\documentclass[aspectratio=169]{beamer}

% Theme and Color Setup
\usetheme{Madrid}
\usecolortheme{whale}
\useinnertheme{rectangles}
\useoutertheme{miniframes}

% Additional Packages
\usepackage[utf8]{inputenc}
\usepackage[T1]{fontenc}
\usepackage{graphicx}
\usepackage{booktabs}
\usepackage{listings}
\usepackage{amsmath}
\usepackage{amssymb}
\usepackage{xcolor}
\usepackage{tikz}
\usepackage{pgfplots}
\pgfplotsset{compat=1.18}
\usetikzlibrary{positioning}
\usepackage{hyperref}

% Custom Colors
\definecolor{myblue}{RGB}{31, 73, 125}
\definecolor{mygray}{RGB}{100, 100, 100}
\definecolor{mygreen}{RGB}{0, 128, 0}
\definecolor{myorange}{RGB}{230, 126, 34}
\definecolor{mycodebackground}{RGB}{245, 245, 245}

% Set Theme Colors
\setbeamercolor{structure}{fg=myblue}
\setbeamercolor{frametitle}{fg=white, bg=myblue}
\setbeamercolor{title}{fg=myblue}
\setbeamercolor{section in toc}{fg=myblue}
\setbeamercolor{item projected}{fg=white, bg=myblue}
\setbeamercolor{block title}{bg=myblue!20, fg=myblue}
\setbeamercolor{block body}{bg=myblue!10}
\setbeamercolor{alerted text}{fg=myorange}

% Set Fonts
\setbeamerfont{title}{size=\Large, series=\bfseries}
\setbeamerfont{frametitle}{size=\large, series=\bfseries}
\setbeamerfont{caption}{size=\small}
\setbeamerfont{footnote}{size=\tiny}

% Code Listing Style
\lstdefinestyle{customcode}{
  backgroundcolor=\color{mycodebackground},
  basicstyle=\footnotesize\ttfamily,
  breakatwhitespace=false,
  breaklines=true,
  commentstyle=\color{mygreen}\itshape,
  keywordstyle=\color{blue}\bfseries,
  stringstyle=\color{myorange},
  numbers=left,
  numbersep=8pt,
  numberstyle=\tiny\color{mygray},
  frame=single,
  framesep=5pt,
  rulecolor=\color{mygray},
  showspaces=false,
  showstringspaces=false,
  showtabs=false,
  tabsize=2,
  captionpos=b
}
\lstset{style=customcode}

% Custom Commands
\newcommand{\hilight}[1]{\colorbox{myorange!30}{#1}}
\newcommand{\source}[1]{\vspace{0.2cm}\hfill{\tiny\textcolor{mygray}{Source: #1}}}
\newcommand{\concept}[1]{\textcolor{myblue}{\textbf{#1}}}
\newcommand{\separator}{\begin{center}\rule{0.5\linewidth}{0.5pt}\end{center}}

% Footer and Navigation Setup
\setbeamertemplate{footline}{
  \leavevmode%
  \hbox{%
  \begin{beamercolorbox}[wd=.3\paperwidth,ht=2.25ex,dp=1ex,center]{author in head/foot}%
    \usebeamerfont{author in head/foot}\insertshortauthor
  \end{beamercolorbox}%
  \begin{beamercolorbox}[wd=.5\paperwidth,ht=2.25ex,dp=1ex,center]{title in head/foot}%
    \usebeamerfont{title in head/foot}\insertshorttitle
  \end{beamercolorbox}%
  \begin{beamercolorbox}[wd=.2\paperwidth,ht=2.25ex,dp=1ex,center]{date in head/foot}%
    \usebeamerfont{date in head/foot}
    \insertframenumber{} / \inserttotalframenumber
  \end{beamercolorbox}}%
  \vskip0pt%
}

% Turn off navigation symbols
\setbeamertemplate{navigation symbols}{}

% Title Page Information
\title[Week 5: Decision Trees]{Week 5: Decision Trees}
\author[J. Smith]{John Smith, Ph.D.}
\institute[University Name]{
  Department of Computer Science\\
  University Name\\
  \vspace{0.3cm}
  Email: email@university.edu\\
  Website: www.university.edu
}
\date{\today}

% Document Start
\begin{document}

\frame{\titlepage}

\begin{frame}[fragile]
    \titlepage
\end{frame}

\begin{frame}[fragile]
    \frametitle{Motivation for Data Mining}
    \begin{block}{Why Do We Need Data Mining?}
        Data mining techniques are essential for extracting meaningful patterns and insights from large datasets in many fields, including:
        \begin{itemize}
            \item **Finance**: Assessing risk and improving decision-making.
            \item **Healthcare**: Identifying trends for better patient outcomes.
            \item **Marketing**: Understanding customer behavior for targeted strategies.
            \item **AI Applications**: Enhancing models like ChatGPT through structured data interpretation.
        \end{itemize}
    \end{block}
    
    \begin{block}{Outline}
        \begin{itemize}
            \item Definition of Decision Trees
            \item Importance in Data Mining
            \item Real-World Applications
            \item Key Points
            \item Example Illustration
        \end{itemize}
    \end{block}
\end{frame}

\begin{frame}[fragile]
    \frametitle{What Are Decision Trees?}
    Decision trees are a type of supervised learning algorithm used for both classification and regression tasks. They are structured as follows:
    
    \begin{itemize}
        \item **Nodes**: Represent features (attributes).
        \item **Branches**: Represent decision rules.
        \item **Leaves**: Denote final outcomes (class labels in classification, continuous values in regression).
    \end{itemize}

    \begin{block}{Importance in Data Mining}
        Decision trees are valued for their:
        \begin{enumerate}
            \item Interpretability: Easy to communicate results.
            \item No need for data scaling: Do not require normalization.
            \item Handling non-linear relationships: Capture complex feature interactions.
        \end{enumerate}
    \end{block}
\end{frame}

\begin{frame}[fragile]
    \frametitle{Real-World Applications}
    Decision trees have numerous applications across various industries:
    
    \begin{itemize}
        \item **Finance**: Assessing credit risk, predicting loan defaults.
        \item **Healthcare**: Diagnosis assistance, patient outcome prediction.
        \item **Marketing**: Customer segmentation, targeting strategies.
        \item **AI and Machine Learning**: Used in models like ChatGPT for feature selection and data distribution understanding.
    \end{itemize}

    \begin{block}{Key Points to Emphasize}
        \begin{itemize}
            \item Model structure: Understanding nodes, branches, and leaves.
            \item Versatility: Suitable for both classification and regression tasks.
            \item Real-world impact: Enhances diverse applications and decision-making processes.
        \end{itemize}
    \end{block}
\end{frame}

\begin{frame}[fragile]
    \frametitle{Example Illustration}
    Consider a simple decision tree for classifying whether a customer will buy a product based on two features: age and income.
    
    \begin{center}
        \begin{lstlisting}
                 [Age]
                  /   \
                <30    >=30
               /        \
             [Income]   Buy
             /    \
          Low       High
           /          \
         No           Yes
        \end{lstlisting}
    \end{center}

    \begin{block}{Conclusion}
        Decision trees serve as a foundational technique in data mining, providing clarity, adaptability, and insights across various fields. Understanding these tools is essential for modern data analysis and AI applications.
    \end{block}
\end{frame}

\begin{frame}[fragile]
    \frametitle{Motivation for Decision Trees - Overview}
    \begin{block}{Why Do We Need Decision Trees?}
        Decision Trees serve as a vital tool in data analysis, particularly for classification and regression problems. They simplify complex decision-making processes, making them interpretable and accessible for both beginners and experienced data scientists.
    \end{block}
    
    \begin{block}{Key Applications}
        \begin{itemize}
            \item Healthcare Diagnosis
            \item Credit Scoring in Finance
            \item Customer Segmentation in Marketing
            \item AI and Machine Learning Integration
        \end{itemize}
    \end{block}
\end{frame}

\begin{frame}[fragile]
    \frametitle{Motivation for Decision Trees - Applications}
    \begin{block}{Healthcare Diagnosis}
        Decision Trees assist doctors in diagnosing illnesses based on patient symptoms.
        \begin{itemize}
            \item Example Questions:
            \begin{itemize}
                \item Is the patient experiencing a fever?
                \item Is there a rash present?
            \end{itemize}
            \item Guides diagnosis towards specific conditions, e.g., classifying types of cancers.
        \end{itemize}
    \end{block}
    
    \begin{block}{Credit Scoring in Finance}
        Financial institutions use Decision Trees to evaluate loan applications based on:
        \begin{itemize}
            \item Income level
            \item Credit history
            \item Outstanding debts
        \end{itemize}
        Example: Determining if an applicant is a high-risk borrower.
    \end{block}
\end{frame}

\begin{frame}[fragile]
    \frametitle{Motivation for Decision Trees - Further Applications}
    \begin{block}{Customer Segmentation in Marketing}
        Marketers segment customers based on purchasing behavior.
        \begin{itemize}
            \item Example: Classifying customers into 'Likely Buy', 'Neutral', and 'Not Likely'.
        \end{itemize}
    \end{block}
    
    \begin{block}{AI and Machine Learning Integration}
        Decision Trees are foundational for more complex algorithms like Random Forests and Gradient Boosted Trees.
        \begin{itemize}
            \item Recent advancements in AI (e.g., ChatGPT) use decision processes influenced by decision trees.
        \end{itemize}
    \end{block}

    \begin{block}{Core Takeaways}
        \begin{itemize}
            \item Simplify complex data into understandable visual structures.
            \item Widely applicable across industries such as healthcare, finance, marketing, and AI.
            \item Provide actionable insights and enhance operational efficiency.
        \end{itemize}
    \end{block}
\end{frame}

\begin{frame}[fragile]
    \frametitle{What is a Decision Tree? - Definition}
    \begin{block}{Definition}
        A Decision Tree is a flowchart-like structure used for making decisions based on predicting outcomes. It breaks down a complex decision into simple, binary questions, guiding users to a decision by following branches that represent different choices or outcomes.
    \end{block}
    
    \begin{itemize}
        \item Flowchart-like structure
        \item Makes decisions through binary questions
        \item Predicts outcomes effectively
    \end{itemize}
\end{frame}

\begin{frame}[fragile]
    \frametitle{What is a Decision Tree? - Structure Overview}
    \begin{block}{Structure}
        \begin{enumerate}
            \item \textbf{Nodes:}
                \begin{itemize}
                    \item \textbf{Decision Nodes:} Points for decision or split (e.g., “Is the weather sunny?”)
                    \item \textbf{Leaf Nodes:} Final outcome or decision (e.g., “Play tennis” or “Do not play tennis”)
                \end{itemize}
            \item \textbf{Branches:} Connect nodes and illustrate outcomes of decisions based on node questions.
        \end{enumerate}
    \end{block}
\end{frame}

\begin{frame}[fragile]
    \frametitle{What is a Decision Tree? - Example and Key Points}
    \begin{block}{Illustration Example}
        \begin{verbatim}
                    [Is the weather sunny?]
                     /                     \
                   Yes                     No
                   /                         \
            [Is it a weekend?]         [Is it raining?]
             /           \                  /         \
           Yes          No                Yes        No
           /            |                  |          |
      [Play Tennis]   [Do not play]   [Do not play] [Play Tennis]
        \end{verbatim}
    \end{block}

    \begin{itemize}
        \item \textbf{Key Points:}
        \begin{itemize}
            \item \textbf{Interpretability:} Visual representation simplifies understanding.
            \item \textbf{Versatility:} Used in classification and regression tasks.
            \item \textbf{Non-Parametric Model:} No assumptions about data distribution.
        \end{itemize}
    \end{itemize}
\end{frame}

\begin{frame}[fragile]
    \frametitle{What is a Decision Tree? - Applications}
    \begin{block}{Applications in Real-World}
        \begin{itemize}
            \item Employed in finance for credit risk assessment.
            \item Used in healthcare for disease diagnosis.
            \item Applied in marketing for customer segmentation.
            \item Forms foundational algorithms for advanced AI (e.g., Random Forests, ChatGPT).
        \end{itemize}
    \end{block}
\end{frame}

\begin{frame}[fragile]
    \frametitle{Types of Decision Trees}
    % Overview of Decision Trees
    Decision trees are powerful tools in machine learning, used for both classification and regression tasks. Understanding the differences between these tree types is crucial for selecting the appropriate model for a given dataset.
\end{frame}

\begin{frame}[fragile]
    \frametitle{1. Classification Trees}
    
    \begin{block}{Definition}
        Classification trees are used to predict categorical outcomes. Each leaf node represents a class label, with branches corresponding to the features leading to these classifications.
    \end{block}
    
    \begin{itemize}
        \item \textbf{Use Case:} Predicting whether an email is Spam or Not Spam.
        \item \textbf{Attributes:} Words in the email, email source, length of the email.
        \item \textbf{Output:} Two classes—Spam or Not Spam.
    \end{itemize}
    
    \begin{block}{How it Works}
        \begin{enumerate}
            \item \textbf{Splitting:} The dataset is divided based on feature values to maximize homogeneity of classes.
            \item \textbf{Decision Nodes:} Questions are posed at each node to guide the splits.
        \end{enumerate}
    \end{block}

    \begin{block}{Evaluation Metrics}
        Accuracy, Precision, Recall, Confusion Matrix.
    \end{block}
\end{frame}

\begin{frame}[fragile]
    \frametitle{2. Regression Trees}
    
    \begin{block}{Definition}
        Regression trees are used to predict continuous numeric outcomes. The leaves represent numerical values, and branches split data based on continuous features.
    \end{block}
    
    \begin{itemize}
        \item \textbf{Use Case:} Estimating house prices.
        \item \textbf{Attributes:} Square footage, number of bedrooms, location.
        \item \textbf{Output:} A predicted price (e.g., \$300,000).
    \end{itemize}

    \begin{block}{How it Works}
        \begin{enumerate}
            \item \textbf{Splitting:} The tree divides the dataset to minimize the variance of the target variable within subsets.
            \item \textbf{Prediction:} At the leaf node, the predicted value is the mean of all instances that reach that node.
        \end{enumerate}
    \end{block}

    \begin{block}{Evaluation Metrics}
        Mean Squared Error (MSE), Root Mean Squared Error (RMSE), R-squared.
    \end{block}
\end{frame}

\begin{frame}[fragile]
    \frametitle{Key Points to Emphasize}
    \begin{itemize}
        \item \textbf{Decision Trees:} Can handle both classification (categories) and regression (numbers).
        \item \textbf{Choosing the Right Tree:}
        \begin{itemize}
            \item Use classification trees for categorical outcomes.
            \item Use regression trees for continuous outcomes.
        \end{itemize}
        \item \textbf{Real-World Applications:} Utilized in areas such as finance (credit scoring), healthcare (disease diagnosis), and more.
    \end{itemize}
    
    By understanding these two main types of decision trees, you can effectively apply them to various predictive modeling scenarios, improving decision-making processes across domains.
\end{frame}

\begin{frame}[fragile]
    \frametitle{How Decision Trees Work - Introduction}
    \begin{block}{What are Decision Trees?}
        Decision Trees are powerful tools for both classification and regression tasks in machine learning. They represent decisions and their possible consequences visually as a tree structure, simplifying complex decision-making processes.
    \end{block}
\end{frame}

\begin{frame}[fragile]
    \frametitle{How Decision Trees Work - Tree-Building Process}
    The tree-building process involves several key steps:
    \begin{enumerate}
        \item \textbf{Select the Root Node:}
            The algorithm evaluates which feature to split on first based on the target variable.
        \item \textbf{Splitting:}
            The dataset is divided into subsets using \textbf{splitting criteria}.
        \item \textbf{Creating Child Nodes:}
            Child nodes are formed recursively until stopping conditions are met.
        \item \textbf{Leaf Nodes:}
            These nodes provide a prediction based on the majority class or mean value.
    \end{enumerate}
\end{frame}

\begin{frame}[fragile]
    \frametitle{How Decision Trees Work - Splitting Criteria}
    \begin{block}{Key Splitting Criteria}
        \begin{itemize}
            \item \textbf{Gini Index:} Measures impurity. A lower Gini indicates better splits.
                \begin{equation}
                Gini(D) = 1 - \sum_{i=1}^{c} p_i^2
                \end{equation}
            \item \textbf{Entropy:} Represents uncertainty in the dataset.
                \begin{equation}
                Entropy(D) = - \sum_{i=1}^{c} p_i \log_2(p_i)
                \end{equation}
            \item \textbf{Mean Squared Error (MSE):} Used in regression trees, measures average squared difference between predicted and actual outcomes.
        \end{itemize}
    \end{block}
\end{frame}

\begin{frame}[fragile]
    \frametitle{Best Practices and Conclusion}
    \begin{block}{Best Practices for Building Decision Trees}
        \begin{itemize}
            \item \textbf{Pre-Pruning:} Stop tree growth to reduce overfitting.
            \item \textbf{Post-Pruning:} Remove less important branches after full growth.
            \item \textbf{Feature Selection:} Prioritize features with the most information gain.
            \item \textbf{Handle Missing Values:} Use strategies to mitigate biases from missing data.
        \end{itemize}
    \end{block}

    \begin{block}{Conclusion}
        Decision trees provide an intuitive approach to both classification and regression tasks, making them valuable in data analysis. Proper construction and optimization lead to robust models that predict effectively on unseen data.
    \end{block}
\end{frame}

\begin{frame}[fragile]
    \frametitle{Understanding Splitting Criteria}
    \begin{itemize}
        \item Decision trees excel in classification tasks.
        \item Effectiveness hinges on optimal data splits.
        \item Focus on three criteria: 
        \begin{itemize}
            \item Gini Index
            \item Entropy
            \item Mean Squared Error (MSE)
        \end{itemize}
    \end{itemize}
\end{frame}

\begin{frame}[fragile]
    \frametitle{1. Gini Index}
    \begin{block}{Definition}
        Measures the impurity of a dataset. Quantifies incorrect label predictions.
    \end{block}
    \begin{block}{Formula}
        \begin{equation}
            Gini(D) = 1 - \sum_{i=1}^{C} (p_i)^2 
        \end{equation}
    \end{block}
    \begin{block}{Example}
        For a dataset with 10 instances (4 A, 6 B):
        \begin{equation}
            Gini(D) = 1 - (0.4^2 + 0.6^2) = 0.48
        \end{equation}
    \end{block}
    \begin{itemize}
        \item Lower Gini Index values indicate better splits.
        \item Efficient for binary classification.
    \end{itemize}
\end{frame}

\begin{frame}[fragile]
    \frametitle{2. Entropy}
    \begin{block}{Definition}
        A measure of uncertainty or disorder in a dataset.
    \end{block}
    \begin{block}{Formula}
        \begin{equation}
            Entropy(D) = - \sum_{i=1}^{C} p_i \log_2(p_i) 
        \end{equation}
    \end{block}
    \begin{block}{Example}
        Using the class distribution:
        \begin{equation}
            Entropy(D) \approx 0.970
        \end{equation}
    \end{block}
    \begin{itemize}
        \item Ranges from 0 (perfectly pure) to $\log_2(C)$ (maximum impurity).
        \item Offers a detailed view of class distributions.
    \end{itemize}
\end{frame}

\begin{frame}[fragile]
    \frametitle{3. Mean Squared Error (MSE)}
    \begin{block}{Definition}
        Measures average squared differences in regression tasks.
    \end{block}
    \begin{block}{Formula}
        \begin{equation}
            MSE = \frac{1}{n} \sum_{i=1}^{n} (y_i - \hat{y}_i)^2 
        \end{equation}
    \end{block}
    \begin{block}{Example}
        True values: $[3, -0.5, 2, 7]$ and predicted values: $[2.5, 0.0, 2, 8]$ yield:
        \begin{equation}
            MSE = 0.375
        \end{equation}
    \end{block}
    \begin{itemize}
        \item Lower MSE suggests better predictive performance.
        \item Crucial for regression tree algorithms.
    \end{itemize}
\end{frame}

\begin{frame}[fragile]
    \frametitle{Conclusion and Outlines}
    \begin{block}{Conclusion}
        Understanding these criteria is vital for effective decision trees. Criteria selection depends on:
        \begin{itemize}
            \item Problem type (classification vs. regression).
            \item Nature of data.
        \end{itemize}
    \end{block}
    \begin{itemize}
        \item Gini Index: Measures impurity.
        \item Entropy: Quantifies uncertainty.
        \item Mean Squared Error: Evaluates regression accuracy.
    \end{itemize}
    \pause
    \textbf{Next Topic}: Advantages of Decision Trees.
\end{frame}

\begin{frame}[fragile]
    \frametitle{Advantages of Decision Trees - Introduction}
    \begin{itemize}
        \item Decision trees are popular methods for classification and regression in data mining and machine learning.
        \item Known for their simplicity and interpretability, making them accessible to users with varying expertise levels.
        \item This presentation explores the key strengths of decision trees.
    \end{itemize}
\end{frame}

\begin{frame}[fragile]
    \frametitle{Advantages of Decision Trees - Key Advantages}
    \begin{enumerate}
        \item \textbf{Interpretability}
        \begin{itemize}
            \item Intuitive Structure: Visual representation of decisions and outcomes.
            \item Ease of Understanding: Non-experts can easily follow decision rules.
            \item Transparency: Users can trace decision paths, fostering trust.
        \end{itemize}
        
        \item \textbf{Ease of Use}
        \begin{itemize}
            \item Minimal Data Preparation: Handles both numerical and categorical data.
            \item Automatic Feature Selection: Determines important features during the tree-building process.
            \item No Extensive Tuning Required: Simple application for beginners.
        \end{itemize}
    \end{enumerate}
\end{frame}

\begin{frame}[fragile]
    \frametitle{Advantages of Decision Trees - Versatility & Visualization}
    \begin{enumerate}
        \setcounter{enumi}{2} % Continue enumeration
        \item \textbf{Versatility}
        \begin{itemize}
            \item Applicable Across Domains: Suitable for customer segmentation, medical diagnosis, etc.
            \item Integration with Ensemble Methods: Foundation for techniques like Random Forests.
        \end{itemize}

        \item \textbf{Visualize Decision Process}
        \begin{itemize}
            \item Graphical Representation: Effective visualization aids communication and decision-making understanding.
        \end{itemize}
    \end{enumerate}
\end{frame}

\begin{frame}[fragile]
    \frametitle{Advantages of Decision Trees - Example}
    \begin{block}{Example: Buying a Car}
        Consider a simplified decision tree to determine whether to buy a car based on:
        \begin{itemize}
            \item Budget (Low/High)
            \item Mileage (Low/High)
        \end{itemize}
        
        Decision paths:
        \begin{itemize}
            \item If Budget is Low and Mileage is High, Do Not Buy
            \item If Budget is High and Mileage is Low, Buy
            \item If Budget is High and Mileage is High, Buy
        \end{itemize}
        
        This illustrates how straightforward decision rules can be understood.
    \end{block}
\end{frame}

\begin{frame}[fragile]
    \frametitle{Advantages of Decision Trees - Conclusion}
    \begin{itemize}
        \item Decision trees offer advantages in interpretability, ease of use, versatility, and decision-tracing.
        \item These strengths make them a preferred choice for both beginners and experienced data analysts.
    \end{itemize}

    \begin{block}{Key Summary Points}
        \begin{itemize}
            \item Interpretability builds trust.
            \item Ease of use reduces preparation and tuning challenges.
            \item Versatility across various fields.
            \item Visual representation enhances understanding.
        \end{itemize}
    \end{block}
\end{frame}

\begin{frame}[fragile]
    \frametitle{Limitations of Decision Trees - Overview}
    \begin{itemize}
        \item Decision trees are interpretable and simple tools in data mining.
        \item However, they have notable limitations:
        \begin{itemize}
            \item Overfitting
            \item Bias towards features
            \item Instability
            \item Challenges with unbalanced data
        \end{itemize}
    \end{itemize}
\end{frame}

\begin{frame}[fragile]
    \frametitle{Limitations of Decision Trees - Overfitting}
    \begin{block}{1. Overfitting}
        \begin{itemize}
            \item \textbf{Definition}: Learning the noise instead of the pattern in training data.
            \item \textbf{Illustration}: A decision tree with complex splits for every data point fails to generalize.
            \item \textbf{Example}: Memorizing every case leads to poor predictions for new customers.
        \end{itemize}
    \end{block}
    \textbf{Outline:}
    \begin{itemize}
        \item Definition and explanation
        \item Illustrative example
    \end{itemize}
\end{frame}

\begin{frame}[fragile]
    \frametitle{Limitations of Decision Trees - Other Key Issues}
    \begin{block}{2. Bias towards Features}
        \begin{itemize}
            \item \textbf{Issue}: Features with more levels may be prioritized.
            \item \textbf{Illustration}: Unique values (e.g., ZIP code) can overshadow more relevant features (e.g., income).
            \item \textbf{Example}: Excessive focus on "ZIP code" may lead to ignoring critical factors in customer behavior prediction.
        \end{itemize}
    \end{block}
  
    \begin{block}{3. Instability}
        \begin{itemize}
            \item \textbf{Definition}: Small data changes lead to different tree structures.
            \item \textbf{Consequence}: Reduced reliability of predictions.
            \item \textbf{Illustration}: A single instance change can topple the entire model, like a house on sand.
        \end{itemize}
    \end{block}
  
    \textbf{Outline:}
    \begin{itemize}
        \item Bias towards features
        \item Instability and its impact
    \end{itemize}
\end{frame}

\begin{frame}[fragile]
    \frametitle{Limitations of Decision Trees - Summary and Transition}
    \begin{block}{4. Handling Unbalanced Data}
        \begin{itemize}
            \item \textbf{Challenge}: Struggles with datasets where one class is overrepresented.
            \item \textbf{Result}: Biased predictions favoring the majority class.
            \item \textbf{Example}: In a dataset of 1000 transactions, predicting "buy" for all cases ignores "not buy".
        \end{itemize}
    \end{block}
    
    \begin{itemize}
        \item \textbf{Summary Points:}
        \begin{itemize}
            \item Prone to overfitting in complex datasets.
            \item Biased towards certain features affecting model quality.
            \item Structural instability limits reliability.
            \item Challenges with unbalanced data are significant.
        \end{itemize}
        \item \textbf{Transition Note:} Next, we will discuss strategies to mitigate these issues, focusing on techniques like pruning.
    \end{itemize}
    
    \textbf{Outline:}
    \begin{itemize}
        \item Summary of key limitations
        \item Transition to solutions
    \end{itemize}
\end{frame}

\begin{frame}[fragile]
    \frametitle{Overfitting and Pruning - Concept Overview}
    \begin{itemize}
        \item \textbf{Overfitting:} Occurs when a decision tree learns not only the trend in the training data but also the noise.
        \begin{itemize}
            \item Result: High accuracy on training data but poor performance on test data.
            \item \textit{Illustration:} A tree perfectly adapted to individual data points, capturing all variations.
        \end{itemize}
        \item \textbf{Symptoms:}
        \begin{itemize}
            \item Excellent training performance with low test accuracy.
            \item Complex tree with many non-generalizing splits.
        \end{itemize}
    \end{itemize}
\end{frame}

\begin{frame}[fragile]
    \frametitle{Overfitting - Why It is a Problem}
    \begin{itemize}
        \item \textbf{Generalization Issue:} The main goal of machine learning is to generalize to new data, which overfitting undermines.
        \item \textbf{Model Complexity:} Complex trees complicate interpretation and can be sensitive to data changes.
    \end{itemize}
    
    \begin{block}{Pruning Techniques}
        \begin{itemize}
            \item \textbf{Definition:} Pruning removes sections of the tree that provide little predictive power, aiding in model simplification and generalization.
        \end{itemize}
    \end{block}
\end{frame}

\begin{frame}[fragile]
    \frametitle{Pruning Techniques - Types}
    \begin{enumerate}
        \item \textbf{Pre-pruning (Early Stopping)}
            \begin{itemize}
                \item Stops tree growth based on thresholds (e.g., minimum samples to split).
                \item \textit{Example:} A node with fewer than 10 samples becomes a leaf node.
            \end{itemize}
        \item \textbf{Post-pruning}
            \begin{itemize}
                \item Full tree constructed first, then trimmed based on validation performance.
                \item \textit{Example:} Evaluate subtrees and remove those that do not improve performance.
            \end{itemize}
    \end{enumerate}
    
    \begin{block}{Key Points}
        \begin{itemize}
            \item Balancing bias and variance is crucial; pruning reduces variance, improving generalization.
            \item Both pre-pruning and post-pruning strategies are valid depending on dataset specifics.
        \end{itemize}
    \end{block}
\end{frame}

\begin{frame}[fragile]
    \frametitle{Ensemble Methods - Introduction}
    % Introduction to Ensemble Methods
    \begin{block}{Definition}
        Ensemble methods are powerful techniques in machine learning that combine multiple models to improve performance and robustness compared to single model approaches.
    \end{block}
    
    \begin{block}{Motivation}
        - Decision trees tend to overfit.
        - Ensemble methods enhance accuracy and robustness.
    \end{block}

    \begin{block}{Primary Techniques}
        - \textbf{Random Forest}
        - \textbf{Boosting}
    \end{block}
\end{frame}

\begin{frame}[fragile]
    \frametitle{Ensemble Methods - Why Use Them?}
    % Reasons to use Ensemble Methods
    \begin{enumerate}
        \item \textbf{Improved Accuracy}:
            Ensemble methods often achieve higher accuracy than individual models.
        \item \textbf{Robustness}:
            They reduce overfitting, improving generalization on unseen data.
        \item \textbf{Handling Data Complexity}:
            They capture complex patterns in large datasets effectively.
    \end{enumerate}
\end{frame}

\begin{frame}[fragile]
    \frametitle{Ensemble Methods - Random Forest}
    % Exploring Random Forest
    \begin{block}{Definition}
        Random Forest is an ensemble learning method that creates a 'forest' of many decision trees trained on random subsets of data and features.
    \end{block}

    \begin{block}{How It Works}
        - \textbf{Bootstrap Aggregating (Bagging)}: Random samples of the dataset (with replacement) train different trees.
        - \textbf{Feature Randomness}: A random subset of features is considered at each split, reducing correlation between trees.
    \end{block}
    
    \begin{block}{Example Use Case}
        Predicting customer churn by analyzing usage patterns across different demographic groups.
    \end{block}
\end{frame}

\begin{frame}[fragile]
    \frametitle{Ensemble Methods - Boosting}
    % Exploring Boosting
    \begin{block}{Definition}
        Boosting builds models sequentially, where each new model attempts to correct the errors made by its predecessor.
    \end{block}

    \begin{block}{How It Works}
        - Each model is trained on the instances mispredicted by its predecessors, focusing learning on hard-to-predict cases.
        - Popular algorithms: AdaBoost, Gradient Boosting Machines (GBM), XGBoost.
    \end{block}

    \begin{block}{Example Use Case}
        Predicting credit risk by refining assessments of loan applications with high default risks.
    \end{block}
\end{frame}

\begin{frame}[fragile]
    \frametitle{Key Formulas}
    % Key mathematical formulations for boosting
    \begin{equation}
        F(x) = F_{t-1}(x) + \alpha_t \cdot h_t(x)
    \end{equation}

    \begin{itemize}
        \item $F(x)$: overall prediction
        \item $h_t(x)$: current model's prediction
        \item $\alpha_t$: weight assigned to the current model
    \end{itemize}
\end{frame}

\begin{frame}[fragile]
    \frametitle{Ensemble Methods - Conclusion and Takeaway}
    % Concluding remarks on ensemble methods
    \begin{block}{Conclusion}
        Ensemble methods like Random Forest and Boosting improve accuracy and reduce overfitting, enhancing traditional decision tree performance.
    \end{block}

    \begin{block}{Key Takeaway}
        Embrace ensemble methods to leverage the strengths of multiple models, resulting in better predictions and insights.
    \end{block}

    \begin{block}{Next Steps}
        Explore real-world applications of decision trees in upcoming slides focused on case studies.
    \end{block}
\end{frame}

\begin{frame}[fragile]
    \frametitle{Decision Trees in Practice}
    % Overview of Decision Trees
    \begin{block}{Overview of Decision Trees}
        Decision trees are versatile and powerful tools in data mining and machine learning, known for:
        \begin{itemize}
            \item Simple and visual representation
            \item Applicability in classification, regression, and decision analysis
        \end{itemize}
    \end{block}
\end{frame}

\begin{frame}[fragile]
    \frametitle{Motivation for Using Decision Trees}
    % Motivation for Using Decision Trees
    \begin{block}{Growing Need}
        Businesses increasingly require data-driven decision-making due to:
        \begin{itemize}
            \item Increased data availability
            \item Complexity of environments
            \item Demand for transparency
        \end{itemize}
    \end{block}
    
    \begin{block}{Benefits of Decision Trees}
        Companies can achieve:
        \begin{itemize}
            \item Efficient handling of large datasets
            \item Clear interpretations for stakeholders
            \item Accessibility for non-technical users
        \end{itemize}
    \end{block}
\end{frame}

\begin{frame}[fragile]
    \frametitle{Case Studies of Decision Tree Implementations}
    % Case Studies
    \begin{enumerate}
        \item \textbf{Healthcare: Patient Diagnosis}
            \begin{itemize}
                \item Use: Assist in diagnosing diseases
                \item Example: Determining diabetes based on age, BMI, and other factors
                \item Outcome: Improved diagnostics and tailored treatment
            \end{itemize}
        \item \textbf{Finance: Credit Scoring}
            \begin{itemize}
                \item Use: Assess creditworthiness
                \item Example: Evaluating credit history and income for loan approval
                \item Outcome: Reduced default rates and improved risk assessment
            \end{itemize}
        \item \textbf{Retail: Customer Segmentation}
            \begin{itemize}
                \item Use: Segment customers based on behavior
                \item Example: Categorizing customers as "frequent buyers" or "discount shoppers"
                \item Outcome: Tailored marketing strategies
            \end{itemize}
        \item \textbf{Manufacturing: Quality Control}
            \begin{itemize}
                \item Use: Identify production defects
                \item Example: Analyzing factors like machine settings and material quality
                \item Outcome: Enhanced quality management and reduced wastage
            \end{itemize}
    \end{enumerate}
\end{frame}

\begin{frame}[fragile]
    \frametitle{Key Points to Emphasize}
    % Key Points
    \begin{itemize}
        \item \textbf{Interpretability}: Decision trees provide clear visual representations.
        \item \textbf{Flexibility}: They can handle both categorical and numerical data.
        \item \textbf{Integration}: Often combined with ensemble methods such as Random Forest and Boosting.
    \end{itemize}
\end{frame}

\begin{frame}[fragile]
    \frametitle{Conclusion}
    % Conclusion
    The implementation of decision trees is versatile across industries, simplifying complex decision-making and yielding actionable insights. Their strengths make them fundamental in today's data-driven landscape. In the next slide, we will explore how to implement these concepts using Python libraries such as scikit-learn.
\end{frame}

\begin{frame}[fragile]
    \frametitle{Decision Trees with Python}
    \begin{block}{Overview}
        Decision Trees are powerful tools used in data mining and machine learning for classification and regression tasks. They work by splitting the data into branches to make decisions based on feature values, ultimately leading to a clear output.
    \end{block}
\end{frame}

\begin{frame}[fragile]
    \frametitle{Why Use Decision Trees?}
    \begin{itemize}
        \item \textbf{Simplicity}: Easy to understand and interpret with a clear visual representation.
        \item \textbf{Non-parametric}: No assumptions about data distribution, making them versatile.
        \item \textbf{Feature Importance}: Identifies significant features for feature selection.
    \end{itemize}
    \begin{block}{Example}
        In a healthcare application, a decision tree can help determine if a patient has a specific disease based on symptoms and medical history.
    \end{block}
\end{frame}

\begin{frame}[fragile]
    \frametitle{Implementation with scikit-learn - Step-by-Step}
    \begin{enumerate}
        \item \textbf{Install Required Libraries}:
        \begin{lstlisting}
        pip install scikit-learn pandas
        \end{lstlisting}
        
        \item \textbf{Import Libraries}:
        \begin{lstlisting}
        import pandas as pd
        from sklearn.model_selection import train_test_split
        from sklearn.tree import DecisionTreeClassifier
        from sklearn.metrics import accuracy_score, classification_report
        \end{lstlisting}
        
        \item \textbf{Load Dataset}:
        \begin{lstlisting}
        data = pd.read_csv('your_dataset.csv')
        \end{lstlisting}
    \end{enumerate}
\end{frame}

\begin{frame}[fragile]
    \frametitle{Implementation with scikit-learn - Continue}
    \begin{enumerate}
        \setcounter{enumi}{3} % Continue from last numbered item
        \item \textbf{Prepare Data}:
        \begin{lstlisting}
        X = data.drop('target_column', axis=1)
        y = data['target_column']
        \end{lstlisting}

        \item \textbf{Split Data}:
        \begin{lstlisting}
        X_train, X_test, y_train, y_test = train_test_split(X, y, test_size=0.2, random_state=42)
        \end{lstlisting}

        \item \textbf{Train the Model}:
        \begin{lstlisting}
        model = DecisionTreeClassifier(random_state=42)
        model.fit(X_train, y_train)
        \end{lstlisting}

        \item \textbf{Make Predictions}:
        \begin{lstlisting}
        predictions = model.predict(X_test)
        \end{lstlisting}

        \item \textbf{Evaluate the Model}:
        \begin{lstlisting}
        accuracy = accuracy_score(y_test, predictions)
        print(f'Accuracy: {accuracy}')
        print(classification_report(y_test, predictions))
        \end{lstlisting}
    \end{enumerate}
\end{frame}

\begin{frame}[fragile]
    \frametitle{Key Points and Conclusion}
    \begin{itemize}
        \item \textbf{Visual Representation}: Decision Trees can be visualized, facilitating understanding.
        \item \textbf{Overfitting}: Be cautious of overfitting; control parameters like \texttt{max\_depth}.
        \item \textbf{Parameter Tuning}: Hyperparameters such as \texttt{min\_samples\_split} can affect performance significantly.
    \end{itemize}
    \begin{block}{Conclusion}
        Decision Trees are intuitive and powerful methods for tackling classification and regression problems. With Python and libraries like scikit-learn, implementing them is straightforward and invaluable in a data scientist's toolkit.
    \end{block}
\end{frame}

\begin{frame}[fragile]
    \frametitle{Evaluation Metrics for Decision Trees}
    \begin{block}{Introduction}
        After implementing decision trees using Python, it's essential to evaluate performance effectively. This slide discusses evaluation metrics that help gauge the effectiveness of a decision tree classifier: 
        \begin{itemize}
            \item Accuracy
            \item Precision
            \item Recall
            \item F1-score
        \end{itemize}
    \end{block}
\end{frame}

\begin{frame}[fragile]
    \frametitle{Key Evaluation Metrics - 1}
    \begin{block}{Accuracy}
        \textbf{Definition}: Measures the proportion of true results (both true positives and true negatives) in the total population.

        \textbf{Formula}:
        \begin{equation}
            \text{Accuracy} = \frac{\text{True Positives} + \text{True Negatives}}{\text{Total Samples}}
        \end{equation}
        
        \textbf{Example}: 
        If a decision tree correctly classifies 70 out of 100 instances, the accuracy is:
        \begin{equation}
            \text{Accuracy} = \frac{70}{100} = 0.70 \text{ or } 70\%
        \end{equation}
        
        \textbf{Key Point}: Accuracy is straightforward but can be misleading in imbalanced datasets.
    \end{block}

    \begin{block}{Precision}
        \textbf{Definition}: Assesses the accuracy of the positive predictions. It indicates how many of the predicted positives are actual positives.
        
        \textbf{Formula}:
        \begin{equation}
            \text{Precision} = \frac{\text{True Positives}}{\text{True Positives} + \text{False Positives}}
        \end{equation}

        \textbf{Example}: 
        If a model predicts 40 positive cases, with 30 true positives and 10 false positives:
        \begin{equation}
            \text{Precision} = \frac{30}{30 + 10} = 0.75 \text{ or } 75\%
        \end{equation}

        \textbf{Key Point}: High precision indicates a low false positive rate, crucial for applications like medical testing.
    \end{block}
\end{frame}

\begin{frame}[fragile]
    \frametitle{Key Evaluation Metrics - 2}
    \begin{block}{Recall (Sensitivity)}
        \textbf{Definition}: Measures the ability of the classifier to find all relevant cases (true positives).

        \textbf{Formula}:
        \begin{equation}
            \text{Recall} = \frac{\text{True Positives}}{\text{True Positives} + \text{False Negatives}}
        \end{equation}

        \textbf{Example}: 
        If there are 40 actual positive instances, and the model detects 30 correctly:
        \begin{equation}
            \text{Recall} = \frac{30}{30 + 10} = 0.75 \text{ or } 75\%
        \end{equation}

        \textbf{Key Point}: Recall is vital where missing a positive case is crucial, such as in fraud detection.
    \end{block}

    \begin{block}{F1-Score}
        \textbf{Definition}: The F1-Score is the harmonic mean of Precision and Recall, providing a balance between the two.

        \textbf{Formula}:
        \begin{equation}
            \text{F1-Score} = 2 \times \frac{\text{Precision} \times \text{Recall}}{\text{Precision} + \text{Recall}}
        \end{equation}

        \textbf{Example}: Given precision = 0.75 and recall = 0.75:
        \begin{equation}
            \text{F1-Score} = 2 \times \frac{0.75 \times 0.75}{0.75 + 0.75} = 0.75 \text{ or } 75\%
        \end{equation}

        \textbf{Key Point}: The F1-Score is useful when seeking a balance between precision and recall, especially with imbalanced classes.
    \end{block}
\end{frame}

\begin{frame}[fragile]
    \frametitle{Summary and Next Steps}
    \begin{block}{Summary}
        \begin{itemize}
            \item \textbf{Accuracy}: Measures overall correctness but may not reflect model performance in unbalanced datasets.
            \item \textbf{Precision}: Critical when false positives are costly.
            \item \textbf{Recall}: Essential in scenarios where false negatives are risky.
            \item \textbf{F1-Score}: Provides a comprehensive measure of a model’s accuracy, balancing precision and recall.
        \end{itemize}
    \end{block}

    \begin{block}{Next Steps}
        In our next lesson, we will explore recent trends and applications of decision trees in AI and machine learning, discussing technologies such as ChatGPT and their dependency on robust data mining techniques.
    \end{block}
\end{frame}

\begin{frame}[fragile]
    \frametitle{Recent Trends and Applications - Introduction}
    \begin{block}{Definition}
        Decision Trees are a powerful and versatile method in machine learning for classification and regression tasks.
    \end{block}
    \begin{itemize}
        \item Model decisions and consequences in a tree-like structure.
        \item Transparent and easy to understand decision-making process.
    \end{itemize}
\end{frame}

\begin{frame}[fragile]
    \frametitle{Recent Trends in Decision Trees}
    \begin{enumerate}
        \item \textbf{Integration with Ensemble Methods}
            \begin{itemize}
                \item Techniques like Random Forests and Gradient Boosting improve predictive performance.
                \item \textit{Example}: A Random Forest combines multiple trees for better accuracy.
            \end{itemize}
        
        \item \textbf{Interpretability in AI}
            \begin{itemize}
                \item Useful in applications requiring transparency (e.g., medical diagnoses).
                \item \textit{Key Point}: Visualizing decision criteria enhances trust.
            \end{itemize}
        
        \item \textbf{Pruning Techniques}
            \begin{itemize}
                \item Modern algorithms apply pruning to avoid overfitting while maintaining accuracy.
            \end{itemize}
    \end{enumerate}
\end{frame}

\begin{frame}[fragile]
    \frametitle{Applications of Decision Trees}
    \begin{block}{Key Domains}
        Decision Trees find applications across various fields, notably:
    \end{block}
    \begin{enumerate}
        \item \textbf{Healthcare}
            \begin{itemize}
                \item Predicting patient outcomes based on symptoms.
                \item \textit{Example}: Classifying diseases based on symptom presentation.
            \end{itemize}

        \item \textbf{Finance}
            \begin{itemize}
                \item Used in credit scoring to assess risk.
                \item \textit{Example}: Classifying customers' risk based on attributes.
            \end{itemize}
        
        \item \textbf{Customer Relationship Management (CRM)}
            \begin{itemize}
                \item Predicts customer churn and segments based on behavior.
                \item \textit{Example}: Identifying at-risk customers by evaluating usage patterns.
            \end{itemize}
    \end{enumerate}
\end{frame}

\begin{frame}[fragile]
    \frametitle{Linking Decision Trees with AI Technologies}
    \begin{block}{Integration with AI}
        Recent advancements like ChatGPT leverage decision trees in Natural Language Processing (NLP):
    \end{block}
    \begin{itemize}
        \item \textbf{Data Mining Integration}
            \begin{itemize}
                \item Filters and classifies data for training AI models.
            \end{itemize}
        
        \item \textbf{Personalized Recommendations}
            \begin{itemize}
                \item Analyzes user preferences to provide tailored responses.
            \end{itemize}
    \end{itemize}
\end{frame}

\begin{frame}[fragile]
    \frametitle{Conclusion and Key Takeaways}
    \begin{block}{Conclusion}
        The evolution of decision tree algorithms enhances applicability across sectors, especially with modern AI developments, making them vital in machine learning.
    \end{block}
    \begin{itemize}
        \item Integrate with ensemble methods for better accuracy.
        \item Crucial interpretability in applications like healthcare and finance.
        \item Leveraged in AI applications like ChatGPT for data classification and recommendations.
    \end{itemize}
\end{frame}

\begin{frame}[fragile]
    \frametitle{Ethics in Decision Trees - Introduction}
    \begin{itemize}
        \item Decision trees are widely used in machine learning for their interpretability.
        \item Ethical considerations are crucial to their implementation.
    \end{itemize}
    \begin{block}{Key Points}
        \begin{itemize}
            \item Understand how decisions impact privacy and fairness.
            \item Ethical usage enhances trust in AI systems.
        \end{itemize}
    \end{block}
\end{frame}

\begin{frame}[fragile]
    \frametitle{Ethics in Decision Trees - Data Privacy}
    \begin{itemize}
        \item \textbf{Definition}: Proper handling and processing of individual personal information.
        \item \textbf{Challenges}:
        \begin{itemize}
            \item Sensitive Data Usage - Risk of exposing personal information.
            \item Compliance with regulations like GDPR requiring consent for data use.
        \end{itemize}
        \item \textbf{Example}: 
        \begin{itemize}
            \item Classifying loan applications must limit identifiers (e.g., age, race).
        \end{itemize}
    \end{itemize}
\end{frame}

\begin{frame}[fragile]
    \frametitle{Ethics in Decision Trees - Fairness and Accountability}
    \begin{itemize}
        \item \textbf{Fairness}:
        \begin{itemize}
            \item Decisions should not discriminate based on protected attributes (race, gender).
            \item \textbf{Types of Bias}:
            \begin{itemize}
                \item Algorithmic Bias - Historical data reflects systemic inequalities.
                \item Data Bias - Unrepresentative datasets can skew results.
            \end{itemize}
        \end{itemize}
        \item \textbf{Example}:
        \begin{itemize}
            \item Biased hiring algorithms favor specific backgrounds undesirably.
        \end{itemize}
    \end{itemize}
    \begin{block}{Conclusion}
        \begin{itemize}
            \item Ethical decision-making in AI technology is essential for protecting individual rights and ensuring fairness.
            \item Always ask: ``Who benefits from this decision?''
        \end{itemize}
    \end{block}
\end{frame}

\begin{frame}[fragile]
    \frametitle{Conclusion \& Key Takeaways - Part 1}
    \begin{block}{Summary of Key Points}
        \begin{enumerate}
            \item \textbf{Understanding Decision Trees}:
            \begin{itemize}
                \item A decision tree is a flowchart-like structure used for decision-making and predictive modeling.
                \item It splits data into subsets based on various conditions, making it visually intuitive and easy to interpret.
            \end{itemize}
            
            \item \textbf{Structure and Components}:
            \begin{itemize}
                \item \textbf{Nodes}: Represent decisions or outcomes.
                \item \textbf{Branches}: Indicate the flow from decision to decision.
                \item \textbf{Leaves}: Terminal nodes that represent the final outcome or classification.
            \end{itemize}
        \end{enumerate}
    \end{block}
\end{frame}

\begin{frame}[fragile]
    \frametitle{Conclusion \& Key Takeaways - Part 2}
    \begin{block}{Benefits and Applications}
        \begin{enumerate}
            \item \textbf{Benefits of Decision Trees}:
            \begin{itemize}
                \item \textbf{Simplicity}: Easily understood by non-experts.
                \item \textbf{Interpretability}: Users can easily follow the decision-making process.
                \item \textbf{No Assumptions}: Decision trees make no assumptions that can lead to misinterpretations, unlike linear models.
            \end{itemize}
            
            \item \textbf{Applications in Data Mining}:
            \begin{itemize}
                \item Used in finance, healthcare, and marketing.
                \item AI advancements, like ChatGPT, leverage decision trees for classification in natural language processing.
            \end{itemize}
        \end{enumerate}
    \end{block}
\end{frame}

\begin{frame}[fragile]
    \frametitle{Conclusion \& Key Takeaways - Part 3}
    \begin{block}{Limitations and Ethical Considerations}
        \begin{enumerate}
            \item \textbf{Limitations}:
            \begin{itemize}
                \item Prone to overfitting. Pruning techniques can help mitigate this.
                \item Sensitive to small changes in data that can affect the structure of the tree.
            \end{itemize}
            
            \item \textbf{Ethical Considerations}:
            \begin{itemize}
                \item Ethical usage is crucial, ensuring fairness and data privacy in decision tree algorithms.
            \end{itemize}
        \end{enumerate}
    \end{block}

    \begin{block}{Key Takeaways}
        \begin{itemize}
            \item Decision trees simplify complex data analysis critical to data mining.
            \item Understanding their structure allows effective utilization in practice.
            \item Continuous exploration of innovations, like ensemble methods, can significantly enhance predictive performance.
        \end{itemize}
    \end{block}
\end{frame}


\end{document}