\documentclass[aspectratio=169]{beamer}

% Theme and Color Setup
\usetheme{Madrid}
\usecolortheme{whale}
\useinnertheme{rectangles}
\useoutertheme{miniframes}

% Additional Packages
\usepackage[utf8]{inputenc}
\usepackage[T1]{fontenc}
\usepackage{graphicx}
\usepackage{booktabs}
\usepackage{listings}
\usepackage{amsmath}
\usepackage{amssymb}
\usepackage{xcolor}
\usepackage{tikz}
\usepackage{pgfplots}
\pgfplotsset{compat=1.18}
\usetikzlibrary{positioning}
\usepackage{hyperref}

% Custom Colors
\definecolor{myblue}{RGB}{31, 73, 125}
\definecolor{mygray}{RGB}{100, 100, 100}
\definecolor{mygreen}{RGB}{0, 128, 0}
\definecolor{myorange}{RGB}{230, 126, 34}
\definecolor{mycodebackground}{RGB}{245, 245, 245}

% Set Theme Colors
\setbeamercolor{structure}{fg=myblue}
\setbeamercolor{frametitle}{fg=white, bg=myblue}
\setbeamercolor{title}{fg=myblue}
\setbeamercolor{section in toc}{fg=myblue}
\setbeamercolor{item projected}{fg=white, bg=myblue}
\setbeamercolor{block title}{bg=myblue!20, fg=myblue}
\setbeamercolor{block body}{bg=myblue!10}
\setbeamercolor{alerted text}{fg=myorange}

% Set Fonts
\setbeamerfont{title}{size=\Large, series=\bfseries}
\setbeamerfont{frametitle}{size=\large, series=\bfseries}
\setbeamerfont{caption}{size=\small}
\setbeamerfont{footnote}{size=\tiny}

% Footer and Navigation Setup
\setbeamertemplate{footline}{
  \leavevmode%
  \hbox{%
  \begin{beamercolorbox}[wd=.3\paperwidth,ht=2.25ex,dp=1ex,center]{author in head/foot}%
    \usebeamerfont{author in head/foot}\insertshortauthor
  \end{beamercolorbox}%
  \begin{beamercolorbox}[wd=.5\paperwidth,ht=2.25ex,dp=1ex,center]{title in head/foot}%
    \usebeamerfont{title in head/foot}\insertshorttitle
  \end{beamercolorbox}%
  \begin{beamercolorbox}[wd=.2\paperwidth,ht=2.25ex,dp=1ex,center]{date in head/foot}%
    \usebeamerfont{date in head/foot}
    \insertframenumber{} / \inserttotalframenumber
  \end{beamercolorbox}}%
  \vskip0pt%
}

% Turn off navigation symbols
\setbeamertemplate{navigation symbols}{}

% Title Page Information
\title[Week 4: Logistic Regression]{Week 4: Logistic Regression}
\subtitle{} % Add subtitle if needed
\author[J. Smith]{John Smith, Ph.D.} % Update with actual author name
\institute[University Name]{
  Department of Computer Science\\
  University Name\\
  \vspace{0.3cm}
  Email: email@university.edu\\
  Website: www.university.edu
}
\date{\today}

% Document Start
\begin{document}

\frame{\titlepage}

\begin{frame}[fragile]
    \titlepage
\end{frame}

\begin{frame}[fragile]
    \frametitle{What is Logistic Regression?}
    \begin{itemize}
        \item Logistic regression is a statistical method for binary classification.
        \item It predicts the probability of a binary outcome based on predictor variables.
        \item Unlike linear regression, logistic regression uses the logistic function.
    \end{itemize}
    
    \begin{block}{Logistic Function}
        \begin{equation}
        P(Y=1|X) = \frac{1}{1 + e^{-(\beta_0 + \beta_1 X_1 + \beta_2 X_2 + ... + \beta_n X_n)}}
        \end{equation}
        Where:
        \begin{itemize}
            \item \( P(Y=1|X) \) is the probability that the outcome is 1.
            \item \( \beta_0 \) is the intercept.
            \item \( \beta_1, \beta_2, ..., \beta_n \) are coefficients for predictor variables.
            \item \( e \) is the base of the natural logarithm.
        \end{itemize}
    \end{block}
\end{frame}

\begin{frame}[fragile]
    \frametitle{Why Logistic Regression?}
    \begin{itemize}
        \item \textbf{Relevance in Data Mining:}
        \begin{itemize}
            \item Models complex decision boundaries.
            \item Efficient to implement and interpret.
            \item Widely applicable to classify data into two groups (e.g., spam vs. not spam).
        \end{itemize}
        \item \textbf{Practical Example:}
        \begin{itemize}
            \item Predicting whether a student will pass or fail an exam based on hours of study and attendance.
            \item Input Variables (X): Hours of Study, Attendance Rate.
            \item Output Variable (Y): Pass (1) or Fail (0).
        \end{itemize}
    \end{itemize}
    
    \begin{block}{Key Points}
        \begin{itemize}
            \item Interpretability of coefficients showing the effect of predictors.
            \item Provides probabilities rather than direct classifications.
            \item Foundational in data mining applications, enhancing AI capabilities like ChatGPT.
        \end{itemize}
    \end{block}
\end{frame}

\begin{frame}[fragile]
    \frametitle{Summary}
    \begin{itemize}
        \item Logistic regression is a powerful supervised learning technique for binary classification.
        \item It uses a probabilistic framework for predictions.
        \item Notable for its simplicity, interpretability, and relevance across fields such as healthcare, marketing, and finance.
    \end{itemize}
    \begin{block}{Outline}
        \begin{enumerate}
            \item Definition and Explanation of Logistic Regression
            \item Formula of the Logistic Function
            \item Importance and Relevance in Data Mining
            \item Practical Example of Application
            \item Key Points and Summary
        \end{enumerate}
    \end{block}
\end{frame}

\begin{frame}[fragile]
    \frametitle{Motivation for Logistic Regression - Understanding Classification Problems}
    \begin{itemize}
        \item \textbf{Definition:} Classification problems involve predicting categorical outcomes based on input features.
        \begin{itemize}
            \item Examples: 
            \begin{itemize}
                \item Spam detection (spam or not spam)
                \item Medical diagnosis (disease or no disease)
                \item Customer churn prediction (churn or retain)
            \end{itemize}
        \end{itemize}
        \item \textbf{Importance in Data Mining:} 
        \begin{itemize}
            \item Essential for making informed decisions based on data insights.
            \item Example: Correct identification of fraudulent transactions can save banks millions.
        \end{itemize}
    \end{itemize}
\end{frame}

\begin{frame}[fragile]
    \frametitle{Motivation for Logistic Regression - The Role of Logistic Regression}
    \begin{itemize}
        \item \textbf{What is Logistic Regression?}
        \begin{itemize}
            \item A statistical method for binary classification, predicting the probability of a binary outcome (0 or 1).
        \end{itemize}
        \item \textbf{Why Use Logistic Regression?}
        \begin{itemize}
            \item \textbf{Ease of Interpretation:} Coefficients indicate how input variables affect outcome probabilities.
            \item \textbf{Probabilistic Framework:} Outputs constrained between 0 and 1 through the logistic function.
        \end{itemize}
    \end{itemize}
\end{frame}

\begin{frame}[fragile]
    \frametitle{Motivation for Logistic Regression - Practical Examples}
    \begin{enumerate}
        \item \textbf{Medical Diagnosis:}
        \begin{itemize}
            \item Predicting disease presence based on various symptoms.
            \item Example features: age, blood pressure, cholesterol levels.
        \end{itemize}
        \item \textbf{Marketing Campaigns:}
        \begin{itemize}
            \item Deciding customer responsiveness based on purchasing behavior.
            \item Example features: age, income, past purchases.
        \end{itemize}
        \item \textbf{Credit Scoring:}
        \begin{itemize}
            \item Assessing loan applicant credit risks based on their financial history.
            \item Example features: credit history, income, employment status.
        \end{itemize}
    \end{enumerate}
\end{frame}

\begin{frame}[fragile]
    \frametitle{Logistic Function}
    \begin{block}{Definition}
        The logistic function is defined as:
        \begin{equation}
            P(Y=1 | X) = \frac{1}{1 + e^{-(\beta_0 + \beta_1 X_1 + \beta_2 X_2 + \ldots + \beta_n X_n)}}
        \end{equation}
    \end{block}
    \begin{itemize}
        \item Where:
        \begin{itemize}
            \item \( P(Y=1 | X) \) is the predicted probability of the positive class.
            \item \( X_1, X_2, \ldots, X_n \) are the input features.
            \item \( \beta_0, \beta_1, \ldots, \beta_n \) are the coefficients of the model.
        \end{itemize}
    \end{itemize}
\end{frame}

\begin{frame}[fragile]
    \frametitle{Conclusion and Transition}
    \begin{itemize}
        \item Logistic regression is foundational in classification contexts due to its simplicity and interpretability.
        \item It transforms linear combinations of input features to produce probabilities for binary outcomes.
        \item Understanding its motivation aids in transitioning to specific concepts like binary classification and its real-world applications in AI and data-driven decision-making.
    \end{itemize}
\end{frame}

\begin{frame}[fragile]
    \frametitle{Understanding Binary Classification - Overview}
    \begin{block}{What is Binary Classification?}
        Binary classification is a predictive modeling technique that categorizes data into two distinct outcomes, typically represented as 0 and 1, or "Yes" and "No." This technique is fundamental across various domains such as medical diagnosis, spam detection, and customer churn prediction.
    \end{block}
    \begin{itemize}
        \item **Two Classes**: Outcomes limited to two categories (e.g., healthy/unhealthy, spam/not spam).
        \item **Prediction Goal**: Develop a model that predicts the class label for new samples based on input features.
    \end{itemize}
\end{frame}

\begin{frame}[fragile]
    \frametitle{Understanding Binary Classification - Logistic Regression}
    \begin{block}{How Logistic Regression Fits in}
        Logistic Regression is suited for binary classification. It maps input features to the probability of belonging to a particular class.
    \end{block}
    \begin{itemize}
        \item **Mapping Features to Outcomes**: Uses a logistic function to convert linear combinations of input features to probabilities between 0 and 1.
    \end{itemize}
    \begin{equation}
        P(Y=1 | X) = \frac{1}{1 + e^{-(\beta_0 + \beta_1X_1 + \beta_2X_2 + \ldots + \beta_nX_n)}}
    \end{equation}
\end{frame}

\begin{frame}[fragile]
    \frametitle{Understanding Binary Classification - Example Application}
    \begin{block}{Example Application: Student Outcomes}
        Predicting whether a student will pass (1) or fail (0) based on study hours and attendance.
    \end{block}
    \begin{enumerate}
        \item **Input Features**: 
            \begin{itemize}
                \item Study Hours (e.g., 2, 4, 6)
                \item Attendance Rate (e.g., 90\%, 75\%, 50\%)
            \end{itemize}
        \item **Model Development**:
            \begin{itemize}
                \item Analyze data from previous students.
                \item Output probability score (e.g., 0.8) indicating the likelihood of passing.
            \end{itemize}
        \item **Decision Threshold**: Set at 0.5 to determine pass or fail.
    \end{enumerate}
\end{frame}

\begin{frame}[fragile]
    \frametitle{Understanding Binary Classification - Key Points}
    \begin{itemize}
        \item Binary classification involves two outcomes.
        \item Logistic Regression uses the logistic function to bind probabilities between 0 and 1.
        \item Proper threshold selection is crucial for decision-making in predictions.
        \item Logistic regression is interpretable and efficient, making it popular for binary classification tasks.
    \end{itemize}
\end{frame}

\begin{frame}[fragile]
    \frametitle{Logistic Function - Introduction}
    \begin{block}{Definition}
        The logistic function is a key concept in statistics and machine learning for modeling binary outcomes. It predicts probabilities that range between 0 and 1, making it essential in binary classification scenarios.
    \end{block}
    
    \begin{itemize}
        \item Models binary outcomes effectively
        \item Applicable in various fields, such as healthcare and finance
    \end{itemize}
    
    \begin{block}{Outline}
        \begin{itemize}
            \item What is the Logistic Function?
            \item Understanding the Output
            \item Purpose in Binary Classification
            \item Key Properties
            \item Example
        \end{itemize}
    \end{block}
\end{frame}

\begin{frame}[fragile]
    \frametitle{Logistic Function - Mathematical Definition}
    The logistic function is mathematically defined as:
    \begin{equation}
        f(x) = \frac{1}{1 + e^{-x}}
    \end{equation}

    Where:
    \begin{itemize}
        \item \( e \) is approximately 2.71828 (Euler's number).
        \item \( x \) can be any linear combination of features, e.g., 
        \[
        x = \beta_0 + \beta_1 x_1 + \beta_2 x_2 + ... + \beta_n x_n
        \]
    \end{itemize}
    
    \begin{block}{Key Points}
        \begin{itemize}
            \item Output \( f(x) \) ranges between 0 and 1
            \item Ideal for modeling probabilities
        \end{itemize}
    \end{block}
\end{frame}

\begin{frame}[fragile]
    \frametitle{Logistic Function - Application and Example}
    The logistic function is vital in binary classification:
    \begin{itemize}
        \item Models the likelihood of an input belonging to a particular class.
        \item Example: Predicting if a student passes or fails based on study hours.
    \end{itemize}

    \begin{equation}
        f(x) = \frac{1}{1 + e^{-(2 + 0.5 \cdot \text{hours})}}
    \end{equation}

    For example, for 6 hours of study:
    \begin{equation}
        f(6) = \frac{1}{1 + e^{-(2 + 0.5 \cdot 6)}} \approx 0.88
    \end{equation}
    \begin{block}{Interpretation}
        This indicates an 88\% probability that the student will pass the exam.
    \end{block}
\end{frame}

\begin{frame}[fragile]
    \frametitle{Modeling with Logistic Regression - Overview}
    \begin{block}{Overview of Logistic Regression}
        Logistic Regression is a statistical method used for binary classification problems—where outcomes are categorical and can take one of two possible values (e.g., success/failure, yes/no).
    \end{block}
    
    \begin{block}{Why Use Logistic Regression?}
        \begin{itemize}
            \item \textbf{Interpretability:} Parameters can be interpreted in terms of odds ratios, which aids in decision-making.
            \item \textbf{Non-linearity:} Captures non-linear relationships using the logistic function.
            \item \textbf{Widely Applicable:} Utilized across fields such as healthcare, marketing, and social sciences.
        \end{itemize}
    \end{block}
\end{frame}

\begin{frame}[fragile]
    \frametitle{Fitting a Logistic Regression Model}
    \begin{block}{Steps to Fit a Model}
        To fit a logistic regression model to training data, follow these steps:
        \begin{enumerate}
            \item \textbf{Select Variables:} Identify independent variables (features) and the dependent variable (outcome).
            \item \textbf{Model Specification:} The logistic model is represented as:
            \begin{equation}
                P(Y=1|X) = \frac{1}{1 + e^{-(\beta_0 + \beta_1 X_1 + \beta_2 X_2 + ... + \beta_n X_n)}}
            \end{equation}
            Where \(P(Y=1|X)\) is the probability of the outcome being 1 given features \(X\).
            \item \textbf{Estimate Parameters:} Parameters (\(\beta\)) are estimated using maximum likelihood estimation (MLE).
        \end{enumerate}
    \end{block}
\end{frame}

\begin{frame}[fragile]
    \frametitle{Example of Logistic Regression Modeling}
    \begin{block}{Scenario: Predicting Customer Purchases}
        \begin{itemize}
            \item \textbf{Data:} 
            \begin{itemize}
                \item Income (X1): Continuous variable in dollars
                \item Age (X2): Continuous variable in years
                \item Purchase (Y): 1 (Yes) or 0 (No)
            \end{itemize}
            \item \textbf{Model Fit:}
            \[
            P(Y=1) = \frac{1}{1 + e^{-(\beta_0 + \beta_1 \cdot \text{Income} + \beta_2 \cdot \text{Age})}}
            \]
            \item \textbf{Interpreting Coefficients:} For example, having \(\beta_1=0.03\) and \(\beta_2=-0.02\) indicates how income and age impact the likelihood of purchase.
        \end{itemize}
    \end{block}
    
    \begin{block}{Key Points to Remember}
        \begin{itemize}
            \item Logistic Regression predicts probabilities and is suited for binary outcomes.
            \item The logistic function transforms linear combinations into probabilities.
            \item Model performance can be evaluated using metrics like accuracy and AUC-ROC curve.
        \end{itemize}
    \end{block}
\end{frame}

\begin{frame}[fragile]
    \frametitle{Cost Function and Optimization - Part 1}

    \textbf{1. Understanding the Cost Function in Logistic Regression}
    
    The cost function, often referred to as the loss function, measures how well the logistic regression model predicts the target variable.
    We use the \textbf{Binary Cross-Entropy Loss} (or Log Loss) as the cost function because it is particularly suited for binary classification problems.
    
    \begin{block}{Formula}
        \begin{equation}
        J(\theta) = -\frac{1}{m} \sum_{i=1}^{m} [y^{(i)} \cdot \log(h_\theta(x^{(i)})) + (1 - y^{(i)}) \cdot \log(1 - h_\theta(x^{(i)}))]
        \end{equation}
    \end{block}
    
    \textbf{Where:}
    \begin{itemize}
        \item $m$ = number of training examples
        \item $y^{(i)}$ = true label (0 or 1) for the $i^{th}$ example
        \item $h_\theta(x) = \frac{1}{1 + e^{-\theta^Tx}}$ = hypothesis function (sigmoid function)
    \end{itemize}
    
    \textbf{Key Points:}
    \begin{itemize}
        \item The cost function quantifies the difference between the predicted probabilities $h_\theta(x)$ and the true labels $y$.
        \item A lower value of the cost function indicates a better fit of the model to the training data.
    \end{itemize}
\end{frame}

\begin{frame}[fragile]
    \frametitle{Cost Function and Optimization - Part 2}

    \textbf{2. Optimization Method: Gradient Descent}

    \textbf{What is Gradient Descent?}
    
    Gradient Descent is an iterative optimization algorithm used to minimize the cost function. It updates the model parameters $\theta$ in the direction that reduces the cost.

    \begin{block}{Formula for Updating Parameters}
        \begin{equation}
        \theta := \theta - \alpha \cdot \nabla J(\theta)
        \end{equation}
    \end{block}
    
    \textbf{Where:}
    \begin{itemize}
        \item $\alpha$ = learning rate (controls how much to update the parameters)
        \item $\nabla J(\theta) = \frac{\partial J(\theta)}{\partial \theta}$ = gradient of the cost function
    \end{itemize}

    \textbf{Steps in Gradient Descent:}
    \begin{enumerate}
        \item Initialization: Start with random values for $\theta$.
        \item Calculate the Cost: Compute $J(\theta)$ using the current parameters.
        \item Compute the Gradient: Calculate the gradients of the cost function.
        \item Update Parameters: Adjust the parameters using the update rule.
        \item Repeat: Continue until convergence (when changes are negligible).
    \end{enumerate}
\end{frame}

\begin{frame}[fragile]
    \frametitle{Cost Function and Optimization - Part 3}

    \textbf{Example:}
    
    Imagine you have a binary classification problem predicting whether an email is spam (1) or not spam (0). 
    By minimizing the cost function through gradient descent, you can optimize the logistic regression model to classify emails with increasing accuracy.

    \textbf{3. Conclusion \& Key Takeaways}
    \begin{itemize}
        \item The cost function in logistic regression quantifies model performance and is minimized using gradient descent.
        \item Gradient descent iteratively adjusts model parameters to find the optimal values that result in the lowest cost.
        \item Understanding these concepts is crucial for effectively implementing and improving logistic regression models.
    \end{itemize}
    
    \textbf{Next Topic:} In the following slide, we will delve into how to make predictions using the trained logistic regression model and how to interpret the results.
\end{frame}

\begin{frame}[fragile]
  \frametitle{Making Predictions - Introduction}
  
  \begin{block}{Overview}
    Logistic regression is a statistical method for binary classification. It predicts the probability of an outcome belonging to one of two categories based on predictor variables.
  \end{block}
  
  \begin{itemize}
    \item Used in binary classification problems.
    \item Outputs the probability of an input belonging to a category.
  \end{itemize}
  
\end{frame}

\begin{frame}[fragile]
  \frametitle{Making Predictions - Core Concept}
  
  \begin{block}{Logistic Function}
    The logistic function is defined as:
    \begin{equation}
      P(Y=1 | X) = \frac{1}{1 + e^{-(\beta_0 + \beta_1X_1 + \beta_2X_2 + \ldots + \beta_nX_n)}}
    \end{equation}
    Where:
    \begin{itemize}
      \item \(P(Y=1 | X)\) is the predicted probability.
      \item \(\beta_0\) is the intercept.
      \item \(\beta_1, \beta_2, \ldots, \beta_n\) are the coefficients.
    \end{itemize}
  \end{block}
  
\end{frame}

\begin{frame}[fragile]
  \frametitle{Making Predictions - Steps}
  
  \begin{enumerate}
    \item \textbf{Gather Predictor Variables:} Collect necessary features for each observation.
    \item \textbf{Calculate Logit:} Compute the logit as:
      \begin{equation}
        z = \beta_0 + \beta_1X_1 + \beta_2X_2 + \ldots + \beta_nX_n
      \end{equation}
    \item \textbf{Apply Logistic Function:} Convert logit to probability:
      \begin{equation}
        P = \frac{1}{1 + e^{-z}}
      \end{equation}
    \item \textbf{Decision Boundary:} Set a threshold (e.g., 0.5) for classification.
  \end{enumerate}

\end{frame}

\begin{frame}[fragile]
  \frametitle{Making Predictions - Example}
  
  \begin{exampleblock}{Example}
    Consider a model predicting a student's exam pass status based on hours studied:
    \begin{itemize}
      \item Coefficients: \(\beta_0 = -4\) and \(\beta_1 = 0.5\)
      \item For a student studying 8 hours \( (X = 8) \):
        \begin{equation}
          z = -4 + 0.5 \times 8 = 0
        \end{equation}
        Applying the logistic function:
        \begin{equation}
          P = \frac{1}{1 + e^{0}} = 0.5
        \end{equation}
    \end{itemize}
    The predicted probability is 0.5.
  \end{exampleblock}

\end{frame}

\begin{frame}[fragile]
  \frametitle{Making Predictions - Interpretation}
  
  \begin{block}{Output Interpretation}
    \begin{itemize}
      \item \textbf{Probabilities:} Represents the likelihood of class membership.
      \item \textbf{Odds:} Can be calculated from probability:
        \begin{equation}
          \text{Odds} = \frac{P}{1-P}
        \end{equation}
    \end{itemize}
  \end{block}
  
  \begin{block}{Key Points}
    \begin{itemize}
      \item Provides a probabilistic framework for binary classification.
      \item Helps in assessing prediction certainty.
      \item Threshold selection can influence prediction outcomes.
    \end{itemize}
  \end{block}

\end{frame}

\begin{frame}[fragile]
  \frametitle{Making Predictions - Conclusion}
  
  Understanding how predictions are made using logistic regression enables data scientists and analysts to implement this powerful technique effectively in real-world binary classification tasks, emphasizing the importance of output interpretation and classification threshold selection.

\end{frame}

\begin{frame}[fragile]
    \frametitle{Performance Metrics - Overview}
    \begin{block}{Understanding Performance Metrics in Logistic Regression}
        Logistic Regression is a powerful statistical method used for binary classification problems. However, simply building a model is not enough; we must evaluate its performance to ensure its effectiveness. 
        Below are key performance metrics specific to logistic regression, crucial for interpreting model results accurately.
    \end{block}
\end{frame}

\begin{frame}[fragile]
    \frametitle{Performance Metrics - Accuracy}
    \begin{block}{1. Accuracy}
        \begin{itemize}
            \item \textbf{Definition}: Measures the proportion of correct predictions in relation to the total predictions made.
            \begin{equation}
                \text{Accuracy} = \frac{\text{TP} + \text{TN}}{\text{TP} + \text{TN} + \text{FP} + \text{FN}}
            \end{equation}
            \item Where:
            \begin{itemize}
                \item \textbf{TP}: True Positives (correct positive predictions)
                \item \textbf{TN}: True Negatives (correct negative predictions)
                \item \textbf{FP}: False Positives (incorrect positive predictions)
                \item \textbf{FN}: False Negatives (incorrect negative predictions)
            \end{itemize}
            \item \textbf{Example}: In a medical diagnosis model, if out of 100 patients, 85 were correctly classified, the accuracy would be:
            \begin{equation}
                \text{Accuracy} = \frac{85}{100} = 0.85 \text{ or } 85\%
            \end{equation}
        \end{itemize}
    \end{block}
\end{frame}

\begin{frame}[fragile]
    \frametitle{Performance Metrics - Precision and Recall}
    \begin{block}{2. Precision}
        \begin{itemize}
            \item \textbf{Definition}: Measures the accuracy of the positive predictions made by the model.
            \begin{equation}
                \text{Precision} = \frac{\text{TP}}{\text{TP} + \text{FP}}
            \end{equation}
            \item \textbf{Example}: If the model correctly identified 30 out of 40 positive cases, the precision would be:
            \begin{equation}
                \text{Precision} = \frac{30}{30 + 10} = \frac{30}{40} = 0.75 \text{ or } 75\%
            \end{equation}
        \end{itemize}
    \end{block}

    \begin{block}{3. Recall (Sensitivity)}
        \begin{itemize}
            \item \textbf{Definition}: Calculates the ability of the model to find all relevant cases (all actual positives).
            \begin{equation}
                \text{Recall} = \frac{\text{TP}}{\text{TP} + \text{FN}}
            \end{equation}
            \item \textbf{Example}: If out of 50 actual positive cases, the model identified 40 correctly, the recall would be:
            \begin{equation}
                \text{Recall} = \frac{40}{40 + 10} = \frac{40}{50} = 0.80 \text{ or } 80\%
            \end{equation}
        \end{itemize}
    \end{block}
\end{frame}

\begin{frame}[fragile]
    \frametitle{Performance Metrics - F1-Score and Key Takeaways}
    \begin{block}{4. F1-Score}
        \begin{itemize}
            \item \textbf{Definition}: The harmonic mean of precision and recall, providing a balance between the two. Useful for imbalanced datasets.
            \begin{equation}
                \text{F1-Score} = 2 \times \frac{\text{Precision} \times \text{Recall}}{\text{Precision} + \text{Recall}}
            \end{equation}
            \item \textbf{Example}: If a model has a precision of 75\% and a recall of 80\%, the F1-Score would be:
            \begin{equation}
                \text{F1-Score} \approx 0.769 \text{ or } 76.9\%
            \end{equation}
        \end{itemize}
    \end{block}
    
    \begin{block}{Key Points to Emphasize}
        \begin{itemize}
            \item No Single Metric is Sufficient: Use multiple metrics for a comprehensive evaluation.
            \item Trade-offs: Understanding the trade-offs between precision and recall is essential.
            \item Context Matters: The choice of metric can vary based on the problem context.
        \end{itemize}
    \end{block}
\end{frame}

\begin{frame}[fragile]
    \frametitle{ROC Curve and AUC - Introduction}
    \begin{itemize}
        \item \textbf{Definition}: The Receiver Operating Characteristic (ROC) curve is a graphical representation that illustrates the diagnostic ability of a binary classifier system as its discrimination threshold is varied.
        \item \textbf{Importance}: It allows us to visualize how different thresholds affect the performance of a classifier in distinguishing between classes.
    \end{itemize}
\end{frame}

\begin{frame}[fragile]
    \frametitle{ROC Curve - Understanding the Axes}
    \begin{itemize}
        \item \textbf{Axes}:
        \begin{itemize}
            \item \textbf{X-axis}: False Positive Rate (FPR) - the proportion of actual negatives that are incorrectly identified as positives \\
            $FPR = \frac{FP}{(FP + TN)}$
            \item \textbf{Y-axis}: True Positive Rate (TPR) (also known as Sensitivity) - the proportion of actual positives that are correctly identified \\
            $TPR = \frac{TP}{(TP + FN)}$
        \end{itemize}
        \item \textbf{Interpretation}: A model with a curve closer to the top left corner indicates better performance.
    \end{itemize}
\end{frame}

\begin{frame}[fragile]
    \frametitle{Area Under the Curve (AUC)}
    \begin{itemize}
        \item \textbf{Definition}: The AUC quantifies the overall performance of a classifier; it ranges from 0 to 1.
        \begin{itemize}
            \item \textbf{AUC = 1}: Perfect model; correctly classifies all instances.
            \item \textbf{AUC = 0.5}: No discrimination; model performs no better than random guessing.
            \item \textbf{AUC < 0.5}: Model is performing worse than random guessing.
        \end{itemize}
        \item \textbf{Significance}: A higher AUC value indicates a better ability of the model to distinguish between the positive and negative classes.
    \end{itemize}
\end{frame}

\begin{frame}[fragile]
    \frametitle{Key Points and Example}
    \begin{itemize}
        \item \textbf{Key Points}:
        \begin{itemize}
            \item ROC curves are especially useful for imbalanced datasets, where accuracy may be misleading.
            \item AUC provides a robust performance measure across all classification thresholds.
        \end{itemize}
        \item \textbf{Example}: Consider a medical diagnostic test. If the AUC is 0.85, it suggests the model is effective at distinguishing between diseased and non-diseased patients.
    \end{itemize}
\end{frame}

\begin{frame}[fragile]
    \frametitle{Assumptions of Logistic Regression - Introduction}
    Logistic regression is a widely used statistical method for binary classification problems. 
    To ensure the validity and effectiveness of a logistic regression model, certain key assumptions must be met. 
    Understanding these assumptions is crucial for interpreting model results correctly and for making informed decisions based on predictive analytics.
\end{frame}

\begin{frame}[fragile]
    \frametitle{Assumptions of Logistic Regression - Key Assumptions}
    \begin{enumerate}
        \item \textbf{Binary Outcome Variable}
            \begin{itemize}
                \item The dependent variable must be binary or dichotomous (e.g., success/failure).
                \item \textit{Example:} Predicting whether a patient has a disease (Yes or No).
            \end{itemize}
        \item \textbf{Independence of Observations}
            \begin{itemize}
                \item Observations should be independent of one another.
                \item \textit{Example:} Responses from individual patients in a recovery study should not affect each other.
            \end{itemize}
    \end{enumerate}
\end{frame}

\begin{frame}[fragile]
    \frametitle{Assumptions of Logistic Regression - Continued}
    \begin{enumerate}
        \setcounter{enumi}{2} % Continue numbering from the previous frame
        \item \textbf{No Multicollinearity}
            \begin{itemize}
                \item Independent variables should not be highly correlated.
                \item \textit{Example:} Including both height and weight as predictors may lead to multicollinearity.
            \end{itemize}
        \item \textbf{Linearity in the Logit}
            \begin{itemize}
                \item There must be a linear relationship between the logit of the outcome and the independent variables.
                \item \textit{Formula:} 
                \[
                \text{Logit}(p) = \ln\left(\frac{p}{1-p}\right) = \beta_0 + \beta_1X_1 + \beta_2X_2 + \ldots + \beta_nX_n
                \]
                \item \textit{Example:} The log-odds should reflect a linearity with respect to predictors (e.g., age, income).
            \end{itemize}
    \end{enumerate}
\end{frame}

\begin{frame}[fragile]
    \frametitle{Assumptions of Logistic Regression - Summary}
    \begin{enumerate}
        \setcounter{enumi}{4} % Continue numbering from the previous frame
        \item \textbf{Large Sample Size}
            \begin{itemize}
                \item Sufficient sample size is required for reliable estimates, typically 10 events per predictor.
                \item \textit{Example:} For 3 predictors, at least 30 events (successes) are recommended.
            \end{itemize}
    \end{enumerate}
    
    \begin{block}{Key Points to Emphasize}
        \begin{itemize}
            \item Ensuring these assumptions are met helps prevent bias and improves model performance.
            \item Violations can lead to unreliable estimates and predictions.
            \item Diagnostic tests and visualizations can help check these assumptions.
        \end{itemize}
    \end{block}

    \begin{block}{Conclusion}
        Understanding these assumptions is essential for building robust models and drawing accurate conclusions from analyses.
    \end{block}
\end{frame}

\begin{frame}[fragile]
    \frametitle{Common Applications of Logistic Regression}
    Logistic regression is a powerful statistical tool utilized across various industries for predicting binary outcomes. Here we explore several practical applications of logistic regression in key fields:
\end{frame}

\begin{frame}[fragile]
    \frametitle{Health Care Applications}
    \begin{block}{Predictive Analytics}
        Logistic regression is frequently used to predict the likelihood of diseases or health conditions based on patient data.
    \end{block}

    \begin{itemize}
        \item \textbf{Example: Diabetes Prediction}
        \begin{itemize}
            \item Analysis of factors like age, BMI, blood pressure, and glucose levels to predict the probability of developing diabetes.
        \end{itemize}

        \item \textbf{Key Points:}
        \begin{itemize}
            \item Features include lifestyle factors, family history, and clinical measurements.
            \item Outcome: Probability of developing a disease (0 = No, 1 = Yes).
        \end{itemize}
    \end{itemize}
\end{frame}

\begin{frame}[fragile]
    \frametitle{Finance Applications}
    \begin{block}{Credit Scoring}
        Financial institutions apply logistic regression to evaluate the creditworthiness of loan applicants.
    \end{block}
    
    \begin{itemize}
        \item \textbf{Example: Evaluation of Loan Applicants}
        \begin{itemize}
            \item Analyzing variables such as income, credit history, and employment status to determine default risk.
        \end{itemize}

        \item \textbf{Key Points:}
        \begin{itemize}
            \item Outcomes labeled as "Default" (1) or "Not Default" (0).
            \item Aids in risk management and risk-based pricing for loans.
        \end{itemize}
    \end{itemize}
\end{frame}

\begin{frame}[fragile]
    \frametitle{Marketing and Social Media Applications}
    
    \begin{block}{Customer Retention}
        Marketers use logistic regression to analyze customer behavior and predict churn.
    \end{block}

    \begin{itemize}
        \item \textbf{Example: Telecom Company Analysis}
        \begin{itemize}
            \item Examining usage data, customer service interactions, and payment history to identify at-risk customers.
        \end{itemize}

        \item \textbf{Key Points:}
        \begin{itemize}
            \item Dependent variable: Churn (1 = Churned, 0 = Retained).
            \item Optimizes marketing efforts and reduces acquisition costs.
        \end{itemize}
    \end{itemize}

    \begin{block}{Content Engagement}
        Social media platforms also utilize logistic regression to analyze user interactions.
    \end{block}

    \begin{itemize}
        \item \textbf{Example: User Engagement Prediction}
        \begin{itemize}
            \item Variables like post type, timing, and demographics predict engagement likelihood.
        \end{itemize}

        \item \textbf{Key Points:}
        \begin{itemize}
            \item Outcome: Engagement (1 = Engaged, 0 = Not Engaged).
            \item Supports targeted content strategies.
        \end{itemize}
    \end{itemize}
\end{frame}

\begin{frame}[fragile]
    \frametitle{Conclusion and Recap}
    \begin{block}{Conclusion}
        Logistic regression is a versatile tool capable of delivering valuable insights across diverse fields. It empowers practitioners by informing decision-making processes.
    \end{block}

    \begin{block}{Quick Formula Recap}
        The logistic regression model is represented by the following logistic function:
        \begin{equation}
            P(Y = 1 | X) = \frac{1}{1 + e^{-(\beta_0 + \beta_1X_1 + \beta_2X_2 + ... + \beta_nX_n)}}
        \end{equation}
        Where:
        \begin{itemize}
            \item \( P(Y = 1) \) = Probability of the outcome occurring.
            \item \( \beta_0 \) = Intercept.
            \item \( \beta_i \) = Coefficients of predictors \( X_i \).
        \end{itemize}
    \end{block}
\end{frame}

\begin{frame}
    \frametitle{Case Study: Logistic Regression in Action}
    \begin{block}{Overview}
        In this case study, we will demonstrate the application of logistic regression on a dataset to predict customer churn for a telecommunications company.
    \end{block}
\end{frame}

\begin{frame}
    \frametitle{Introduction to Logistic Regression}
    \begin{itemize}
        \item Logistic regression is used for binary classification problems.
        \item It estimates the probability that a given input belongs to a particular class.
        \item Output variable is categorical with two possible outcomes (e.g., success/failure).
    \end{itemize}
\end{frame}

\begin{frame}
    \frametitle{Case Study Context: Predicting Customer Churn}
    \begin{itemize}
        \item \textbf{Dataset}: Telecom company dataset containing customer info and their churn status.
        \item \textbf{Objective}: Predict customer churn based on:
        \begin{itemize}
            \item Age
            \item Monthly charges
            \item Customer service calls
            \item Contract type
        \end{itemize}
    \end{itemize}
\end{frame}

\begin{frame}
    \frametitle{Step 1: Data Preparation}
    \begin{itemize}
        \item \textbf{Data Cleaning}: Remove duplicates and handle missing values.
        \item \textbf{Feature Selection}: Identify significant predictors (e.g., Monthly Charges).
        \item \textbf{Encoding Categorical Variables}: Convert categories into numerical formats.
    \end{itemize}

    \begin{block}{Data Preparation Example}
    Before:
    \begin{tabular}{|c|c|c|}
        \hline
        Age & Monthly Charges & Churn \\
        \hline
        25  & \$70            & 0     \\
        30  & \$50            & 1     \\
        \hline
    \end{tabular}

    After (One-Hot Encoding):
    \begin{tabular}{|c|c|c|c|}
        \hline
        Age & Monthly Charges & Contract (One-Hot) & Churn \\
        \hline
        25  & 70             & Monthly            & 0     \\
        30  & 50             & One-Year           & 1     \\
        \hline
    \end{tabular}
    \end{block}
\end{frame}

\begin{frame}[fragile]
    \frametitle{Step 2: Model Building}
    \begin{itemize}
        \item \textbf{Logistic Regression Formula}:
        \begin{equation}
            P(Y=1|X) = \frac{1}{1 + e^{-(\beta_0 + \beta_1X_1 + \beta_2X_2 + ... + \beta_nX_n)}}
        \end{equation}
        Where:
        \begin{itemize}
            \item \( P(Y=1|X) \) = predicted probability of churn
            \item \( \beta_0 \) = intercept
            \item \( \beta_1, \ldots, \beta_n \) = coefficients for each feature
        \end{itemize}
        \item \textbf{Fitting the Model}:
        Use libraries like `scikit-learn` to fit the model.
    \end{itemize}

    \begin{block}{Code Snippet Example}
    \begin{lstlisting}[language=Python]
from sklearn.model_selection import train_test_split
from sklearn.linear_model import LogisticRegression
import pandas as pd

data = pd.read_csv('customer_data.csv')
X = data[['age', 'monthly_charges', 'service_calls']]  # Features
y = data['churn']  # Target

X_train, X_test, y_train, y_test = train_test_split(X, y, test_size=0.3, random_state=42)
model = LogisticRegression()
model.fit(X_train, y_train)
    \end{lstlisting}
    \end{block}
\end{frame}

\begin{frame}
    \frametitle{Step 3: Model Evaluation}
    \begin{itemize}
        \item \textbf{Metrics}:
        \begin{itemize}
            \item \textbf{Accuracy}: Proportion of true results within total population.
            \item \textbf{Confusion Matrix}: 
            \begin{center}
            \begin{tabular}{|c|c|c|}
                \hline
                & Predicted Positive & Predicted Negative \\
                \hline
                Actual Positive & TP & FN \\
                \hline
                Actual Negative & FP & TN \\
                \hline
            \end{tabular}
            \end{center}
            \item \textbf{ROC Curve \& AUC}: Measures the trade-off between sensitivity and specificity.
        \end{itemize}
    \end{itemize}
\end{frame}

\begin{frame}
    \frametitle{Conclusion and Key Points}
    \begin{itemize}
        \item Logistic regression helps understand relationships between predictors and binary outcomes.
        \item Effective data preparation is vital for reliable predictions.
        \item Evaluation metrics are crucial for assessing model performance.
    \end{itemize}
    \begin{block}{Summary}
        By predicting customer churn, businesses can significantly improve their retention strategies using insights from data.
    \end{block}
\end{frame}

\begin{frame}[fragile]
    \frametitle{Handling Multiclass Classification - Introduction}
    \begin{itemize}
        \item Logistic regression is typically used for binary classification.
        \item Real-world problems often require multiclass classification.
        \item Extending logistic regression to address multiclass scenarios is essential.
    \end{itemize}
\end{frame}

\begin{frame}[fragile]
    \frametitle{One-vs-Rest (OvR) Approach}
    \begin{block}{Key Concept: One-vs-Rest}
        \begin{enumerate}
            \item **Divide the Classes**: Train \( K \) models for \( K \) classes.
            \item **Model Training**: Each model \( m_k \) treats its class as positive (1) and others as negative (0).
                \begin{itemize}
                    \item Example for classes {A, B, C}:
                    \begin{itemize}
                        \item Model 1: Class A vs. {B, C}
                        \item Model 2: Class B vs. {A, C}
                        \item Model 3: Class C vs. {A, B}
                    \end{itemize}
                \end{itemize}
            \item **Making Predictions**: Predict scores from all models; class with highest score is chosen.
        \end{enumerate}
    \end{block}
\end{frame}

\begin{frame}[fragile]
    \frametitle{Illustration of One-vs-Rest}
    \begin{itemize}
        \item Consider 3 classes: Apples, Oranges, Bananas.
        \item Model Training:
        \begin{enumerate}
            \item Model 1: Distinguish Apples (1) from Non-Apples (0)
            \item Model 2: Distinguish Oranges (1) from Non-Oranges (0)
            \item Model 3: Distinguish Bananas (1) from Non-Bananas (0)
        \end{enumerate}
        \item **Prediction Process**: For predictions:
        \begin{itemize}
            \item Apples: 0.80
            \item Oranges: 0.15
            \item Bananas: 0.05
        \end{itemize}
        \item Predicted class is **Apples** (highest probability).
    \end{itemize}
\end{frame}

\begin{frame}[fragile]
    \frametitle{Key Points and Conclusion}
    \begin{itemize}
        \item **Flexibility**: OvR enables binary classifiers to handle multiclass problems.
        \item **Interpretable Probabilities**: Insights from predicted probabilities aid in understanding classifications.
        \item **Scalability**: OvR can be costly in terms of computation with many classes.
    \end{itemize}
    \begin{block}{Conclusion}
        Extending logistic regression with techniques like OvR enhances its application in the data scientist's toolkit.
    \end{block}
\end{frame}

\begin{frame}[fragile]
    \frametitle{Formula Recap}
    For the logistic regression probability \( P(y=k|x) \) for class \( k \):
    \begin{equation}
        P(y = k | x) = \frac{e^{z_k}}{ \sum_{j=1}^{K} e^{z_j} }
    \end{equation}
    where \( z_k \) is the linear combination of weights and features for class \( k \).
\end{frame}

\begin{frame}[fragile]
    \frametitle{Next Steps}
    \begin{itemize}
        \item Next, we will discuss **Challenges and Limitations** of implementing logistic regression for multiclass classification.
        \item Common pitfalls to avoid during model building will also be addressed.
    \end{itemize}
\end{frame}

\begin{frame}[fragile]
  \frametitle{Overview of Challenges and Limitations}
  Logistic regression is a powerful statistical method for binary classification. However, it has several limitations and challenges:

  \begin{itemize}
    \item Understanding these challenges is crucial for refining models and improving predictions.
    \item Avoiding common pitfalls can lead to more reliable outcomes.
  \end{itemize}
\end{frame}

\begin{frame}[fragile]
  \frametitle{Model Assumptions}
  \begin{block}{Key Assumptions}
    \begin{itemize}
      \item \textbf{Linearity}: Assumes a linear relationship between independent variables and log-odds of the dependent variable.
      \item \textbf{Independence}: Observations must be independent; dependencies can lead to incorrect parameter estimates.
    \end{itemize}
  \end{block}
  
  \begin{example}
    In predicting pass/fail based on study hours and attendance without considering prior performance, the linear assumption may be violated.
  \end{example}
\end{frame}

\begin{frame}[fragile]
  \frametitle{Multicollinearity and Outliers}
  \begin{block}{Multicollinearity}
    \begin{itemize}
      \item High correlation among independent variables can inflate variance and instability of coefficient estimates.
      \item \textbf{Tip}: Use Variance Inflation Factor (VIF); a VIF above 10 indicates significant multicollinearity.
    \end{itemize}
  \end{block}
  
  \begin{block}{Outliers}
    \begin{itemize}
      \item Outliers can skew results and lead to biased estimates.
      \item \textbf{Example}: Extreme values in age/test results in medical data can disproportionately influence the model.
    \end{itemize}
  \end{block}
\end{frame}

\begin{frame}[fragile]
  \frametitle{Sample Size, Imbalance, and Overfitting}
  \begin{block}{Sample Size and Imbalance}
    \begin{itemize}
      \item Logistic regression struggles with small sample sizes or imbalanced classes.
      \item \textbf{Solution}: Techniques like oversampling the minority class or undersampling the majority class.
    \end{itemize}
  \end{block}

  \begin{block}{Overfitting}
    \begin{itemize}
      \item Happens when the model learns noise instead of underlying patterns, diminishing predictive power on unseen data.
      \item \textbf{Strategy}: Use regularization (L1/Lasso or L2/Ridge) to penalize complex models.
    \end{itemize}
  \end{block}
\end{frame}

\begin{frame}[fragile]
  \frametitle{Interpretability Challenges and Recommendations}
  \begin{block}{Interpretability and Complexity}
    \begin{itemize}
      \item Complex models (e.g., involving polynomials or interactions) are harder to interpret, affecting decision-making.
      \item \textbf{Tip}: Maintain clarity by limiting predictors and using feature importance analysis.
    \end{itemize}
  \end{block}

  \begin{block}{Conclusion and Recommendations}
    \begin{itemize}
      \item Regularly check assumptions of logistic regression.
      \item Manage multicollinearity with VIF.
      \item Handle outliers carefully.
      \item Utilize techniques for class imbalance.
      \item Apply regularization to combat overfitting.
    \end{itemize}
  \end{block}
\end{frame}

\begin{frame}[fragile]
  \frametitle{Future of Logistic Regression and Trends}
  
  \begin{block}{Introduction}
    Logistic Regression has been a foundational method in statistical modeling and machine learning. As we look to the future, the integration of logistic regression with emerging technologies and methodologies presents exciting opportunities for enhancing predictive modeling across various domains.
  \end{block}
  
\end{frame}

\begin{frame}[fragile]
  \frametitle{Key Emerging Trends}

  \begin{enumerate}
    \item \textbf{Integration with Deep Learning}
      \begin{itemize}
        \item Often used as a baseline model integrated with neural networks like MLPs.
        \item Example: Logistic regression in the final classification layer of CNNs.
      \end{itemize}
      
    \item \textbf{Automated Machine Learning (AutoML)}
      \begin{itemize}
        \item Enables automatic selection and tuning of logistic regression models.
        \item Example: Tools like H2O.ai optimize features and hyperparameters.
      \end{itemize}
      
    \item \textbf{Handling Large Scale Data}
      \begin{itemize}
        \item Advances in computing allow logistic regression to handle big data.
        \item Example: Use of Apache Spark for model fitting on massive datasets.
      \end{itemize}
  \end{enumerate}
  
\end{frame}

\begin{frame}[fragile]
  \frametitle{Key Emerging Trends (Continued)}

  \begin{enumerate}
    \setcounter{enumi}{3}
    \item \textbf{Improvements in Interpretability}
      \begin{itemize}
        \item Techniques like SHAP and LIME enhance model interpretability.
        \item Example: Identifying influential features for model transparency.
      \end{itemize}
      
    \item \textbf{Incorporation of External Data}
      \begin{itemize}
        \item Models are enhanced by integrating external data sources.
        \item Example: Social media patterns in credit scoring.
      \end{itemize}
      
    \item \textbf{Regularization Techniques}
      \begin{itemize}
        \item Lasso (L1) and Ridge (L2) help manage overfitting in high-dimensional data.
        \item Example: Improved model performance through regularization.
      \end{itemize}
  \end{enumerate}

\end{frame}

\begin{frame}[fragile]
  \frametitle{Challenges and Future Directions}

  \begin{itemize}
    \item \textbf{Bias and Fairness}: Ensuring fairness in complex models.
    \item \textbf{Scalability}: Developing methods to scale with increasing data.
    \item \textbf{Real-time Predictions}: Enhancements for applications like fraud detection.
  \end{itemize}
  
  \begin{block}{Conclusion}
    The future of logistic regression lies at the intersection of traditional statistical methods and modern machine learning innovations. By embracing these emerging trends and addressing ongoing challenges, logistic regression can continue to play a crucial role in predictive analytics across diverse domains.
  \end{block}
  
\end{frame}

\begin{frame}[fragile]
  \frametitle{Key Points Recap}

  \begin{itemize}
    \item Integration with deep learning and AutoML for efficiency.
    \item Enhanced interpretability through tools like SHAP and LIME.
    \item Importance of addressing bias and ensuring scalability as data grows.
  \end{itemize}
  
\end{frame}

\begin{frame}[fragile]
    \frametitle{Summary and Key Takeaways - Part 1}
    \begin{block}{Understanding Logistic Regression}
        \begin{itemize}
            \item \textbf{Definition:} Logistic regression is a statistical method for binary classification, predicting an outcome with two categories (e.g., success/failure).
            \item \textbf{Motivation:} Essential for predicting event probabilities based on prior data, aiding decision-making across domains such as healthcare, finance, and marketing.
        \end{itemize}
    \end{block}
\end{frame}

\begin{frame}[fragile]
    \frametitle{Summary and Key Takeaways - Part 2}
    \begin{block}{Key Concepts Recap}
        \begin{enumerate}
            \item \textbf{Probability Interpretation:}
                \begin{itemize}
                    \item Logistic regression uses the logistic function to predict probabilities.
                    \item The logistic function:
                    \[
                    P(Y=1|X) = \frac{1}{1 + e^{-(\beta_0 + \beta_1X_1 + \beta_2X_2 + \ldots + \beta_nX_n)}}
                    \]
                    \item Example: Predicting a student's pass/fail based on study hours and attendance.
                \end{itemize}
            \item \textbf{Understanding Coefficients:}
                \begin{itemize}
                    \item Each coefficient (\(\beta\)) indicates the change in log-odds for a one-unit increase in a predictor variable.
                    \item Positive coefficients increase event likelihood; negative coefficients decrease it.
                \end{itemize}
        \end{enumerate}
    \end{block}
\end{frame}

\begin{frame}[fragile]
    \frametitle{Summary and Key Takeaways - Part 3}
    \begin{block}{Model Evaluation Metrics and Integration with Trends}
        \begin{itemize}
            \item \textbf{Model Evaluation:}
                \begin{itemize}
                    \item \textbf{Confusion Matrix:} Visualizes model performance (e.g., True Positives, False Positives).
                    \item Metrics: Accuracy, Precision, Recall, F1 Score assess model performance.
                \end{itemize}
            \item \textbf{Recent Trends:}
                \begin{itemize}
                    \item Applications in AI (e.g., ChatGPT) utilize logistic regression for predictive tasks like text classification.
                    \item Effective in analyzing big data to uncover actionable insights.
                \end{itemize}
        \end{itemize}
    \end{block}
\end{frame}

\begin{frame}[fragile]
    \frametitle{Summary and Key Takeaways - Conclusion}
    \begin{block}{Implications for Data Mining Practices}
        \begin{itemize}
            \item Crucial for feature selection and dimensionality reduction, improving model accuracy and interpretability.
            \item Provides foundational understanding for complex models: Supports understanding of SVM and neural networks.
            \item \textbf{Conclusion:} Logistic regression bridges statistical methods and machine learning, invaluable in academia and industry.
        \end{itemize}
    \end{block}
\end{frame}


\end{document}