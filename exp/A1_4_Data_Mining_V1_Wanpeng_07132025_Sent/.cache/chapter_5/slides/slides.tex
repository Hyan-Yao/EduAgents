\documentclass[aspectratio=169]{beamer}

% Theme and Color Setup
\usetheme{Madrid}
\usecolortheme{whale}
\useinnertheme{rectangles}
\useoutertheme{miniframes}

% Additional Packages
\usepackage[utf8]{inputenc}
\usepackage[T1]{fontenc}
\usepackage{graphicx}
\usepackage{booktabs}
\usepackage{listings}
\usepackage{amsmath}
\usepackage{amssymb}
\usepackage{xcolor}
\usepackage{tikz}
\usepackage{pgfplots}
\pgfplotsset{compat=1.18}
\usetikzlibrary{positioning}
\usepackage{hyperref}

% Custom Colors
\definecolor{myblue}{RGB}{31, 73, 125}
\definecolor{mygray}{RGB}{100, 100, 100}
\definecolor{mygreen}{RGB}{0, 128, 0}
\definecolor{myorange}{RGB}{230, 126, 34}
\definecolor{mycodebackground}{RGB}{245, 245, 245}

% Set Theme Colors
\setbeamercolor{structure}{fg=myblue}
\setbeamercolor{frametitle}{fg=white, bg=myblue}
\setbeamercolor{title}{fg=myblue}
\setbeamercolor{section in toc}{fg=myblue}
\setbeamercolor{item projected}{fg=white, bg=myblue}
\setbeamercolor{block title}{bg=myblue!20, fg=myblue}
\setbeamercolor{block body}{bg=myblue!10}
\setbeamercolor{alerted text}{fg=myorange}

% Set Fonts
\setbeamerfont{title}{size=\Large, series=\bfseries}
\setbeamerfont{frametitle}{size=\large, series=\bfseries}
\setbeamerfont{caption}{size=\small}
\setbeamerfont{footnote}{size=\tiny}

% Document Start
\begin{document}

\frame{\titlepage}

\begin{frame}[fragile]
    \frametitle{Introduction to Ensemble Methods}
    
    \begin{block}{Definition}
        Ensemble methods are techniques in machine learning that create multiple models and combine their predictions to improve overall performance. 
    \end{block}
    
    \begin{block}{Significance}
        Ensemble methods leverage the strengths of several models to achieve:
        \begin{itemize}
            \item Improved accuracy
            \item Reduction of overfitting
            \item Enhanced stability
            \item Versatility in applications
        \end{itemize}
    \end{block}
    
\end{frame}

\begin{frame}[fragile]
    \frametitle{Types of Ensemble Methods}
    
    \begin{enumerate}
        \item \textbf{Bagging (Bootstrap Aggregating)} 
            \begin{itemize}
                \item Creates multiple versions of a model trained on random subsets.
                \item Example: Random Forests, averaging multiple decision trees.
            \end{itemize}
        \item \textbf{Boosting} 
            \begin{itemize}
                \item Sequential training of models where each attempts to correct the errors of its predecessor.
                \item Example: AdaBoost, Gradient Boosting Machines (GBMs) like XGBoost.
            \end{itemize}
        \item \textbf{Stacking} 
            \begin{itemize}
                \item Combines predictions of various models by training a meta-learner.
                \item Example: Logistic regression as a meta-learner on predictions of decision trees and SVMs.
            \end{itemize}
    \end{enumerate}

\end{frame}

\begin{frame}[fragile]
    \frametitle{Recent Applications in AI}
    
    Ensemble methods play a crucial role in advanced AI applications:
    \begin{itemize}
        \item \textbf{ChatGPT and NLP} 
            \begin{itemize}
                \item Used in stages of natural language processing to enhance language understanding and response accuracy.
            \end{itemize}
        \item \textbf{Healthcare Predictions} 
            \begin{itemize}
                \item Employed to predict patient outcomes by analyzing various features from different models.
            \end{itemize}
    \end{itemize}
    
    \begin{block}{Key Takeaways}
        \begin{itemize}
            \item Ensemble methods improve accuracy and robustness.
            \item They help reduce overfitting and provide stable predictions.
            \item Techniques include Bagging, Boosting, and Stacking.
            \item Applications include cutting-edge AI systems and critical fields like healthcare.
        \end{itemize}
    \end{block}
    
\end{frame}

\begin{frame}[fragile]
    \frametitle{What Are Random Forests?}
    \begin{block}{Introduction to Random Forests}
        Random Forests are an ensemble learning method used for both classification and regression tasks. They construct multiple decision trees during training and output the mode (for classification) or mean prediction (for regression) of the individual trees.
    \end{block}
\end{frame}

\begin{frame}[fragile]
    \frametitle{Motivation Behind Random Forests}
    \begin{itemize}
        \item \textbf{Limitations of Single Decision Trees}:
        \begin{itemize}
            \item Sensitive to data variations.
            \item Prone to overfitting, leading to completely different tree structures with minor data changes.
        \end{itemize}
        
        \item \textbf{Ensemble Approach}:
        \begin{itemize}
            \item Aggregates predictions to smooth out noise.
            \item Similar to a survey, where multiple opinions yield a more accurate consensus.
        \end{itemize}
    \end{itemize}
\end{frame}

\begin{frame}[fragile]
    \frametitle{How Random Forests Work}
    \begin{enumerate}
        \item \textbf{Bootstrap Aggregating (Bagging)}:
            \begin{itemize}
                \item Creates multiple subsets of training data by sampling with replacement.
                \item Each tree is trained on a different subset, reducing variance.
            \end{itemize}
        
        \item \textbf{Building Decision Trees}:
            \begin{itemize}
                \item Randomly select a subset of features for splitting at each node (feature randomness).
                \item Trees are grown to maximum depth without pruning.
            \end{itemize}
        
        \item \textbf{Majority Voting / Averaging}:
            \begin{itemize}
                \item In classification, each tree votes, and the majority class is the final prediction.
                \item In regression, the average output of trees becomes the final prediction.
            \end{itemize}
    \end{enumerate}
\end{frame}

\begin{frame}[fragile]
    \frametitle{Key Advantages}
    \begin{itemize}
        \item \textbf{Robustness}: High tolerance to outliers and noise.
        \item \textbf{Feature Importance}: Insights into which features significantly contribute to the model.
        \item \textbf{Reduced Overfitting}: Averaging multiple trees mitigates the risk of overfitting.
    \end{itemize}
\end{frame}

\begin{frame}[fragile]
    \frametitle{Example Use Case}
    \begin{block}{Fraud Detection in Financial Transactions}
        A single decision tree may create complex rules affected by a few incorrect entries. In contrast, a Random Forest builds many trees using different samples and features, leading to a more generalized model that accurately identifies fraudulent patterns without being misled by noise.
    \end{block}
\end{frame}

\begin{frame}[fragile]
    \frametitle{Key Points to Remember}
    \begin{itemize}
        \item Random Forests combine multiple decision trees to enhance predictive performance.
        \item Utilize bagging and feature randomness to reduce variance and prevent overfitting.
        \item Widely used in finance, healthcare, and marketing due to robustness and accuracy.
    \end{itemize}
\end{frame}

\begin{frame}[fragile]
    \frametitle{Conclusion}
    By understanding and employing Random Forests, an essential technique in data mining and machine learning, you can leverage state-of-the-art models applicable in various fields, including predictive analytics in tools like ChatGPT.
\end{frame}

\begin{frame}[fragile]
    \frametitle{The Concept of Ensemble Learning}
    \begin{block}{What is Ensemble Learning?}
        Ensemble learning is a powerful machine learning approach that combines predictions from multiple models to enhance overall accuracy. A group of weak learners can come together to form a robust learner.
    \end{block}
    \begin{block}{Key Definition}
        \begin{itemize}
            \item \textbf{Ensemble Learning}: A technique in which multiple models (often of varying types) are trained to solve the same problem, and their predictions are combined to achieve better performance than any individual model.
        \end{itemize}
    \end{block}
\end{frame}

\begin{frame}[fragile]
    \frametitle{Why Do We Use Ensemble Learning?}
    \begin{itemize}
        \item \textbf{Improved Accuracy}: Aggregating predictions reduces errors and enhances generalization to unseen data.
        \item \textbf{Mitigation of Overfitting}: Combines several models to produce a more stable model that generalizes better.
        \item \textbf{Diversity is Key}: Different algorithms or varying hyperparameters lead to diverse predictions, which can yield better combined results.
    \end{itemize}
\end{frame}

\begin{frame}[fragile]
    \frametitle{How Does Ensemble Learning Work?}
    \begin{enumerate}
        \item \textbf{Model Generation}: Train multiple models on the same dataset:
            \begin{itemize}
                \item Homogeneous (e.g., multiple decision trees)
                \item Heterogeneous (e.g., combining a decision tree, logistic regression, neural network)
            \end{itemize}
        \item \textbf{Combining Predictions}:
            \begin{itemize}
                \item \textbf{Voting}: Majority vote for classification tasks
                \item \textbf{Averaging}: Average predictions for regression tasks
            \end{itemize}
    \end{enumerate}
\end{frame}

\begin{frame}[fragile]
    \frametitle{Common Ensemble Strategies}
    \begin{itemize}
        \item \textbf{Bagging (Bootstrap Aggregating)}: 
            \begin{itemize}
                \item Example: Random Forests
                \item Reduces variance by using different subsets of the training data.
            \end{itemize}
        
        \item \textbf{Boosting}:
            \begin{itemize}
                \item Example: AdaBoost, Gradient Boosting Machines
                \item Focuses on hard-to-predict instances by adjusting weights based on previous performance.
            \end{itemize}
        
        \item \textbf{Stacking}:
            \begin{itemize}
                \item Combines predictions from multiple models (base learners) and uses another model (meta-learner) for final predictions.
            \end{itemize}
    \end{itemize}
\end{frame}

\begin{frame}[fragile]
    \frametitle{Example Illustration}
    Consider a voting system for an election:
    \begin{itemize}
        \item Model A: Predicts Candidate X
        \item Model B: Predicts Candidate Y
        \item Model C: Predicts Candidate X
    \end{itemize}
    The ensemble method would tally the votes:
    \begin{itemize}
        \item Candidate X wins with 2 votes to 1, highlighting how ensemble learning can yield better results despite individual model inaccuracies.
    \end{itemize}
\end{frame}

\begin{frame}[fragile]
    \frametitle{Key Takeaways}
    \begin{itemize}
        \item Ensemble learning enhances predictive performance by leveraging the strengths of multiple models.
        \item It effectively addresses overfitting and provides more reliable results.
        \item Widely used across various applications, including fraud detection, recommendation systems, and image recognition, with advancements like ChatGPT benefiting from these techniques.
    \end{itemize}
\end{frame}

\begin{frame}[fragile]
    \frametitle{Summary}
    Ensemble learning is a fundamental technique in machine learning that enhances model performance by combining predictions from diverse models. Through strategies like bagging, boosting, and stacking, ensemble methods leverage collective decision-making, paving the way for significant improvements in AI applications.
\end{frame}

\begin{frame}[fragile]
    \frametitle{Advantages of Using Random Forests - Introduction}
    \begin{block}{Overview}
        Random Forests are a powerful ensemble learning technique that constructs multiple decision trees and aggregates their predictions. This method not only improves prediction accuracy but also helps in mitigating issues such as overfitting.
    \end{block}
\end{frame}

\begin{frame}[fragile]
    \frametitle{Advantages of Using Random Forests - Key Advantages}
    \begin{enumerate}
        \item \textbf{Improved Accuracy}:
        \begin{itemize}
            \item Combines predictions from numerous trees for enhanced accuracy and reduced variance.
            \item Example: In classification tasks, Random Forests often outperform single decision trees by averaging multiple predictions.
        \end{itemize}

        \item \textbf{Robustness to Overfitting}:
        \begin{itemize}
            \item Uses bagging and feature randomness to reduce overfitting.
            \item Key Point: Perform well on unseen data.
        \end{itemize}

        \item \textbf{Handling Missing Values}:
        \begin{itemize}
            \item Naturally manages missing values, maintaining accuracy by preserving available data.
        \end{itemize}
    \end{enumerate}
\end{frame}

\begin{frame}[fragile]
    \frametitle{Advantages of Using Random Forests - Continued}
    \begin{enumerate}
        \setcounter{enumi}{3} % resume enumeration
        \item \textbf{Feature Importance Estimation}:
        \begin{itemize}
            \item Identifies influential variables by measuring accuracy drop when permuting feature values.
            \item Example: In medical datasets, age and symptoms can be crucial for disease prediction.
        \end{itemize}

        \item \textbf{Versatile Applications}:
        \begin{itemize}
            \item Applicable in various domains such as finance and healthcare, solving both classification and regression problems.
            \item Example: Predicting customer churn based on demographic and behavioral analysis.
        \end{itemize}
    \end{enumerate}
    
    \begin{block}{Summary}
        - Improved accuracy by combining tree predictions\\
        - Mitigated overfitting through bagging and feature randomness\\
        - Effective handling of missing values\\
        - Insights into feature importance for enhanced interpretability\\
        - Broad applicability across different fields
    \end{block}
\end{frame}

\begin{frame}[fragile]
    \frametitle{How Random Forests Work - Introduction}
    \begin{block}{Introduction to Random Forests}
        Random Forest is an ensemble learning algorithm that combines multiple decision trees to improve predictive performance and control overfitting. It is widely used in classification and regression tasks.
    \end{block}
\end{frame}

\begin{frame}[fragile]
    \frametitle{How Random Forests Work - Key Concepts}
    \begin{itemize}
        \item \textbf{Ensemble Learning}: The process of combining multiple models to produce a better performing model. Random Forests use a technique known as \textit{bagging} (Bootstrap Aggregating).
        
        \item \textbf{Bagging}:
            \begin{itemize}
                \item \textit{Definition}: Involves training multiple models on different subsets of the training data.
                \item \textit{Process}:
                    \begin{itemize}
                        \item Bootstrapping: Randomly sample data points with replacement to create different training datasets.
                        \item Parallel model training: Each decision tree is built on a different subset.
                    \end{itemize}
                \item \textit{Result}: Predictions from all trees are aggregated (majority voting for classification, averaging for regression) to improve accuracy.
            \end{itemize}
        
        \item \textbf{Feature Randomness}:
            \begin{itemize}
                \item During the construction of each tree, a random subset of features is considered for splitting nodes.
                \item This helps:
                    \begin{itemize}
                        \item Reduce correlation among the trees.
                        \item Different trees learn different patterns.
                    \end{itemize}
                \item Leads to a diverse set of trees, enhancing the effectiveness of the Random Forest algorithm.
            \end{itemize}
    \end{itemize}
\end{frame}

\begin{frame}[fragile]
    \frametitle{How Random Forests Work - Functioning and Example}
    \begin{block}{Step-by-Step Process}
        \begin{enumerate}
            \item \textbf{Create Bootstrap Samples}: Generate 'n' bootstrap samples from the original dataset.
            \item \textbf{Train Trees}: For each bootstrap sample, grow a decision tree, randomly selecting a subset of features at each split.
            \item \textbf{Aggregate Predictions}: 
                \begin{itemize}
                    \item For classification: Class receiving most votes from all trees is chosen.
                    \item For regression: Take the average of predictions.
                \end{itemize}
        \end{enumerate}
    \end{block}

    \begin{block}{Example}
        Imagine classifying whether an email is spam:
        \begin{itemize}
            \item Create datasets by randomly sampling from original emails.
            \item Build individual trees that analyze different attributes (like subject line, sender).
            \item Combine results for final prediction to determine if the email is spam.
        \end{itemize}
    \end{block}
\end{frame}

\begin{frame}[fragile]
    \frametitle{How Random Forests Work - Key Points and Summary}
    \begin{itemize}
        \item \textbf{Robustness}: Resistant to overfitting due to the averaging effect of multiple trees.
        \item \textbf{Versatility}: Applicable to both classification and regression tasks.
        \item \textbf{Performance}: Generally outperform single decision trees on various datasets.
    \end{itemize}

    \begin{block}{Summary}
        Random Forests enhance predictive accuracy through ensemble learning, utilizing bagging and feature randomness for a robust model capable of generalizing well to unseen data.
    \end{block}

    \begin{block}{Next Steps}
        Follow along to see how to implement Random Forests in Python using scikit-learn!
    \end{block}
\end{frame}

\begin{frame}
    \frametitle{Implementation of Random Forests in Python - Overview}
    \begin{itemize}
        \item Introduction to implementing Random Forests with Python.
        \item Focus on the `scikit-learn` library for a seamless experience.
        \item Contains data preparation, model training, and evaluation steps.
    \end{itemize}
\end{frame}

\begin{frame}
    \frametitle{Implementation of Random Forests in Python - Why Use Random Forests?}
    \begin{block}{Advantages}
        \begin{itemize}
            \item Robust to high-dimensional datasets.
            \item Excellent predictive performance with minimal tuning.
            \item Uses ensemble learning to increase accuracy and reduce overfitting.
        \end{itemize}
    \end{block}
\end{frame}

\begin{frame}[fragile]
    \frametitle{Implementation of Random Forests in Python - Step-by-Step Guide (Part 1)}
    \begin{enumerate}
        \item \textbf{Import Necessary Libraries}
        \begin{lstlisting}[language=Python]
import pandas as pd
from sklearn.model_selection import train_test_split
from sklearn.ensemble import RandomForestClassifier
from sklearn.metrics import classification_report, accuracy_score
        \end{lstlisting}
        
        \item \textbf{Load and Prepare Data}
        \begin{lstlisting}[language=Python]
from sklearn.datasets import load_iris
iris = load_iris()
X = iris.data  # Features
y = iris.target  # Target variable

X_train, X_test, y_train, y_test = train_test_split(X, y, test_size=0.2, random_state=42)
        \end{lstlisting}
    \end{enumerate}
\end{frame}

\begin{frame}[fragile]
    \frametitle{Implementation of Random Forests in Python - Step-by-Step Guide (Part 2)}
    \begin{enumerate}
        \setcounter{enumi}{2} % Continue numbering from previous frame
        \item \textbf{Initialize and Train the Random Forest Model}
        \begin{lstlisting}[language=Python]
rf_model = RandomForestClassifier(n_estimators=100, random_state=42)
rf_model.fit(X_train, y_train)
        \end{lstlisting}

        \item \textbf{Make Predictions}
        \begin{lstlisting}[language=Python]
y_pred = rf_model.predict(X_test)
        \end{lstlisting}
    \end{enumerate}
\end{frame}

\begin{frame}[fragile]
    \frametitle{Implementation of Random Forests in Python - Step-by-Step Guide (Part 3)}
    \begin{enumerate}
        \setcounter{enumi}{4} % Continue numbering from previous frame
        \item \textbf{Evaluate the Model}
        \begin{lstlisting}[language=Python]
print("Accuracy:", accuracy_score(y_test, y_pred))
print(classification_report(y_test, y_pred))
        \end{lstlisting}
    \end{enumerate}
\end{frame}

\begin{frame}
    \frametitle{Implementation of Random Forests in Python - Key Points}
    \begin{itemize}
        \item Random Forest models improve accuracy by utilizing multiple decision trees.
        \item The `scikit-learn` library simplifies implementation.
        \item Attention to hyperparameters, such as `n_estimators`, affects performance.
        \item Evaluating model effectiveness using metrics like accuracy and recall is vital.
    \end{itemize}
\end{frame}

\begin{frame}
    \frametitle{Implementation of Random Forests in Python - Conclusion}
    \begin{itemize}
        \item Following these steps enables the effective application of Random Forests in Python.
        \item Next slide will focus on evaluation metrics for assessing model performance.
        \item Useful in various data mining tasks improving decision-making in predictive analytics.
    \end{itemize}
\end{frame}

\begin{frame}
    \frametitle{Evaluation Metrics for Random Forests - Introduction}
    When building and assessing predictive models, especially with ensemble methods like Random Forests, it's crucial to employ appropriate evaluation metrics. These metrics provide insight into the model’s performance, enabling data scientists to determine how well the model generalizes to unseen data.
\end{frame}

\begin{frame}
    \frametitle{Evaluation Metrics for Random Forests - Key Metrics}
    \begin{enumerate}
        \item \textbf{Accuracy}
        \begin{itemize}
            \item \textbf{Definition:} The ratio of correctly predicted instances to the total instances in the dataset.
            \item \textbf{Formula:}  
            \begin{equation}
                \text{Accuracy} = \frac{\text{True Positives} + \text{True Negatives}}{\text{Total Instances}}
            \end{equation}
            \item \textbf{Example:} If a model correctly predicts 80 out of 100 instances, its accuracy is 0.80 or 80\%.
        \end{itemize}
        
        \item \textbf{Precision}
        \begin{itemize}
            \item \textbf{Definition:} The ratio of correctly predicted positive observations to the total predicted positives.
            \item \textbf{Formula:}
            \begin{equation}
                \text{Precision} = \frac{\text{True Positives}}{\text{True Positives} + \text{False Positives}}
            \end{equation}
            \item \textbf{Example:} If a model predicts 50 instances as positive, and 40 of those are actually positive, the precision is 0.80 (80\%).
        \end{itemize}
    \end{enumerate}
\end{frame}

\begin{frame}
    \frametitle{Evaluation Metrics for Random Forests - Recall and F1-Score}
    \begin{enumerate}
        \setcounter{enumi}{2}
        \item \textbf{Recall (Sensitivity)}
        \begin{itemize}
            \item \textbf{Definition:} The ratio of correctly predicted positive observations to all actual positives.
            \item \textbf{Formula:}
            \begin{equation}
                \text{Recall} = \frac{\text{True Positives}}{\text{True Positives} + \text{False Negatives}}
            \end{equation}
            \item \textbf{Example:} If there are 60 actual positive cases, and the model identifies 40 as positive, the recall is approximately 0.67 (67\%).
        \end{itemize}
        
        \item \textbf{F1-Score}
        \begin{itemize}
            \item \textbf{Definition:} The harmonic mean of precision and recall.
            \item \textbf{Formula:}
            \begin{equation}
                F1 = 2 \times \frac{\text{Precision} \times \text{Recall}}{\text{Precision} + \text{Recall}}
            \end{equation}
            \item \textbf{Example:} For a precision of 0.80 and a recall of 0.67, the F1-Score is 0.73 (73\%).
        \end{itemize}
    \end{enumerate}
\end{frame}

\begin{frame}
    \frametitle{Evaluation Metrics for Random Forests - Why Use Multiple Metrics?}
    \begin{itemize}
        \item \textbf{Addressing Imbalance:} In cases of class imbalance, accuracy can be misleading; thus, metrics like precision, recall, and F1-Score provide a clearer picture of performance.
        \item \textbf{Model Evaluation:} Different metrics highlight various strengths and weaknesses of a model, enabling targeted improvements.
    \end{itemize}
\end{frame}

\begin{frame}[fragile]
    \frametitle{Evaluation Metrics for Random Forests - Code Snippet Example}
    \begin{lstlisting}[language=Python]
from sklearn.metrics import accuracy_score, precision_score, recall_score, f1_score

# y_true: actual labels, y_pred: predicted labels
accuracy = accuracy_score(y_true, y_pred)
precision = precision_score(y_true, y_pred)
recall = recall_score(y_true, y_pred)
f1 = f1_score(y_true, y_pred)

print(f"Accuracy: {accuracy:.2f}")
print(f"Precision: {precision:.2f}")
print(f"Recall: {recall:.2f}")
print(f"F1-Score: {f1:.2f}")
    \end{lstlisting}
\end{frame}

\begin{frame}
    \frametitle{Evaluation Metrics for Random Forests - Conclusion}
    Understanding and correctly applying these evaluation metrics is essential for assessing the performance of Random Forest models, leading to better decision-making and model refinement in real-world applications. 
    As we move towards the next topic, we will compare Random Forests with other modeling techniques to explore their relative advantages and limitations.
\end{frame}

\begin{frame}[fragile]
  \frametitle{Random Forests vs Other Models - Introduction}
  \begin{block}{Random Forests Overview}
    Random Forests is an ensemble learning method used for classification and regression by constructing multiple decision trees and outputting the mode class (for classification) or mean prediction (for regression). This method effectively mitigates overfitting, a common issue with single decision trees.
  \end{block}
  
  \begin{block}{Key Motivations for Model Selection}
    - Need for robust models to handle diverse data characteristics.\\
    - Balancing performance and interpretability in predictive modeling.\\
    - When to prefer ensemble methods like Random Forests over single models.
  \end{block}
\end{frame}

\begin{frame}[fragile]
  \frametitle{Random Forests vs Other Models - Comparison Breakdown}
  
  \begin{block}{1. Random Forests vs Individual Decision Trees}
    \begin{itemize}
      \item \textbf{Performance}:
      \begin{itemize}
        \item Random Forests reduce variance, leading to better accuracy on unseen data.
        \item Single trees prone to overfitting, particularly with complex datasets.
      \end{itemize}
      \item \textbf{Example}: A single tree might classify a dataset perfectly, but fail on new data due to sensitivity to noise, while Random Forests would average errors.
    \end{itemize}
  \end{block}
  
  \begin{block}{2. Random Forests vs Support Vector Machines (SVM)}
    \begin{itemize}
      \item \textbf{Performance}:
      \begin{itemize}
        \item SVMs work well for high-dimensional data but are sensitive to outliers and require hyperparameter tuning.
        \item Random Forests perform robustly with minimal tuning and are less affected by noise.
      \end{itemize}
      \item \textbf{Example}: In image recognition, SVMs may need numerous adjustments, whereas Random Forests yield decent results with defaults.
    \end{itemize}
  \end{block}
\end{frame}

\begin{frame}[fragile]
  \frametitle{Random Forests vs Other Models - Key Points}

  \begin{block}{3. Trade-offs and Strengths}
    \begin{itemize}
      \item \textbf{Interpretability}:
      \begin{itemize}
        \item Individual trees are easy to interpret; Random Forests are considered "black boxes."
        \item SVMs can be challenging to interpret, especially with non-linear kernels.
      \end{itemize}
      \item \textbf{Training Time}:
      \begin{itemize}
        \item Training multiple trees in Random Forests takes longer than a single decision tree but is often faster than SVMs for large datasets.
      \end{itemize}
    \end{itemize}
  \end{block}

  \begin{block}{4. Key Points to Emphasize}
    \begin{itemize}
      \item Random Forests excel in generalization due to their ensemble approach.
      \item Robustness compared to individual trees and SVMs across varying data characteristics.
      \item Often a first choice in predictive modeling due to excellent performance and minimal tuning needs.
    \end{itemize}
  \end{block}
  
  \begin{block}{Summarizing Remarks}
    Random Forests provide a compelling alternative by combining strengths of individual trees and SVMs, achieving balanced performance across various applications.
  \end{block}
\end{frame}

\begin{frame}
    \frametitle{Tuning Random Forest Models}
    \begin{block}{Introduction to Hyperparameter Tuning}
        Tuning hyperparameters is essential for optimizing Random Forest models. Unlike parameters learned from the data, hyperparameters are set before training and can significantly impact the model's performance.
    \end{block}
    
    \begin{block}{Why Tune Hyperparameters?}
        \begin{itemize}
            \item \textbf{Performance Optimization}: Enhances model accuracy, reduces overfitting, and improves generalization to unseen data.
            \item \textbf{Complexity Management}: Strikes a balance between model complexity and interpretability.
        \end{itemize}
    \end{block}
\end{frame}

\begin{frame}
    \frametitle{Key Hyperparameters to Tune}
    \begin{enumerate}
        \item \textbf{Number of Trees ($n\_estimators$)}:
            \begin{itemize}
                \item Total number of trees in the forest.
                \item More trees can improve accuracy but also increase computational cost.
                \item \textit{Example:} Start with 100, then test increments of 50.
            \end{itemize}
        \item \textbf{Maximum Depth ($max\_depth$)}:
            \begin{itemize}
                \item Controls how deep each tree can grow.
                \item \textit{Example:} Test values like 10, 20, and None for unlimited depth.
            \end{itemize}
        \item \textbf{Minimum Samples Split ($min\_samples\_split$)}:
            \begin{itemize}
                \item Minimum number of data points required to split a node.
                \item \textit{Example:} Values can be 2, 5, and 10.
            \end{itemize}
    \end{enumerate}
\end{frame}

\begin{frame}
    \frametitle{Key Hyperparameters to Tune (Continued)}
    \begin{enumerate}[resume]
        \item \textbf{Minimum Samples Leaf ($min\_samples\_leaf$)}:
            \begin{itemize}
                \item Minimum number of samples required to be at a leaf node.
                \item \textit{Example:} Options might include 1, 2, and 4.
            \end{itemize}
        \item \textbf{Max Features ($max\_features$)}:
            \begin{itemize}
                \item Number of features to consider for the best split.
                \item \textit{Example:} Test strategies such as 'sqrt', 'log2', and an absolute number (e.g., 10).
            \end{itemize}
    \end{enumerate}
\end{frame}

\begin{frame}[fragile]
    \frametitle{Methods for Hyperparameter Tuning}
    \begin{itemize}
        \item \textbf{Grid Search}:
        \begin{itemize}
            \item Explores all possible combinations of hyperparameters.
            \item \textit{Code Snippet}:
            \begin{lstlisting}[language=Python]
from sklearn.model_selection import GridSearchCV
param_grid = {
    'n_estimators': [100, 200],
    'max_depth': [None, 10, 20],
    'min_samples_split': [2, 5],
    'min_samples_leaf': [1, 2],
    'max_features': ['auto', 'sqrt']
}
grid_search = GridSearchCV(RandomForestClassifier(), param_grid, cv=3)
grid_search.fit(X_train, y_train)
            \end{lstlisting}
        \end{itemize}
        
        \item \textbf{Random Search}:
            \begin{itemize}
                \item Samples hyperparameters randomly, reducing computational time.
            \end{itemize}

        \item \textbf{Bayesian Optimization}:
            \begin{itemize}
                \item A method based on probability to select the next evaluation point.
            \end{itemize}
    \end{itemize}
\end{frame}

\begin{frame}
    \frametitle{Key Points to Emphasize}
    \begin{itemize}
        \item Hyperparameter tuning is crucial for maximizing model performance.
        \item Different hyperparameters can lead to vastly different model behaviors.
        \item Using cross-validation during tuning helps evaluate generalization capabilities.
    \end{itemize}
\end{frame}

\begin{frame}
    \frametitle{Conclusion}
    Tuning hyperparameters effectively allows for the creation of robust Random Forest models tailored to specific datasets, enhancing their predictive power and generalization.
    
    \textit{Note: Consider using visual aids such as performance graphs to illustrate varying hyperparameter values.}
\end{frame}

\begin{frame}
    \frametitle{Case Study: Application of Random Forests}
    \begin{block}{Introduction to Data Mining}
        \begin{itemize}
            \item \textbf{Motivation:} In today’s data-driven world, data mining is essential for extracting valuable insights from large datasets.
            \item \textbf{Relevance:} Ubiquitous applications including recommendation systems, fraud detection, and medical diagnostics necessitate robust algorithms.
        \end{itemize}
    \end{block}
\end{frame}

\begin{frame}
    \frametitle{Random Forest Overview}
    \begin{block}{Definition}
        Random Forest is an ensemble learning method that constructs multiple decision trees at training and outputs the mode of their classes (classification) or mean prediction (regression).
    \end{block}
    \begin{block}{Advantages}
        \begin{itemize}
            \item Reduces overfitting compared to individual decision trees.
            \item Handles both categorical and numerical data effectively.
            \item Robust against noise and missing values.
        \end{itemize}
    \end{block}
\end{frame}

\begin{frame}
    \frametitle{Case Study: Predicting Customer Churn}
    \begin{block}{Context}
        A telecommunications company aims to predict which customers are likely to churn based on various metrics.
    \end{block}
    \begin{block}{Data Collection}
        \begin{itemize}
            \item Historical customer data: contract length, service usage, payment methods.
            \item Key features: age, monthly charges, number of complaints, account tenure.
        \end{itemize}
    \end{block}
\end{frame}

\begin{frame}
    \frametitle{Methodology}
    \begin{enumerate}
        \item \textbf{Data Preprocessing}
            \begin{itemize}
                \item Handle missing values using imputation techniques.
                \item One-hot encoding for categorical variables.
            \end{itemize}
        \item \textbf{Model Training}
            \begin{itemize}
                \item Data split: 70\% training, 30\% testing.
                \item Train a Random Forest model with default hyperparameters.
            \end{itemize}
        \item \textbf{Hyperparameter Tuning}
            \begin{itemize}
                \item Optimize parameters such as $n\_estimators$, $max\_depth$, $min\_samples\_split$ based on grid search.
            \end{itemize}
    \end{enumerate}
\end{frame}

\begin{frame}
    \frametitle{Results and Performance Metrics}
    \begin{block}{Performance Evaluation}
        \begin{itemize}
            \item Metrics: accuracy, precision, recall, and F1 score.
            \item Example Result: Tuned model achieves 87\% accuracy on the test dataset, up from 65\%.
        \end{itemize}
    \end{block}
\end{frame}

\begin{frame}
    \frametitle{Key Takeaways and Conclusion}
    \begin{block}{Key Takeaways}
        \begin{itemize}
            \item Random Forests classify customer behavior and enhance retention strategies effectively.
            \item Feature importance analysis identifies key predictors like number of complaints and contract length.
        \end{itemize}
    \end{block}
    \begin{block}{Conclusion}
        \begin{itemize}
            \item Random Forests are a powerful tool for classification tasks.
            \item Mastery of frameworks like Random Forests is essential for developing predictive models.
        \end{itemize}
    \end{block}
\end{frame}

\begin{frame}[fragile]
    \frametitle{Example Code Snippet (Python)}
    \begin{lstlisting}[language=Python]
from sklearn.model_selection import train_test_split
from sklearn.ensemble import RandomForestClassifier
from sklearn.metrics import classification_report

# Load dataset
data = pd.read_csv('telecom_churn.csv')

# Data preprocessing
data = pd.get_dummies(data)  # One-hot encoding for categorical variables
X = data.drop('churn', axis=1)  # Features
y = data['churn']  # Target variable

# Train-test split
X_train, X_test, y_train, y_test = train_test_split(X, y, test_size=0.3, random_state=42)

# Train Random Forest model
model = RandomForestClassifier(n_estimators=100)
model.fit(X_train, y_train)

# Predictions and evaluation
y_pred = model.predict(X_test)
print(classification_report(y_test, y_pred))
    \end{lstlisting}
\end{frame}

\begin{frame}[fragile]
    \frametitle{Challenges and Limitations of Random Forests}
    % Introduction
    \begin{block}{Introduction}
        Random Forests are a powerful ensemble learning method that combines multiple decision trees. However, certain challenges and limitations arise in practical scenarios.
    \end{block}
\end{frame}

\begin{frame}[fragile]
    \frametitle{Challenges and Limitations - Overfitting and Interpretability}
    % Overfitting
    \begin{block}{1. Overfitting in Some Scenarios}
        \begin{itemize}
            \item **Explanation**: Random Forests can still overfit with too many trees or excessively deep trees.
            \item **Example**: Complex models may capture noise in datasets with many features and few observations.
        \end{itemize}
    \end{block}

    % Interpretability
    \begin{block}{2. Interpretability Issues}
        \begin{itemize}
            \item **Explanation**: Random Forests are often seen as "black boxes," which complicates understanding individual predictor contributions.
            \item **Example**: In healthcare, it may be difficult to extract insights about the contributions of factors like age and blood pressure.
        \end{itemize}
    \end{block}
\end{frame}

\begin{frame}[fragile]
    \frametitle{Challenges and Limitations - Computational Complexity and Parameter Tuning}
    % Computational Complexity
    \begin{block}{3. Computational Complexity}
        \begin{itemize}
            \item **Explanation**: Training numerous decision trees can be computationally expensive.
            \item **Example**: Training on a dataset with millions of observations requires significant memory and processing power.
        \end{itemize}
    \end{block}

    % Need for Parameter Tuning
    \begin{block}{4. Need for Parameter Tuning}
        \begin{itemize}
            \item **Explanation**: Optimal performance often requires fine-tuning hyperparameters (e.g., $n\_estimators$, max depth).
            \item **Key Point**: Default parameters may not yield the best results; methods like cross-validation are necessary.
        \end{itemize}
    \end{block}
\end{frame}

\begin{frame}[fragile]
    \frametitle{Challenges and Limitations - Bias in Imbalanced Datasets and Conclusion}
    % Bias in Imbalanced Datasets
    \begin{block}{5. Bias in Imbalanced Datasets}
        \begin{itemize}
            \item **Explanation**: Random Forests may perform poorly with imbalanced classes due to bias towards the majority class.
            \item **Example**: In fraud detection, models may predict only the majority class unless class weighting or resampling techniques are used.
        \end{itemize}
    \end{block}

    % Conclusion
    \begin{block}{Conclusion}
        While Random Forests have several advantages, awareness of their challenges is crucial for effective application and optimal model performance.
    \end{block}
\end{frame}

\begin{frame}[fragile]
    \frametitle{Other Ensemble Methods - Overview}
    \begin{block}{Ensemble Learning}
        Ensemble learning combines multiple models to improve predictive performance compared to individual models. Beyond Random Forests, two popular methods are \textbf{Boosting} and \textbf{Stacking}.
    \end{block}

    \begin{itemize}
        \item Boosting: Focuses on correcting errors of weak learners.
        \item Stacking: Blends predictions from multiple models using a meta-model.
    \end{itemize}
\end{frame}

\begin{frame}[fragile]
    \frametitle{Boosting}
    \begin{block}{Definition}
        Boosting is an ensemble technique that sequentially applies weak learners to adjust the weights of previously misclassified observations, aiming to create a strong classifier.
    \end{block}

    \begin{itemize}
        \item \textbf{How Boosting Works:}
            \begin{enumerate}
                \item Initialization: All instances hold equal weight.
                \item Training Weak Learners:
                    \begin{itemize}
                        \item First model is trained on original data.
                        \item Subsequent models focus on correcting errors.
                    \end{itemize}
                \item Weighted Voting: Each model votes based on accuracy; more accurate models receive higher weight.
            \end{enumerate}
        \item \textbf{Popular Algorithms:}
            \begin{itemize}
                \item AdaBoost
                \item Gradient Boosting
                \item XGBoost
            \end{itemize}
    \end{itemize}

    \begin{block}{Key Points}
        \begin{itemize}
            \item Boosting corrects errors of weak learners.
            \item Sensitive to noisy data and outliers.
            \item Best suited for scenarios with less interpretability.
        \end{itemize}
    \end{block}
\end{frame}

\begin{frame}[fragile]
    \frametitle{Stacking}
    \begin{block}{Definition}
        Stacking employs multiple models and blends their predictions by training a meta-model on the outputs of base models.
    \end{block}

    \begin{itemize}
        \item \textbf{How Stacking Works:}
            \begin{enumerate}
                \item Base Learners: Train multiple diverse models (e.g., decision trees, SVMs).
                \item Level-0 Data: Use predictions on a validation set to form a new dataset.
                \item Meta-learner: Train a higher-level model to combine predictions effectively.
            \end{enumerate}
    \end{itemize}

    \begin{block}{Key Points}
        \begin{itemize}
            \item Combines different model types for improved performance.
            \item Requires careful tuning to avoid overfitting.
            \item Leverages diverse perspectives for better outcomes.
        \end{itemize}
    \end{block}
\end{frame}

\begin{frame}[fragile]
    \frametitle{Future Trends in Ensemble Learning - Overview}
    \begin{block}{Overview}
        Ensemble learning methods have dramatically advanced machine learning by combining multiple models to improve accuracy, robustness, and generalization. 
        This slide discusses emerging trends and future directions in ensemble learning that are shaping the field.
    \end{block}
\end{frame}

\begin{frame}[fragile]
    \frametitle{Future Trends in Ensemble Learning - 1. Integration with Deep Learning}
    \begin{itemize}
        \item \textbf{Description:} Combining traditional ensemble techniques with deep learning architectures for enhanced performance.
        \item \textbf{Examples:}
        \begin{itemize}
            \item Hybrid Models: Integrating ensemble methods like boosting with neural networks (e.g., Neural Boosting) for improved feature extraction.
            \item Ensemble of Deep Networks: Using model ensembles composed of various deep learning architectures to tackle specific tasks, such as image classification or natural language processing.
        \end{itemize}
    \end{itemize}
    \begin{block}{Key Point}
        The fusion of ensemble learning and deep learning exploits the strengths of both domains, leading to superior performance in complex tasks.
    \end{block}
\end{frame}

\begin{frame}[fragile]
    \frametitle{Future Trends in Ensemble Learning - 2. AutoML}
    \begin{itemize}
        \item \textbf{Description:} Emergence of AutoML frameworks simplifies the model selection and hyperparameter tuning process in ensemble methods.
        \item \textbf{Examples:}
        \begin{itemize}
            \item Automated selection of ensemble strategies (e.g., selecting between bagging, boosting, and stacking) based on data characteristics and problem type.
        \end{itemize}
        \item \textbf{Tools:} Libraries like TPOT or Auto-Sklearn can automate the ensemble-building process.
    \end{itemize}
    \begin{block}{Key Point}
        AutoML makes ensemble methods more accessible to practitioners with limited machine learning expertise, democratizing advanced modeling techniques.
    \end{block}
\end{frame}

\begin{frame}[fragile]
    \frametitle{Future Trends in Ensemble Learning - 3. Explainability and Interpretability}
    \begin{itemize}
        \item \textbf{Description:} With the rise of model interpretability, ensemble models are being enhanced to provide insights into their decision-making processes.
        \item \textbf{Examples:} 
        \begin{itemize}
            \item Development of techniques to visualize feature importances in ensemble methods, such as SHAP (SHapley Additive exPlanations) and LIME (Local Interpretable Model-agnostic Explanations).
        \end{itemize}
    \end{itemize}
    \begin{block}{Key Point}
        As ensemble methods are implemented in safety-sensitive applications (like healthcare), understanding their predictions becomes crucial.
    \end{block}
\end{frame}

\begin{frame}[fragile]
    \frametitle{Future Trends in Ensemble Learning - 4. Handling Uncertainty}
    \begin{itemize}
        \item \textbf{Description:} Future ensemble methods will increasingly focus on quantifying uncertainty in predictions, which is vital in fields like finance and medicine.
        \item \textbf{Examples:}
        \begin{itemize}
            \item Incorporating Bayesian approaches into ensemble learning to provide probabilistic outputs alongside predictions.
        \end{itemize}
    \end{itemize}
    \begin{block}{Key Point}
        Managing uncertainty allows for more informed decision-making based on ensemble predictions.
    \end{block}
\end{frame}

\begin{frame}[fragile]
    \frametitle{Future Trends in Ensemble Learning - 5. Real-World Applications}
    \begin{itemize}
        \item \textbf{Examples of Applications:}
        \begin{itemize}
            \item \textbf{Healthcare:} Predicting patient outcomes using ensembles that combine clinical data, genomic information, and lifestyle factors.
            \item \textbf{Finance:} Risk assessment and fraud detection by utilizing ensemble models that analyze transaction patterns in real-time.
        \end{itemize}
    \end{itemize}
    \begin{block}{Key Point}
        The versatility and adaptability of ensemble methods make them valuable tools across diverse industries, driving innovation and improving efficiencies.
    \end{block}
\end{frame}

\begin{frame}[fragile]
    \frametitle{Future Trends in Ensemble Learning - Conclusion}
    \begin{block}{Conclusion}
        Ensemble learning is poised for continued evolution as it integrates with cutting-edge technologies and addresses critical challenges in transparency, automation, and prediction reliability. 
        Future advancements will not only enhance model performance but also ensure more ethical and interpretable applications of machine learning.
    \end{block}
\end{frame}

\begin{frame}[fragile]
    \frametitle{Future Trends in Ensemble Learning - Outline}
    \begin{enumerate}
        \item Integration with Deep Learning
        \item Automated Machine Learning (AutoML)
        \item Explainability and Interpretability
        \item Handling Uncertainty
        \item Real-World Applications
    \end{enumerate}
\end{frame}

\begin{frame}[fragile]
    \frametitle{Ethical Considerations - Introduction}
    \begin{block}{Introduction to Ethical Considerations}
        Ethical considerations are critical in the application of ensemble methods such as Random Forests.
        This discussion revolves around two major themes: 
        \begin{itemize}
            \item Data Privacy
            \item Model Transparency
        \end{itemize}
        Understanding these themes ensures responsible usage of machine learning models.
    \end{block}
\end{frame}

\begin{frame}[fragile]
    \frametitle{Ethical Considerations - Data Privacy}
    \begin{block}{Data Privacy}
        \textbf{Definition:} Data privacy relates to the handling of sensitive information to safeguard against unauthorized access and misuse.

        \textbf{Key Concerns:}
        \begin{enumerate}
            \item Data Collection: Aggregation of large datasets may include personal information.
            \item Data Anonymization: Ensuring identities cannot be traced back from the training data.
            \item Regulatory Compliance: Adherence to laws like GDPR and HIPAA governing data usage.
        \end{enumerate}

        \textbf{Example:} A healthcare dataset predicting patient outcomes requires anonymization to protect identities from exposure.
    \end{block}
\end{frame}

\begin{frame}[fragile]
    \frametitle{Ethical Considerations - Model Transparency}
    \begin{block}{Model Transparency}
        \textbf{Definition:} Model transparency denotes how easily stakeholders comprehend the reasoning behind model predictions.

        \textbf{Key Issues:}
        \begin{enumerate}
            \item Complexity: Ensemble methods often function as black boxes, making understanding predictions difficult.
            \item Accountability: A lack of transparency complicates addressing biases or errors in predictions.
        \end{enumerate}

        \textbf{Example:} In hiring decisions, organizations need to explain how a Random Forest model reached a conclusion to ensure fairness and accountability.
    \end{block}
\end{frame}

\begin{frame}[fragile]
    \frametitle{Summary of Random Forests - Overview}
    \begin{block}{Overview}
        Random Forests represent a powerful ensemble learning technique commonly used for classification and regression tasks. 
        By combining multiple decision trees, Random Forests enhance model accuracy, robustness, and reduce the risk of overfitting.
    \end{block}
\end{frame}

\begin{frame}[fragile]
    \frametitle{Summary of Random Forests - Key Points}
    \begin{enumerate}
        \item \textbf{Ensemble Learning Concept}:
        \begin{itemize}
            \item Ensemble methods leverage multiple models to improve predictive performance compared to individual models.
            \item Motivations include increased accuracy and robustness in predictions, particularly in complex datasets.
        \end{itemize}

        \item \textbf{How Random Forests Work}:
        \begin{itemize}
            \item \textbf{Bootstrap Aggregating (Bagging)}: Randomly samples subsets of the training data with replacement.
            \item \textbf{Feature Randomness}: Randomly selects a subset of features when splitting nodes, enhancing tree diversity.
            \item \textbf{Vote Aggregation}: Each tree contributes to the final prediction via majority voting or averaging.
        \end{itemize}

        \item \textbf{Advantages}:
        \begin{itemize}
            \item Improved accuracy and reduced variance.
            \item Capable of handling missing values through Surrogate Splits.
            \item Less prone to overfitting due to the averaging effect.
        \end{itemize}

        \item \textbf{Limitations}:
        \begin{itemize}
            \item Complexity and lack of transparency.
            \item Resource-intensive training process.
        \end{itemize}
    \end{enumerate}
\end{frame}

\begin{frame}[fragile]
    \frametitle{Summary of Random Forests - Applications and Conclusion}
    \begin{block}{Applications}
        Random Forests are widely used in various domains such as:
        \begin{itemize}
            \item Finance (credit scoring)
            \item Healthcare (disease prediction)
            \item Customer segmentation
        \end{itemize}
    \end{block}

    \begin{block}{Conclusion}
        Understanding Random Forests equips you with advanced skills in machine learning, enabling you to tackle complex prediction tasks while being aware of ethical considerations related to model transparency and data privacy.
    \end{block}

    \begin{block}{Next Steps}
        Prepare for a Q\&A session to discuss any complexities or applications of Random Forests that need clarification!
    \end{block}
\end{frame}

\begin{frame}[fragile]
  \frametitle{Q\&A Session - Introduction}
  \begin{itemize}
    \item This session serves as an opportunity to deepen your understanding of Random Forests and Ensemble Methods.
    \item Feel free to ask questions to clarify concepts, seek examples, or discuss applications in data mining and AI.
  \end{itemize}
\end{frame}

\begin{frame}[fragile]
  \frametitle{Key Topics to Consider}
  \begin{enumerate}
    \item \textbf{What are Random Forests?}
    \begin{itemize}
      \item Ensemble learning method using multiple decision trees. 
      \item Outputs mode of classes (classification) or mean prediction (regression).
      \item \textbf{Key Point:} Addresses overfitting and improves accuracy vs. single decision trees.
    \end{itemize}

    \item \textbf{Importance of Ensemble Methods}
    \begin{itemize}
      \item Combining multiple models leads to better performance (diversity effect).
      \item Examples: Bagging (Bootstrap Aggregating), Boosting, Stacking.
    \end{itemize}

    \item \textbf{Applications in AI}
    \begin{itemize}
      \item Credit scoring, Medical diagnosis, Fraud detection, Recommendation systems.
    \end{itemize}
  \end{enumerate}
\end{frame}

\begin{frame}[fragile]
  \frametitle{Discussion and Engagement}
  \begin{itemize}
    \item \textbf{Why Use Random Forests?}
    \begin{itemize}
      \item High accuracy and robustness against overfitting.
      \item Handles large datasets with high dimensionality without variable deletion.
    \end{itemize}

    \item \textbf{Interpretation of Random Forest Models}
    \begin{itemize}
      \item Feature Importance: Identifies key features contributing to predictions.
      \item Out-of-Bag Error Estimation: Instant validation during training.
    \end{itemize}

    \item \textbf{Example Questions to Stimulate Discussion}
    \begin{itemize}
      \item How does Random Forest mitigate the issue of overfitting in decision trees?
      \item Can you provide an example of where ensemble methods notably outperform single models?
      \item How does data preprocessing impact the performance of Random Forests?
      \item What role does randomness play in the formation of trees within Random Forests?
    \end{itemize}
  \end{itemize}
\end{frame}


\end{document}