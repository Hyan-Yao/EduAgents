\documentclass[aspectratio=169]{beamer}

% Theme and Color Setup
\usetheme{Madrid}
\usecolortheme{whale}
\useinnertheme{rectangles}
\useoutertheme{miniframes}

% Additional Packages
\usepackage[utf8]{inputenc}
\usepackage[T1]{fontenc}
\usepackage{graphicx}
\usepackage{booktabs}
\usepackage{listings}
\usepackage{amsmath}
\usepackage{amssymb}
\usepackage{xcolor}
\usepackage{tikz}
\usepackage{pgfplots}
\pgfplotsset{compat=1.18}
\usetikzlibrary{positioning}
\usepackage{hyperref}

% Custom Colors
\definecolor{myblue}{RGB}{31, 73, 125}
\definecolor{mygray}{RGB}{100, 100, 100}
\definecolor{mygreen}{RGB}{0, 128, 0}
\definecolor{myorange}{RGB}{230, 126, 34}
\definecolor{mycodebackground}{RGB}{245, 245, 245}

% Set Theme Colors
\setbeamercolor{structure}{fg=myblue}
\setbeamercolor{frametitle}{fg=white, bg=myblue}
\setbeamercolor{title}{fg=myblue}
\setbeamercolor{section in toc}{fg=myblue}
\setbeamercolor{item projected}{fg=white, bg=myblue}
\setbeamercolor{block title}{bg=myblue!20, fg=myblue}
\setbeamercolor{block body}{bg=myblue!10}
\setbeamercolor{alerted text}{fg=myorange}

% Set Fonts
\setbeamerfont{title}{size=\Large, series=\bfseries}
\setbeamerfont{frametitle}{size=\large, series=\bfseries}
\setbeamerfont{caption}{size=\small}
\setbeamerfont{footnote}{size=\tiny}

% Custom Commands
\newcommand{\hilight}[1]{\colorbox{myorange!30}{#1}}
\newcommand{\concept}[1]{\textcolor{myblue}{\textbf{#1}}}

% Title Page Information
\title[Decision Trees and Random Forests]{Chapter 6: Decision Trees and Random Forests}
\author[J. Smith]{John Smith, Ph.D.}
\institute[University Name]{
  Department of Computer Science\\
  University Name\\
  \vspace{0.3cm}
  Email: email@university.edu\\
  Website: www.university.edu
}
\date{\today}

% Document Start
\begin{document}

\frame{\titlepage}

\begin{frame}[fragile]
    \titlepage
\end{frame}

\begin{frame}[fragile]
    \frametitle{Overview of Decision Trees}
    \begin{block}{Definition}
        \textbf{Decision Trees} are intuitive models used for classification and regression tasks in data mining. They represent decisions and their possible consequences using a tree-like structure.
    \end{block}
    \begin{itemize}
        \item \textbf{Nodes:} represent features or attributes
        \item \textbf{Branches:} signify decision rules based on features
        \item \textbf{Leaves:} denote final outcomes or class labels
    \end{itemize}
    
    \begin{block}{Key Characteristics}
        \begin{itemize}
            \item Easy to Interpret
            \item Non-Parametric
            \item Versatile: Handles both categorical and numerical data
        \end{itemize}
    \end{block}
\end{frame}

\begin{frame}[fragile]
    \frametitle{Example Decision Tree}
    \begin{itemize}
        \item Imagine deciding whether to play outside based on weather conditions:
        \begin{itemize}
            \item \textbf{Root Node:} Weather
            \item If sunny:
            \begin{itemize}
                \item \textbf{Node:} Wind
                \item If windy $\rightarrow$ \textbf{Leave:} Don’t play
                \item If not windy $\rightarrow$ \textbf{Leave:} Play
            \end{itemize}
            \item If not sunny $\rightarrow$ \textbf{Leave:} Don’t play
        \end{itemize}
    \end{itemize}
\end{frame}

\begin{frame}[fragile]
    \frametitle{Introduction to Random Forests}
    \begin{block}{Definition}
        \textbf{Random Forests} are an ensemble learning method that builds multiple decision trees and merges them for improved accuracy and stability in predictions.
    \end{block}
    \begin{block}{Key Characteristics}
        \begin{itemize}
            \item Reduced Overfitting
            \item Robustness to noise and missing values
            \item Quantification of feature importance
        \end{itemize}
    \end{block}
    
    \begin{block}{Example}
        In predicting house prices, a random forest might create 100 decision trees based on different samples and variable subsets. The final prediction is the average of all tree predictions.
    \end{block}
\end{frame}

\begin{frame}[fragile]
    \frametitle{Importance in Data Mining}
    \begin{itemize}
        \item **Interpretability**: Decision trees reveal paths for understanding model decisions, valuable in healthcare and finance.
        \item **Performance**: Random forests significantly improve predictive performance, making them a go-to algorithm for data science.
    \end{itemize}

    \begin{block}{Key Points to Emphasize}
        \begin{itemize}
            \item Simplicity of Decision Trees
            \item Advancements with Random Forests
            \item Application Areas: customer segmentation, fraud detection, recommendation systems
        \end{itemize}
    \end{block}
\end{frame}

\begin{frame}[fragile]
    \frametitle{Mathematical Foundations}
    \begin{itemize}
        \item \textbf{Entropy:} 
        \begin{equation}
            H(S) = -\sum_{i=1}^{C} p_i \log_2(p_i)
        \end{equation}
        \item \textbf{Gini Impurity:} 
        \begin{equation}
            Gini(S) = 1 - \sum_{i=1}^{C} p_i^2
        \end{equation}
    \end{itemize}
\end{frame}

\begin{frame}[fragile]
    \frametitle{Python Code Snippet}
    \begin{lstlisting}[language=Python]
from sklearn.tree import DecisionTreeClassifier
from sklearn.ensemble import RandomForestClassifier

# Create a Decision Tree Classifier
dtree = DecisionTreeClassifier()
# Create a Random Forest Classifier
rforest = RandomForestClassifier(n_estimators=100)
    \end{lstlisting}
\end{frame}

\begin{frame}[fragile]
    \frametitle{Understanding Decision Trees - Structure}
    A \textbf{decision tree} is a flowchart-like structure used for decision-making and predictive modeling. It consists of:
    
    \begin{itemize}
        \item \textbf{Nodes}:
        \begin{itemize}
            \item \textbf{Root Node}: The top node representing the entire dataset.
            \item \textbf{Decision Nodes}: Represent tests or questions about specific features, splitting the data based on feature values.
            \item \textbf{Leaf Nodes}: Terminal nodes providing final classification outcomes.
        \end{itemize}
        
        \item \textbf{Branches}: Connect nodes representing the outcome of a decision, leading to further nodes or leaf nodes based on answers to questions.
    \end{itemize}
\end{frame}

\begin{frame}[fragile]
    \frametitle{Understanding Decision Trees - Algorithm}
    \textbf{Basic Algorithm for Constructing a Decision Tree}
    
    The steps involved in constructing a decision tree are:
    \begin{enumerate}
        \item \textbf{Select the Best Feature}: Evaluate features in the dataset to determine the best split (using Gini impurity or Information Gain).
        \item \textbf{Splitting the Dataset}: Divide the dataset into subsets based on the selected feature, aiming for homogeneous subsets.
        \item \textbf{Create Decision Nodes}: Determine if further splits are needed; create leaf nodes if a subset is pure or stopping criteria are met.
        \item \textbf{Repeat}: Recursively continue the process for each decision node until all nodes are leaf nodes or stopping criteria are met.
        \item \textbf{Tree Pruning (optional)}: Remove nodes providing little predictive power to avoid overfitting.
    \end{enumerate}
\end{frame}

\begin{frame}[fragile]
    \frametitle{Understanding Decision Trees - Example and Formulas}
    \textbf{Example: Predicting Product Purchase}
    
    Imagine a dataset where you want to predict a purchase based on income and age:

    \begin{itemize}
        \item Root Node: "Is income > \$50,000?"
        \item If \textbf{Yes}: Next question "Is age < 30?"
            \begin{itemize}
                \item If \textbf{Yes}: Predict \textbf{Buy} (Leaf Node).
                \item If \textbf{No}: Predict \textbf{Do Not Buy} (Leaf Node).
            \end{itemize}
        \item If \textbf{No} (income question): Directly lead to leaf node predicting \textbf{Do Not Buy}.
    \end{itemize}
    
    \textbf{Formulas:}
    \begin{equation}
        Gini(D) = 1 - \sum_{k=1}^{K} (p_k)^2
    \end{equation}
    \begin{equation}
        IG(D, A) = Entropy(D) - \sum_{v \in A} \frac{|D_v|}{|D|} \cdot Entropy(D_v)
    \end{equation}
\end{frame}

\begin{frame}` to `\end{frame}

\begin{frame}[fragile]
  \frametitle{Splitting Criteria - Overview}
  \begin{block}{Understanding Splitting Criteria in Decision Trees}
    When constructing decision trees, selecting the right attribute to split nodes is crucial for model accuracy and efficiency. The splitting criteria determine how well the decision tree classifies data. 
  \end{block}
  
  \begin{itemize}
    \item **Gini Impurity**: Measures the likelihood of incorrect labeling.
    \item **Information Gain**: Measures the reduction in entropy after a dataset split.
  \end{itemize}
\end{frame}

\begin{frame}[fragile]
  \frametitle{Splitting Criteria - Gini Impurity}
  \begin{block}{Gini Impurity}
    \begin{itemize}
      \item **Definition**: Measures likelihood for incorrect labeling based on class distribution.
      \item **Formula**: 
      \begin{equation}
      G = 1 - \sum_{i=1}^{k} p_i^2
      \end{equation}
      where \( p_i \) is the proportion of class \( i \).
    \end{itemize}
    
    \begin{block}{Interpretation}
      \begin{itemize}
        \item Ranges from 0 (pure) to 0.5 (impure).
        \item Higher values indicate increased disorder.
      \end{itemize}
    \end{block}
    
    \begin{block}{Example}
    Consider a node with:
    \begin{itemize}
      \item Class A: 4 instances
      \item Class B: 6 instances
    \end{itemize}
    Calculating Gini Impurity:
    \begin{equation}
      G = 1 - \left( 0.4^2 + 0.6^2 \right) = 0.48
    \end{equation}
    \end{block}
  \end{block}
\end{frame}

\begin{frame}[fragile]
  \frametitle{Splitting Criteria - Information Gain}
  \begin{block}{Information Gain}
    \begin{itemize}
      \item **Definition**: Reduction in entropy after a dataset split.
      \item **Formula**:
      \begin{equation}
      IG(S, A) = H(S) - \sum_{v \in values(A)} \frac{|S_v|}{|S|} H(S_v)
      \end{equation}
      \item **Entropy Formula**:
      \begin{equation}
      H(S) = -\sum_{i=1}^{k} p_i \log_2(p_i)
      \end{equation}
    \end{itemize}

    \begin{block}{Interpretation}
      \begin{itemize}
        \item Chooses the attribute providing the highest reduction in uncertainty.
      \end{itemize}
    \end{block}

    \begin{block}{Example}
    \begin{itemize}
      \item \( H(S) = 1 \) (before split)
      \item \( H(S_A) = 0.5 \), \( H(S_B) = 0.7 \)
    \end{itemize}
    Information Gain:
    \begin{equation}
      IG(S, A) = 1 - \left( \frac{|S_A|}{|S|} H(S_A) + \frac{|S_B|}{|S|} H(S_B) \right)
    \end{equation}
    \end{block}
  \end{block}
\end{frame}

\begin{frame}[fragile]
    \frametitle{Advantages and Limitations of Decision Trees - Part 1}
    
    \begin{block}{Advantages of Decision Trees}
        \begin{enumerate}
            \item \textbf{Interpretability}
            \begin{itemize}
                \item Visually represent decisions and consequences.
                \item Example: Predicting customer purchase behavior based on features (e.g., age, income).
            \end{itemize}

            \item \textbf{Ease of Use}
            \begin{itemize}
                \item Minimal data preparation required.
                \item Handles both numerical and categorical data.
                \item Results are interpretable without advanced statistical knowledge.
            \end{itemize}

            \item \textbf{Non-Parametric Nature}
            \begin{itemize}
                \item No assumption of linear relationship between features and target variable.
                \item Versatile for various datasets.
            \end{itemize}

            \item \textbf{Feature Importance}
            \begin{itemize}
                \item Identifies significant predictors without extra feature selection.
            \end{itemize}
        \end{enumerate}
    \end{block}
\end{frame}

\begin{frame}[fragile]
    \frametitle{Advantages and Limitations of Decision Trees - Part 2}

    \begin{block}{Limitations of Decision Trees}
        \begin{enumerate}
            \item \textbf{Overfitting}
            \begin{itemize}
                \item Complexity leads to perfect fits for training data but poor performance on unseen data.
                \item \textit{Illustration}: A tree classifying noise as signal.
            \end{itemize}

            \item \textbf{Instability}
            \begin{itemize}
                \item Small changes in data can lead to drastically different tree structures.
            \end{itemize}

            \item \textbf{Biased with Imbalanced Datasets}
            \begin{itemize}
                \item Bias toward majority classes in the dataset.
            \end{itemize}

            \item \textbf{Lack of Generalization}
            \begin{itemize}
                \item Struggles with complex relationships and situations requiring sophisticated modeling.
            \end{itemize}
        \end{enumerate}
    \end{block}
\end{frame}

\begin{frame}[fragile]
    \frametitle{Mathematical Foundations of Decision Trees}

    \begin{block}{Key Points to Emphasize}
        \begin{itemize}
            \item \textbf{Versatile and Intuitive:} Suitable for classification and regression.
            \item \textbf{Model Integrity:} Pruning techniques or setting depth limits can prevent overfitting.
            \item \textbf{Use Case Suitability:} Ideal in contexts needing interpretability, such as medical diagnoses.
        \end{itemize}
    \end{block}

    \begin{block}{Mathematical Foundations}
        \begin{itemize}
            \item \textbf{Gini Impurity:} 
            \[
            Gini(D) = 1 - \sum_{j=1}^{n} p_j^2
            \]
            where \( p_j \) is the probability of class \( j \).

            \item \textbf{Information Gain:}
            \[
            IG(D, A) = H(D) - \sum_{v \in A} \frac{|D_v|}{|D|} H(D_v)
            \]
            where \( H(D) \) is the entropy of dataset \( D \).
        \end{itemize}
    \end{block}
\end{frame}

\begin{frame}[fragile]
    \frametitle{Introduction to Random Forests}
    \begin{block}{Overview of Random Forests}
        Random Forests is an ensemble learning technique that improves accuracy and generalization by aggregating multiple decision trees.
    \end{block}
    \begin{itemize}
        \item Combines diversity of decision trees
        \item Mitigates overfitting
        \item Effective for high-dimensional data
    \end{itemize}
\end{frame}

\begin{frame}[fragile]
    \frametitle{What is an Ensemble Learning Method?}
    \begin{itemize}
        \item Ensemble learning combines multiple models to enhance prediction accuracy.
        \item Random Forests leverage multiple decision trees for strong predictive performance.
    \end{itemize}
\end{frame}

\begin{frame}[fragile]
    \frametitle{How Random Forests Work}
    \begin{enumerate}
        \item \textbf{Building Decision Trees}
        \begin{itemize}
            \item Trained using bagging (Bootstrap Aggregating).
            \item Each tree is built from a random subset of training data.
        \end{itemize}
        \item \textbf{Making Predictions}
        \begin{itemize}
            \item For classification: majority voting from all trees.
            \item For regression: average of predictions from all trees.
        \end{itemize}
    \end{enumerate}
\end{frame}

\begin{frame}[fragile]
    \frametitle{Key Steps in Building a Random Forest}
    \begin{enumerate}
        \item \textbf{Data Sampling:} Randomly select samples with replacement (bootstrapping).
        \item \textbf{Feature Sampling:} Consider a random subset of features at each split.
        \item \textbf{Tree Training:} Build each decision tree on the sampled data.
        \item \textbf{Aggregation:} Combine predictions to produce the final prediction.
    \end{enumerate}
\end{frame}

\begin{frame}[fragile]
    \frametitle{Example of Random Forests}
    \begin{itemize}
        \item Dataset: predicting whether a customer will purchase a product based on features such as age, income, and previous purchases.
        \item Example predictions:
        \begin{itemize}
            \item Tree 1: based primarily on age.
            \item Tree 2: focused more on income.
        \end{itemize}
        \item Final prediction: majority vote from all trees reduces overfitting risk.
    \end{itemize}
\end{frame}

\begin{frame}[fragile]
    \frametitle{Key Points to Emphasize}
    \begin{itemize}
        \item \textbf{Accuracy and Robustness:} Better accuracy than single trees due to reduced variance.
        \item \textbf{Overfitting Mitigation:} Combining many trees mitigates the overfitting problem.
        \item \textbf{Handling High Dimensionality:} Random feature selection enhances performance on high-dimensional data.
    \end{itemize}
\end{frame}

\begin{frame}[fragile]
    \frametitle{Conclusion}
    Random Forests are a powerful machine learning tool that utilizes the strengths of ensemble methods.
    \begin{itemize}
        \item Introduces randomness in both sampling and feature selection.
        \item Capable of handling a variety of prediction tasks effectively.
    \end{itemize}
\end{frame}

\begin{frame}[fragile]
    \frametitle{Bagging Technique in Random Forests - Introduction}
    \begin{itemize}
        \item Bagging (Bootstrap Aggregating) is a fundamental ensemble method.
        \item It improves stability and accuracy of machine learning algorithms.
        \item Key benefits:
        \begin{itemize}
            \item Reduces overfitting
            \item Enhances model performance by combining predictions
        \end{itemize}
    \end{itemize}
\end{frame}

\begin{frame}[fragile]
    \frametitle{Bagging Technique in Random Forests - Bootstrapping Process}
    \begin{block}{Bootstrapping Explained}
        \begin{itemize}
            \item Resampling technique to create multiple subsets.
            \item From original dataset of size \( n \), form \( m \) bootstrapped datasets with replacement.
        \end{itemize}
    \end{block}
    
    \begin{block}{Illustration of Bootstrapping}
        \begin{itemize}
            \item Original Dataset: \{A, B, C, D, E\}
            \item Bootstrapped Samples:
            \begin{itemize}
                \item Sample 1: \{A, A, C, D, E\}
                \item Sample 2: \{B, C, C, E, A\}
            \end{itemize}
            \item Points may repeat or be omitted in each sample.
        \end{itemize}
    \end{block}
\end{frame}

\begin{frame}[fragile]
    \frametitle{Bagging Technique in Random Forests - Decision Trees & Voting Mechanism}
    \begin{itemize}
        \item For each bootstrapped dataset, train a distinct decision tree.
        \item Each tree may differ due to unique training data.
    \end{itemize}
    
    \begin{block}{Majority Voting Mechanism}
        \begin{itemize}
            \item Predictions aggregated from all trees.
            \item Classification:
            \begin{itemize}
                \item Final prediction is determined by majority vote.
                \item Example: 
                \begin{itemize}
                    \item Tree 1: Class A, Tree 2: Class B, Tree 3: Class B,
                    \item Final prediction: Class B (2 vs 1).
                \end{itemize}
            \end{itemize}
            \item Regression:
            \begin{itemize}
                \item Predictions averaged for final value.
            \end{itemize}
        \end{itemize}
        
        \begin{equation}
        Y = \text{arg max}_{y} \bigg( \sum_{i=1}^{m} \mathbb{I}(h_i(X) = y) \bigg)
        \end{equation}
    \end{block}
\end{frame}

\begin{frame}[fragile]
    \frametitle{Bagging Technique in Random Forests - Key Points and Summary}
    \begin{itemize}
        \item \textbf{Reduced Overfitting:} Aggregation stabilizes predictions.
        \item \textbf{Robustness:} Resistant to noise and outliers.
        \item \textbf{Parallel Processing:} Trees can be grown independently, improving efficiency.
    \end{itemize}

    \begin{block}{Summary}
        \begin{itemize}
            \item Bagging leverages bootstrapping to generate multiple trees.
            \item Majority voting provides final predictions.
            \item Balances bias and variance for more accurate models.
        \end{itemize}
    \end{block}
\end{frame}

\begin{frame}[fragile]
    \frametitle{Feature Importance in Random Forests - Introduction}
    \begin{block}{Introduction}
        Understanding which features contribute the most to the predictive power of a model is crucial in machine learning. Random forests utilize a robust method to assess feature importance, shedding light on how different features impact predictions and decision-making.
    \end{block}
\end{frame}

\begin{frame}[fragile]
    \frametitle{Feature Importance in Random Forests - Key Concepts}
    \begin{enumerate}
        \item \textbf{What is Feature Importance?}
            \begin{itemize}
                \item Quantifies the contribution of each feature in making predictions.
                \item High importance indicates significant influence, low importance suggests a minimal effect.
            \end{itemize}
        
        \item \textbf{Assessing Feature Importance in Random Forests}
            \begin{itemize}
                \item \textbf{Mean Decrease Impurity (MDI)}: Measures how much each feature reduces impurity when making splits.
                \item \textbf{Mean Decrease Accuracy (MDA)}: Assesses how permuting the feature values affects model accuracy.
            \end{itemize}
    \end{enumerate}
\end{frame}

\begin{frame}[fragile]
    \frametitle{Feature Importance in Random Forests - Calculating Importance}
    \begin{block}{Mean Decrease Impurity (MDI)}
        \begin{itemize}
            \item For each decision tree, the impurity decrease attributed to a feature is summed.
        \end{itemize}
        \begin{equation}
            Importance(f) = \sum_{t \in T} \sum_{j \in J_t} (p_j^t \cdot Impurity_{j}^{before} - Impurity_{j}^{after})
        \end{equation}
        Where:
        \begin{itemize}
            \item \( T \) = total trees in the forest
            \item \( J_t \) = splits made by tree \( t \)
            \item \( p_j^t \) = proportion of samples reaching split \( j \)
            \item \( Impurity_{j}^{before/after} \) = impurity before and after the split
        \end{itemize}
    \end{block}
    
    \begin{block}{Mean Decrease Accuracy (MDA)}
        \begin{itemize}
            \item Evaluates how model accuracy changes with feature value permutation.
            \item Important if accuracy decreases significantly upon permuting a feature.
        \end{itemize}
    \end{block}
\end{frame}

\begin{frame}[fragile]
    \frametitle{Model Evaluation Metrics - Overview}
    \begin{block}{Understanding Performance Metrics}
        When evaluating the effectiveness of decision trees and random forests, several key metrics provide insights into model performance. These metrics help assess how well the models predict outcomes based on input data.
    \end{block}
\end{frame}

\begin{frame}[fragile]
    \frametitle{Model Evaluation Metrics - Key Metrics}
    \begin{enumerate}
        \item \textbf{Accuracy}
            \begin{itemize}
                \item \textbf{Definition}: The ratio of correctly predicted instances to the total instances in the dataset.
                \item \textbf{Formula}:
                \[
                \text{Accuracy} = \frac{\text{True Positives} + \text{True Negatives}}{\text{Total Instances}}
                \]
                \item \textbf{Example}: If a model predicts 80 out of 100 instances correctly, the accuracy is 80\%.
            \end{itemize}
        
        \item \textbf{Precision}
            \begin{itemize}
                \item \textbf{Definition}: The ratio of correctly predicted positive observations to the total predicted positives.
                \item \textbf{Formula}:
                \[
                \text{Precision} = \frac{\text{True Positives}}{\text{True Positives} + \text{False Positives}}
                \]
                \item \textbf{Example}: If a model identifies 30 out of 50 predicted positive cases as actually positive, its precision is 60\%.
            \end{itemize}
    \end{enumerate}
\end{frame}

\begin{frame}[fragile]
    \frametitle{Model Evaluation Metrics - Continued}
    \begin{enumerate}
        \setcounter{enumi}{2} % Continue the enumeration
        \item \textbf{Recall (Sensitivity)}
            \begin{itemize}
                \item \textbf{Definition}: The ratio of correctly predicted positive observations to the actual positives.
                \item \textbf{Formula}:
                \[
                \text{Recall} = \frac{\text{True Positives}}{\text{True Positives} + \text{False Negatives}}
                \]
                \item \textbf{Example}: If there are 70 actual positive cases and the model identifies 50 correctly, the recall is approximately 71.4\%.
            \end{itemize}
        
        \item \textbf{F1 Score}
            \begin{itemize}
                \item \textbf{Definition}: The harmonic mean of precision and recall; provides a balance between the two metrics.
                \item \textbf{Formula}:
                \[
                F1 = 2 \times \frac{\text{Precision} \times \text{Recall}}{\text{Precision} + \text{Recall}}
                \]
                \item \textbf{Example}: If precision is 60\% and recall is 71.4\%, the F1 score is around 65.3\%.
            \end{itemize}
    \end{enumerate}
\end{frame}

\begin{frame}[fragile]
    \frametitle{Model Evaluation Metrics - Key Points}
    \begin{itemize}
        \item \textbf{Purpose of Metrics}: Evaluating models helps in understanding their strengths and weaknesses, guiding further improvements.
        \item \textbf{Trade-offs}: A high accuracy doesn't always imply a good model, especially in imbalanced datasets. Hence, precision and recall are crucial.
        \item \textbf{F1 Score}: Particularly valuable when both false positives and false negatives carry significant penalties.
    \end{itemize}
\end{frame}

\begin{frame}[fragile]
    \frametitle{Model Evaluation Metrics - Practical Application}
    To compute these metrics in Python using Scikit-learn, you can use the following code:
    \begin{lstlisting}[language=Python]
from sklearn.metrics import accuracy_score, precision_score, recall_score, f1_score

# Assuming y_true are your actual labels and y_pred are your predicted labels
accuracy = accuracy_score(y_true, y_pred)
precision = precision_score(y_true, y_pred, average='weighted')
recall = recall_score(y_true, y_pred, average='weighted')
f1 = f1_score(y_true, y_pred, average='weighted')

print(f'Accuracy: {accuracy:.2f}')
print(f'Precision: {precision:.2f}')
print(f'Recall: {recall:.2f}')
print(f'F1 Score: {f1:.2f}')
    \end{lstlisting}
\end{frame}

\begin{frame}[fragile]
    \frametitle{Avoiding Overfitting - Understanding Overfitting}
    \begin{itemize}
        \item Overfitting occurs when a model learns both the underlying patterns and the noise in training data. 
        \item This leads to excellent performance on training data but poor performance on unseen test data.
        \item In decision trees and random forests, overfitting is common due to their capacity to create complex models.
    \end{itemize}
\end{frame}

\begin{frame}[fragile]
    \frametitle{Avoiding Overfitting - Strategies to Prevent Overfitting}
    \begin{enumerate}
        \item \textbf{Pruning}
        \begin{itemize}
            \item Definition: Removing parts of the tree with little predictive power to reduce complexity.
            \item \textbf{Types of Pruning:}
            \begin{itemize}
                \item \textbf{Pre-pruning (Early Stopping)}: Prevents the tree from growing too deep.
                \item \textbf{Post-pruning}: Full tree growth followed by the removal of insignificant branches based on validation data.
            \end{itemize}
        \end{itemize}
        
        \item \textbf{Tuning Hyperparameters}
        \begin{itemize}
            \item Hyperparameters influence the training process. Examples include:
            \begin{itemize}
                \item Max Depth: Limits tree depth.
                \item Min Samples Split: Minimum samples required to split a node.
                \item Min Samples Leaf: Minimum samples in a leaf node.
                \item Number of Trees (in Random Forests): Affects performance and complexity.
            \end{itemize}
        \end{itemize}
    \end{enumerate}
\end{frame}

\begin{frame}[fragile]
    \frametitle{Avoiding Overfitting - Key Points and Conclusion}
    \begin{itemize}
        \item Overfitting is a concern in model evaluation; it affects generalizability.
        \item Essential techniques like pruning and hyperparameter tuning help balance bias and variance.
        \item Use cross-validation during tuning to estimate model performance effectively.
        \item \textbf{Conclusion:} Implementing these strategies enhances the ability of decision trees and random forests to generalize to new data.
    \end{itemize}
\end{frame}

\begin{frame}[fragile]
    \frametitle{Practical Applications of Decision Trees and Random Forests}

    \begin{block}{Introduction}
        Decision Trees and Random Forests are powerful machine learning algorithms widely used across industries for decision-making and predictions. Their interpretability and versatility make them valuable tools in real-world scenarios.
    \end{block}
\end{frame}

\begin{frame}[fragile]
    \frametitle{Applications in Various Industries}

    \begin{enumerate}
        \item \textbf{Finance}
            \begin{itemize}
                \item \textit{Credit Scoring}: Used to assess loan applicants' creditworthiness.
                \item \textit{Example}: Predicting loan repayment based on income and debts.
            \end{itemize}

        \item \textbf{Healthcare}
            \begin{itemize}
                \item \textit{Disease Diagnosis}: Classifies patients based on various health metrics, aiding in timely interventions.
                \item \textit{Example}: Predicting diabetes likelihood using age, BMI, and family history.
                \item \textit{Treatment Recommendation}: Suggests treatments based on patient data.
            \end{itemize}

        \item \textbf{Marketing}
            \begin{itemize}
                \item \textit{Customer Segmentation}: Segments audiences based on behavior, demographics, and activity.
                \item \textit{Example}: Classifying customers as ‘frequent buyers’ or ‘discount shoppers.’
                \item \textit{Churn Prediction}: Identifies at-risk customers to help retain them.
            \end{itemize}
    \end{enumerate}
\end{frame}

\begin{frame}[fragile]
    \frametitle{Key Points and Conclusion}

    \begin{block}{Key Points}
        \begin{itemize}
            \item \textbf{Interpretability}: Decision Trees are easy to visualize, making them accessible for non-technical stakeholders.
            \item \textbf{Robustness of Random Forests}: Average predictions from multiple trees reduce overfitting, leading to more accurate results.
            \item \textbf{Versatility}: Applicable across various data types and sectors.
        \end{itemize}
    \end{block}

    \begin{block}{Conclusion}
        Decision Trees and Random Forests play a crucial role in data-driven decision-making, providing actionable insights in finance, healthcare, and marketing. As industries evolve, their importance will likely increase.
    \end{block}
\end{frame}

\begin{frame}[fragile]
    \frametitle{Ethical Considerations - Introduction}
    As decision trees and random forests become increasingly integrated into decision-making processes across various sectors, it is essential to address the ethical implications associated with their use. This includes:
    \begin{itemize}
        \item Concerns related to data privacy
        \item Algorithmic bias
        \item Potential consequences of automated decisions
    \end{itemize}
\end{frame}

\begin{frame}[fragile]
    \frametitle{Ethical Considerations - Data Privacy}
    \begin{block}{Definition}
        Data privacy refers to the proper handling of sensitive and personal information, ensuring that individuals' data is protected from unauthorized access and misuse.
    \end{block}
    
    \begin{itemize}
        \item \textbf{Data Collection}: Significant data is required for training decision trees and random forests. Compliance with legal frameworks is essential (e.g., GDPR, HIPAA).
        \item \textbf{Anonymization}: Proper anonymization techniques must be employed before data processing.
    \end{itemize}
    
    \begin{exampleblock}{Example}
        A healthcare application using random forests to predict patient outcomes must ensure that all data is anonymized and stored securely, adhering to regulatory standards.
    \end{exampleblock}
\end{frame}

\begin{frame}[fragile]
    \frametitle{Ethical Considerations - Algorithmic Bias}
    \begin{block}{Definition}
        Algorithmic bias occurs when a decision-making process leads to unfair outcomes, often due to imbalanced training data or flawed assumptions in the model.
    \end{block}
    
    \begin{itemize}
        \item \textbf{Sources of Bias}:
            \begin{itemize}
                \item Historical Bias: Training data reflecting societal inequalities may reproduce these biases.
                \item Feature Selection: Incorrectly chosen features can lead to misleading conclusions.
            \end{itemize}
    \end{itemize}
    
    \begin{exampleblock}{Example}
        A decision tree model for hiring might inadvertently favor certain demographics if previous practices are reflected in the training data, leading to discrimination.
    \end{exampleblock}
\end{frame}

\begin{frame}[fragile]
    \frametitle{Ethical Considerations - Consequences & Mitigation}
    \begin{block}{Impact of Unethical Decisions}
        \begin{itemize}
            \item Loss of trust in technological systems.
            \item Legal ramifications and financial penalties for organizations.
            \item Societal harm, perpetuating stereotypes and inequality.
        \end{itemize}
    \end{block}
    
    \begin{block}{Mitigating Ethical Risks}
        \begin{enumerate}
            \item Transparency: Companies must be transparent about model development and data usage.
            \item Bias Audits: Regular audits for bias and fairness should be conducted.
            \item Diverse Data Sets: Using diverse data helps avoid bias and improves model generalization.
        \end{enumerate}
    \end{block}
\end{frame}

\begin{frame}[fragile]
    \frametitle{Ethical Considerations - Conclusion}
    While decision trees and random forests are powerful tools, their application must be governed by strict ethical guidelines to protect data privacy and eliminate bias. 
    \begin{itemize}
        \item Ethical use is critical to maintaining public trust and ensuring fair outcomes.
        \item Compliance with data privacy laws and proactive bias mitigation are crucial for ethical AI development.
    \end{itemize}
    \begin{block}{Note}
        Ethical policies and practices in data science are evolving and require continuous education and dialogue among practitioners.
    \end{block}
\end{frame}

\begin{frame}[fragile]
  \frametitle{Future Trends - Overview}
  \begin{block}{Emerging Directions}
    As we explore the future of decision trees and random forests, several trends are influencing their development:
    \begin{itemize}
      \item Integration with deep learning.
      \item Automation in machine learning (AutoML).
      \item Emphasis on explainable AI (XAI).
      \item Advances in scalability for big data.
      \item Optimization techniques for improved model training.
      \item Strategies for handling imbalanced datasets.
    \end{itemize}
  \end{block}
\end{frame}

\begin{frame}[fragile]
  \frametitle{Future Trends - Integration of Deep Learning}
  \begin{block}{1. Integration of Deep Learning Techniques}
    \begin{itemize}
      \item \textbf{Concept:} Hybrid models combining decision trees with deep learning frameworks.
      \item \textbf{Explanation:} Integrating decision tree principles with neural networks enhances interpretability and predictive power.
      \item \textbf{Example:} Decision trees can generate features for neural networks, improving image classification accuracy.
    \end{itemize}
  \end{block}
\end{frame}

\begin{frame}[fragile]
  \frametitle{Future Trends - Automated ML and Explainable AI}
  \begin{block}{2. Automated Machine Learning (AutoML)}
    \begin{itemize}
      \item \textbf{Concept:} Automation in machine learning processes.
      \item \textbf{Explanation:} Simplifies model selection and tuning, making tree-based algorithms accessible to non-experts.
      \item \textbf{Example:} H2O.ai optimizes parameters for random forests without extensive manual tuning.
    \end{itemize}
  \end{block}

  \begin{block}{3. Explainable AI (XAI)}
    \begin{itemize}
      \item \textbf{Concept:} The necessity for transparency in complex models.
      \item \textbf{Explanation:} Decision trees offer clear decision paths, aiding ethical evaluations in AI.
      \item \textbf{Key Point:} Comprehensible decisions in models are vital for user trust in sensitive fields like healthcare and finance.
    \end{itemize}
  \end{block}
\end{frame}

\begin{frame}[fragile]
  \frametitle{Future Trends - Scalability and Optimization}
  \begin{block}{4. Scalability and Big Data Approaches}
    \begin{itemize}
      \item \textbf{Concept:} Efficiently handling large datasets.
      \item \textbf{Explanation:} Enhancements are aimed at improving scalability through frameworks like Apache Spark.
      \item \textbf{Example:} Cloud platforms will facilitate large-scale training of random forest models in real-time.
    \end{itemize}
  \end{block}

  \begin{block}{5. Optimization Techniques in Model Training}
    \begin{itemize}
      \item \textbf{Concept:} Improving training methodologies.
      \item \textbf{Explanation:} Research into novel optimization methods speeds up decision tree training and enhances accuracy.
      \item \textbf{Formula:} A typical optimization objective in regression trees is minimizing mean squared error (MSE):
      \begin{equation}
        \text{MSE} = \frac{1}{n}\sum_{i=1}^{n} (y_i - \hat{y}_i)^2
      \end{equation}
    \end{itemize}
  \end{block}
\end{frame}

\begin{frame}[fragile]
  \frametitle{Future Trends - Handling Imbalanced Data}
  \begin{block}{6. Handling Imbalanced Data}
    \begin{itemize}
      \item \textbf{Concept:} Developing strategies for skewed datasets.
      \item \textbf{Explanation:} New algorithms are designed to improve decision tree robustness against imbalanced classes.
      \item \textbf{Example:} SMOTE (Synthetic Minority Over-sampling Technique) enhances predictive performance with decision trees.
    \end{itemize}
  \end{block}
\end{frame}


\end{document}