\documentclass[aspectratio=169]{beamer}
% Theme and Color Setup
\usetheme{Madrid}
\usecolortheme{whale}
\useinnertheme{rectangles}
\useoutertheme{miniframes}
% Additional Packages
\usepackage[utf8]{inputenc}
\usepackage[T1]{fontenc}
\usepackage{graphicx}
\usepackage{booktabs}
\usepackage{listings}
\usepackage{amsmath}
\usepackage{amssymb}
\usepackage{xcolor}
\usepackage{tikz}
\usepackage{pgfplots}
\pgfplotsset{compat=1.18}
\usetikzlibrary{positioning}
\usepackage{hyperref}
% Custom Colors
\definecolor{myblue}{RGB}{31, 73, 125}
\definecolor{mygray}{RGB}{100, 100, 100}
\definecolor{mygreen}{RGB}{0, 128, 0}
\definecolor{myorange}{RGB}{230, 126, 34}
\definecolor{mycodebackground}{RGB}{245, 245, 245}
% Set Theme Colors
\setbeamercolor{structure}{fg=myblue}
\setbeamercolor{frametitle}{fg=white, bg=myblue}
\setbeamercolor{title}{fg=myblue}
\setbeamercolor{section in toc}{fg=myblue}
\setbeamercolor{item projected}{fg=white, bg=myblue}
\setbeamercolor{block title}{bg=myblue!20, fg=myblue}
\setbeamercolor{block body}{bg=myblue!10}
\setbeamercolor{alerted text}{fg=myorange}
% Set Fonts
\setbeamerfont{title}{size=\Large, series=\bfseries}
\setbeamerfont{frametitle}{size=\large, series=\bfseries}
\setbeamerfont{caption}{size=\small}
\setbeamerfont{footnote}{size=\tiny}
% Custom Commands
\newcommand{\hilight}[1]{\colorbox{myorange!30}{#1}}
\newcommand{\concept}[1]{\textcolor{myblue}{\textbf{#1}}}
% Title Page Information
\title[Introduction to Data Mining]{Chapter 1: Introduction to Data Mining}
\author[J. Smith]{John Smith, Ph.D.}
\date{\today}
% Document Start
\begin{document}

\frame{\titlepage}

\begin{frame}[fragile]
    \titlepage
\end{frame}

\begin{frame}[fragile]
    \frametitle{What is Data Mining?}
    \begin{block}{Definition}
        Data mining is the process of discovering patterns, correlations, and anomalies within large sets of data through various machine learning, statistical, and computational techniques. It transforms raw data into meaningful information, enabling informed decision-making.
    \end{block}
\end{frame}

\begin{frame}[fragile]
    \frametitle{Significance of Data Mining}
    Data mining plays a crucial role across various industries by:
    \begin{itemize}
        \item \textbf{Healthcare:} Analyzing patient records to identify treatment effectiveness and disease outbreaks.
        \item \textbf{Finance:} Detecting fraudulent transactions by identifying unusual spending behaviors.
        \item \textbf{Retail:} Personalizing customer experiences through purchase history analysis to optimize product recommendations.
        \item \textbf{Telecommunications:} Improving customer retention by analyzing call patterns and service usage.
    \end{itemize}
\end{frame}

\begin{frame}[fragile]
    \frametitle{Key Methodologies in Data Mining}
    \begin{enumerate}
        \item \textbf{Classification:} Assigning items to predefined categories based on their attributes.  
              \begin{itemize}
                  \item Example: Classifying emails as spam or non-spam using algorithms like Decision Trees or Support Vector Machines (SVM).
              \end{itemize}
        
        \item \textbf{Regression:} Predicting a continuous outcome variable based on input variables.  
              \begin{equation}
                  Y = a + bX
              \end{equation}
              where \( Y \) is the dependent variable, \( X \) is the independent variable, and \( a, b \) are coefficients.  
              Example: Predicting housing prices based on features like square footage, location, etc.
        
        \item \textbf{Clustering:} Grouping similar data points within a dataset without predefined labels.   
              \begin{itemize}
                  \item Example: Customer segmentation in marketing can identify different purchasing behaviors using algorithms such as K-Means or Hierarchical Clustering.
              \end{itemize}
        
        \item \textbf{Association Rule Learning:} Discovering interesting relationships between variables in large databases.  
              \begin{itemize}
                  \item Example: Using the Apriori algorithm to identify products frequently bought together, such as bread and butter (often used in market basket analysis).
              \end{itemize}
    \end{enumerate}
\end{frame}

\begin{frame}[fragile]
    \frametitle{Key Points to Emphasize}
    \begin{itemize}
        \item Data mining leverages advanced analytics to extract actionable insights from data.
        \item Understanding methodologies such as classification, regression, clustering, and association rules is fundamental for effective data analysis.
        \item The significance of data mining in real-world applications illustrates its impactful use across various sectors.
    \end{itemize}
\end{frame}

\begin{frame}[fragile]
    \frametitle{Next Steps: Learning Objectives}
    In the following slide, we will outline the key learning objectives for this course, focusing on:
    \begin{itemize}
        \item Understanding data mining fundamentals
        \item Techniques and tools for data mining
        \item Ethical implications of data mining
    \end{itemize}
\end{frame}

\begin{frame}[fragile]{Learning Objectives - Overview}
    \begin{block}{Welcome}
        Welcome to Data Mining! In this course, we will embark on a journey to uncover the fundamentals and intricacies of data mining. Our key learning objectives are designed to provide you with a solid foundation and critical understanding of the field.
    \end{block}
\end{frame}

\begin{frame}[fragile]{Learning Objectives - Fundamentals}
    \begin{enumerate}
        \item \textbf{Understanding Data Mining Fundamentals}
        \begin{itemize}
            \item \textbf{Definition}: Data mining is the process of discovering patterns and knowledge from large amounts of data using techniques from statistics, machine learning, and database systems.
            \item \textbf{Key Concepts}:
            \begin{itemize}
                \item \textbf{Data Sources}: Structured vs. unstructured data.
                \item \textbf{Data Preprocessing}: Cleaning and preparing data for analysis.
            \end{itemize}
            \item \textbf{Example}: Consider a retail store that analyzes transaction data to identify shopping patterns among customers.
        \end{itemize}
    \end{enumerate}
\end{frame}

\begin{frame}[fragile]{Learning Objectives - Techniques}
    \begin{enumerate}
        \setcounter{enumi}{1}
        \item \textbf{Familiarity with Data Mining Techniques}
        \begin{itemize}
            \item \textbf{Techniques Overview}:
            \begin{itemize}
                \item \textbf{Classification}: Assigning items to predefined categories (e.g., spam detection in emails).
                \item \textbf{Regression}: Predicting a continuous-valued attribute (e.g., predicting house prices).
                \item \textbf{Clustering}: Grouping similar data points (e.g., customer segmentation).
                \item \textbf{Association Rule Learning}: Discovering interesting relationships (e.g., "Customers who bought bread also bought butter").
            \end{itemize}
            \item \textbf{Key Point}: Understanding which technique to apply in various scenarios is crucial for effective data mining.
        \end{itemize}
    \end{enumerate}
\end{frame}

\begin{frame}[fragile]{Learning Objectives - Tools}
    \begin{enumerate}
        \setcounter{enumi}{2}
        \item \textbf{Proficiency in Data Mining Tools}
        \begin{itemize}
            \item \textbf{Popular Tools and Software}:
            \begin{itemize}
                \item \textbf{RapidMiner}: User-friendly, with a drag-and-drop interface for data preparation and modeling.
                \item \textbf{Weka}: Powerful tool for machine learning and data mining, written in Java.
                \item \textbf{Python Libraries}:
                \begin{itemize}
                    \item \textbf{Pandas} for data manipulation.
                    \item \textbf{Scikit-learn} for implementing various algorithms.
                \end{itemize}
            \end{itemize}
            \item \textbf{Example}: Using Python’s Scikit-learn to implement a simple classification model:
            \begin{lstlisting}[language=Python]
from sklearn.datasets import load_iris
from sklearn.model_selection import train_test_split
from sklearn.ensemble import RandomForestClassifier

# Load data
data = load_iris()
X = data.data
y = data.target

# Split the dataset
X_train, X_test, y_train, y_test = train_test_split(X, y, test_size=0.2)

# Model training
model = RandomForestClassifier()
model.fit(X_train, y_train)
            \end{lstlisting}
        \end{itemize}
    \end{enumerate}
\end{frame}

\begin{frame}[fragile]{Learning Objectives - Ethics}
    \begin{enumerate}
        \setcounter{enumi}{3}
        \item \textbf{Awareness of Ethical Implications}
        \begin{itemize}
            \item \textbf{Importance of Ethics in Data Mining}: Recognizing the impact of data mining decisions on privacy, security, and fairness.
            \item \textbf{Key Issues}:
            \begin{itemize}
                \item \textbf{Data Privacy}: Understanding how to handle personal data and the consequences of breaches.
                \item \textbf{Bias in Algorithms}: Acknowledging and mitigating bias to avoid discrimination.
            \end{itemize}
            \item \textbf{Key Point}: As future data miners, it is imperative to adhere to ethical standards to protect individuals' rights and data integrity.
        \end{itemize}
    \end{enumerate}
\end{frame}

\begin{frame}[fragile]{Learning Objectives - Conclusion}
    \begin{block}{Conclusion}
        By the end of this course, you should be confident in your abilities to understand, apply, and ethically navigate the world of data mining. Let's get started on this exciting journey!
    \end{block}
\end{frame}

\begin{frame}[fragile]
    \frametitle{History of Data Mining}
    \begin{block}{Overview}
        Data mining is an interdisciplinary field that has evolved significantly, shaped by advancements in technology and methodologies. This slide highlights key milestones in data mining's history, illustrating its growth from simple data analysis techniques to sophisticated modern approaches.
    \end{block}
\end{frame}

\begin{frame}[fragile]
    \frametitle{Origins (1960s - 1980s)}
    \begin{itemize}
        \item \textbf{Early Data Analysis}
            \begin{itemize}
                \item Data analysis was primarily reliant on statistical methods.
                \item Development of techniques such as regression analysis and cluster analysis.
            \end{itemize}
        \item \textbf{Database Management Systems (DBMS)}
            \begin{itemize}
                \item Emergence of DBMS in the 1970s enabled complex data manipulation.
                \item Improved data storage and retrieval processes.
            \end{itemize}
    \end{itemize}
\end{frame}

\begin{frame}[fragile]
    \frametitle{The Birth of Data Mining (1990s)}
    \begin{itemize}
        \item \textbf{Terminology}
            \begin{itemize}
                \item The term "data mining" became widely used in the early 1990s.
            \end{itemize}
        \item \textbf{Algorithm Development}
            \begin{itemize}
                \item Introduction of algorithms such as Decision Trees, Neural Networks, and Support Vector Machines (SVM).
            \end{itemize}
        \item \textbf{Commercial Interest}
            \begin{itemize}
                \item Companies recognized data mining's potential for market analysis and customer segmentation.
            \end{itemize}
    \end{itemize}
\end{frame}

\begin{frame}[fragile]
    \frametitle{Expansion and Proliferation (2000s)}
    \begin{itemize}
        \item \textbf{Big Data Revolution}
            \begin{itemize}
                \item Explosion of digital data (social media, IoT) led to advanced data mining techniques.
                \item Emergence of tools and platforms (e.g., Apache Hadoop) for big data processing.
            \end{itemize}
        \item \textbf{Techniques Diversification}
            \begin{itemize}
                \item Development of ensemble methods like Random Forests and boosting algorithms improved prediction accuracy.
            \end{itemize}
    \end{itemize}
\end{frame}

\begin{frame}[fragile]
    \frametitle{Modern Practices (2010s to Present)}
    \begin{itemize}
        \item \textbf{Machine Learning and AI Integration}
            \begin{itemize}
                \item Heavy incorporation of machine learning enables predictive analytics and real-time data processing.
            \end{itemize}
        \item \textbf{Ethics and Regulation}
            \begin{itemize}
                \item Increased awareness of data privacy and biases.
                \item Initiatives and regulations in data handling (e.g., GDPR).
            \end{itemize}
    \end{itemize}
\end{frame}

\begin{frame}[fragile]
    \frametitle{Key Points to Emphasize}
    \begin{itemize}
        \item Transition from basic statistical methods to complex machine learning algorithms.
        \item The expansion of big data has necessitated innovative data mining approaches.
        \item Ethical considerations are now critical, impacting data collection, analysis, and usage.
    \end{itemize}
\end{frame}

\begin{frame}[fragile]
    \frametitle{Illustrative Example}
    \begin{itemize}
        \item \textbf{Example of Evolution:}
            \begin{itemize}
                \item A retail company using frequency analysis in the 1990s now employs predictive models for sales trends, accounting for customer behavior, seasonal variations, and regional preferences.
            \end{itemize}
    \end{itemize}
\end{frame}

\begin{frame}[fragile]{Key Concepts in Data Mining - Overview}
    The essential concepts in data mining include:
    \begin{itemize}
        \item Data Preprocessing
        \item Data Exploration
        \item Model Development
    \end{itemize}
\end{frame}

\begin{frame}[fragile]{Key Concepts in Data Mining - Data Preprocessing}
    Data preprocessing is crucial for preparing raw data for analysis. Key processes include:
    \begin{itemize}
        \item \textbf{Data Cleaning}: Removing noise and inconsistencies.
        \begin{itemize}
            \item Example: Correcting non-numeric entries in survey data.
        \end{itemize}
        \item \textbf{Data Transformation}: Converting data into a suitable format.
        \begin{itemize}
            \item Example: Normalizing income data to a standard range.
        \end{itemize}
        \item \textbf{Data Reduction}: Reducing volume while maintaining data integrity.
        \begin{itemize}
            \item Example: Using PCA to reduce dimensionality.
        \end{itemize}
    \end{itemize}
\end{frame}

\begin{frame}[fragile]{Key Concepts in Data Mining - Data Exploration}
    Data exploration involves analyzing data to find patterns and relationships. Key elements include:
    \begin{itemize}
        \item \textbf{Descriptive Statistics}: Measures such as mean, median, mode.
        \begin{block}{Key Formula}
            \begin{equation}
                \text{Mean} = \frac{\sum_{i=1}^{n} x_i}{n}
            \end{equation}
        \end{block}
        \item \textbf{Visualization Tools}: Graphs such as histograms and scatter plots help identify trends.
        \item \textbf{Association Rules}: Finding interesting relationships between variables.
        \begin{itemize}
            \item Example: Market basket analysis to find frequently bought items.
        \end{itemize}
    \end{itemize}
\end{frame}

\begin{frame}[fragile]{Key Concepts in Data Mining - Model Development}
    Model development focuses on creating predictive models from explored data. Key aspects include:
    \begin{itemize}
        \item \textbf{Supervised Learning}: Training models on labeled data.
        \begin{itemize}
            \item Example: Predicting housing prices based on various features.
        \end{itemize}
        \item \textbf{Unsupervised Learning}: Discovering patterns in unlabeled data.
        \begin{itemize}
            \item Example: Clustering customers based on buying behaviors.
        \end{itemize}
        \item \textbf{Model Evaluation}: Using metrics to assess model performance.
        \begin{block}{Key Formula}
            \begin{equation}
                \text{Accuracy} = \frac{TP + TN}{TP + TN + FP + FN}
            \end{equation}
        \end{block}
    \end{itemize}
\end{frame}

\begin{frame}[fragile]{Key Concepts in Data Mining - Conclusion}
    Understanding these key concepts is vital for grasping more advanced data mining techniques. 
    \begin{itemize}
        \item Data preprocessing ensures quality.
        \item Data exploration reveals data relationships.
        \item Model development utilizes insights for prediction.
    \end{itemize}
\end{frame}

\begin{frame}{Data Analysis Techniques}
    \frametitle{Overview}
    Data analysis techniques are essential in data mining as they provide mechanisms for extracting insights from complex datasets. These techniques can range from basic statistical methods to advanced machine learning algorithms, enabling effective interpretation, visualization, and management of data.
\end{frame}

\begin{frame}{Key Techniques in Data Analysis}
    \frametitle{Key Techniques}
    \begin{enumerate}
        \item \textbf{Descriptive Statistics}
        \item \textbf{Inferential Statistics}
        \item \textbf{Regression Analysis}
        \item \textbf{Clustering Techniques}
        \item \textbf{Data Visualization}
    \end{enumerate}
\end{frame}

\begin{frame}{Descriptive Statistics}
    \frametitle{Descriptive Statistics}
    \begin{itemize}
        \item \textbf{Purpose:} Summarize and describe data features.
        \item \textbf{Key Measures:}
        \begin{itemize}
            \item Mean: \(\bar{x} = \frac{1}{n} \sum_{i=1}^{n} x_i\)
            \item Median: Middle value when data is sorted.
            \item Standard Deviation (SD): 
            \begin{equation}
                SD = \sqrt{\frac{1}{n} \sum_{i=1}^{n} (x_i - \bar{x})^2} 
            \end{equation}
        \end{itemize}
        \item \textbf{Example:} Mean exam scores of students to gauge performance.
    \end{itemize}
\end{frame}

\begin{frame}{Inferential Statistics}
    \frametitle{Inferential Statistics}
    \begin{itemize}
        \item \textbf{Purpose:} Make predictions about a population from a sample.
        \item \textbf{Key Concepts:}
        \begin{itemize}
            \item Confidence Intervals: Represents uncertainty around a sample estimate, e.g., a 95\% CI for mean might be [70, 80].
            \item Hypothesis Testing: Uses p-values to test claims (typically \(p < 0.05\)).
        \end{itemize}
        \item \textbf{Example:} Testing a new teaching method's effect on test scores.
    \end{itemize}
\end{frame}

\begin{frame}{Regression Analysis}
    \frametitle{Regression Analysis}
    \begin{itemize}
        \item \textbf{Purpose:} Explore relationships between variables to make predictions.
        \item \textbf{Key Models:}
        \begin{itemize}
            \item Linear Regression: 
            \begin{equation}
                Y = \beta_0 + \beta_1X + \epsilon
            \end{equation}
            \item \textbf{Example:} Predicting sales based on advertising spend.
        \end{itemize}
    \end{itemize}
\end{frame}

\begin{frame}{Clustering Techniques}
    \frametitle{Clustering Techniques}
    \begin{itemize}
        \item \textbf{Purpose:} Group similar data points without prior labels.
        \item \textbf{Key Algorithms:}
        \begin{itemize}
            \item K-Means Clustering: 
            \begin{itemize}
                \item Initialize \(k\) centroids.
                \item Assign data points to nearest centroid.
                \item Recalculate centroids based on assignments.
            \end{itemize}
        \end{itemize}
        \item \textbf{Example:} Customer segmentation for targeted marketing.
    \end{itemize}
\end{frame}

\begin{frame}{Data Visualization}
    \frametitle{Data Visualization}
    \begin{itemize}
        \item \textbf{Purpose:} Present data in visually intuitive formats.
        \item \textbf{Common Tools:}
        \begin{itemize}
            \item Bar graphs for categorical data.
            \item Scatter plots for relationship representation.
            \item Heat maps for correlation matrices.
        \end{itemize}
        \item \textbf{Example:} Visualizing sales data to identify trends.
    \end{itemize}
\end{frame}

\begin{frame}{Conclusion}
    \frametitle{Conclusion}
    Using the right data analysis techniques is critical for informed decision-making and generating insights. Different methods can be applied based on the analysis objectives to extract valuable information effectively.
\end{frame}

\begin{frame}[fragile]{Code Snippet: Linear Regression in Python}
    \frametitle{Python Code Example}
    \begin{lstlisting}[language=Python]
import pandas as pd
import statsmodels.api as sm

# Sample Data
X = pd.DataFrame({'Advertising': [100, 200, 300, 400, 500]})
y = pd.DataFrame({'Sales': [10, 20, 30, 40, 50]})

# Adding a constant for intercept
X = sm.add_constant(X)

# Fitting the model
model = sm.OLS(y, X).fit()
predictions = model.predict(X)

# Summary of the regression
print(model.summary())
    \end{lstlisting}
\end{frame}

\begin{frame}
    \frametitle{Industry-Standard Tools}
    \begin{block}{Overview of Data Mining Tools}
        Data mining relies on various tools to process and extract insights from large datasets. We will explore three popular tools: 
        \begin{itemize}
            \item Python
            \item R
            \item Weka
        \end{itemize}
    \end{block}
\end{frame}

\begin{frame}[fragile]
    \frametitle{Python - Overview}
    \begin{block}{Description}
        Python is a versatile programming language known for its simplicity and readability. It has extensive libraries and frameworks for data mining.
    \end{block}
    
    \begin{block}{Key Libraries}
        \begin{itemize}
            \item \textbf{Pandas:} Data manipulation and analysis.
            \begin{lstlisting}[language=Python]
import pandas as pd
data = pd.read_csv('data.csv')
            \end{lstlisting}
            \item \textbf{NumPy:} Numerical computations.
            \item \textbf{Scikit-learn:} Machine learning library.
            \begin{lstlisting}[language=Python]
from sklearn.tree import DecisionTreeClassifier

X = [[0, 0], [1, 1]]  # feature matrix
y = [0, 1]  # target variable
clf = DecisionTreeClassifier()
clf.fit(X, y)
            \end{lstlisting}
        \end{itemize}
    \end{block}

    \begin{block}{Applications}
        \begin{itemize}
            \item Data preprocessing
            \item Building predictive models
            \item Visualizing data (Matplotlib, Seaborn)
        \end{itemize}
    \end{block}
\end{frame}

\begin{frame}[fragile]
    \frametitle{R and Weka - Overview}
    \begin{block}{R - Overview}
        R is specifically designed for statistical computing and data analysis. It excels in statistical modeling.
    \end{block}

    \begin{block}{Key Packages}
        \begin{itemize}
            \item \textbf{dplyr:} Data manipulation.
            \begin{lstlisting}[language=R]
library(dplyr)
filtered_data <- filter(data, value > 10)
            \end{lstlisting}
            \item \textbf{caret:} Building predictive models.
            \item \textbf{ggplot2:} Data visualization.
            \begin{lstlisting}[language=R]
ggplot(data, aes(x = variable1, y = variable2)) + geom_point()
            \end{lstlisting}
        \end{itemize}
    \end{block}

    \begin{block}{Applications}
        \begin{itemize}
            \item Statistical analysis
            \item Data visualization
            \item Reporting for research
        \end{itemize}
    \end{block}

    \begin{block}{Weka}
        Weka is an open-source suite for data mining with a GUI, allowing users to apply algorithms and visualize results without coding.
        \begin{itemize}
            \item Examples of classifiers: Random Forest, Naive Bayes.
            \item Ideal for educational and prototyping purposes.
        \end{itemize}
    \end{block}
\end{frame}

\begin{frame}
    \frametitle{Summary of Key Points}
    \begin{itemize}
        \item \textbf{Python:} Best for flexibility and extensive libraries.
        \item \textbf{R:} Ideal for statisticians focusing on statistical analysis.
        \item \textbf{Weka:} Excellent for beginners exploring machine learning without programming.
    \end{itemize}

    By understanding these tools, you can choose the right one for your data mining projects.
\end{frame}

\begin{frame}[fragile]
    \frametitle{Predictive and Classification Models - Overview}
    Predictive and classification models are essential in data mining, allowing us to predict outcomes and categorize data based on input features. 
    In this slide, we will explore key model development techniques: 
    \begin{itemize}
        \item \textbf{Decision Trees}
        \item \textbf{Neural Networks}
        \item \textbf{Clustering Algorithms}
    \end{itemize}
\end{frame}

\begin{frame}[fragile]
    \frametitle{Predictive and Classification Models - Decision Trees}
    \textbf{Definition:} A decision tree is a flowchart-like structure where each internal node represents a feature, each branch a decision rule, and each leaf node an outcome.

    \textbf{How it Works:} 
    \begin{itemize}
        \item Decision trees split data into subsets based on input feature values using rules (e.g., “If age > 30, then…”).
        \item The process is recursive, aiming to maximize information gain and minimize impurity in resultant nodes.
    \end{itemize}
    
    \textbf{Example:} Predicting whether a person will purchase insurance based on features like age and income.
    
    \textbf{Advantages:}
    \begin{itemize}
        \item Easy to understand and interpret
        \item Requires little data preparation
        \item Handles both numerical and categorical data
    \end{itemize}
\end{frame}

\begin{frame}[fragile]
    \frametitle{Predictive and Classification Models - Neural Networks}
    \textbf{Definition:} Neural networks are computational models inspired by the human brain, consisting of interconnected layers of nodes (neurons).

    \textbf{How it Works:} 
    \begin{itemize}
        \item Input is processed through layers of neurons using activation functions to produce output.
        \item Weights associated with connections adjust through training to minimize prediction error.
    \end{itemize}

    \textbf{Application:} Used in image recognition, natural language processing, and forecasting.
    
    \textbf{Example:}
    \begin{itemize}
        \item Input Layer: Features (e.g., pixel values)
        \item Hidden Layer: Processes features
        \item Output Layer: Predictions (e.g., classifying an image)
    \end{itemize}
    
    \textbf{Advantages:}
    \begin{itemize}
        \item Captures complex patterns effectively
        \item Scalable to large datasets
    \end{itemize}
\end{frame}

\begin{frame}[fragile]
    \frametitle{Predictive and Classification Models - Clustering Algorithms}
    \textbf{Definition:} Clustering is the task of dividing a dataset into groups so that data points in the same group are more similar to one another.

    \textbf{How it Works:} 
    \begin{itemize}
        \item Common algorithms: \textbf{K-means}, \textbf{Hierarchical Clustering}, and \textbf{DBSCAN}.
        \item K-means partitions data into K clusters, iterating until centroids stabilize.
    \end{itemize}

    \textbf{Example:} Segmenting customers based on purchasing behavior for targeted marketing.

    \textbf{Advantages:}
    \begin{itemize}
        \item Helps discover underlying data patterns
        \item Useful for exploratory data analysis
    \end{itemize}
\end{frame}

\begin{frame}[fragile]
    \frametitle{Key Points and Code Snippets}
    \textbf{Key Points to Emphasize:}
    \begin{itemize}
        \item Model selection depends on data characteristics and desired outcomes.
        \item Understanding model strengths and weaknesses helps tailor solutions.
        \item Often, a combination of models increases predictive accuracy.
    \end{itemize}

    \textbf{Decision Tree Example (Pseudocode):}
    \begin{lstlisting}
function buildTree(data):
    if isPure(data):
        return leafNode(data)
    bestFeature = findBestFeature(data)
    tree = createNode(bestFeature)
    for each value in bestFeature:
        subset = split(data, bestFeature, value)
        tree.addBranch(value, buildTree(subset))
    return tree
    \end{lstlisting}

    \textbf{K-means Algorithm (Pseudocode):}
    \begin{lstlisting}
repeat:
    for each data point:
        assign to nearest centroid
    update centroids by averaging assigned points
until centroids do not change
    \end{lstlisting}
\end{frame}

\begin{frame}[fragile]
    \frametitle{Model Evaluation Metrics - Introduction}
    In data mining, evaluating the performance of predictive and classification models is crucial to ensure they make accurate predictions based on learned patterns. Common evaluation metrics include:
    \begin{itemize}
        \item \textbf{Accuracy}
        \item \textbf{Precision}
        \item \textbf{Recall}
    \end{itemize}
    Understanding these metrics will help us choose the best model for our task.
\end{frame}

\begin{frame}[fragile]
    \frametitle{Model Evaluation Metrics - Accuracy}
    \begin{block}{Accuracy}
        \textbf{Definition}: The proportion of correctly predicted instances (both positive and negative) out of the total instances.
    \end{block}
    \begin{equation}
        \text{Accuracy} = \frac{TP + TN}{TP + TN + FP + FN}
    \end{equation}
    \begin{itemize}
        \item \(TP\): True Positives
        \item \(TN\): True Negatives
        \item \(FP\): False Positives
        \item \(FN\): False Negatives
    \end{itemize}
    \textbf{Example}: In a model predicting whether a patient has a disease, 70 correct predictions (true positives + true negatives) out of 100 total predictions yields an accuracy of $70\%$.
\end{frame}

\begin{frame}[fragile]
    \frametitle{Model Evaluation Metrics - Precision and Recall}
    \begin{block}{Precision}
        \textbf{Definition}: The proportion of true positive predictions among all positive predictions made by the model.
    \end{block}
    \begin{equation}
        \text{Precision} = \frac{TP}{TP + FP}
    \end{equation}
    \textbf{Example}: If a model predicts 30 patients as having the disease, but only 20 truly have it, the precision is $\frac{20}{30} = \frac{2}{3} \approx 67\%$.
    
    \begin{block}{Recall (Sensitivity)}
        \textbf{Definition}: The proportion of true positive predictions made out of all actual positive instances.
    \end{block}
    \begin{equation}
        \text{Recall} = \frac{TP}{TP + FN}
    \end{equation}
    \textbf{Example}: If there are 25 actual positive cases and the model identifies 20 of them, the recall is $\frac{20}{25} = 0.8 = 80\%$.
    
    \textbf{Importance}: High recall is vital in medical diagnoses where missing a true positive could have severe consequences.
\end{frame}

\begin{frame}[fragile]
    \frametitle{Model Evaluation Metrics - Confusion Matrix}
    The \textbf{Confusion Matrix} is a quick visual tool to evaluate models based on their predictions:

    \begin{center}
    \begin{tabular}{|c|c|c|}
        \hline
        & \textbf{Predicted Positive} & \textbf{Predicted Negative} \\
        \hline
        \textbf{Actual Positive} & TP & FN \\
        \hline
        \textbf{Actual Negative} & FP & TN \\
        \hline
    \end{tabular}
    \end{center}

    From this matrix, we can derive all our metrics, illustrating how each contributes to our understanding of model performance.
\end{frame}

\begin{frame}[fragile]
    \frametitle{Model Evaluation Metrics - Key Takeaways}
    \begin{itemize}
        \item \textbf{Accuracy} gives a general idea of performance but can be misleading, especially in imbalanced datasets.
        \item \textbf{Precision} and \textbf{Recall} provide deeper insights, especially relevant when the costs of false positives and negatives are significant.
        \item Often, a balance between Precision and Recall is sought, especially using metrics like \textbf{F1 Score}, which combines both.
    \end{itemize}
    Understanding these metrics will enhance our ability to evaluate and select the most appropriate model based on the specific requirements of the task at hand.
\end{frame}

\begin{frame}[fragile]
    \frametitle{Ethical Considerations in Data Mining}
    \begin{block}{Overview}
        Exploration of the ethical implications of data mining and data governance frameworks that inform responsible practices.
    \end{block}
\end{frame}

\begin{frame}[fragile]
    \frametitle{Understanding Ethical Implications in Data Mining}
    Data mining involves extracting valuable insights from vast datasets, but it also raises critical ethical concerns. Key considerations:
    \begin{enumerate}
        \item \textbf{Data Privacy}: Individuals have the right to control their personal information. Sensitive data must be anonymized, and users must be informed about data usage.
        \item \textbf{Informed Consent}: Transparency about data collection and usage is crucial; stakeholders should clearly understand what they are consenting to.
        \item \textbf{Bias and Fairness}: Machine learning might reflect societal biases, necessitating ongoing evaluation to mitigate discriminatory outcomes.
        \item \textbf{Accountability and Responsibility}: Organizations must take responsibility for data mining consequences, guided by governance frameworks.
    \end{enumerate}
\end{frame}

\begin{frame}[fragile]
    \frametitle{Key Examples and Illustrations}
    \begin{itemize}
        \item \textbf{Data Privacy Example}:
          In 2020, a company faced backlash after mishandling customer data, leading to reputational damage.
          
        \item \textbf{Informed Consent Illustration}:
          A mobile app tracking user location must clearly state this in its terms of service, allowing for informed user choices.
          
        \item \textbf{Bias and Fairness Example}:
          Recruitment algorithms trained on historical data may unintentionally favor specific demographics, resulting in discrimination.
          
        \item \textbf{Accountability Key Point}:
          Implementing an ethical review board can ensure scrutiny of decisions involving data mining.
    \end{itemize}
\end{frame}

\begin{frame}[fragile]
    \frametitle{Data Governance Frameworks}
    \begin{block}{Definition}
        Data governance involves processes, roles, policies, and metrics ensuring ethical data management.
    \end{block}
    Essential components include:
    \begin{itemize}
        \item \textbf{Policies}: Clear guidelines on data usage, retention, and security.
        \item \textbf{Compliance Programs}: Adhering to frameworks like GDPR or HIPAA helps prevent data misuse.
        \item \textbf{Training and Awareness}: Regular ethics training for data practitioners fosters a culture of responsibility.
    \end{itemize}
\end{frame}

\begin{frame}[fragile]
    \frametitle{Conclusion}
    As data mining techniques evolve, integrating strong ethical standards is crucial. 
    \begin{itemize}
        \item Ethical data mining protects individuals while building trust in organizations.
        \item A robust data governance framework allows the responsible unlocking of data's power while upholding rights and societal values.
    \end{itemize}
    \textbf{Key Takeaway:} Ethical considerations are fundamental to sustaining public trust and ensuring equitable outcomes in our data-driven world.
\end{frame}

\begin{frame}[fragile]
    \frametitle{Data Privacy Laws}
    \begin{block}{Overview}
        Data privacy laws govern the collection, storage, and sharing of personally identifiable information (PII) by organizations, particularly as data mining practices expand.
    \end{block}
\end{frame}

\begin{frame}[fragile]
    \frametitle{Key Concepts}
    \begin{enumerate}
        \item \textbf{Personally Identifiable Information (PII)} 
        \begin{itemize}
            \item Any data that can identify an individual, e.g., names, addresses.
            \item \textit{Example}: Targeting ads based on purchase data may infringe on privacy if mishandled.
        \end{itemize}
        
        \item \textbf{Data Protection Regulation}
        \begin{itemize}
            \item Legal frameworks designed to protect individual privacy rights and guide proper data usage.
        \end{itemize}
    \end{enumerate}
\end{frame}

\begin{frame}[fragile]
    \frametitle{Major Data Privacy Laws}
    \begin{enumerate}
        \item \textbf{GDPR (General Data Protection Regulation) - EU}
        \begin{itemize}
            \item Enforced since May 2018, applies to organizations handling EU citizen data.
            \item \textit{Key Principles:}
            \begin{itemize}
                \item \textbf{Consent}: Explicit consent required for data processing.
                \item \textbf{Right to Access}: Individuals can request their data.
                \item \textbf{Right to Erasure}: Individuals can request deletion of their data.
            \end{itemize}
        \end{itemize}
        
        \item \textbf{CCPA (California Consumer Privacy Act) - USA}
        \begin{itemize}
            \item Effective from January 2020, enhances control over personal information for Californians.
            \item \textit{Key Features:}
            \begin{itemize}
                \item \textbf{Right to Know}: Consumers can request how their data is used.
                \item \textbf{Opt-out}: Individuals can opt-out of personal data sales.
            \end{itemize}
        \end{itemize}
        
        \item \textbf{HIPAA (Health Insurance Portability and Accountability Act) - USA}
        \begin{itemize}
            \item Protects sensitive patient health information.
            \item \textit{Key Requirement}: Healthcare providers must secure patient data.
        \end{itemize}
    \end{enumerate}
\end{frame}

\begin{frame}[fragile]
    \frametitle{Impact on Data Handling \& Processing}
    \begin{itemize}
        \item \textbf{Compliance Necessities}: Organizations must adopt data governance policies to avoid fines.
        \item \textbf{Anonymization \& De-identification}: Data mining often requires data to be anonymized or de-identified.
    \end{itemize}
\end{frame}

\begin{frame}[fragile]
    \frametitle{Example Application}
    A retail company utilizes customer transaction data for market basket analysis:
    \begin{itemize}
        \item Under GDPR, prior user consent is mandatory for analyzing transactions.
        \item Customers must have opt-out options at any time.
    \end{itemize}
\end{frame}

\begin{frame}[fragile]
    \frametitle{Key Points to Emphasize}
    \begin{itemize}
        \item Understanding data privacy laws is essential for ethical data mining.
        \item Compliance protects users and enhances organizational trust.
        \item Data mining can be performed responsibly using anonymized data.
    \end{itemize}
\end{frame}

\begin{frame}[fragile]
    \frametitle{Conclusion}
    Data privacy laws heavily influence data mining practices. Awareness and adherence to these laws are crucial for responsible and ethical data analysis.
\end{frame}

\begin{frame}{Collaborative Data Mining Projects}
    Insights on conducting collaborative projects in data mining, from problem definition to model deployment.
\end{frame}

\begin{frame}{Introduction}
    \begin{itemize}
        \item Collaborative data mining involves multiple stakeholders working together to extract meaningful patterns from large datasets.
        \item Essential for tackling complex problems requiring diverse expertise, resources, and perspectives.
    \end{itemize}
\end{frame}

\begin{frame}{Phases of Collaborative Data Mining Projects}
    \begin{enumerate}
        \item \textbf{Problem Definition}
            \begin{itemize}
                \item Define project goals and desired outcomes.
                \item Example: Forecast customer churn; specify metrics like churn rate.
            \end{itemize}
        
        \item \textbf{Data Collection and Preparation}
            \begin{itemize}
                \item Gather relevant datasets; clean data.
                \item Identify sources, handle missing values.
            \end{itemize}
    \end{enumerate}
\end{frame}

\begin{frame}[fragile]{Data Collection and Preparation - Example Code}
    \begin{lstlisting}[language=Python]
import pandas as pd

# Load dataset
data = pd.read_csv('customer_data.csv')

# Handling missing values
data.fillna(method='ffill', inplace=True)
    \end{lstlisting}
\end{frame}

\begin{frame}{Phases of Collaborative Data Mining Projects (Continued)}
    \begin{enumerate}
        \setcounter{enumi}{2}
        \item \textbf{Data Exploration and Feature Selection}
            \begin{itemize}
                \item Analyze data structure and identify key features.
                \item Techniques: statistics, visualization (e.g., histograms).
            \end{itemize}

        \item \textbf{Model Building}
            \begin{itemize}
                \item Implement best predictive algorithms.
                \item Common algorithms: Decision Trees, Random Forest, SVM.
            \end{itemize}
    \end{enumerate}
\end{frame}

\begin{frame}[fragile]{Model Building - Example Code}
    \begin{lstlisting}[language=Python]
from sklearn.ensemble import RandomForestClassifier

model = RandomForestClassifier(n_estimators=100)
model.fit(X_train, y_train)
    \end{lstlisting}
\end{frame}

\begin{frame}{Phases of Collaborative Data Mining Projects (Continued)}
    \begin{enumerate}
        \setcounter{enumi}{4}
        \item \textbf{Model Evaluation}
            \begin{itemize}
                \item Assess performance using accuracy, precision, recall, F1 score.
                \item Example: F1 Score is crucial for imbalanced datasets.
            \end{itemize}

        \item \textbf{Model Deployment}
            \begin{itemize}
                \item Integrate model into production for real-time predictions.
                \item Monitor performance; update as necessary.
                \item Example: Deploy as REST API.
            \end{itemize}
    \end{enumerate}
\end{frame}

\begin{frame}{Key Points to Emphasize}
    \begin{itemize}
        \item \textbf{Collaboration is Key}: Diverse skills and perspectives enhance problem-solving.
        \item \textbf{Iterative Process}: Revisiting phases can lead to new insights.
        \item \textbf{Data Governance}: Ensure compliance with data privacy and ethical standards.
    \end{itemize}
\end{frame}

\begin{frame}{Conclusion}
    Collaborative data mining projects combine technical expertise with teamwork to deliver actionable insights. Through well-defined phases, teams can manage complexities effectively and drive impactful results.
\end{frame}

\begin{frame}{Formula to Remember}
    \begin{equation}
        F1 = 2 \times \frac{\text{Precision} \times \text{Recall}}{\text{Precision} + \text{Recall}}
    \end{equation}
\end{frame}

\begin{frame}[fragile]{Communication and Reporting Skills}
    Importance of effectively communicating technical findings to diverse audiences and preparing structured reports.
\end{frame}

\begin{frame}[fragile]{Importance of Effective Communication in Data Mining}
    Effective communication and reporting of technical findings are crucial in the field of data mining due to the diverse audiences involved. Key points include:
    \begin{itemize}
        \item \textbf{Diverse Audience}: Stakeholders may have varying levels of technical expertise.
        \item \textbf{Avoiding Misinterpretation}: Reduce the chance of confusion by using clear language.
    \end{itemize}
\end{frame}

\begin{frame}[fragile]{Structured Reporting and Visual Aids}
    Structured reports allow for an organized presentation of data analysis, findings, and recommendations. A typical report structure includes:
    \begin{enumerate}
        \item \textbf{Introduction}: Overview of the problem and objectives.
        \item \textbf{Methodology}: Description of data sources and analytical techniques used.
        \item \textbf{Findings}: Key insights derived from the analysis.
        \item \textbf{Conclusion and Recommendations}: Summary and actionable items based on findings.
    \end{enumerate}
    Incorporating visual aids (charts, graphs, tables) can enhance comprehension of trends and findings. 
\end{frame}

\begin{frame}[fragile]{Key Points to Emphasize}
    Remember to focus on these aspects during your presentations:
    \begin{itemize}
        \item \textbf{Audience Engagement}: Tailor your communication style to suit your audience.
        \item \textbf{Clarity and Brevity}: Keep communication concise and focus on key findings.
        \item \textbf{Active Listening}: Encourage questions to clarify misunderstandings and build rapport.
    \end{itemize}
\end{frame}

\begin{frame}[fragile]{Conclusion}
    Mastering communication and reporting skills is vital for sharing technical findings in data mining projects. By emphasizing clarity, structure, and audience engagement, you can ensure that your insights lead to informed decision-making and drive meaningful action.
\end{frame}

\begin{frame}[fragile]{Essential Math Skills for Data Mining - Overview}
    Data mining is a field that relies heavily on mathematical foundations. 
    Practitioners must be adept in key areas of mathematics, particularly \textbf{Statistics} and \textbf{Linear Algebra}. 
    This presentation will explore fundamental concepts in these areas and their relevance to data mining.
\end{frame}

\begin{frame}[fragile]{Essential Math Skills for Data Mining - Statistics}
    \textbf{Statistics} is crucial for making sense of data. Here are the essential statistical concepts:

    \begin{itemize}
        \item \textbf{Descriptive Statistics}: Tools to summarize and describe the main features of a dataset.
        \item \textbf{Probability Distributions}:
            \begin{itemize}
                \item \textbf{Normal Distribution}: Bell-shaped curve; many statistical tests assume normality.
            \end{itemize}
        \item \textbf{Inferential Statistics}: Techniques to draw conclusions about a population based on a sample.
    \end{itemize}
\end{frame}

\begin{frame}[fragile]{Essential Math Skills for Data Mining - Key Concepts in Statistics}
    \textbf{Hypothesis Testing}: Framework for making inferences (e.g., t-tests, chi-square tests).

    \textbf{Example}: Testing if a new product affects sales compared to an old one.

    \begin{block}{Key Formula: Z-score}
        \begin{equation}
            Z = \frac{(X - \mu)}{\sigma}
        \end{equation}
        Where \(X\) is a value, \(\mu\) is the mean, and \(\sigma\) is the standard deviation. This helps analyze and compare data points relative to the overall dataset.
    \end{block}
\end{frame}

\begin{frame}[fragile]{Essential Math Skills for Data Mining - Linear Algebra}
    \textbf{Linear Algebra} provides the structure for data representation and manipulation in multidimensional space.

    \begin{itemize}
        \item \textbf{Vectors and Matrices}: Fundamental structures for organizing data.
        \item \textbf{Matrix Operations}:
            \begin{itemize}
                \item Addition, subtraction, scalar multiplication, and matrix multiplication.
            \end{itemize}
        \item \textbf{Eigenvalues and Eigenvectors}: Crucial for understanding dimensionality reduction techniques like PCA (Principal Component Analysis).
    \end{itemize}
\end{frame}

\begin{frame}[fragile]{Essential Math Skills for Data Mining - Key Formulas}
    \textbf{Example}: A dataset with multiple features can be represented as a matrix.

    \begin{block}{Key Formula: Dot Product of Vectors}
        \begin{equation}
            \mathbf{a} \cdot \mathbf{b} = \sum_{i=1}^{n} a_i \cdot b_i
        \end{equation}
    \end{block}

    Indicates the degree of similarity between two data points.
\end{frame}

\begin{frame}[fragile]{Essential Math Skills for Data Mining - Importance}
    Mastering these mathematical concepts equips you to:

    \begin{itemize}
        \item Analyze Data Efficiently: Draw actionable insights with robust statistical methods.
        \item Develop Algorithms: Use linear algebra for building and optimizing machine learning models.
        \item Communicate Results: Present findings in a statistically sound and understandable manner.
    \end{itemize}
    
    \textbf{Remember:} A solid foundation in statistics and linear algebra is essential for data mining.
\end{frame}

\begin{frame}[fragile]
    \frametitle{Assessment and Evaluation Methods - Overview}
    \begin{block}{Introduction}
        Assessment and evaluation in data mining courses are essential for measuring student understanding, skills development, and the overall effectiveness of instructional strategies. A diverse range of assessment methods can provide insights into student learning outcomes and support continuous improvement.
    \end{block}
\end{frame}

\begin{frame}[fragile]
    \frametitle{Assessment and Evaluation Methods - Types of Assessment}
    \begin{itemize}
        \item \textbf{Formative Assessments}
            \begin{itemize}
                \item \textbf{Definition:} Ongoing assessments to monitor student learning and provide immediate feedback.
                \item \textbf{Examples:}
                    \begin{itemize}
                        \item Quizzes on key concepts (e.g., data preprocessing techniques)
                        \item Class participation in discussions on case studies
                        \item Peer reviews of student work
                    \end{itemize}
            \end{itemize}
        
        \item \textbf{Summative Assessments}
            \begin{itemize}
                \item \textbf{Definition:} Evaluations at the end of an instructional unit to assess cumulative learning.
                \item \textbf{Examples:}
                    \begin{itemize}
                        \item Comprehensive exams covering all topics
                        \item Final projects applying data mining techniques to real-world data
                    \end{itemize}
            \end{itemize}

        \item \textbf{Practical Assessments}
            \begin{itemize}
                \item \textbf{Definition:} Hands-on evaluations assessing the application of data mining tools and techniques.
                \item \textbf{Examples:}
                    \begin{itemize}
                        \item Coding assignments (e.g., K-means clustering)
                        \item Data analysis projects with specified datasets
                    \end{itemize}
            \end{itemize}
    \end{itemize}
\end{frame}

\begin{frame}[fragile]
    \frametitle{Assessment and Evaluation Methods - Evaluation Criteria}
    \begin{itemize}
        \item \textbf{Understanding of Concepts:} 
            Ability to explain underlying principles of data mining methods.
        \item \textbf{Application of Techniques:} 
            Proficiency in applying algorithms and tools effectively.
        \item \textbf{Analytical Skills:} 
            Capability to interpret results and make data-driven decisions.
        \item \textbf{Communication:} 
            Clarity and effectiveness in presenting findings, both in written and oral forms.
    \end{itemize}

    \begin{block}{Tools and Platforms for Assessment}
        \begin{itemize}
            \item Learning Management Systems (LMS), such as Moodle or Canvas
            \item Version Control (Git) for programming assignments
            \item Data Visualization Tools (e.g., Tableau, Python libraries)
        \end{itemize}
    \end{block}
\end{frame}

\begin{frame}[fragile]
    \frametitle{Assessment and Evaluation Methods - Key Points}
    \begin{itemize}
        \item A combination of assessment methods enhances learning and keeps students engaged.
        \item Real-world applicability of data mining techniques is crucial for student motivation.
        \item Providing constructive feedback is essential for students to improve and understand their learning paths.
    \end{itemize}
    
    \begin{block}{Summary}
        By utilizing a variety of assessment methods, educators can ensure that students not only grasp theoretical concepts but are also equipped with practical skills necessary for a successful career in data mining.
    \end{block}
\end{frame}

\begin{frame}[fragile]{Weekly Schedule Overview - Part 1}
  \frametitle{Overview}
  This week's schedule delineates our journey through the fascinating world of Data Mining. Each week, we will delve into vital topics, supported by:
  \begin{itemize}
    \item Key readings
    \item Practical assignments
    \item Structured evaluations
  \end{itemize}
  All aimed at enhancing your understanding of data mining concepts and techniques.
\end{frame}

\begin{frame}[fragile]{Weekly Schedule Overview - Part 2}
  \frametitle{Weekly Breakdown}
  
  \textbf{Week 1: Introduction to Data Mining}
  \begin{itemize}
    \item \textbf{Topics:}
    \begin{itemize}
      \item Definition and Scope of Data Mining
      \item Importance in Industry and Research
    \end{itemize}
    \item \textbf{Readings:} 
    \begin{itemize}
      \item "Data Mining: Concepts and Techniques" by Han et al., Chapters 1 \& 2
    \end{itemize}
    \item \textbf{Assignments:} 
    Discuss a case study where data mining transformed a business strategy.
    \item \textbf{Evaluation:} Participation in class discussion (10\% of final grade).
  \end{itemize}

  \textbf{Week 2: Data Preprocessing and Exploration} 
  \begin{itemize}
    \item \textbf{Topics:}
    \begin{itemize}
      \item Data Cleaning, Integration, and Transformation
      \item Exploratory Data Analysis (EDA) Techniques
    \end{itemize}
    \item \textbf{Readings:} 
    \begin{itemize}
      \item "An Introduction to Data Mining" by Tan et al., Chapter 3
    \end{itemize}
    \item \textbf{Assignments:} Submit a small dataset for preprocessing steps.
    \item \textbf{Evaluation:} Homework Quiz (15\% of final grade).
  \end{itemize}  
\end{frame}

\begin{frame}[fragile]{Weekly Schedule Overview - Part 3}
  \frametitle{Weekly Breakdown Continued}
  
  \textbf{Week 3: Classification Techniques}
  \begin{itemize}
    \item \textbf{Topics:} Overview of classification algorithms (e.g., Decision Trees, SVM, Random Forest).
    \item \textbf{Readings:} Online Articles on Classification Fundamentals.
    \item \textbf{Assignments:} Implement a Decision Tree using Python on a provided dataset.
    \item \textbf{Evaluation:} Code Review and Quality (20\% of final grade).
  \end{itemize}

  \textbf{Week 4: Clustering Methods}
  \begin{itemize}
    \item \textbf{Topics:} Types of Clustering: K-Means, Hierarchical Clustering.
    \item \textbf{Readings:} "Data Mining Methodologies" by Yao, Chapters 5 \& 6.
    \item \textbf{Assignments:} Perform K-Means clustering on real-world data and report findings.
    \item \textbf{Evaluation:} Group Presentation (20\% of final grade).
  \end{itemize}
  
\end{frame}


\end{document}