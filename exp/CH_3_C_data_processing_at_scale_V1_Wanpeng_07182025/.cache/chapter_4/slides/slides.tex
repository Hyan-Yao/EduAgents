\documentclass[aspectratio=169]{beamer}

% Theme and Color Setup
\usetheme{Madrid}
\usecolortheme{whale}
\useinnertheme{rectangles}
\useoutertheme{miniframes}

% Additional Packages
\usepackage[utf8]{inputenc}
\usepackage[T1]{fontenc}
\usepackage{graphicx}
\usepackage{booktabs}
\usepackage{listings}
\usepackage{amsmath}
\usepackage{amssymb}
\usepackage{xcolor}
\usepackage{tikz}
\usepackage{pgfplots}
\pgfplotsset{compat=1.18}
\usetikzlibrary{positioning}
\usepackage{hyperref}

% Custom Colors
\definecolor{myblue}{RGB}{31, 73, 125}
\definecolor{mygray}{RGB}{100, 100, 100}
\definecolor{mygreen}{RGB}{0, 128, 0}
\definecolor{myorange}{RGB}{230, 126, 34}
\definecolor{mycodebackground}{RGB}{245, 245, 245}

% Set Theme Colors
\setbeamercolor{structure}{fg=myblue}
\setbeamercolor{frametitle}{fg=white, bg=myblue}
\setbeamercolor{title}{fg=myblue}
\setbeamercolor{section in toc}{fg=myblue}
\setbeamercolor{item projected}{fg=white, bg=myblue}
\setbeamercolor{block title}{bg=myblue!20, fg=myblue}
\setbeamercolor{block body}{bg=myblue!10}
\setbeamercolor{alerted text}{fg=myorange}

% Set Fonts
\setbeamerfont{title}{size=\Large, series=\bfseries}
\setbeamerfont{frametitle}{size=\large, series=\bfseries}
\setbeamerfont{caption}{size=\small}
\setbeamerfont{footnote}{size=\tiny}

% Custom Commands
\newcommand{\hilight}[1]{\colorbox{myorange!30}{#1}}
\newcommand{\concept}[1]{\textcolor{myblue}{\textbf{#1}}}

% Footer and Navigation Setup
\setbeamertemplate{footline}{
  \leavevmode%
  \hbox{%
  \begin{beamercolorbox}[wd=.3\paperwidth,ht=2.25ex,dp=1ex,center]{author in head/foot}%
    \usebeamerfont{author in head/foot}\insertshortauthor
  \end{beamercolorbox}%
  \begin{beamercolorbox}[wd=.5\paperwidth,ht=2.25ex,dp=1ex,center]{title in head/foot}%
    \usebeamerfont{title in head/foot}\insertshorttitle
  \end{beamercolorbox}%
  \begin{beamercolorbox}[wd=.2\paperwidth,ht=2.25ex,dp=1ex,center]{date in head/foot}%
    \usebeamerfont{date in head/foot}
    \insertframenumber{} / \inserttotalframenumber
  \end{beamercolorbox}}%
  \vskip0pt%
}

% Turn off navigation symbols
\setbeamertemplate{navigation symbols}{}

% Title Page Information
\title[Week 4: Data Storage Solutions]{Week 4: Data Storage Solutions}
\author{Your Name}
\institute{Your Institution}
\date{\today}

% Document Start
\begin{document}

\frame{\titlepage}

\begin{frame}[fragile]
    \frametitle{Introduction to Data Storage Solutions}
    Data storage solutions are essential in data management, providing methods for storing, retrieving, and analyzing data efficiently and securely. Understanding these solutions is crucial for maximizing data's potential in the data-driven world.
\end{frame}

\begin{frame}[fragile]
    \frametitle{Importance in Data Management}
    \begin{itemize}
        \item \textbf{Data Availability:} Ensures accessibility for timely decision-making.
        \item \textbf{Data Integrity:} Protects against unauthorized access and corruption.
        \item \textbf{Scalability:} Facilitates growth through easy expansion of data capacity.
        \item \textbf{Data Backup and Recovery:} Safeguards data against loss through regular backups.
    \end{itemize}
\end{frame}

\begin{frame}[fragile]
    \frametitle{Key Concepts of Data Storage}
    \begin{enumerate}
        \item \textbf{Data Storage Types:}
            \begin{itemize}
                \item \textbf{Primary Storage:} Temporary storage (e.g., RAM) for immediate CPU access.
                \item \textbf{Secondary Storage:} Permanent solutions (e.g., HDDs, SSDs) for long-term data.
            \end{itemize}
        \item \textbf{Storage Technologies:}
            \begin{itemize}
                \item \textbf{File Storage:} Data stored in files, accessed via file systems.
                \item \textbf{Block Storage:} Data divided into blocks, popular in cloud services.
                \item \textbf{Object Storage:} Data stored as objects, accessed via APIs.
            \end{itemize}
    \end{enumerate}
\end{frame}

\begin{frame}[fragile]
    \frametitle{Examples of Data Storage Solutions}
    \begin{itemize}
        \item \textbf{Traditional Storage Solutions:} 
            \begin{itemize}
                \item Hard Disk Drives (HDDs)
                \item Solid-State Drives (SSDs)
            \end{itemize}
        \item \textbf{Cloud Storage Solutions:}
            \begin{itemize}
                \item Google Drive
                \item Amazon S3
                \item Microsoft Azure
            \end{itemize}
    \end{itemize}
\end{frame}

\begin{frame}[fragile]
    \frametitle{Key Points to Emphasize}
    \begin{itemize}
        \item Choice of storage solution depends on performance, cost, and data type requirements.
        \item Importance of selecting suitable solutions is critical as data volumes grow.
    \end{itemize}
\end{frame}

\begin{frame}[fragile]
    \frametitle{Illustrative Example}
    Consider a company processing large datasets for business analytics. By leveraging cloud-based object storage, they can efficiently store and access data globally, facilitating real-time insights and decision-making without physical storage limitations.
\end{frame}

\begin{frame}[fragile]
    \frametitle{Summary}
    Data storage solutions are pivotal in effective data management strategies, offering tailored options for organizational needs. Understanding these solutions optimizes data handling, ensuring data is available, secure, and scalable for growth.
\end{frame}

\begin{frame}[fragile]
    \frametitle{Types of Data Storage - Introduction}
    \begin{block}{Overview}
        Data storage solutions can be classified into two main categories: 
        \textbf{SQL (Structured Query Language)} and \textbf{NoSQL (Not Only SQL)}.
        Understanding the differences between these two types is crucial for choosing the right database solution for your applications.
    \end{block}
\end{frame}

\begin{frame}[fragile]
    \frametitle{Types of Data Storage - SQL Databases}
    \begin{itemize}
        \item \textbf{Definition}: 
            SQL databases are structured and use defined schemas to organize data, 
            maintaining data integrity and enabling complex queries.
        
        \item \textbf{Key Characteristics}:
            \begin{itemize}
                \item \textbf{Structured Data}: Data stored in tabular form (rows and columns).
                \item \textbf{Schema}: Requires a predefined schema that defines the structure of the data.
                \item \textbf{ACID Compliance}: Adheres to ACID (Atomicity, Consistency, Isolation, Durability) properties.
            \end{itemize}
        
        \item \textbf{Examples}:
            \begin{itemize}
                \item \textbf{MySQL}: Widely used for web applications.
                \item \textbf{PostgreSQL}: Known for advanced features and standards compliance.
            \end{itemize}
        
        \item \textbf{Usage}: 
            Best suited for applications requiring complex queries, such as financial systems and enterprise applications.
    \end{itemize}
    
    \begin{block}{SQL Query Example}
        \begin{lstlisting}[language=SQL]
SELECT * FROM Customers WHERE Country = 'USA';
        \end{lstlisting}
    \end{block}
\end{frame}

\begin{frame}[fragile]
    \frametitle{Types of Data Storage - NoSQL Databases}
    \begin{itemize}
        \item \textbf{Definition}: 
            NoSQL databases are designed for unstructured data storage and can handle a variety of data formats (e.g., document, graph, key-value).
        
        \item \textbf{Key Characteristics}:
            \begin{itemize}
                \item \textbf{Flexible Schema}: Allows for dynamic schemas; data stored without a predefined structure.
                \item \textbf{Scalability}: Can scale horizontally, making it suitable for large-scale applications.
                \item \textbf{Eventual Consistency}: Offers eventual consistency rather than immediate consistency.
            \end{itemize}
        
        \item \textbf{Examples}:
            \begin{itemize}
                \item \textbf{MongoDB}: A document store handling JSON-like documents.
                \item \textbf{Cassandra}: A wide-column store known for high availability and scalability.
            \end{itemize}
        
        \item \textbf{Usage}: 
            Ideal for applications like social networks, real-time data analytics, and content management systems.
    \end{itemize}
    
    \begin{block}{NoSQL Document Example}
        \begin{lstlisting}[language=json]
{
    "name": "John Doe",
    "age": 30,
    "country": "USA"
}
        \end{lstlisting}
    \end{block}
\end{frame}

\begin{frame}[fragile]
    \frametitle{Understanding SQL Databases - Introduction}
    \begin{block}{Definition}
        SQL (Structured Query Language) databases are relational databases that store data in structured formats using tables. Each table consists of rows and columns, where rows represent records and columns represent attributes of the data.
    \end{block}
\end{frame}

\begin{frame}[fragile]
    \frametitle{Understanding SQL Databases - Key Components}
    \begin{itemize}
        \item \textbf{Tables:} The basic unit of data storage, where data is organized into rows (records) and columns (fields).
        \item \textbf{Schema:} The structure or blueprint of the database, defining tables, fields, data types, and relationships between tables.
        \item \textbf{Primary Key:} A unique identifier for each record in a table, ensuring that each entry can be distinguished from others.
        \item \textbf{Foreign Key:} A field that creates a link between two tables, referring back to the primary key of another table, establishing relationships.
    \end{itemize}
\end{frame}

\begin{frame}[fragile]
    \frametitle{Understanding SQL Databases - Functionality}
    \begin{block}{Data Manipulation}
        SQL is the standard programming language for interacting with relational databases, allowing users to perform tasks such as:
        \begin{itemize}
            \item \texttt{SELECT:} Retrieve data from one or more tables.
            \item \texttt{INSERT:} Add new records to a table.
            \item \texttt{UPDATE:} Modify existing records.
            \item \texttt{DELETE:} Remove records from a table.
        \end{itemize}
    \end{block}
\end{frame}

\begin{frame}[fragile]
    \frametitle{Understanding SQL Databases - Example SQL Queries}
    \begin{lstlisting}[language=SQL]
    -- Retrieve all records from 'Employees' table
    SELECT * FROM Employees;

    -- Add a new employee to the 'Employees' table
    INSERT INTO Employees (Name, Position, Department) VALUES ('John Doe', 'Analyst', 'Finance');

    -- Update the salary of an employee
    UPDATE Employees SET Salary = 75000 WHERE Name = 'John Doe';

    -- Delete an employee record
    DELETE FROM Employees WHERE Name = 'John Doe';
    \end{lstlisting}
\end{frame}

\begin{frame}[fragile]
    \frametitle{Understanding SQL Databases - Example Scenario}
    Consider a company that uses a SQL database to manage its employees. The “Employees” table might have the following columns:
    \begin{itemize}
        \item EmployeeID (Primary Key)
        \item Name
        \item Position
        \item Department
        \item Salary
    \end{itemize}
\end{frame}

\begin{frame}[fragile]
    \frametitle{Understanding SQL Databases - Key Points}
    \begin{itemize}
        \item \textbf{Data Integrity:} SQL databases enforce data integrity through constraints (like primary keys and foreign keys), ensuring accuracy and consistency of data.
        \item \textbf{ACID Compliance:} SQL databases adhere to ACID properties (Atomicity, Consistency, Isolation, Durability), which are critical for reliable transactions.
        \item \textbf{Structured Querying:} SQL provides a standardized way to query data, making it accessible even for individuals without programming backgrounds.
    \end{itemize}
\end{frame}

\begin{frame}[fragile]
    \frametitle{Understanding SQL Databases - Conclusion}
    SQL databases are a powerful solution for structured data storage and management, offering robust data integrity, relationship mapping, and ease of use through structured querying. Understanding the foundational elements of SQL databases is essential for effective data management in various applications.
\end{frame}

\begin{frame}[fragile]
    \frametitle{Benefits of SQL Databases}
    SQL databases provide crucial advantages like ACID compliance, data integrity, and structured querying. These features ensure reliable data management and efficient querying capabilities, making them essential for numerous applications.
\end{frame}

\begin{frame}[fragile]
    \frametitle{1. ACID Compliance}
    \begin{block}{Definition}
        ACID stands for Atomicity, Consistency, Isolation, and Durability.
        \begin{itemize}
            \item \textbf{Atomicity}: All-or-nothing transactions; failure in one part fails the entire transaction.
            \item \textbf{Consistency}: Transitions the database from one valid state to another.
            \item \textbf{Isolation}: Independent transaction execution prevents interference.
            \item \textbf{Durability}: Committed transactions persist even after a failure.
        \end{itemize}
    \end{block}
    
    \begin{block}{Example}
        Consider a bank transfer: transferring \$100 from Account A to Account B. Both debit and credit must complete for the operation to succeed, ensuring data integrity.
    \end{block}
\end{frame}

\begin{frame}[fragile]
    \frametitle{2. Data Integrity}
    \begin{block}{Definition}
        Data integrity ensures data accuracy, consistency, and reliability over its lifecycle.
    \end{block}
    
    \begin{block}{Types of Data Integrity}
        \begin{itemize}
            \item \textbf{Entity Integrity}: Unique identifier (primary key) in each table.
            \item \textbf{Referential Integrity}: Foreign keys must reference valid primary keys in another table.
        \end{itemize}
    \end{block}
    
    \begin{block}{Illustration}
        \begin{itemize}
            \item \textbf{Primary Key}: `EmployeeID` in an `Employees` table must be unique.
            \item \textbf{Foreign Key}: `DepartmentID` in `Employees` must correspond to an existing `DepartmentID` in the `Departments` table.
        \end{itemize}
    \end{block}
\end{frame}

\begin{frame}[fragile]
    \frametitle{3. Structured Querying}
    \begin{block}{Definition}
        SQL enables users to write queries that are easy to understand while efficiently executing complex database operations.
    \end{block}
    
    \begin{block}{Example Query}
        \begin{lstlisting}
        SELECT EmployeeName, Salary
        FROM Employees
        WHERE DepartmentID = 1
        ORDER BY Salary DESC;
        \end{lstlisting}
    \end{block}
    
    \begin{block}{Key Points to Emphasize}
        \begin{itemize}
            \item SQL ensures \textbf{data reliability} through ACID properties, vital for critical applications.
            \item \textbf{Data integrity constraints} prevent anomalies and ensure valid relationships.
            \item \textbf{Structured Querying} facilitates strong data manipulation capabilities.
        \end{itemize}
    \end{block}
\end{frame}

\begin{frame}[fragile]
    \frametitle{Summarizing Takeaways}
    \begin{itemize}
        \item SQL databases are powerful tools for managing relational data, ensuring consistent and reliable transactions.
        \item Mastering SQL enhances your ability to handle data effectively across various domains, increasing your value in data-driven environments.
    \end{itemize}
    
    \begin{block}{Additional Note}
        For advanced users, learning about indexing and optimization techniques can enhance database performance and efficiency further.
    \end{block}
\end{frame}

\begin{frame}[fragile]
    \frametitle{Limitations of SQL Databases - Introduction}
    \begin{block}{Introduction}
        While SQL databases offer numerous benefits, they also come with inherent limitations that can impact their effectiveness for certain applications. Understanding these drawbacks is crucial for making informed decisions about data storage solutions.
    \end{block}
\end{frame}

\begin{frame}[fragile]
    \frametitle{Limitations of SQL Databases - Key Limitations}
    \begin{enumerate}
        \item \textbf{Scalability Issues}
            \begin{itemize}
                \item SQL databases typically require vertical scaling (upgrading to more powerful hardware).
                \item Challenging to manage performance bottlenecks as user demand grows.
            \end{itemize}
        \item \textbf{Flexibility Constraints}
            \begin{itemize}
                \item Rigid schema necessitates detailed migration processes for changes.
                \item Hinders rapid development in fast-evolving environments.
            \end{itemize}
    \end{enumerate}
\end{frame}

\begin{frame}[fragile]
    \frametitle{Limitations of SQL Databases - Further Drawbacks}
    \begin{enumerate}
        \setcounter{enumi}{2} % Continue from previous frame
        \item \textbf{Handling Unstructured Data}
            \begin{itemize}
                \item SQL databases struggle with unstructured data (e.g., images, videos).
                \item Example: Social media platforms may find it difficult to efficiently store multimedia content.
            \end{itemize}
        \item \textbf{JOIN Operations Performance}
            \begin{itemize}
                \item Complex JOIN operations across large tables can slow down performance.
                \item Highly interconnected data can lead to inefficiency.
            \end{itemize}
        \item \textbf{Cost of High Availability Solutions}
            \begin{itemize}
                \item High availability often requires additional infrastructure.
                \item Example: Master-slave replication increases both complexity and operational costs.
            \end{itemize}
    \end{enumerate}
\end{frame}

\begin{frame}[fragile]
    \frametitle{Limitations of SQL Databases - Conclusion and Key Takeaways}
    \begin{block}{Conclusion}
        Understanding these limitations of SQL databases is crucial for developers and database administrators, as they may necessitate alternative solutions like NoSQL based on specific project needs.
    \end{block}
    \begin{block}{Key Takeaways}
        \begin{itemize}
            \item SQL databases may struggle to scale effectively.
            \item Schema rigidity can impede rapid development efforts.
            \item Performance bottlenecks can arise from complex data relationships.
        \end{itemize}
    \end{block}
\end{frame}

\begin{frame}[fragile]
    \frametitle{Understanding NoSQL Databases - Definition}
    \begin{block}{Definition of NoSQL Databases}
        NoSQL databases, or "not only SQL," are designed to provide flexible storage and retrieval of data in a way that surpasses traditional relational databases. 
        They are particularly well-suited for applications involving large volumes of unstructured or semi-structured data, real-time analytics, and horizontal scalability.
    \end{block}
\end{frame}

\begin{frame}[fragile]
    \frametitle{Understanding NoSQL Databases - Structure}
    \begin{itemize}
        \item \textbf{Schema-less Design}: NoSQL databases typically do not require a fixed schema, allowing fields to be added or removed without affecting existing data.
        
        \item \textbf{Distributed Architecture}: They often use a distributed architecture, which allows horizontal scaling by adding more servers, providing flexibility and high availability.
    \end{itemize}
\end{frame}

\begin{frame}[fragile]
    \frametitle{Types of NoSQL Databases}
    \begin{enumerate}
        \item \textbf{Key-Value Stores}
            \begin{itemize}
                \item \textbf{Examples}: Redis, DynamoDB
                \item \textbf{Use Case}: Best for caching and session data.
                \item \textbf{Illustration}:
                \begin{lstlisting}[basicstyle=\small]
                { "user123": "John Doe", "user456": "Jane Smith" }
                \end{lstlisting}
            \end{itemize}
        
        \item \textbf{Document Stores}
            \begin{itemize}
                \item \textbf{Examples}: MongoDB, CouchDB
                \item \textbf{Use Case}: Ideal for content management systems and varied data structures.
                \item \textbf{Illustration}:
                \begin{lstlisting}[basicstyle=\small]
                {
                  "name": "John Doe",
                  "age": 30,
                  "interests": ["coding", "music"]
                }
                \end{lstlisting}
            \end{itemize}
        
        \item \textbf{Wide-Column Stores}
            \begin{itemize}
                \item \textbf{Examples}: Apache Cassandra, HBase
                \item \textbf{Use Case}: Suitable for analytical applications with dynamic column definitions.
                \item \textbf{Illustration}:
                \begin{lstlisting}[basicstyle=\small]
                Row Key: 1
                Columns: Name -> John Doe, Age -> 30, Occupation -> Developer
                \end{lstlisting}
            \end{itemize}
        
        \item \textbf{Graph Databases}
            \begin{itemize}
                \item \textbf{Examples}: Neo4j, Amazon Neptune
                \item \textbf{Use Case}: Perfect for understanding complex relationships like social networks.
                \item \textbf{Illustration}:
                \begin{lstlisting}[basicstyle=\small]
                John (Node) --[FRIEND]--> Doe (Node)
                \end{lstlisting}
            \end{itemize}
    \end{enumerate}
\end{frame}

\begin{frame}[fragile]
    \frametitle{Key Points to Emphasize}
    \begin{itemize}
        \item NoSQL databases provide flexibility and scalability absent in traditional SQL databases.
        \item They cater to specific use cases based on data types and relationships.
        \item Understanding the structure and purpose of each type can guide effective database design and application development.
    \end{itemize}
    
    By grasping the concepts of NoSQL architecture and its various forms, students can effectively leverage these technologies in modern application development.
\end{frame}

\begin{frame}[fragile]
    \frametitle{Benefits of NoSQL Databases}
    \begin{block}{Introduction to NoSQL Databases}
        NoSQL databases are designed to handle various data types and structures that traditional relational databases may struggle with. They excel in environments requiring large-scale data storage and management, particularly with unstructured and semi-structured data.
    \end{block}
\end{frame}

\begin{frame}[fragile]
    \frametitle{Key Benefits of NoSQL Databases - Part 1}
    \begin{enumerate}
        \item \textbf{Scalability}
            \begin{itemize}
                \item \textbf{Horizontal Scaling:} Designed to scale out with more servers (nodes) rather than upgrading a single server.
                \item \textbf{Example:} A social media platform distributing user activity across multiple servers.
            \end{itemize}
            
        \item \textbf{Flexibility}
            \begin{itemize}
                \item \textbf{Schema-less Design:} Allows varying data structures without downtime.
                \item \textbf{Example:} Document-based databases storing different user profiles with varying attributes.
            \end{itemize}
    \end{enumerate}
\end{frame}

\begin{frame}[fragile]
    \frametitle{Key Benefits of NoSQL Databases - Part 2}
    \begin{enumerate}[resume]
        \item \textbf{Performance with Unstructured Data}
            \begin{itemize}
                \item \textbf{High Throughput and Low Latency:} Optimized for speed with vast amounts of unstructured data.
                \item \textbf{Example:} Streaming services efficiently managing video metadata.
            \end{itemize}
            
        \item \textbf{Data Variety}
            \begin{itemize}
                \item \textbf{Handling Diverse Data Types:} Accommodates text, JSON, images, and graphs easily.
                \item \textbf{Key-Value Pairs:} Stored as collections allowing efficient access and modification.
            \end{itemize}
            
        \item \textbf{Eventual Consistency}
            \begin{itemize}
                \item \textbf{Optimized for Availability:} Allows faster write operations, useful in applications where perfect consistency is not critical.
                \item \textbf{Example:} Shopping cart applications synchronizing states in the background.
            \end{itemize}
    \end{enumerate}
\end{frame}

\begin{frame}[fragile]
    \frametitle{Key Points to Emphasize}
    \begin{itemize}
        \item \textbf{Adapting to Change:} Quick adaptation leads to faster development cycles and innovation.
        \item \textbf{Cost Efficiency:} Utilizes commodity hardware for horizontal scaling, saving infrastructure costs.
        \item \textbf{Real-world Applications:} Modern web applications, online retailers, and big data analytics heavily rely on NoSQL solutions.
    \end{itemize}
\end{frame}

\begin{frame}[fragile]
    \frametitle{Summary and Conclusion}
    \begin{block}{Summary}
        NoSQL databases offer numerous advantages for handling large volumes of unstructured data. These benefits guide organizations to choose suitable data storage solutions.
    \end{block}
    \begin{block}{Conclusion}
        In the next section, we will explore the limitations of NoSQL databases to provide a balanced view of their use in modern data environments.
    \end{block}
\end{frame}

\begin{frame}[fragile]{Limitations of NoSQL Databases - Overview}
    \begin{itemize}
        \item NoSQL databases have transformed data storage, particularly for unstructured and semi-structured data.
        \item Understanding their limitations is crucial for informed architectural decisions.
    \end{itemize}
\end{frame}

\begin{frame}[fragile]{Limitations of NoSQL Databases - Part 1: Eventual Consistency}
    \begin{block}{Explanation}
        \begin{itemize}
            \item **Eventual Consistency** guarantees that, if no new updates are made, all accesses will eventually return the last updated value.
            \item Data may not be immediately consistent across all nodes, allowing faster write operations.
        \end{itemize}
    \end{block}

    \begin{block}{Key Points}
        \begin{itemize}
            \item \textbf{Pros}: Improves performance by prioritizing availability and partition tolerance.
            \item \textbf{Cons}: Temporary inconsistencies can lead to users seeing different data.
        \end{itemize}
    \end{block}

    \begin{block}{Example}
        Consider an online shopping platform where...
    \end{block}
\end{frame}

\begin{frame}[fragile]{Limitations of NoSQL Databases - Part 2: Complex Querying}
    \begin{block}{Explanation}
        \begin{itemize}
            \item Many NoSQL databases lack complex querying capabilities compared to SQL databases.
        \end{itemize}
    \end{block}

    <br>

    \begin{block}{Key Points}
        \begin{itemize}
            \item \textbf{Lack of Joins}: NoSQL systems often do not support JOIN operations.
            \item \textbf{Limited Query Language}: Query languages may lack the richness of SQL.
        \end{itemize}
    \end{block}

    \begin{block}{Example}
        A user wanting to retrieve all orders for a...
    \end{block}
\end{frame}

\begin{frame}[fragile]{Limitations of NoSQL Databases - Part 3: Other Limitations}
    \begin{block}{Data Model Constraints}
        \begin{itemize}
            \item **Schema Flexibility** can lead to data integrity issues without a fixed schema.
            \item Maintaining organization can be challenging as applications evolve.
        \end{itemize}
    \end{block}

    <br>

    \begin{block}{Transaction Support}
        \begin{itemize}
            \item Many NoSQL databases offer limited support for ACID transactions.
            \item Strong transactional guarantees may be lacking for critical applications.
        \end{itemize}
    \end{block}

    \begin{block}{Conclusion}
        Understanding these limitations aids in selecting the right database technology...
    \end{block}
\end{frame}

\begin{frame}[fragile]
    \frametitle{Choosing the Right Solution}
    When selecting a database solution, various factors come into play, particularly when comparing SQL (Structured Query Language) and NoSQL (Not Only SQL) databases. Each type has distinct characteristics that align with specific needs of applications.
\end{frame}

\begin{frame}[fragile]
    \frametitle{Key Factors to Consider}
    \begin{enumerate}
        \item \textbf{Data Structure}:
            \begin{itemize}
                \item \textbf{SQL}: Uses structured schemas defined by tables. Useful for applications requiring complex queries.
                \item \textbf{NoSQL}: Employs flexible schemas, allowing various data formats like key-value pairs or documents.
            \end{itemize}
        \item \textbf{Scalability Needs}:
            \begin{itemize}
                \item \textbf{SQL}: Vertically scalable, which can limit performance as user load increases.
                \item \textbf{NoSQL}: Horizontally scalable, allows adding more servers without downtime.
            \end{itemize}
        \item \textbf{Use Cases}:
            \begin{itemize}
                \item \textbf{SQL}: Best for applications needing ACID compliance (e.g., banking systems).
                \item \textbf{NoSQL}: Ideal for large datasets and real-time analytics (e.g., real-time web apps like Netflix).
            \end{itemize}
    \end{enumerate}
\end{frame}

\begin{frame}[fragile]
    \frametitle{Key Points to Emphasize}
    \begin{itemize}
        \item SQL excels in complex queries, while NoSQL is faster for simple read/write operations.
        \item NoSQL often provides eventual consistency, less ideal for strong transactional guarantees.
        \item Cost implications vary based on infrastructure and scaling challenges.
    \end{itemize}
\end{frame}

\begin{frame}[fragile]
    \frametitle{Summary}
    Choosing between SQL and NoSQL is fundamentally about understanding your data needs: the structure of your data, the expected growth, and the nature of queries performed. The right choice is crucial for your application's success and efficiency.
\end{frame}

\begin{frame}[fragile]
    \frametitle{Code Snippets}
    \begin{block}{Example Code}
        \textbf{SQL Example}:
        \begin{lstlisting}[language=SQL]
SELECT customer_id, order_id 
FROM orders 
WHERE order_date > '2023-01-01';
        \end{lstlisting}

        \textbf{NoSQL Example (MongoDB)}:
        \begin{lstlisting}
db.orders.find({"order_date": {$gt: new ISODate("2023-01-01")}});
        \end{lstlisting}
    \end{block}
\end{frame}

\begin{frame}[fragile]
    \frametitle{Real-World Applications - Introduction}
    \begin{itemize}
        \item SQL (Structured Query Language) and NoSQL (Not Only SQL) databases are widely used in various industries.
        \item Understanding real-world applications highlights the strengths and use cases of each type.
    \end{itemize}
\end{frame}

\begin{frame}[fragile]
    \frametitle{Real-World Applications - SQL Database Examples}
    \begin{enumerate}
        \item \textbf{Banking Systems}
            \begin{itemize}
                \item Relies on SQL for transaction processing.
                \item SQL's ACID properties ensure data integrity.
                \item Example systems: Oracle Database, MySQL.
            \end{itemize}
        \item \textbf{E-commerce Platforms}
            \begin{itemize}
                \item Used by online retailers like Amazon.
                \item Manages inventories, customer data, and transactions effectively.
                \item Supports complex JOIN operations ensuring data integrity.
            \end{itemize}
    \end{enumerate}
\end{frame}

\begin{frame}[fragile]
    \frametitle{Real-World Applications - NoSQL Database Examples}
    \begin{enumerate}
        \item \textbf{Social Media Networks}
            \begin{itemize}
                \item Companies like Facebook and Twitter use NoSQL.
                \item Document-oriented databases (e.g., MongoDB) manage varied content.
                \item High scalability and flexibility for unstructured data.
            \end{itemize}
        \item \textbf{Real-Time Analytics}
            \begin{itemize}
                \item Companies such as Airbnb utilize NoSQL.
                \item Supports high-throughput operations for dynamic datasets.
                \item Fast data ingestion and retrieval is critical.
            \end{itemize}
    \end{enumerate}
\end{frame}

\begin{frame}[fragile]
    \frametitle{Real-World Applications - Hybrid Use Cases}
    \begin{itemize}
        \item \textbf{Online Gaming}
            \begin{itemize}
                \item Games like Clash of Clans leverage both SQL and NoSQL.
                \item SQL handles user account transactions.
                \item NoSQL manages game state and player interactions.
                \item Combining both offers optimal performance and scalability.
            \end{itemize}
    \end{itemize}
\end{frame}

\begin{frame}[fragile]
    \frametitle{Real-World Applications - Summary}
    \begin{itemize}
        \item \textbf{Key Takeaways:}
            \begin{itemize}
                \item \textbf{SQL Databases}: Best choice for structured data needing accuracy.
                \item \textbf{NoSQL Databases}: Ideal for unstructured data and scalability.
                \item \textbf{Hybrid Solutions}: Utilize both to leverage respective strengths.
            \end{itemize}
    \end{itemize}
\end{frame}

\begin{frame}[fragile]
    \frametitle{Real-World Applications - Closing Thoughts}
    \begin{itemize}
        \item Understanding the use of SQL and NoSQL databases enhances data management capabilities.
        \item Informed decision-making leads to strategic growth in businesses.
    \end{itemize}
\end{frame}

\begin{frame}[fragile]
    \frametitle{Future Trends in Data Storage}
    \begin{block}{Introduction}
        As the digital landscape evolves, so do the technologies and methodologies that underpin data storage. This slide explores two significant trends: 
        \begin{itemize}
            \item \textbf{Hybrid databases}
            \item \textbf{Cloud storage solutions}
        \end{itemize}
        Understanding these advancements is essential for leveraging data effectively in various applications.
    \end{block}
\end{frame}

\begin{frame}[fragile]
    \frametitle{Hybrid Databases}
    \begin{block}{Definition}
        A hybrid database combines different database technologies to optimize performance and scalability, typically incorporating both SQL (structured) and NoSQL (unstructured) capabilities.
    \end{block}
    
    \begin{block}{Purpose}
        By merging the strengths of both types of databases, organizations can handle diverse data types and workloads more efficiently.
    \end{block}
    
    \begin{block}{Example}
        \textbf{Company:} Netflix uses a hybrid database system to manage user data (SQL) while handling massive amounts of video streaming data (NoSQL), allowing for personalized recommendations and fast data retrieval.
    \end{block}
\end{frame}

\begin{frame}[fragile]
    \frametitle{Cloud Storage Solutions}
    \begin{block}{Definition}
        Cloud storage is a service that allows users to store and manage data on remote servers accessed via the internet rather than on local servers or personal devices.
    \end{block}
    
    \begin{block}{Benefits}
        \begin{itemize}
            \item \textbf{Scalability:} Easily expand storage capacity as needed.
            \item \textbf{Cost-Effectiveness:} Pay for what you use, reducing costs associated with maintaining physical hardware.
            \item \textbf{Accessibility:} Data can be accessed from anywhere, facilitating collaboration and remote work.
        \end{itemize}
    \end{block}
    
    \begin{block}{Example}
        \textbf{Service:} Amazon Web Services (AWS) S3 (Simple Storage Service) provides scalable object storage for cloud applications, enabling organizations to upload, manage, and retrieve data globally.
    \end{block}
\end{frame}

\begin{frame}[fragile]
    \frametitle{Emphasizing Key Points}
    \begin{itemize}
        \item \textbf{Flexibility:} Hybrid databases provide the ability to leverage both SQL and NoSQL, allowing for tailored solutions based on data needs.
        \item \textbf{Innovation and Efficiency:} Cloud storage solutions drive innovation by simplifying backup, sharing, and collaboration processes.
    \end{itemize}
\end{frame}

\begin{frame}[fragile]
    \frametitle{Technical Insights}
    \begin{block}{Performance}
        Hybrid databases can increase performance by routing queries to the most appropriate database type.
    \end{block}

    \textbf{SQL Query Example:}
    \begin{lstlisting}[language=SQL]
    SELECT * FROM users WHERE age > 30;
    \end{lstlisting}

    \textbf{NoSQL Query Example:}
    \begin{lstlisting}[language=json]
    db.collection.find({tags: 'technology'});
    \end{lstlisting}
\end{frame}

\begin{frame}[fragile]
    \frametitle{Conclusion}
    Staying ahead of data storage trends like hybrid databases and cloud storage solutions is critical for businesses looking to innovate and optimize their data management strategies. By understanding and utilizing these technologies, organizations can enhance their ability to handle vast amounts of data effectively.

    \begin{block}{Next Steps}
        In the conclusion slide, we will summarize the key takeaways and their implications for future data storage strategies.
    \end{block}
\end{frame}

\begin{frame}[fragile]
    \frametitle{Conclusion - Key Takeaways}
    In our exploration of data storage solutions, we have highlighted several fundamental concepts crucial for managing data efficiently.

    \begin{itemize}
        \item Understanding different types of storage solutions
        \item Recognizing future trends in data management
        \item Evaluating considerations for choosing storage solutions
        \item Importance of key metrics in cost and performance assessment
    \end{itemize}
\end{frame}

\begin{frame}[fragile]
    \frametitle{Types of Data Storage Solutions}
    \begin{enumerate}
        \item \textbf{Traditional Storage}
            \begin{itemize}
                \item Local devices: HDDs (cost-effective, high capacity) and SSDs (faster performance)
            \end{itemize}
        \item \textbf{Cloud Storage}
            \begin{itemize}
                \item Internet-based access, managed by third-party providers (e.g., Amazon S3, Google Cloud)
            \end{itemize}
        \item \textbf{Hybrid Storage}
            \begin{itemize}
                \item Combination of on-premises and cloud solutions for optimized performance
            \end{itemize}
    \end{enumerate}
\end{frame}

\begin{frame}[fragile]
    \frametitle{Future Trends and Considerations}
    \begin{block}{Future Trends}
        \begin{itemize}
            \item Emergence of data lakes for unstructured data
            \item Increasing use of AI in predictive storage management
            \item Growing focus on data security (encrypted storage, access controls)
        \end{itemize}
    \end{block}

    \begin{block}{Considerations in Choosing a Storage Solution}
        \begin{itemize}
            \item Scalability: Can it grow with your needs?
            \item Cost-Effectiveness: Understand upfront and ongoing costs
            \item Performance: Speed and reliability requirements
        \end{itemize}
    \end{block}
\end{frame}

\begin{frame}[fragile]
    \frametitle{Key Metrics and Final Thoughts}
    \begin{itemize}
        \item \textbf{Total Cost of Ownership (TCO)}: Includes purchase, maintenance, and operational costs
        \item \textbf{Access Time}: Speed of data retrieval impacting performance
        \item \textbf{Cost per Gigabyte (CPGB)}:
            \begin{equation}
            \text{CPGB} = \frac{\text{Total Cost}}{\text{Total Storage in GB}}
            \end{equation}
    \end{itemize}

    \begin{block}{Final Thoughts}
        Understanding data storage solutions aids in making informed decisions that ensure data integrity, security, and accessibility. 
    \end{block}
    
    \begin{block}{Engaging Question}
        How do you envision the evolution of data storage impacting your future career in tech? 
    \end{block}
\end{frame}


\end{document}