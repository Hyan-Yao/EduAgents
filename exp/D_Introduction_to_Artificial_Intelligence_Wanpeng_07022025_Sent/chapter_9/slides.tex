\documentclass{beamer}

% Theme choice
\usetheme{Madrid} % You can change to e.g., Warsaw, Berlin, CambridgeUS, etc.

% Encoding and font
\usepackage[utf8]{inputenc}
\usepackage[T1]{fontenc}

% Graphics and tables
\usepackage{graphicx}
\usepackage{booktabs}

% Code listings
\usepackage{listings}
\lstset{
basicstyle=\ttfamily\small,
keywordstyle=\color{blue},
commentstyle=\color{gray},
stringstyle=\color{red},
breaklines=true,
frame=single
}

% Math packages
\usepackage{amsmath}
\usepackage{amssymb}

% Colors
\usepackage{xcolor}

% TikZ and PGFPlots
\usepackage{tikz}
\usepackage{pgfplots}
\pgfplotsset{compat=1.18}
\usetikzlibrary{positioning}

% Hyperlinks
\usepackage{hyperref}

% Title information
\title{Week 9: Hands-on Workshop: Building a Model}
\author{Your Name}
\institute{Your Institution}
\date{\today}

\begin{document}

\frame{\titlepage}

\begin{frame}[fragile]
    \frametitle{Introduction to Hands-on Workshop}
    \begin{block}{Overview of the Practical Coding Session}
        In this hands-on workshop, we will delve into the exciting world of Artificial Intelligence (AI) by building our own AI models. 
        You will gain valuable experience implementing models using provided datasets, enhancing your understanding of model development from data preprocessing to evaluation.
    \end{block}
\end{frame}

\begin{frame}[fragile]
    \frametitle{Key Concepts to Understand}
    \begin{enumerate}
        \item \textbf{AI Models:} Algorithms designed to mimic human cognitive functions, including:
        \begin{itemize}
            \item Linear regression
            \item Decision trees
            \item Neural networks
        \end{itemize}
        
        \item \textbf{Datasets:} Collections of data for training/testing models, consisting of features (inputs) and labels (outputs). E.g., images of handwritten digits with corresponding labels.
    \end{enumerate}
\end{frame}

\begin{frame}[fragile]
    \frametitle{Workshop Structure}
    \begin{block}{Data Loading}
        We will start by loading the provided datasets into our coding environment:
        \begin{lstlisting}[language=Python]
import pandas as pd

# Load a sample dataset
data = pd.read_csv('dataset.csv')
        \end{lstlisting}
    \end{block}
    
    \begin{block}{Data Preprocessing}
        Cleaning and transforming data to make it suitable for model training:
        \begin{lstlisting}[language=Python]
# Example of filling missing values
data.fillna(method='ffill', inplace=True)
        \end{lstlisting}
    \end{block}
    
    \begin{block}{Model Building}
        Choosing algorithms and implementing them:
        \begin{lstlisting}[language=Python]
from sklearn.model_selection import train_test_split
from sklearn.ensemble import RandomForestClassifier

# Splitting data into training and testing sets
X = data.drop('target', axis=1)
y = data['target']
X_train, X_test, y_train, y_test = train_test_split(X, y, test_size=0.2)

# Building a Random Forest model
model = RandomForestClassifier()
model.fit(X_train, y_train)
        \end{lstlisting}
    \end{block}
\end{frame}

\begin{frame}[fragile]
    \frametitle{Model Evaluation}
    After building your model, we will assess its performance:
    \begin{block}{Evaluation Metrics}
        Using accuracy, precision, recall, and F1-score:
        \begin{lstlisting}[language=Python]
from sklearn.metrics import accuracy_score

# Making predictions
predictions = model.predict(X_test)

# Evaluating accuracy
accuracy = accuracy_score(y_test, predictions)
print(f'Accuracy: {accuracy * 100:.2f}%')
        \end{lstlisting}
    \end{block}
    
    \begin{itemize}
        \item \textbf{Hands-on Learning:} Emphasizes practical skills through real coding experience.
        \item \textbf{Iterative Process:} Model building is iterative; refine your models based on evaluations.
        \item \textbf{Collaboration:} Engage with peers to share insights and challenges.
        \item \textbf{Ethical AI:} Consider ethical implications and AI's societal impact during model building.
    \end{itemize}
\end{frame}

\begin{frame}[fragile]
    \frametitle{Conclusion}
    By the end of this workshop, you will have:
    \begin{itemize}
        \item Developed a working AI model
        \item Gained a deeper understanding of AI principles, tools, and best practices
    \end{itemize}
    \textbf{Next up:} Objectives of the Workshop - discussing key goals, including model building, evaluation, and the importance of ethical considerations in AI development.
\end{frame}

\begin{frame}[fragile]
    \frametitle{Workshop Objectives - Overview}
    In this workshop, we aim to provide students with hands-on experience in building and evaluating AI models. This session emphasizes practical skills while also addressing the essential ethical considerations involved in AI development.
\end{frame}

\begin{frame}[fragile]
    \frametitle{Workshop Objectives - Key Concepts}
    \begin{enumerate}
        \item \textbf{Model Building}
        \item \textbf{Model Evaluation}
        \item \textbf{Ethical Considerations}
    \end{enumerate}
\end{frame}

\begin{frame}[fragile]
    \frametitle{Workshop Objectives - Model Building}
    \begin{block}{Concept}
        Participants will learn how to create an AI model from scratch, utilizing provided datasets as a foundation.
    \end{block}
    \begin{itemize}
        \item \textbf{Example}: Building a model to classify handwritten digits (using the MNIST dataset).
        \item Participants will go through steps of:
            \begin{itemize}
                \item Data preprocessing
                \item Model architecture design
                \item Training
                \item Testing
            \end{itemize}
        \item \textbf{Key Point}: Importance of selecting the right architecture (e.g., Convolutional Neural Network for image classification) and configuring hyperparameters (learning rate, batch size).
    \end{itemize}
\end{frame}

\begin{frame}[fragile]
    \frametitle{Workshop Objectives - Model Evaluation}
    \begin{block}{Concept}
        After building the model, it is crucial to assess its performance rigorously.
    \end{block}
    \begin{itemize}
        \item \textbf{Example}: Using metrics like accuracy, precision, recall, and F1-score to evaluate the model’s effectiveness.
        \item For instance, in a binary classification task:
        \begin{equation}
            \text{Precision} = \frac{\text{True Positives}}{\text{True Positives} + \text{False Positives}}
        \end{equation}
        \item \textbf{Key Point}: Learn tools and techniques (like confusion matrices) to analyze model output, ensuring identification of strengths and weaknesses.
    \end{itemize}
\end{frame}

\begin{frame}[fragile]
    \frametitle{Workshop Objectives - Ethical Considerations}
    \begin{block}{Concept}
        Understanding the ethical implications of AI models is vital in today’s tech landscape.
    \end{block}
    \begin{itemize}
        \item \textbf{Discussion Points}:
            \begin{itemize}
                \item Bias in AI: Recognizing how biased data can lead to unfair outcomes.
                \item Informed Consent: Ensuring individuals are fully aware of how their data will be used.
                \item Accountability: Discussing responsibility if the model causes harm.
            \end{itemize}
        \item \textbf{Key Point}: Emphasize the importance of transparency and fairness in AI, advocating for inclusive practices.
    \end{itemize}
\end{frame}

\begin{frame}[fragile]
    \frametitle{Workshop Objectives - Conclusion}
    By the end of this workshop, students will have practical experience in building and evaluating a model while being equipped to critically examine the ethical dimensions of AI technology. This holistic approach prepares them to not just implement AI solutions, but to do so responsibly and with awareness of their wider impact.
\end{frame}

\begin{frame}[fragile]
    \frametitle{Preparation Steps - Hardware Requirements}
    \begin{block}{Hardware Requirements}
        \begin{itemize}
            \item \textbf{Computer Specifications:}
            \begin{itemize}
                \item \textbf{Processor:} 
                \begin{itemize}
                    \item Minimum: Intel Core i5 or equivalent
                    \item Recommended: Intel Core i7 or AMD Ryzen 7
                \end{itemize}
                \item \textbf{RAM:} 
                \begin{itemize}
                    \item Minimum: 8 GB
                    \item Recommended: 16 GB or more (for deep learning tasks)
                \end{itemize}
                \item \textbf{Graphics Card (GPU):}
                \begin{itemize}
                    \item Minimum: Integrated GPU (for basic tasks)
                    \item Recommended: NVIDIA GTX 1060 or higher (for performance in training models)
                \end{itemize}
            \end{itemize}
            \item \textbf{Storage:}
            \begin{itemize}
                \item SSD Recommended: Provides faster read/write speeds
                \item At least 50 GB of free space for datasets and framework installations
            \end{itemize}
        \end{itemize}
    \end{block}
\end{frame}

\begin{frame}[fragile]
    \frametitle{Preparation Steps - Software Requirements}
    \begin{block}{Software Requirements}
        \begin{itemize}
            \item \textbf{Operating System:}
            \begin{itemize}
                \item Windows 10/11, macOS, or a Linux distribution (Ubuntu preferred)
            \end{itemize}
            \item \textbf{Integrated Development Environment (IDE):}
            \begin{itemize}
                \item Jupyter Notebook: Great for interactive coding and visualization.
                \item PyCharm: Full-featured IDE for Python with excellent support for libraries.
                \item Visual Studio Code: Lightweight editor with extensions for Python and Jupyter support.
            \end{itemize}
        \end{itemize}
    \end{block}
\end{frame}

\begin{frame}[fragile]
    \frametitle{Preparation Steps - AI Frameworks}
    \begin{block}{AI Frameworks}
        \begin{itemize}
            \item \textbf{TensorFlow:}
            \begin{itemize}
                \item Developed by Google for building machine learning models. Industry-standard for production.
                \item Installation: 
                \begin{lstlisting}
                    pip install tensorflow
                \end{lstlisting}
                \item Example:
                \begin{lstlisting}[language=Python]
                    import tensorflow as tf
                    model = tf.keras.models.Sequential([
                        tf.keras.layers.Dense(128, activation='relu', input_shape=(input_shape,)),
                        tf.keras.layers.Dense(10, activation='softmax')
                    ])
                \end{lstlisting}
            \end{itemize}
            \item \textbf{PyTorch:}
            \begin{itemize}
                \item Developed by Facebook, favored for research and academia due to its flexibility.
                \item Installation: 
                \begin{lstlisting}
                    pip install torch torchvision
                \end{lstlisting}
                \item Example:
                \begin{lstlisting}[language=Python]
                    import torch
                    import torch.nn as nn
                    model = nn.Sequential(
                        nn.Linear(input_size, 128),
                        nn.ReLU(),
                        nn.Linear(128, 10)
                    )
                \end{lstlisting}
            \end{itemize}
        \end{itemize}
    \end{block}
\end{frame}

\begin{frame}[fragile]
    \frametitle{Dataset Overview}
    Datasets are the cornerstone of any AI model. In this workshop, we will explore three distinct datasets, each tailored for various AI applications. 
\end{frame}

\begin{frame}[fragile]
    \frametitle{Key Characteristics of the Datasets}
    \begin{enumerate}
        \item \textbf{Size and Structure:}
            \begin{itemize}
                \item Dataset 1: MNIST - 70,000 hand-written digit images (28x28 pixels)
                \item Dataset 2: CIFAR-10 - 60,000 32x32 color images across 10 classes
                \item Dataset 3: UCI Adult Income Dataset - 32,000 entries with features like age, education, and income
            \end{itemize}
        \item \textbf{Data Types:}
            \begin{itemize}
                \item Numerical: Continuous values (e.g., age, income)
                \item Categorical: Discrete groups (e.g., gender, job type)
                \item Image/Text: Unstructured data requiring specialized preprocessing
            \end{itemize}
        \item \textbf{Label Availability:}
            \begin{itemize}
                \item Supervised: Data points with corresponding labels (e.g., image categorization)
                \item Unsupervised: Unlabeled data for clustering or pattern recognition
            \end{itemize}
    \end{enumerate}
\end{frame}

\begin{frame}[fragile]
    \frametitle{Relevance to AI Modeling}
    \begin{block}{Training \& Validation}
        Datasets are essential for training AI models and validating their performance. Well-structured datasets enhance model generalization on unseen data.
    \end{block}

    \begin{block}{Real-World Applications}
        \begin{itemize}
            \item Dataset 1: Digit recognition in postal services
            \item Dataset 2: Image classification in self-driving cars
            \item Dataset 3: Income prediction for loan approvals
        \end{itemize}
    \end{block}
\end{frame}

\begin{frame}[fragile]
    \frametitle{Key Points and Example Code}
    \begin{itemize}
        \item \textbf{Quality of Data:} Essential for building effective AI models
        \item \textbf{Diversity of Features:} Enhances the model's learning capability
        \item \textbf{Understanding Limitations:} Awareness of biases and data quality issues is crucial
    \end{itemize}

    \begin{lstlisting}[language=Python]
import pandas as pd

# Load the UCI Adult Income Dataset
data = pd.read_csv('adult.csv')
print(data.head())
    \end{lstlisting}
\end{frame}

\begin{frame}[fragile]
    \frametitle{Conclusion}
    Understanding the datasets will significantly impact your ability to build, train, and evaluate AI models effectively during this workshop. Let’s explore each dataset and leverage its strengths in our AI modeling journey.
\end{frame}

\begin{frame}

\frametitle{AI Model Development Workflow - Overview}

Building an AI model is a structured process that involves several key steps, each crucial to achieving a functional and effective model. Understanding this workflow is essential for successful AI development.

\end{frame}

\begin{frame}[fragile]

\frametitle{AI Model Development Workflow - Data Preprocessing}

\begin{block}{1. Data Preprocessing}
    \begin{itemize}
        \item \textbf{Definition:} Preparing and cleaning the data before it can be used to train the model.
        
        \item \textbf{Steps:}
        \begin{itemize}
            \item \textbf{Data Collection:} Gathering data from various sources (databases, CSV files, APIs).
            \item \textbf{Data Cleaning:} Handling missing values and removing duplicates.
            \item \textbf{Data Transformation:} Scaling features using techniques like StandardScaler or MinMaxScaler.
            \item \textbf{Feature Engineering:} Creating new features to improve model performance.
        \end{itemize}
    \end{itemize}
\end{block}

\end{frame}

\begin{frame}[fragile]

\frametitle{AI Model Development Workflow - Example}

\begin{block}{Example of Data Preprocessing}
If we have a dataset with heights in cm and we want to predict weights, we might normalize heights to a scale of 0 to 1.
\end{block}

\end{frame}

\begin{frame}[fragile]

\frametitle{AI Model Development Workflow - Model Training}

\begin{block}{2. Model Training}
    \begin{itemize}
        \item \textbf{Definition:} Using the prepared dataset to train the AI model.
        
        \item \textbf{Concepts:}
        \begin{itemize}
            \item \textbf{Choosing a Model:} Options include decision trees, neural networks, or support vector machines.
            \item \textbf{Training Process:} The model learns from the data using optimization algorithms like gradient descent.
        \end{itemize}
    \end{itemize}
\end{block}

\end{frame}

\begin{frame}[fragile]

\frametitle{AI Model Development Workflow - Code Snippet}

\begin{block}{Code Snippet for Model Training}
\begin{lstlisting}[language=Python]
from sklearn.model_selection import train_test_split
from sklearn.ensemble import RandomForestRegressor

# Splitting the dataset into training and testing sets
X_train, X_test, y_train, y_test = train_test_split(X, y, test_size=0.2, random_state=42)

# Training the model
model = RandomForestRegressor()
model.fit(X_train, y_train)
\end{lstlisting}
\end{block}

\end{frame}

\begin{frame}[fragile]

\frametitle{AI Model Development Workflow - Evaluation & Testing}

\begin{block}{3. Evaluation}
    \begin{itemize}
        \item \textbf{Definition:} Assess how well the model performs on unseen data.
        
        \item \textbf{Techniques:}
        \begin{itemize}
            \item \textbf{Cross-validation:} Ensures model robustness.
            \item \textbf{Metrics:} Common metrics include accuracy, precision, recall, and F1 score.
        \end{itemize}
    \end{itemize}
\end{block}

\end{frame}

\begin{frame}[fragile]

\frametitle{AI Model Development Workflow - Testing}

\begin{block}{4. Testing}
    \begin{itemize}
        \item \textbf{Definition:} Test the model on a separate dataset to confirm performance before deployment.
        
        \item \textbf{Key Point:} Always keep a part of your dataset as a test set to avoid data leakage and overfitting.
    \end{itemize}
\end{block}

\end{frame}

\begin{frame}

\frametitle{AI Model Development Workflow - Conclusion}

The AI model development workflow is iterative; revisiting earlier steps can enhance model robustness. Each phase contributes to the creation of a reliable AI model capable of real-world application.

\end{frame}

\begin{frame}
    \frametitle{Hands-on Coding Session}
    Welcome to the Hands-on Coding Session! In this interactive workshop, you will have the opportunity to implement your own AI models using pre-provided datasets. This live coding experience is designed to translate your theoretical knowledge from previous sessions into practical skills.
\end{frame}

\begin{frame}
    \frametitle{Objectives}
    \begin{enumerate}
        \item \textbf{Develop an Understanding of AI Model Implementation}: Put theory into practice through real coding scenarios.
        \item \textbf{Collaborative Learning}: Engage with peers and instructors to solve coding challenges and troubleshoot issues.
        \item \textbf{Hands-on Experience}: Gain experience using common libraries and frameworks for AI development.
    \end{enumerate}
\end{frame}

\begin{frame}[fragile]
    \frametitle{Key Concepts to Cover - Part 1}
    \begin{itemize}
        \item \textbf{Setting Up Your Environment}:
        \begin{itemize}
            \item Ensure you have the necessary tools installed, such as Python, Jupyter Notebook, and relevant libraries (e.g., NumPy, pandas, scikit-learn, TensorFlow/PyTorch).
            \item Configuration Example:
            \begin{lstlisting}[language=bash]
pip install numpy pandas scikit-learn tensorflow
            \end{lstlisting}
        \end{itemize}
        
        \item \textbf{Loading the Dataset}:
        \begin{itemize}
            \item Utilize pandas to load and explore the dataset.
            \item Example Code:
            \begin{lstlisting}[language=python]
import pandas as pd
data = pd.read_csv('dataset.csv')
print(data.head())
            \end{lstlisting}
        \end{itemize}
    \end{itemize}
\end{frame}

\begin{frame}[fragile]
    \frametitle{Key Concepts to Cover - Part 2}
    \begin{itemize}
        \item \textbf{Data Preprocessing}:
        \begin{itemize}
            \item Discuss techniques such as handling missing values, normalization, and feature encoding.
            \item Example: Filling missing values
            \begin{lstlisting}[language=python]
data.fillna(method='ffill', inplace=True)
            \end{lstlisting}
        \end{itemize}
        
        \item \textbf{Building the Model}:
        \begin{itemize}
            \item Introduce the principles of selecting an appropriate algorithm (e.g., regression, classification).
            \item Example: Implementing a simple linear regression model using scikit-learn.
            \begin{lstlisting}[language=python]
from sklearn.model_selection import train_test_split
from sklearn.linear_model import LinearRegression

# Separating features and target variables
X = data[['feature1', 'feature2']]
y = data['target']

# Train-test split
X_train, X_test, y_train, y_test = train_test_split(X, y, test_size=0.2, random_state=42)

# Creating and training the model
model = LinearRegression()
model.fit(X_train, y_train)
            \end{lstlisting}
        \end{itemize}
    \end{itemize}
\end{frame}

\begin{frame}[fragile]
    \frametitle{Key Concepts to Cover - Part 3}
    \begin{itemize}
        \item \textbf{Evaluating the Model}:
        \begin{itemize}
            \item Brief overview of evaluating model performance, including splitting data into training and testing sets, and using metrics like accuracy or RMSE (Root Mean Squared Error).
            \item Example:
            \begin{lstlisting}[language=python]
from sklearn.metrics import mean_squared_error

y_pred = model.predict(X_test)
rmse = mean_squared_error(y_test, y_pred, squared=False)
print(f'RMSE: {rmse}')
            \end{lstlisting}
        \end{itemize}
    \end{itemize}
\end{frame}

\begin{frame}
    \frametitle{Key Points to Emphasize}
    \begin{itemize}
        \item \textbf{Iterative Process}: Building AI models is an iterative process; you may need to tweak parameters and preprocess data multiple times.
        \item \textbf{Collaboration Is Key}: Don’t hesitate to ask questions or share your screen if you're stuck.
        \item \textbf{Learning from Errors}: Debugging is a learning opportunity; embrace it!
    \end{itemize}
\end{frame}

\begin{frame}
    \frametitle{Conclusion}
    This coding session is your chance to bring your ideas to life using AI models. Be prepared to explore, experiment, and most importantly, learn through doing! Feel free to ask questions or seek clarification at any step of the way. Let's get coding!
\end{frame}

\begin{frame}[fragile]
    \frametitle{Evaluating Model Performance - Overview}
    \begin{itemize}
        \item Evaluating model performance is crucial for understanding how well an AI model meets its goals.
        \item Various metrics are used depending on: 
        \begin{itemize}
            \item The model type (classification, regression, etc.)
            \item The nature of the task
        \end{itemize}
        \item This session covers key evaluation metrics, interpreting results, and improving model performance.
    \end{itemize}
\end{frame}

\begin{frame}[fragile]
    \frametitle{Evaluating Model Performance - Key Metrics}
    \textbf{1. Classification Metrics}
    \begin{itemize}
        \item \textbf{Accuracy:} 
        \begin{equation}
            \text{Accuracy} = \frac{\text{TP} + \text{TN}}{\text{TP} + \text{TN} + \text{FP} + \text{FN}}
        \end{equation}
        \begin{itemize}
            \item Example: Predicting 80 out of 100 test cases correctly gives 80\% accuracy.
        \end{itemize}
        
        \item \textbf{Precision:} 
        \begin{equation}
            \text{Precision} = \frac{\text{TP}}{\text{TP} + \text{FP}}
        \end{equation}
        \begin{itemize}
            \item Example: If a model predicted 30 positive cases and 25 were correct, $\text{Precision} \approx 0.83$.
        \end{itemize}

        \item \textbf{Recall:} 
        \begin{equation}
            \text{Recall} = \frac{\text{TP}}{\text{TP} + \text{FN}}
        \end{equation}
        \begin{itemize}
            \item Example: 25 correct out of 40 actual positives gives Recall = 0.625.
        \end{itemize}

        \item \textbf{F1 Score:} 
        \begin{equation}
            F1 = 2 \times \frac{\text{Precision} \times \text{Recall}}{\text{Precision} + \text{Recall}}
        \end{equation}
    \end{itemize}
\end{frame}

\begin{frame}[fragile]
    \frametitle{Evaluating Model Performance - Regression Metrics and Summary}
    \textbf{2. Regression Metrics}
    \begin{itemize}
        \item \textbf{Mean Absolute Error (MAE):} 
        \begin{equation}
            \text{MAE} = \frac{1}{n} \sum_{i=1}^{n} |y_i - \hat{y}_i|
        \end{equation}

        \item \textbf{Mean Squared Error (MSE):} 
        \begin{equation}
            \text{MSE} = \frac{1}{n} \sum_{i=1}^{n} (y_i - \hat{y}_i)^2
        \end{equation}

        \item \textbf{R-squared (\(R^2\)):} 
        \begin{equation}
            R^2 = 1 - \frac{\text{SS}_{\text{res}}}{\text{SS}_{\text{tot}}}
        \end{equation}
        \begin{itemize}
            \item Where $\text{SS}_{\text{res}}$ = sum of squares of residuals.
            \item $\text{SS}_{\text{tot}}$ = total sum of squares.
        \end{itemize}
    \end{itemize}

    \textbf{Key Takeaway:}
    \begin{itemize}
        \item Use appropriate metrics tailored for evaluation to guide improvements and refine model performance.
    \end{itemize}
\end{frame}

\begin{frame}[fragile]
    \frametitle{Ethical Considerations}
    \begin{block}{Overview of Ethical Implications in AI Models}
        AI models have significant ethical responsibilities that transform industries and society.
        Understanding these implications is crucial for all stakeholders.
    \end{block}
\end{frame}

\begin{frame}[fragile]
    \frametitle{Key Ethical Concerns - Bias Detection}
    \begin{enumerate}
        \item \textbf{Bias Detection}
            \begin{itemize}
                \item \textbf{Definition}: Bias occurs when algorithms produce systematically prejudiced results.
                \item \textbf{Example}: A hiring algorithm trained on predominantly male resumes may favor male candidates.
                \item \textbf{Methods for Detection}:
                    \begin{itemize}
                        \item \textbf{Data Auditing}: Analyze datasets for imbalances and assess fairness with statistical tests.
                        \item \textbf{Performance Evaluation}: Compare model results across different demographic groups.
                    \end{itemize}
            \end{itemize}
    \end{enumerate}
\end{frame}

\begin{frame}[fragile]
    \frametitle{Key Ethical Concerns - Accountability}
    \begin{enumerate}
        \setcounter{enumi}{1} % Continue numbering from the previous frame
        \item \textbf{Accountability in AI Applications}
            \begin{itemize}
                \item \textbf{Explanation}: Accountability refers to the obligation for model outcomes.
                \item \textbf{Example}: In an autonomous vehicle accident, where lies the responsibility?
                \item \textbf{Strategies for Enhancing Accountability}:
                    \begin{itemize}
                        \item \textbf{Clear Documentation}: Maintain records of model development and performance.
                        \item \textbf{Transparency Frameworks}: Guidelines to disclose model operations and decision-making.
                    \end{itemize}
            \end{itemize}
    \end{enumerate}
\end{frame}

\begin{frame}[fragile]
    \frametitle{Key Points and Conclusion}
    \begin{block}{Key Points to Emphasize}
        \begin{itemize}
            \item Importance of Bias Detection for equality.
            \item Responsibility of developers in mitigating ethical risks.
            \item Collaboration for establishing ethical guidelines is essential.
        \end{itemize}
    \end{block}
    
    \begin{block}{Conclusion}
        Understanding ethical considerations in AI is integral for responsible development.
        \textbf{Next Step}: In our upcoming group activity, we will discuss implementing these ethical principles.
    \end{block}
\end{frame}

\begin{frame}[fragile]
    \frametitle{Formulas for Assessing Bias}
    \begin{block}{Disparate Impact Ratio}
        A measure to evaluate if an AI model disproportionately affects certain groups:
        \begin{equation}
            \text{Disparate Impact Ratio} = \frac{P(\text{favorable outcome for group A})}{P(\text{favorable outcome for group B})}
        \end{equation}
        A ratio significantly different from 1 indicates potential bias.
    \end{block}
\end{frame}

\begin{frame}[fragile]
    \frametitle{Group Collaboration}
    Encourage teamwork and collaborative problem-solving during the workshop, emphasizing shared learning.
\end{frame}

\begin{frame}[fragile]
    \frametitle{Encouraging Teamwork in Model Building}
    Group collaboration is essential in the workshop as it fosters a dynamic environment of shared learning and collective problem-solving. 
    \begin{itemize}
        \item Enhances individual skills 
        \item Strengthens group synergy
    \end{itemize}
\end{frame}

\begin{frame}[fragile]
    \frametitle{Key Concepts}
    \begin{enumerate}
        \item \textbf{Importance of Teamwork}
            \begin{itemize}
                \item \textbf{Diverse Perspectives:} Unique skills lead to innovative solutions.
                \item \textbf{Shared Responsibility:} Workload distribution reduces individual pressure.
                \item \textbf{Enhanced Creativity:} Group brainstorming encourages creative ideas.
            \end{itemize}
        \item \textbf{Collaborative Problem-Solving}
            \begin{itemize}
                \item \textbf{Identification of Challenges:} Compile insights to efficiently identify issues.
                \item \textbf{Collective Brainstorming:} Explore solutions and reach consensus.
            \end{itemize}
    \end{enumerate}
\end{frame}

\begin{frame}[fragile]
    \frametitle{Examples and Illustrations}
    \textbf{Scenario:} Tasked with building a predictive model for housing prices.
    \begin{enumerate}
        \item \textbf{Step 1:} 
        \begin{itemize}
            \item Data gathering 
            \item Feature selection 
            \item Model evaluation 
        \end{itemize}
        \item \textbf{Step 2:} Regular check-ins to discuss findings.
    \end{enumerate}
    \textbf{Case Study:} Teams with effective communication saw a 25\% improvement in model accuracy.
\end{frame}

\begin{frame}[fragile]
    \frametitle{Key Points to Emphasize}
    \begin{itemize}
        \item \textbf{Effective Communication:} Use shared documents and chat platforms (e.g., Slack).
        \item \textbf{Conflict Resolution:} Open dialogue for constructive discussion.
        \item \textbf{Peer Feedback:} Regularly review each other's work for improvement.
        \item \textbf{Reflection and Learning:} Assess group dynamics post-workshop.
    \end{itemize}
\end{frame}

\begin{frame}[fragile]
    \frametitle{Collaboration Tools}
    \begin{itemize}
        \item \textbf{Project Management:} Trello or Asana for tasks and timelines.
        \item \textbf{Coding Collaboration:} GitHub for version control.
        \item \textbf{Virtual Collaboration:} Zoom or Microsoft Teams for discussions.
    \end{itemize}
\end{frame}

\begin{frame}[fragile]
    \frametitle{Conclusion}
    Collaboration is a mindset that leverages collective strengths. 
    \begin{itemize}
        \item Embrace teamwork for robust model building.
        \item Develop essential teamwork skills invaluable in real-world applications.
    \end{itemize}
\end{frame}

\begin{frame}[fragile]
    \frametitle{Wrap-Up and Q\&A}
    \begin{itemize}
        \item Summarize the key takeaways from the workshop
        \item Open the floor for:
        \begin{itemize}
            \item Questions
            \item Discussions
        \end{itemize}
    \end{itemize}
\end{frame}

\begin{frame}[fragile]
    \frametitle{Key Takeaways from the Workshop}
    \begin{enumerate}
        \item \textbf{Importance of Collaboration:} 
        \begin{itemize}
            \item Emphasized the role of teamwork in model building
            \item Diverse perspectives enhance problem-solving
        \end{itemize}
        
        \item \textbf{Model Building Process:} 
        \begin{itemize}
            \item Define the Problem
            \item Data Collection
            \item Model Selection
            \item Evaluation
            \item Iteration
        \end{itemize}
        
        \item \textbf{Tools and Techniques:}
        \begin{itemize}
            \item Python, R, Excel
            \item Key techniques: Data preprocessing, Feature selection, Cross-validation
        \end{itemize}
    \end{enumerate}
\end{frame}

\begin{frame}[fragile]
    \frametitle{Discussion Points for Q\&A}
    \begin{enumerate}
        \item \textbf{Challenges Faced:} 
        \begin{itemize}
            \item Share difficulties encountered 
            \item Encourage collective problem-solving
        \end{itemize}
        
        \item \textbf{Real-world Applications:}
        \begin{itemize}
            \item Discuss potential use cases in various fields
            \item Engage participants in sharing their applications
        \end{itemize}
        
        \item \textbf{Feedback on the Workshop:}
        \begin{itemize}
            \item Gather feedback for improvement
            \item Identify valuable components and areas to refine
        \end{itemize}
    \end{enumerate}
\end{frame}

\begin{frame}[fragile]
    \frametitle{Next Steps and Closing Thought}
    \begin{enumerate}
        \item \textbf{Further Learning Resources:} 
        \begin{itemize}
            \item Books, online courses, communities
        \end{itemize}
        
        \item \textbf{Networking Opportunities:} 
        \begin{itemize}
            \item Connect with peers beyond the workshop
            \item Consider participating in forums or study groups
        \end{itemize}
        
        \item \textbf{Closing Thought:} 
        \begin{itemize}
            \item Encourage a culture of inquiry and continual learning
            \item Emphasize the iterative nature of model building
        \end{itemize}
    \end{enumerate}
\end{frame}

\begin{frame}[fragile]
    \frametitle{Example: Simple Linear Regression}
    \begin{block}{Code Snippet}
    \begin{lstlisting}[language=Python]
import pandas as pd
from sklearn.model_selection import train_test_split
from sklearn.linear_model import LinearRegression

# Load the dataset
data = pd.read_csv('data.csv')
X = data[['feature1', 'feature2']]
y = data['target']

# Split into training and testing sets
X_train, X_test, y_train, y_test = train_test_split(X, y, test_size=0.2, random_state=42)

# Create and fit the model
model = LinearRegression()
model.fit(X_train, y_train)

# Evaluate the model
accuracy = model.score(X_test, y_test)
print('Model Accuracy:', accuracy)
    \end{lstlisting}
    \end{block}
\end{frame}


\end{document}