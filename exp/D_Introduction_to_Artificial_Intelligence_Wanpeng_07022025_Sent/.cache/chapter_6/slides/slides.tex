\documentclass{beamer}

% Theme choice
\usetheme{Madrid} % You can change to e.g., Warsaw, Berlin, CambridgeUS, etc.

% Encoding and font
\usepackage[utf8]{inputenc}
\usepackage[T1]{fontenc}

% Graphics and tables
\usepackage{graphicx}
\usepackage{booktabs}

% Code listings
\usepackage{listings}
\lstset{
    basicstyle=\ttfamily\small,
    keywordstyle=\color{blue},
    commentstyle=\color{gray},
    stringstyle=\color{red},
    breaklines=true,
    frame=single
}

% Math packages
\usepackage{amsmath}
\usepackage{amssymb}

% Colors
\usepackage{xcolor}

% TikZ and PGFPlots
\usepackage{tikz}
\usepackage{pgfplots}
\pgfplotsset{compat=1.18}
\usetikzlibrary{positioning}

% Hyperlinks
\usepackage{hyperref}

% Title information
\title{Week 6: Midterm Review and Exam}
\author{Your Name}
\institute{Your Institution}
\date{\today}

\begin{document}

\frame{\titlepage}

\begin{frame}[fragile]
    \frametitle{Introduction to Midterm Review}
    \begin{block}{Overview}
        Overview of the midterm exam format and its significance in assessing foundational knowledge in AI.
    \end{block}
\end{frame}

\begin{frame}[fragile]
    \frametitle{Midterm Exam Format}
    \begin{itemize}
        \item \textbf{Type of Questions:}
        \begin{itemize}
            \item Multiple-choice questions
            \item Short answer questions
            \item Practical coding exercises
        \end{itemize}
        \item \textbf{Difficulty Level:} 
        Questions range from basic recall to higher-order application.
        \item \textbf{Duration:} 
        You will have \textbf{2 hours} to complete the exam.
    \end{itemize}
\end{frame}

\begin{frame}[fragile]
    \frametitle{Significance of the Midterm Exam}
    \begin{itemize}
        \item \textbf{Assessment of Foundational Knowledge:} 
        Aims to gauge understanding of essential AI principles.
        \item \textbf{Feedback Mechanism:} 
        Offers insights into student performance and teaching effectiveness.
        \item \textbf{Preparation for Final Exam:} 
        Mastery of these concepts aids in preparation for final assessments.
    \end{itemize}
    \begin{block}{Key Points for Review}
        \begin{itemize}
            \item Review your notes and emphasize key concepts.
            \item Practice coding examples from class.
            \item Engage in group study for collaborative learning.
        \end{itemize}
    \end{block}
\end{frame}

\begin{frame}[fragile]
    \frametitle{Course Learning Objectives Recap - Introduction}
    As we prepare for the midterm exam, it's essential to revisit our course learning objectives. These objectives outline what you've learned and what to expect moving forward in the Introduction to Artificial Intelligence course.
\end{frame}

\begin{frame}[fragile]
    \frametitle{Course Learning Objectives Recap - Key Learning Objectives}
    \begin{enumerate}
        \item \textbf{Understanding AI Concepts}
        \begin{itemize}
            \item \textbf{Description:} Define Artificial Intelligence and its key components (Machine Learning, NLP).
            \item \textbf{Example:} Distinguish between "weak AI" (like chess) and "strong AI" (human-like cognition).
        \end{itemize}
        
        \item \textbf{Machine Learning Techniques}
        \begin{itemize}
            \item \textbf{Description:} Explore algorithms including supervised, unsupervised, and reinforcement learning.
            \item \textbf{Example:} 
            \begin{itemize}
                \item \textbf{Supervised Learning:} Regression and classification tasks (e.g., predicting house prices).
                \item \textbf{Unsupervised Learning:} Clustering methods (e.g., K-means for customer segmentation).
            \end{itemize}
        \end{itemize}
    \end{enumerate}
\end{frame}

\begin{frame}[fragile]
    \frametitle{Course Learning Objectives Recap - More Key Objectives}
    \begin{enumerate}[resume]
        \item \textbf{Deep Learning Fundamentals}
        \begin{itemize}
            \item \textbf{Description:} Understand principles of deep learning and neural networks, like architecture and backpropagation.
            \item \textbf{Illustration:} Basic neural network example with layers and data flow.
        \end{itemize}
        
        \item \textbf{Natural Language Processing (NLP)}
        \begin{itemize}
            \item \textbf{Description:} Key techniques in NLP (tokenization, stemming, sentiment analysis).
            \item \textbf{Example:} Analyzing customer reviews for sentiment using Python libraries.
        \end{itemize}
        
        \item \textbf{Ethical Considerations in AI}
        \begin{itemize}
            \item \textbf{Description:} Discuss ethical implications (bias, privacy, societal impacts).
            \item \textbf{Key Point:} AI systems must be developed responsibly to ensure fairness and minimize harm.
        \end{itemize}
    \end{enumerate}
\end{frame}

\begin{frame}[fragile]
    \frametitle{Course Learning Objectives Recap - Emphasis Points for Midterm}
    \begin{itemize}
        \item Familiarize yourself with definitions and examples from each learning objective.
        \item Be prepared to explain concepts clearly; understanding is key, not memorization.
        \item Review case studies discussed in class illustrating practical applications.
    \end{itemize}
    \vspace{1em}
    \textbf{Conclusion:} Reflect on each objective to prepare for the midterm exam. Clarity and understanding are your allies!
\end{frame}

\begin{frame}[fragile]
    \frametitle{AI Fundamentals - Part 1}
    % Introduction to Artificial Intelligence
    \begin{block}{Introduction to AI}
        Artificial Intelligence (AI) refers to the simulation of human intelligence processes by machines, especially computer systems. It encompasses:
        \begin{itemize}
            \item Problem-solving
            \item Learning
            \item Reasoning
            \item Perception
            \item Language understanding
        \end{itemize}
    \end{block}
\end{frame}

\begin{frame}[fragile]
    \frametitle{AI Fundamentals - Key Terms}
    % Key Terms for AI, ML, DL, and NLP
    \begin{enumerate}
        \item \textbf{Artificial Intelligence (AI):} 
            \begin{itemize}
                \item \textit{Definition:} Broad discipline encompassing all techniques and methods for making machines 'intelligent.'
                \item \textit{Example:} Robots performing tasks, AI chess-playing programs.
            \end{itemize}
        
        \item \textbf{Machine Learning (ML):}
            \begin{itemize}
                \item \textit{Definition:} A subset of AI focused on enabling machines to learn from data without being explicitly programmed.
                \item \textit{How it Works:} Algorithms analyze patterns in data to make predictions or decisions.
                \item \textit{Example:} Email spam filters that improve by learning from user feedback.
            \end{itemize}
        
        \item \textbf{Deep Learning (DL):}
            \begin{itemize}
                \item \textit{Definition:} A specialized form of ML that uses neural networks with many layers (deep neural networks).
                \item \textit{Key Point:} Handles vast amounts of unstructured data such as images and text.
                \item \textit{Example:} Image recognition systems used in social media platforms.
            \end{itemize}
        
        \item \textbf{Natural Language Processing (NLP):}
            \begin{itemize}
                \item \textit{Definition:} A branch of AI that focuses on the interaction between computers and humans through natural language.
                \item \textit{Application:} Includes text analysis, sentiment analysis, and language translation.
                \item \textit{Example:} Virtual assistants like Siri and chatbots that can understand and respond to human queries.
            \end{itemize}
    \end{enumerate}
\end{frame}

\begin{frame}[fragile]
    \frametitle{Differentiating Key Concepts}
    % Differentiating Key Concepts
    \begin{block}{AI vs. ML vs. DL}
        \begin{itemize}
            \item \textbf{AI:} The umbrella term for all intelligent processes.
            \item \textbf{ML:} A method/approach within AI.
            \item \textbf{DL:} A further specialization of ML that mimics human brain operations.
        \end{itemize}
    \end{block}
    
    \begin{block}{Recap Points}
        \begin{itemize}
            \item AI encompasses all intelligence-related processes.
            \item Machine Learning leverages data for improvement.
            \item Deep Learning focuses on neural networks.
            \item Natural Language Processing enables human language interaction.
        \end{itemize}
    \end{block}
\end{frame}

\begin{frame}[fragile]
    \frametitle{Critical Analysis of AI Applications}
    \begin{block}{Overview}
        Examine three case studies demonstrating AI applications in various industries and analyze their challenges and opportunities.
    \end{block}
\end{frame}

\begin{frame}[fragile]
    \frametitle{Case Study 1: Healthcare – Predictive Analytics in Patient Care}
    \begin{itemize}
        \item \textbf{Overview}: AI algorithms analyze patient data to predict health risks and improve patient outcomes.
        \item \textbf{Example}: IBM Watson Health uses AI to sift through vast amounts of medical literature and patient data to provide treatment recommendations.
    \end{itemize}

    \begin{block}{Challenges}
        \begin{itemize}
            \item Data privacy concerns: Ensuring patient data is secure and compliant with regulations such as HIPAA.
            \item Integration issues: Incorporating AI tools into existing healthcare systems can be cumbersome.
        \end{itemize}
    \end{block}

    \begin{block}{Opportunities}
        \begin{itemize}
            \item Improved diagnostic accuracy: AI can identify patterns that human doctors might miss.
            \item Personalized medicine: Tailoring treatments based on individual patient data can lead to better health outcomes.
        \end{itemize}
    \end{block}
\end{frame}

\begin{frame}[fragile]
    \frametitle{Case Study 2: Finance – Fraud Detection Systems}
    \begin{itemize}
        \item \textbf{Overview}: AI is employed to detect and prevent fraudulent transactions in real-time.
        \item \textbf{Example}: PayPal utilizes machine learning models to evaluate transactions for suspicious activity by analyzing behavioral patterns.
    \end{itemize}

    \begin{block}{Challenges}
        \begin{itemize}
            \item False positives: Legitimate transactions may be flagged, leading to customer dissatisfaction.
            \item Evolving tactics: Fraudsters continuously change their methods, requiring ongoing model updates.
        \end{itemize}
    \end{block}

    \begin{block}{Opportunities}
        \begin{itemize}
            \item Real-time monitoring: Immediate detection of fraud enhances security.
            \item Cost savings: Reducing losses from fraudulent activities improves financial performance.
        \end{itemize}
    \end{block}
\end{frame}

\begin{frame}[fragile]
    \frametitle{Case Study 3: Retail – Personalized Marketing}
    \begin{itemize}
        \item \textbf{Overview}: AI-driven algorithms assess customer behavior to create personalized shopping experiences.
        \item \textbf{Example}: Amazon uses recommendation systems to suggest products based on past purchases and browsing history.
    \end{itemize}

    \begin{block}{Challenges}
        \begin{itemize}
            \item Data dependency: Effectiveness relies on the availability of quality data.
            \item Ethical considerations: Balancing personalization with consumer privacy can be complex.
        \end{itemize}
    \end{block}

    \begin{block}{Opportunities}
        \begin{itemize}
            \item Increased sales: Targeted marketing can significantly boost conversion rates.
            \item Customer loyalty: Creating tailored experiences enhances customer satisfaction and retention.
        \end{itemize}
    \end{block}
\end{frame}

\begin{frame}[fragile]
    \frametitle{Key Takeaways and Conclusion}
    \begin{itemize}
        \item \textbf{Multi-faceted Nature of AI}: AI applications offer both significant advantages and daunting challenges across diverse industries.
        \item \textbf{Importance of Ethics and Compliance}: Careful consideration of data ethics and privacy regulations is crucial.
        \item \textbf{Continuous Improvement}: AI models require constant updates and fine-tuning to adapt to new challenges and maximize effectiveness.
    \end{itemize}

    \begin{block}{Conclusion}
        Understanding these case studies illustrates how AI transforms industries and the importance of strategically addressing challenges to fully harness its potential.
    \end{block}
\end{frame}

\begin{frame}
    \frametitle{Hands-on Experience with AI Tools}
    \begin{block}{Introduction to AI Tools}
        Explore three powerful AI frameworks: TensorFlow, Keras, and PyTorch. 
        Practical experience with these tools will enhance understanding and skills in implementing AI projects.
    \end{block}
\end{frame}

\begin{frame}
    \frametitle{Key AI Frameworks - TensorFlow}
    \begin{itemize}
        \item \textbf{Description}: Open-source library developed by Google for numerical computation and large-scale machine learning. Offers flexibility and scalability.
        \item \textbf{Use Case}: Ideal for deploying deep learning models in production due to its performance capabilities.
        \item \textbf{Practical Assignment}: Build and train a neural network to recognize handwritten digits using the MNIST dataset.
    \end{itemize}
    \begin{block}{Example Code}
    \begin{lstlisting}[language=Python]
    import tensorflow as tf
    from tensorflow.keras import layers, models

    model = models.Sequential()
    model.add(layers.Dense(128, activation='relu', input_shape=(784,)))
    model.add(layers.Dense(10, activation='softmax'))

    model.compile(optimizer='adam', loss='sparse_categorical_crossentropy', metrics=['accuracy'])
    \end{lstlisting}
    \end{block}
\end{frame}

\begin{frame}
    \frametitle{Key AI Frameworks - Keras \& PyTorch}
    \begin{itemize}
        \item \textbf{Keras}
            \begin{itemize}
                \item \textbf{Description}: High-level neural networks API in Python, runs on top of TensorFlow. 
                \item \textbf{Use Case}: Quickly prototype deep learning models.
                \item \textbf{Practical Assignment}: Create a CNN for classifying images from the CIFAR-10 dataset.
            \end{itemize}
        \item \textbf{PyTorch}
            \begin{itemize}
                \item \textbf{Description}: Known for flexibility, dynamic computational graphs, popular among researchers.
                \item \textbf{Use Case}: Suited for rapid experimentation or custom functions in NLP.
                \item \textbf{Practical Assignment}: Implement an RNN for text generation based on a dataset.
            \end{itemize}
    \end{itemize}
    \begin{block}{Example Code (Keras)}
    \begin{lstlisting}[language=Python]
    model = tf.keras.models.Sequential([
        layers.Conv2D(32, (3, 3), activation='relu', input_shape=(32, 32, 3)),
        layers.MaxPooling2D(pool_size=(2, 2)),
        layers.Flatten(),
        layers.Dense(64, activation='relu'),
        layers.Dense(10, activation='softmax')
    ])
    \end{lstlisting}
    \end{block}
    \begin{block}{Example Code (PyTorch)}
    \begin{lstlisting}[language=Python]
    import torch
    import torch.nn as nn

    class RNN(nn.Module):
        def __init__(self, input_size, hidden_size, output_size):
            super(RNN, self).__init__()
            self.rnn = nn.RNN(input_size, hidden_size)
            self.fc = nn.Linear(hidden_size, output_size)

        def forward(self, x):
            out, _ = self.rnn(x)
            out = self.fc(out[-1])
            return out
    \end{lstlisting}
    \end{block}
\end{frame}

\begin{frame}
    \frametitle{Key Points \& Conclusion}
    \begin{itemize}
        \item \textbf{Practical Skills}: Hands-on assignments solidify theoretical knowledge.
        \item \textbf{Versatility of Tools}: Each framework has strengths; choose based on project needs.
        \item \textbf{Community and Resources}: Vast communities for learning and troubleshooting all three frameworks.
    \end{itemize}
    \begin{block}{Conclusion}
        Hands-on experience with TensorFlow, Keras, and PyTorch prepares you for real-world AI applications. 
        Explore tutorials and documentation for deeper insights and additional functionalities!
    \end{block}
\end{frame}

\begin{frame}[fragile]
    \frametitle{Ethical Considerations in AI - Part 1}
    \begin{block}{Understanding Ethical Issues in AI}
        AI technologies have made remarkable advancements, but they also pose significant ethical challenges. It's essential to consider ethical frameworks to guide AI development and deployment.
    \end{block}
\end{frame}

\begin{frame}[fragile]
    \frametitle{Ethical Considerations in AI - Part 2}
    \begin{block}{Key Ethical Issues}
        \begin{itemize}
            \item \textbf{Bias}:
            \begin{itemize}
                \item AI systems can learn and perpetuate biases from training data.
                \item Example: A hiring algorithm may favor certain demographics based on unbalanced datasets.
            \end{itemize}
            
            \item \textbf{Accountability}:
            \begin{itemize}
                \item Complexities arise in determining responsibility for AI actions.
                \item Example: Accidents involving autonomous vehicles raise questions about liability.
            \end{itemize}
            
            \item \textbf{Societal Impact}:
            \begin{itemize}
                \item AI affects job markets, privacy, and decision-making processes.
                \item Example: Automation can lead to job displacement and economic inequality.
            \end{itemize}
        \end{itemize}
    \end{block}
\end{frame}

\begin{frame}[fragile]
    \frametitle{Ethical Considerations in AI - Part 3}
    \begin{block}{Key Points to Emphasize}
        \begin{itemize}
            \item Addressing bias is essential for fair and equitable AI systems.
            \item Establish accountability frameworks to manage AI's moral implications.
            \item Engage in proactive discussions about AI's societal impacts.
        \end{itemize}
    \end{block}

    \begin{block}{Additional Frameworks for Ethical AI}
        \begin{itemize}
            \item \textbf{Fairness}: Ensure equal treatment across all AI users.
            \item \textbf{Transparency}: Promote understandability and scrutiny of algorithms.
            \item \textbf{Privacy}: Respect user data and adhere to privacy regulations like GDPR.
        \end{itemize}
    \end{block}
\end{frame}

\begin{frame}[fragile]
    \frametitle{Ethical Considerations in AI - Conclusion}
    \begin{block}{Conclusion}
        As AI technologies advance, it's crucial to remain vigilant about ethical considerations. Incorporating ethical frameworks into AI design enhances trust and leads to socially beneficial outcomes.
    \end{block}
    
    \begin{block}{Discussion Points for Class}
        \begin{itemize}
            \item Can you think of other real-world examples where AI has introduced ethical dilemmas?
            \item How might we design systems to minimize bias and improve accountability in AI?
        \end{itemize}
    \end{block}
\end{frame}

\begin{frame}[fragile]
    \frametitle{Collaborative Problem-Solving - Teamwork in AI}
    \begin{block}{Importance of Teamwork}
        Collaboration is essential in AI development due to its complexity and multidisciplinary nature.
        Successful projects require expertise from data science, software engineering, design, and domain knowledge.
    \end{block}
\end{frame}

\begin{frame}[fragile]
    \frametitle{Collaborative Problem-Solving - Key Points}
    \begin{enumerate}
        \item \textbf{Diverse Skill Sets}
        \begin{itemize}
            \item Data Scientists analyze and model data.
            \item Software Engineers create and optimize algorithms.
            \item Domain Experts ensure relevance to the target industry.
        \end{itemize}
        
        \item \textbf{Enhanced Creativity}
        \begin{itemize}
            \item Team collaboration fosters innovation.
            \item Example: Brainstorming sessions lead to novel approaches.
        \end{itemize}
        
        \item \textbf{Shared Responsibility}
        \begin{itemize}
            \item Teamwork distributes tasks, managing workload and reducing stress.
            \item Example: In healthcare AI, team roles focus on privacy and algorithms.
        \end{itemize}
        
        \item \textbf{Problem-Solving Efficiency}
        \begin{itemize}
            \item Teams solve complex problems faster due to diverse perspectives.
            \item Example: Collaborative reviews unveil data issues quickly.
        \end{itemize}
    \end{enumerate}
\end{frame}

\begin{frame}[fragile]
    \frametitle{Collaborative Problem-Solving - Agile Methodologies}
    \begin{block}{Project Management in AI}
        Agile methodologies enhance collaboration and adaptability in AI development.
    \end{block}
    
    \begin{itemize}
        \item \textbf{Agile Framework}
        \begin{itemize}
            \item Sprints for focused goals.
            \item Daily stand-up meetings for alignment.
        \end{itemize}

        \item \textbf{User Feedback}
        \begin{itemize}
            \item Frequent work increments gather user feedback early.
            \item Example: Chatbot development adjusts design based on user needs.
        \end{itemize}
        
        \item \textbf{Iterative Testing}
        \begin{itemize}
            \item Continuous integration leads to reliable AI solutions.
            \item Bugs are addressed swiftly during development cycles.
        \end{itemize}
    \end{itemize}
\end{frame}

\begin{frame}[fragile]
    \frametitle{Collaborative Problem-Solving - Conclusion}
    \begin{block}{Conclusion}
        Effective teamwork and agile methodologies in AI projects maximize innovation and efficiency.
        Being adaptable and cohesive is crucial in the fast-evolving AI landscape.
    \end{block}

    \begin{block}{Final Note}
        Encourage participation in discussions and exercises that simulate collaborative problem-solving in AI.
        This fosters real-world application of these concepts.
    \end{block}
\end{frame}

\begin{frame}[fragile]
    \frametitle{Research Literacy in AI - Overview}
    Research literacy in AI is crucial for understanding the rapidly evolving landscape of artificial intelligence. 
    This competency encompasses:
    \begin{itemize}
        \item Locating credible sources
        \item Assessing quality and reliability of information
        \item Synthesizing information to support informed discussions
    \end{itemize}
\end{frame}

\begin{frame}[fragile]
    \frametitle{Research Literacy in AI - Identifying Reputable Sources}
    Finding reliable and high-quality literature is essential. Key sources to consider include:
    \begin{itemize}
        \item \textbf{Peer-Reviewed Journals}: Such as *Journal of Artificial Intelligence Research* and *IEEE Transactions on Neural Networks*.
        \item \textbf{Academic Conferences}: Proceedings from events like NeurIPS and ICML showcase the latest research.
        \item \textbf{Books and Textbooks}: Use established authors and publishers for comprehensive insights.
        \item \textbf{Governmental and Institutional Reports}: Agencies like NIST publish guidelines and standards.
    \end{itemize}
    \begin{block}{Example}
        For "Machine Learning for Predictive Analytics," search for recent articles in reputable journals.
    \end{block}
\end{frame}

\begin{frame}[fragile]
    \frametitle{Research Literacy in AI - Synthesizing Information}
    To synthesize information meaningfully:
    \begin{itemize}
        \item \textbf{Summarization}: Extract key findings and identify patterns.
        \item \textbf{Critical Thinking}: Assess validity, reliability, and relevance of findings.
        \item \textbf{Comparative Analysis}: Create visuals like tables or charts for comparison.
    \end{itemize}
    \begin{block}{Example}
        \begin{tabular}{|c|c|c|}
            \hline
            \textbf{Study} & \textbf{Purpose} & \textbf{Key Finding} \\
            \hline
            Study A & AI in diagnostics & 85$\%$ accuracy in cancer detection \\
            Study B & AI in patient management & Reduced readmissions by 20$\%$ \\
            Study C & AI in health monitoring & Improved engagement rates by 30$\%$ \\
            \hline
        \end{tabular}
    \end{block}
\end{frame}

\begin{frame}[fragile]
    \frametitle{Research Literacy in AI - Key Points and Conclusion}
    \begin{itemize}
        \item Emphasize the use of \textbf{peer-reviewed and high-quality sources}.
        \item Note that \textbf{synthesis} involves critiquing and relating sources.
        \item Stay updated with current literature for the latest advancements.
    \end{itemize}
    \begin{block}{Conclusion}
        Mastering research literacy in AI enables effective participation in discussions about AI technologies and their societal implications.
    \end{block}
    \begin{block}{Tips for Success}
        \begin{itemize}
            \item Maintain a \textbf{Research Log} to document your process.
            \item Use tools like \textbf{Zotero} or \textbf{Mendeley} for managing citations.
        \end{itemize}
    \end{block}
\end{frame}

\begin{frame}[fragile]
    \frametitle{Preparing for the Midterm Exam - Introduction}
    \begin{block}{Overview}
        Midterm exams can be a significant milestone in your academic journey. 
        Preparation is key to doing well, and adopting strategic study techniques can greatly enhance your understanding and retention of material covered throughout the course.
    \end{block}
\end{frame}

\begin{frame}[fragile]
    \frametitle{Preparing for the Midterm Exam - Key Strategies}
    \begin{enumerate}
        \item \textbf{Create a Study Schedule}
            \begin{itemize}
                \item Break It Down: Allocate specific times for studying different topics.
                \item Prioritize Topics: Focus on less confident areas first.
            \end{itemize}

        \item \textbf{Utilize Review Resources}
            \begin{itemize}
                \item Lecture Notes: Highlight key concepts.
                \item Textbook Summaries: Review chapter summaries and key points.
                \item Online Resources: Use platforms like Khan Academy for additional help.
            \end{itemize}

        \item \textbf{Group Study Sessions}
            \begin{itemize}
                \item Collaborate with Peers: Discuss concepts and quiz each other.
            \end{itemize}
    \end{enumerate}
\end{frame}

\begin{frame}[fragile]
    \frametitle{Preparing for the Midterm Exam - Example Study Plan}
    \begin{table}[h!]
        \centering
        \begin{tabular}{|l|l|l|}
            \hline
            \textbf{Day} & \textbf{Topics to Study} & \textbf{Activities} \\
            \hline
            Day 1 & AI Ethics and Applications  & Review notes, read chapter, discuss in study group \\
            Day 2 & Machine Learning Basics     & Solve practice questions \\
            Day 3 & Data Analysis Techniques    & Create flashcards for key terms \\
            Day 4 & Review All Topics          & Group discussion, timed quiz \\
            Day 5 & Relax and Self-Review      & Light reviewing and rest \\
            \hline
        \end{tabular}
    \end{table}
\end{frame}

\begin{frame}[fragile]
    \frametitle{Q\&A Session - Overview}
    % Overview of the Q&A session to address questions and clarify concepts
    \begin{block}{Overview}
        This session provides an opportunity for students to ask questions about the material covered in the midterm review. Engaging in dialogue strengthens understanding and clarifies any lingering doubts.
    \end{block}

    \begin{block}{Key Objectives}
        \begin{itemize}
            \item Address specific student concerns.
            \item Clarify concepts that may be confusing.
            \item Reinforce understanding of key topics in preparation for the midterm.
        \end{itemize}
    \end{block}
\end{frame}

\begin{frame}[fragile]
    \frametitle{Q\&A Session - Common Topics}
    % Discuss common topics for Q&A session
    \begin{block}{Common Topics to Discuss}
        \begin{enumerate}
            \item \textbf{Review Strategies}
                \begin{itemize}
                    \item Importance of active engagement while studying.
                    \item Utilization of practice questions and example problems.
                \end{itemize}
                
            \item \textbf{Concept Integration}
                \begin{itemize}
                    \item How different topics relate to each other.
                    \item Applications of theoretical concepts in practical scenarios.
                \end{itemize}

            \item \textbf{Exam Format}
                \begin{itemize}
                    \item Types of questions: multiple-choice, short answer, problem-solving.
                    \item Time management tips for each section of the exam.
                \end{itemize}

            \item \textbf{Concept Clarification}
                \begin{itemize}
                    \item Key formulas or concepts needing further explanation (e.g., the Pythagorean theorem: \( a^2 + b^2 = c^2 \)).
                    \item Examples of common misunderstandings (e.g., distinguishing between mean, median, and mode).
                \end{itemize}
        \end{enumerate}
    \end{block}
\end{frame}

\begin{frame}[fragile]
    \frametitle{Q\&A Session - Participation and Conclusion}
    % Encouragement for questions and session conclusion
    \begin{block}{Example Questions to Consider}
        \begin{itemize}
            \item "Can you explain how to apply [specific formula] in an exam question?"
            \item "What are the most important concepts to focus on based on what we've covered?"
            \item "How do I best manage my time during the midterm exam?"
        \end{itemize}
    \end{block}

    \begin{block}{Encouragement for Participation}
        \begin{itemize}
            \item Think about your questions beforehand.
            \item No question is too small; if it’s on your mind, it’s valid!
            \item Engage with peers to discuss questions they might have as well.
        \end{itemize}
    \end{block}

    \begin{block}{Conclusion}
        This Q\&A session aims to empower you with knowledge and confidence as you approach the midterm exam. Remember, asking questions is a vital part of the learning process!
    \end{block}

    \begin{block}{Reminder}
        Write down your questions and actively participate in the session! Your engagement will not only help you but also your classmates.
    \end{block}
\end{frame}


\end{document}