\documentclass{beamer}

% Theme choice
\usetheme{Madrid} % You can change to e.g., Warsaw, Berlin, CambridgeUS, etc.

% Encoding and font
\usepackage[utf8]{inputenc}
\usepackage[T1]{fontenc}

% Graphics and tables
\usepackage{graphicx}
\usepackage{booktabs}

% Code listings
\usepackage{listings}
\lstset{
basicstyle=\ttfamily\small,
keywordstyle=\color{blue},
commentstyle=\color{gray},
stringstyle=\color{red},
breaklines=true,
frame=single
}

% Math packages
\usepackage{amsmath}
\usepackage{amssymb}

% Colors
\usepackage{xcolor}

% TikZ and PGFPlots
\usepackage{tikz}
\usepackage{pgfplots}
\pgfplotsset{compat=1.18}
\usetikzlibrary{positioning}

% Hyperlinks
\usepackage{hyperref}

% Title information
\title{Week 8: Frameworks for AI Development}
\author{Your Name}
\institute{Your Institution}
\date{\today}

\begin{document}

\frame{\titlepage}

\begin{frame}[fragile]
    \frametitle{Introduction to AI Development Frameworks}
    \begin{block}{Overview of the Importance of Frameworks in AI Development}
        AI Development Frameworks are structured tools and libraries that facilitate the creation, training, evaluation, and deployment of artificial intelligence models. They streamline complex tasks, enabling developers to focus on problem-solving rather than managing underlying complexities.
    \end{block}
\end{frame}

\begin{frame}[fragile]
    \frametitle{Importance of AI Frameworks - Part 1}
    \begin{enumerate}
        \item \textbf{Simplification of Complex Processes:}
            \begin{itemize}
                \item AI development involves multiple stages including data preparation, model selection, training, testing, and deployment.
                \item Frameworks provide pre-defined functions and libraries, making development more efficient.
                \item \textit{Example:} Instead of coding a neural network from scratch, use TensorFlow or PyTorch.
            \end{itemize}
        
        \item \textbf{Standardization:}
            \begin{itemize}
                \item Frameworks promote consistency in AI model building and evaluation.
                \item Facilitates easier collaboration and code sharing among teams.
                \item \textit{Illustration:} Code is understandable and modifiable by different team members.
            \end{itemize}
    \end{enumerate}
\end{frame}

\begin{frame}[fragile]
    \frametitle{Importance of AI Frameworks - Part 2}
    \begin{enumerate}
        \setcounter{enumi}{2}
        \item \textbf{Access to Cutting-Edge Techniques:}
            \begin{itemize}
                \item Frameworks come with the latest algorithms and models.
                \item \textit{Example:} Keras provides easy access to CNNs and RNNs for quick experimentation.
            \end{itemize}

        \item \textbf{Performance Optimization:}
            \begin{itemize}
                \item Optimized for performance, leveraging hardware acceleration (e.g., GPUs).
                \item Crucial for handling large datasets and complex architectures.
                \item \textit{Code Snippet:}
                \begin{lstlisting}[language=Python]
                import tensorflow as tf

                # Enable GPU acceleration
                physical_devices = tf.config.list_physical_devices('GPU')
                tf.config.experimental.set_memory_growth(physical_devices[0], True)
                \end{lstlisting}
            \end{itemize}
        
        \item \textbf{Ecosystem and Community Support:}
            \begin{itemize}
                \item Extensive documentation and tutorials available.
                \item Community support facilitates troubleshooting and accelerates learning.
                \item Engaging with the community leads to discovering best practices and innovative solutions.
            \end{itemize}
    \end{enumerate}
\end{frame}

\begin{frame}[fragile]
    \frametitle{Key Takeaways}
    \begin{itemize}
        \item AI development frameworks simplify and streamline the model creation process.
        \item They provide a standardized approach, enhance collaboration, and promote code sharing.
        \item Frameworks enable access to cutting-edge techniques and optimize performance through hardware acceleration.
        \item Community support significantly enhances the development experience and outcomes.
    \end{itemize}
\end{frame}

\begin{frame}[fragile]
    \frametitle{What are AI Frameworks?}
    \begin{block}{Definition of AI Frameworks}
        AI frameworks are software libraries designed to simplify the process of developing artificial intelligence (AI) models. They provide a structured approach that includes pre-built components, tools, and resources for building, training, and deploying machine learning algorithms and neural networks.
    \end{block}
    
    \begin{block}{Role in Simplifying AI Model Development}
        AI frameworks allow developers to focus on model design and experimentation rather than spending excessive time coding from scratch. They abstract complex tasks and handle low-level details, thereby accelerating the development cycle.
    \end{block}
\end{frame}

\begin{frame}[fragile]
    \frametitle{Key Features of AI Frameworks}
    \begin{enumerate}
        \item \textbf{Pre-built Functions and Modules}
        \begin{itemize}
            \item AI frameworks come with various pre-built components, such as neural network layers, loss functions, and optimizers, easily utilized to build complex models.
            \item Example: In TensorFlow, creating a simple neural network can be achieved with just a few lines of code.
        \end{itemize}
        
        \begin{lstlisting}[language=Python]
import tensorflow as tf

model = tf.keras.models.Sequential([
    tf.keras.layers.Dense(64, activation='relu', input_shape=(32,)),
    tf.keras.layers.Dense(10, activation='softmax')
])
        \end{lstlisting}
        
        \item \textbf{Consistent APIs}
        \begin{itemize}
            \item Standard interfaces make it easier for developers to switch between models and frameworks without needing to relearn the entire API.
        \end{itemize}
        
        \item \textbf{Community and Documentation}
        \begin{itemize}
            \item Supported by vast communities that provide extensive documentation, tutorials, and forums for troubleshooting and knowledge sharing.
        \end{itemize}
        
        \item \textbf{Support for Multiple Platforms}
        \begin{itemize}
            \item Cross-platform support enables deployment to cloud services, edge devices, and more.
        \end{itemize}
    \end{enumerate}
\end{frame}

\begin{frame}[fragile]
    \frametitle{Key Points and Conclusion}
    \begin{itemize}
        \item \textbf{Efficiency}
        \begin{itemize}
            \item Using an AI framework can dramatically reduce time spent on development.
        \end{itemize}

        \item \textbf{Accessibility}
        \begin{itemize}
            \item They make advanced AI technologies accessible to developers without a deep background in machine learning, promoting innovation.
        \end{itemize}

        \item \textbf{Rapid Prototyping}
        \begin{itemize}
            \item Developers can quickly test ideas and iterate on designs, facilitating innovation and experimentation.
        \end{itemize}
    \end{itemize}

    \begin{block}{Conclusion}
        AI frameworks play a pivotal role in the evolution of AI development by streamlining processes, enhancing productivity, and providing powerful tools to both novice and experienced developers. They not only simplify the technical aspects of building AI models but also foster a collaborative environment for sharing knowledge and best practices.
    \end{block}
\end{frame}

\begin{frame}[fragile]
    \frametitle{Key Benefits of Using AI Frameworks - Part 1}
    \begin{block}{1. Ease of Use}
        AI frameworks are designed to simplify the development process. They provide intuitive APIs and pre-built functions that allow developers to build AI models without dealing with the underlying complexities of the algorithms.
        
        \begin{itemize}
            \item \textbf{Example:}
            \begin{itemize}
                \item \textbf{Scikit-learn}: This Python library offers a simple interface for classification, regression, and clustering tasks. For instance, a beginner can easily implement a decision tree model with just a few lines of code:
                \begin{lstlisting}[language=Python]
from sklearn.tree import DecisionTreeClassifier
model = DecisionTreeClassifier()
model.fit(X_train, y_train)
predictions = model.predict(X_test)
                \end{lstlisting}
            \end{itemize}
        \end{itemize}
    \end{block}
\end{frame}

\begin{frame}[fragile]
    \frametitle{Key Benefits of Using AI Frameworks - Part 2}
    \begin{block}{2. Rapid Prototyping}
        Using AI frameworks accelerates the prototype development process. With pre-defined structures and components, developers can quickly create and test models.
        
        \begin{itemize}
            \item \textbf{Example:}
            \begin{itemize}
                \item \textbf{Keras}: Known for its user-friendly approach, Keras allows for the rapid construction of neural networks. Developers can define a neural network in a few straightforward steps:
                \begin{lstlisting}[language=Python]
from keras.models import Sequential
from keras.layers import Dense

model = Sequential()
model.add(Dense(64, activation='relu', input_shape=(input_dim,)))
model.add(Dense(1, activation='sigmoid'))
model.compile(optimizer='adam', loss='binary_crossentropy', metrics=['accuracy'])
                \end{lstlisting}
            \end{itemize}
        \end{itemize}
    \end{block}
\end{frame}

\begin{frame}[fragile]
    \frametitle{Key Benefits of Using AI Frameworks - Part 3}
    \begin{block}{3. Community Support}
        Most popular AI frameworks have extensive community support. This means that developers can find solutions to common problems and share knowledge through forums, documentation, and tutorials.
        
        \begin{itemize}
            \item \textbf{Example:}
            \begin{itemize}
                \item \textbf{TensorFlow}: With a large user base, TensorFlow boasts a wealth of resources. From detailed API documentation to community-driven platforms like Stack Overflow, users can easily access support and examples that enhance their learning experience.
            \end{itemize}
        \end{itemize}
    \end{block}
    
    \begin{block}{Key Points to Emphasize}
        \begin{itemize}
            \item \textbf{Accessibility}: Frameworks lower the barrier to entry for newcomers to AI development.
            \item \textbf{Productivity}: They allow for faster iteration over models, enabling developers to focus on refining algorithms rather than building from scratch.
            \item \textbf{Collaboration}: Communities around these frameworks encourage collaboration and sharing of best practices.
        \end{itemize}
    \end{block}
    
    \begin{block}{Summary}
        Incorporating AI frameworks into model development provides significant advantages such as ease of use, rapid prototyping, and robust community support. These benefits empower developers of all skill levels to innovate and efficiently build sophisticated AI systems.
    \end{block}
\end{frame}

\begin{frame}
    \frametitle{Overview of TensorFlow}
    % Slide description
    TensorFlow is an open-source machine learning framework developed by Google. It allows developers to build a range of machine learning models from simple linear models to complex neural networks.
\end{frame}

\begin{frame}
    \frametitle{Key Features of TensorFlow}
    \begin{enumerate}
        \item \textbf{Flexibility:}
        \begin{itemize}
            \item Accommodates various workflows with high-level APIs and low-level operations.
            \item Models can be trained and deployed on-premises or in the cloud.
        \end{itemize}
        
        \item \textbf{Data Flow Graphs:}
        \begin{itemize}
            \item Utilizes data flow graphs for efficient numerical computations.
            \item Nodes represent operations, edges represent tensors.
        \end{itemize}

        \item \textbf{Ecosystem and Community:}
        \begin{itemize}
            \item Supports TensorBoard for visualization, TensorFlow Hub for components.
            \item Strength in community, documentation, and tutorials.
        \end{itemize}

        \item \textbf{Scalability:}
        \begin{itemize}
            \item Optimized performance on CPUs and GPUs for complex tasks.
        \end{itemize}
    \end{enumerate}
\end{frame}

\begin{frame}[fragile]
    \frametitle{Example: TensorFlow Code for Linear Transformation}
    % Code snippet example
    \begin{lstlisting}[language=Python]
import tensorflow as tf

# Define a simple computation
W = tf.Variable([[2.0]])
b = tf.Variable([[3.0]])
x = tf.constant([[1.0]])
y = tf.add(tf.matmul(x, W), b)  # y = Wx + b
    \end{lstlisting}
\end{frame}

\begin{frame}
    \frametitle{Common Use Cases in AI Development}
    \begin{itemize}
        \item \textbf{Deep Learning:}
        \begin{itemize}
            \item Image recognition and natural language processing using deep learning models like CNNs.
        \end{itemize}

        \item \textbf{Reinforcement Learning:}
        \begin{itemize}
            \item Developing AI agents for decision-making in game development.
        \end{itemize}

        \item \textbf{Predictive Analytics:}
        \begin{itemize}
            \item Detecting patterns and making predictions in finance, healthcare, etc.
        \end{itemize}

        \item \textbf{Generative Models:}
        \begin{itemize}
            \item Creating models such as GANs to generate new content based on training data.
        \end{itemize}
    \end{itemize}
\end{frame}

\begin{frame}
    \frametitle{Key Points to Emphasize}
    \begin{itemize}
        \item Versatility of TensorFlow supports beginners and advanced users.
        \item Abundant community resources accelerate development and troubleshooting.
        \item Learning TensorFlow provides essential skills for careers in AI and data science.
    \end{itemize}
\end{frame}

\begin{frame}
    \frametitle{Next Steps}
    % Transition to next section
    As we move forward, we will cover the initial steps for installing TensorFlow and setting up your first model, which will empower you to start experimenting with this powerful framework.
\end{frame}

\begin{frame}[fragile]
    \frametitle{Getting Started with TensorFlow - Introduction}
    \begin{block}{Introduction to TensorFlow}
        TensorFlow is an open-source library developed by Google for numerical computation that makes machine learning faster and easier. It is specifically designed for building machine learning and deep learning models.
    \end{block}
    \begin{itemize}
        \item Explore the basic installation process of TensorFlow
        \item Set up your first model
    \end{itemize}
\end{frame}

\begin{frame}[fragile]
    \frametitle{Getting Started with TensorFlow - Installation}
    \begin{block}{1. Installation of TensorFlow}
        To use TensorFlow, the first step is to install it. This can be done using pip:
    \end{block}
    \begin{itemize}
        \item Open your command line interface (Terminal or Command Prompt)
        \item Run the following command:
            \begin{lstlisting}
# For stable version
pip install tensorflow

# If you want GPU support:
pip install tensorflow-gpu
            \end{lstlisting}
        \item Key Points:
        \begin{itemize}
            \item Ensure Python (3.6 or later) is installed
            \item Consider using a virtual environment to manage libraries
        \end{itemize}
    \end{itemize}
\end{frame}

\begin{frame}[fragile]
    \frametitle{Getting Started with TensorFlow - Setting Up Your First Model}
    \begin{block}{2. Setting Up Your First Model}
        Here's how to create a simple sequential model:
    \end{block}
    
    \begin{enumerate}
        \item Import Libraries:
            \begin{lstlisting}
import tensorflow as tf
from tensorflow import keras
            \end{lstlisting}
        \item Load a Dataset:
            \begin{lstlisting}
mnist = keras.datasets.mnist
(x_train, y_train), (x_test, y_test) = mnist.load_data()
            \end{lstlisting}
        \item Preprocess the Data:
            \begin{lstlisting}
x_train = x_train / 255.0
x_test = x_test / 255.0
            \end{lstlisting}
        \item Build the Model:
            \begin{lstlisting}
model = keras.Sequential([
    keras.layers.Flatten(input_shape=(28, 28)),  # Flatten images
    keras.layers.Dense(128, activation='relu'),  # Hidden layer
    keras.layers.Dense(10, activation='softmax')  # Output layer
])
            \end{lstlisting}
        \item Compile the Model:
            \begin{lstlisting}
model.compile(optimizer='adam',
              loss='sparse_categorical_crossentropy',
              metrics=['accuracy'])
            \end{lstlisting}
        \item Train the Model:
            \begin{lstlisting}
model.fit(x_train, y_train, epochs=5)
            \end{lstlisting}
        \item Evaluate the Model:
            \begin{lstlisting}
test_loss, test_acc = model.evaluate(x_test, y_test)
print('Test accuracy:', test_acc)
            \end{lstlisting}
    \end{enumerate}
\end{frame}

\begin{frame}[fragile]
    \frametitle{Getting Started with TensorFlow - Summary}
    \begin{block}{Summary of Key Points}
        \begin{itemize}
            \item **Installation**: Use pip for TensorFlow installation and consider GPU support.
            \item **Model Creation**: Utilize sequential models for structure, flatten inputs, and define output layers.
            \item **Training \& Evaluation**: Use \texttt{.fit} to train and \texttt{.evaluate} to gauge model performance.
        \end{itemize}
    \end{block}
    By following these steps, you can efficiently set up your first machine learning model using TensorFlow!
\end{frame}

\begin{frame}
    \frametitle{Introduction to PyTorch}
    \begin{itemize}
        \item Overview of PyTorch
        \item Key Features
        \item Comparative Advantage over TensorFlow
    \end{itemize}
\end{frame}

\begin{frame}
    \frametitle{Overview of PyTorch}
    \begin{block}{Definition}
        PyTorch is an open-source machine learning framework developed by Facebook's AI Research lab. It's widely used in academia and industry for deep learning applications due to its flexibility and ease of use.
    \end{block}
\end{frame}

\begin{frame}
    \frametitle{Key Features of PyTorch - Part 1}
    \begin{enumerate}
        \item \textbf{Dynamic Computation Graphs:}
        \begin{itemize}
            \item Built at runtime, allowing for intuitive and debuggable code.
            \item \textit{Example:} Modify neural network architecture on-the-fly based on input.
        \end{itemize}

        \item \textbf{Tensor Library:}
        \begin{itemize}
            \item Provides high-performance operations on GPU.
            \item \textit{Illustration:}
            \begin{lstlisting}[language=Python]
import torch

# Creating a tensor
x = torch.tensor([[1, 2], [3, 4]])
print(x)
            \end{lstlisting}
        \end{itemize}
    \end{enumerate}
\end{frame}

\begin{frame}
    \frametitle{Key Features of PyTorch - Part 2}
    \begin{enumerate}[start=3]
        \item \textbf{Ecosystem and Libraries:}
        \begin{itemize}
            \item Integrates with libraries like torchvision and torchaudio for various applications.
        \end{itemize}

        \item \textbf{Community Support:}
        \begin{itemize}
            \item A vibrant community provides numerous tutorials, forums, and resources.
        \end{itemize}
    \end{enumerate}
\end{frame}

\begin{frame}
    \frametitle{Comparative Advantage Over TensorFlow}
    \begin{itemize}
        \item \textbf{Ease of Use:} More Pythonic API, simpler for Python users.
        \item \textbf{Flexibility:} Dynamic computation graphs allow for runtime architecture changes.
        \item \textbf{Debugging:} Supports Python debuggers, easier to inspect variables.
        \item \textbf{Performance:} Often faster for straightforward tasks, less overhead in graph handling.
    \end{itemize}
\end{frame}

\begin{frame}
    \frametitle{Conclusion}
    \begin{itemize}
        \item PyTorch is highly flexible and user-friendly, gaining popularity for tasks that require iterative experimentation.
        \item Explore PyTorch for enhanced AI development capabilities.
    \end{itemize}
\end{frame}

\begin{frame}
    \frametitle{Next Steps}
    \begin{itemize}
        \item Check the next slide for a hands-on approach to getting started with PyTorch.
        \item Topics include installation and creating your first model.
    \end{itemize}
\end{frame}

\begin{frame}[fragile]
    \frametitle{Getting Started with PyTorch}
    % Overview of PyTorch
    \begin{itemize}
        \item Open-source deep learning framework
        \item Flexible and efficient for building neural networks
        \item Dynamic computation graph simplifies model development
        \item Ideal for researchers and practitioners
    \end{itemize}
\end{frame}

\begin{frame}[fragile]
    \frametitle{Installation of PyTorch}
    \begin{enumerate}
        \item \textbf{Check your Python version}:
            \begin{lstlisting}
            python --version
            \end{lstlisting}
            Ensure Python version >= 3.6
        \item \textbf{Install Using pip}:
            \begin{lstlisting}
            pip install torch torchvision torchaudio
            \end{lstlisting}
            \begin{itemize}
                \item \texttt{torch}: Core PyTorch package
                \item \texttt{torchvision}: Datasets, model architectures, and image transformations
                \item \texttt{torchaudio}: Audio processing capabilities
            \end{itemize}
        \item \textbf{Verify Installation}:
            \begin{lstlisting}
            import torch
            print(torch.__version__)
            \end{lstlisting}
    \end{enumerate}
\end{frame}

\begin{frame}[fragile]
    \frametitle{Creating a Simple Model with PyTorch}
    \begin{enumerate}
        \item \textbf{Import Necessary Libraries}:
            \begin{lstlisting}
            import torch
            import torch.nn as nn
            import torch.optim as optim
            \end{lstlisting}
        \item \textbf{Define the Model}:
            \begin{lstlisting}
            class SimpleNN(nn.Module):
                def __init__(self):
                    super(SimpleNN, self).__init__()
                    self.fc1 = nn.Linear(10, 5)  # Input to hidden
                    self.fc2 = nn.Linear(5, 1)    # Hidden to output

                def forward(self, x):
                    x = torch.relu(self.fc1(x))
                    x = self.fc2(x)
                    return x
            \end{lstlisting}
        \item \textbf{Instantiate and Use the Model}:
            \begin{lstlisting}
            model = SimpleNN()
            input_data = torch.randn(1, 10)  # Example input
            output = model(input_data)
            print(output)  # Model output
            \end{lstlisting}
    \end{enumerate}
\end{frame}

\begin{frame}
    \frametitle{Hands-on Experience}
    \begin{block}{Introduction to Hands-on Learning}
        Hands-on experience is vital for developing practical skills in AI development. In this session, students will engage with two of the most widely-used frameworks in the field: \textbf{TensorFlow} and \textbf{PyTorch}.
    \end{block}
\end{frame}

\begin{frame}
    \frametitle{Overview of TensorFlow and PyTorch}
    \begin{itemize}
        \item \textbf{TensorFlow}: Developed by Google, TensorFlow is an open-source library for numerical computation and machine learning, enabling the development of deep learning models.
        \item \textbf{PyTorch}: Developed by Facebook’s AI Research lab, PyTorch is also an open-source machine learning library, known for its dynamic computation graph which enables flexibility in model building.
    \end{itemize}
\end{frame}

\begin{frame}
    \frametitle{Learning Objectives}
    \begin{enumerate}
        \item Understand key features and differences between TensorFlow and PyTorch.
        \item Gain practical experience in model creation and training.
        \item Develop the ability to select appropriate frameworks based on project needs.
    \end{enumerate}
\end{frame}

\begin{frame}[fragile]
    \frametitle{Lab Activities}
    \begin{enumerate}
        \item \textbf{Installation and Setup}:
        \begin{itemize}
            \item Students will install TensorFlow and PyTorch in a lab environment. Instructions and troubleshooting tips will be provided.
            \item Example Snippet for Installation:
            \begin{lstlisting}[language=bash]
# Install TensorFlow
pip install tensorflow

# Install PyTorch (with CUDA support)
pip install torch torchvision torchaudio --extra-index-url https://download.pytorch.org/whl/cu113
            \end{lstlisting}
        \end{itemize}
        \item \textbf{Building a Simple Neural Network}
        \begin{itemize}
            \item Creating a simple feedforward neural network using both frameworks.
            \item Tasks include defining the model architecture, choosing the loss function, optimizer, and training the model on a sample dataset (e.g., MNIST digits).
        \end{itemize}
    \end{enumerate}
\end{frame}

\begin{frame}[fragile]
    \frametitle{Neural Network Code Examples}
    \begin{block}{TensorFlow Code Example}
    \begin{lstlisting}[language=python]
import tensorflow as tf
from tensorflow.keras import layers, models

# Define a Sequential model
model = models.Sequential([
    layers.Flatten(input_shape=(28, 28)),
    layers.Dense(128, activation='relu'),
    layers.Dense(10, activation='softmax')
])

# Compile the model
model.compile(optimizer='adam', loss='sparse_categorical_crossentropy', metrics=['accuracy'])
    \end{lstlisting}
    \end{block}
    
    \begin{block}{PyTorch Code Example}
    \begin{lstlisting}[language=python]
import torch
import torch.nn as nn
import torch.optim as optim

class SimpleNN(nn.Module):
    def __init__(self):
        super(SimpleNN, self).__init__()
        self.fc1 = nn.Linear(28*28, 128)
        self.fc2 = nn.Linear(128, 10)

    def forward(self, x):
        x = torch.relu(self.fc1(x.view(-1, 28*28)))
        x = self.fc2(x)
        return x

model = SimpleNN()
criterion = nn.CrossEntropyLoss()
optimizer = optim.Adam(model.parameters(), lr=0.001)
    \end{lstlisting}
    \end{block}
\end{frame}

\begin{frame}
    \frametitle{Key Points to Emphasize}
    \begin{itemize}
        \item \textbf{Flexibility}: PyTorch's dynamic computation graph allows for easier debugging and experimentation.
        \item \textbf{Deployment}: TensorFlow has robust support for deploying models to production via TensorFlow Serving, TensorFlow Lite, and TensorFlow.js.
        \item \textbf{Community \& Resources}: Both frameworks have extensive documentation and community support, making troubleshooting and finding resources straightforward.
    \end{itemize}
\end{frame}

\begin{frame}
    \frametitle{Conclusion}
    This hands-on experience not only reinforces theoretical knowledge but also builds the foundational skills required to implement AI solutions using TensorFlow and PyTorch. Each student will leave with practical insights and model-building skills that will be beneficial in their future work or studies.
\end{frame}

\begin{frame}
    \frametitle{Collaboration and Peer Feedback}
    Through collaborative work, peer feedback, and instructor guidance, students will have the opportunity to bridge the gap between theoretical concepts and practical application, ensuring a comprehensive understanding of AI frameworks.
\end{frame}

\begin{frame}
    \frametitle{Case Studies}
    \begin{block}{Introduction to Case Studies}
        To solidify your understanding of AI development frameworks, we will explore real-world applications built using two popular frameworks: TensorFlow and PyTorch. These examples will illustrate how these tools are transforming industries and solving complex problems.
    \end{block}
\end{frame}

\begin{frame}[fragile]
    \frametitle{TensorFlow Case Study: Healthcare with DeepMind}
    \begin{itemize}
        \item \textbf{Overview}
            \begin{itemize}
                \item DeepMind partnered with healthcare institutions to develop algorithms for analyzing medical images.
                \item Employs TensorFlow for training deep neural networks.
            \end{itemize}
        \item \textbf{Key Features}
            \begin{itemize}
                \item \textbf{Objective}: Early diagnosis of diseases (like cancer).
                \item \textbf{Data Utilization}: High-quality datasets with thousands of annotated medical images.
                \item \textbf{Outcome}: Comparable performance to expert radiologists.
            \end{itemize}
        \item \textbf{Technical Insight}
            \begin{lstlisting}[language=Python]
import tensorflow as tf
model = tf.keras.models.Sequential([
    tf.keras.layers.Conv2D(32, (3, 3), activation='relu', input_shape=(img_height, img_width, channels)),
    tf.keras.layers.MaxPooling2D(pool_size=(2, 2)),
    tf.keras.layers.Flatten(),
    tf.keras.layers.Dense(1, activation='sigmoid')
])
model.compile(optimizer='adam', loss='binary_crossentropy', metrics=['accuracy'])
            \end{lstlisting}
        \item \textbf{Impact}: Improved patient outcomes, reduced load on healthcare professionals.
    \end{itemize}
\end{frame}

\begin{frame}[fragile]
    \frametitle{PyTorch Case Study: Autonomous Driving with Tesla}
    \begin{itemize}
        \item \textbf{Overview}
            \begin{itemize}
                \item Tesla uses PyTorch for self-driving technology.
                \item Develops neural networks for real-time data interpretation.
            \end{itemize}
        \item \textbf{Key Features}
            \begin{itemize}
                \item \textbf{Objective}: Enhance vehicle navigation systems and safety features.
                \item \textbf{Data Utilization}: Large amounts of data from Tesla vehicles.
                \item \textbf{Outcome}: Rapid feature iterations with dynamic computation graphs.
            \end{itemize}
        \item \textbf{Technical Insight}
            \begin{lstlisting}[language=Python]
import torch
import torch.nn as nn

class SimpleNN(nn.Module):
    def __init__(self):
        super(SimpleNN, self).__init__()
        self.fc1 = nn.Linear(input_size, hidden_size)
        self.fc2 = nn.Linear(hidden_size, output_size)

    def forward(self, x):
        x = torch.relu(self.fc1(x))
        x = self.fc2(x)
        return x

model = SimpleNN()
            \end{lstlisting}
        \item \textbf{Impact}: Reduces accidents, improves overall driving experience.
    \end{itemize}
\end{frame}

\begin{frame}
    \frametitle{Key Points to Emphasize}
    \begin{itemize}
        \item \textbf{Framework Selection}: TensorFlow excels in production, while PyTorch is favored for research.
        \item \textbf{Application Diversity}: Both frameworks handle applications in healthcare, automotive, finance, etc.
        \item \textbf{Community and Support}: Extensive resources are available for both frameworks, encouraging collaboration.
    \end{itemize}
\end{frame}

\begin{frame}
    \frametitle{Conclusion}
    Understanding how TensorFlow and PyTorch are applied in real-world scenarios enhances your insight into their capabilities. Consider how these technologies could apply to your interests or future AI development projects.
\end{frame}

\begin{frame}[fragile]
    \frametitle{Conclusion and Future Directions - Key Points Summary}
    \begin{enumerate}
        \item \textbf{Framework Overview}
            \begin{itemize}
                \item AI development frameworks like TensorFlow and PyTorch streamline the creation, training, and testing of machine learning models.
                \item They serve as foundational tools for efficient model building and deployment.
            \end{itemize}
        \item \textbf{Real-World Applications}
            \begin{itemize}
                \item Applied across healthcare, finance, automotive, and entertainment.
                \item Demonstrated benefits include enhanced prediction accuracy and improved operational efficiency.
            \end{itemize}
        \item \textbf{Community Support and Ecosystem}
            \begin{itemize}
                \item Supported by large communities and rich ecosystems.
                \item Extensive libraries, tools, and forums available for knowledge sharing and skill enhancement.
            \end{itemize}
    \end{enumerate}
\end{frame}

\begin{frame}[fragile]
    \frametitle{Conclusion and Future Directions - Future Trends}
    \begin{enumerate}
        \setcounter{enumi}{3}
        \item \textbf{Increased Accessibility}
            \begin{itemize}
                \item Future frameworks will prioritize accessibility for non-technical users.
                \item Rise of no-code and low-code platforms to simplify model creation for business professionals.
            \end{itemize}
        \item \textbf{Integration of AutoML}
            \begin{itemize}
                \item Automation in model building will become more prevalent.
                \item Users can automatically select algorithms and optimize hyperparameters easily.
            \end{itemize}
        \item \textbf{Interoperability}
            \begin{itemize}
                \item Need for frameworks to work seamlessly together as AI integrates into various systems.
                \item Standardization efforts expected to increase.
            \end{itemize}
        \item \textbf{Ethics and Governance}
            \begin{itemize}
                \item Incorporation of tools for ethical AI usage, including bias detection and fairness assessment.
                \item Focus on responsible AI will become a key differentiator among frameworks.
            \end{itemize}
        \item \textbf{Emphasis on Edge AI}
            \begin{itemize}
                \item Shift towards frameworks that support edge computing for improved latency and data privacy.
            \end{itemize}
    \end{enumerate}
\end{frame}

\begin{frame}[fragile]
    \frametitle{Conclusion and Future Directions - Key Takeaways}
    The landscape of AI development frameworks is rapidly evolving. Embracing these trends will equip practitioners to leverage the best tools and methodologies for their projects. This includes:
    
    \begin{block}{Conclusion}
        \begin{itemize}
            \item Stay informed about trends to select the right frameworks.
            \item Contribute to the ethical advancement of AI technologies.
        \end{itemize}
    \end{block}
\end{frame}

\begin{frame}[fragile]
    \frametitle{Example of Future Framework Component: AutoML}
    \begin{lstlisting}[language=Python]
# Example of a simple AutoML implementation using TPOT
from tpot import TPOTClassifier
from sklearn.datasets import load_iris
from sklearn.model_selection import train_test_split

# Load dataset
iris = load_iris()
X_train, X_test, y_train, y_test = train_test_split(iris.data, iris.target, random_state=42)

# Initialize and fit TPOT
tpot = TPOTClassifier(verbosity=2, generations=5, population_size=20)
tpot.fit(X_train, y_train)

# Score the model
print(tpot.score(X_test, y_test))

# Export the best model found
tpot.export('best_model.py')
    \end{lstlisting}
    This code snippet illustrates how straightforward it can be to build a predictive model using AutoML techniques, reducing barriers for entry into AI development.
\end{frame}


\end{document}