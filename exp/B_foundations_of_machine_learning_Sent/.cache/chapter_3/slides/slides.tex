\documentclass{beamer}

% Theme choice
\usetheme{Madrid} % You can change to e.g., Warsaw, Berlin, CambridgeUS, etc.

% Encoding and font
\usepackage[utf8]{inputenc}
\usepackage[T1]{fontenc}

% Graphics and tables
\usepackage{graphicx}
\usepackage{booktabs}

% Code listings
\usepackage{listings}
\lstset{
basicstyle=\ttfamily\small,
keywordstyle=\color{blue},
commentstyle=\color{gray},
stringstyle=\color{red},
breaklines=true,
frame=single
}

% Math packages
\usepackage{amsmath}
\usepackage{amssymb}

% Colors
\usepackage{xcolor}

% TikZ and PGFPlots
\usepackage{tikz}
\usepackage{pgfplots}
\pgfplotsset{compat=1.18}
\usetikzlibrary{positioning}

% Hyperlinks
\usepackage{hyperref}

% Title information
\title{Chapter 3: Mathematical Foundations}
\author{Your Name}
\institute{Your Institution}
\date{\today}

\begin{document}

\frame{\titlepage}

\begin{frame}[fragile]
    \frametitle{Introduction to Linear Algebra in Machine Learning}
    \begin{block}{Overview of Linear Algebra}
        Linear algebra is a branch of mathematics that studies vectors, vector spaces (linear spaces), linear transformations, and systems of linear equations. It provides the tools needed to analyze and solve problems involving linear relationships.
    \end{block}
\end{frame}

\begin{frame}[fragile]
    \frametitle{Importance of Linear Algebra in Machine Learning}
    Linear algebra forms the mathematical foundation for numerous algorithms and models utilized in machine learning. Here are the primary areas of application:
    \begin{enumerate}
        \item \textbf{Data Representation}
        \begin{itemize}
            \item \textbf{Vectors}: Data points (features) are often represented as vectors (e.g., a grayscale image of size 28x28 as a vector of 784 elements).
        \end{itemize}
        
        \item \textbf{Matrices}
        \begin{itemize}
            \item Datasets are structured as matrices, allowing for efficient computations (e.g., a 5x3 matrix for 5 samples with 3 features).
        \end{itemize}
        
        \item \textbf{Linear Transformations}
        \begin{itemize}
            \item Functions that map vectors linearly are crucial for applications like PCA. Key equation: \( \mathbf{y} = \mathbf{Ax} + \mathbf{b} \).
        \end{itemize}
        
        \item \textbf{Solving Systems of Equations}
        \begin{itemize}
            \item Many algorithms rely on solving systems, using methods like matrix inversion (\( \mathbf{x} = \mathbf{A}^{-1}\mathbf{b} \) when \( \mathbf{A} \) is invertible).
        \end{itemize}
        
        \item \textbf{Gradient Descent Optimization}
        \begin{itemize}
            \item Used to compute gradients essential for optimizing cost functions during model training.
        \end{itemize}
    \end{enumerate}
\end{frame}

\begin{frame}[fragile]
    \frametitle{Key Points and Conclusion}
    \begin{block}{Key Points to Emphasize}
        \begin{itemize}
            \item \textbf{Foundation for Algorithms}: Essential for understanding various machine learning algorithms.
            \item \textbf{Efficient Computations}: Matrix operations can be efficiently performed using libraries like NumPy.
            \item \textbf{Real-world Applications}: Fundamental to systems from recommendation algorithms to computer vision tasks.
        \end{itemize}
    \end{block}

    \begin{block}{Conclusion}
        Understanding linear algebraic concepts equips students with skills to tackle complex machine learning problems, improving their analytical capabilities and comprehension of underlying models.
    \end{block}
\end{frame}

\begin{frame}[fragile]
    \frametitle{Relevant Example Code Snippet (Python with NumPy)}
    \begin{lstlisting}[language=Python]
import numpy as np

# Create a dataset matrix (5 samples, 3 features)
data = np.array([[5, 2, 3],
                 [1, 0, 4],
                 [3, 1, 2],
                 [2, 5, 1],
                 [4, 3, 0]])

# Perform matrix operations: mean centering
mean = np.mean(data, axis=0)
centered_data = data - mean
    \end{lstlisting}
\end{frame}

\begin{frame}[fragile]
    \frametitle{Core Concepts of Machine Learning}
    \begin{block}{What is Machine Learning?}
        Machine Learning (ML) is a branch of artificial intelligence that focuses on building systems that can learn from and make decisions based on data. Instead of being programmed explicitly for each task, ML algorithms analyze patterns in data to improve their performance over time.
    \end{block}
\end{frame}

\begin{frame}[fragile]
    \frametitle{Categories of Machine Learning}
    ML can be broadly categorized into three types:
    \begin{enumerate}
        \item \textbf{Supervised Learning}
        \item \textbf{Unsupervised Learning}
        \item \textbf{Reinforcement Learning}
    \end{enumerate}
\end{frame}

\begin{frame}[fragile]
    \frametitle{Supervised Learning}
    \begin{itemize}
        \item \textbf{Definition:} Involves training a model on a labeled dataset. The algorithm learns to map input data (features) to output labels (targets).
        \item \textbf{Example:} 
            Predicting house prices based on features such as size, number of bedrooms, and location.
        \item \textbf{Key Algorithms:} Linear Regression, Decision Trees, Support Vector Machines (SVM), Neural Networks.
    \end{itemize}
    \begin{block}{Illustrative Example}
        \begin{itemize}
            \item \textbf{Dataset:} 
                \begin{itemize}
                    \item Features: [Size (sq ft), Bedrooms]
                    \item Labels: [Price ($)]
                \end{itemize}
            \item \textbf{Training Data:}
                \begin{itemize}
                    \item (2000, 3) $\rightarrow$ 500,000
                    \item (1500, 2) $\rightarrow$ 350,000
                \end{itemize}
            \item \textbf{Goal:} Learn a function that can predict price based on size and bedrooms.
        \end{itemize}
    \end{block}
\end{frame}

\begin{frame}[fragile]
    \frametitle{Unsupervised Learning}
    \begin{itemize}
        \item \textbf{Definition:} Trains the model on data without explicit labels. The goal is to discover underlying patterns or groupings within the data.
        \item \textbf{Example:} 
            Customer segmentation in retail, where the algorithm identifies distinct customer groups based on purchasing behavior without prior labels.
        \item \textbf{Key Algorithms:} K-Means Clustering, Hierarchical Clustering, Principal Component Analysis (PCA).
    \end{itemize}
    \begin{block}{Illustrative Example}
        \begin{itemize}
            \item \textbf{Dataset:}
                \begin{itemize}
                    \item Features: [Age, Income]
                    \item Goal: Group customers into clusters based on their age and income patterns.
                \end{itemize}
            \item \textbf{Result:} Possible clusters of young high-income individuals and middle-aged low-income individuals.
        \end{itemize}
    \end{block}
\end{frame}

\begin{frame}[fragile]
    \frametitle{Reinforcement Learning}
    \begin{itemize}
        \item \textbf{Definition:} A type of learning where an agent interacts with an environment to maximize cumulative reward. The agent learns by receiving feedback from its actions.
        \item \textbf{Example:} 
            Training a robot to navigate a maze, receiving positive rewards for reaching the exit and negative rewards for hitting walls.
        \item \textbf{Key Algorithms:} Q-Learning, Deep Q-Networks (DQN), Policy Gradients.
    \end{itemize}
    \begin{block}{Illustrative Example}
        \begin{itemize}
            \item \textbf{Scenario:} Self-driving car (agent) must navigate through traffic (environment).
            \item \textbf{Feedback:} Receives positive feedback for safely reaching a destination and negative feedback for dangerous driving.
            \item \textbf{Goal:} Maximize rewards while minimizing risks.
        \end{itemize}
    \end{block}
\end{frame}

\begin{frame}[fragile]
    \frametitle{Key Points to Emphasize}
    \begin{itemize}
        \item The type of machine learning approach chosen depends on the nature of the data and the specific problem to solve.
        \item Supervised approaches require labeled data, making them more resource-intensive, while unsupervised can work with unlabeled data.
        \item Reinforcement learning focuses on learning through interaction rather than just from passive observation.
    \end{itemize}
\end{frame}

\begin{frame}[fragile]
    \frametitle{Formulas and Functions}
    \begin{block}{Supervised Learning Example - Linear Regression Model}
        \begin{equation}
            Y = \beta_0 + \beta_1 X_1 + \beta_2 X_2 + \ldots + \beta_n X_n + \epsilon
        \end{equation}
    \end{block}
    \begin{itemize}
        \item Where:
            \begin{itemize}
                \item $Y$ = output/target (e.g., price)
                \item $X_1, X_2, \ldots, X_n$ = features
                \item $\beta_0$ = intercept
                \item $\beta_1, \ldots, \beta_n$ = coefficients
                \item $\epsilon$ = error term.
            \end{itemize}
    \end{itemize}
\end{frame}

\begin{frame}[fragile]
    \frametitle{Formulas and Functions (cont.)}
    \begin{block}{Basic Concept in Reinforcement Learning (Q-Function)}
        \begin{equation}
            Q(s, a) \leftarrow R(s, a) + \gamma \max_{a'} Q(s', a')
        \end{equation}
    \end{block}
    \begin{itemize}
        \item Where:
            \begin{itemize}
                \item $Q(s, a)$ = expected future rewards for action $a$ in state $s$
                \item $R(s, a)$ = immediate reward received for action $a$ in state $s$
                \item $\gamma$ = discount factor.
            \end{itemize}
    \end{itemize}
\end{frame}

\begin{frame}[fragile]
    \frametitle{Mathematical Foundations}
    
    \begin{block}{Importance of Mathematics in Machine Learning}
        Mathematics forms the backbone of machine learning (ML). Key areas include:
        \begin{itemize}
            \item \textbf{Linear Algebra}
            \item \textbf{Statistics}
            \item \textbf{Probability}
        \end{itemize}
        Each field contributes essential tools and concepts for developing and understanding ML algorithms.
    \end{block}
\end{frame}

\begin{frame}[fragile]
    \frametitle{1. Linear Algebra}
    
    \begin{itemize}
        \item \textbf{Definition}: A branch concerning vector spaces and linear mappings between these spaces.
        \item \textbf{Relevance}: Fundamental for understanding data structures such as vectors and matrices.
    \end{itemize}
    
    \begin{block}{Example: Linear Regression}
        In ML models like linear regression, relationships can be expressed as:
        \begin{equation}
            \mathbf{y} = \mathbf{X} \mathbf{w} + \mathbf{e}
        \end{equation}
        Where:
        \begin{itemize}
            \item $\mathbf{y}$ is the output vector
            \item $\mathbf{X}$ is the input feature matrix
            \item $\mathbf{w}$ is the weights vector
            \item $\mathbf{e}$ is the error term
        \end{itemize}
    \end{block}
\end{frame}

\begin{frame}[fragile]
    \frametitle{2. Statistics and 3. Probability}
    
    \begin{block}{Statistics}
        \begin{itemize}
            \item \textbf{Definition}: The science of collecting, analyzing, interpreting, and presenting data.
            \item \textbf{Relevance}: Critical for estimating behavior of data and validating ML model performance.
        \end{itemize}
        
        \begin{block}{Example: Classification Algorithms}
            Understanding the distribution of classes is essential for metrics like accuracy and precision. A confusion matrix summarizes performance across classes.
        \end{block}
    \end{block}
    
    \begin{block}{Probability}
        \begin{itemize}
            \item \textbf{Definition}: The measure of likelihood that an event will occur.
            \item \textbf{Relevance}: Used to make predictions in uncertain scenarios, with models like Naïve Bayes and Bayesian networks.
        \end{itemize}
        
        \begin{block}{Example: Bayesian Inference}
            Conditional probability used in:
            \begin{equation}
                P(A|B) = \frac{P(B|A)P(A)}{P(B)}
            \end{equation}
            Where $P(A|B)$ represents the probability of event A given B is true.
        \end{block}
    \end{block}
\end{frame}

\begin{frame}[fragile]
    \frametitle{Key Points and Conclusion}
    
    \begin{itemize}
        \item Mastery of linear algebra simplifies data manipulation and model understanding.
        \item Statistics is critical for data interpretation and validating model results.
        \item Probability provides a framework for dealing with uncertainty in predictions and inferences.
    \end{itemize}
    
    \begin{block}{Conclusion}
        Understanding these mathematical foundations enhances your ability to design, implement, and evaluate machine learning algorithms, enriching your problem-solving toolkit.
    \end{block}
    
    \begin{block}{Next Steps}
        In the next slide, we will dive deeper into \textbf{Linear Algebra Essentials}, starting with vectors and matrices in machine learning.
    \end{block}
\end{frame}

\begin{frame}[fragile]
    \frametitle{Linear Algebra Essentials - Introduction}
    \begin{block}{What is Linear Algebra?}
        Linear algebra is a cornerstone of machine learning, providing essential frameworks to handle and manipulate data efficiently. It deals with vectors, matrices, and their operations, which are foundational in transforming data for algorithms.
    \end{block}
\end{frame}

\begin{frame}[fragile]
    \frametitle{Linear Algebra Essentials - Key Concepts}
    \begin{enumerate}
        \item \textbf{Vectors}
        \begin{itemize}
            \item \textbf{Definition}: An ordered list of numbers, which can represent points in space or data features.
            \item \textbf{Example}: A 3-dimensional vector:  
            \[
            \mathbf{v} = 
            \begin{bmatrix}
            3 \\
            4 \\
            5
            \end{bmatrix}
            \]
            \item \textbf{In Machine Learning}: Features of a dataset can be represented as vectors.
        \end{itemize} 
        
        \item \textbf{Matrices}
        \begin{itemize}
            \item \textbf{Definition}: A rectangular array of numbers arranged in rows and columns.
            \item \textbf{Example}: A 2x3 matrix:  
            \[
            \mathbf{A} = 
            \begin{bmatrix}
            1 & 2 & 3 \\
            4 & 5 & 6
            \end{bmatrix}
            \]
            \item \textbf{In Machine Learning}: Matrices represent datasets where rows are observations and columns are features.
        \end{itemize}
    \end{enumerate}
\end{frame}

\begin{frame}[fragile]
    \frametitle{Linear Algebra Essentials - Matrix Operations}
    \begin{enumerate}
        \item \textbf{Addition and Subtraction}
        \begin{itemize}
            \item \textbf{Rule}: Two matrices can be added/subtracted if they have the same dimensions.
            \item \textbf{Example}:  
            \[
            \mathbf{A} + \mathbf{B} = 
            \begin{bmatrix}
            1 & 2 \\
            3 & 4
            \end{bmatrix}
            +
            \begin{bmatrix}
            5 & 6 \\
            7 & 8
            \end{bmatrix}
            =
            \begin{bmatrix}
            6 & 8 \\
            10 & 12
            \end{bmatrix}
            \]
        \end{itemize} 

        \item \textbf{Scalar Multiplication}
        \begin{itemize}
            \item \textbf{Definition}: Multiplying each element of a matrix by a scalar.
            \item \textbf{Example}: If $\mathbf{k} = 2$ and $\mathbf{A} = 
            \begin{bmatrix}
            1 & 2 \\
            3 & 4
            \end{bmatrix}$, then 
            $\mathbf{kA} = 
            \begin{bmatrix}
            2 & 4 \\
            6 & 8
            \end{bmatrix}
            $
        \end{itemize}
        
        \item \textbf{Matrix Multiplication}
        \begin{itemize}
            \item \textbf{Rule}: The number of columns of the first matrix must equal the number of rows of the second.
            \item \textbf{Example}: 
            \[
            \mathbf{A} = 
            \begin{bmatrix}
            1 & 2 \\
            3 & 4
            \end{bmatrix}
            , \quad 
            \mathbf{B} = 
            \begin{bmatrix}
            5 & 6 \\
            7 & 8
            \end{bmatrix}
            \]
            The product $\mathbf{AB} = 
            \begin{bmatrix}
            19 & 22 \\
            43 & 50
            \end{bmatrix}
            $
        \end{itemize}
    \end{enumerate}
\end{frame}

\begin{frame}[fragile]
    \frametitle{Linear Algebra Essentials - Relevance to Machine Learning}
    \begin{block}{Importance of Linear Algebra}
        \begin{itemize}
            \item \textbf{Data Transformation}: Matrix operations enable dimensionality reduction, normalization, or feature extraction.
            \item \textbf{Performance}: Efficient operations enhance computational performance in tasks such as linear regression and neural networks.
        \end{itemize}
    \end{block}

    \begin{block}{Key Points to Emphasize}
        \begin{itemize}
            \item Vectors and matrices are crucial for data representation in machine learning.
            \item Understanding matrix operations is vital for data manipulation and model training.
            \item Effective execution of linear algebra computations significantly impacts the success of machine learning algorithms.
        \end{itemize}
    \end{block}
\end{frame}

\begin{frame}[fragile]
    \frametitle{Vector Spaces and Dimensions - Overview}
    \begin{block}{Definition}
        A **vector space** is a collection of vectors defined by two operations: vector addition and scalar multiplication.
    \end{block}
    \begin{itemize}
        \item **Properties of Vector Spaces:**
            \begin{itemize}
                \item Closure under addition and scalar multiplication
                \item Associativity
                \item Identity element exists (zero vector)
                \item Inverse elements for every vector
            \end{itemize}
        \item **Significance:**
            \begin{itemize}
                \item Foundation of linear algebra 
                \item Essential for data representation
            \end{itemize}
    \end{itemize}
\end{frame}

\begin{frame}[fragile]
    \frametitle{Basis of a Vector Space}
    \begin{block}{Basis Definition}
        A **basis** is a set of linearly independent vectors in a vector space that spans the entire space.
    \end{block}
    \begin{itemize}
        \item **Characteristics:**
            \begin{itemize}
                \item Linear Independence: No vector can be expressed as a combination of others.
                \item Spanning: The basis vectors cover the entire vector space.
            \end{itemize}
        \item **Example in** \( \mathbb{R}^3 \):
            \begin{equation}
                \mathbf{v} = x\mathbf{e_1} + y\mathbf{e_2} + z\mathbf{e_3}
            \end{equation}
    \end{itemize}
\end{frame}

\begin{frame}[fragile]
    \frametitle{Dimension of a Vector Space and its Significance}
    \begin{block}{Dimension}
        The **dimension** of a vector space is the number of vectors in a basis for that space.
    \end{block}
    \begin{itemize}
        \item **Key Points:**
            \begin{itemize}
                \item Finite Dimensional Spaces: A finite basis defines the dimension 
                \item Infinite Dimensional Spaces: Lack a finite basis (e.g., function spaces)
            \end{itemize}
        \item **Significance in Feature Representation:**
            \begin{itemize}
                \item Dimensionality Reduction helps preserve essential features in data
                \item Applications in feature engineering in machine learning
            \end{itemize}
        \item **Example:** Principal Component Analysis (PCA) identifies new basis vectors maximizing data variance.
    \end{itemize}
\end{frame}

\begin{frame}
    \frametitle{Matrix Factorization}
    Matrix Factorization is a mathematical technique used to decompose a matrix into the product of two or more matrices. 
    \begin{itemize}
        \item Used in statistics, machine learning, and data compression.
        \item Key benefits:
        \begin{itemize}
            \item \textbf{Dimensionality Reduction}: Reduces features while retaining information.
            \item \textbf{Collaborative Filtering}: Predicts user preferences in recommendation systems.
        \end{itemize}
    \end{itemize}
\end{frame}

\begin{frame}
    \frametitle{Singular Value Decomposition (SVD)}
    Singular Value Decomposition (SVD) is a common matrix factorization technique expressing an \( m \times n \) matrix \( A \) as:
    \begin{equation}
        A = U \Sigma V^T
    \end{equation}
    
    Where:
    \begin{itemize}
        \item \( U \): left singular vectors
        \item \( \Sigma \): diagonal matrix of singular values
        \item \( V^T \): transpose of right singular vectors
    \end{itemize}
    
    \textbf{Example User-Item Ratings Matrix:}
    \begin{equation}
    A = \begin{bmatrix}
        5 & 3 & 0 \\
        4 & 0 & 0 \\
        0 & 1 & 5 \\
        0 & 0 & 4 \\
    \end{bmatrix}
    \end{equation}
\end{frame}

\begin{frame}[fragile]
    \frametitle{Applications of SVD}
    \begin{block}{Dimensionality Reduction}
        \begin{itemize}
            \item Select top \( k \) singular values to create a lower-dimensional approximation of \( A \).
            \item \textbf{Example}: Image processing for compression.
        \end{itemize}
    \end{block}
    
    \begin{block}{Collaborative Filtering}
        \begin{itemize}
            \item Predict missing ratings in user-item matrices using reduced dimensions.
            \item \textbf{Example}: Netflix uses SVD for movie recommendations based on user preferences.
        \end{itemize}
    \end{block}
\end{frame}

\begin{frame}[fragile]
    \frametitle{Conclusion and Key Points}
    \begin{itemize}
        \item \textbf{Matrix Factorization} is essential for dimensionality reduction and recommendation system efficiency.
        \item \textbf{SVD} allows for effective matrix decomposition and analysis.
        \item Implementing SVD enhances data science and machine learning techniques.
    \end{itemize}
\end{frame}

\begin{frame}[fragile]
    \frametitle{Additional Notes}
    \begin{itemize}
        \item Familiarity with matrix decomposition is valuable for coding applications (e.g., NumPy or SciPy).
    \end{itemize}
    
    \begin{lstlisting}[language=Python]
import numpy as np
from numpy.linalg import svd

# Example matrix
A = np.array([[5, 3, 0],
              [4, 0, 0],
              [0, 1, 5],
              [0, 0, 4]])

# Performing SVD
U, sigma, Vt = svd(A)

# Sigma needs to be converted to diagonal matrix
Sigma = np.zeros((A.shape[0], A.shape[1]))
Sigma[:len(sigma), :len(sigma)] = np.diag(sigma)
    \end{lstlisting}
\end{frame}

\begin{frame}[fragile]
    \frametitle{Importance of Statistical Foundations - Introduction}
    Understanding the fundamentals of probability and statistics is crucial for modeling and evaluating machine learning algorithms. These mathematical foundations enable data scientists to make informed decisions based on data, enhancing both the development process and the performance of models.
\end{frame}

\begin{frame}[fragile]
    \frametitle{Importance of Statistical Foundations - Key Concepts}
    \begin{itemize}
        \item \textbf{Probability}: The study of uncertainty and the analysis of random events. It helps to quantify the uncertainty in predictions.
        \item \textbf{Statistics}: The science of collecting, analyzing, interpreting, and presenting data. It is crucial for understanding data distributions and validating models.
    \end{itemize}
\end{frame}

\begin{frame}[fragile]
    \frametitle{Importance of Statistical Foundations - Relevance in Machine Learning}
    \begin{enumerate}
        \item \textbf{Model Evaluation:}
        \begin{itemize}
            \item Evaluates models with statistical measures such as accuracy, precision, recall, and F1-score.
            \item \textbf{Example:} In a binary classification task:
            \[
            \text{Accuracy} = \frac{\text{True Positives} + \text{True Negatives}}{\text{Total Samples}}
            \]
        \end{itemize}

        \item \textbf{Hypothesis Testing:}
        \begin{itemize}
            \item Determines if model performance is statistically significant using tests like t-tests or chi-squared tests.
            \item \textbf{Example:} Testing if a new algorithm outperforms an existing model.
        \end{itemize}

        \item \textbf{Understanding Distributions:}
        \begin{itemize}
            \item Many algorithms assume a certain data distribution (e.g., Gaussian).
            \item Identifying distributions aids in better preprocessing and feature selection.
        \end{itemize}

        \item \textbf{Bayesian Approaches:}
        \begin{itemize}
            \item Incorporate prior beliefs with new evidence using Bayes' theorem.
            \item \textbf{Formula:}
            \[
            P(A|B) = \frac{P(B|A) \times P(A)}{P(B)}
            \]
        \end{itemize}
    \end{enumerate}
\end{frame}

\begin{frame}[fragile]
    \frametitle{Importance of Statistical Foundations - Key Points to Emphasize}
    \begin{itemize}
        \item \textbf{Informed Decision-Making}: Statistical methods empower data scientists to make decisions based on evidence.
        \item \textbf{Performance Metrics}: Understanding metrics aids in choosing and tuning algorithms effectively.
        \item \textbf{Data Interpretation}: Statistics help unveil insights hidden within data, leading to better predictions.
    \end{itemize}
\end{frame}

\begin{frame}[fragile]
    \frametitle{Importance of Statistical Foundations - Conclusion}
    A solid grasp of probability and statistics is essential for anyone involved in machine learning. It not only aids in developing robust models but also fosters critical analysis of their performance, ultimately driving advancements in the field. Utilizing these foundations enables practitioners to navigate complex data landscapes with greater confidence and clarity.
\end{frame}

\begin{frame}[fragile]
    \frametitle{Ethical Considerations in Machine Learning Model Development}
    \begin{block}{Introduction}
        Machine learning (ML) is revolutionizing various fields, but it also raises important ethical questions, particularly regarding \textbf{fairness} and \textbf{accountability}. As ML systems make decisions that can impact people's lives, understanding these implications is crucial for developers.
    \end{block}
\end{frame}

\begin{frame}[fragile]
    \frametitle{Key Concepts: Fairness}
    \begin{enumerate}
        \item \textbf{Fairness}: ML models should make decisions without bias against individuals or groups. 
        \begin{itemize}
            \item \textbf{Data Bias}: Training data reflecting discriminatory patterns (e.g., race, gender).
            \item \textbf{Model Bias}: Inherent in algorithms that might favor one demographic over another.
        \end{itemize}

        \item \textbf{Example}: A hiring algorithm that favors candidates with certain educational backgrounds. If the training data predominantly consists of male applicants, the model may inadvertently disadvantage female candidates.
    \end{enumerate}
\end{frame}

\begin{frame}[fragile]
    \frametitle{Key Concepts: Accountability}
    \begin{enumerate}
        \item \textbf{Accountability}: Developers have the responsibility to ensure models operate transparently and that decisions can be explained and justified.
        \begin{itemize}
            \item \textbf{Model Explainability}: Stakeholders should understand how and why decisions are made.
            \item \textbf{Error Accountability}: Establishing who is responsible for harm caused by incorrect predictions.
        \end{itemize}

        \item \textbf{Example}: An autonomous vehicle making a decision leading to an accident raises the question of responsibility: the developer, the manufacturer, or the user?
    \end{enumerate}
\end{frame}

\begin{frame}[fragile]
    \frametitle{Key Points to Emphasize}
    \begin{itemize}
        \item \textbf{Identify and Mitigate Bias}: Use diverse datasets and pre-processing techniques to reduce bias before training models.
        
        \item \textbf{Use Explainable AI (XAI)}: Incorporate techniques such as SHAP (SHapley Additive exPlanations) or LIME (Local Interpretable Model-agnostic Explanations) to enhance model transparency.
        
        \item \textbf{Establish Ethical Guidelines}: Develop and adhere to ethical standards within organizations to foster a culture of accountability.
    \end{itemize}
\end{frame}

\begin{frame}[fragile]
    \frametitle{Formula & Techniques}
    To assess fairness, metrics like \textbf{Equal Opportunity} and \textbf{Demographic Parity} can be applied:
    \begin{itemize}
        \item \textbf{Equal Opportunity}: True Positive Rate for different groups should be equivalent.
        \item \textbf{Demographic Parity}: Decisions should have equal probability across different groups.
    \end{itemize}

    \textbf{Example Metric Calculation}:
    For a binary classification model:
    \begin{equation}
        \text{True Positive Rate (TPR)} = \frac{TP}{TP + FN}
    \end{equation}

    \begin{lstlisting}[language=python]
# Example code snippet for calculating True Positive Rate
def true_positive_rate(confusion_matrix, group_label):
    TP = confusion_matrix[group_label][1]  # True Positives
    FN = confusion_matrix[group_label][0]  # False Negatives
    return TP / (TP + FN)

# Assuming confusion_matrix = {'Group_A': [40, 10], 'Group_B': [30, 20]}
print(true_positive_rate({'Group_A': [40, 10], 'Group_B': [30, 20]}, 'Group_A'))
    \end{lstlisting}
\end{frame}

\begin{frame}[fragile]
    \frametitle{Conclusion}
    As we delve deeper into machine learning's capabilities, addressing ethical considerations related to fairness and accountability will be fundamental to building trustworthy and equitable systems. Engaging with these issues not only promotes better model performance but also fosters trust and acceptance in society.
\end{frame}

\begin{frame}[fragile]
    \frametitle{Application Case Studies}
    \begin{block}{Introduction to Linear Algebra in Machine Learning}
        Linear algebra is a foundational tool in machine learning, providing essential techniques to manipulate and analyze data. Concepts such as vectors, matrices, and transformations are pivotal in building and understanding advanced algorithms.
    \end{block}
\end{frame}

\begin{frame}[fragile]
    \frametitle{Case Study 1: Image Recognition}
    \begin{itemize}
        \item \textbf{Concepts Applied:}
            \begin{itemize}
                \item Matrices: Each image can be represented as a matrix of pixel values.
                \item For a color image, this can involve three matrices (for RGB channels).
            \end{itemize}

        \item \textbf{Application:}
            \begin{itemize}
                \item Convolutional Neural Networks (CNNs): Utilize convolutions, a linear algebra operation, to extract features from images for classification tasks.
            \end{itemize}

        \item \textbf{Example:}
            Given a 28x28 grayscale image represented as a matrix \( X \):
            \begin{equation}
              X = 
              \begin{bmatrix}
              0 & 1 & 0 & \dots & 0 \\
              1 & 0 & 1 & \dots & 1 \\
              \vdots & \vdots & \vdots & \ddots & \vdots \\
              \end{bmatrix}
            \end{equation}
            Convolution operations with filters can significantly reduce dimensions while capturing essential features, leading to predictions such as identifying numbers in the MNIST dataset.
    \end{itemize}
\end{frame}

\begin{frame}[fragile]
    \frametitle{Case Study 2: Natural Language Processing (NLP)}
    \begin{itemize}
        \item \textbf{Concepts Applied:}
            \begin{itemize}
                \item Vector Spaces: Words can be embedded in high-dimensional space (word embeddings), allowing conceptual similarities to be captured.
            \end{itemize}

        \item \textbf{Application:}
            \begin{itemize}
                \item Word2Vec: This algorithm transforms words into vectors. Similar words have similar vectors in the vector space, facilitating clustering and other machine learning tasks.
            \end{itemize}

        \item \textbf{Example:}
            If "king" and "queen" are two words represented by vectors \( v_{king} \) and \( v_{queen} \), then the vector difference 
            \[
            v_{queen} - v_{king} + v_{woman} \approx v_{man}.
            \]
    \end{itemize}
\end{frame}

\begin{frame}[fragile]
    \frametitle{Case Study 3: Recommendation Systems}
    \begin{itemize}
        \item \textbf{Concepts Applied:}
            \begin{itemize}
                \item Matrix Factorization: A technique to decompose large matrices (like user-item interactions) into simpler, latent features.
            \end{itemize}

        \item \textbf{Application:}
            \begin{itemize}
                \item Collaborative Filtering: By representing users and items in a matrix, the system predicts user preferences through matrix factorization techniques like Singular Value Decomposition (SVD).
            \end{itemize}

        \item \textbf{Example:}
            For user-item ratings matrix \( R \):
            \begin{equation}
            R \approx U \cdot V
            \end{equation}
            where \( U \) and \( V \) are lower-dimensional matrices representing user characteristics and item features, respectively.
    \end{itemize}
\end{frame}

\begin{frame}[fragile]
    \frametitle{Key Points to Emphasize}
    \begin{itemize}
        \item Linear Algebra's Role: It is crucial in processing, transforming, and representing data.
        \item Practical Applications: Case studies show the applicability of theoretical concepts in real-world machine learning scenarios.
        \item Interconnectedness: Understanding different fields (computer vision, NLP, and recommendation systems) is enhanced through mathematical foundations.
    \end{itemize}

    \begin{block}{Conclusion}
        The application of linear algebra in machine learning is not just theoretical; it permeates various domains, allowing the development of powerful models capable of solving complex problems. Understanding these applications solidifies the importance of mathematical foundations in the field of machine learning.
    \end{block}
\end{frame}

\begin{frame}[fragile]
    \frametitle{Conclusion \& Review - Overview}
    \begin{itemize}
        \item Summary of key learnings
        \item Significance of mathematical foundations in machine learning
    \end{itemize}
\end{frame}

\begin{frame}[fragile]
    \frametitle{Conclusion \& Review - Key Learnings}
    \begin{enumerate}
        \item \textbf{Importance of Mathematical Foundations}
            \begin{itemize}
                \item Backbone of machine learning; essential for understanding algorithms
                \item Key areas: Linear Algebra, Probability and Statistics, Calculus
            \end{itemize}
        
        \item \textbf{Integration of Concepts}
            \begin{itemize}
                \item Model training and performance evaluation rely on these foundations
                \item Applications extend to various domains (e.g., healthcare, finance)
            \end{itemize}
    \end{enumerate}
\end{frame}

\begin{frame}[fragile]
    \frametitle{Conclusion \& Review - Detailed Insights}
    \begin{block}{Linear Algebra}
        \begin{itemize}
            \item Vectors and Matrices: Data representation in multi-dimensional spaces
            \item Eigenvalues and Eigenvectors: Important for PCA and dimensionality reduction
        \end{itemize}
    \end{block}
    
    \begin{block}{Probability and Statistics}
        \begin{itemize}
            \item Modeling Uncertainty: Using probabilistic methods in predictive modeling
        \end{itemize}
    \end{block}
    
    \begin{block}{Optimization in Calculus}
        Updates in parameters via Gradient Descent:
        \begin{equation}
            \theta_{\text{new}} = \theta_{\text{old}} - \alpha \nabla J(\theta)
        \end{equation}
    \end{block}
\end{frame}

\begin{frame}[fragile]
    \frametitle{Conclusion \& Review - Key Points and Closing}
    \begin{itemize}
        \item A strong grasp of mathematical foundations enhances understanding and innovation in ML
        \item Translating real-world problems into mathematical formulations is crucial for effective applications
        \item Continuous learning and practice of mathematical tools are essential
    \end{itemize}
    
    \begin{block}{Closing Note}
        The link between mathematical principles and machine learning emphasizes the necessity of a solid math foundation to navigate data-driven innovation.
    \end{block}
\end{frame}


\end{document}