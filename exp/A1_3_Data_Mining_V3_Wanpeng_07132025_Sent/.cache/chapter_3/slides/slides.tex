\documentclass[aspectratio=169]{beamer}

% Theme and Color Setup
\usetheme{Madrid}
\usecolortheme{whale}
\useinnertheme{rectangles}
\useoutertheme{miniframes}

% Additional Packages
\usepackage[utf8]{inputenc}
\usepackage[T1]{fontenc}
\usepackage{graphicx}
\usepackage{booktabs}
\usepackage{listings}
\usepackage{amsmath}
\usepackage{amssymb}
\usepackage{xcolor}
\usepackage{tikz}
\usepackage{pgfplots}
\pgfplotsset{compat=1.18}
\usetikzlibrary{positioning}
\usepackage{hyperref}

% Custom Colors
\definecolor{myblue}{RGB}{31, 73, 125}
\definecolor{mygray}{RGB}{100, 100, 100}
\definecolor{mygreen}{RGB}{0, 128, 0}
\definecolor{myorange}{RGB}{230, 126, 34}
\definecolor{mycodebackground}{RGB}{245, 245, 245}

% Set Theme Colors
\setbeamercolor{structure}{fg=myblue}
\setbeamercolor{frametitle}{fg=white, bg=myblue}
\setbeamercolor{title}{fg=myblue}
\setbeamercolor{section in toc}{fg=myblue}
\setbeamercolor{item projected}{fg=white, bg=myblue}
\setbeamercolor{block title}{bg=myblue!20, fg=myblue}
\setbeamercolor{block body}{bg=myblue!10}
\setbeamercolor{alerted text}{fg=myorange}

% Set Fonts
\setbeamerfont{title}{size=\Large, series=\bfseries}
\setbeamerfont{frametitle}{size=\large, series=\bfseries}
\setbeamerfont{caption}{size=\small}
\setbeamerfont{footnote}{size=\tiny}

% Code Listing Style
\lstdefinestyle{customcode}{
  backgroundcolor=\color{mycodebackground},
  basicstyle=\footnotesize\ttfamily,
  breakatwhitespace=false,
  breaklines=true,
  commentstyle=\color{mygreen}\itshape,
  keywordstyle=\color{blue}\bfseries,
  stringstyle=\color{myorange},
  numbers=left,
  numbersep=8pt,
  numberstyle=\tiny\color{mygray},
  frame=single,
  framesep=5pt,
  rulecolor=\color{mygray},
  showspaces=false,
  showstringspaces=false,
  showtabs=false,
  tabsize=2,
  captionpos=b
}
\lstset{style=customcode}

% Custom Commands
\newcommand{\hilight}[1]{\colorbox{myorange!30}{#1}}
\newcommand{\source}[1]{\vspace{0.2cm}\hfill{\tiny\textcolor{mygray}{Source: #1}}}
\newcommand{\concept}[1]{\textcolor{myblue}{\textbf{#1}}}
\newcommand{\separator}{\begin{center}\rule{0.5\linewidth}{0.5pt}\end{center}}

% Document Start
\begin{document}

\frame{\titlepage}

\begin{frame}[fragile]
    \frametitle{Introduction to Classification Algorithms - Overview}
    \begin{block}{Overview of Classification Algorithms}
        Classification algorithms are a subset of supervised machine learning techniques used in data mining and artificial intelligence. They analyze input data and categorize it into predefined classes or labels based on learned patterns.
    \end{block}
    Their primary goal is to predict the category of new observations based on historical data.

    \begin{block}{Key Points}
        \begin{itemize}
            \item Classification operates in a supervised learning framework.
            \item Aims to predict unseen data accurately.
            \item Evaluation metrics: accuracy, precision, recall, F1-score.
        \end{itemize}
    \end{block}
\end{frame}

\begin{frame}[fragile]
    \frametitle{Importance of Classification in Data Mining}
    \begin{enumerate}
        \item \textbf{Decision-Making:} Provides support for decision-making in fields such as finance, healthcare, and marketing. 
        \item \textbf{Efficient Data Processing:} Sorts vast amounts of data quickly, extracting valuable insights.
        \item \textbf{Real-World Applications:}
            \begin{itemize}
                \item Spam Detection: Identifies spam emails based on sender and message content.
                \item Medical Diagnosis: Classifies medical images such as cancerous cells.
                \item Sentiment Analysis: Assesses public sentiment through social media and reviews.
            \end{itemize}
    \end{enumerate}
\end{frame}

\begin{frame}[fragile]
    \frametitle{Common Classification Algorithms}
    \begin{enumerate}
        \item \textbf{Decision Trees}: Visual representation of decisions that split data based on feature values.
        \item \textbf{Random Forests}: Ensemble method that builds multiple decision trees and merges them for improved accuracy.
        \item \textbf{Support Vector Machines (SVM)}: Finds the optimal hyperplane that separates classes with maximum margin.
        \item \textbf{Neural Networks}: Effective in complex classification tasks, especially in image recognition.
    \end{enumerate}
\end{frame}

\begin{frame}[fragile]
    \frametitle{Example: Classifying Iris Species}
    Consider the famous Iris dataset, which contains measurements of iris flowers. The goal is to classify the species into three types (Setosa, Versicolor, Virginica) based on features like:
    \begin{itemize}
        \item Sepal length
        \item Sepal width
        \item Petal length
        \item Petal width
    \end{itemize}
    
    The classification algorithm learns from the training data (known features and species) and predicts the species of new iris flower measurements.
\end{frame}

\begin{frame}[fragile]
    \frametitle{Conclusion}
    In summary, classification algorithms are critical in data mining as they empower businesses and researchers to make informed decisions based on data. By understanding their significance and foundational concepts, we set the groundwork for exploring deeper motivations and applications in upcoming slides.
\end{frame}

\begin{frame}[fragile]{Why Classification?}
  \begin{block}{Understanding Classification}
    \begin{itemize}
      \item \textbf{Motivations for Classification:}
      \begin{itemize}
        \item \textbf{Decision-Making:} Helps in making informed decisions by predicting outcomes based on existing data.
        \item \textbf{Automation:} Reduces human intervention and errors (e.g., spam filters).
        \item \textbf{Insights and Discovery:} Uncovers patterns within data for better business understanding.
      \end{itemize}
      \item \textbf{Key Points:}
      \begin{itemize}
        \item Assigns items to predefined categories.
        \item Essential in various domains for structured data analysis.
      \end{itemize}
    \end{itemize}
  \end{block}
\end{frame}

\begin{frame}[fragile]{Real-World Use Cases of Classification}
  \begin{enumerate}
    \item \textbf{Healthcare:}
    \begin{itemize}
      \item Predicting patient conditions (e.g., diabetes).
      \item Early diagnosis improves treatment efficiency.
    \end{itemize}
    
    \item \textbf{Finance:}
    \begin{itemize}
      \item Credit scoring of loan applicants.
      \item Categorization helps minimize loan defaults.
    \end{itemize}
    
    \item \textbf{E-commerce:}
    \begin{itemize}
      \item Personalized product recommendations.
      \item Increases sales through tailored experiences.
    \end{itemize}
    
    \item \textbf{Social Media:}
    \begin{itemize}
      \item Classifying posts as 'relevant' or 'irrelevant'.
      \item Enhances user engagement and content relevancy.
    \end{itemize}
  \end{enumerate}
\end{frame}

\begin{frame}[fragile]{The Role of Data Mining in Classification}
  \begin{block}{Data Mining Background}
    \begin{itemize}
      \item Extracts useful information from large datasets.
      \item Classification is a key technique for categorizing data efficiently.
    \end{itemize}
  \end{block}
  
  \begin{block}{Recent AI Applications}
    \begin{itemize}
      \item Language models like ChatGPT utilize classification for:
      \begin{itemize}
        \item Intent recognition
        \item Sentiment analysis
      \end{itemize}
      \item Enhances processing and responding to user queries.
    \end{itemize}
  \end{block}
  
  \begin{block}{Key Insights}
    \begin{itemize}
      \item Vital for transforming raw data into actionable insights.
      \item Critical role in digital transformation where data drives decisions.
    \end{itemize}
  \end{block}
\end{frame}

\begin{frame}[fragile]
    \frametitle{What Are Classification Algorithms?}
    \begin{block}{Definition}
        Classification algorithms are machine learning techniques that categorize data into predefined classes based on input features.
    \end{block}
    \begin{block}{Significance}
        They enable data-driven decisions, allowing organizations to forecast outcomes, identify patterns, and improve operational efficiency.
    \end{block}
\end{frame}

\begin{frame}[fragile]
    \frametitle{Key Concepts in Classification Algorithms}
    \begin{itemize}
        \item \textbf{Supervised Learning:} Algorithms learn from labeled training data to classify new instances.
        \item \textbf{Decision Boundary:} Creates boundaries that separate classes in feature space, varying in complexity based on algorithm.
        \item \textbf{Performance Metrics:} Effectiveness is measured using accuracy, precision, recall, and F1 score.
    \end{itemize}
\end{frame}

\begin{frame}[fragile]
    \frametitle{Examples and Applications of Classification}
    \begin{enumerate}
        \item \textbf{Email Filtering:} Classifies emails as 'spam' or 'not spam'.
        \item \textbf{Medical Diagnosis:} Predicts diseases based on symptoms and test results.
        \item \textbf{Credit Scoring:} Classifies loan applicants as ‘risk’ or ‘no risk’.
    \end{enumerate}
\end{frame}

\begin{frame}[fragile]
    \frametitle{Recent Applications in AI}
    \begin{block}{Example: ChatGPT}
        Utilizes classification approaches to understand and categorize user inputs for contextually relevant responses.
    \end{block}
    \begin{block}{Key Points to Emphasize}
        \begin{itemize}
            \item Classification algorithms improve decision-making across various domains.
            \item Understanding principles and metrics is crucial for choosing the right algorithm for applications.
        \end{itemize}
    \end{block}
\end{frame}

\begin{frame}[fragile]
    \frametitle{Conclusion}
    \begin{block}{Summary}
        Classification algorithms are fundamental in predictive analytics, converting raw data into actionable insights across various industries.
    \end{block}
\end{frame}

\begin{frame}[fragile]
    \frametitle{Popular Classification Algorithms - Introduction}
    \begin{block}{What are Classification Algorithms?}
        Classification algorithms are crucial tools in predictive analytics that categorize or classify data points into predefined classes based on their features. 
    \end{block}
    
    \begin{itemize}
        \item Widely used in applications like spam detection and medical diagnosis.
        \item Integral to AI applications, e.g., ChatGPT.
    \end{itemize}

    \begin{block}{Focus of This Section}
        We will introduce two commonly used classification algorithms: 
        \begin{itemize}
            \item Decision Trees
            \item k-Nearest Neighbors (k-NN)
        \end{itemize}
    \end{block}
\end{frame}

\begin{frame}[fragile]
    \frametitle{Popular Classification Algorithms - Decision Trees}
    \begin{block}{Definition}
        A Decision Tree is a flowchart-like structure used for decision-making, splitting data into branches based on feature values.
    \end{block}
    
    \begin{itemize}
        \item \textbf{Node Splitting:} Each internal node represents a feature.
        \item \textbf{Leaf Nodes:} Endpoints signify final classification outcomes.
    \end{itemize}

    \begin{block}{Example}
        To determine if a person will buy a computer based on age and income:
        \begin{itemize}
            \item If age $\leq$ 30 and income $\leq$ \$50k, classify as "No Buy."
            \item If age $>$ 30 and income $>$ \$50k, classify as "Buy."
        \end{itemize}
    \end{block}
    
    \begin{itemize}
        \item \textbf{Advantages:} Easy to interpret, little preprocessing, handles numerical and categorical data well.
        \item \textbf{Disadvantages:} Prone to overfitting and sensitive to noisy data.
    \end{itemize}
\end{frame}

\begin{frame}[fragile]
    \frametitle{Popular Classification Algorithms - k-Nearest Neighbors (k-NN)}
    \begin{block}{Definition}
        k-NN is an instance-based learning algorithm that classifies a data point based on the classification of its *k* nearest neighbors.
    \end{block}
    
    \begin{itemize}
        \item Choose a value for *k* (number of neighbors).
        \item Calculate distance (e.g., Euclidean distance) from the new data point to existing data points.
        \item Assign the most common class label among the *k* nearest neighbors.
    \end{itemize}

    \begin{block}{Example}
        Given a dataset of animals based on weight and height:
        \begin{itemize}
            \item For a new animal at weight = 20 kg, height = 50 cm:
            \begin{itemize}
                \item If 2 nearest are labeled "Cat" and 1 as "Dog," classify as "Cat."
            \end{itemize}
        \end{itemize}
    \end{block}
    
    \begin{itemize}
        \item \textbf{Advantages:} Simple to understand, effective for small datasets.
        \item \textbf{Disadvantages:} Computationally expensive for large datasets; performance depends on choice of *k* and distance metric.
    \end{itemize}
\end{frame}

\begin{frame}[fragile]
    \frametitle{Key Takeaways}
    \begin{itemize}
        \item Classification algorithms play critical roles in data classification.
        \item Understanding mechanisms, advantages, and limitations is vital for choosing the right algorithm.
        \item Decision Trees offer interpretable models, while k-NN is intuitive but computationally intensive.
    \end{itemize}

    \begin{block}{Conclusion}
        By exploring algorithms like Decision Trees and k-NN, we bridge theory and practical application in data science and predictive analytics.
    \end{block}
\end{frame}

\begin{frame}[fragile]
    \frametitle{Decision Trees - Introduction}
    \begin{itemize}
        \item Decision Trees are a popular method for classification and regression in machine learning.
        \item They model decisions and consequences in a tree-like structure, making them intuitive and easy to visualize.
    \end{itemize}
\end{frame}

\begin{frame}[fragile]
    \frametitle{Decision Trees - Structure}
    \begin{itemize}
        \item A Decision Tree consists of:
        \begin{itemize}
            \item \textbf{Nodes}: Represent the feature or condition being examined.
            \item \textbf{Branches}: Indicate the outcome of a decision (i.e., the split based on the feature).
            \item \textbf{Leaves}: Final output or classification at the end of the branches.
        \end{itemize}
        \item \textbf{Example Structure:}
        \begin{verbatim}
                [Outlook]
                  /    \
                Sunny   Rainy
                /  \      \
              Humid Windy  [Play]
              /      \      
            Yes      No
        \end{verbatim}
    \end{itemize}
\end{frame}

\begin{frame}[fragile]
    \frametitle{Decision Trees - Working Mechanism}
    \begin{enumerate}
        \item \textbf{Data Splitting}: Starts at the root node, data is split based on the best feature.
        \item \textbf{Recursive Partitioning}: Continues splitting at each node until:
        \begin{itemize}
            \item All samples belong to a single class.
            \item No further splits significantly improve the model.
            \item A predefined maximum depth is reached.
        \end{itemize}
    \end{enumerate}
\end{frame}

\begin{frame}[fragile]
    \frametitle{Decision Trees - Advantages and Disadvantages}
    \begin{block}{Advantages}
        \begin{itemize}
            \item Intuitive and easy to interpret.
            \item No need for data normalization.
            \item Handles non-linear relationships.
            \item Versatile for both classification and regression.
        \end{itemize}
    \end{block}

    \begin{block}{Disadvantages}
        \begin{itemize}
            \item Prone to overfitting especially if deep.
            \item Instability to small changes in data.
            \item Biased towards dominant classes.
        \end{itemize}
    \end{block}
\end{frame}

\begin{frame}[fragile]
    \frametitle{Key Points and Conclusion}
    \begin{itemize}
        \item Decision Trees provide a simple, visual method for decision-making.
        \item They are integral to the larger family of algorithms for pattern recognition in data.
        \item Implement pruning strategies to mitigate overfitting.
    \end{itemize}

    \begin{block}{Conclusion}
        Decision Trees are foundational in data mining and machine learning. Understanding their properties prepares you for further exploration of complex algorithms.
    \end{block}
\end{frame}

\begin{frame}[fragile]
    \frametitle{References and Recent Applications}
    \begin{itemize}
        \item Implementations using Python libraries like \texttt{scikit-learn}.
        \item Recent applications in AI, including tools like ChatGPT, leverage decision trees in underlying features.
    \end{itemize}
\end{frame}

\begin{frame}
    \frametitle{Implementation of Decision Trees - Overview}
    \begin{itemize}
        \item Decision Trees are intuitive methods for classification and regression in machine learning.
        \item They divide datasets into manageable subsets and create a decision-making tree.
        \item Using Python's \texttt{scikit-learn}, this guide illustrates building a Decision Tree Classifier.
    \end{itemize}
\end{frame}

\begin{frame}
    \frametitle{Motivation for Using Decision Trees}
    \begin{itemize}
        \item Handle both categorical and continuous data.
        \item Offer clear interpretations via visualized tree structures.
        \item Perform well without extensive data preprocessing.
        \item Uncover hidden patterns that are hard to analyze.
    \end{itemize}
\end{frame}

\begin{frame}[fragile]
    \frametitle{Implementation Steps in Python - Part 1}
    \begin{enumerate}
        \item \textbf{Import Required Libraries:}
        \begin{lstlisting}[language=Python]
import pandas as pd
import numpy as np
import matplotlib.pyplot as plt
from sklearn.model_selection import train_test_split
from sklearn.tree import DecisionTreeClassifier
from sklearn import tree
from sklearn.metrics import accuracy_score, classification_report, confusion_matrix
        \end{lstlisting}
        \item \textbf{Load the Dataset:}
        \begin{lstlisting}[language=Python]
from sklearn.datasets import load_iris
iris = load_iris()
X = iris.data  # features
y = iris.target  # target variable
        \end{lstlisting}
    \end{enumerate}
\end{frame}

\begin{frame}[fragile]
    \frametitle{Implementation Steps in Python - Part 2}
    \begin{enumerate}
        \setcounter{enumi}{2}
        \item \textbf{Split the Data:}
        \begin{lstlisting}[language=Python]
X_train, X_test, y_train, y_test = train_test_split(X, y, test_size=0.3, random_state=42)
        \end{lstlisting}
        \item \textbf{Create the Decision Tree Classifier:}
        \begin{lstlisting}[language=Python]
clf = DecisionTreeClassifier(random_state=42)
clf.fit(X_train, y_train)
        \end{lstlisting}
        \item \textbf{Make Predictions:}
        \begin{lstlisting}[language=Python]
y_pred = clf.predict(X_test)
        \end{lstlisting}
    \end{enumerate}
\end{frame}

\begin{frame}[fragile]
    \frametitle{Implementation Steps in Python - Part 3}
    \begin{enumerate}
        \setcounter{enumi}{5}
        \item \textbf{Evaluate the Model:}
        \begin{lstlisting}[language=Python]
print("Accuracy:", accuracy_score(y_test, y_pred))
print("Classification Report:\n", classification_report(y_test, y_pred))
print("Confusion Matrix:\n", confusion_matrix(y_test, y_pred))
        \end{lstlisting}
        \item \textbf{Visualize the Decision Tree:}
        \begin{lstlisting}[language=Python]
plt.figure(figsize=(15,10))
tree.plot_tree(clf, filled=True, feature_names=iris.feature_names, class_names=iris.target_names)
plt.show()
        \end{lstlisting}
    \end{enumerate}
\end{frame}

\begin{frame}
    \frametitle{Key Points to Emphasize}
    \begin{itemize}
        \item \textbf{Interpretability:} Decision Trees are interpretable and easy to visualize.
        \item \textbf{Model Performance:} Use accuracy, precision, and recall for performance assessment.
        \item \textbf{Overfitting Consideration:} Techniques like pruning and max depth can mitigate overfitting.
        \item \textbf{Conclusion:} Demonstrated the creation of a Decision Tree Classifier using Python's \texttt{scikit-learn}.
    \end{itemize}
\end{frame}

\begin{frame}
    \frametitle{Next Steps}
    \begin{itemize}
        \item Explore hyperparameter tuning to enhance performance.
        \item Investigate ensemble methods like Random Forests for improved accuracy.
        \item Understand and leverage Decision Trees in data analysis projects.
    \end{itemize}
\end{frame}

\begin{frame}[fragile]
    \frametitle{k-Nearest Neighbors (k-NN) - Concept}
    \begin{block}{What is k-NN?}
        k-Nearest Neighbors (k-NN) is a simple, supervised machine learning algorithm used for classification and regression tasks. 
        The central idea is to predict the class of a data point based on the classes of its nearest neighbors in the feature space.
    \end{block}
    
    \begin{block}{Why k-NN?}
        \begin{itemize}
            \item Easy to understand and implement.
            \item Takes advantage of the local structure of the data.
            \item Particularly effective for small-to-medium datasets.
        \end{itemize}
    \end{block}
\end{frame}

\begin{frame}[fragile]
    \frametitle{k-Nearest Neighbors (k-NN) - Functioning}
    \begin{block}{Step-by-Step Process}
        \begin{enumerate}
            \item \textbf{Select the Number of Neighbors (k):} Choose an odd value of k to avoid ties.
            \item \textbf{Calculate Distance:} Use a distance metric (commonly Euclidean).
                \begin{equation}
                D(x, y) = \sqrt{\sum_{i=1}^{n} (x_i - y_i)^2}
                \end{equation}
            \item \textbf{Identify Neighbors:} Find the k closest data points to the test instance.
            \item \textbf{Voting Mechanism:} 
                \begin{itemize}
                    \item For classification: Choose the class of the majority.
                    \item For regression: Use the average of the values.
                \end{itemize}
        \end{enumerate}
    \end{block}
\end{frame}

\begin{frame}[fragile]
    \frametitle{k-Nearest Neighbors (k-NN) - Applications}
    \begin{block}{Real-World Application Scenarios}
        \begin{itemize}
            \item \textbf{Image Recognition:} Recognizing characters or objects in images.
            \item \textbf{Recommendation Systems:} Suggesting products or movies based on similar user preferences.
            \item \textbf{Medical Diagnosis:} Classifying health conditions based on symptoms matching previous cases.
            \item \textbf{Anomaly Detection:} Identifying deviations in network traffic or financial transactions for fraud detection.
        \end{itemize}
    \end{block}

    \begin{block}{Key Points to Emphasize}
        \begin{itemize}
            \item k-NN depends on the distance metric chosen.
            \item The choice of k can influence performance: a small k can be sensitive to noise, while a large k smooths over anomalies.
        \end{itemize}
    \end{block}
    
    \begin{block}{Conclusion}
        k-NN is a foundational algorithm in machine learning, bridging intuitive concepts with practical applications.
    \end{block}
\end{frame}

\begin{frame}[fragile]
    \frametitle{Implementation of k-NN - Overview}
    \begin{block}{What is k-NN?}
        The k-Nearest Neighbors (k-NN) algorithm is a simple yet powerful classification method used in statistical classification and data mining. It classifies new inputs based on the majority class of the k closest training samples in the feature space.
    \end{block}
    
    \begin{block}{Why Use k-NN?}
        \begin{itemize}
            \item \textbf{Simplicity}: Easy to implement and understand.
            \item \textbf{Flexibility}: Works with any distance metric (e.g., Euclidean, Manhattan).
            \item \textbf{No Training Phase}: All computation occurs during prediction, making it efficient with small datasets.
        \end{itemize}
    \end{block}
\end{frame}

\begin{frame}[fragile]
    \frametitle{Implementation of k-NN - Practical Example}
    \begin{block}{Step 1: Import Libraries}
        \begin{lstlisting}[language=Python]
import numpy as np
import pandas as pd
from sklearn.model_selection import train_test_split
from sklearn.neighbors import KNeighborsClassifier
from sklearn import datasets
from sklearn.metrics import accuracy_score
        \end{lstlisting}
    \end{block}

    \begin{block}{Step 2: Load Dataset}
        \begin{lstlisting}[language=Python]
# Load Iris dataset
iris = datasets.load_iris()
X = iris.data  # Features
y = iris.target  # Labels
        \end{lstlisting}
    \end{block}
\end{frame}

\begin{frame}[fragile]
    \frametitle{Implementation of k-NN - Continuing Steps}
    \begin{block}{Step 3: Split the Data}
        \begin{lstlisting}[language=Python]
# Split the data into training and testing sets
X_train, X_test, y_train, y_test = train_test_split(X, y, test_size=0.2, random_state=42)
        \end{lstlisting}
    \end{block}

    \begin{block}{Step 4: Initialize k-NN Classifier}
        \begin{lstlisting}[language=Python]
# Initialize the k-NN classifier with k=3
knn = KNeighborsClassifier(n_neighbors=3)
        \end{lstlisting}
    \end{block}

    \begin{block}{Step 5: Fit the Model}
        \begin{lstlisting}[language=Python]
# Fit the model
knn.fit(X_train, y_train)
        \end{lstlisting}
    \end{block}
    
    \begin{block}{Step 6: Make Predictions}
        \begin{lstlisting}[language=Python]
# Make predictions
y_pred = knn.predict(X_test)
        \end{lstlisting}
    \end{block}
    
    \begin{block}{Step 7: Evaluate the Model}
        \begin{lstlisting}[language=Python]
# Evaluate the model
accuracy = accuracy_score(y_test, y_pred)
print(f'Accuracy: {accuracy * 100:.2f}%')
        \end{lstlisting}
    \end{block}
\end{frame}

\begin{frame}[fragile]
    \frametitle{Implementation of k-NN - Key Points}
    \begin{itemize}
        \item \textbf{k Value}: The choice of k can significantly affect accuracy.
        \begin{itemize}
            \item A small k may lead to overfitting.
            \item A large k can cause underfitting.
        \end{itemize}
        
        \item \textbf{Distance Metric}: Default is Euclidean distance; exploration of other metrics may improve performance.
        
        \item \textbf{Computational Load}: k-NN can be computationally expensive with large datasets due to distance calculations from all training points.
    \end{itemize}
\end{frame}

\begin{frame}
    \frametitle{Implementation of k-NN - Summary}
    \begin{block}{Summary}
        k-NN is a foundational machine learning algorithm that is simple to use yet offers substantial versatility. The provided example illustrates its application on the Iris dataset and emphasizes the importance of experimenting with k and different datasets to deepen understanding of the algorithm's behavior.
    \end{block}
\end{frame}

\begin{frame}[fragile]
    \frametitle{Performance Metrics for Classification - Introduction}
    \begin{itemize}
        \item Evaluating classification algorithms requires clear performance metrics.
        \item This slide covers:
        \begin{itemize}
            \item Accuracy
            \item Precision
            \item Recall
            \item F1-score
        \end{itemize}
    \end{itemize}
\end{frame}

\begin{frame}[fragile]
    \frametitle{Performance Metrics for Classification - Accuracy}
    \begin{block}{Definition}
        Accuracy measures the proportion of true results (both true positives and true negatives) among all instances evaluated.
    \end{block}
    
    \begin{equation}
    \text{Accuracy} = \frac{\text{TP} + \text{TN}}{\text{TP} + \text{TN} + \text{FP} + \text{FN}}
    \end{equation}
    
    \begin{block}{Example}
        In a dataset with 100 predictions where 70 are correct:
        \begin{equation}
        \text{Accuracy} = \frac{70}{100} = 0.7 \text{ or } 70\%
        \end{equation}
    \end{block}
    
    \begin{alertblock}{Key Point}
        While intuitive, accuracy can be misleading in imbalanced datasets.
    \end{alertblock}
\end{frame}

\begin{frame}[fragile]
    \frametitle{Performance Metrics for Classification - Precision, Recall, and F1-Score}
    \begin{block}{Precision}
        \begin{itemize}
            \item \textbf{Definition}: Ratio of true positives to total predicted positives.
            \item \textbf{Formula}:
            \begin{equation}
            \text{Precision} = \frac{\text{TP}}{\text{TP} + \text{FP}}
            \end{equation}
            \item \textbf{Example}: If a model predicts 50 positives and 30 are true positives:
            \begin{equation}
            \text{Precision} = \frac{30}{50} = 0.6 \text{ or } 60\%
            \end{equation}
            \item \textbf{Key Point}: High precision is crucial in applications like spam detection.
        \end{itemize}
    \end{block}

    \begin{block}{Recall (Sensitivity)}
        \begin{itemize}
            \item \textbf{Definition}: Ratio of true positives to actual total positives.
            \item \textbf{Formula}:
            \begin{equation}
            \text{Recall} = \frac{\text{TP}}{\text{TP} + \text{FN}}
            \end{equation}
            \item \textbf{Example}: If there are 40 actual positive cases and 30 are identified:
            \begin{equation}
            \text{Recall} = \frac{30}{40} = 0.75 \text{ or } 75\%
            \end{equation}
            \item \textbf{Key Point}: Essential for applications with high costs of missed positives, like medical diagnoses.
        \end{itemize}
    \end{block}

    \begin{block}{F1-Score}
        \begin{itemize}
            \item \textbf{Definition}: Harmonic mean of precision and recall, balances both metrics.
            \item \textbf{Formula}:
            \begin{equation}
            \text{F1} = 2 \cdot \frac{\text{Precision} \cdot \text{Recall}}{\text{Precision} + \text{Recall}}
            \end{equation}
            \item \textbf{Example}: If precision is 60\% and recall is 75\%:
            \begin{equation}
            \text{F1} = 2 \cdot \frac{0.6 \cdot 0.75}{0.6 + 0.75} \approx 0.666 \text{ or } 66.6\%
            \end{equation}
            \item \textbf{Key Point}: Useful when class distribution is uneven.
        \end{itemize}
    \end{block}
\end{frame}

\begin{frame}[fragile]
    \frametitle{Performance Metrics for Classification - Summary and Conclusion}
    \begin{itemize}
        \item \textbf{Accuracy}: Overall effectiveness measure; can be deceptive.
        \item \textbf{Precision} and \textbf{Recall}: Critical for the performance concerning the positive class.
        \item \textbf{F1-Score}: Combines precision and recall for a balanced view.
    \end{itemize}
    
    \begin{block}{Conclusion}
        Understanding these metrics will help in evaluating and selecting classification models suitable for your dataset and application needs.
    \end{block}
\end{frame}

\begin{frame}[fragile]
    \frametitle{Comparative Analysis - Introduction}
    \begin{block}{Introduction to Comparative Analysis}
        In this section, we will perform a comparative analysis of two popular classification algorithms: 
        Decision Trees and k-Nearest Neighbors (k-NN). Understanding the strengths and weaknesses 
        of these algorithms helps in selecting the right one for specific data mining tasks.
    \end{block}
\end{frame}

\begin{frame}[fragile]
    \frametitle{Comparative Analysis - Decision Trees}
    \begin{block}{1. Decision Trees}
        \textbf{Description:} \\
        Decision Trees are a non-parametric supervised learning method used for classification and regression. 
        They operate by splitting a dataset into subsets based on the value of input features.
        
        \textbf{Key Points:}
        \begin{itemize}
            \item \textbf{Interpretability:} Easy to interpret and visualize as they mimic human decision-making.
            \item \textbf{Performance:} Useful for datasets with complex interactions between features.
            \item \textbf{Overfitting:} Tends to overfit with deeper trees, leading to poor generalization.
        \end{itemize}
    \end{block}
\end{frame}

\begin{frame}[fragile]
    \frametitle{Comparative Analysis - Decision Tree Example}
    \begin{block}{Decision Tree Example}
        To classify whether a fruit is an apple or orange based on features such as weight and color:
        \begin{lstlisting}
            IF weight < 150g THEN
                IF color == "red" THEN "Apple"
                ELSE "Orange"
            ELSE "Orange"
        \end{lstlisting}
    \end{block}
\end{frame}

\begin{frame}[fragile]
    \frametitle{Comparative Analysis - k-NN}
    \begin{block}{2. k-Nearest Neighbors (k-NN)}
        \textbf{Description:} \\
        k-NN is a simple, instance-based learning algorithm that classifies data points based on the 
        'k' nearest neighbors in the feature space.

        \textbf{Key Points:}
        \begin{itemize}
            \item \textbf{Simplicity:} Very easy to implement with no training phase; directly uses the training 
            data for classification.
            \item \textbf{Distance Metric:} Sensitive to the choice of distance metric (Euclidean, Manhattan, etc.), 
            which can affect results.
            \item \textbf{Scalability:} Computationally expensive for large datasets, as it requires calculating 
            distances to all training data points.
        \end{itemize}
    \end{block}
\end{frame}

\begin{frame}[fragile]
    \frametitle{Comparative Analysis - k-NN Example}
    \begin{block}{k-NN Example}
        To classify a new fruit with known properties, use a chosen 'k' (e.g., k=3) to consider the 3 
        closest fruits in the training set and majority vote for classification.

        \textbf{Distance Calculation:}
        For a two-dimensional dataset, the Euclidean distance formula is:
        \begin{equation}
            d = \sqrt{(x_2 - x_1)^2 + (y_2 - y_1)^2}
        \end{equation}
    \end{block}
\end{frame}

\begin{frame}[fragile]
    \frametitle{Comparative Analysis - Comparison Table}
    \begin{center}
        \begin{tabular}{|l|l|l|}
            \hline
            \textbf{Feature} & \textbf{Decision Trees} & \textbf{k-NN} \\
            \hline
            Interpretability & High (visual representation) & Low (no clear model) \\
            \hline
            Training Phase & Requires learning (training tree) & No explicit training phase \\
            \hline
            Processing Time & Fast for predictions (query time) & Slow for large datasets \\
            \hline
            Handling of Non-linearity & Good for non-linear data & Good for all types, depending on distance \\
            \hline
            Overfitting Tendency & Prone if not pruned & Not prone but sensitive to noise \\
            \hline
        \end{tabular}
    \end{center}
\end{frame}

\begin{frame}[fragile]
    \frametitle{Comparative Analysis - Conclusion}
    \begin{block}{Conclusion}
        The choice between Decision Trees and k-NN depends on the specific requirements of the dataset 
        and the task at hand. Decision Trees excel in interpretability while k-NN is advantageous for 
        datasets where proximity in feature space is meaningful.
    \end{block}
    
    \begin{block}{Summary}
        \begin{itemize}
            \item Decision Trees offer clarity and visualization.
            \item k-NN provides simplicity but can struggle with large datasets.
            \item Selection based on application context and dataset characteristics is crucial.
        \end{itemize}
    \end{block}
\end{frame}

\begin{frame}[fragile]
    \frametitle{Comparative Analysis - Discussion}
    \begin{block}{Discussion Points}
        Engage students with discussions around when to choose which algorithm based on the 
        characteristics of the data!
    \end{block}
\end{frame}

\begin{frame}
    \frametitle{Case Studies and Applications}
    \begin{block}{Introduction to Classification Algorithms}
        Classification algorithms, such as Decision Trees and k-Nearest Neighbors (k-NN), play crucial roles in various domains by creating predictive models that automate decision-making.
    \end{block}
\end{frame}

\begin{frame}
    \frametitle{1. Decision Trees}
    \begin{block}{Explanation}
        Decision Trees partition a dataset into subsets based on feature values, resulting in a tree structure where each node represents a decision point.
    \end{block}
    
    \begin{block}{Case Study: Predicting Loan Default}
        \begin{itemize}
            \item \textbf{Context:} A financial institution wants to assess the risk of borrowers defaulting on loans.
            \item \textbf{Application:} Decision Trees analyze historical data (age, income, credit score) to classify applicants into 'default' or 'no default'.
            \item \textbf{Outcome:} The model provides transparency, allowing stakeholders to understand the decision-making process.
        \end{itemize}
    \end{block}

    \begin{itemize}
        \item Key Points:
        \begin{itemize}
            \item Easily interpretable; provides decision pathways.
            \item Handles both numerical and categorical data effectively.
        \end{itemize}
    \end{itemize}
\end{frame}

\begin{frame}[fragile]
    \frametitle{Illustration of Decision Trees}
    \begin{center}
        \includegraphics[width=0.5\textwidth]{tree_structure.png} % Placeholder for visual representation of a decision tree
        % Replace with an actual graphic of a decision tree for your presentation
    \end{center}
\end{frame}

\begin{frame}
    \frametitle{2. k-Nearest Neighbors (k-NN)}
    \begin{block}{Explanation}
        k-NN classifies data points based on the majority class of their 'k' nearest neighbors in the feature space.
    \end{block}

    \begin{block}{Case Study: Disease Classification}
        \begin{itemize}
            \item \textbf{Context:} Medical researchers are studying symptoms to classify types of diseases.
            \item \textbf{Application:} Using patient data vectors (symptoms as features), k-NN assigns a class to new patients based on the most frequent diseases among their k-nearest neighbors.
            \item \textbf{Outcome:} Rapid adaptation to new cases with high accuracy.
        \end{itemize}
    \end{block}

    \begin{itemize}
        \item Key Points:
        \begin{itemize}
            \item Non-parametric and flexible; requires no training phase.
            \item Sensitive to the choice of k and the distance metric used.
        \end{itemize}
    \end{itemize}
\end{frame}

\begin{frame}[fragile]
    \frametitle{Example Code Snippet for k-NN}
    \begin{lstlisting}[language=Python]
from sklearn.neighbors import KNeighborsClassifier

# Example dataset
X = [[0, 0], [1, 1], [1, 0], [0, 1]]  # Features
y = [0, 1, 1, 0]  # Class labels

# Create classifier instance
knn = KNeighborsClassifier(n_neighbors=3)
knn.fit(X, y)  # Fit model

# Predict the class for a new sample
prediction = knn.predict([[0.2, 0.3]])
print('Predicted Class:', prediction)
    \end{lstlisting}
\end{frame}

\begin{frame}
    \frametitle{Conclusion}
    \begin{itemize}
        \item Both Decision Trees and k-NN highlight the power of classification algorithms in diverse fields, from finance to healthcare. 
        \item They transform raw data into actionable insights, facilitating informed decision-making. 
    \end{itemize}
    
    \begin{block}{Key Takeaways}
        \begin{itemize}
            \item \textbf{Decision Trees:} Provide clarity and interpretability in predictions.
            \item \textbf{k-NN:} Offers flexibility and adapts quickly to new data.
        \end{itemize}
    \end{block}
\end{frame}

\begin{frame}[fragile]
    \frametitle{Ethical Considerations in Classification Algorithms}
    \begin{block}{Introduction}
        The deployment of classification algorithms in data mining is pivotal in fields like healthcare, finance, and social media. However, their use brings ethical implications that must address bias, discrimination, and privacy violations.
    \end{block}
\end{frame}

\begin{frame}[fragile]
    \frametitle{Key Ethical Implications}
    \begin{itemize}
        \item \textbf{Bias and Fairness}
            \begin{itemize}
                \item Classification algorithms can perpetuate existing biases.
                \item Example: Historical biases can lead to unfair outcomes in predictive policing or hiring.
                \item \textit{Key Point:} Assess and mitigate bias to ensure fairness.
            \end{itemize}

        \item \textbf{Transparency and Accountability}
            \begin{itemize}
                \item Stakeholders should understand decision-making processes.
                \item Example: Misclassification in healthcare can result in harmful outcomes.
                \item \textit{Key Point:} Ensure interpretability and transparency in algorithms.
            \end{itemize}
    \end{itemize}
\end{frame}

\begin{frame}[fragile]
    \frametitle{Continuing Ethical Implications}
    \begin{itemize}
        \item \textbf{Privacy Concerns}
            \begin{itemize}
                \item Data collection may infringe on privacy rights.
                \item Example: Creditworthiness predictions based on sensitive data can lead to violations.
                \item \textit{Key Point:} Implement data governance to protect sensitive information.
            \end{itemize}
        
        \item \textbf{Impact on Society}
            \begin{itemize}
                \item Algorithms can influence societal outcomes.
                \item Example: Automated job screenings may adversely affect underrepresented groups.
                \item \textit{Key Point:} Consider societal implications, especially for marginalized communities.
            \end{itemize}
    \end{itemize}
\end{frame}

\begin{frame}[fragile]
    \frametitle{Best Practices for Ethical Use}
    \begin{itemize}
        \item Conduct impact assessments prior to algorithm deployment.
        \item Implement continuous monitoring for biases or changes in data trends.
        \item Engage diverse stakeholders in the development process to identify ethical concerns.
    \end{itemize}
\end{frame}

\begin{frame}[fragile]
    \frametitle{Conclusion}
    The use of classification algorithms carries profound ethical responsibilities. We must balance innovation with fairness, transparency, and privacy to ensure responsible applications.
\end{frame}

\begin{frame}[fragile]
    \frametitle{Conclusion: Key Takeaways from Classification Algorithms}
    In this chapter, we explored classification algorithms, essential tools in data mining and machine learning. Here are the crucial points:
    \begin{itemize}
        \item \textbf{Definition and Purpose}: Categorizes data into predetermined classes based on input features.
        \item \textbf{Real-world Applications}: Email filtering, medical diagnosis, and sentiment analysis.
        \item \textbf{Data-Driven Decision Making}: Enhances operations, targets marketing, and personalizes user experiences.
    \end{itemize}
\end{frame}

\begin{frame}[fragile]
    \frametitle{Conclusion: Key Algorithms and Evaluation}
    \begin{itemize}
        \item \textbf{Key Algorithms Discussed}:
        \begin{itemize}
            \item \textbf{Decision Trees}: Flowchart-like structure for classification.
            \item \textbf{Support Vector Machines (SVM)}: Maximizes margin between classes.
            \item \textbf{Logistic Regression}: Predicts probabilities of binary outcomes.
        \end{itemize}
        \item \textbf{Model Evaluation Metrics}:
        \begin{itemize}
            \item \textbf{Accuracy} = (True Positives + True Negatives) / Total Predictions
            \item \textbf{Precision} = True Positives / (True Positives + False Positives)
            \item \textbf{Recall} = True Positives / (True Positives + False Negatives)
            \item \textbf{F1 Score} = 2 * (Precision * Recall) / (Precision + Recall)
        \end{itemize}
    \end{itemize}
\end{frame}

\begin{frame}[fragile]
    \frametitle{Conclusion: Ethical Considerations and Future Insights}
    \begin{itemize}
        \item \textbf{Ethical Considerations}:
        \begin{itemize}
            \item Algorithmic bias and fairness are vital for developing responsible AI systems.
        \end{itemize}
        \item \textbf{Looking Ahead}:
        \begin{itemize}
            \item Next chapter will address deeper questions arising from classification algorithms to enhance understanding.
            \item Classification techniques are pivotal for innovation across various fields including finance, healthcare, and AI applications like ChatGPT.
        \end{itemize}
    \end{itemize}
\end{frame}

\begin{frame}[fragile]
  \frametitle{Questions and Answers - Overview}
  Open the floor for questions and discussions to clarify concepts around classification algorithms.
\end{frame}

\begin{frame}[fragile]
  \frametitle{Why Do We Need Classification?}
  
  Classification algorithms are essential for:
  
  \begin{itemize}
    \item **Decision-making:** Facilitating informed decisions based on data.
    \item **Applications:**
    \begin{enumerate}
      \item Medical Diagnosis: Identifying diseases through patient data.
      \item Financial Assessment: Evaluating credit risk for loan approvals.
      \item Customer Analytics: Targeting customers by analyzing purchasing behavior.
    \end{enumerate}
  \end{itemize}
\end{frame}

\begin{frame}[fragile]
  \frametitle{Examples of Classification Algorithms}

  \begin{itemize}
    \item **Logistic Regression:**
      \begin{itemize}
        \item Binary classification method.
        \item Example: Spam detection in emails.
        \item Formula:
        \begin{equation}
          P(Y=1|X) = \frac{1}{1 + e^{-(\beta_0 + \beta_1X_1 + \ldots + \beta_nX_n)}}
        \end{equation}
      \end{itemize}

    \item **Decision Trees:**
      \begin{itemize}
        \item Flowchart-like structure for classification.
        \item Example: Classifying fruits based on features.
      \end{itemize}

    \item **Random Forest:**
      \begin{itemize}
        \item Ensemble method with multiple decision trees.
        \item Example: Classifying species in wildlife.
      \end{itemize}

    \item **Support Vector Machine (SVM):**
      \begin{itemize}
        \item Separates classes using hyperplanes.
        \item Example: Risk classification of customers.
      \end{itemize}
  \end{itemize}
\end{frame}

\begin{frame}[fragile]
  \frametitle{Key Points and Discussion}

  \begin{block}{Evaluation Metrics}
    \begin{itemize}
      \item **Accuracy:** Ratio of correct predictions.
      \item **Precision:** True positives / (True positives + False positives).
      \item **Recall:** True positives / (True positives + False negatives).
      \item **F1 Score:** Harmonic mean of precision and recall.
    \end{itemize}
  \end{block}  

  \begin{itemize}
    \item Recent Application: ChatGPT utilizes classification for sentiment analysis.
  \end{itemize}
  
  \begin{block}{Discussion Questions}
    \begin{enumerate}
      \item How do we choose the right classification algorithm?
      \item What challenges do you face when implementing these algorithms?
      \item Are there ethical considerations in classification?
    \end{enumerate}
  \end{block}
\end{frame}

\begin{frame}[fragile]
  \frametitle{Closing}
  Let's open the floor for questions! Feel free to ask about any concepts, examples, or applications of classification algorithms that need clarification.
\end{frame}


\end{document}