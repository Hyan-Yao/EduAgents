\documentclass[aspectratio=169]{beamer}

% Theme and Color Setup
\usetheme{Madrid}
\usecolortheme{whale}
\useinnertheme{rectangles}
\useoutertheme{miniframes}

% Additional Packages
\usepackage[utf8]{inputenc}
\usepackage[T1]{fontenc}
\usepackage{graphicx}
\usepackage{booktabs}
\usepackage{listings}
\usepackage{amsmath}
\usepackage{amssymb}
\usepackage{xcolor}
\usepackage{tikz}
\usepackage{pgfplots}
\pgfplotsset{compat=1.18}
\usetikzlibrary{positioning}
\usepackage{hyperref}

% Custom Colors
\definecolor{myblue}{RGB}{31, 73, 125}
\definecolor{mygray}{RGB}{100, 100, 100}
\definecolor{mygreen}{RGB}{0, 128, 0}
\definecolor{myorange}{RGB}{230, 126, 34}
\definecolor{mycodebackground}{RGB}{245, 245, 245}

% Set Theme Colors
\setbeamercolor{structure}{fg=myblue}
\setbeamercolor{frametitle}{fg=white, bg=myblue}
\setbeamercolor{title}{fg=myblue}
\setbeamercolor{section in toc}{fg=myblue}
\setbeamercolor{item projected}{fg=white, bg=myblue}
\setbeamercolor{block title}{bg=myblue!20, fg=myblue}
\setbeamercolor{block body}{bg=myblue!10}
\setbeamercolor{alerted text}{fg=myorange}

% Set Fonts
\setbeamerfont{title}{size=\Large, series=\bfseries}
\setbeamerfont{frametitle}{size=\large, series=\bfseries}
\setbeamerfont{caption}{size=\small}
\setbeamerfont{footnote}{size=\tiny}

% Document Start
\begin{document}

\frame{\titlepage}

\begin{frame}[fragile]
    \titlepage
\end{frame}

\begin{frame}[fragile]
    \frametitle{Understanding Data Processing Architectures}
    Data processing architectures refer to structured frameworks that enable organizations to collect, store, process, and analyze data efficiently. In today's data-driven world, the demand for real-time insights and data manipulation has never been higher. Therefore, it’s crucial to adopt an architecture that can scale as the volume and variety of data grow.
\end{frame}

\begin{frame}[fragile]
    \frametitle{Importance of Scalable Architectures}
    \begin{enumerate}
        \item \textbf{Adaptability to Growth}
            \begin{itemize}
                \item \textbf{Definition:} Scalable architectures can handle increasing data volumes without compromising performance.
                \item \textbf{Example:} A social media platform must manage millions of posts as its user base expands.
            \end{itemize}
        
        \item \textbf{Cost Efficiency}
            \begin{itemize}
                \item \textbf{Definition:} They enable efficient resource allocation, expanding capacity only when necessary.
                \item \textbf{Example:} Cloud architectures allow pay-as-you-go pricing based on actual resource usage.
            \end{itemize}

        \item \textbf{Performance and Speed}
            \begin{itemize}
                \item \textbf{Definition:} A well-designed architecture reduces latency and enhances processing speeds.
                \item \textbf{Example:} Streaming platforms like Apache Kafka ensure low latency in real-time data handling.
            \end{itemize}

        \item \textbf{Flexibility and Modularity}
            \begin{itemize}
                \item \textbf{Definition:} Users can update or replace components without full overhauls.
                \item \textbf{Example:} In microservices, individual services can be optimized independently.
            \end{itemize}
    \end{enumerate}
\end{frame}

\begin{frame}[fragile]
    \frametitle{Key Points to Emphasize}
    \begin{itemize}
        \item \textbf{Importance of Scalability:} Essential for maintaining performance and user satisfaction as data grows.
        \item \textbf{Market Competitiveness:} Scalable architectures enable quicker responses to market demand.
        \item \textbf{Future-proofing:} They help prepare businesses for future data challenges, decreasing obsolescence risk.
    \end{itemize}
\end{frame}

\begin{frame}[fragile]
    \frametitle{Conclusion}
    In summary, scalable data processing architectures are integral to effective and efficient data management. By understanding and implementing these architectures, organizations can ensure they are well-equipped to handle the complexities of modern data landscapes.
\end{frame}

\begin{frame}[fragile]
    \frametitle{Objectives of Data Processing Architectures}
    \begin{block}{Understanding the Goals}
        Data processing architectures are critical frameworks that facilitate the ingestion, storage, processing, and analysis of data in various forms. Their effectiveness directly impacts the quality, efficiency, and scalability of data-driven solutions.
    \end{block}
\end{frame}

\begin{frame}[fragile]
    \frametitle{Key Objectives - Part 1}
    \begin{enumerate}
        \item **Scalability**
        \begin{itemize}
            \item \textbf{Definition:} The ability to handle increasing amounts of data without compromising performance.
            \item \textbf{Example:} An architecture efficiently processing a surge in data during peak transactions (e.g., Black Friday).
        \end{itemize}

        \item **Efficiency**
        \begin{itemize}
            \item \textbf{Definition:} Maximizing resource utilization to minimize processing time and costs.
            \item \textbf{Example:} Using specialized engines like Apache Spark for optimization.
        \end{itemize}

        \item **Flexibility**
        \begin{itemize}
            \item \textbf{Definition:} The capability to support various data types and workloads.
            \item \textbf{Example:} A system that processes historical data (batch) and live sensor data (streaming).
        \end{itemize}
    \end{enumerate}
\end{frame}

\begin{frame}[fragile]
    \frametitle{Key Objectives - Part 2}
    \begin{enumerate}
        \setcounter{enumi}{3}
        \item **Reliability**
        \begin{itemize}
            \item \textbf{Definition:} Ensuring data integrity and availability.
            \item \textbf{Example:} Architectures using replication and fault-tolerance in distributed databases.
        \end{itemize}

        \item **Interoperability**
        \begin{itemize}
            \item \textbf{Definition:} The ability for systems to communicate seamlessly.
            \item \textbf{Example:} Integration with APIs and data lakes for a unified data view.
        \end{itemize}

        \item **Cost-Effectiveness**
        \begin{itemize}
            \item \textbf{Definition:} Utilizing resources economically while delivering high performance.
            \item \textbf{Example:} Leveraging cloud services to reduce on-premise hardware needs.
        \end{itemize}
    \end{enumerate}
\end{frame}

\begin{frame}[fragile]
    \frametitle{Significance of Effective Data Processing Architectures}
    \begin{itemize}
        \item **Enhanced Decision Making:** Accurate data allows organizations to make informed decisions quickly (e.g., real-time fraud detection).
        
        \item **Improved Customer Experience:** Personalization through data processing tailored to user preferences.
        
        \item **Competitive Advantage:** Efficient data utilization provides crucial insights for strategic initiatives.
    \end{itemize}
\end{frame}

\begin{frame}[fragile]
    \frametitle{Conclusion}
    Effective data processing architectures are foundational for modern data-driven applications. They ensure scalability, efficiency, flexibility, reliability, interoperability, and cost-effectiveness. Recognizing these objectives is crucial as we explore core principles like batch and stream processing in upcoming slides.
\end{frame}

\begin{frame}[fragile]
    \frametitle{Key Points to Emphasize}
    \begin{itemize}
        \item **Scalability \& Efficiency** are critical for handling data growth.
        \item **Flexibility \& Interoperability** ensure that various data types and systems can work together.
        \item **Cost-Effectiveness** allows organizations to optimize expenditures while maintaining performance.
    \end{itemize}
\end{frame}

\begin{frame}[fragile]
    \frametitle{Suggested Diagrams}
    Consider including a diagram illustrating a typical data processing architecture that demonstrates component interactions (data ingestion, storage layers, processing engines, backend services).
\end{frame}

\begin{frame}[fragile]
    \frametitle{Core Principles of Data Processing}
    \begin{block}{Distinguishing Between Batch and Stream Processing}
        Data processing methods can be categorized into two main approaches: 
        \begin{itemize}
            \item \textbf{Batch Processing}
            \item \textbf{Stream Processing}
        \end{itemize}
    \end{block}
\end{frame}

\begin{frame}[fragile]
    \frametitle{Understanding Data Processing}
    Data processing refers to methods used to collect, manipulate, and analyze data. The two primary methodologies are:
    \begin{itemize}
        \item \textbf{Batch Processing}: Collects and processes large volumes of data as a single unit.
        \item \textbf{Stream Processing}: Processes continuous data streams in real time.
    \end{itemize}
\end{frame}

\begin{frame}[fragile]
    \frametitle{Batch Processing}
    \begin{block}{Definition}
        Batch processing involves collecting large volumes of data over time and processing them as a single batch.
    \end{block}
    \begin{itemize}
        \item \textbf{Characteristics}:
        \begin{itemize}
            \item High latency; results are produced after accumulation.
            \item Ideal for non-real-time tasks.
            \item Efficient resource utilization, often executed during off-peak hours.
        \end{itemize}
        
        \item \textbf{Use Cases}:
        \begin{itemize}
            \item Monthly payroll processing
            \item End-of-day banking transactions
            \item Data transformation in data warehousing
        \end{itemize}
        
        \item \textbf{Example}:
        An e-commerce platform processes customer orders at the end of the day, generating reports in a single batch job.
    \end{itemize}
\end{frame}

\begin{frame}[fragile]
    \frametitle{Stream Processing}
    \begin{block}{Definition}
        Stream processing allows for real-time processing of continuous data streams, enabling immediate analysis and response.
    \end{block}
    \begin{itemize}
        \item \textbf{Characteristics}:
        \begin{itemize}
            \item Low latency; processes data as it arrives.
            \item Supports dynamic, real-time data requirements.
            \item Utilizes distributed processing for efficiency.
        \end{itemize}
        
        \item \textbf{Use Cases}:
        \begin{itemize}
            \item Real-time fraud detection
            \item Social media feed analysis
            \item IoT device data processing
        \end{itemize}
        
        \item \textbf{Example}:
        A financial institution monitors transactions in real-time to detect fraudulent activity as it happens, allowing for immediate intervention.
    \end{itemize}
\end{frame}

\begin{frame}[fragile]
    \frametitle{Key Points to Emphasize}
    \begin{itemize}
        \item **Timing**: 
            \begin{itemize}
                \item Batch processing accumulates data, while stream processing reacts instantly.
            \end{itemize}
        \item **Complexity vs. Simplicity**: 
            \begin{itemize}
                \item Batch jobs can be simpler but less flexible than stream processing.
            \end{itemize}
        \item **Technologies**: 
            \begin{itemize}
                \item Batch: Apache Hadoop, Apache Spark
                \item Stream: Apache Kafka, Apache Flink
            \end{itemize}
        \item **Choosing the Right Approach**: 
            \begin{itemize}
                \item The decision between batch and stream processing depends on real-time requirements and processing logic.
            \end{itemize}
    \end{itemize}
\end{frame}

\begin{frame}[fragile]
    \frametitle{Conclusion}
    Both batch and stream processing methodologies provide valuable solutions tailored to specific data processing needs. Understanding their differences is essential for designing effective data architectures that align with organizational goals.
\end{frame}

\begin{frame}[fragile]
    \frametitle{Data Storage Options}
    \begin{block}{Introduction}
        In this section, we will explore various data storage solutions, focusing on three primary types:
        \begin{itemize}
            \item \textbf{Data Lakes}
            \item \textbf{Data Warehouses}
            \item \textbf{NoSQL databases}
        \end{itemize}
        Understanding these options is crucial for efficiently managing and analyzing data.
    \end{block}
\end{frame}

\begin{frame}[fragile]
    \frametitle{1. Data Lakes}
    \begin{block}{Definition}
        A Data Lake is a centralized repository that allows you to store all your structured and unstructured data at scale.
    \end{block}
    \begin{itemize}
        \item \textbf{Key Features:}
        \begin{itemize}
            \item Schema on Read: Data is stored in its raw form and only structured when read.
            \item Scalability: Can handle vast amounts of data, from terabytes to petabytes.
        \end{itemize}
        \item \textbf{Example Use Case:} An e-commerce company collects user behavior data, product logs, and sales transactions in a data lake.
    \end{itemize}
\end{frame}

\begin{frame}[fragile]
    \frametitle{2. Data Warehouses}
    \begin{block}{Definition}
        A Data Warehouse is a centralized repository designed for query and analysis of structured data.
    \end{block}
    \begin{itemize}
        \item \textbf{Key Features:}
        \begin{itemize}
            \item Schema on Write: Data is processed and structured before storage.
            \item Performance Optimization: Tailored for fast query performance.
        \end{itemize}
        \item \textbf{Example Use Case:} A retail chain uses a data warehouse to combine sales records, inventory levels, and customer demographics for insightful reporting.
        \item \textbf{Note:} Consider using a diagram to illustrate schema on read versus schema on write.
    \end{itemize}
\end{frame}

\begin{frame}[fragile]
    \frametitle{3. NoSQL Databases}
    \begin{block}{Definition}
        NoSQL databases are designed to provide flexible schemas and scale horizontally.
    \end{block}
    \begin{itemize}
        \item \textbf{Key Features:}
        \begin{itemize}
            \item Variety of Data Models: Include document, key-value, column-family, and graph databases.
            \item High Availability: Often distributed across multiple nodes.
        \end{itemize}
        \item \textbf{Example Use Case:} A social media platform uses a document-based NoSQL database to store user profiles, posts, and interactions.
    \end{itemize}
\end{frame}

\begin{frame}[fragile]
    \frametitle{Conclusion and Key Points}
    \begin{itemize}
        \item \textbf{Choosing the Right Solution:} This depends on data type, scale, and analysis requirements.
        \item \textbf{Integration:} Each storage option's compatibility with data processing frameworks and analytics tools is critical.
    \end{itemize}
    \begin{block}{Conclusion}
        Understanding these data storage options is essential for robust data processing architectures, as each has its advantages suitable for specific use cases.
    \end{block}
\end{frame}

\begin{frame}[fragile]
    \frametitle{References}
    \begin{itemize}
        \item Further reading on Data Lakes, Data Warehouses, and NoSQL databases.
        \item Relevant industry examples highlighting advancements and trends in data storage.
    \end{itemize}
\end{frame}

\begin{frame}[fragile]
    \frametitle{Analyzing Existing Architectures}
    \begin{block}{Introduction}
        Data processing architectures define how data is:
        \begin{itemize}
            \item Collected
            \item Processed
            \item Stored
            \item Accessed
        \end{itemize}
        We review three prominent architectures: 
        \begin{itemize}
            \item Batch Processing
            \item Stream Processing
            \item Lambda Architecture
        \end{itemize}
    \end{block}
\end{frame}

\begin{frame}[fragile]
    \frametitle{Batch Processing Architecture}
    \begin{block}{Definition}
        Batch processing executes jobs on large datasets over time. 
        \begin{itemize}
            \item Ideal for processing historical data
            \item Not suitable for real-time analysis
        \end{itemize}
    \end{block}
    
    \begin{block}{Key Considerations}
        \begin{itemize}
            \item \textbf{Data Latency}: Results are available after the entire batch is processed.
            \item \textbf{Resource Utilization}: Efficient for large volumes of data.
            \item \textbf{Use Cases}: Reports, sales summaries, data transformations.
        \end{itemize}
    \end{block}
    
    \begin{block}{Example}
        An e-commerce company runs weekly batch jobs to analyze customer purchases for trends.
    \end{block}
\end{frame}

\begin{frame}[fragile]
    \frametitle{Stream Processing Architecture}
    \begin{block}{Definition}
        Stream processing allows continuous real-time data processing. 
        \begin{itemize}
            \item Ideal for instant insights
        \end{itemize}
    \end{block}
    
    \begin{block}{Key Considerations}
        \begin{itemize}
            \item \textbf{Low Latency}: Provides immediate feedback for faster decisions.
            \item \textbf{Scalability}: Handles high-velocity data inflow.
            \item \textbf{Use Cases}: Fraud detection, real-time analytics, sensor monitoring.
        \end{itemize}
    \end{block}
    
    \begin{block}{Example}
        Stock trading platforms monitor price changes and execute trades instantly.
    \end{block}
\end{frame}

\begin{frame}[fragile]
    \frametitle{Lambda Architecture}
    \begin{block}{Definition}
        Lambda architecture combines batch and stream processing to leverage their strengths.
    \end{block}

    \begin{block}{Key Considerations}
        \begin{itemize}
            \item \textbf{Complexity}: Requires management of both processing layers.
            \item \textbf{Data Redundancy}: Data may be processed twice.
            \item \textbf{Use Cases}: Large-scale systems needing real-time and historical processing.
        \end{itemize}
    \end{block}

    \begin{block}{Example}
        A social media platform analyzes trending topics using real-time interactions and historical data.
    \end{block}
\end{frame}

\begin{frame}[fragile]
    \frametitle{Summary of Key Points}
    \begin{itemize}
        \item \textbf{Batch Processing}: Effective for large volume analytics with latency.
        \item \textbf{Stream Processing}: Real-time needs require robust infrastructure.
        \item \textbf{Lambda Architecture}: Hybrid approach combining strengths of both methods.
    \end{itemize}
    
    \begin{block}{Conclusion}
        Choosing the appropriate architecture depends on specific business needs and data characteristics.
    \end{block}
\end{frame}

\begin{frame}[fragile]
    \frametitle{Designing Scalable Solutions}
    \begin{block}{Overview}
        Designing a scalable data processing solution is critical for ensuring that systems can handle increased workloads without a decrease in performance. Scalability involves designing architectures that can grow with user demands by adding resources or efficiently managing existing resources.
    \end{block}
\end{frame}

\begin{frame}[fragile]
    \frametitle{Fundamental Steps to Designing Scalable Solutions - Part 1}
    \begin{enumerate}
        \item \textbf{Understand the Workload:}
            \begin{itemize}
                \item \textbf{Definition:} Analyze the type of data and processing demands.
                \item \textbf{Example:} Classify workload types, such as batch processing, streaming data, or interactive queries.
                \item \textbf{Consideration:} Determine peak loads and average usage patterns.
            \end{itemize}
        
        \item \textbf{Choose an Appropriate Architecture:}
            \begin{itemize}
                \item \textbf{Types:}
                    \begin{itemize}
                        \item \textbf{Batch Processing:} Ideal for large volumes of data processed at once (e.g., Hadoop).
                        \item \textbf{Stream Processing:} Best for real-time data processing (e.g., Apache Kafka).
                        \item \textbf{Hybrid Solutions:} Combine batch and stream for flexible workloads.
                    \end{itemize}
                \item \textbf{Key Point:} Match the architecture to workload characteristics.
            \end{itemize}
    \end{enumerate}
\end{frame}

\begin{frame}[fragile]
    \frametitle{Fundamental Steps to Designing Scalable Solutions - Part 2}
    \begin{enumerate}[resume]
        \item \textbf{Design for Flexibility and Modularity:}
            \begin{itemize}
                \item \textbf{Microservices Architecture:} Break down the application into independent modules that can be scaled individually.
                \item \textbf{Example:} A payment processing service that can be independently scaled from a user management service.
                \item \textbf{Benefit:} Isolates issues and allows for independent updates without affecting the entire system.
            \end{itemize}
        
        \item \textbf{Implement Load Balancing:}
            \begin{itemize}
                \item \textbf{Purpose:} Distribute workloads evenly across resources.
                \item \textbf{Techniques:} Use round-robin DNS, session persistence, or application load balancers.
                \item \textbf{Illustration:} Visualize traffic being spread across multiple servers to optimize resource usage.
            \end{itemize}

        \item \textbf{Incorporate Caching Strategies:}
            \begin{itemize}
                \item \textbf{Definition:} Use caches to temporarily store frequently accessed data to improve response times.
                \item \textbf{Examples:} Redis and Memcached for storing session data or frequently read database queries.
                \item \textbf{Key Point:} Reduces database load and speeds up data retrieval.
            \end{itemize}
    \end{enumerate}
\end{frame}

\begin{frame}[fragile]
    \frametitle{Fundamental Steps to Designing Scalable Solutions - Part 3}
    \begin{enumerate}[resume]
        \item \textbf{Utilize Elasticity:}
            \begin{itemize}
                \item \textbf{Cloud Services:} Leverage platforms like AWS, Azure, or Google Cloud for elastic scalability.
                \item \textbf{Automatic Scaling:} Configure resources to automatically scale up/down based on demand.
                \item \textbf{Key Point:} Cost-efficient resource management optimizes budget as well.
            \end{itemize}

        \item \textbf{Monitor and Optimize Performance:}
            \begin{itemize}
                \item \textbf{Tools:} Utilize monitoring solutions (e.g., Prometheus, Grafana) to identify bottlenecks and performance issues.
                \item \textbf{Regular Reviews:} Assess performance metrics and adjust resources based on user feedback and usage patterns.
                \item \textbf{Key Point:} Continuous monitoring ensures that the system adapts to changing demands.
            \end{itemize}
    \end{enumerate}
\end{frame}

\begin{frame}[fragile]
    \frametitle{Key Takeaway and Example Formula}
    \begin{block}{Key Takeaway}
        Scalability in data processing is not just about adding more hardware; it’s about thoughtfully designing systems and architectures that can grow dynamically while maintaining efficiency and performance.
    \end{block}

    \begin{block}{Example Formula}
        \textbf{Throughput (T) Calculation:}
        \begin{equation}
          T = \frac{Total\_Processed\_Data}{Time}
        \end{equation}
        Where `Total_Processed_Data` is the amount of data processed and `Time` is the duration of the processing cycle.
    \end{block}
\end{frame}

\begin{frame}[fragile]
    \frametitle{Implementing Data Processing Workflows - Introduction}
    Data processing workflows are integral to extracting insights from large volumes of data. 
    This slide outlines the steps necessary to develop a data processing workflow using two powerful frameworks: 
    \textbf{Apache Hadoop} and \textbf{Apache Spark}. Each has unique strengths suited for various data processing tasks.
\end{frame}

\begin{frame}[fragile]
    \frametitle{Implementing Data Processing Workflows - Step 1}
    \textbf{1. Define Objectives and Requirements}
    \begin{itemize}
        \item \textbf{Identify Data Sources}: Understand where your data is stored (e.g., databases, data lakes).
        \item \textbf{Determine Use Cases}: Clarify what problems you’re trying to solve (e.g., batch processing, real-time analytics).
        \item \textbf{Set Performance Goals}: Establish expected performance metrics and scale requirements.
    \end{itemize}
    \textbf{Example:} Analyze user behavior data from a web application; define expected load, frequency of analysis, and latency requirements.
\end{frame}

\begin{frame}[fragile]
    \frametitle{Implementing Data Processing Workflows - Step 2}
    \textbf{2. Choose the Right Framework}
    \begin{itemize}
        \item \textbf{Apache Hadoop}: Best suited for batch processing of large datasets using the MapReduce paradigm.
        \item \textbf{Apache Spark}: Offers in-memory processing, suitable for real-time data analysis with support for iterative algorithms.
    \end{itemize}
    \textbf{Key Point:} Choose based on your use case; Spark can handle both batch and stream processing, while Hadoop excels at batch processing.
\end{frame}

\begin{frame}[fragile]
    \frametitle{Implementing Data Processing Workflows - Step 3}
    \textbf{3. Data Ingestion}
    \begin{itemize}
        \item Use tools such as Apache Flume for streaming data or Apache Sqoop for batch data transfer from relational databases.
        \item \textbf{Hadoop Example:} Use HDFS (Hadoop Distributed File System) to store large datasets.
        \item \textbf{Spark Example:} Read data directly from sources like HDFS, S3, or databases using Spark APIs.
    \end{itemize}

    \begin{block}{Code Snippet (Spark)}
    \begin{lstlisting}[language=python]
from pyspark.sql import SparkSession
spark = SparkSession.builder.appName("DataProcessing").getOrCreate()
df = spark.read.csv("hdfs://path/to/data.csv", header=True, inferSchema=True)
    \end{lstlisting}
    \end{block}
\end{frame}

\begin{frame}[fragile]
    \frametitle{Implementing Data Processing Workflows - Steps 4-6}
    \textbf{4. Data Processing Logic}
    \begin{itemize}
        \item \textbf{Define Data Transformations}: Use either MapReduce jobs (Hadoop) or DataFrame/Dataset APIs (Spark) to transform data.
        \item \textbf{Example Workflows:}
        \begin{itemize}
            \item \textbf{Hadoop:} Write Mapper and Reducer functions.
            \item \textbf{Spark:} Leverage built-in functions and transformations (e.g., \texttt{filter}, \texttt{groupBy}).
        \end{itemize}
    \end{itemize}

    \begin{block}{Code Snippet (Hadoop MapReduce)}
    \begin{lstlisting}[language=java]
public class WordCount {
    public static class TokenizerMapper extends Mapper<Object, Text, Text, IntWritable> {
        // Implement map function
    }
    public static class IntSumReducer extends Reducer<Text, IntWritable, Text, IntWritable> {
        // Implement reduce function
    }
}
    \end{lstlisting}
    \end{block}
    
    \textbf{5. Job Submission and Monitoring}
    \begin{itemize}
        \item Submit jobs using tools like Apache YARN (for Hadoop) or Spark's built-in cluster manager.
        \item Monitor job status via dashboards (e.g., Spark UI, Hadoop Resource Manager).
        \item \textbf{Key Point:} Regularly check for job failures or bottlenecks.
    \end{itemize}

    \textbf{6. Output Management}
    \begin{itemize}
        \item Store processed data back in HDFS or a database.
        \item Use tools like Apache Hive or Presto for efficient querying.
    \end{itemize}

    \begin{block}{Example of Storing Output in Spark}
    \begin{lstlisting}[language=python]
df.write.mode("overwrite").parquet("hdfs://path/to/output")
    \end{lstlisting}
    \end{block}
\end{frame}

\begin{frame}[fragile]
    \frametitle{Implementing Data Processing Workflows - Conclusion}
    Implementing a data processing workflow with Apache Hadoop or Spark involves defining clear objectives, 
    selecting the appropriate framework, processing data effectively, and managing outputs. 
    This systematic approach ensures the efficient handling of big data for meaningful insights.
\end{frame}

\begin{frame}[fragile]
    \frametitle{Performance Evaluation of Data Systems}
    Key metrics for assessing the performance of data systems.
\end{frame}

\begin{frame}[fragile]
    \frametitle{Key Metrics for Assessing Performance}
    \begin{enumerate}
        \item \textbf{Throughput}
            \begin{itemize}
                \item \textbf{Definition:} Amount of data processed in a specific time frame.
                \item \textbf{Importance:} High throughput indicates efficiency.
                \item \textbf{Example:} 1 million records in 10 minutes $\Rightarrow$ 100,000 records/min.
            \end{itemize}
        
        \item \textbf{Latency}
            \begin{itemize}
                \item \textbf{Definition:} Time taken for a single data request.
                \item \textbf{Importance:} Low latency is crucial for real-time applications.
                \item \textbf{Example:} 200 milliseconds to retrieve data $\Rightarrow$ latency of 200ms.
            \end{itemize}
    \end{enumerate}
\end{frame}

\begin{frame}[fragile]
    \frametitle{Key Metrics Continued}
    \begin{enumerate}
        \setcounter{enumi}{2}
        \item \textbf{Scalability}
            \begin{itemize}
                \item \textbf{Definition:} Capacity to handle increased loads without performance loss.
                \item \textbf{Types:}
                    \begin{itemize}
                        \item \textbf{Vertical Scaling:} Adding resources to a machine.
                        \item \textbf{Horizontal Scaling:} Adding more machines.
                    \end{itemize}
                \item \textbf{Importance:} Allows growth alongside demand.
            \end{itemize}

        \item \textbf{Resource Utilization}
            \begin{itemize}
                \item \textbf{Definition:} Efficiency of resource usage (CPU, memory, disk I/O).
                \item \textbf{Importance:} High utilization signals efficiency, but contention may arise near 100%.
                \item \textbf{Example:} 85\% CPU usage indicates good utilization with headroom.
            \end{itemize}

        \item \textbf{Fault Tolerance}
            \begin{itemize}
                \item \textbf{Definition:} Continued operation despite component failures.
                \item \textbf{Importance:} Improves reliability and availability.
                \item \textbf{Example:} Apache Spark rerouting tasks in case of node failures.
            \end{itemize}
    \end{enumerate}
\end{frame}

\begin{frame}[fragile]
    \frametitle{Key Points to Emphasize}
    \begin{itemize}
        \item Understanding key metrics aids in selecting the right data processing architecture.
        \item Performance evaluation is vital for optimizing systems to meet business needs.
        \item A holistic approach involves monitoring and using metrics for architectural decisions.
    \end{itemize}
\end{frame}

\begin{frame}[fragile]
    \frametitle{Performance Metrics Assessment Example}
    \begin{block}{Performance Metrics Table}
        \begin{center}
        \begin{tabular}{|c|l|}
            \hline
            \textbf{Metric} & \textbf{Description} \\
            \hline
            Throughput     & Records processed/time \\
            Latency        & Time taken per request \\
            Scalability    & Capacity to handle growth \\
            Resource Util. & Efficiency of resource use \\
            Fault Tolerance & Resilience to component failures \\
            \hline
        \end{tabular}
        \end{center}
    \end{block}
\end{frame}

\begin{frame}[fragile]
    \frametitle{Identifying and Addressing Bottlenecks}
    Common bottlenecks in data processing systems and strategies for overcoming them.
\end{frame}

\begin{frame}[fragile]
    \frametitle{Understanding Bottlenecks}
    Bottlenecks are points in a data processing system where performance is limited due to inadequate resource availability. They can significantly impact throughput, latency, and overall system efficiency.

    \begin{block}{Types of Bottlenecks}
        \begin{itemize}
            \item \textbf{CPU Bottlenecks:} Insufficient processing power to handle the workload.
            \item \textbf{Memory Bottlenecks:} Inadequate RAM causing slow processing speeds.
            \item \textbf{I/O Bottlenecks:} Slow read/write speeds in disk operations or network communication.
            \item \textbf{Network Bottlenecks:} Insufficient bandwidth for data load, causing delays.
        \end{itemize}
    \end{block}
\end{frame}

\begin{frame}[fragile]
    \frametitle{Common Symptoms of Bottlenecks}
    \begin{itemize}
        \item Increased processing time for data queries.
        \item High latency in data retrieval.
        \item Low throughput rates.
        \item System crashes or errors during high-load operations.
    \end{itemize}
\end{frame}

\begin{frame}[fragile]
    \frametitle{Strategies for Overcoming Bottlenecks}
    \begin{enumerate}
        \item \textbf{Performance Monitoring:}
            \begin{itemize}
                \item Tools: Utilize monitoring tools (e.g., Prometheus, Grafana).
                \item Key Metrics: CPU usage, Memory usage, Disk I/O rates, Network bandwidth.
            \end{itemize}
        \item \textbf{Scaling Resources:}
            \begin{itemize}
                \item Vertical Scaling: Increase CPU or Memory resources.
                \item Horizontal Scaling: Add more machines for load handling.
            \end{itemize}
        \item \textbf{Optimizing Queries:}
            \begin{itemize}
                \item Indexing Data: Create indexes in databases.
                \item Query Optimization: Analyze query plans.
            \end{itemize}
            \begin{lstlisting}[language=SQL]
-- Example of a SQL query with an index
CREATE INDEX idx_customer_name ON customers(name);
SELECT * FROM customers WHERE name = 'John Doe';
            \end{lstlisting}
    \end{enumerate}
\end{frame}

\begin{frame}[fragile]
    \frametitle{Strategies for Overcoming Bottlenecks (Continued)}
    \begin{enumerate}[resume]
        \item \textbf{Caching Strategies:}
            \begin{itemize}
                \item In-Memory Caching: Use Redis or Memcached.
                \item Content Delivery Networks (CDNs): Improve data retrieval times.
            \end{itemize}
        \item \textbf{Load Balancing:}
            \begin{itemize}
                \item Distribute workloads across multiple servers.
                \item Use load balancers for traffic direction.
            \end{itemize}
        \item \textbf{Refactoring Architecture:}
            \begin{itemize}
                \item Implement Microservices to break monolithic applications.
                \item Data Sharding: Distribute data across databases.
            \end{itemize}
    \end{enumerate}
\end{frame}

\begin{frame}[fragile]
    \frametitle{Key Points to Remember}
    \begin{itemize}
        \item Identifying and addressing bottlenecks is crucial for maintaining high performance.
        \item Performance metrics provide insights into bottlenecks.
        \item A combination of strategies may be required to optimize system performance.
    \end{itemize}
\end{frame}

\begin{frame}[fragile]
    \frametitle{Conclusion}
    Understanding and addressing bottlenecks in data processing systems can lead to enhanced performance, improved user experience, and better resource utilization, ultimately leading to more efficient data processing architectures.
\end{frame}

\begin{frame}
    \frametitle{Integrating APIs in Data Solutions}
    \begin{block}{Overview of APIs in Data Processing}
        \begin{itemize}
            \item \textbf{API Definition}: A mechanism for software applications to communicate.
            \item \textbf{Role}: Enhance data collection, transformation, integration, and delivery.
        \end{itemize}
    \end{block}
\end{frame}

\begin{frame}
    \frametitle{Why Integrate APIs?}
    \begin{itemize}
        \item \textbf{Enhanced Functionality}: Access powerful capabilities without building from scratch.
        \item \textbf{Data Enrichment}: Access external datasets to enhance insights.
        \item \textbf{Efficiency}: Automate processes to save time and reduce errors.
    \end{itemize}
\end{frame}

\begin{frame}[fragile]
    \frametitle{Examples of API Integration in Data Projects}
    \begin{enumerate}
        \item \textbf{Data Retrieval API}:
        \begin{itemize}
            \item \textit{Example}: Using a weather API to fetch weather data for analysis.
            \item \textit{Code Snippet:}
            \begin{lstlisting}[language=Python]
import requests

api_key = 'your_api_key_here'
city = 'London'
url = f"http://api.openweathermap.org/data/2.5/weather?q={city}&appid={api_key}"
response = requests.get(url)
data = response.json()
print(data)
            \end{lstlisting}
        \end{itemize}

        \item \textbf{Data Processing API}:
        \begin{itemize}
            \item \textit{Example}: Using a machine learning API for trend analysis.
            \item \textit{Code Snippet:}
            \begin{lstlisting}[language=Python]
from sklearn.externals import joblib

model = joblib.load('model.pkl')
prediction = model.predict(new_data)
            \end{lstlisting}
        \end{itemize}

        \item \textbf{Data Visualization API}:
        \begin{itemize}
            \item \textit{Example}: Integrating a visualization API to create charts.
            \item \textit{Code Snippet:}
            \begin{lstlisting}[language=JavaScript]
const data = [{x: [1, 2, 3], y: [2, 3, 5], mode: 'lines'}];
Plotly.newPlot('myDiv', data);
            \end{lstlisting}
        \end{itemize}
    \end{enumerate}
\end{frame}

\begin{frame}
    \frametitle{Key Points to Emphasize}
    \begin{itemize}
        \item APIs bridge data processing platforms and external services for complex solutions.
        \item Choose APIs based on project needs: data quality, reliability, and cost.
        \item Handle API limitations, such as rate limits and access restrictions.
    \end{itemize}
\end{frame}

\begin{frame}
    \frametitle{Diagram: API Integration Architecture}
    \begin{block}{Architecture Overview}
        Consider an architecture where a data pipeline consists of:
        \begin{itemize}
            \item Data Sources
            \item API Requests
            \item Data Processing
            \item Visualization and Reporting
        \end{itemize}
    \end{block}
\end{frame}

\begin{frame}[fragile]
    \frametitle{Collaborative Team Work in Data Projects}
    \begin{block}{Importance of Teamwork in Data Science}
        Data science projects are inherently complex, requiring collaboration among team members to ensure successful and efficient completion.
    \end{block}
\end{frame}

\begin{frame}[fragile]
    \frametitle{Overview of Collaboration in Data Science}
    \begin{itemize}
        \item Effective teamwork fosters innovation.
        \item Minimizes errors in data analysis.
        \item Leads to higher-quality insights.
    \end{itemize}
\end{frame}

\begin{frame}[fragile]
    \frametitle{Key Roles in a Data Science Team}
    \begin{itemize}
        \item \textbf{Data Engineers}: Responsible for data architecture and ETL processes.
        \item \textbf{Data Scientists}: Analyze data and build predictive models.
        \item \textbf{Data Analysts}: Focus on data visualization and reporting.
        \item \textbf{Business Analysts}: Align data-driven insights with business strategies.
    \end{itemize}
\end{frame}

\begin{frame}[fragile]
    \frametitle{Benefits of Team Collaboration}
    \begin{itemize}
        \item \textbf{Diverse Perspectives}: Varying expertise leads to innovation.
        \item \textbf{Shared Knowledge}: Encourages collective problem-solving.
        \item \textbf{Increased Accountability}: Team reliance promotes accountability.
    \end{itemize}
\end{frame}

\begin{frame}[fragile]
    \frametitle{Tools for Collaboration in Data Projects}
    \begin{itemize}
        \item \textbf{Version Control Systems (e.g., Git)}: Manage code changes and collaboration.
        \item \textbf{Collaboration Platforms (e.g., Slack, Microsoft Teams)}: Facilitate communication and file sharing.
        \item \textbf{Project Management Tools (e.g., Trello, JIRA)}: Organize tasks, set milestones, and track progress.
    \end{itemize}
\end{frame}

\begin{frame}[fragile]
    \frametitle{Real-World Example}
    \begin{block}{Collaborative Project Approach}
        In a retail project to improve customer categorization:
        \begin{enumerate}
            \item Data Engineers set up a data pipeline.
            \item Data Scientists develop models with feedback from Analysts.
            \item Business Analysts test outcomes against metrics and feedback.
        \end{enumerate}
    \end{block}
\end{frame}

\begin{frame}[fragile]
    \frametitle{Key Points to Emphasize}
    \begin{itemize}
        \item Collaboration leverages diverse skills in data science.
        \item Regular communication enhances project efficiency and innovation.
        \item Appropriate tools improve productivity and stakeholder engagement.
    \end{itemize}
\end{frame}

\begin{frame}[fragile]
    \frametitle{Conclusion}
    \begin{block}{Why Collaboration is Essential}
        Collaboration in data projects is critical for achieving goals and maximizing data potential. Technology and a team-oriented culture are vital for success.
    \end{block}
\end{frame}

\begin{frame}[fragile]
    \frametitle{Optional Code Snippet}
    \begin{lstlisting}[language=Python]
# Using Git for version control in a data science project
# Example of basic Git commands
git init                 # Initialize a new Git repository
git add <filename>       # Stage changes
git commit -m "Message"  # Commit changes
git push origin main     # Push changes to remote repository
    \end{lstlisting}
\end{frame}

\begin{frame}[fragile]
    \frametitle{Optional Diagram Description}
    \textbf{Team Collaboration Workflow:} A flowchart illustrating interactions between Data Engineers, Data Scientists, Data Analysts, and Business Analysts, showcasing feedback loops and knowledge sharing.
\end{frame}

\begin{frame}[fragile]
    \frametitle{Conclusion - Key Takeaways}
    \begin{block}{Understanding Data Processing Architectures}
        \begin{itemize}
            \item Foundational frameworks for data handling
            \item Three primary types:
            \begin{itemize}
                \item \textbf{Batch Processing}: Scheduled large volume processing (e.g., nightly data uploads)
                \item \textbf{Stream Processing}: Real-time data processing (e.g., social media feeds)
                \item \textbf{Micro-batch Processing}: Small batches with low latency (e.g., every few seconds)
            \end{itemize}
        \end{itemize}
    \end{block}
\end{frame}

\begin{frame}[fragile]
    \frametitle{Conclusion - Key Takeaways (Cont'd)}
    \begin{block}{Importance of Scalability and Flexibility}
        \begin{itemize}
            \item Must handle increasing data volumes
            \item Flexibility for integrating various data sources
            \item \textbf{Elasticity}: Dynamic resource allocation based on workloads
        \end{itemize}
    \end{block}

    \begin{block}{Integration with Collaboration Tools}
        \begin{itemize}
            \item Facilitate collaboration among data teams
            \item Tools: Jupyter, GitHub, Slack
            \item Collaboration Flow: Data Scientists $\rightarrow$ Data Engineers $\rightarrow$ Stakeholders
        \end{itemize}
    \end{block}
\end{frame}

\begin{frame}[fragile]
    \frametitle{Conclusion - Real-World Applications}
    \begin{block}{Real-World Applications}
        \begin{itemize}
            \item Insights for informed decision-making across industries
            \item Example sectors:
            \begin{itemize}
                \item \textbf{Finance}: Real-time risk assessment
                \item \textbf{Healthcare}: Patient monitoring using streaming data
                \item \textbf{Retail}: Personalized marketing through consumer behavior analysis
            \end{itemize}
        \end{itemize}
    \end{block}

    \begin{block}{Key Concepts to Remember}
        \begin{itemize}
            \item Architecture choices affect latency and efficiency
            \item Specific project requirements guide architecture selection
        \end{itemize}
    \end{block}

    \begin{block}{Looking Ahead}
        \begin{itemize}
            \item Explore architectural design strategies
            \item Prepare for deeper discussions in upcoming Q\&A session
        \end{itemize}
    \end{block}
\end{frame}

\begin{frame}[fragile]
  \frametitle{Q\&A Session - Introduction}
  This session serves as an open forum for exploring the concepts and practical applications of data processing architectures. It's an opportunity to clarify doubts, discuss scenarios, and deepen our understanding of how data infrastructure supports modern applications.
\end{frame}

\begin{frame}[fragile]
  \frametitle{Key Concepts to Discuss}
  \begin{enumerate}
    \item \textbf{Definition of Data Processing Architecture}
      \begin{itemize}
        \item A framework that dictates how data is ingested, processed, and analyzed.
        \item Integral to informing how workflows are designed and how systems interact.
      \end{itemize}

    \item \textbf{Types of Data Processing Architectures}
      \begin{itemize}
        \item \textbf{Batch Processing}: Processes large volumes of data at once. Ideal for scheduled tasks, e.g., payroll.
        \item \textbf{Stream Processing}: Handles continuous data flows in real-time. Common in applications like fraud detection.
        \item \textbf{Hybrid Processing}: Combines batch and stream processing for flexibility.
      \end{itemize}
  \end{enumerate}
\end{frame}

\begin{frame}[fragile]
  \frametitle{Examples of Data Processing Architectures}
  \begin{itemize}
    \item \textbf{Lambda Architecture}: Combines batch and real-time processing for comprehensive insights.
      \begin{itemize}
        \item \textbf{Components}:
          \begin{itemize}
            \item \textbf{Batch layer}: Manages the master dataset and pre-computes batch views.
            \item \textbf{Speed layer}: Processes real-time data and updates results quickly.
            \item \textbf{Serving layer}: Merges views from the batch and speed layers for complete insights.
          \end{itemize}
      \end{itemize}

    \item \textbf{Kappa Architecture}: Simplified architecture that processes all data as a stream, eliminating the batch layer.
  \end{itemize}
\end{frame}

\begin{frame}[fragile]
  \frametitle{Discussion Points}
  \begin{itemize}
    \item \textbf{Choosing the Right Architecture:} Factors include data volume, velocity, variety, and business needs.
    \item \textbf{Integration of APIs and Tools:}
      \begin{itemize}
        \item Importance of understanding tools like Apache Kafka, Spark, or AWS Lambda within architectures.
        \item Real-world implications of incorrect tool choices causing performance bottlenecks.
      \end{itemize}
    \item \textbf{Real-World Applications:}
      \begin{itemize}
        \item E-commerce transaction analysis (batch processing) vs. live sports updates (stream processing).
      \end{itemize}
  \end{itemize}
\end{frame}

\begin{frame}[fragile]
  \frametitle{Open Floor for Questions}
  Please feel free to ask about:
  \begin{itemize}
    \item Specific architectures, their pros and cons
    \item Integration practices
    \item Any concepts we have discussed this week
  \end{itemize}
  
  \textbf{Key Points to Emphasize:}
  \begin{itemize}
    \item Understanding the underlying architecture is crucial for effective data processing.
    \item The choice of architecture affects performance, scalability, and utility.
    \item Engaging with current technologies shapes your skills in real-world applications.
  \end{itemize}
\end{frame}

\begin{frame}[fragile]
  \frametitle{Conclusion}
  Utilize this opportunity to clarify doubts, expand on examples, and share insights or experiences related to data processing architectures. Let's deepen our understanding together!
\end{frame}


\end{document}