\documentclass[aspectratio=169]{beamer}

% Theme and Color Setup
\usetheme{Madrid}
\usecolortheme{whale}
\useinnertheme{rectangles}
\useoutertheme{miniframes}

% Additional Packages
\usepackage[utf8]{inputenc}
\usepackage[T1]{fontenc}
\usepackage{graphicx}
\usepackage{booktabs}
\usepackage{listings}
\usepackage{amsmath}
\usepackage{amssymb}
\usepackage{xcolor}
\usepackage{tikz}
\usepackage{pgfplots}
\pgfplotsset{compat=1.18}
\usetikzlibrary{positioning}
\usepackage{hyperref}

% Custom Colors
\definecolor{myblue}{RGB}{31, 73, 125}
\definecolor{mygray}{RGB}{100, 100, 100}
\definecolor{mygreen}{RGB}{0, 128, 0}
\definecolor{myorange}{RGB}{230, 126, 34}
\definecolor{mycodebackground}{RGB}{245, 245, 245}

% Set Theme Colors
\setbeamercolor{structure}{fg=myblue}
\setbeamercolor{frametitle}{fg=white, bg=myblue}
\setbeamercolor{title}{fg=myblue}
\setbeamercolor{section in toc}{fg=myblue}
\setbeamercolor{item projected}{fg=white, bg=myblue}
\setbeamercolor{block title}{bg=myblue!20, fg=myblue}
\setbeamercolor{block body}{bg=myblue!10}
\setbeamercolor{alerted text}{fg=myorange}

% Set Fonts
\setbeamerfont{title}{size=\Large, series=\bfseries}
\setbeamerfont{frametitle}{size=\large, series=\bfseries}
\setbeamerfont{caption}{size=\small}
\setbeamerfont{footnote}{size=\tiny}

% Footer and Navigation Setup
\setbeamertemplate{footline}{
  \leavevmode%
  \hbox{%
  \begin{beamercolorbox}[wd=.3\paperwidth,ht=2.25ex,dp=1ex,center]{author in head/foot}%
    \usebeamerfont{author in head/foot}\insertshortauthor
  \end{beamercolorbox}%
  \begin{beamercolorbox}[wd=.5\paperwidth,ht=2.25ex,dp=1ex,center]{title in head/foot}%
    \usebeamerfont{title in head/foot}\insertshorttitle
  \end{beamercolorbox}%
  \begin{beamercolorbox}[wd=.2\paperwidth,ht=2.25ex,dp=1ex,center]{date in head/foot}%
    \usebeamerfont{date in head/foot}
    \insertframenumber{} / \inserttotalframenumber
  \end{beamercolorbox}}%
  \vskip0pt%
}
\setbeamertemplate{navigation symbols}{}

% Title Page Information
\title[Classification Techniques]{Week 4: Classification Techniques}
\author[J. Smith]{John Smith, Ph.D.}
\institute[University Name]{
  Department of Computer Science\\
  University Name\\
  \vspace{0.3cm}
  Email: email@university.edu\\
  Website: www.university.edu
}
\date{\today}

% Document Start
\begin{document}

\frame{\titlepage}

\begin{frame}[fragile]
    \frametitle{Introduction to Classification Techniques}
    \begin{block}{Overview}
        Classification techniques are crucial in data mining as they enable the analysis and prediction of categorical outcomes based on input data. 
        We will introduce three primary techniques:
        \begin{itemize}
            \item Decision Trees
            \item Naive Bayes
            \item Support Vector Machines (SVM)
        \end{itemize}
    \end{block}
\end{frame}

\begin{frame}[fragile]
    \frametitle{Decision Trees}
    \begin{block}{Definition}
        A decision tree is a flowchart-like structure where:
        \begin{itemize}
            \item Each internal node represents a test on an attribute.
            \item Each branch corresponds to the outcome of the test.
            \item Each leaf node represents a class label.
        \end{itemize}
    \end{block}

    \begin{block}{How It Works}
        \begin{itemize}
            \item The tree is constructed by splitting the dataset based on feature values providing the highest information gain or greatest reduction in impurity.
        \end{itemize}
    \end{block}

    \begin{block}{Example}
        Predicting if a student passes or fails based on hours studied and prior grades.
    \end{block}

    \begin{block}{Key Points}
        \begin{itemize}
            \item Easy to interpret and visualize.
            \item Prone to overfitting without pruning techniques.
        \end{itemize}
    \end{block}
\end{frame}

\begin{frame}[fragile]
    \frametitle{Naive Bayes}
    \begin{block}{Definition}
        A family of probabilistic algorithms based on Bayes' theorem, assuming independence among predictors.
    \end{block}

    \begin{block}{How It Works}
        \begin{equation}
            P(Class|Features) = \frac{P(Features|Class) \times P(Class)}{P(Features)}
        \end{equation}
    \end{block}

    \begin{block}{Example}
        Classifying emails as spam or not based on the presence of specific keywords.
    \end{block}

    \begin{block}{Key Points}
        \begin{itemize}
            \item Performs well with large datasets.
            \item Assumes independence between features.
        \end{itemize}
    \end{block}
\end{frame}

\begin{frame}[fragile]
    \frametitle{Support Vector Machines (SVM)}
    \begin{block}{Definition}
        A classification technique that finds the optimal hyperplane maximizing the margin between different classes.
    \end{block}

    \begin{block}{How It Works}
        \begin{itemize}
            \item Transforms data into a higher dimension to find the hyperplane that best separates classes.
        \end{itemize}
    \end{block}

    \begin{block}{Example}
        Imagine a 2D plane with red and blue points; SVM finds the best separating line with the largest margin.
    \end{block}
    
    \begin{block}{Key Points}
        \begin{itemize}
            \item Effective in high-dimensional spaces.
            \item Best when classes are distinct; may struggle with overlap.
        \end{itemize}
    \end{block}
\end{frame}

\begin{frame}[fragile]
    \frametitle{Conclusion and Next Steps}
    \begin{block}{Conclusion}
        Understanding classification techniques enhances your ability to select the appropriate method for various data mining problems. Each technique has distinct use cases and parameters.
    \end{block}

    \begin{block}{Next Steps}
        In the following slides, we will delve deeper into each technique, exploring their mechanisms, advantages, disadvantages, and practical applications.
    \end{block}
\end{frame}

\begin{frame}[fragile]
    \frametitle{What are Classification Techniques?}
    \begin{block}{Definition}
        Classification techniques are a subset of data mining methods that assign labels or categories to instances based on their characteristics. The primary goal is to create a model that can predict the class label for new, unseen data based on the patterns learned from a training dataset.
    \end{block}
\end{frame}

\begin{frame}[fragile]
    \frametitle{Context in the Data Mining Process}
    \begin{itemize}
        \item \textbf{Data Preprocessing}: Cleaning, normalizing, and transforming the data into a suitable format.
        \item \textbf{Model Building}: Applying classification algorithms, such as Decision Trees, Naive Bayes, and Support Vector Machines, to build a model.
        \item \textbf{Training and Testing}: Dividing data into training and testing sets to train the model and evaluate performance.
        \item \textbf{Deployment and Monitoring}: Deploying the model for live predictions and continuously monitoring its performance.
    \end{itemize}
\end{frame}

\begin{frame}[fragile]
    \frametitle{Key Points and Example of Classification}
    \begin{enumerate}
        \item \textbf{Supervised Learning}: Classification is a type of supervised learning trained with labeled data.
        \item \textbf{Use Cases}: Examples include spam detection, sentiment analysis, medical diagnosis, and credit scoring.
        \item \textbf{Performance Evaluation}: Models are evaluated using metrics like accuracy, precision, recall, and F1 score.
    \end{enumerate}
    
    \begin{block}{Example}
        Suppose we have a dataset of emails labeled as either "spam" or "not spam". Using classification techniques, we develop a model to identify spam characteristics (e.g., words and sender domains) and classify incoming emails into the respective categories.
    \end{block}
\end{frame}

\begin{frame}[fragile]
    \frametitle{Decision Trees - Overview}
    
    \begin{block}{What are Decision Trees?}
        A Decision Tree is a graphical representation used for making decisions and predictions in classification problems. It involves breaking down a dataset into smaller subsets while developing a tree incrementally. They are easy to visualize and interpret, making them popular for both beginners and seasoned data scientists.
    \end{block}
\end{frame}

\begin{frame}[fragile]
    \frametitle{Decision Trees - Structure}
    
    \begin{itemize}
        \item \textbf{Nodes}: Represent features (attribute) or decision points.
            \begin{itemize}
                \item \textbf{Root Node}: Topmost node representing the entire dataset, divided into sub-nodes.
                \item \textbf{Internal Nodes}: Represent features and indicate a test for splitting the data.
                \item \textbf{Leaf Nodes}: End nodes indicating the outcome or class label.
            \end{itemize}
    \end{itemize}

    \begin{block}{Visualization of a Decision Tree}
        \begin{verbatim}
              [Outlook]
             /    |    \
          Sunny  Rainy  Overcast
          /        |        
       [Humidity]  No
        /     \
      High   Normal
       |       |
      No      Yes
        \end{verbatim}
    \end{block}
\end{frame}

\begin{frame}[fragile]
    \frametitle{Decision Trees - How They Work}

    \begin{enumerate}
        \item \textbf{Selecting the Best Feature to Split}: Begin at root node, selecting the best feature using criteria like Gini impurity or Information Gain.
        \item \textbf{Splitting}: Data is divided based on the selected feature, forming branches leading to further nodes.
        \item \textbf{Stopping Criteria}: Continue recursively splitting until all data points belong to a class (leaf node) or a stopping criterion is met (e.g., maximum depth).
    \end{enumerate}

    \begin{block}{Example Dataset}
        Consider a dataset determining whether to play outside based on weather conditions:
        
        \begin{tabular}{|c|c|c|c|c|}
            \hline
            Outlook  & Temperature & Humidity & Windy & Play \\
            \hline
            Sunny    & Hot        & High     & False & No   \\
            Sunny    & Hot        & High     & True  & No   \\
            Overcast & Hot        & High     & False & Yes  \\
            Rainy    & Mild       & High     & False & Yes  \\
            Rainy    & Cool       & Normal   & False & Yes  \\
            Rainy    & Cool       & Normal   & True  & No   \\
            Overcast & Cool       & Normal   & True  & Yes  \\
            \hline
        \end{tabular}
    \end{block}
\end{frame}

\begin{frame}[fragile]
    \frametitle{Decision Trees - Advantages and Disadvantages}
    
    \begin{block}{Advantages}
        \begin{itemize}
            \item Easy to understand and interpret.
            \item Handles both numerical and categorical data without need for normalization.
            \item Easily visualizable, understandable by non-experts.
            \item Flexible in modeling non-linear relationships.
        \end{itemize}
    \end{block}

    \begin{block}{Disadvantages}
        \begin{itemize}
            \item Can overfit the data, capturing noise rather than patterns.
            \item Sensitive to small changes in data leading to different trees.
            \item Bias towards features with more levels or categories.
        \end{itemize}
    \end{block}
\end{frame}

\begin{frame}[fragile]
    \frametitle{How Decision Trees Work - Overview}
    \begin{block}{Understanding Decision Trees}
        Decision Trees are a popular and intuitive method for classification and regression tasks in machine learning. They model decisions and their possible consequences, resembling a tree structure. Each node represents a feature, each branch represents a decision rule, and each leaf node represents an outcome.
    \end{block}
\end{frame}

\begin{frame}[fragile]
    \frametitle{How Decision Trees Work - Key Concepts}
    \begin{enumerate}
        \item \textbf{Splitting Criteria}
        \begin{itemize}
            \item \textbf{Definition:} Determine how to divide the dataset to maximize the separation of classes.
            \item \textbf{Common Methods:}
            \begin{itemize}
                \item \textbf{Gini Impurity:}
                \begin{equation}
                    Gini(D) = 1 - \sum (p_i^2)
                \end{equation}
                \item \textbf{Entropy:}
                \begin{equation}
                    Entropy(D) = -\sum (p_i \log_2 p_i)
                \end{equation}
                \item \textbf{Information Gain:} Decrease in entropy after splitting on an attribute.
            \end{itemize}
        \end{itemize}
    \end{enumerate}
\end{frame}

\begin{frame}[fragile]
    \frametitle{How Decision Trees Work - Tree Construction}
    \begin{enumerate}
        \item \textbf{Choosing the Best Features}
        \begin{itemize}
            \item Importance: Select features based on their predictive power.
            \item Pruning: Trim branches that add complexity without improving accuracy.
        \end{itemize}
        
        \item \textbf{Tree Construction Process}
        \begin{itemize}
            \item Recursive Partitioning:
            \begin{enumerate}
                \item Start with the whole dataset at the root.
                \item For each node:
                \begin{itemize}
                    \item Apply splitting criterion.
                    \item Create branches for each outcome.
                    \item Repeat recursively until certain conditions are met.
                \end{itemize}
            \end{enumerate}
        \end{itemize}
    \end{enumerate}
\end{frame}

\begin{frame}[fragile]
    \frametitle{How Decision Trees Work - Example}
    \begin{block}{Example of Decision Tree Construction}
        \textbf{Dataset:} Features include Weather, Temperature, and Outcome (Play).
    \end{block}
    \begin{itemize}
        \item \textbf{Step 1:} Initial Split using Weather: Sunny, Overcast, Rainy.
        \item \textbf{Step 2:} For each category, apply Gini Impurity or Entropy to decide on Temperature.
        \item \textbf{Step 3:} Final Leaf Nodes represent options like Play (Yes) or Don't Play (No).
    \end{itemize}
\end{frame}

\begin{frame}[fragile]
    \frametitle{How Decision Trees Work - Key Points and Conclusion}
    \begin{itemize}
        \item Decision Trees are interpretable and easy to visualize.
        \item Prone to overfitting; consider pruning or constraints.
        \item Regularly assess model performance using validation techniques.
    \end{itemize}
    \begin{block}{Conclusion}
        Understanding Decision Trees helps leverage this method for classification and regression tasks effectively.
    \end{block}
\end{frame}

\begin{frame}[fragile]
  \frametitle{Applications of Decision Trees - Introduction}
  \begin{block}{Introduction to Decision Trees}
    Decision trees are versatile and easily interpretable machine learning models that can be used for both classification and regression tasks. 
    The tree structure breaks down a dataset into smaller subsets, making decisions based on feature values.
  \end{block}
\end{frame}

\begin{frame}[fragile]
  \frametitle{Applications of Decision Trees - Overview}
  \begin{block}{Practical Applications of Decision Trees}
    Here are some notable examples where decision trees are effectively utilized:
  \end{block}
\end{frame}

\begin{frame}[fragile]
  \frametitle{Applications of Decision Trees - Examples}
  \begin{enumerate}
    \item \textbf{Medical Diagnosis:}
      \begin{itemize}
        \item Used to diagnose diseases based on symptoms and medical history.
        \item Example: A tree splitting on "Fever" leads to further questions guiding to a diagnosis.
        \item \textit{Benefit:} Assists doctors in decision-making.
      \end{itemize}
      
    \item \textbf{Customer Churn Prediction:}
      \begin{itemize}
        \item Used to identify customers likely to leave services.
        \item Analyzes contract length, usage, etc.
        \item \textit{Benefit:} Targets at-risk customers with tailored strategies.
      \end{itemize}

    \item \textbf{Credit Risk Assessment:}
      \begin{itemize}
        \item Employed to assess the creditworthiness of loan applicants.
        \item Factors include income and credit history.
        \item \textit{Benefit:} Faster loan approvals and reduced defaults.
      \end{itemize}

    \item \textbf{Fraud Detection:}
      \begin{itemize}
        \item Analyzes transaction data to flag potentially fraudulent activities.
        \item Evaluates patterns such as transaction amount and frequency.
        \item \textit{Benefit:} Enhanced security by flagging unusual behavior.
      \end{itemize}

    \item \textbf{Marketing Campaign Analysis:}
      \begin{itemize}
        \item Helps evaluate the effectiveness of marketing strategies.
        \item Analyzes customer responses based on demographics.
        \item \textit{Benefit:} Optimizes resource allocation on high-impact campaigns.
      \end{itemize}
  \end{enumerate}
\end{frame}

\begin{frame}[fragile]
  \frametitle{Applications of Decision Trees - Key Points and Conclusion}
  \begin{block}{Key Points to Emphasize}
    \begin{itemize}
      \item Provide simple visualizations for decision-making.
      \item Their interpretability builds trust in AI-driven decisions.
      \item Handle both numerical and categorical data.
    \end{itemize}
  \end{block}

  \begin{block}{Conclusion}
    Decision trees are a powerful tool for analytical decision-making across sectors, offering clear insights that enhance understanding and facilitate strategic actions.
  \end{block}
\end{frame}

\begin{frame}[fragile]
  \frametitle{Naive Bayes Classifier - Introduction}
  \begin{block}{Introduction}
    The Naive Bayes classifier is a family of probabilistic algorithms based on applying Bayes' theorem with strong (naive) independence assumptions. 
    It's widely used in various applications, particularly in text classification, such as spam detection and sentiment analysis.
  \end{block}
\end{frame}

\begin{frame}[fragile]
  \frametitle{Naive Bayes Classifier - Key Concepts}
  \begin{enumerate}
    \item \textbf{Bayes' Theorem}: 
    \begin{equation}
      P(A | B) = \frac{P(B | A) \cdot P(A)}{P(B)}
    \end{equation}
    \begin{itemize}
      \item \textbf{P(A | B)}: Probability of event A occurring given that B is true.
      \item \textbf{P(B | A)}: Probability of event B occurring given that A is true.
      \item \textbf{P(A)}: Probability of event A.
      \item \textbf{P(B)}: Probability of event B.
    \end{itemize}

    \item \textbf{Naive Assumption}: 
    The "naive" assumption involves assuming that the presence of one feature in a class is independent of the presence of any other feature. 
    This simplifies computations significantly but may not hold true in practice.
  \end{enumerate}
\end{frame}

\begin{frame}[fragile]
  \frametitle{Naive Bayes Classifier - Assumptions and Calculations}
  \begin{block}{Assumptions}
    \begin{itemize}
      \item \textbf{Feature Independence}: Each feature contributes independently to the final classification.
      \item \textbf{Feature Presence}: Features can be either present or absent, effective in text classification.
    \end{itemize}
  \end{block}

  \begin{block}{Calculating Probabilities}
    \begin{enumerate}
      \item \textbf{Training Phase}:
      \begin{itemize}
        \item Calculate prior probabilities of each class.
        \item Calculate conditional probabilities for each feature given a class.
      \end{itemize}
      \item \textbf{Prediction Phase}:
      \begin{itemize}
        \item Compute posterior probability for each class using Bayes' theorem.
        \item Select class with the highest posterior probability as output.
      \end{itemize}
    \end{enumerate}
  \end{block}
\end{frame}

\begin{frame}[fragile]
  \frametitle{Naive Bayes Classifier - Example}
  \begin{block}{Example}
    Let's classify an email as Spam or Not Spam based on keywords:

    \begin{itemize}
      \item \textbf{Training Data}:
      \begin{itemize}
        \item Spam: [“Buy now”, “Limited offer”, “Free gift”]
        \item Not Spam: [“Meeting at 10am”, “Project deadline”, “Thank you”]
      \end{itemize}
      
      \item \textbf{Calculations}:
      \begin{itemize}
        \item Prior probabilities:
        \begin{itemize}
          \item P(Spam) = 3/6
          \item P(Not Spam) = 3/6
        \end{itemize}
        \item Conditional probabilities:
        \begin{itemize}
          \item P(“Buy now” | Spam) = 1/3
          \item P(“Buy now” | Not Spam) = 0/3
        \end{itemize}
        \item To classify a new email containing “Buy now”, calculate P(Spam | “Buy now”) and P(Not Spam | “Buy now”).
      \end{itemize}
    \end{itemize}
  \end{block}
\end{frame}

\begin{frame}[fragile]
    \frametitle{Understanding Naive Bayes}
    \begin{block}{What is Naive Bayes?}
        Naive Bayes is a family of probabilistic classifiers based on Bayes’ Theorem. It assumes independence among predictors and performs well in text classification.
    \end{block}
\end{frame}

\begin{frame}[fragile]
    \frametitle{How Does Naive Bayes Work?}
    \begin{block}{Bayes' Theorem}
        The key equation:
        \[
        P(C|X) = \frac{P(X|C) \cdot P(C)}{P(X)}
        \]
        Where:
        \begin{itemize}
            \item \(P(C|X)\): Probability of class \(C\) given features \(X\)
            \item \(P(X|C)\): Probability of feature set \(X\) given class \(C\)
            \item \(P(C)\): Prior probability of class \(C\)
            \item \(P(X)\): Probability of feature set \(X\)
        \end{itemize}
    \end{block}
\end{frame}

\begin{frame}[fragile]
    \frametitle{Key Assumptions and Applications}
    \begin{itemize}
        \item \textbf{Feature Independence}: Features are independent given the class label.
        \item \textbf{Text Classification Example}:
        \begin{enumerate}
            \item Data Collection: Using labeled emails (Spam/Not Spam).
            \item Feature Extraction: Identify indicative words.
            \item Calculate Probabilities: Determine \(P(Spam)\), \(P(Not Spam)\), and likelihoods \(P(Word|Spam)\), \(P(Word|Not Spam)\).
            \item Making Predictions: Classify new emails based on calculated probabilities.
        \end{enumerate}
    \end{itemize}
\end{frame}

\begin{frame}[fragile]
    \frametitle{Key Points and Conclusion}
    \begin{itemize}
        \item \textbf{Efficiency}: Computationally efficient for large datasets.
        \item \textbf{Simplicity}: Easy to implement and interpret.
        \item \textbf{Good Baseline}: Excellent benchmark against more complex models.
    \end{itemize}
    
    \begin{block}{Conclusion}
        Naive Bayes is a powerful tool for classification tasks and serves as a foundation for exploring more complex algorithms in data science.
    \end{block}
\end{frame}

\begin{frame}[fragile]
    \frametitle{Advantages of Naive Bayes}
    \begin{enumerate}
        \item \textbf{Simplicity and Efficiency}:
        \begin{itemize}
            \item Easy to understand and implement.
            \item Computationally efficient with minimal data requirements.
            \item \textit{Example}: Rapidly classifies emails as spam or not spam.
        \end{itemize}
        
        \item \textbf{Fast Training and Prediction}:
        \begin{itemize}
            \item Quick training due to feature independence assumption.
            \item \textit{Illustration}: Training on 10,000 documents is faster than SVM or neural networks.
        \end{itemize}
        
        \item \textbf{Works Well with High Dimensional Data}:
        \begin{itemize}
            \item Effective when features exceed observations—typical in text classification.
        \end{itemize}

        \item \textbf{Robustness to Irrelevant Features}:
        \begin{itemize}
            \item Can effectively manage irrelevant attributes, simplifying the model.
        \end{itemize}
        
        \item \textbf{Good Performance with Small Datasets}:
        \begin{itemize}
            \item Provides reliable estimates even with limited data.
        \end{itemize}
    \end{enumerate}
\end{frame}

\begin{frame}[fragile]
    \frametitle{Limitations of Naive Bayes}
    \begin{enumerate}
        \item \textbf{Assumption of Feature Independence}:
        \begin{itemize}
            \item Often unrealistic in practice, leading to suboptimal performance.
            \item \textit{Example}: In sentiment analysis, word dependency affects meaning (e.g., "not good").
        \end{itemize}

        \item \textbf{Zero Probability Problem}:
        \begin{itemize}
            \item Occurs when a feature is absent in training data for a class.
            \item \textit{Solution}: Laplace smoothing to adjust probabilities.
        \end{itemize}
        
        \item \textbf{Limited Expressiveness}:
        \begin{itemize}
            \item Cannot capture complex feature relationships.
        \end{itemize}
        
        \item \textbf{Preference for Certain Classes}:
        \begin{itemize}
            \item Might favor majority classes in imbalanced datasets, leading to bias.
        \end{itemize}
    \end{enumerate}
\end{frame}

\begin{frame}[fragile]
    \frametitle{Summary and Key Points}
    \begin{block}{Summary}
        Naive Bayes is a powerful and efficient classifier, particularly effective for text data. However, its assumptions and limitations should be carefully considered to ensure effective application in classification tasks.
    \end{block}

    \begin{itemize}
        \item \textbf{Simplicity}: Easy to implement and understand.
        \item \textbf{Speed}: Quick training and prediction.
        \item \textbf{Independence Assumption}: Critical for performance; often flawed in real applications.
        \item \textbf{Use Cases}: Particularly suited for text classification (e.g., emails, documents).
    \end{itemize}

    \begin{block}{Conclusion}
        Understanding both strengths and limitations of Naive Bayes is crucial for its effective application.
    \end{block}
\end{frame}

\begin{frame}[fragile]
    \frametitle{Support Vector Machines (SVM) - Introduction}
    \begin{itemize}
        \item Support Vector Machines (SVM) are powerful classification algorithms used in machine learning.
        \item Particularly effective in high-dimensional spaces and scenarios with more dimensions than samples.
        \item \textbf{Purpose}: To find the optimal hyperplane that separates different classes in a dataset.
    \end{itemize}
\end{frame}

\begin{frame}[fragile]
    \frametitle{Support Vector Machines (SVM) - Hyperplanes}
    \begin{block}{Basic Concept of Hyperplanes}
        \begin{itemize}
            \item \textbf{Definition}: In an N-dimensional space, a hyperplane is a flat affine subspace of dimension \(N-1\).
            \item It serves as a decision boundary dividing the space into two parts, each for a different class.
            \item \textbf{Geometric Representation}:
            \begin{itemize}
                \item 2D: A line.
                \item 3D: A plane.
                \item Higher dimensions: Conceptually remains a boundary for categorizing data points.
            \end{itemize}
        \end{itemize}
    \end{block}
    
    \begin{equation}
        w \cdot x + b = 0
    \end{equation}
    where:
    \begin{itemize}
        \item \(w\): Weight vector (normal to the hyperplane)
        \item \(x\): Feature vector of a data point
        \item \(b\): Bias term (a constant)
    \end{itemize}
\end{frame}

\begin{frame}[fragile]
    \frametitle{Support Vector Machines (SVM) - Key Points}
    \begin{enumerate}
        \item \textbf{Maximal Margin}:
        \begin{itemize}
            \item SVM maximizes the distance (margin) between the hyperplane and nearest data points (support vectors).
            \item A larger margin is associated with better generalization.
        \end{itemize}
        
        \item \textbf{Support Vectors}:
        \begin{itemize}
            \item Nearest data points to the hyperplane, critical for defining the hyperplane's position and orientation.
        \end{itemize}
        
        \item \textbf{Example}:
        \begin{itemize}
            \item Two classes represented in 2D:
            \begin{itemize}
                \item Class A: Blue points
                \item Class B: Red points
            \end{itemize}
            \item SVM finds the line (hyperplane) that separates these classes while maximizing the margin.
        \end{itemize}
    \end{enumerate}
\end{frame}

\begin{frame}[fragile]
    \frametitle{How SVM Works - Overview}
    % This frame provides a brief overview of Support Vector Machines (SVM)
    \begin{block}{Understanding Support Vector Machines (SVM)}
        Support Vector Machines (SVM) are powerful supervised learning models used for classification and regression tasks. The central concept of SVM involves finding the optimal hyperplane that separates classes in the feature space.
    \end{block}
\end{frame}

\begin{frame}[fragile]
    \frametitle{How SVM Works - Training Steps}
    % This frame outlines the key steps in training an SVM
    \begin{block}{Training SVM: Key Steps}
        \begin{enumerate}
            \item \textbf{Data Preparation}:
            \begin{itemize}
                \item Input labeled data points with features and class labels.
                \item Example: For classifying fruits, features could include weight and color, while labels are "apple," "banana," etc.
            \end{itemize}
            
            \item \textbf{Choosing the Hyperplane}:
            \begin{itemize}
                \item A hyperplane is a decision boundary separating different classes; it is a line in 2D, a plane in 3D.
                \item SVM aims to maximize the margin – the distance between the hyperplane and the nearest data points (support vectors).
            \end{itemize}
            
            \item \textbf{Optimization Problem}:
            \begin{equation}
            \text{Minimize } \frac{1}{2} \| \mathbf{w} \|^2
            \end{equation}
            \begin{equation}
            \text{Subject to } y_i(\mathbf{w}^T \mathbf{x}_i + b) \geq 1, \forall i
            \end{equation}
        \end{enumerate}
    \end{block}
\end{frame}

\begin{frame}[fragile]
    \frametitle{How SVM Works - Role of Kernel Functions}
    % This frame explains the role of kernel functions in SVM
    \begin{block}{Role of Kernel Functions}
        Kernel functions allow SVM to operate in higher-dimensional spaces without explicitly mapping data points, accommodating non-linearly separable data.
        
        \begin{itemize}
            \item \textbf{Linear Kernel}:
            \begin{equation}
            K(x_i, x_j) = x_i^T x_j
            \end{equation}
            Suitable for linearly separable data.
            
            \item \textbf{Polynomial Kernel}:
            \begin{equation}
            K(x_i, x_j) = (x_i^T x_j + c)^d
            \end{equation}
            Useful for polynomial relationships in data.

            \item \textbf{Radial Basis Function (RBF) Kernel}:
            \begin{equation}
            K(x_i, x_j) = e^{-\gamma \|x_i - x_j\|^2}
            \end{equation}
            Effectively handles non-linear relationships; \(\gamma\) controls individual sample influence.
        \end{itemize}
    \end{block}
\end{frame}

\begin{frame}[fragile]
    \frametitle{Applications of SVM - Introduction}
    \begin{itemize}
        \item Support Vector Machines (SVM) are powerful supervised learning algorithms.
        \item Commonly used for classification and regression tasks.
        \item They create optimal hyperplanes, making them suitable for various applications.
        \item High accuracy is critical in these real-world applications.
    \end{itemize}
\end{frame}

\begin{frame}[fragile]
    \frametitle{Applications of SVM - Key Applications}
    \begin{enumerate}
        \item \textbf{Text Classification}
        \begin{itemize}
            \item \textit{Example: Spam Detection}
            \item Classifies emails as "spam" or "not spam" based on keyword frequency.
            \item Trained on labeled datasets to find optimal hyperplane for separation.
        \end{itemize}
        
        \item \textbf{Image Recognition}
        \begin{itemize}
            \item \textit{Example: Handwritten Digit Recognition}
            \item Differentiates between digits (0-9) using pixel data as features.
            \item SVM finds the optimal separating hyperplane in high-dimensional space.
        \end{itemize}
        
        \item \textbf{Biological Classification}
        \begin{itemize}
            \item \textit{Example: Cancer Diagnosis}
            \item Classifies tissue samples using gene expression profiles.
            \item Aids in quick and accurate diagnoses by distinguishing cell types.
        \end{itemize}
    \end{enumerate}
\end{frame}

\begin{frame}[fragile]
    \frametitle{Applications of SVM - Additional Applications}
    \begin{enumerate}[resume]
        \item \textbf{Face Detection}
        \begin{itemize}
            \item \textit{Example: Facial Recognition Technology}
            \item Differentiates human faces from non-faces in images.
            \item Trained on various facial features to capture complex patterns.
        \end{itemize}

        \item \textbf{Financial Forecasting}
        \begin{itemize}
            \item \textit{Example: Stock Market Prediction}
            \item Analyzes historical stock prices and volumes to predict trends.
            \item Supports traders in decision-making by identifying market patterns.
        \end{itemize}
    \end{enumerate}
\end{frame}

\begin{frame}[fragile]
    \frametitle{Applications of SVM - Key Points and Conclusion}
    \begin{block}{Key Points to Emphasize}
        \begin{itemize}
            \item \textbf{High Dimensional Data Handling:} Works effectively in many features.
            \item \textbf{Versatility:} Applicable in fields like healthcare and finance.
            \item \textbf{Kernel Trick:} Enables efficient handling of non-linear data for better class separation.
        \end{itemize}
    \end{block}
    
    \begin{block}{Conclusion}
        Support Vector Machines play a critical role in machine learning applications across various industries, managing complex datasets and making accurate predictions. 
        We will now compare SVM with other classification techniques to explore its strengths and weaknesses.
    \end{block}
\end{frame}

\begin{frame}[fragile]
    \frametitle{Comparative Analysis of Techniques}
    \textbf{Introduction to Classification Techniques} \\
    In machine learning, classification techniques are essential for predicting the category of new observations based on training data. This slide focuses on three common techniques: \textbf{Decision Trees}, \textbf{Naive Bayes}, and \textbf{Support Vector Machines (SVM)}. Each technique has its unique strengths and weaknesses, making them suited for different situations.
\end{frame}

\begin{frame}[fragile]
    \frametitle{Decision Trees}
    \begin{itemize}
        \item \textbf{Definition}: A flowchart-like structure where:
            \begin{itemize}
                \item each internal node represents a test on an attribute,
                \item each branch represents the outcome of the test,
                \item each leaf node represents a class label.
            \end{itemize}
        \item \textbf{Performance}:
            \begin{itemize}
                \item \textit{Pros}: Simple to understand and interpret; handles both numerical and categorical data; requires little data preprocessing.
                \item \textit{Cons}: Prone to overfitting, especially with deep trees; sensitive to noisy data.
            \end{itemize}
        \item \textbf{Usability}: 
            \begin{itemize}
                \item User-friendly visual representation.
                \item Great for initial analysis and feature importance assessment.
            \end{itemize}
        \item \textbf{Best Suited For}: Situations with a clear decision-making process, e.g., customer classification and risk assessment.
    \end{itemize}
\end{frame}

\begin{frame}[fragile]
    \frametitle{Naive Bayes}
    \begin{itemize}
        \item \textbf{Definition}: Based on Bayes' theorem, assumes features are independent given the class label.
        \item \textbf{Performance}:
            \begin{itemize}
                \item \textit{Pros}: Fast to train and predict; performs well on large datasets; effective for text classification.
                \item \textit{Cons}: Assumes feature independence; may not perform well when this assumption is violated.
            \end{itemize}
        \item \textbf{Usability}: 
            \begin{itemize}
                \item Minimal training time; few hyperparameters to tune; interpretable results.
            \end{itemize}
        \item \textbf{Best Suited For}: High-dimensional problems, e.g., email classification, sentiment analysis.
    \end{itemize}
\end{frame}

\begin{frame}[fragile]
    \frametitle{Support Vector Machines (SVM)}
    \begin{itemize}
        \item \textbf{Definition}: Constructs a hyperplane in high-dimensional space to separate different class labels.
        \item \textbf{Performance}:
            \begin{itemize}
                \item \textit{Pros}: Effective in high-dimensional spaces; robust against overfitting; effective with kernel tricks.
                \item \textit{Cons}: Less effective with large datasets; computationally intensive; requires careful parameter tuning.
            \end{itemize}
        \item \textbf{Usability}: Requires more understanding of parameters; visualizations are less intuitive.
        \item \textbf{Best Suited For}: Complex problems in high-dimensional datasets, e.g., image recognition.
    \end{itemize}
\end{frame}

\begin{frame}[fragile]
    \frametitle{Key Points to Emphasize}
    \begin{itemize}
        \item \textbf{Model Interpretation}: 
            \begin{itemize}
                \item Decision trees are easily interpretable.
                \item Naive Bayes is highly efficient for specific applications.
                \item SVM excels in complex, high-dimensional spaces.
            \end{itemize}
        \item \textbf{Vulnerability}: 
            \begin{itemize}
                \item Decision trees may overfit to noise.
                \item Naive Bayes assumes feature independence.
                \item SVM requires careful parameter tuning.
            \end{itemize}
        \item \textbf{Application Context}: 
            The choice of technique should align with the dataset characteristics and the problem's requirements.
    \end{itemize}
\end{frame}

\begin{frame}[fragile]
    \frametitle{Conclusion}
    Understanding the nuances of these classification techniques allows practitioners to choose the appropriate method based on specific use cases, performance requirements, and resource constraints. This slide aids in understanding the comparative landscape of classification techniques and prepares students to make informed decisions in their applications.
\end{frame}

\begin{frame}[fragile]
    \frametitle{Ethical Considerations - Introduction}
    \begin{block}{Introduction to Ethical Considerations}
        Classification techniques, such as decision trees, Naive Bayes, and Support Vector Machines (SVM), play a vital role in data analysis across various sectors (healthcare, finance, hiring, etc.). However, their application raises significant ethical concerns that must be addressed to prevent negative consequences.
    \end{block}
\end{frame}

\begin{frame}[fragile]
    \frametitle{Ethical Considerations - Implications}
    \begin{block}{Ethical Implications}
        \begin{enumerate}
            \item \textbf{Bias and Fairness}
                \begin{itemize}
                    \item Algorithms can reinforce societal biases if trained on biased datasets.
                    \item Example: Hiring algorithms favor certain genders or ethnicities based on historical patterns.
                    \item Key Point: Assess training data for bias to ensure fairness.
                \end{itemize}
            \item \textbf{Transparency and Accountability}
                \begin{itemize}
                    \item Complex models can act as “black boxes,” creating a lack of transparency.
                    \item Example: Healthcare models might hinder trust due to unclear decision-making.
                    \item Key Point: Model explainability is essential in high-stakes areas.
                \end{itemize}
            \item \textbf{Privacy Concerns}
                \begin{itemize}
                    \item Data collection for training can breach individual privacy.
                    \item Example: Predictive policing raises concerns about surveillance.
                    \item Key Point: Safeguard individual privacy rights.
                \end{itemize}
            \item \textbf{Informed Consent}
                \begin{itemize}
                    \item Users should understand how their data is used and consent.
                    \item Example: E-commerce users contribute to recommendation systems unknowingly.
                    \item Key Point: Ethical data usage includes transparent communication.
                \end{itemize}
        \end{enumerate}
    \end{block}
\end{frame}

\begin{frame}[fragile]
    \frametitle{Ethical Considerations - Implications Continued}
    \begin{block}{Ethical Implications Continued}
        \begin{enumerate}
            \setcounter{enumi}{4}
            \item \textbf{Impact on Society}
                \begin{itemize}
                    \item Deployment of classification can have social implications.
                    \item Example: Social media algorithms can create echo chambers and polarize opinions.
                    \item Key Point: Consider the societal impact of model deployment.
                \end{itemize}
        \end{enumerate}
    \end{block}
\end{frame}

\begin{frame}[fragile]
    \frametitle{Best Practices for Ethical Classification}
    \begin{block}{Best Practices}
        \begin{itemize}
            \item \textbf{Data Auditing}: Regularly check datasets for biases and representation.
            \item \textbf{Model Evaluation}: Use fairness indices and transparency scores beyond accuracy.
            \item \textbf{Stakeholder Involvement}: Engage community stakeholders in discussions.
            \item \textbf{Adopt Ethical Frameworks}: Follow ethical guidelines specific to the sector.
        \end{itemize}
    \end{block}
\end{frame}

\begin{frame}[fragile]
    \frametitle{Conclusion and Engagement}
    \begin{block}{Conclusion}
        Addressing ethical considerations in classification techniques is essential for responsible data science. By prioritizing bias reduction, transparency, privacy, consent, and societal impact, we can promote equitable outcomes and foster trust in automated systems.
    \end{block}
  
    \begin{block}{Discussion Questions}
        \begin{itemize}
            \item How can we ensure our models are fair?
            \item What steps would you take to communicate data usage to users effectively?
        \end{itemize}
    \end{block}
\end{frame}

\begin{frame}[fragile]
    \frametitle{Bias Detection Formula}
    \begin{block}{Bias Detection Formula}
        Consider a simple bias detection formula:
        \begin{equation}
            \text{Bias Score} = \frac{\text{Number of misclassifications for a group}}{\text{Total predictions for that group}}
        \end{equation}
        This score can help quantify bias in model predictions.
    \end{block}
\end{frame}

\begin{frame}[fragile]
    \frametitle{Summary and Key Takeaways - Overview}
    Classification Techniques are essential methods in data mining that involve predicting the categorical labels of new observations based on past data. They play a pivotal role in various domains, including:
    \begin{itemize}
        \item Finance
        \item Healthcare
        \item Marketing
        \item And more...
    \end{itemize}
\end{frame}

\begin{frame}[fragile]
    \frametitle{Summary and Key Takeaways - Key Concepts Recap}
    \begin{enumerate}
        \item \textbf{Definition of Classification:}
        \begin{itemize}
            \item A supervised learning approach where the model learns from labeled training data to categorize new instances.
        \end{itemize}

        \item \textbf{Types of Classification Algorithms:}
        \begin{itemize}
            \item \textbf{Decision Trees:} Model decisions in a tree-like structure. \\ 
            Easy to interpret and visualize.
            \begin{itemize}
                \item Example: Predicting loan approval based on credit score, income, etc.
            \end{itemize}
            \item \textbf{Naïve Bayes Classifier:} Based on Bayes' theorem with an independence assumption. \\
            Easy to implement.
            \begin{itemize}
                \item Example: Email spam detection.
            \end{itemize}
            \item \textbf{Support Vector Machines (SVM):} Finds the hyperplane that best separates different classes.
            \begin{itemize}
                \item Example: Image classification.
            \end{itemize}
            \item \textbf{K-Nearest Neighbors (KNN):} Classifies based on the labels of nearest neighbors.
            \begin{itemize}
                \item Example: Identifying fruits based on features.
            \end{itemize}
        \end{itemize}
    \end{enumerate}
\end{frame}

\begin{frame}[fragile]
    \frametitle{Summary and Key Takeaways - Evaluation Metrics}
    \begin{block}{Evaluation Metrics}
        \begin{itemize}
            \item \textbf{Accuracy:} Proportion of correctly classified instances.
            \item \textbf{Precision:} True positives divided by total predicted positives.
            \item \textbf{Recall (Sensitivity):} True positives divided by total actual positives.
            \item \textbf{F1 Score:} Harmonic mean of precision and recall; balances both metrics.
        \end{itemize}
    \end{block}

    \begin{block}{Formula}
        For evaluating a classification model's performance, the \textbf{F1 Score} is defined as:
        \begin{equation}
        F1 = 2 \times \frac{(Precision \times Recall)}{(Precision + Recall)}
        \end{equation}
    \end{block}
\end{frame}

\begin{frame}[fragile]
    \frametitle{Summary and Key Takeaways - Applications and Considerations}
    \begin{enumerate}
        \item \textbf{Applications in Data Mining:}
        \begin{itemize}
            \item Customer Segmentation
            \item Medical Diagnosis
            \item Fraud Detection
        \end{itemize}

        \item \textbf{Key Points to Emphasize:}
        \begin{itemize}
            \item Importance of Data Quality: High-quality, well-labeled data is crucial.
            \item Ethical Considerations: Address bias to avoid unfair predictions.
            \item Choosing the Right Technique: Depends on problem domain and dataset nature.
        \end{itemize}
        
        \item \textbf{Closing Thoughts:} Classification techniques are powerful tools that enable insightful predictions. A deep understanding will enhance responsible use in practical scenarios.
    \end{enumerate}
\end{frame}

\begin{frame}[fragile]
    \frametitle{Summary and Key Takeaways - Next Steps}
    Prepare for the following discussion on practical applications and clarify any doubts regarding classification techniques!
\end{frame}

\begin{frame}[fragile]
  \frametitle{Discussion and Questions}
  % Open the floor for questions and discussion regarding classification techniques and their applications.
\end{frame}

\begin{frame}[fragile]
  \frametitle{Overview of Classification Techniques}
  \begin{block}{Definition}
    Classification is a supervised learning method where an algorithm learns from a labeled dataset (training set) to predict the class labels for new data (test set).
  \end{block}

  \begin{block}{Common Classification Algorithms}
    \begin{itemize}
      \item \textbf{Decision Trees}: Tree-like models that make decisions based on asking a series of questions about the data features.
      \item \textbf{Support Vector Machines (SVM)}: Algorithms that find the hyperplane that best separates data into different classes.
      \item \textbf{K-Nearest Neighbors (KNN)}: Classifies data points based on the majority class of their nearest neighbors.
      \item \textbf{Logistic Regression}: Models the probability of the default class for binary classification.
      \item \textbf{Neural Networks}: Deep learning models particularly powerful for complex datasets.
    \end{itemize}
  \end{block}
\end{frame}

\begin{frame}[fragile]
  \frametitle{Applications of Classification}
  \begin{block}{Typical Applications}
    \begin{itemize}
      \item \textbf{Spam Detection}: Classifying emails as "spam" or "not spam".
      \item \textbf{Medical Diagnosis}: Using patient data to classify diseases, enabling timely treatment.
      \item \textbf{Credit Scoring}: Assessing whether an applicant is a good credit risk based on historical data.
      \item \textbf{Image Recognition}: Classifying images into categories like "cat", "dog", etc.
    \end{itemize}
  \end{block}

  \begin{block}{Example Scenario}
    Imagine building a model to classify whether an email is spam:
    \begin{itemize}
      \item \textbf{Input Features}: Presence of certain keywords, email metadata, and frequency of links.
      \item \textbf{Output Classes}: "Spam" and "Not Spam".
      \item A decision tree could be used where each node represents a question about these features.
    \end{itemize}
  \end{block}
\end{frame}

\begin{frame}[fragile]
  \frametitle{Key Discussion Points}
  \begin{enumerate}
    \item \textbf{Feature Selection}: The choice of features critically influences model accuracy.
    \item \textbf{Overfitting vs. Underfitting}: Balancing model complexity to avoid fitting noise while ensuring generalization.
    \item \textbf{Evaluation Metrics}: 
    \begin{itemize}
      \item \textbf{Accuracy}: Overall correctness of the model.
      \item \textbf{Precision and Recall}: Important for understanding trade-offs in imbalanced datasets.
      \item \textbf{F1 Score}: The harmonic mean of precision and recall for evaluating binary classification systems.
    \end{itemize}
  \end{enumerate}

  \begin{block}{Questions for Discussion}
    \begin{itemize}
      \item What challenges have you faced with classification tasks in your projects?
      \item How do different algorithms compare in terms of performance and application?
      \item Can you think of real-world scenarios where misclassification could have serious implications?
    \end{itemize}
  \end{block}
\end{frame}


\end{document}