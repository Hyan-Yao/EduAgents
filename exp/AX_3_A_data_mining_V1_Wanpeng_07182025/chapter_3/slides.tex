\documentclass[aspectratio=169]{beamer}

% Theme and Color Setup
\usetheme{Madrid}
\usecolortheme{whale}
\useinnertheme{rectangles}
\useoutertheme{miniframes}

% Additional Packages
\usepackage[utf8]{inputenc}
\usepackage[T1]{fontenc}
\usepackage{graphicx}
\usepackage{booktabs}
\usepackage{listings}
\usepackage{amsmath}
\usepackage{amssymb}
\usepackage{xcolor}
\usepackage{tikz}
\usepackage{pgfplots}
\pgfplotsset{compat=1.18}
\usetikzlibrary{positioning}
\usepackage{hyperref}

% Custom Colors
\definecolor{myblue}{RGB}{31, 73, 125}
\definecolor{mygray}{RGB}{100, 100, 100}
\definecolor{mygreen}{RGB}{0, 128, 0}
\definecolor{myorange}{RGB}{230, 126, 34}
\definecolor{mycodebackground}{RGB}{245, 245, 245}

% Set Theme Colors
\setbeamercolor{structure}{fg=myblue}
\setbeamercolor{frametitle}{fg=white, bg=myblue}
\setbeamercolor{title}{fg=myblue}
\setbeamercolor{section in toc}{fg=myblue}
\setbeamercolor{item projected}{fg=white, bg=myblue}
\setbeamercolor{block title}{bg=myblue!20, fg=myblue}
\setbeamercolor{block body}{bg=myblue!10}
\setbeamercolor{alerted text}{fg=myorange}

% Set Fonts
\setbeamerfont{title}{size=\Large, series=\bfseries}
\setbeamerfont{frametitle}{size=\large, series=\bfseries}
\setbeamerfont{caption}{size=\small}
\setbeamerfont{footnote}{size=\tiny}

% Document Start
\begin{document}

\frame{\titlepage}

\begin{frame}[fragile]
    \title{Week 3: Exploratory Data Analysis}
    \author{John Smith, Ph.D.}
    \date{\today}
    \maketitle
\end{frame}

\begin{frame}[fragile]
    \frametitle{Introduction to Exploratory Data Analysis (EDA)}
    
    \begin{block}{Overview of EDA}
        Exploratory Data Analysis (EDA) is a critical approach for analyzing datasets to summarize their main characteristics, often employing graphical representations.
    \end{block}
    
    \begin{itemize}
        \item Understand data distributions
        \item Identify patterns and relationships
        \item Spot anomalies or outliers
        \item Formulate hypotheses for further analysis
    \end{itemize}
\end{frame}

\begin{frame}[fragile]
    \frametitle{Importance of EDA in the Data Mining Process}
    
    EDA is a crucial step in the data mining process because:
    
    \begin{enumerate}
        \item \textbf{Data Understanding}: Helps data scientists understand the data to make informed decisions.
        \item \textbf{Data Cleaning}: Aids in identifying data quality issues, like missing values.
        \item \textbf{Feature Selection}: Visualizations help in identifying relevant features for modeling.
        \item \textbf{Hypothesis Generation}: Uncovers insights that can lead to new hypotheses.
    \end{enumerate}
\end{frame}

\begin{frame}[fragile]
    \frametitle{Techniques for Summarizing and Visualizing Data}
    
    Common techniques during EDA:
    
    \begin{itemize}
        \item \textbf{Descriptive Statistics}
            \begin{itemize}
                \item Mean, Median, Mode (Central Tendency)
                \item Standard Deviation and Variance (Spread)
            \end{itemize}
        \item \textbf{Data Visualization}
            \begin{itemize}
                \item Histograms
                \item Boxplots
                \item Scatter Plots
            \end{itemize}
        \item \textbf{Correlation Analysis}
            \begin{equation}
            r = \frac{Cov(X, Y)}{\sigma_x \sigma_y}
            \end{equation}
        \item \textbf{Data Transformation}
            \begin{itemize}
                \item Normalization or log-transformation techniques
            \end{itemize}
    \end{itemize}
\end{frame}

\begin{frame}[fragile]
    \frametitle{Key Points to Emphasize}
    
    \begin{itemize}
        \item EDA is both a diagnostic and investigative process.
        \item Effective EDA involves both quantitative and qualitative approaches.
        \item Visualizations are powerful tools for conveying findings and insights.
    \end{itemize}

    By mastering EDA techniques, students will be better equipped to extract meaningful insights from data.
\end{frame}

\begin{frame}[fragile]
    \frametitle{Objectives of Exploratory Data Analysis (EDA)}
    \begin{block}{Understanding EDA}
        Exploratory Data Analysis (EDA) is a critical phase in the data analysis process where the primary goal is to uncover meaningful insights and patterns within the data before carrying out formal modeling or hypothesis testing.
    \end{block}
\end{frame}

\begin{frame}[fragile]
    \frametitle{Key Objectives of EDA}
    \begin{enumerate}
        \item \textbf{Identifying Patterns}
            \begin{itemize}
                \item \textbf{Definition:} Recognizing regularities or trends within the dataset.
                \item \textbf{Example:} Analyzing sales data over time can reveal seasonal trends in product demand.
                \item \textbf{Illustration:} A line graph showing sales trends over several months can easily identify peak seasons.
            \end{itemize}

        \item \textbf{Spotting Anomalies}
            \begin{itemize}
                \item \textbf{Definition:} Detecting data points that deviate significantly from other observations, which may indicate errors or novel phenomena.
                \item \textbf{Example:} A sudden spike in a customer's transaction history could indicate fraudulent activity or a new promotional impact.
                \item \textbf{Illustration:} Use of a scatter plot to highlight outliers, where most points cluster closely around a trend but a few points lie far outside this cluster.
            \end{itemize}

        \item \textbf{Formulating Hypotheses}
            \begin{itemize}
                \item \textbf{Definition:} Generating questions or educated guesses based on observed patterns and anomalies that can be tested using further statistical analysis.
                \item \textbf{Example:} If a solar panel company observes that panels in a particular region consistently generate less power, they might hypothesize that local weather conditions affect efficiency.
                \item \textbf{Key Consideration:} It’s vital to base these hypotheses on evidence found during EDA to ensure they are grounded in reality.
            \end{itemize}
    \end{enumerate}
\end{frame}

\begin{frame}[fragile]
    \frametitle{Techniques and Key Points in EDA}
    \begin{block}{Techniques used in EDA}
        \begin{itemize}
            \item \textbf{Descriptive Statistics:} Utilizing measures such as mean, median, mode, and standard deviation to summarize data.
            \item \textbf{Visualizations:} Employing various plots (e.g., histograms, scatter plots) to visually interpret the data, aiding in the identification of trends, correlations, and outliers.
        \end{itemize}
    \end{block}
    
    \begin{block}{Key Points to Emphasize}
        \begin{itemize}
            \item EDA is iterative; insights from one phase can lead to deeper exploration.
            \item It is not just about confirming assumptions but also about discovering new insights.
            \item Effective EDA combines quantitative analysis with qualitative interpretation.
        \end{itemize}
    \end{block}
    
    By conducting a thorough EDA, analysts equip themselves with a well-rounded understanding of their data, paving the way for more robust modeling and decision-making processes.
\end{frame}

\begin{frame}[fragile]
    \frametitle{Types of Data Visualizations - Introduction}
    \begin{block}{Introduction}
        Data visualization is a cornerstone of Exploratory Data Analysis (EDA), providing a graphical representation of information that allows analysts to intuitively understand complex datasets. 
    \end{block}
    In this section, we will cover four fundamental types of visualizations:
    \begin{itemize}
        \item Histograms
        \item Scatter Plots
        \item Box Plots
        \item Heatmaps
    \end{itemize}
\end{frame}

\begin{frame}[fragile]
    \frametitle{Types of Data Visualizations - Histograms}
    \begin{block}{1. Histograms}
        \begin{itemize}
            \item \textbf{Definition:} A histogram is a graphical representation that organizes a group of data points into specified ranges (bins). 
            \item \textbf{Purpose:} Used to assess the distribution of a dataset and identify patterns such as normality, skewness, or the presence of outliers.
            \item \textbf{Example:} A histogram of students' test scores can reveal if most students scored in a particular range, indicating performance trends.
        \end{itemize}
    \end{block}
    \begin{lstlisting}[language=Python]
    import matplotlib.pyplot as plt
    import numpy as np

    data = np.random.normal(loc=70, scale=10, size=100)  # Example data
    plt.hist(data, bins=10, alpha=0.7, color='blue')
    plt.title('Histogram of Test Scores')
    plt.xlabel('Score Range')
    plt.ylabel('Frequency')
    plt.show()
    \end{lstlisting}
\end{frame}

\begin{frame}[fragile]
    \frametitle{Types of Data Visualizations - Scatter Plots}
    \begin{block}{2. Scatter Plots}
        \begin{itemize}
            \item \textbf{Definition:} A scatter plot displays values for typically two variables for a set of data. Points are plotted on a Cartesian plane based on their values.
            \item \textbf{Purpose:} Useful for identifying relationships, correlations, or patterns between two numerical variables.
            \item \textbf{Example:} In a scatter plot of height versus weight, each point represents an individual, helping visualize the correlation between height and weight.
        \end{itemize}
    \end{block}
    \begin{lstlisting}[language=Python]
    import matplotlib.pyplot as plt

    height = [150, 160, 165, 170, 180]
    weight = [50, 60, 65, 75, 90]
    plt.scatter(height, weight, color='red')
    plt.title('Height vs Weight Scatter Plot')
    plt.xlabel('Height (cm)')
    plt.ylabel('Weight (kg)')
    plt.show()
    \end{lstlisting}
\end{frame}

\begin{frame}[fragile]
    \frametitle{Types of Data Visualizations - Box Plots}
    \begin{block}{3. Box Plots}
        \begin{itemize}
            \item \textbf{Definition:} A box plot (or whisker plot) provides a summary of a set of data values, showing properties like median, quartiles, and potential outliers.
            \item \textbf{Purpose:} Highlights central tendency, dispersion, and asymmetry of data while readily identifying outliers.
            \item \textbf{Example:} Box plots are particularly effective in comparing test score distributions across different classes.
        \end{itemize}
    \end{block}
    \begin{lstlisting}[language=Python]
    import seaborn as sns
    import matplotlib.pyplot as plt

    data = sns.load_dataset("tips")  # Example dataset
    sns.boxplot(x="day", y="total_bill", data=data)
    plt.title('Box Plot of Total Bill Amount by Day')
    plt.show()
    \end{lstlisting}
\end{frame}

\begin{frame}[fragile]
    \frametitle{Types of Data Visualizations - Heatmaps}
    \begin{block}{4. Heatmaps}
        \begin{itemize}
            \item \textbf{Definition:} A heatmap shows the magnitude of a phenomenon as color in two dimensions. Values in a matrix are represented as colors.
            \item \textbf{Purpose:} Ideal for visualizing correlation matrices or frequency counts, making patterns or anomalies visually searchable.
            \item \textbf{Example:} A heatmap showing correlations between various factors in a dataset can help identify strong relationships at a glance.
        \end{itemize}
    \end{block}
    \begin{lstlisting}[language=Python]
    import seaborn as sns
    import matplotlib.pyplot as plt

    data = sns.load_dataset("flights").pivot("month", "year", "passengers")
    sns.heatmap(data, cmap="YlGnBu")
    plt.title('Heatmap of Flight Passengers')
    plt.show()
    \end{lstlisting}
\end{frame}

\begin{frame}[fragile]
    \frametitle{Types of Data Visualizations - Key Points}
    \begin{itemize}
        \item Each visualization type serves a unique purpose; select based on data type and analysis goals.
        \item Histograms help understand distributions, while scatter plots reveal relationships.
        \item Box plots summarize data and highlight outliers efficiently.
        \item Heatmaps excel at visualizing complex data relationships and variations.
    \end{itemize}
    By familiarizing yourself with these visualizations, you'll enhance your ability to explore and analyze data effectively, laying the groundwork for deeper statistical analysis and insights.
\end{frame}

\begin{frame}[fragile]
    \frametitle{Descriptive Statistics - Introduction}
    \begin{itemize}
        \item Descriptive statistics summarize key characteristics of a data set.
        \item Crucial for exploratory data analysis (EDA).
        \item Provide insights into the data and highlight main aspects effectively.
    \end{itemize}
\end{frame}

\begin{frame}[fragile]
    \frametitle{Descriptive Statistics - Measures of Central Tendency}
    \begin{block}{Central Tendency}
        Measures that identify the central position within a data set:
    \end{block}
    \begin{itemize}
        \item \textbf{Mean (Average)}:
            \begin{itemize}
                \item Formula: 
                \[
                \text{Mean} = \frac{\sum_{i=1}^n x_i}{n}
                \]
                \item Example: For \{2, 4, 6, 8\}, Mean = (2+4+6+8)/4 = 5.
            \end{itemize}
        \item \textbf{Median}:
            \begin{itemize}
                \item Middle value when data is ordered.
                \item Example: For \{3, 1, 4, 2\} (ordered \{1, 2, 3, 4\}), Median = (2+3)/2 = 2.5.
            \end{itemize}
        \item \textbf{Mode}:
            \begin{itemize}
                \item Most frequently occurring value.
                \item Example: For \{1, 2, 2, 3\}, Mode = 2.
            \end{itemize}
    \end{itemize}
\end{frame}

\begin{frame}[fragile]
    \frametitle{Descriptive Statistics - Measures of Dispersion}
    \begin{block}{Dispersion}
        Measures that reflect the variation or spread of the data:
    \end{block}
    \begin{itemize}
        \item \textbf{Range}:
            \begin{itemize}
                \item Formula: 
                \[
                \text{Range} = \text{Max} - \text{Min}
                \]
                \item Example: For \{1, 3, 5, 7\}, Range = 7 - 1 = 6.
            \end{itemize}
        \item \textbf{Variance}:
            \begin{itemize}
                \item Measures average of squared differences from the mean.
                \item Formula: 
                \[
                \text{Variance} (σ^2) = \frac{\sum_{i=1}^n (x_i - \text{Mean})^2}{n}
                \]
                \item Example: For \{2, 4, 4, 4, 5, 5, 7, 9\}, Variance = 4.
            \end{itemize}
        \item \textbf{Standard Deviation (SD)}:
            \begin{itemize}
                \item Square root of the variance.
                \item Formula: 
                \[
                \text{Standard Deviation} (σ) = \sqrt{\text{Variance}}
                \]
                \item Example: If Variance = 4, then SD = \(\sqrt{4}\) = 2.
            \end{itemize}
    \end{itemize}
\end{frame}

\begin{frame}[fragile]
    \frametitle{Descriptive Statistics - Key Points and Summary}
    \begin{itemize}
        \item Descriptive statistics are foundational for understanding data distributions.
        \item Measures of central tendency summarize location; measures of dispersion provide insights into variability.
        \item Essential for making informed decisions based on data.
    \end{itemize}
    \begin{block}{Summary}
        Descriptive statistics distill complex data into understandable summaries. Mean, median, and mode define the center, while range, variance, and standard deviation reveal spread. These concepts are pivotal in EDA for data-driven decisions.
    \end{block}
\end{frame}

\begin{frame}[fragile]
    \frametitle{Data Distribution and Normality}
    \begin{block}{Overview}
        Understanding data distributions, normality tests, and the significance of normal distribution in data analysis.
    \end{block}
\end{frame}

\begin{frame}[fragile]
    \frametitle{Understanding Data Distributions}
    \begin{itemize}
        \item \textbf{Data Distribution}: How values are spread across a range, providing insights into dataset structure.
        \item \textbf{Main Types of Distributions}:
            \begin{enumerate}
                \item \textbf{Normal Distribution}: Symmetric, bell-shaped curve, most observations cluster around the central peak.
                \item \textbf{Skewed Distribution}: Data points are uneven; examples include income distribution (right-skewed) and biological measures (left-skewed).
                \item \textbf{Uniform Distribution}: All outcomes are equally likely.
            \end{enumerate}
    \end{itemize}
\end{frame}

\begin{frame}[fragile]
    \frametitle{Importance of Normal Distribution}
    \begin{itemize}
        \item \textbf{Central Limit Theorem}: Mean of a sufficiently large number of samples will be approximately normally distributed, enabling population parameter inferences.
        \item \textbf{Easier Analysis}: Many statistical tests (e.g., t-tests, ANOVAs) assume normality to be valid.
    \end{itemize}
\end{frame}

\begin{frame}[fragile]
    \frametitle{Normality Tests}
    \begin{itemize}
        \item \textbf{Shapiro-Wilk Test}:
            \begin{itemize}
                \item \textbf{Hypotheses}:
                    \begin{enumerate}
                        \item Null (H0): The data is normally distributed.
                        \item Alternative (H1): The data is not normally distributed.
                    \end{enumerate}
                \item \textbf{Interpretation}: A p-value < 0.05 indicates data significantly deviates from normality.
            \end{itemize}
        \item \textbf{Kolmogorov-Smirnov Test}: Compares the sample distribution with a reference normal distribution.
        \item \textbf{Q-Q Plot}: Graphical method to visually assess normality; points on the 45-degree line indicate normal distribution.
    \end{itemize}
\end{frame}

\begin{frame}[fragile]
    \frametitle{Key Points and Visualizations}
    \begin{itemize}
        \item \textbf{Visualizations Matter}: Use histograms and boxplots to visually assess distribution shape.
        \item \textbf{Understanding Skewness and Kurtosis}:
            \begin{itemize}
                \item \textbf{Skewness}: Asymmetry measurement; values close to 0 indicate normal distribution.
                \item \textbf{Kurtosis}: Measures "tailedness"; normal distribution has a kurtosis of 3.
            \end{itemize}
    \end{itemize}
\end{frame}

\begin{frame}[fragile]
    \frametitle{Example Illustrations}
    \begin{itemize}
        \item \textbf{Visual Aids}:
            \begin{enumerate}
                \item Histogram displaying normal distribution, left-skewed, and right-skewed distributions.
                \item Q-Q plot illustrating linear versus non-linear fits (normality vs. non-normality).
            \end{enumerate}
    \end{itemize}
\end{frame}

\begin{frame}[fragile]
    \frametitle{Conclusion}
    By grasping data distributions and the concept of normality, you can better understand and analyze your data, making more informed and valid conclusions.
\end{frame}

\begin{frame}[fragile]
    \frametitle{Outlier Detection Techniques - Introduction}
    \begin{block}{Introduction to Outliers}
        Outliers are data points that differ significantly from other observations. They can skew and mislead the interpretation of statistical analyses. Understanding how to identify and handle outliers is crucial in making data-driven conclusions.
    \end{block}
\end{frame}

\begin{frame}[fragile]
    \frametitle{Outlier Detection Techniques - Importance}
    \begin{block}{Why Do Outliers Matter?}
        \begin{itemize}
            \item \textbf{Impact on Analysis:} Outliers can affect the mean and standard deviation, leading to misleading results.
            \item \textbf{Indicate Important Findings:} Sometimes, outliers reflect critical information, such as fraud in financial transactions or significant events in time series data.
        \end{itemize}
    \end{block}
\end{frame}

\begin{frame}[fragile]
    \frametitle{Outlier Detection Techniques - Methods}
    \begin{block}{Methods for Outlier Detection}
        \begin{enumerate}
            \item \textbf{Visual Techniques}
            \begin{itemize}
                \item \textbf{Box Plots:} A graphical representation that displays the data's minimum, first quartile (Q1), median, third quartile (Q3), and maximum. 
                \item Outliers are typically defined as points that fall below $Q1 - 1.5 \times \text{IQR}$ or above $Q3 + 1.5 \times \text{IQR}$, where IQR (Interquartile Range) = $Q3 - Q1$.
                
                \begin{center}
                \textbf{Example:} \\
                Box Plot: \\
                |----------------------|  \\         
                |         |    |      |   \\         
                |---------|----|------|--------- \\
                |         Q1   Median   Q3     |
                \end{center}
                
                \item \textbf{Scatter Plots:} Useful for visualizing the relationship between two variables. Outliers appear as points distant from the majority of data points.
            \end{itemize}
            
            \item \textbf{Statistical Techniques}
            \begin{itemize}
                \item \textbf{Z-score:} 
                \begin{equation}
                    Z = \frac{(X - \mu)}{\sigma}
                \end{equation}
                Where: 
                \begin{itemize}
                    \item $X$ = value being examined
                    \item $\mu$ = mean of the data
                    \item $\sigma$ = standard deviation
                \end{itemize}
                A Z-score greater than 3 (or less than -3) often indicates an outlier.
                
                \item \textbf{Modified Z-score:} An alternative robust against outliers using median and median absolute deviation (MAD).
                \begin{equation}
                    M = 0.6745 \cdot \frac{(X_i - \text{Median})}{\text{MAD}}
                \end{equation}

                \item \textbf{Isolation Forest:} A machine learning approach that isolates observations by randomly selecting a feature and splitting the data. Anomalies are easy to isolate due to their unique attributes.
            \end{itemize}
        \end{enumerate}
    \end{block}
\end{frame}

\begin{frame}[fragile]
    \frametitle{Outlier Detection Techniques - Key Points}
    \begin{block}{Key Points to Emphasize}
        \begin{itemize}
            \item \textbf{Importance of Context:} Not all outliers are errors. Investigate reasons behind outliers.
            \item \textbf{Multiple Methods:} Use a combination of visual and statistical methods for robust outlier detection.
            \item \textbf{Prevention of Misleading Results:} Identifying outliers helps maintain the integrity of analysis, improving decision-making based on data.
        \end{itemize}
    \end{block}
    
    \begin{block}{Summary}
        Outlier detection is an essential part of exploratory data analysis. By effectively using visualization and robust statistical methods, you can identify and appropriately handle outliers, thus enhancing the quality of your data analysis.
    \end{block}
    
    \begin{block}{Next Steps}
        These techniques will prepare you for the next slide on \textbf{Correlation Analysis}, where we will explore relationships between your cleaned and validated data.
    \end{block}
\end{frame}

\begin{frame}[fragile]
    \frametitle{Correlation Analysis - Understanding Correlation}
    \begin{block}{Definition}
        Correlation measures the strength and direction of a linear relationship between two quantitative variables. It provides insight into how the change in one variable may be associated with the change in another.
    \end{block}
\end{frame}

\begin{frame}[fragile]
    \frametitle{Correlation Analysis - Correlation Coefficients}
    \begin{itemize}
        \item \textbf{1. Pearson Correlation Coefficient (r):}
        \begin{itemize}
            \item \textbf{Range:} -1 to 1 
                \begin{itemize}
                    \item \textbf{1:} Perfect positive correlation
                    \item \textbf{-1:} Perfect negative correlation
                    \item \textbf{0:} No correlation
                \end{itemize}
            \item \textbf{Formula:}
            \begin{equation}
                r = \frac{n(\sum xy) - (\sum x)(\sum y)}{\sqrt{[n\sum x^2 - (\sum x)^2][n\sum y^2 - (\sum y)^2]}}
            \end{equation}
        \end{itemize}
        
        \item \textbf{2. Spearman's Rank Correlation Coefficient:} Non-parametric measure of the strength and direction of association between two ranked variables.
        
        \item \textbf{3. Kendall’s Tau:} Another non-parametric measure that assesses the strength of association based on the ranks.
    \end{itemize}
\end{frame}

\begin{frame}[fragile]
    \frametitle{Correlation Analysis - Interpreting Correlation Matrices}
    \begin{block}{Definition}
        A correlation matrix displays the correlation coefficients between multiple variables.
    \end{block}
    
    \begin{exampleblock}{Example Data}
        Consider a dataset with variables: Height, Weight, and Age.
        \begin{center}
            \begin{tabular}{|c|c|c|c|}
                \hline
                & Height & Weight & Age \\
                \hline
                Height & 1 & 0.85 & 0.10 \\
                Weight & 0.85 & 1 & 0.20 \\
                Age & 0.10 & 0.20 & 1 \\
                \hline
            \end{tabular}
        \end{center}
    \end{exampleblock}
    
    \begin{itemize}
        \item \textbf{Interpretation:}
        \begin{itemize}
            \item Height \& Weight (0.85): Strong positive correlation.
            \item Height \& Age (0.10): Very weak positive correlation.
            \item Weight \& Age (0.20): Weak positive correlation.
        \end{itemize}
    \end{itemize}
\end{frame}

\begin{frame}
    \frametitle{Using Software Tools for EDA}
    \begin{block}{Introduction to Exploratory Data Analysis (EDA)}
        \begin{itemize}
            \item EDA is a critical process in data science.
            \item It summarizes main characteristics of a dataset.
            \item Involves visualizations and descriptive statistics.
            \item Helps find patterns or anomalies before inferential statistics.
        \end{itemize}
    \end{block}
\end{frame}

\begin{frame}
    \frametitle{Popular Software Tools for EDA}
    \begin{enumerate}
        \item \textbf{Python's Pandas}
        \item \textbf{R's ggplot2}
        \item \textbf{Tableau}
        \item \textbf{Microsoft Excel}
    \end{enumerate}
\end{frame}

\begin{frame}[fragile]
    \frametitle{Python's Pandas}
    \begin{block}{Overview}
        Pandas is a powerful library for data manipulation and analysis.
        It provides DataFrames and Series for structured data.
    \end{block}
    \begin{block}{Key Functions}
        \begin{itemize}
            \item \texttt{read\_csv()}: Import data from CSV files.
            \item \texttt{describe()}: Generate descriptive statistics.
            \item \texttt{groupby()}: Group data for aggregation.
            \item \texttt{plot()}: Basic plotting capabilities.
        \end{itemize}
    \end{block}
    \begin{block}{Example}
        \begin{lstlisting}[language=Python]
        import pandas as pd
        df = pd.read_csv('data.csv')
        print(df.describe())
        df['column_name'].hist()
        \end{lstlisting}
    \end{block}
\end{frame}

\begin{frame}[fragile]
    \frametitle{R's ggplot2}
    \begin{block}{Overview}
        ggplot2 is a visualization package in R based on the grammar of graphics.
        It allows easy creation of complex and customized visualizations.
    \end{block}
    \begin{block}{Key Functions}
        \begin{itemize}
            \item \texttt{ggplot()}: Initialize the plotting system.
            \item \texttt{aes()}: Define aesthetic mappings.
            \item \texttt{geom\_*()}: Specify types of plots (e.g., \texttt{geom\_point()}, \texttt{geom\_histogram()}).
        \end{itemize}
    \end{block}
    \begin{block}{Example}
        \begin{lstlisting}[language=R]
        library(ggplot2)
        ggplot(data = df, aes(x = column_name)) +
            geom_histogram(binwidth = 0.5)
        \end{lstlisting}
    \end{block}
\end{frame}

\begin{frame}
    \frametitle{Other Tools for EDA}
    \begin{itemize}
        \item \textbf{Tableau}
        \begin{itemize}
            \item Visual analytics platform for interactive dashboards.
            \item Connects to various data sources with an intuitive interface.
        \end{itemize}
        
        \item \textbf{Microsoft Excel}
        \begin{itemize}
            \item Versatile spreadsheet application for data analysis.
            \item Features include PivotTables and various chart types.
        \end{itemize}
    \end{itemize}
\end{frame}

\begin{frame}
    \frametitle{Key Points}
    \begin{itemize}
        \item Choice of tool depends on analysis complexity and data volume.
        \item Combining visualizations with statistics is crucial for understanding.
        \item Open-source tools like Pandas and ggplot2 are widely accessible.
    \end{itemize}
\end{frame}

\begin{frame}
    \frametitle{Conclusion}
    \begin{itemize}
        \item EDA is essential for guiding further analysis or model building.
        \item Familiarity with software tools enhances the exploration process.
        \item Effective EDA leads to better decision-making based on data insights.
    \end{itemize}
\end{frame}

\begin{frame}
    \frametitle{Case Study: Applying EDA}
    \begin{block}{Introduction to the Case Study}
        This slide explores a case study where Exploratory Data Analysis (EDA) techniques have been effectively applied to a real-world dataset.
        The purpose of this case study is to illustrate how EDA can uncover valuable insights, reveal patterns, and assist in informed decision-making.
    \end{block}
\end{frame}

\begin{frame}
    \frametitle{Dataset Overview}
    \begin{block}{Context}
        We will analyze a dataset from the UCI Machine Learning Repository focusing on the Iris Species dataset.
        This dataset contains 150 samples of iris flowers with features: 
        \begin{itemize}
            \item Sepal Length
            \item Sepal Width
            \item Petal Length
            \item Petal Width
        \end{itemize}
        and labels indicating the species (setosa, versicolor, virginica).
    \end{block}

    \begin{block}{Feature Summary}
        \begin{itemize}
            \item \textbf{Sepal Length}: Continuous variable (cm)
            \item \textbf{Sepal Width}: Continuous variable (cm)
            \item \textbf{Petal Length}: Continuous variable (cm)
            \item \textbf{Petal Width}: Continuous variable (cm)
            \item \textbf{Species}: Categorical variable (setosa, versicolor, virginica)
        \end{itemize}
    \end{block}
\end{frame}

\begin{frame}[fragile]
    \frametitle{EDA Techniques Applied}
    \begin{enumerate}
        \item \textbf{Descriptive Statistics}
        \begin{itemize}
            \item Basic statistics (mean, median, mode, std. deviation) for each feature.
        \end{itemize}
        \begin{lstlisting}
import pandas as pd
iris_data = pd.read_csv('iris.csv')
descriptive_stats = iris_data.describe()
        \end{lstlisting}
        
        \item \textbf{Visualizations}
        \begin{itemize}
            \item Pairplot: Matrix of scatter plots, color-coded by species.
            \item Boxplots: Distribution of petal lengths and widths, revealing outliers.
        \end{itemize}
        \begin{lstlisting}
import seaborn as sns
sns.pairplot(iris_data, hue='Species')
        \end{lstlisting}

        \item \textbf{Correlation Matrix}
        \begin{itemize}
            \item Shows how features correlate, identifying potential multicollinearity.
        \end{itemize}
        \begin{lstlisting}
correlation_matrix = iris_data.corr()
sns.heatmap(correlation_matrix, annot=True)
        \end{lstlisting}
    \end{enumerate}
\end{frame}

\begin{frame}
    \frametitle{Key Findings}
    \begin{itemize}
        \item \textbf{Species Distribution}: Iris Setosa is distinguishable due to smaller petal sizes.
        \item \textbf{Feature Relationships}: Strong positive correlation between petal length and petal width.
        \item \textbf{Outliers}: Outliers found in petal length of versicolor species warrant further investigation.
    \end{itemize}
\end{frame}

\begin{frame}
    \frametitle{Conclusion and Key Points}
    \begin{block}{Conclusion}
        The application of EDA techniques on the Iris dataset revealed insights into the characteristics of the iris species, emphasizing their distinct features.
    \end{block}

    \begin{block}{Key Points to Remember}
        \begin{itemize}
            \item EDA is essential for understanding data characteristics before formal modeling.
            \item Key visualizations clarify complex relationships.
            \item Document insights gained during EDA for future reference.
        \end{itemize}
    \end{block}
\end{frame}

\begin{frame}[fragile]
    \frametitle{Best Practices in Exploratory Data Analysis (EDA)}
    \begin{block}{Overview}
        Exploratory Data Analysis (EDA) is critical for understanding data distributions, relationships, and anomalies. Following best practices enhances EDA's effectiveness.
    \end{block}
\end{frame}

\begin{frame}[fragile]
    \frametitle{Best Practices in EDA - Part 1}
    \begin{enumerate}
        \item \textbf{Understand Your Data:}
        \begin{itemize}
            \item Gain familiarity with the dataset.
            \item Identify variable types: categorical, numerical, etc.
            \item \textit{Example:} In customer datasets, discern demographic from purchase history data.
        \end{itemize}

        \item \textbf{Data Cleaning:}
        \begin{itemize}
            \item Handle missing values, duplicates, and inconsistencies.
            \item \textit{Example:} Use Pandas to manage missing values:
            \begin{lstlisting}[language=Python]
import pandas as pd
df = pd.read_csv('data.csv')
df.dropna(inplace=True)  # Remove rows with missing values
            \end{lstlisting}
        \end{itemize}

        \item \textbf{Visualize Data:}
        \begin{itemize}
            \item Create visualizations like histograms, boxplots, and scatterplots to identify patterns.
            \item \textit{Example:} Plotting distributions helps in assessing outliers:
            \begin{lstlisting}[language=Python]
import matplotlib.pyplot as plt
plt.hist(df['age'], bins=10)  # Histogram of 'age' variable
plt.show()
            \end{lstlisting}
        \end{itemize}
    \end{enumerate}
\end{frame}

\begin{frame}[fragile]
    \frametitle{Best Practices in EDA - Part 2}
    \begin{enumerate}
        \setcounter{enumi}{3} % Continue enumeration
        \item \textbf{Descriptive Statistics:}
        \begin{itemize}
            \item Use measures such as mean, median, and standard deviation to summarize characteristics.
            \item \textit{Example:} Calculate average income to understand central tendencies.
        \end{itemize}

        \begin{block}{Key Formula}
            \begin{equation}
            \text{Mean} (\mu) = \frac{\Sigma x_i}{n}
            \end{equation}
            Where $\Sigma x_i$ is the sum of all observations and $n$ is the total number of observations.
        \end{block}

        \item \textbf{Document Findings:}
        \begin{itemize}
            \item Record insights and analysis notes for repeatability.
        \end{itemize}

        \item \textbf{Iterate and Refine:}
        \begin{itemize}
            \item EDA is iterative; revisit analysis based on initial findings.
            \item \textit{Example:} Exploring relationships observed via scatterplots.
        \end{itemize}
    \end{enumerate}
\end{frame}

\begin{frame}[fragile]
    \frametitle{Best Practices in EDA - Part 3}
    \begin{enumerate}
        \setcounter{enumi}{6} % Continue enumeration
        \item \textbf{Engage with Domain Knowledge:}
        \begin{itemize}
            \item Collaborate with domain experts to validate findings.
            \item Domain knowledge aids in interpreting relationships and anomalies.
        \end{itemize}
    \end{enumerate}

    \begin{block}{Summary}
        Implementing best practices in EDA enhances efficiency and effectiveness. Key practices include:
        \begin{itemize}
            \item Understanding data
            \item Thorough cleaning
            \item Effective visualization
            \item Descriptive statistics
            \item Comprehensive documentation
            \item Continuous iteration
            \item Leveraging domain expertise
        \end{itemize}
        Stay proactive and adaptable in your EDA journey!
    \end{block}
\end{frame}


\end{document}