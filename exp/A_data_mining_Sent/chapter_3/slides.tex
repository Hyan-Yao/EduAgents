\documentclass{beamer}

% Theme choice
\usetheme{Madrid} % You can change to e.g., Warsaw, Berlin, CambridgeUS, etc.

% Encoding and font
\usepackage[utf8]{inputenc}
\usepackage[T1]{fontenc}

% Graphics and tables
\usepackage{graphicx}
\usepackage{booktabs}

% Code listings
\usepackage{listings}
\lstset{
    basicstyle=\ttfamily\small,
    keywordstyle=\color{blue},
    commentstyle=\color{gray},
    stringstyle=\color{red},
    breaklines=true,
    frame=single
}

% Math packages
\usepackage{amsmath}
\usepackage{amssymb}

% Colors
\usepackage{xcolor}

% TikZ and PGFPlots
\usepackage{tikz}
\usepackage{pgfplots}
\pgfplotsset{compat=1.18}
\usetikzlibrary{positioning}

% Hyperlinks
\usepackage{hyperref}

% Title information
\title{Week 3: Exploratory Data Analysis (EDA)}
\author{Your Name}
\institute{Your Institution}
\date{\today}

\begin{document}

\frame{\titlepage}

\begin{frame}[fragile]
    \frametitle{Introduction to Exploratory Data Analysis (EDA)}
    \begin{block}{What is Exploratory Data Analysis (EDA)?}
        Exploratory Data Analysis (EDA) is a fundamental step in the data analysis process, focusing on examining data sets to summarize their main characteristics often using visual methods.
    \end{block}
    EDA involves breaking down complex data to reveal patterns, trends, and relationships that can inform subsequent analysis.
\end{frame}

\begin{frame}[fragile]
    \frametitle{Purpose of EDA}
    \begin{enumerate}
        \item \textbf{Understanding Data:} Comprehend data sources and structures for better quality assessments.
        \item \textbf{Identifying Patterns:} Visualize data to detect underlying trends or patterns.
        \item \textbf{Highlighting Anomalies:} Identify outliers that can indicate errors or unique insights.
        \item \textbf{Informing Future Analysis:} Insights from EDA shape subsequent data analysis processes and guide feature selection.
    \end{enumerate}
\end{frame}

\begin{frame}[fragile]
    \frametitle{Significance in Data Mining}
    \begin{itemize}
        \item \textbf{Data Quality Assessment:} Identifies missing values and anomalies impacting analysis quality.
        \item \textbf{Enhanced Model Building:} Understanding relationships improves predictive model effectiveness.
        \item \textbf{Decision-Making Framework:} Provides actionable insights for business strategies.
    \end{itemize}
\end{frame}

\begin{frame}[fragile]
    \frametitle{Key Points to Emphasize}
    \begin{itemize}
        \item \textbf{Visualization Tools:} Common tools include histograms, scatter plots, box plots, and correlation matrices.
        \item \textbf{Statistical Techniques:} Descriptive statistics summarize data distributions effectively (mean, median, mode).
        \item \textbf{Iterative Process:} EDA is not linear; it may lead to additional cleansing, transformation, or data collection.
    \end{itemize}
\end{frame}

\begin{frame}[fragile]
    \frametitle{Example of EDA}
    Consider a dataset containing employee information, such as age, salary, department, and job satisfaction.
    \begin{itemize}
        \item \textbf{Visualization:} A box plot can reveal salary distribution disparities across departments.
        \item \textbf{Statistical Insight:} Calculating mean job satisfaction scores identifies areas needing improvement.
    \end{itemize}
\end{frame}

\begin{frame}[fragile]
    \frametitle{Conclusion}
    Exploratory Data Analysis (EDA) is a critical step in data mining, bridging the gap between raw data and informative insights.
    \begin{itemize}
        \item It equips analysts to make data-driven decisions.
        \item Through visualization and statistical techniques, EDA paves the way for successful data analysis and business outcomes.
    \end{itemize}
\end{frame}

\begin{frame}[fragile]
    \frametitle{Importance of EDA - Understanding the Role of EDA}
    Exploratory Data Analysis (EDA) is a critical first step in the data analysis process, which involves examining datasets to summarize their main characteristics using visual methods.
    
    EDA helps analysts understand the data before applying formal statistical modeling or machine learning techniques.

    \begin{block}{Key Aspects of EDA}
        \begin{enumerate}
            \item Uncovering Patterns
            \item Detecting Anomalies
            \item Informed Business Decisions
        \end{enumerate}
    \end{block}
\end{frame}

\begin{frame}[fragile]
    \frametitle{Importance of EDA - Uncovering Patterns and Detecting Anomalies}
    \begin{itemize}
        \item \textbf{Uncovering Patterns:}
        \begin{itemize}
            \item \textbf{Definition:} Recurring trends or relationships within the dataset.
            \item \textbf{Example:} Seasonal sales peaks in retail datasets.
            \item \textbf{Importance:} Enables data-driven decisions for inventory optimization and targeted marketing.
        \end{itemize}

        \item \textbf{Detecting Anomalies:}
        \begin{itemize}
            \item \textbf{Definition:} Outliers that diverge significantly from the overall data distribution.
            \item \textbf{Example:} Sudden spikes in customer purchases indicating potential fraud.
            \item \textbf{Importance:} Crucial for maintaining data quality and preventing skewed analysis results.
        \end{itemize}
    \end{itemize}
\end{frame}

\begin{frame}[fragile]
    \frametitle{Importance of EDA - Informed Business Decisions and Key Points}
    \begin{itemize}
        \item \textbf{Informed Business Decisions:}
        \begin{itemize}
            \item \textbf{Definition:} Decisions made based on data insights rather than intuition.
            \item \textbf{Example:} Adjusting marketing strategies based on underperforming products.
            \item \textbf{Importance:} Leads to improved efficiency, customer satisfaction, and better business outcomes.
        \end{itemize}
        
        \item \textbf{Key Points to Emphasize:}
        \begin{itemize}
            \item EDA enhances understanding and prepares for deeper inquiry.
            \item Visualization tools are critical for effective EDA.
            \item EDA is an iterative process that refines exploration strategies.
        \end{itemize}
        
        \begin{block}{Example Code Snippet (Python)}
            \begin{lstlisting}
import pandas as pd
import matplotlib.pyplot as plt

# Load dataset
data = pd.read_csv('sales_data.csv')

# Plotting a histogram for sales
plt.figure(figsize=(10, 6))
plt.hist(data['Sales'], bins=20, color='blue', alpha=0.7)
plt.title('Sales Distribution')
plt.xlabel('Sales Amount')
plt.ylabel('Frequency')
plt.show()
            \end{lstlisting}
        \end{block}
    \end{itemize}
\end{frame}

\begin{frame}
    \frametitle{Key Visualization Tools: Matplotlib}
    \begin{block}{Introduction to Matplotlib}
        Matplotlib is a powerful and widely-used Python library for data visualization. It provides a flexible platform to create a variety of static, animated, and interactive graphs and plots.
    \end{block}
    \begin{itemize}
        \item Ideal for Exploratory Data Analysis (EDA)
        \item Helps in presenting complex data insights visually
        \item Aids in identifying patterns and trends effectively
    \end{itemize}
\end{frame}

\begin{frame}
    \frametitle{Key Features of Matplotlib}
    \begin{enumerate}
        \item \textbf{Versatility:} Create line plots, bar charts, scatter plots, histograms, pie charts, and more.
        \item \textbf{Customizable:} Customize colors, markers, line styles, legends, and annotations.
        \item \textbf{Integration:} Works seamlessly with NumPy and Pandas for data manipulation.
        \item \textbf{Animation Support:} Create animated visualizations to show changes over time.
        \item \textbf{Interactive Plots:} Allows interaction with plots in Jupyter Notebooks.
    \end{enumerate}
\end{frame}

\begin{frame}[fragile]
    \frametitle{Basic Usage Example}
    \begin{block}{Python Code for a Simple Plot}
        \begin{lstlisting}[language=Python]
import matplotlib.pyplot as plt

# Sample data
x = [1, 2, 3, 4, 5]
y = [2, 3, 5, 7, 11]

# Creating a plot
plt.plot(x, y, marker='o', linestyle='-', color='b', label='Prime Numbers')

# Adding title and labels
plt.title('Simple Line Plot')
plt.xlabel('X-axis: Numbers')
plt.ylabel('Y-axis: Corresponding Prime')

# Display the legend
plt.legend()

# Show the plot
plt.show()
        \end{lstlisting}
    \end{block}
    \begin{itemize}
        \item Illustrates key functions used in plotting
        \item Highlights importance of titles and labels for context
    \end{itemize}
\end{frame}

\begin{frame}
    \frametitle{Key Points to Emphasize}
    \begin{itemize}
        \item \textbf{Understanding Data through Visualization:} Plots reveal insights that raw data cannot easily communicate.
        \item \textbf{Adaptability:} Matplotlib's flexibility accommodates all types of data.
        \item \textbf{Foundation for Other Libraries:} It's foundational for other visualization libraries, such as Seaborn.
    \end{itemize}
    \begin{block}{Conclusion}
        Mastering Matplotlib equips you with a key tool for your EDA toolkit, allowing effective communication of data-driven stories.
    \end{block}
\end{frame}

\begin{frame}
    \frametitle{Next Steps}
    \begin{block}{Exploring Seaborn}
        In the following slide, we will explore \textbf{Seaborn}, which enhances Matplotlib’s functionality with sophisticated statistical plots and improved aesthetics.
    \end{block}
\end{frame}

\begin{frame}[fragile]
    \frametitle{Key Visualization Tools: Seaborn}
    
    \begin{block}{Overview of Seaborn}
        Seaborn is a powerful visualization library in Python, built on top of Matplotlib. It provides a high-level interface for creating attractive statistical graphics. While Matplotlib offers versatility, Seaborn makes creating visually appealing representations of data easier with simplified syntax.
    \end{block}
\end{frame}

\begin{frame}[fragile]
    \frametitle{Seaborn Enhancements Over Matplotlib}
    
    \begin{itemize}
        \item \textbf{Simplified Syntax:} 
        \begin{itemize}
            \item Seaborn simplifies many plotting tasks, allowing complex visualizations with fewer lines of code.
            
            \item \textit{Example: Scatter Plot with Regression Line}
            \begin{lstlisting}[language=Python]
import seaborn as sns
import matplotlib.pyplot as plt

# Load example dataset
tips = sns.load_dataset('tips')

# Create scatter plot with regression line
sns.regplot(x='total_bill', y='tip', data=tips)
plt.show()
            \end{lstlisting}
        \end{itemize}
        
        \item \textbf{Built-in Themes:}
        \begin{itemize}
            \item Several aesthetic themes can be applied globally for visual enhancements:
            \begin{itemize}
                \item \texttt{darkgrid}, \texttt{whitegrid}, \texttt{dark}, \texttt{white}, \texttt{ticks}
            \end{itemize}
            
            \item \textit{Setting a Theme:}
            \begin{lstlisting}[language=Python]
sns.set_theme(style="whitegrid")  # Apply the whitegrid theme
            \end{lstlisting}
        \end{itemize}
    \end{itemize}
\end{frame}

\begin{frame}[fragile]
    \frametitle{Built-in Themes for Visual Aesthetics}
    
    \begin{enumerate}
        \item \textbf{Color Palettes:} 
        \begin{itemize}
            \item Seaborn allows easy usage of color palettes.
            \item You can switch between different palettes (e.g., "deep", "muted", "pastel", "dark").
            \item \textit{Example:}
            \begin{lstlisting}[language=Python]
sns.set_palette('pastel')
sns.barplot(x='day', y='total_bill', data=tips)
plt.show()
            \end{lstlisting}
        \end{itemize}

        \item \textbf{Faceting:}
        \begin{itemize}
            \item Use \texttt{facetgrid} to create grids of plots for visualizing relationships across categorical variables.
            \item \textit{Example:}
            \begin{lstlisting}[language=Python]
g = sns.FacetGrid(tips, col="time")
g.map(sns.scatterplot, "total_bill", "tip")
plt.show()
            \end{lstlisting}
        \end{itemize}
    \end{enumerate}
\end{frame}

\begin{frame}
    \frametitle{Basic Plotting with Matplotlib}
    \begin{block}{Overview}
        Matplotlib is a powerful library in Python for static, animated, and interactive visualizations. In this presentation, we'll cover:
        \begin{itemize}
            \item Line Plots
            \item Scatter Plots
            \item Bar Charts
        \end{itemize}
    \end{block}
\end{frame}

\begin{frame}[fragile]
    \frametitle{1. Line Plots}
    \begin{block}{Description}
        Line plots display data points sequentially, typically over time.
    \end{block}
    
    \begin{block}{Code Example}
    \begin{lstlisting}[language=Python]
import matplotlib.pyplot as plt

# Sample Data
x = [1, 2, 3, 4, 5]
y = [2, 3, 5, 7, 11]

# Creating a Line Plot
plt.plot(x, y)
plt.title("Line Plot Example")
plt.xlabel("X-axis")
plt.ylabel("Y-axis")
plt.grid()
plt.show()
    \end{lstlisting}
    \end{block}
    
    \begin{block}{Key Points}
        \begin{itemize}
            \item Use \texttt{plt.plot()} to create line plots.
            \item Labels, titles, and grids enhance readability.
        \end{itemize}
    \end{block}
\end{frame}

\begin{frame}[fragile]
    \frametitle{2. Scatter Plots}
    \begin{block}{Description}
        Scatter plots show the relationship between two numerical variables, which may reveal patterns or correlations.
    \end{block}
    
    \begin{block}{Code Example}
    \begin{lstlisting}[language=Python]
import matplotlib.pyplot as plt

# Sample Data
x = [5, 7, 8, 5, 6, 6, 2, 1]
y = [7, 4, 3, 7, 8, 9, 10, 12]

# Creating a Scatter Plot
plt.scatter(x, y, color='blue', marker='o')
plt.title("Scatter Plot Example")
plt.xlabel("X-axis")
plt.ylabel("Y-axis")
plt.grid()
plt.show()
    \end{lstlisting}
    \end{block}
    
    \begin{block}{Key Points}
        \begin{itemize}
            \item Use \texttt{plt.scatter()} to create scatter plots.
            \item Markers and colors can represent different categories.
        \end{itemize}
    \end{block}
\end{frame}

\begin{frame}[fragile]
    \frametitle{3. Bar Charts}
    \begin{block}{Description}
        Bar charts are useful for comparing quantities associated with different categories.
    \end{block}
    
    \begin{block}{Code Example}
    \begin{lstlisting}[language=Python]
import matplotlib.pyplot as plt

# Sample Data
categories = ['A', 'B', 'C', 'D']
values = [10, 15, 7, 12]

# Creating a Bar Chart
plt.bar(categories, values, color='orange')
plt.title("Bar Chart Example")
plt.xlabel("Categories")
plt.ylabel("Values")
plt.show()
    \end{lstlisting}
    \end{block}
    
    \begin{block}{Key Points}
        \begin{itemize}
            \item Use \texttt{plt.bar()} to create bar charts.
            \item Effective for displaying categorical data.
        \end{itemize}
    \end{block}
\end{frame}

\begin{frame}
    \frametitle{Conclusion and Next Steps}
    \begin{block}{Conclusion}
        Matplotlib is essential for visual data exploration. Mastering basic plots enhances analytical capability, enabling clearer insights.
    \end{block}
    
    \begin{block}{Next Steps}
        Explore advanced plotting techniques in Seaborn for enhanced aesthetics and intuitiveness in your visualizations!
    \end{block}
\end{frame}

\begin{frame}[fragile]
    \frametitle{Advanced Plotting with Seaborn}
    \begin{block}{Introduction to Seaborn}
        Seaborn is a powerful Python data visualization library based on Matplotlib that simplifies the creation of complex visualizations. It is particularly effective for visualizing statistical relationships, making it a significant tool for Exploratory Data Analysis (EDA).
    \end{block}
\end{frame}

\begin{frame}[fragile]
    \frametitle{Key Techniques Covered}
    \begin{enumerate}
        \item \textbf{Heatmaps}
        \item \textbf{Violin Plots}
        \item \textbf{Pair Plots}
    \end{enumerate}
\end{frame}

\begin{frame}[fragile]
    \frametitle{1. Heatmaps}
    \begin{block}{Explanation}
        A heatmap is a visualization technique that uses color to represent values in a matrix. It is particularly effective for displaying correlation matrices or visualizing data density.
    \end{block}
    \begin{block}{Example}
        \begin{lstlisting}[language=Python]
import seaborn as sns
import matplotlib.pyplot as plt

# Load sample dataset
data = sns.load_dataset('flights')

# Create a pivot table
pivot_table = data.pivot_table(index='month', columns='year', values='passengers', aggfunc='sum')

# Generate a heatmap
sns.heatmap(pivot_table, annot=True, fmt='d', cmap='YlGnBu')
plt.title("Heatmap of Flight Passengers Over Years")
plt.show()
        \end{lstlisting}
    \end{block}
\end{frame}

\begin{frame}[fragile]
    \frametitle{2. Violin Plots}
    \begin{block}{Explanation}
        A violin plot combines elements of a box plot and a density plot, representing data distributions across categories and highlighting the probability density at various values.
    \end{block}
    \begin{block}{Example}
        \begin{lstlisting}[language=Python]
# Load sample dataset
tips = sns.load_dataset('tips')

# Create a violin plot
sns.violinplot(x='day', y='total_bill', data=tips, inner='quartile', palette='muted')
plt.title("Violin Plot of Total Bill Amount by Day")
plt.show()
        \end{lstlisting}
    \end{block}
\end{frame}

\begin{frame}[fragile]
    \frametitle{3. Pair Plots}
    \begin{block}{Explanation}
        A pair plot is a grid of scatter plots that visualizes relationships among all pairs of numeric variables in a dataset, allowing for quick examination of their relationships and distributions.
    \end{block}
    \begin{block}{Example}
        \begin{lstlisting}[language=Python]
# Load sample dataset
iris = sns.load_dataset('iris')

# Create a pair plot
sns.pairplot(iris, hue='species', markers=["o", "s", "D"], palette="bright")
plt.title("Pair Plot of Iris Species")
plt.show()
        \end{lstlisting}
    \end{block}
\end{frame}

\begin{frame}[fragile]
    \frametitle{Key Points to Emphasize}
    \begin{itemize}
        \item Seaborn provides easy-to-use interfaces for complex visualizations.
        \item Each plotting technique serves a unique purpose in EDA, enhancing dataset understanding.
        \item Customize plots (titles, color palettes) for clarity and presentation enhancement.
    \end{itemize}
\end{frame}

\begin{frame}[fragile]
    \frametitle{Conclusion}
    Utilizing advanced plotting techniques like heatmaps, violin plots, and pair plots in Seaborn significantly enhances exploratory data analysis. Effective visualizations can highlight areas of concern in datasets as we progress into data cleaning techniques.
    
    \textbf{Next Steps:} Explore essential data cleaning techniques that set the stage for efficient EDA!
\end{frame}

\begin{frame}
    \frametitle{Data Cleaning Techniques}
    \begin{itemize}
        \item Data cleaning is essential before Exploratory Data Analysis (EDA).
        \item Ensures accuracy, quality, and efficiency in data analysis.
        \item Focus on handling missing values and identifying outliers.
    \end{itemize}
\end{frame}

\begin{frame}
    \frametitle{Importance of Data Cleaning}
    \begin{itemize}
        \item \textbf{Accuracy}: Dirty data can lead to misleading results. Cleaning ensures data reflects reality.
        \item \textbf{Quality}: High-quality data enhances the credibility of findings for decision-making.
        \item \textbf{Efficiency}: Reduces noise in data, streamlining further analysis.
    \end{itemize}
\end{frame}

\begin{frame}
    \frametitle{Handling Missing Values}
    \begin{itemize}
        \item \textbf{Remove Rows or Columns}: 
            \begin{itemize}
                \item If missing values are minimal (e.g., less than 5\%), consider dropping affected rows/columns.
            \end{itemize}
        \item \textbf{Imputation}: Replace missing values using statistical methods:
            \begin{itemize}
                \item Mean/Median/Mode Imputation.
                \begin{lstlisting}[language=Python]
import pandas as pd

# Assume df is a DataFrame with a missing value in 'Age'
df['Age'].fillna(df['Age'].mean(), inplace=True)  # Mean Imputation
                \end{lstlisting}
            \end{itemize}
        \item \textbf{Predictive Models}: Use machine learning to predict and fill missing values based on other features.
    \end{itemize}
\end{frame}

\begin{frame}
    \frametitle{Identifying Outliers}
    \begin{itemize}
        \item \textbf{Visual Inspection}: Use boxplots or scatter plots to detect outliers.
        \item \textbf{Statistical Methods}:
            \begin{itemize}
                \item Z-score Method:
                    \begin{equation}
                    Z = \frac{(X - \mu)}{\sigma}
                    \end{equation}
                \item IQR Method:
                    \begin{equation}
                    \text{Lower Bound} = Q1 - 1.5 \times \text{IQR}
                    \end{equation}
                    \begin{equation}
                    \text{Upper Bound} = Q3 + 1.5 \times \text{IQR}
                    \end{equation}
            \end{itemize}
    \end{itemize}
\end{frame}

\begin{frame}
    \frametitle{Key Points to Emphasize}
    \begin{itemize}
        \item \textbf{Data Integrity}: Cleaning data is essential for maintaining reliability of analysis.
        \item \textbf{Choice of Method}: The method chosen can significantly affect analysis results.
        \item \textbf{Iterative Process}: Data cleaning is not one-time; it may require revisiting as new insights emerge.
    \end{itemize}
\end{frame}

\begin{frame}
    \frametitle{Conclusion}
    Effective data cleaning is foundational for successful EDA. By addressing missing values and outliers, analysts can uncover important patterns, ensuring robust analysis processes.
\end{frame}

\begin{frame}[fragile]
    \frametitle{Statistical Summaries in EDA - Overview}
    \begin{block}{Overview of Key Statistical Measures}
        In Exploratory Data Analysis (EDA), understanding the underlying distributions of data is critical. Statistical summaries provide a foundation for this understanding.
    \end{block}
    We will discuss four essential statistical measures:
    \begin{itemize}
        \item Mean
        \item Median
        \item Mode
        \item Variance
    \end{itemize}
\end{frame}

\begin{frame}[fragile]
    \frametitle{Statistical Summaries in EDA - Mean, Median, Mode}
    \begin{block}{1. Mean}
        \begin{itemize}
            \item \textbf{Definition}: The mean is the average of a dataset.
            \item \textbf{Formula}:
            \begin{equation}
                \text{Mean} (\bar{x}) = \frac{\sum_{i=1}^{n} x_i}{n}
            \end{equation}
            \item \textbf{Example}: For the dataset [4, 8, 6, 5, 3]:
            \begin{equation}
                \text{Mean} = \frac{4+8+6+5+3}{5} = 5.2
            \end{equation}
            \item \textbf{Relevance}: Influenced by outliers; useful for symmetrically distributed data.
        \end{itemize}
    \end{block}

    \begin{block}{2. Median}
        \begin{itemize}
            \item \textbf{Definition}: The middle value in an ordered dataset.
            \item \textbf{Example}: For [4, 8, 6, 5, 3] (ordered: [3, 4, 5, 6, 8]): Median = 5
            \item For [4, 8, 6, 5] (ordered: [4, 5, 6, 8]): 
            \begin{equation}
                \text{Median} = \frac{5 + 6}{2} = 5.5
            \end{equation}
            \item \textbf{Relevance}: Robust against outliers and useful for skewed distributions.
        \end{itemize}
    \end{block}
\end{frame}

\begin{frame}[fragile]
    \frametitle{Statistical Summaries in EDA - Mode and Variance}
    \begin{block}{3. Mode}
        \begin{itemize}
            \item \textbf{Definition}: The value that appears most frequently in a dataset.
            \item \textbf{Example}: In [1, 2, 2, 3, 4]: Mode = 2; In [1, 1, 2, 3, 3]: modes are 1 and 3 (bimodal).
            \item \textbf{Relevance}: Useful for categorical data to identify the most common category.
        \end{itemize}
    \end{block}

    \begin{block}{4. Variance}
        \begin{itemize}
            \item \textbf{Definition}: Measures the spread of data points around the mean.
            \item \textbf{Formula}:
            \begin{equation}
                \text{Variance} (s^2) = \frac{\sum_{i=1}^{n} (x_i - \bar{x})^2}{n}
            \end{equation}
            \item \textbf{Example}: For the dataset [4, 8, 6, 5, 3], Mean = 5.2
            \begin{equation}
                s^2 = \frac{(4-5.2)^2 + (8-5.2)^2 + (6-5.2)^2 + (5-5.2)^2 + (3-5.2)^2}{5} = 2.96
            \end{equation}
            \item \textbf{Relevance}: Indicates degree of dispersion; higher variance indicates more spread out data.
        \end{itemize}
    \end{block}
\end{frame}

\begin{frame}
    \frametitle{Case Study: EDA Application}
    \begin{block}{Overview of Exploratory Data Analysis (EDA)}
        EDA is a critical process to understand patterns and insights in a dataset.
        Visualization tools like Matplotlib and Seaborn help uncover hidden trends, detect anomalies, and formulate hypotheses.
    \end{block}
\end{frame}

\begin{frame}[fragile]
    \frametitle{Real-World Example: Analyzing a Housing Dataset}
    \begin{itemize}
        \item Features of the dataset:
        \begin{itemize}
            \item \texttt{Square Footage}
            \item \texttt{Number of Bedrooms}
            \item \texttt{Age of House}
            \item \texttt{Price}
        \end{itemize}
        \item Goal: Explore the dataset to identify factors affecting housing prices.
    \end{itemize}
\end{frame}

\begin{frame}[fragile]
    \frametitle{Steps of EDA}
    \begin{enumerate}
        \item \textbf{Load the Dataset and Libraries}:
        \begin{lstlisting}[language=Python]
import pandas as pd
import matplotlib.pyplot as plt
import seaborn as sns

# Load dataset
df = pd.read_csv('housing_data.csv')
        \end{lstlisting}

        \item \textbf{Data Overview}:
        Use \texttt{.info()} and \texttt{.describe()} for data types and summary statistics.
        \begin{lstlisting}[language=Python]
print(df.info())
print(df.describe())
        \end{lstlisting}
    \end{enumerate}
\end{frame}

\begin{frame}[fragile]
    \frametitle{Visualizing the Data}
    \begin{itemize}
        \item \textbf{Distribution of House Prices}:
        \begin{lstlisting}[language=Python]
plt.figure(figsize=(10, 6))
sns.histplot(df['Price'], bins=30, kde=True)
plt.title('Distribution of Housing Prices')
plt.xlabel('Price')
plt.ylabel('Frequency')
plt.show()
        \end{lstlisting}
        \item \textbf{Correlation Matrix}:
        \begin{lstlisting}[language=Python]
plt.figure(figsize=(8, 6))
correlation_matrix = df.corr()
sns.heatmap(correlation_matrix, annot=True, cmap='coolwarm')
plt.title('Correlation Matrix')
plt.show()
        \end{lstlisting}
    \end{itemize}
\end{frame}

\begin{frame}[fragile]
    \frametitle{Visualizing the Data (continued)}
    \begin{itemize}
        \item \textbf{Boxplot for Prices by Bedrooms}:
        \begin{lstlisting}[language=Python]
plt.figure(figsize=(10, 6))
sns.boxplot(x='Number of Bedrooms', y='Price', data=df)
plt.title('Boxplot of Prices by Number of Bedrooms')
plt.xlabel('Number of Bedrooms')
plt.ylabel('Price')
plt.show()
        \end{lstlisting}
    \end{itemize}
\end{frame}

\begin{frame}
    \frametitle{Key Findings from EDA}
    \begin{itemize}
        \item The histogram reveals a right-skewed distribution, indicating most houses are sold at lower prices.
        \item The correlation matrix shows a strong positive correlation between \texttt{Square Footage} and \texttt{Price}.
        \item Boxplots indicate houses with more bedrooms tend to have higher prices, with notable outliers present.
    \end{itemize}
\end{frame}

\begin{frame}
    \frametitle{Conclusions from EDA}
    \begin{itemize}
        \item Factors such as size and number of bedrooms significantly affect housing prices.
        \item Identified anomalies and trends suggest further investigation on outlier prices.
    \end{itemize}
\end{frame}

\begin{frame}
    \frametitle{Key Points to Emphasize}
    \begin{itemize}
        \item EDA is iterative; initial findings often lead to more questions and deeper analyses.
        \item Visualization is essential for effectively communicating insights.
        \item Understanding correlations and distributions is fundamental to data-driven decision-making.
    \end{itemize}
\end{frame}

\begin{frame}[fragile]
    \frametitle{Summary and Best Practices - Overview}
    \begin{block}{Key Takeaways}
        \begin{enumerate}
            \item \textbf{Purpose of EDA:} EDA is essential for understanding data, identifying patterns, anomalies, and relationships.
            \item \textbf{Data Visualization:} Utilize tools like Matplotlib and Seaborn to visualize data through graphs (e.g., scatter plots, histograms).
            \item \textbf{Descriptive Statistics:} Summarize data using measures like mean, median, and standard deviation.
            \item \textbf{Data Cleaning:} Address missing values, duplicates, and outliers through various techniques.
        \end{enumerate}
    \end{block}
\end{frame}

\begin{frame}[fragile]
    \frametitle{Summary and Best Practices - Detailed Insights}
    \begin{block}{Best Practices}
        \begin{enumerate}
            \item \textbf{Start with Visualizations:} Create visual representations to quickly identify trends and patterns.
            \item \textbf{Engage in Data Profiling:} Understand variable types, missing values, and summary statistics.
            \item \textbf{Iterative Process:} EDA is an iterative journey; revisit previous steps as new insights arise.
            \item \textbf{Leverage Multiple Visualization Techniques:} Combine various visualizations for a comprehensive data view.
            \item \textbf{Document Findings:} Record key observations to ensure reproducibility in future analyses.
            \item \textbf{Automation with Code:} Utilize libraries like Pandas and Matplotlib for efficient data handling and visualization.
        \end{enumerate}
    \end{block}
\end{frame}

\begin{frame}[fragile]
    \frametitle{Summary and Best Practices - Example Code}
    \begin{block}{Example Code for EDA}
        \begin{lstlisting}[language=Python]
import pandas as pd
import seaborn as sns
import matplotlib.pyplot as plt

# Load dataset
data = pd.read_csv('data.csv')

# Visualize correlations
sns.heatmap(data.corr(), annot=True)
plt.show()
        \end{lstlisting}
    \end{block}
\end{frame}

\begin{frame}[fragile]
    \frametitle{Summary and Best Practices - Conclusion}
    \begin{block}{Emphasize the Importance of EDA}
        By adhering to these best practices, practitioners can enhance data insights and make informed, data-driven decisions. Approach EDA with curiosity and rigor to significantly improve analytical competency.
    \end{block}
\end{frame}


\end{document}