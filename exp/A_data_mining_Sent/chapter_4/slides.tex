\documentclass{beamer}

% Theme choice
\usetheme{Madrid} % You can change to e.g., Warsaw, Berlin, CambridgeUS, etc.

% Encoding and font
\usepackage[utf8]{inputenc}
\usepackage[T1]{fontenc}

% Graphics and tables
\usepackage{graphicx}
\usepackage{booktabs}

% Code listings
\usepackage{listings}
\lstset{
basicstyle=\ttfamily\small,
keywordstyle=\color{blue},
commentstyle=\color{gray},
stringstyle=\color{red},
breaklines=true,
frame=single
}

% Math packages
\usepackage{amsmath}
\usepackage{amssymb}

% Colors
\usepackage{xcolor}

% TikZ and PGFPlots
\usepackage{tikz}
\usepackage{pgfplots}
\pgfplotsset{compat=1.18}
\usetikzlibrary{positioning}

% Hyperlinks
\usepackage{hyperref}

% Title information
\title{Week 4: Classification Algorithms}
\author{Your Name}
\institute{Your Institution}
\date{\today}

\begin{document}

\frame{\titlepage}

\begin{frame}[fragile]
    \frametitle{Introduction to Classification Algorithms}
    \begin{block}{Overview}
        Classification algorithms are a subset of supervised learning methods in data mining and machine learning. Their primary function is to predict categorical labels for new observations based on a training dataset containing input features and known labels.
    \end{block}
\end{frame}

\begin{frame}[fragile]
    \frametitle{Importance in Data Mining}
    \begin{enumerate}
        \item \textbf{Decision-Making}: Enables data-driven decisions in various fields.
        \item \textbf{Automation}: Saves time and reduces errors in tasks like email filtering and sentiment analysis.
        \item \textbf{Predictive Analytics}: Uncovers patterns that help anticipate future events and trends.
    \end{enumerate}
\end{frame}

\begin{frame}[fragile]
    \frametitle{Key Concepts}
    \begin{itemize}
        \item \textbf{Training Data}: Labeled dataset used to train a classifier, including input features and corresponding output labels.
        \item \textbf{Model}: The mathematical representation created by the algorithm based on training data, used for predictions.
        \item \textbf{Evaluation Metrics}: Performance measured by accuracy, precision, recall, and F1-score.
    \end{itemize}
\end{frame}

\begin{frame}[fragile]
    \frametitle{Common Classification Algorithms}
    \begin{enumerate}
        \item \textbf{Decision Trees}: 
            - Splits data into branches based on feature values.
            - Example: Classifying emails as spam or not spam.
        \item \textbf{Support Vector Machines (SVM)}:
            - Finds the hyperplane that separates classes in high-dimensional space.
            - Example: Classifying images of cats vs. dogs.
        \item \textbf{Naive Bayes}:
            - A probabilistic classifier based on Bayes' theorem, assuming independence among predictors.
            - Example: Classifying text documents by topics.
        \item \textbf{Random Forest}:
            - An ensemble method constructing multiple decision trees and outputting the mode of their predictions.
            - Example: Predicting customer churn based on behavior features.
    \end{enumerate}
\end{frame}

\begin{frame}[fragile]
    \frametitle{Examples and Key Points}
    \begin{itemize}
        \item \textbf{Examples for Context}:
            - Spam Detection: Classifies emails as spam or legitimate.
            - Medical Diagnosis: Classifies symptoms to predict conditions like diabetes.
        \item \textbf{Key Points to Emphasize}:
            \begin{itemize}
                \item Classification falls under supervised learning using labeled input data.
                \item Many algorithms can be applied across various domains.
                \item Quality features are crucial for model accuracy.
            \end{itemize}
    \end{itemize}
\end{frame}

\begin{frame}[fragile]
    \frametitle{Conclusion}
    By understanding classification algorithms, students will enhance their data analytics and decision-making skills, applying these powerful techniques to real-world problems.
\end{frame}

\begin{frame}[fragile]
    \frametitle{Understanding Classification - Part 1}
    \textbf{What is Classification?} \\
    Classification is a fundamental concept in machine learning and artificial intelligence. It involves predicting the category or class label of a new observation based on past observations with known labels. 
    \begin{itemize}
        \item Identifying the category an object belongs to based on its features.
    \end{itemize}
\end{frame}

\begin{frame}[fragile]
    \frametitle{Understanding Classification - Part 2}
    \textbf{Role of Classification in Predictive Modeling:}
    \begin{itemize}
        \item \textbf{Data Input:} Classification algorithms receive data inputs with various features describing observations.
        \item \textbf{Learning Phase:} Algorithms learn patterns from a training dataset with known features and labels.
        \item \textbf{Prediction Phase:} Algorithms predict class labels for unseen observations.
        \item \textbf{Evaluation:} Model performance is assessed using metrics such as accuracy, precision, recall, and F1-score.
    \end{itemize}
\end{frame}

\begin{frame}[fragile]
    \frametitle{Understanding Classification - Part 3}
    \textbf{Examples of Classification:}
    \begin{enumerate}
        \item \textbf{Email Spam Detection:}
            \begin{itemize}
                \item Task: Classify emails as "Spam" or "Not Spam."
                \item Features: Keywords, email length, sender information.
            \end{itemize}
        \item \textbf{Medical Diagnosis:}
            \begin{itemize}
                \item Task: Predict disease presence based on symptoms.
                \item Features: Age, gender, blood test results.
            \end{itemize}
        \item \textbf{Sentiment Analysis:}
            \begin{itemize}
                \item Task: Determine if a movie review is positive, negative, or neutral.
                \item Features: Words and phrases in the review.
            \end{itemize}
    \end{enumerate}

    \textbf{Key Points to Emphasize:}
    \begin{itemize}
        \item Binary vs. Multiclass Classification.
        \item Importance of Training Data.
        \item Familiarity with common algorithms (e.g., Decision Trees, SVM).
        \item Real-world applications in finance, healthcare, and marketing.
    \end{itemize}
\end{frame}

\begin{frame}[fragile]
    \frametitle{Types of Classification Algorithms - Overview}
    Classification algorithms are vital tools in machine learning that help to categorize data into predefined classes based on input features. In this slide, we'll explore three popular classification algorithms:
    \begin{itemize}
        \item Decision Trees
        \item Support Vector Machines (SVM)
        \item Neural Networks
    \end{itemize}
    Understanding these algorithms will equip you with the ability to select the right tool for a given predictive modeling task.
\end{frame}

\begin{frame}[fragile]
    \frametitle{Types of Classification Algorithms - Decision Trees}
    \begin{block}{Description}
    Decision Trees are intuitive models that use a flowchart-like structure to make decisions based on input features.
    \end{block}
    
    \begin{itemize}
        \item **Interpretability:** Visual and straightforward explanation of decisions.
        \item **Non-parametric:** No specific assumption about data distribution.
        \item **Handles both numeric and categorical data.**
    \end{itemize}
    
    \begin{block}{Example}
    \texttt{
    [Attendance > 75\%] \\
    \hspace{2cm}/\hspace{2cm} \\
    \hspace{2cm}Yes\hspace{2cm}No\\
    \hspace{2cm}/\hspace{2cm} \hspace{2cm}/\hspace{2cm} \\
    [Hours > 5] \hspace{1cm}[Fail] \\
    \hspace{4cm}/\hspace{2cm} \hspace{2cm} \\
    \hspace{4cm}Pass\hspace{2cm}Fail  
    }
    \end{block}
\end{frame}

\begin{frame}[fragile]
    \frametitle{Types of Classification Algorithms - Support Vector Machines}
    \begin{block}{Description}
    SVM finds the hyperplane that best separates data points of different classes, maximizing the margin between them.
    \end{block}

    \begin{itemize}
        \item **Effective in high dimensions:** Good performance with a large number of features.
        \item **Versatile:** Can implement different kernel functions for non-linear separation.
    \end{itemize}

    \begin{equation}
    f(x) = w \cdot x + b
    \end{equation}
    Where:
    \begin{itemize}
        \item \( w \) is the weight vector determining the hyperplane orientation.
        \item \( b \) is the bias term adjusting the hyperplane position.
    \end{itemize}
    
    \begin{block}{Example}
    Classifying whether emails are spam or not using word frequencies.
    \end{block}
\end{frame}

\begin{frame}[fragile]
    \frametitle{Types of Classification Algorithms - Neural Networks}
    \begin{block}{Description}
    Neural Networks are inspired by the human brain, consisting of interconnected nodes (neurons) that process input data.
    \end{block}

    \begin{itemize}
        \item **Deep Learning Capabilities:** Learn hierarchical feature representations.
        \item **Highly flexible:** Capable of handling unstructured data (images, text).
    \end{itemize}

    \begin{block}{Basic Structure}
    A typical neural network consists of:
    \begin{itemize}
        \item **Input Layer:** Takes input features.
        \item **Hidden Layers:** Intermediate layers for computations.
        \item **Output Layer:** Produces final classification outputs.
    \end{itemize}
    \end{block}

    \begin{block}{Example}
    Classifying images of cats vs. dogs by learning features such as shape and color.
    \end{block}
\end{frame}

\begin{frame}[fragile]
    \frametitle{Types of Classification Algorithms - Key Points}
    \begin{itemize}
        \item Each algorithm has strengths and weaknesses; the choice depends on the problem and dataset.
        \item Decision Trees provide clear interpretability but may overfit.
        \item SVMs are powerful for separation tasks, especially in high-dimensional spaces.
        \item Neural Networks offer flexibility and performance but require careful training to avoid overfitting.
    \end{itemize}
    
    In summary, choosing the right classification algorithm involves understanding the problem, the data, and the specific characteristics of each algorithm. Next, we will explore real-world case studies to see these algorithms in action.
\end{frame}

\begin{frame}[fragile]
    \frametitle{Case Studies Overview - Introduction}
    
    Classification algorithms are essential tools in data science and machine learning, allowing us to categorize data points into distinct classes based on their features. 
    This slide introduces real-world case studies that highlight the practical applications of classification algorithms.

    \begin{block}{Key Concepts}
        \begin{itemize}
            \item \textbf{Classification Algorithms}:
            These are supervised learning models that predict discrete class labels.
            \item \textbf{Real-World Application}:
            Explore various case studies illustrating the effectiveness of classification algorithms across different domains.
        \end{itemize}
    \end{block}
\end{frame}

\begin{frame}[fragile]
    \frametitle{Case Studies Overview - Examples}
    
    \begin{enumerate}
        \item \textbf{Healthcare}:
            \begin{itemize}
                \item \textbf{Application}: Predicting disease outcomes.
                \item \textbf{Example}: Using Support Vector Machines to classify patients as high or low risk for diabetes.
                \item \textbf{Outcome}: Enhanced patient management and preventive care strategies. 
            \end{itemize}
            
        \item \textbf{Customer Segmentation}:
            \begin{itemize}
                \item \textbf{Application}: Classifying customers for targeted marketing.
                \item \textbf{Example}: Employing Decision Trees to segment customers based on purchase behavior and demographics.
                \item \textbf{Outcome}: Improved marketing strategies and higher conversion rates through personalized offers.
            \end{itemize}
    \end{enumerate}
\end{frame}

\begin{frame}[fragile]
    \frametitle{Case Studies Overview - More Examples}
    
    \begin{enumerate}
        \setcounter{enumi}{2} % Resume numbering from the previous frame
        \item \textbf{Image Recognition}:
            \begin{itemize}
                \item \textbf{Application}: Classifying images into categories (e.g., cat vs. dog).
                \item \textbf{Example}: Utilizing Neural Networks to distinguish between different objects in images.
                \item \textbf{Outcome}: Enhanced capabilities in automation for tasks like sorting images on social media platforms.
            \end{itemize}
            
        \item \textbf{Spam Detection}:
            \begin{itemize}
                \item \textbf{Application}: Identifying spam emails.
                \item \textbf{Example}: Implementing Naïve Bayes Classifier to classify emails as 'spam' or 'not spam'.
                \item \textbf{Outcome}: Improved email filtering and user experience.
            \end{itemize}
    \end{enumerate}
\end{frame}

\begin{frame}[fragile]
    \frametitle{Case Studies Overview - Key Points and Conclusion}

    \begin{block}{Key Points}
        \begin{itemize}
            \item \textbf{Real-World Impact}: Classification algorithms enhance decision-making processes and operational efficiencies across industries.
            \item \textbf{Diversity of Applications}: Their versatility makes them invaluable, from healthcare to marketing.
            \item \textbf{Adaptability}: Understanding various classification algorithms aids in selecting the best model for specific problems.
        \end{itemize}
    \end{block}

    \textbf{Conclusion}: This overview sets the stage for in-depth case studies starting with Decision Trees in Action. Prepare to explore the mechanics and outcomes of these applications!
\end{frame}

\begin{frame}[fragile]
    \frametitle{Decision Trees in Action}
    Decision Trees are powerful classification algorithms in machine learning and data mining. They model decisions and outcomes in a tree-like structure where:
    \begin{itemize}
        \item Each \textbf{internal node} represents a feature (attribute).
        \item Each \textbf{branch} represents a decision rule.
        \item Each \textbf{leaf node} represents an outcome (class label).
    \end{itemize}
    This structure allows for intuitive visualization and clear interpretation.
\end{frame}

\begin{frame}[fragile]
    \frametitle{Case Study: Classifying Customer Data}
    A retail company wants to classify customers into:
    \begin{itemize}
        \item High Value
        \item Medium Value
        \item Low Value
    \end{itemize}
    based on purchasing patterns using a Decision Tree.

    \textbf{Step 1: Data Collection}
    \begin{itemize}
        \item Age: Continuous feature
        \item Income: Continuous feature
        \item Purchase Frequency: Continuous feature
        \item Previous Purchases: Categorical feature (Yes/No)
    \end{itemize}
\end{frame}

\begin{frame}[fragile]
    \frametitle{Building the Decision Tree}
    \textbf{Step 3: Building the Decision Tree}  
    Use Gini Impurity as the splitting criterion:  
    \begin{equation}
    Gini(D) = 1 - \sum (p_i^2)
    \end{equation}  
    where \( p_i \) is the proportion of class \( i \) instances in dataset \( D \).
    
    \textbf{Example Tree Structure:}
    \begin{enumerate}
        \item \textbf{Root Node:} Income > \$50,000?
        \begin{itemize}
            \item Yes → Node 2
            \item No → Classify as Low Value
        \end{itemize}
        \item \textbf{Node 2:} Purchase Frequency > 5 times?
        \begin{itemize}
            \item Yes → Classify as High Value
            \item No → Classify as Medium Value
        \end{itemize}
    \end{enumerate}
\end{frame}

\begin{frame}[fragile]
    \frametitle{Model Training and Evaluation}
    \textbf{Step 4: Model Training and Evaluation}
    \begin{itemize}
        \item Split dataset into training and test sets (e.g., 80/20 split).
        \item Train the Decision Tree on the training data.
        \item Evaluate using metrics such as:
        \begin{itemize}
            \item Accuracy
            \item Precision
            \item Recall
        \end{itemize}
    \end{itemize}
\end{frame}

\begin{frame}[fragile]
    \frametitle{Key Points and Conclusion}
    \textbf{Key Points to Emphasize:}
    \begin{itemize}
        \item \textbf{Interpretability}: Easy to interpret and visualize.
        \item \textbf{Handling Non-linear Data}: Capable of capturing non-linear patterns.
        \item \textbf{Overfitting}: Prone to overfitting; mitigate with pruning and maximum depth settings.
    \end{itemize}

    \textbf{Conclusion:}  
    Decision Trees provide a robust method for classification tasks, enabling improved customer segmentation and satisfaction.
\end{frame}

\begin{frame}[fragile]
    \frametitle{Example Code Snippet (Python)}
    \begin{lstlisting}[language=Python]
import pandas as pd
from sklearn.tree import DecisionTreeClassifier
from sklearn.model_selection import train_test_split

# Sample Data
data = pd.DataFrame({
    'Age': [25, 35, 45, 30, 50],
    'Income': [48000, 60000, 72000, 58000, 80000],
    'Purchase_Frequency': [3, 7, 5, 6, 8],
    'Previous_Purchases': [1, 1, 0, 0, 1],
    'Value': ['Low', 'High', 'Medium', 'Medium', 'High']
})

# Split Data
X = data[['Age', 'Income', 'Purchase_Frequency', 'Previous_Purchases']]
y = data['Value']
X_train, X_test, y_train, y_test = train_test_split(X, y, test_size=0.2)

# Train Decision Tree
model = DecisionTreeClassifier()
model.fit(X_train, y_train)

# Evaluate Model
accuracy = model.score(X_test, y_test)
print(f'Model Accuracy: {accuracy}')
    \end{lstlisting}
\end{frame}

\begin{frame}[fragile]
    \frametitle{Support Vector Machines}
    \begin{block}{Description}
        Case study demonstrating the use of SVM in text categorization.
    \end{block}
\end{frame}

\begin{frame}[fragile]
    \frametitle{What are Support Vector Machines (SVM)?}
    Support Vector Machines (SVM) are a powerful supervised learning algorithm used for classification and regression tasks. SVM works by finding the hyperplane that best separates different classes in the feature space. The key idea is to maximize the margin between the classes, making the model more robust to unseen data.
\end{frame}

\begin{frame}[fragile]
    \frametitle{Key Concepts of SVM}
    \begin{itemize}
        \item \textbf{Hyperplane}: A decision boundary that separates different classes in the feature space.
        \item \textbf{Margin}: The distance between the closest points (support vectors) of the different classes from the hyperplane.
        \item \textbf{Support Vectors}: Data points that lie closest to the hyperplane and define its position and orientation.
    \end{itemize}
\end{frame}

\begin{frame}[fragile]
    \frametitle{SVM in Text Categorization}
    \begin{itemize}
        \item \textbf{Step 1: Data Acquisition} 
            \begin{itemize}
                \item Collect a labeled dataset of news articles categorized into predefined classes.
            \end{itemize}
        \item \textbf{Step 2: Text Preprocessing}
            \begin{itemize}
                \item Tokenization: Split articles into individual words.
                \item Vectorization: Convert text data into numerical format using TF-IDF.
            \end{itemize}
    \end{itemize}
\end{frame}

\begin{frame}[fragile]
    \frametitle{Example of TF-IDF Representation}
    \begin{block}{TF-IDF Representation}
        \texttt{Article: "The team won the game."} \\
        TF-IDF Vector: \([0.7, 0, 0.4, \ldots]\)
    \end{block}
\end{frame}

\begin{frame}[fragile]
    \frametitle{Model Training and Evaluation}
    \begin{itemize}
        \item \textbf{Step 3: Model Training} 
            \begin{itemize}
                \item Train the SVM model on the vectorized dataset to find the hyperplane.
            \end{itemize}
        \item \textbf{Step 4: Model Evaluation} 
            \begin{itemize}
                \item Use metrics like accuracy, precision, and recall on a separate test set.
            \end{itemize}
    \end{itemize}
\end{frame}

\begin{frame}[fragile]
    \frametitle{Advantages of SVM in Text Categorization}
    \begin{enumerate}
        \item \textbf{High Dimensionality}: Efficiently handles high-dimensional spaces.
        \item \textbf{Robust against Overfitting}: Less prone to overfitting compared to other algorithms.
        \item \textbf{Flexibility}: Can learn complex boundaries with kernel functions.
    \end{enumerate}
\end{frame}

\begin{frame}[fragile]
    \frametitle{Important Formulas}
    \begin{block}{Decision Function}
        \begin{equation}
            f(x) = \sum_{i=1}^N \alpha_i y_i K(x_i, x) + b
        \end{equation}
        Where:
        \begin{itemize}
            \item \(K\) is the kernel function
            \item \(\alpha\) are the weights learned
            \item \(y\) are the labels
            \item \(b\) is the bias
        \end{itemize}
    \end{block}
\end{frame}

\begin{frame}[fragile]
    \frametitle{Example Code Snippet (Python)}
    \begin{lstlisting}[language=Python]
from sklearn import SVC
from sklearn.feature_extraction.text import TfidfVectorizer
from sklearn.model_selection import train_test_split

# Example dataset
documents = ["Article 1 text...", "Article 2 text..."]
labels = ["Sports", "Politics"]

# Vectorization
vectorizer = TfidfVectorizer()
X = vectorizer.fit_transform(documents)

# Train-test split
X_train, X_test, y_train, y_test = train_test_split(X, labels, test_size=0.25)

# Model training
model = SVC(kernel='linear')
model.fit(X_train, y_train)

# Model prediction
predictions = model.predict(X_test)
    \end{lstlisting}
\end{frame}

\begin{frame}[fragile]
    \frametitle{Key Points to Emphasize}
    \begin{itemize}
        \item SVM is effective for text categorization due to its handling of high-dimensional data.
        \item Understanding preprocessing steps like tokenization and vectorization is crucial.
        \item Evaluate SVM performance with various metrics for reliable predictions.
    \end{itemize}
\end{frame}

\begin{frame}[fragile]
    \frametitle{Conclusion}
    By employing SVM in text categorization, organizations can efficiently automate the classification of vast amounts of textual data, enabling better information retrieval and management.
\end{frame}

\begin{frame}[fragile]
    \frametitle{Neural Networks for Classification}
    \begin{block}{Overview}
        Neural networks are a powerful subset of machine learning models, especially suited for complex classification tasks like image recognition.
        This slide examines a case study on applying neural networks to an image recognition problem, showcasing how they can classify objects in images efficiently.
    \end{block}
\end{frame}

\begin{frame}[fragile]
    \frametitle{Key Concepts}
    \begin{enumerate}
        \item \textbf{Neural Network Structure:}
        \begin{itemize}
            \item \textbf{Layers:} Input layer, hidden layers, output layer.
            \item \textbf{Neurons:} Computational units that process inputs.
            \item \textbf{Activation Functions:} ReLU, Sigmoid, Softmax introduce non-linearity.
        \end{itemize}
        
        \item \textbf{Forward Propagation:}
        \begin{itemize}
            \item Input images pass through layers, producing class probabilities.
        \end{itemize}

        \item \textbf{Backpropagation:}
        \begin{itemize}
            \item Adjusts weights based on error, employing optimization techniques like Adam or SGD.
        \end{itemize}
    \end{enumerate}
\end{frame}

\begin{frame}[fragile]
    \frametitle{Case Study: Image Recognition}
    \textbf{Example: Handwritten Digit Recognition} \\
    \textbf{Dataset:} MNIST (60,000 training and 10,000 testing images)

    \begin{block}{Architecture}
        \begin{itemize}
            \item Input Layer: 784 neurons (28x28 pixels)
            \item Hidden Layers: 2 layers with 128 and 64 neurons
            \item Output Layer: 10 neurons (one for each digit class)
        \end{itemize}
    \end{block}

    \textbf{Training Process:}
    \begin{itemize}
        \item Data Preprocessing: Normalize pixel values.
        \item Training: Cross-entropy loss for performance evaluation.
        \item Evaluation: Accuracy by comparing predicted against actual values.
    \end{itemize}
\end{frame}

\begin{frame}[fragile]
    \frametitle{Example Code Snippet}
    \begin{lstlisting}[language=Python]
from keras.models import Sequential
from keras.layers import Dense, Flatten
from keras.utils import to_categorical
from keras.datasets import mnist

# Load and preprocess the dataset
(x_train, y_train), (x_test, y_test) = mnist.load_data()
x_train, x_test = x_train / 255.0, x_test / 255.0
y_train = to_categorical(y_train, 10)
y_test = to_categorical(y_test, 10)

# Build the model
model = Sequential()
model.add(Flatten(input_shape=(28, 28)))
model.add(Dense(128, activation='relu'))
model.add(Dense(64, activation='relu'))
model.add(Dense(10, activation='softmax'))

# Compile the model
model.compile(optimizer='adam', loss='categorical_crossentropy', metrics=['accuracy'])

# Train the model
model.fit(x_train, y_train, epochs=10, batch_size=32, validation_split=0.2)
    \end{lstlisting}
\end{frame}

\begin{frame}[fragile]
    \frametitle{Evaluation Metrics}
    \begin{block}{Overview of Classification Metrics}
        In the realm of classification algorithms, assessing performance using metrics is essential. This slide focuses on three key evaluation metrics: 
        \begin{itemize}
            \item Precision
            \item Recall
            \item F1 Score
        \end{itemize}
        Understanding these metrics helps choose the most suitable model, particularly in scenarios with varying costs for false positives and false negatives.
    \end{block}
\end{frame}

\begin{frame}[fragile]
    \frametitle{Evaluation Metrics - Key Metrics Explained}
    \begin{enumerate}
        \item \textbf{Precision}
            \begin{itemize}
                \item \textbf{Definition}: Ratio of correctly predicted positives to total predicted positives. Answers: "Of all instances predicted as positive, how many were actually positive?"
                \item \textbf{Formula}:
                \begin{equation}
                    \text{Precision} = \frac{TP}{TP + FP}
                \end{equation}
                Where:
                \begin{itemize}
                    \item \(TP\) = True Positives
                    \item \(FP\) = False Positives
                \end{itemize}
                \item \textbf{Example}: 
                In email classification, if a spam filter marks 8 emails as spam (TP), but 2 were not spam (FP), then: 
                \begin{equation}
                    \text{Precision} = \frac{8}{8 + 2} = 0.8 \text{ or } 80\%
                \end{equation}
            \end{itemize}
    \end{enumerate}
\end{frame}

\begin{frame}[fragile]
    \frametitle{Evaluation Metrics - Continued}
    \begin{enumerate}[resume]
        \item \textbf{Recall (Sensitivity)}
            \begin{itemize}
                \item \textbf{Definition}: Ratio of correctly predicted positives to all actual positives. Answers: "Of all actual positive instances, how many did we predict correctly?"
                \item \textbf{Formula}:
                \begin{equation}
                    \text{Recall} = \frac{TP}{TP + FN}
                \end{equation}
                Where:
                \begin{itemize}
                    \item \(FN\) = False Negatives
                \end{itemize}
                \item \textbf{Example}: If there were actually 10 spam emails, and our filter correctly identified 8 but missed 2, then:
                \begin{equation}
                    \text{Recall} = \frac{8}{8 + 2} = 0.8 \text{ or } 80\%
                \end{equation}
        \end{itemize}

        \item \textbf{F1 Score}
            \begin{itemize}
                \item \textbf{Definition}: Harmonic mean of Precision and Recall, balancing the two metrics. Useful for imbalanced datasets.
                \item \textbf{Formula}:
                \begin{equation}
                    \text{F1 Score} = 2 \cdot \frac{\text{Precision} \cdot \text{Recall}}{\text{Precision} + \text{Recall}}
                \end{equation}
                \item \textbf{Example}: With Precision and Recall both at 80%:
                \begin{equation}
                    \text{F1 Score} = 2 \cdot \frac{0.8 \cdot 0.8}{0.8 + 0.8} = 0.8 \text{ or } 80\%
                \end{equation}
            \end{itemize}
    \end{enumerate}
\end{frame}

\begin{frame}[fragile]
    \frametitle{Ethical Considerations - Introduction}
    Classification algorithms are powerful tools in data analysis, but their impact extends beyond technical performance. 
    It is crucial to consider the ethical implications, particularly concerning:
    \begin{itemize}
        \item Bias in data
        \item Model transparency
    \end{itemize}
\end{frame}

\begin{frame}[fragile]
    \frametitle{Ethical Considerations - Bias in Data}
    \begin{block}{Definition}
        Bias occurs when certain groups are disproportionately represented in the training data, leading to unfair predictions for specific demographics.
    \end{block}
    
    \begin{itemize}
        \item \textbf{Impact:} Biased algorithms can perpetuate stereotypes, discriminate against marginalized groups, and lead to unjust outcomes.
        \item \textbf{Example:} 
        \begin{itemize}
            \item If a hiring algorithm is trained on data that has historically favored male candidates, the model may undervalue female applicants' qualifications.
        \end{itemize}
    \end{itemize}

    \begin{block}{Key Points to Emphasize}
        \begin{itemize}
            \item Recognize the source of bias: historical injustice, representation, and selection bias.
            \item Techniques to mitigate bias include:
            \begin{itemize}
                \item Diverse data collection
                \item Oversampling underrepresented classes
            \end{itemize}
        \end{itemize}
    \end{block}
\end{frame}

\begin{frame}[fragile]
    \frametitle{Ethical Considerations - Model Transparency}
    \begin{block}{Definition}
        Model transparency refers to how understandable and interpretable an algorithm's decision-making process is.
    \end{block}
    
    \begin{itemize}
        \item \textbf{Importance:} Users and stakeholders should understand how decisions are made for accountability and trust.
        \item \textbf{Illustration:} 
        \begin{itemize}
            \item A \textbf{black-box model} (like deep learning) provides high accuracy but lacks transparency.
            \item A \textbf{transparent model} (like decision trees) allows users to visualize and interpret decisions.
        \end{itemize}
    \end{itemize}

    \begin{block}{Key Points to Emphasize}
        \begin{itemize}
            \item The importance of explainability in high-stakes decisions (e.g., hiring, criminal justice).
            \item Tools for increasing transparency:
            \begin{itemize}
                \item Local Interpretable Model-agnostic Explanations (LIME)
                \item SHapley Additive exPlanations (SHAP)
            \end{itemize}
        \end{itemize}
    \end{block}
\end{frame}

\begin{frame}[fragile]
    \frametitle{Ethical Considerations - Legal and Social Implications}
    \begin{itemize}
        \item Algorithms are increasingly subject to legal scrutiny (e.g., GDPR in the EU mandates data transparency).
        \item Organizations must foster ethical AI practices to build trust and ensure compliance.
    \end{itemize}

    \begin{block}{Key Points to Emphasize}
        \begin{itemize}
            \item Recognize the societal impact of automated decisions.
            \item Aim for accountability through ethical design principles and user-inclusive practices.
        \end{itemize}
    \end{block}
\end{frame}

\begin{frame}[fragile]
    \frametitle{Ethical Considerations - Conclusion}
    Ethical considerations are paramount in the development and deployment of classification algorithms. 
    Addressing bias and enhancing transparency ensures fairness, accountability, and trust in AI systems, which is essential for their acceptance and effective implementation.
    \begin{block}{Summary}
        Always evaluate the ethical implications of classification algorithms to work towards technologies that not only perform well but also serve society equitably and responsibly.
    \end{block}
\end{frame}

\begin{frame}[fragile]
    \frametitle{Conclusion and Future Trends - Key Learning Points}
    
    \begin{enumerate}
        \item \textbf{Fundamentals of Classification Algorithms}:
            \begin{itemize}
                \item Used to categorize data into predefined classes.
                \item Examples:
                    \begin{itemize}
                        \item \textbf{Logistic Regression}: Predicts binary outcomes.
                        \item \textbf{Decision Trees}: Decisions based on a series of questions.
                        \item \textbf{Support Vector Machines (SVM)}: Finds optimal hyperplane.
                        \item \textbf{Neural Networks}: Learns complex patterns in large datasets.
                    \end{itemize}
            \end{itemize}
        
        \item \textbf{Performance Evaluation}:
            \begin{itemize}
                \item Key metrics: \textbf{Accuracy}, \textbf{Precision}, \textbf{Recall}, \textbf{F1 Score}.
                \item Example: \textbf{Precision} = True Positives / (True Positives + False Positives).
            \end{itemize}
        
        \item \textbf{Data Preprocessing}:
            \begin{itemize}
                \item Importance of cleaning data: handle missing values, scale features, encode categorical variables.
            \end{itemize}
    \end{enumerate}
\end{frame}

\begin{frame}[fragile]
    \frametitle{Conclusion and Future Trends - Key Learning Points (Continued)}
    
    \begin{enumerate}
        \setcounter{enumi}{3}
        \item \textbf{Overfitting and Underfitting}:
            \begin{itemize}
                \item \textbf{Overfitting}: Model learns noise, performs poorly on unseen data.
                \item \textbf{Underfitting}: Model cannot capture underlying trends.
                \item Techniques: Cross-validation, Regularization.
            \end{itemize}
    \end{enumerate}
\end{frame}

\begin{frame}[fragile]
    \frametitle{Conclusion and Future Trends - Emerging Trends}
    
    \begin{enumerate}
        \item \textbf{Automated Machine Learning (AutoML)}:
            \begin{itemize}
                \item Automation of machine learning processes.
                \item Tools: Google AutoML, H2O.ai.
            \end{itemize}
        
        \item \textbf{Explainable AI (XAI)}:
            \begin{itemize}
                \item Importance of model transparency in critical fields.
                \item Techniques like LIME to explain predictions from complex models.
            \end{itemize}
        
        \item \textbf{Ensemble Learning}:
            \begin{itemize}
                \item Combining models to improve performance (e.g., Random Forest, AdaBoost).
            \end{itemize}
    \end{enumerate}
\end{frame}

\begin{frame}[fragile]
    \frametitle{Conclusion and Future Trends - Additional Emerging Trends}
    
    \begin{enumerate}
        \setcounter{enumi}{3}
        \item \textbf{Robustness to Adversarial Attacks}:
            \begin{itemize}
                \item Resilience against adversarial inputs (small perturbations causing incorrect predictions).
            \end{itemize}
        
        \item \textbf{Fairness and Bias Mitigation}:
            \begin{itemize}
                \item Addressing bias in classification algorithms.
                \item Techniques: Re-sampling, re-weighting, and algorithmic adjustments.
            \end{itemize}
        
        \item \textbf{Conclusion}:
            \begin{itemize}
                \item Continuous evolution demands practitioners stay updated.
                \item Embracing emerging trends leads to accurate, fair, and interpretable models.
            \end{itemize}
    \end{enumerate}
\end{frame}


\end{document}