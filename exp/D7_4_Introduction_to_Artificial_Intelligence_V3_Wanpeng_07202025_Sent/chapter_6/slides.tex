\documentclass[aspectratio=169]{beamer}

% Theme and Color Setup
\usetheme{Madrid}
\usecolortheme{whale}
\useinnertheme{rectangles}
\useoutertheme{miniframes}

% Additional Packages
\usepackage[utf8]{inputenc}
\usepackage[T1]{fontenc}
\usepackage{graphicx}
\usepackage{booktabs}
\usepackage{listings}
\usepackage{amsmath}
\usepackage{amssymb}
\usepackage{xcolor}
\usepackage{tikz}
\usepackage{pgfplots}
\pgfplotsset{compat=1.18}
\usetikzlibrary{positioning}
\usepackage{hyperref}

% Custom Colors
\definecolor{myblue}{RGB}{31, 73, 125}
\definecolor{mygray}{RGB}{100, 100, 100}
\definecolor{mygreen}{RGB}{0, 128, 0}
\definecolor{myorange}{RGB}{230, 126, 34}
\definecolor{mycodebackground}{RGB}{245, 245, 245}

% Set Theme Colors
\setbeamercolor{structure}{fg=myblue}
\setbeamercolor{frametitle}{fg=white, bg=myblue}
\setbeamercolor{title}{fg=myblue}
\setbeamercolor{section in toc}{fg=myblue}
\setbeamercolor{item projected}{fg=white, bg=myblue}
\setbeamercolor{block title}{bg=myblue!20, fg=myblue}
\setbeamercolor{block body}{bg=myblue!10}
\setbeamercolor{alerted text}{fg=myorange}

% Set Fonts
\setbeamerfont{title}{size=\Large, series=\bfseries}
\setbeamerfont{frametitle}{size=\large, series=\bfseries}
\setbeamerfont{caption}{size=\small}
\setbeamerfont{footnote}{size=\tiny}

% Footer and Navigation Setup
\setbeamertemplate{footline}{
  \leavevmode%
  \hbox{%
    \begin{beamercolorbox}[wd=.3\paperwidth,ht=2.25ex,dp=1ex,center]{author in head/foot}%
      \usebeamerfont{author in head/foot}\insertshortauthor
    \end{beamercolorbox}%
    \begin{beamercolorbox}[wd=.5\paperwidth,ht=2.25ex,dp=1ex,center]{title in head/foot}%
      \usebeamerfont{title in head/foot}\insertshorttitle
    \end{beamercolorbox}%
    \begin{beamercolorbox}[wd=.2\paperwidth,ht=2.25ex,dp=1ex,center]{date in head/foot}%
      \usebeamerfont{date in head/foot}
      \insertframenumber{} / \inserttotalframenumber
    \end{beamercolorbox}}%
  \vskip0pt%
}

% Turn off navigation symbols
\setbeamertemplate{navigation symbols}{}

% Title Page Information
\title[Week 6: Logic Reasoning]{Week 6: Logic Reasoning: Propositional Logic}
\author[J. Smith]{John Smith, Ph.D.}
\institute[University Name]{
  Department of Computer Science\\
  University Name\\
  \vspace{0.3cm}
  Email: email@university.edu\\
  Website: www.university.edu
}
\date{\today}

% Document Start
\begin{document}

\frame{\titlepage}

\begin{frame}[fragile]
    \titlepage
\end{frame}

\begin{frame}[fragile]
    \frametitle{Introduction to Propositional Logic - Overview}
    \begin{block}{What is Propositional Logic?}
        Propositional Logic, also known as Boolean Logic, studies propositions which are declarative statements that can be either true (T) or false (F). It acts as a foundational element for reasoning in artificial intelligence, enabling machines to interpret complex scenarios.
    \end{block}
    
    \begin{block}{Importance of Propositional Logic}
        \begin{enumerate}
            \item \textbf{Understanding Complexity}: Simplifies complex scenarios into manageable components.
            \item \textbf{Logical Reasoning}: Forms the basis for algorithms to make inferences and decisions in AI.
            \item \textbf{Mathematical Foundation}: Essential for grasping advanced topics in logic and computer science.
        \end{enumerate}
    \end{block}
\end{frame}

\begin{frame}[fragile]
    \frametitle{Key Concepts of Propositional Logic}
    \begin{itemize}
        \item \textbf{Propositions}: A statement that can be either true (T) or false (F). 
        \begin{block}{Example}
            "The sky is blue" can either be true or false.
        \end{block}
        
        \item \textbf{Logical Connectives}: Operators that combine propositions:
        \begin{itemize}
            \item \textbf{AND ($\land$)}: True if both propositions are true (e.g., $P \land Q$ is true only when both $P$ and $Q$ are true).
            \item \textbf{OR ($\lor$)}: True if at least one proposition is true (e.g., $P \lor Q$ is true if either $P$ or $Q$, or both are true).
            \item \textbf{NOT ($\neg$)}: Negates the truth of a proposition (e.g., if $P$ is true, then $\neg P$ is false).
        \end{itemize}
        
        \item \textbf{Truth Tables}: A method to show all possible truth values for a set of propositions.
    \end{itemize}
\end{frame}

\begin{frame}[fragile]
    \frametitle{Truth Table Example}
    \begin{block}{Example of a Truth Table for $P \land Q$}
        \begin{center}
        \begin{tabular}{|c|c|c|}
            \hline
            $P$ & $Q$ & $P \land Q$ \\
            \hline
            T   & T   & T         \\
            T   & F   & F         \\
            F   & T   & F         \\
            F   & F   & F         \\
            \hline
        \end{tabular}
        \end{center}
    \end{block}

    \begin{block}{Key Points to Emphasize}
        \begin{itemize}
            \item Propositional logic is fundamental for logical reasoning in AI.
            \item It reduces problems into smaller, more manageable logical constituents.
            \item Understanding logical connectives and truth tables is essential for accurate reasoning in AI.
        \end{itemize}
    \end{block}
\end{frame}

\begin{frame}[fragile]
    \frametitle{Conclusion}
    Propositional logic serves as the bedrock of logical reasoning within artificial intelligence. It provides the necessary tools for students to tackle more complex reasoning systems, clarifying foundational logical principles. Mastering these concepts enhances reasoning skills and aids in the development of sophisticated AI systems capable of complex decision-making.
\end{frame}

\begin{frame}[fragile]
    \frametitle{What is Propositional Logic?}
    \begin{block}{Definition}
        Propositional logic is a branch of logic that deals with propositions, which are declarative statements that can be either true or false, but not both.
    \end{block}

    \begin{block}{Significance in AI and Reasoning}
        \begin{itemize}
            \item \textbf{Automated Reasoning}: Essential for automated reasoning systems, enabling computers to infer from knowledge bases.
            \item \textbf{Decision Making}: Facilitates AI decision-making processes by evaluating complex propositions.
            \item \textbf{Knowledge Representation}: Provides a structured way to represent facts about the world.
        \end{itemize}
    \end{block}
\end{frame}

\begin{frame}[fragile]
    \frametitle{Key Components of Propositional Logic}
    \begin{enumerate}
        \item \textbf{Propositions}: Fundamental units expressing statements.
            \begin{itemize}
                \item Example: 
                    \begin{itemize}
                        \item p: "It is raining."
                        \item q: "I will take an umbrella."
                    \end{itemize}
            \end{itemize}
        \item \textbf{Logical Connectives}: 
            \begin{itemize}
                \item \textbf{AND (∧)}: Both must be true (e.g., \( p \land q \)).
                \item \textbf{OR (∨)}: At least one must be true (e.g., \( p \lor q \)).
                \item \textbf{NOT (¬)}: Negates (e.g., \( \neg p \)).
                \item \textbf{IMPLICATION (→)}: If the first is true, the second must be true (e.g., \( p \rightarrow q \)).
            \end{itemize}
        \item \textbf{Truth Values}: Each proposition has a truth value (True or False).
    \end{enumerate}
\end{frame}

\begin{frame}[fragile]
    \frametitle{Example of Propositional Logic Application}
    \begin{block}{Scenario}
        Consider:
        \begin{itemize}
            \item p: "It is sunny."
            \item q: "We will go to the park."
        \end{itemize}
        This can be expressed as: 
        \[
        p \rightarrow q
        \]
    \end{block}

    \begin{block}{Truth Table}
        \begin{table}[h]
            \centering
            \begin{tabular}{|c|c|c|}
                \hline
                p (It is sunny) & q (Go to the park) & p → q (Implication) \\
                \hline
                True  & True   & True   \\
                True  & False  & False  \\
                False & True   & True   \\
                False & False  & True   \\
                \hline
            \end{tabular}
            \caption{Truth table for implication}
        \end{table}
    \end{block}

    \begin{block}{Key Points}
        \begin{itemize}
            \item Propositional logic provides a systematic approach to represent and reason about information.
            \item Essential for developing intelligent systems to automate logical decision-making.
        \end{itemize}
    \end{block}
\end{frame}

\begin{frame}[fragile]
    \frametitle{Components of Propositional Logic - Overview}
    \begin{block}{Learning Objectives}
        \begin{itemize}
            \item Understand the fundamental components of propositional logic.
            \item Identify and differentiate between propositions, logical connectives, and truth values.
            \item Apply these concepts to evaluate logical statements.
        \end{itemize}
    \end{block}
\end{frame}

\begin{frame}[fragile]
    \frametitle{Components of Propositional Logic - Propositions}
    \begin{block}{1. Propositions}
        \begin{itemize}
            \item \textbf{Definition}: A proposition is a declarative statement that can be either true or false, but not both.
            \item \textbf{Examples}:
                \begin{itemize}
                    \item "The sky is blue." (True)
                    \item "2 + 2 = 5." (False)
                \end{itemize}
            \item Propositions serve as the building blocks for more complex logical expressions.
            \item Non-propositional statements (e.g., questions, commands) cannot be evaluated for truth values.
        \end{itemize}
    \end{block}
\end{frame}

\begin{frame}[fragile]
    \frametitle{Components of Propositional Logic - Logical Connectives}
    \begin{block}{2. Logical Connectives}
        \begin{itemize}
            \item Logical connectives are symbols or words used to connect propositions and form new propositions. The four primary connectives are:
            \begin{enumerate}
                \item \textbf{AND (∧)}: True if both propositions are true.
                \item \textbf{OR (∨)}: True if at least one proposition is true (inclusive OR).
                \item \textbf{NOT (¬)}: Negates the truth value of a proposition.
                \item \textbf{IMPLIES (→)}: Represents a conditional relationship; true unless a true proposition leads to a false one.
            \end{enumerate}
            \item Truth tables illustrate the truth values for these connectives.
        \end{itemize}
    \end{block}
\end{frame}

\begin{frame}[fragile]
    \frametitle{Truth Tables for Logical Connectives}
    \begin{block}{AND (∧)}
        \begin{tabular}{|c|c|c|}
            \hline
            P & Q & P ∧ Q \\
            \hline
            T & T & T \\
            T & F & F \\
            F & T & F \\
            F & F & F \\
            \hline
        \end{tabular}
    \end{block}

    \begin{block}{OR (∨)}
        \begin{tabular}{|c|c|c|}
            \hline
            P & Q & P ∨ Q \\
            \hline
            T & T & T \\
            T & F & T \\
            F & T & T \\
            F & F & F \\
            \hline
        \end{tabular}
    \end{block}
\end{frame}

\begin{frame}[fragile]
    \frametitle{Truth Tables Continued}
    \begin{block}{NOT (¬)}
        \begin{tabular}{|c|c|}
            \hline
            P & ¬P \\
            \hline
            T & F \\
            F & T \\
            \hline
        \end{tabular}
    \end{block}

    \begin{block}{IMPLIES (→)}
        \begin{tabular}{|c|c|c|}
            \hline
            P & Q & P → Q \\
            \hline
            T & T & T \\
            T & F & F \\
            F & T & T \\
            F & F & T \\
            \hline
        \end{tabular}
    \end{block}
\end{frame}

\begin{frame}[fragile]
    \frametitle{Components of Propositional Logic - Truth Values}
    \begin{block}{3. Truth Values}
        \begin{itemize}
            \item Each proposition has a truth value: \textbf{True (T)} or \textbf{False (F)}.
            \item The combination of propositions and logical connectives yields new propositions, with truth values determined through operations.
            \item Understanding how truth values change with different logical connectives is crucial for evaluating logical statements correctly.
            \item Truth tables provide a systematic way to visualize the outcomes of logical operations.
        \end{itemize}
    \end{block}
\end{frame}

\begin{frame}[fragile]
    \frametitle{Summary of Propositional Logic}
    \begin{itemize}
        \item \textbf{Propositions} form the basis of logical reasoning.
        \item \textbf{Logical connectives} allow us to combine propositions, leading to more complex expressions.
        \item \textbf{Truth values} help in determining the validity of logical statements made through propositions and connectives.
    \end{itemize}
\end{frame}

\begin{frame}[fragile]
    \frametitle{Types of Propositions - Overview}
    \begin{block}{Overview}
        In propositional logic, \textbf{propositions} are statements that can be either true or false. Understanding the different types of propositions is critical for logical reasoning, as they form the foundation for constructing logical arguments and evaluating their validity.
    \end{block}
\end{frame}

\begin{frame}[fragile]
    \frametitle{Types of Propositions - Simple and Compound}
    \begin{enumerate}
        \item \textbf{Simple Propositions}
            \begin{itemize}
                \item \textbf{Definition}: A simple proposition is a statement that does not contain any other propositions as parts.
                \item \textbf{Example}: ``The sky is blue.''
            \end{itemize}
        
        \item \textbf{Compound Propositions}
            \begin{itemize}
                \item \textbf{Definition}: A compound proposition is formed by combining two or more simple propositions using logical connectives.
                \item \textbf{Example}: ``The sky is blue and the grass is green.''
            \end{itemize}
    \end{enumerate}
\end{frame}

\begin{frame}[fragile]
    \frametitle{Types of Propositions - Universal, Existential, and Conditional}
    \begin{enumerate}
        \setcounter{enumi}{2}
        \item \textbf{Universal Propositions}
            \begin{itemize}
                \item \textbf{Definition}: Assert that a statement is true for all members of a particular group.
                \item \textbf{Example}: ``All humans are mortal.''
            \end{itemize}
        
        \item \textbf{Existential Propositions}
            \begin{itemize}
                \item \textbf{Definition}: Claim that there exists at least one member of a category for which the statement is true.
                \item \textbf{Example}: ``Some birds can fly.''
            \end{itemize}
        
        \item \textbf{Conditional Propositions}
            \begin{itemize}
                \item \textbf{Definition}: Express a relationship of dependence between two propositions, typically in an 'if...then' format.
                \item \textbf{Example}: ``If it rains, then the ground will be wet.''
            \end{itemize}
    \end{enumerate}
\end{frame}

\begin{frame}[fragile]
    \frametitle{Types of Propositions - Negation and Key Points}
    \begin{enumerate}
        \setcounter{enumi}{5}
        \item \textbf{Negation}
            \begin{itemize}
                \item \textbf{Definition}: The negation of a proposition asserts the opposite truth value.
                \item \textbf{Symbol}: $\neg P$ (not P)
                \item \textbf{Example}: If P is ``It is raining,'' then $\neg P$ would be ``It is not raining.''
            \end{itemize}
    \end{enumerate}
    
    \begin{block}{Key Points to Remember}
        \begin{itemize}
            \item Propositions are essential in logical reasoning.
            \item Recognize the difference between simple and compound propositions.
            \item Be mindful of how universal and existential propositions affect generalizations.
            \item Understanding conditional relationships and negations is crucial for evaluating arguments logically.
        \end{itemize}
    \end{block}
    
    \begin{block}{Next Steps}
        In the next slide, we will explore \textbf{Logical Connectives}, which are critical for forming compound propositions.
    \end{block}
\end{frame}

\begin{frame}[fragile]
    \frametitle{Logical Connectives - Introduction}
    \begin{block}{Overview}
        Logical connectives are crucial in propositional logic as they allow us to combine or modify propositions (statements that can be either true or false). Here are the primary logical connectives:
    \end{block}
    
    \begin{itemize}
        \item AND (Conjunction)
        \item OR (Disjunction)
        \item NOT (Negation)
        \item IMPLIES (Conditional)
    \end{itemize}
\end{frame}

\begin{frame}[fragile]
    \frametitle{Logical Connectives - Definitions and Examples}
    \begin{enumerate}
        \item \textbf{AND (Conjunction)} 
        \begin{itemize}
            \item \textbf{Symbol:} $\land$ 
            \item \textbf{Definition:} True if both propositions are true.
            \item Example: $A \land B$ (It is raining AND it is cold).
            \item Truth Table:
        \end{itemize}
        \begin{center}
            \begin{tabular}{|c|c|c|}
                \hline
                A & B & A $\land$ B \\
                \hline
                T & T & T \\
                T & F & F \\
                F & T & F \\
                F & F & F \\
                \hline
            \end{tabular}
        \end{center}
        
        \item \textbf{OR (Disjunction)} 
        \begin{itemize}
            \item \textbf{Symbol:} $\lor$
            \item \textbf{Definition:} True if at least one proposition is true.
            \item Example: $A \lor B$ (I will go for a walk OR I will read a book).
            \item Truth Table:
        \end{itemize}
        \begin{center}
            \begin{tabular}{|c|c|c|}
                \hline
                A & B & A $\lor$ B \\
                \hline
                T & T & T \\
                T & F & T \\
                F & T & T \\
                F & F & F \\
                \hline
            \end{tabular}
        \end{center}
    \end{enumerate}
\end{frame}

\begin{frame}[fragile]
    \frametitle{Logical Connectives - Continued}
    \begin{enumerate}[resume]
        \item \textbf{NOT (Negation)} 
        \begin{itemize}
            \item \textbf{Symbol:} $\neg$ 
            \item \textbf{Definition:} True if the proposition is false.
            \item Example: $¬A$ (It is NOT sunny).
            \item Truth Table:
        \end{itemize}
        \begin{center}
            \begin{tabular}{|c|c|}
                \hline
                A & ¬A \\
                \hline
                T & F \\
                F & T \\
                \hline
            \end{tabular}
        \end{center}
        
        \item \textbf{IMPLIES (Conditional)} 
        \begin{itemize}
            \item \textbf{Symbol:} $\rightarrow$ 
            \item \textbf{Definition:} True unless A is true and B is false.
            \item Example: $A \rightarrow B$ (If it rains, THEN the ground will be wet).
            \item Truth Table:
        \end{itemize}
        \begin{center}
            \begin{tabular}{|c|c|c|}
                \hline
                A & B & A $\rightarrow$ B \\
                \hline
                T & T & T \\
                T & F & F \\
                F & T & T \\
                F & F & T \\
                \hline
            \end{tabular}
        \end{center}
    \end{enumerate}
\end{frame}

\begin{frame}[fragile]
  \frametitle{Truth Tables - Introduction}
  \begin{block}{What is a Truth Table?}
    A \textbf{truth table} is a mathematical table used to determine the truth values of propositions based on their logical connectives. It systematically lists all possible truth values of a logical expression, showcasing how the truth or falsity of individual propositions combines to affect the overall truth of more complex statements.
  \end{block}
  
  \begin{itemize}
    \item \textbf{Propositions:} Basic statements that can be true (T) or false (F).
    \item \textbf{Logical Connectives:} Symbols connecting propositions:
      \begin{itemize}
        \item AND ( $\land$ )
        \item OR ( $\lor$ )
        \item NOT ( $\neg$ )
        \item IMPLIES ( $\to$ )
      \end{itemize}
  \end{itemize}
\end{frame}

\begin{frame}[fragile]
  \frametitle{Truth Tables - Example and Construction}
  \textbf{Example: Simple Propositions}
  
  Let’s consider the propositions:
  \begin{itemize}
    \item \( P \): "It is raining."
    \item \( Q \): "I will take an umbrella."
  \end{itemize}
  
  \textbf{Constructing a Truth Table for \( P \land Q \):}
  
  \begin{table}[h]
    \centering
    \begin{tabular}{|c|c|c|}
      \hline
      $P$ & $Q$ & $P \land Q$ \\
      \hline
      T & T & T \\
      T & F & F \\
      F & T & F \\
      F & F & F \\
      \hline
    \end{tabular}
  \end{table}
  
  \textbf{Interpretation:} The statement \( P \land Q \) is true only when both \( P \) and \( Q \) are true.
\end{frame}

\begin{frame}[fragile]
  \frametitle{Truth Tables - Complex Examples}
  \textbf{Example with More Complex Propositions}
  
  Now, consider \( P \lor Q \) (P OR Q) and \( \neg P \) (NOT P):
  
  \begin{table}[h]
    \centering
    \begin{tabular}{|c|c|c|c|}
      \hline
      $P$ & $Q$ & $P \lor Q$ & $\neg P$ \\
      \hline
      T & T & T & F \\
      T & F & T & F \\
      F & T & T & T \\
      F & F & F & T \\
      \hline
    \end{tabular}
  \end{table}
  
  \textbf{Key Points to Emphasize:}
  \begin{itemize}
    \item Truth tables are essential in fields like computer science, mathematics, and philosophy.
    \item As the number of propositions increases, the truth table grows exponentially; for \( n \) propositions, there are \( 2^n \) combinations.
    \item They help test the validity of logical arguments by revealing if a conclusion follows true premises.
  \end{itemize}
  
  \textbf{Conclusion:} Truth tables are fundamental tools for evaluating logical statements.
\end{frame}

\begin{frame}[fragile]
    \frametitle{Constructing Truth Tables}
    \begin{block}{Learning Objectives}
        \begin{itemize}
            \item Understand the purpose and structure of truth tables.
            \item Learn the step-by-step process to construct truth tables for compound propositions.
            \item Apply truth tables to evaluate the truth values of logical expressions.
        \end{itemize}
    \end{block}
\end{frame}

\begin{frame}[fragile]
    \frametitle{What is a Truth Table?}
    A truth table is a systematic way to evaluate the truth values of a logical expression based on the truth values of its components. 
    It lays out all possible combinations of truth values for the propositions involved.
\end{frame}

\begin{frame}[fragile]
    \frametitle{Steps to Construct Truth Tables}
    \begin{enumerate}
        \item \textbf{Identify the Propositions:} Count the basic propositions (e.g., P, Q) to determine the table size.
        \item \textbf{Determine the Number of Rows:} For each combination of truth values, create rows.
        \item \textbf{Create the Truth Table Structure:} Set up your table with headers for the propositions.
    \end{enumerate}
    
    \begin{block}{Example Setup}
        \begin{table}[]
            \centering
            \begin{tabular}{|c|c|c|c|}
                \hline
                P & Q & P AND Q & P OR Q \\ \hline
                T & T & ? & ? \\ \hline
                T & F & ? & ? \\ \hline
                F & T & ? & ? \\ \hline
                F & F & ? & ? \\ \hline
            \end{tabular}
        \end{table}
    \end{block}
\end{frame}

\begin{frame}[fragile]
    \frametitle{Filling in the Truth Table}
    \begin{itemize}
        \item Use logical operators to fill in values:
        \begin{itemize}
            \item \textbf{AND (∧):} True only if both propositions are True.
            \item \textbf{OR (∨):} True if at least one proposition is True.
        \end{itemize}
    \end{itemize}
    
    \begin{block}{Example Evaluation}
        \begin{itemize}
            \item P AND Q:
            \begin{itemize}
                \item Row 1: T AND T = T
                \item Row 2: T AND F = F
                \item Row 3: F AND T = F
                \item Row 4: F AND F = F
            \end{itemize}
            \item P OR Q:
            \begin{itemize}
                \item Row 1: T OR T = T
                \item Row 2: T OR F = T
                \item Row 3: F OR T = T
                \item Row 4: F OR F = F
            \end{itemize}
        \end{itemize}
    \end{block}
\end{frame}

\begin{frame}[fragile]
    \frametitle{Completed Truth Table}
    \begin{block}{Final Table}
    \begin{table}[]
        \centering
        \begin{tabular}{|c|c|c|c|}
            \hline
            P & Q & P AND Q & P OR Q \\ \hline
            T & T & T & T \\ \hline
            T & F & F & T \\ \hline
            F & T & F & T \\ \hline
            F & F & F & F \\ \hline
        \end{tabular}
    \end{table}
    \end{block}
    
    \begin{block}{Key Points}
        \begin{itemize}
            \item Truth tables help visualize complex logical expressions.
            \item Evaluate compound propositions based on assigned truth values.
            \item Include all combinations for accurate evaluation.
        \end{itemize}
    \end{block}
\end{frame}

\begin{frame}[fragile]
    \frametitle{Example Compound Propositions}
    Construct similar tables for expressions:
    \begin{itemize}
        \item \textbf{NOT P}
        \item \textbf{P OR (Q AND NOT P)}
        \item \textbf{(P AND Q) OR NOT Q}
    \end{itemize}

    With these steps, you can construct truth tables for any compound proposition, clarifying logic problems effectively.
\end{frame}

\begin{frame}[fragile]
    \frametitle{Methods of Inference}
    \begin{block}{Learning Objectives}
        \begin{itemize}
            \item Understand the basic methods of inference in propositional logic.
            \item Identify and apply Modus Ponens and Modus Tollens correctly.
        \end{itemize}
    \end{block}
\end{frame}

\begin{frame}[fragile]
    \frametitle{Overview of Inference in Propositional Logic}
    \begin{block}{Overview}
        Inference is a fundamental aspect of logical reasoning in propositional logic. It allows us to derive conclusions from premises using established rules. Two primary methods of inference are \textbf{Modus Ponens} and \textbf{Modus Tollens}.
    \end{block}
\end{frame}

\begin{frame}[fragile]
    \frametitle{Modus Ponens}
    \begin{block}{Definition}
        Modus Ponens is a rule of inference that states if we have a conditional statement (if-then) and the antecedent (the "if" part) is true, then we can conclude that the consequent (the "then" part) is also true.
    \end{block}
    \begin{block}{Structure}
        \begin{itemize}
            \item Premise 1: \( P \rightarrow Q \) (If P, then Q)
            \item Premise 2: \( P \) (P is true)
            \item Conclusion: \( Q \) (Therefore, Q is true)
        \end{itemize}
    \end{block}
    \begin{block}{Example}
        \begin{itemize}
            \item Premise 1: If it rains, then the ground will be wet. ( \( P \rightarrow Q \) )
            \item Premise 2: It is raining. ( \( P \) )
            \item Conclusion: Therefore, the ground is wet. ( \( Q \) )
        \end{itemize}
    \end{block}
\end{frame}

\begin{frame}[fragile]
    \frametitle{Modus Tollens}
    \begin{block}{Definition}
        Modus Tollens is another rule of inference that enables us to conclude that if the consequent of a conditional statement is false, then the antecedent must also be false.
    \end{block}
    \begin{block}{Structure}
        \begin{itemize}
            \item Premise 1: \( P \rightarrow Q \) (If P, then Q)
            \item Premise 2: \( \neg Q \) (Q is not true)
            \item Conclusion: \( \neg P \) (Therefore, P is not true)
        \end{itemize}
    \end{block}
    \begin{block}{Example}
        \begin{itemize}
            \item Premise 1: If it rains, then the ground will be wet. ( \( P \rightarrow Q \) )
            \item Premise 2: The ground is not wet. ( \( \neg Q \) )
            \item Conclusion: Therefore, it is not raining. ( \( \neg P \) )
        \end{itemize}
    \end{block}
\end{frame}

\begin{frame}[fragile]
    \frametitle{Key Points and Summary}
    \begin{itemize}
        \item \textbf{Modus Ponens} affirms the antecedent to conclude the consequent.
        \item \textbf{Modus Tollens} negates the consequent to conclude the negation of the antecedent.
        \item Both methods are essential for valid reasoning in propositional logic.
    \end{itemize}
    \begin{block}{Summary}
        Understanding Modus Ponens and Modus Tollens is crucial for constructing valid arguments. These methods allow us to deduce conclusions logically from given premises, forming the backbone of logical reasoning.
    \end{block}
\end{frame}

\begin{frame}[fragile]
    \frametitle{Overview}
    \begin{block}{Definition}
        Rules of inference are fundamental principles that allow us to derive new propositions from existing ones in propositional logic. These rules help in constructing valid arguments and reasoning processes.
    \end{block}

    \begin{block}{Learning Objectives}
        \begin{itemize}
            \item Understand key rules of inference in propositional logic.
            \item Apply these rules to deduce conclusions from given premises.
            \item Identify valid argument forms through practice problems.
        \end{itemize}
    \end{block}
\end{frame}

\begin{frame}[fragile]
    \frametitle{Key Rules of Inference - Part 1}
    \begin{enumerate}
        \item \textbf{Modus Ponens (MP)}
            \begin{itemize}
                \item Structure: If P, then Q (P $\rightarrow$ Q); P is true; therefore, Q is true.
                \item Example: If it rains, then the ground is wet (P $\rightarrow$ Q); It is raining (P); Therefore, the ground is wet (Q).
            \end{itemize}
        
        \item \textbf{Modus Tollens (MT)}
            \begin{itemize}
                \item Structure: If P, then Q (P $\rightarrow$ Q); Q is false; therefore, P is false.
                \item Example: If it is a cat, then it has whiskers (P $\rightarrow$ Q); It does not have whiskers (¬Q); Therefore, it is not a cat (¬P).
            \end{itemize}
    \end{enumerate}
\end{frame}

\begin{frame}[fragile]
    \frametitle{Key Rules of Inference - Part 2}
    \begin{enumerate}
        \setcounter{enumi}{2}
        \item \textbf{Hypothetical Syllogism (HS)}
            \begin{itemize}
                \item Structure: If P then Q (P $\rightarrow$ Q); If Q then R (Q $\rightarrow$ R); therefore, If P then R (P $\rightarrow$ R).
                \item Example: If I study, I will pass (P $\rightarrow$ Q); If I pass, I will celebrate (Q $\rightarrow$ R); Therefore, If I study, I will celebrate (P $\rightarrow$ R).
            \end{itemize}

        \item \textbf{Disjunctive Syllogism (DS)}
            \begin{itemize}
                \item Structure: P or Q (P $\lor$ Q); not P (¬P); therefore, Q is true.
                \item Example: Either I will go to the park or stay home (P $\lor$ Q); I will not go to the park (¬P); Therefore, I will stay home (Q).
            \end{itemize}

        \item \textbf{Constructive Dilemma (CD)}
            \begin{itemize}
                \item Structure: If P then Q (P $\rightarrow$ Q); If R then S (R $\rightarrow$ S); P or R (P $\lor$ R); therefore, Q or S (Q $\lor$ S).
                \item Example: If I exercise, I will be fit (P $\rightarrow$ Q); If I eat healthy, I will be healthy (R $\rightarrow$ S); I will either exercise or eat healthy (P $\lor$ R); Therefore, I will either be fit or healthy (Q $\lor$ S).
            \end{itemize}
    \end{enumerate}
\end{frame}

\begin{frame}[fragile]
    \frametitle{Key Points to Emphasize}
    \begin{itemize}
        \item Rules of inference form the foundation for logical reasoning.
        \item These rules help in verifying the soundness of arguments.
        \item Mastery of these rules is essential for logical problem-solving in mathematics and computer science.
    \end{itemize}
\end{frame}

\begin{frame}[fragile]
    \frametitle{Practice Problems}
    \begin{enumerate}
        \item Given: If the light is on, then the room is lit (P $\rightarrow$ Q). The light is off (¬P). What can we conclude?
        \item Given: If it is a weekend, I will relax. If I relax, I will watch a movie. It is a weekend. What is the conclusion?
    \end{enumerate}

    \begin{block}{Summary}
        By understanding and applying these rules of inference, students will improve their logical reasoning skills and ability to form valid arguments in propositional logic.
    \end{block}
\end{frame}

\begin{frame}[fragile]
    \frametitle{Valid Argument Forms}
    \begin{block}{Learning Objectives}
        \begin{enumerate}
            \item Understand the significance of valid argument forms in propositional logic.
            \item Identify and apply common valid argument forms to evaluate logical statements.
        \end{enumerate}
    \end{block}
\end{frame}

\begin{frame}[fragile]
    \frametitle{Explanation of Valid Argument Forms}
    \begin{block}{Definition}
        A \textbf{valid argument form} refers to a specific structure of reasoning where, if the premises are true, the conclusion must also be true. Understanding these forms allows us to construct rational arguments and assess the validity of reasoning.
    \end{block}
    
    \begin{itemize}
        \item \textbf{Premises}: Statements that provide the groundwork for the argument.
        \item \textbf{Conclusion}: The statement that follows logically from the premises.
    \end{itemize}
\end{frame}

\begin{frame}[fragile]
    \frametitle{Importance of Valid Argument Forms}
    \begin{itemize}
        \item \textbf{Logical Rigor}: Ensures that reasoning is sound and conclusions are justified.
        \item \textbf{Problem-Solving}: Aids in analyzing complex arguments in programming, mathematics, and everyday decision-making.
        \item \textbf{Foundation of Formal Proofs}: Many formal proofs in mathematics and computer science rely on these valid forms.
    \end{itemize}
\end{frame}

\begin{frame}[fragile]
    \frametitle{Common Valid Argument Forms}
    \begin{itemize}
        \item \textbf{Modus Ponens} (Affirming the Antecedent):
        \begin{quote}
            If P, then Q.\\
            P.\\
            Therefore, Q.
        \end{quote}
        \textbf{Example}: If it rains, then the ground is wet. It is raining. Therefore, the ground is wet.

        \item \textbf{Modus Tollens} (Denying the Consequent):
        \begin{quote}
            If P, then Q.\\
            Not Q.\\
            Therefore, not P.
        \end{quote}
        \textbf{Example}: If it rains, then the ground is wet. The ground is not wet. Therefore, it is not raining.
    \end{itemize}
\end{frame}

\begin{frame}[fragile]
    \frametitle{Common Valid Argument Forms (Cont'd)}
    \begin{itemize}
        \item \textbf{Disjunctive Syllogism}:
        \begin{quote}
            P or Q.\\
            Not P.\\
            Therefore, Q.
        \end{quote}
        \textbf{Example}: Either the cat is inside or the cat is outside. The cat is not inside. Therefore, the cat is outside.

        \item \textbf{Constructive Dilemma}:
        \begin{quote}
            If P, then Q.\\
            If R, then S.\\
            P or R.\\
            Therefore, Q or S.
        \end{quote}
        \textbf{Example}: If you study hard, you will pass. If you cheat, you will get caught. Either you study hard or you cheat. Therefore, you will pass or you will get caught.
    \end{itemize}
\end{frame}

\begin{frame}[fragile]
    \frametitle{Key Points to Emphasize}
    \begin{itemize}
        \item Valid argument forms help establish a logical connection between premises and conclusion.
        \item Recognizing these forms enhances critical thinking skills applicable in various fields, including programming and philosophy.
        \item Practice constructing arguments using these forms to reinforce your understanding of logical reasoning.
    \end{itemize}
\end{frame}

\begin{frame}[fragile]
    \frametitle{Conclusion}
    \begin{block}{Summary}
        Mastering valid argument forms is crucial for evaluating logical consistency and ensuring sound reasoning in both theoretical and practical applications of propositional logic. Remember that every valid form is a tool for building a robust argument framework.
    \end{block}
    
    \begin{block}{Engagement}
        Feel free to engage with these forms through exercises or examples you encounter in everyday reasoning or academic problems!
    \end{block}
\end{frame}

\begin{frame}[fragile]
    \frametitle{Example Applications - Introduction}
    \begin{block}{Introduction to Propositional Logic in AI}
        Propositional logic forms the foundation for decision-making processes in Artificial Intelligence (AI). It uses statements that can be either true (T) or false (F) to automate reasoning and facilitate complex problem-solving.
    \end{block}
\end{frame}

\begin{frame}[fragile]
    \frametitle{Example Applications - Overview}
    \begin{block}{Real-World Applications}
        \begin{enumerate}
            \item Automated Deduction Systems
            \item Smart Home Automation
            \item Recommendation Systems
            \item Game AI Decision Making
        \end{enumerate}
    \end{block}
\end{frame}

\begin{frame}[fragile]
    \frametitle{Example Applications - Automated Deduction Systems}
    \begin{block}{Automated Deduction Systems}
        \begin{itemize}
            \item \textbf{Description:} Deductive reasoning systems used in legal reasoning or medical diagnosis apply propositional logic to arrive at conclusions based on the truth of premises.
            \item \textbf{Example:}
            \begin{align*}
                \text{P1:} & \quad \text{"If the defendant has an alibi (A), they are not guilty (G)."} \\
                \text{P2:} & \quad \text{"The defendant has an alibi."} \\
                \text{Conclusion:} & \quad \text{"Therefore, the defendant is not guilty." (Using modus ponens)}
            \end{align*}
        \end{itemize}
    \end{block}
\end{frame}

\begin{frame}[fragile]
    \frametitle{Example Applications - Smart Home Automation}
    \begin{block}{Smart Home Automation}
        \begin{itemize}
            \item \textbf{Description:} Smart home devices utilize propositional logic to enable conditional actions based on the status of inputs (e.g., sensors).
            \item \textbf{Example:}
            \begin{align*}
                \text{P1:} & \quad \text{"If the front door is open (D), then turn on the security camera (C)."} \\
                \text{P2:} & \quad \text{"The front door is open."} \\
                \text{Conclusion:} & \quad \text{"Turn on the security camera."}
            \end{align*}
        \end{itemize}
    \end{block}
\end{frame}

\begin{frame}[fragile]
    \frametitle{Example Applications - Recommendation Systems}
    \begin{block}{Recommendation Systems}
        \begin{itemize}
            \item \textbf{Description:} AI systems suggest products or content to users based on logical conditions derived from user preferences and past behaviors.
            \item \textbf{Example:}
            \begin{align*}
                \text{P1:} & \quad \text{"If the user liked action movies (X), recommend action movie A."} \\
                \text{P2:} & \quad \text{"The user liked action movies."} \\
                \text{Conclusion:} & \quad \text{"Recommend action movie A."}
            \end{align*}
        \end{itemize}
    \end{block}
\end{frame}

\begin{frame}[fragile]
    \frametitle{Example Applications - Game AI Decision Making}
    \begin{block}{Game AI Decision Making}
        \begin{itemize}
            \item \textbf{Description:} In strategic games, AI agents utilize propositional logic to make decisions based on the state of the game.
            \item \textbf{Example:}
            \begin{align*}
                \text{P1:} & \quad \text{"If the opponent's health is low (H), then attack aggressively (A)."} \\
                \text{P2:} & \quad \text{"The opponent's health is low."} \\
                \text{Conclusion:} & \quad \text{"Attack aggressively."}
            \end{align*}
        \end{itemize}
    \end{block}
\end{frame}

\begin{frame}[fragile]
    \frametitle{Example Applications - Key Points and Conclusion}
    \begin{block}{Key Points to Emphasize}
        \begin{itemize}
            \item \textbf{Flexibility:} Propositional logic allows for representation of various real-life scenarios in a formalized manner that machines can process.
            \item \textbf{Automation:} Systematic application of rules leads to automated reasoning in complex decision-making environments.
            \item \textbf{Accuracy:} Logical conclusions derived from valid premises lead to accurate and reliable outcomes in AI systems.
        \end{itemize}
    \end{block}
    
    \textbf{Conclusion:} Propositional logic is a fundamental tool for AI in reasoning and automating decision-making across various fields, including law, home automation, recommendations, and gaming strategies.
\end{frame}

\begin{frame}[fragile]
    \frametitle{Common Misconceptions in Propositional Logic}
    \begin{block}{Understanding Propositional Logic}
        Propositional logic is a form of logic where statements, known as propositions, can either be true or false. Understanding this framework is crucial in logical reasoning, especially in fields like computer science and artificial intelligence.
    \end{block}
\end{frame}

\begin{frame}[fragile]
    \frametitle{Common Misconceptions - Part 1}
    \begin{enumerate}
        \item \textbf{Misconception: “If A is true, then B must also be true.”}
        \begin{itemize}
            \item \textbf{Clarification:} This assumes a direct causal relationship that does not exist in propositional logic.
        \end{itemize}
        
        \item \textbf{Misconception: "A proposition is always either true or false."}
        \begin{itemize}
            \item \textbf{Clarification:} In classical propositional logic, this is true; however, fuzzy logic contexts allow degrees of truth.
        \end{itemize}
        
        \item \textbf{Misconception: “Negation of a false proposition is true, but there are no other implications.”}
        \begin{itemize}
            \item \textbf{Clarification:} Compound propositions must be understood, for example:
            \begin{equation}
                A \land B \text{ is true only if both A and B are true.}
            \end{equation}
        \end{itemize}
    \end{enumerate}
\end{frame}

\begin{frame}[fragile]
    \frametitle{Common Misconceptions - Part 2}
    \begin{enumerate}
        \setcounter{enumi}{3}
        \item \textbf{Misconception: “The symbols in propositional logic are just randomized letters.”}
        \begin{itemize}
            \item \textbf{Clarification:} Each letter (p, q, r) represents specific propositions, vital for correct logical interpretation.
        \end{itemize}
        
        \item \textbf{Misconception: “Logical equivalences can be applied in any context without consideration.”}
        \begin{itemize}
            \item \textbf{Clarification:} Logical equivalences, like De Morgan’s laws, must be applied within their contexts:
            \begin{equation}
                \neg (A \land B) \equiv \neg A \lor \neg B
            \end{equation}
            \begin{equation}
                \neg (A \lor B) \equiv \neg A \land \neg B
            \end{equation}
        \end{itemize}
    \end{enumerate}
\end{frame}

\begin{frame}[fragile]
    \frametitle{Key Points and Conclusion}
    \begin{itemize}
        \item \textbf{Understanding Connectors:} Logical connectors (AND, OR, NOT) define relationships between propositions.
        \item \textbf{Evaluating Compound Statements:} Analyze all components to determine the truth value of complex statements.
        \item \textbf{Context Matters:} Engage with propositions in their specific context to avoid misinterpretations.
    \end{itemize}

    \begin{block}{Conclusion}
        Addressing these misconceptions enhances comprehension and application of propositional logic which is essential for effective logical reasoning in both academic and practical scenarios.
    \end{block}
\end{frame}

\begin{frame}[fragile]
    \frametitle{Practice Problems}
    \begin{block}{Learning Objectives}
        \begin{itemize}
            \item Reinforce understanding of propositional logic.
            \item Apply logical operators to solve problems.
            \item Develop skills in constructing and evaluating logical statements.
        \end{itemize}
    \end{block}
\end{frame}

\begin{frame}[fragile]
    \frametitle{Introduction to Propositional Logic}
    Propositional logic is fundamental in the study of logic, which involves creating and analyzing statements that can either be true or false. The primary operators include:
    
    \begin{itemize}
        \item **AND ($\land$)**: True if both propositions are true.
        \item **OR ($\lor$)**: True if at least one proposition is true.
        \item **NOT ($\neg$)**: True if the proposition is false.
        \item **IMPLIES ($\rightarrow$)**: True unless a true proposition implies a false one.
        \item **BICONDITIONAL ($\leftrightarrow$)**: True if both propositions are either true or false.
    \end{itemize}
\end{frame}

\begin{frame}[fragile]
    \frametitle{Practice Problems}
    \begin{enumerate}
        \item \textbf{Basic Truth Tables:}
        \begin{itemize}
            \item Construct the truth table for the expression $P \lor Q$.
            \item Here, $P$ and $Q$ are two propositions that can either be true (T) or false (F).
        \end{itemize}
        \begin{center}
            \begin{tabular}{|c|c|c|}
                \hline
                P & Q & P $\lor$ Q \\
                \hline
                T & T & T \\
                T & F & T \\
                F & T & T \\
                F & F & F \\
                \hline
            \end{tabular}
        \end{center}
        
        \item \textbf{Evaluating Statements:}
        \begin{itemize}
            \item Determine the truth value of the compound statement assuming $P$ is true and $Q$ is false: $(P \land \neg Q) \rightarrow Q$.
            \item \textbf{Steps:}
            \begin{itemize}
                \item Determine $\neg Q$: Since $Q$ is false, $\neg Q$ is true.
                \item Calculate $P \land \neg Q$: Both $P$ (true) and $\neg Q$ (true) yield true.
                \item Evaluate $(P \land \neg Q) \rightarrow Q$: True implies false is false.
            \end{itemize}
        \end{itemize}
    \end{enumerate}
\end{frame}

\begin{frame}[fragile]
    \frametitle{Logical Expressions and Puzzles}
    \begin{enumerate}
        \setcounter{enumi}{2}
        \item \textbf{Logic Puzzle:}
        \begin{itemize}
            \item A farmer has a dog (D) and a cat (C). The dog barks if and only if the cat is outside. If the cat is inside and the dog is barking, what can we conclude about the dog and the cat?
            \item \textbf{Solution Steps:}
            \begin{itemize}
                \item The dog barks (D) $\leftrightarrow$ the cat is outside (C).
                \item Assume $C$ is false (the cat is inside).
                \item If $C$ is false (cat inside), then $D$ must also be false (the dog is not barking).
            \end{itemize}
        \end{itemize}
        
        \item \textbf{Constructing Logical Expressions:}
        \begin{itemize}
            \item Write a logical expression for the following statement: "It is not raining or it is sunny."
            \item \textbf{Answer:}
            \begin{itemize}
                \item The logical expression: $\neg R \lor S$ where $R$ = "It is raining" and $S$ = "It is sunny".
            \end{itemize}
        \end{itemize}
    \end{enumerate}
\end{frame}

\begin{frame}[fragile]
    \frametitle{Key Points and Next Steps}
    \begin{block}{Key Points to Emphasize}
        \begin{itemize}
            \item Remember to utilize truth tables to visualize logical statements’ outcomes.
            \item Practice converting phrases into logical expressions to strengthen understanding of propositional logic.
            \item Review logical implications and equivalences thoroughly; they are crucial for more complex logical reasoning.
        \end{itemize}
    \end{block}
    \begin{block}{Next Steps}
        Prepare for the upcoming slide on "Review and Key Takeaways," where we will summarize the key concepts covered in this chapter and how they relate to reasoning in AI.
    \end{block}
\end{frame}

\begin{frame}[fragile]
    \frametitle{Review and Key Takeaways}
    % Brief Summary of Key Concepts
    \begin{block}{Summary}
        This presentation recaps the core concepts of propositional logic and highlights their implications for reasoning in AI. You will learn about propositions, logical connectives, truth tables, and logical equivalence, along with their significance in AI applications.
    \end{block}
\end{frame}

\begin{frame}[fragile]
    \frametitle{Key Concepts of Propositional Logic}
    % Key Concepts Overview
    \begin{enumerate}
        \item \textbf{Propositions}
        \begin{itemize}
            \item Declarative statements that are either true or false.
            \item Examples: "It is raining." and "2 + 2 = 4."
        \end{itemize}

        \item \textbf{Logical Connectives}
        \begin{itemize}
            \item AND ( ∧ ): True if both propositions are true.
            \item OR ( ∨ ): True if at least one proposition is true.
            \item NOT ( ¬ ): Inverts the truth value of a proposition.
            \item IMPLICATION ( → ): True unless a true proposition implies a false one.
            \item BICONDITIONAL ( ↔ ): True if both propositions have the same truth value.
        \end{itemize}
    \end{enumerate}
\end{frame}

\begin{frame}[fragile]
    \frametitle{Truth Tables and Logical Equivalence}
    % Truth Tables and Equivalence
    \begin{itemize}
        \item \textbf{Truth Tables}
        \begin{itemize}
            \item Systematic method for exploring truth values under different conditions.
            \item Example for AND ( p ∧ q ):
            \begin{center}
                \begin{tabular}{|c|c|c|}
                    \hline
                    p & q & p ∧ q \\
                    \hline
                    T & T & T \\
                    T & F & F \\
                    F & T & F \\
                    F & F & F \\
                    \hline
                \end{tabular}
            \end{center}
        \end{itemize}

        \item \textbf{Logical Equivalence}
        \begin{itemize}
            \item Two statements are logically equivalent if they share truth values in all situations.
            \item Example: De Morgan's Laws:
            \begin{itemize}
                \item ¬(p ∧ q) ≡ (¬p ∨ ¬q)
                \item ¬(p ∨ q) ≡ (¬p ∧ ¬q)
            \end{itemize}
        \end{itemize}
    \end{itemize}
\end{frame}

\begin{frame}[fragile]
    \frametitle{Implications for Reasoning in AI}
    % Key Implications Overview
    \begin{itemize}
        \item \textbf{Decision Making}
            \begin{itemize}
                \item Provides a foundation for AI systems to deduce conclusions from known information.
            \end{itemize}

        \item \textbf{Knowledge Representation}
            \begin{itemize}
                \item Logical constructs are used to represent knowledge in a machine-processable format.
            \end{itemize}

        \item \textbf{Automated Theorem Proving}
            \begin{itemize}
                \item Enables systems to automatically prove statements or theorems.
            \end{itemize}

        \item \textbf{Natural Language Processing}
            \begin{itemize}
                \item Parsing human language into propositional forms enhances comprehension and output generation.
            \end{itemize}
    \end{itemize}
\end{frame}

\begin{frame}[fragile]
    \frametitle{Key Points to Emphasize}
    % Important Points to Keep in Mind
    \begin{itemize}
        \item Distinguish between propositions and logical operators.
        \item Practice creating and interpreting truth tables for validation of logical expressions.
        \item Familiarize with logical equivalences to simplify complex expressions.
    \end{itemize}

    \begin{block}{Conclusion}
        Mastering these concepts will enhance your reasoning skills within AI, enabling the development of systems that replicate human logical reasoning efficiently.
    \end{block}
\end{frame}

\begin{frame}[fragile]
    \frametitle{Connecting to Larger Themes in AI}
    \begin{block}{Understanding Propositional Logic in AI and Machine Learning}
        Propositional logic is critical in AI, serving as a foundational aspect of logical reasoning. It allows machines to analyze truth values and build knowledge representation systems.
    \end{block}
\end{frame}

\begin{frame}[fragile]
    \frametitle{Defining Propositional Logic}
    \begin{itemize}
        \item Propositional Logic deals with propositions that are either true or false, combined using logical connectives (AND, OR, NOT).
        \item Forms the basis of logical reasoning in AI systems.
    \end{itemize}
\end{frame}

\begin{frame}[fragile]
    \frametitle{Importance and Applications of Propositional Logic}
    \begin{enumerate}
        \item \textbf{Foundation for Reasoning}: Enables machines to deduce new information.
        \item \textbf{Simplifying Complex Problems}: Breaks down challenges into simpler propositions.
        \item \textbf{Decision-making}: Essential in evaluating conditions and executing actions.
    \end{enumerate}
\end{frame}

\begin{frame}[fragile]
    \frametitle{Key Applications in AI}
    \begin{itemize}
        \item \textbf{Expert Systems}: Automate reasoning and solutions in knowledge databases (e.g. medical diagnosis).
        \item \textbf{Natural Language Processing (NLP)}: Used for parsing sentences and understanding human language.
        \item \textbf{Circuit Design and Verification}: Logical propositions represent inputs/outputs for circuit designs.
    \end{itemize}
\end{frame}

\begin{frame}[fragile]
    \frametitle{Key Points to Emphasize}
    \begin{itemize}
        \item \textbf{Bridging Human and Machine Intelligence}: Replicates human-like reasoning capabilities.
        \item \textbf{Scalability}: Essential for constructing scalable AI systems.
        \item \textbf{Critical Role in Algorithms}: Underpins many algorithms in rule-based systems and classifiers.
    \end{itemize}
\end{frame}

\begin{frame}[fragile]
    \frametitle{Conclusion}
    By understanding propositional logic, students gain insights into the logical foundations that enable advanced machine learning algorithms, vital for comprehending AI's capabilities and limitations.
\end{frame}

\begin{frame}[fragile]
    \frametitle{Q\&A Session - Objectives}
    \begin{itemize}
        \item Promote understanding of key concepts in propositional logic.
        \item Address any misconceptions or challenges faced in the study of logic.
        \item Reinforce learning through discussion and clarifications.
    \end{itemize}
\end{frame}

\begin{frame}[fragile]
    \frametitle{Q\&A Session - Core Concepts of Propositional Logic}
    \begin{enumerate}
        \item \textbf{Propositions}
        \begin{itemize}
            \item Declarative statements that are either true or false.
            \item Example: "The sky is blue."
        \end{itemize}
        
        \item \textbf{Logical Connectives}
        \begin{itemize}
            \item \textbf{AND ( $\land$ )}: True if both propositions are true.
            \item \textbf{OR ( $\lor$ )}: True if at least one proposition is true.
            \item \textbf{NOT ( $\neg$ )}: Negates the truth value of a proposition.
            \item \textbf{IMPLIES ( $\rightarrow$ )}: True unless a true proposition implies a false one.
        \end{itemize}
        
        \item \textbf{Truth Tables}
        \begin{itemize}
            \item Used to determine the truth value of compound propositions.
        \end{itemize}
    \end{enumerate}
\end{frame}

\begin{frame}[fragile]
    \frametitle{Q\&A Session - Key Points and Sample Questions}
    \begin{block}{Key Points to Emphasize}
        \begin{itemize}
            \item Understanding logical structure of arguments is crucial.
            \item Propositional logic is foundational in AI for reasoning and decision-making.
            \item Practice constructing and analyzing propositions for proficiency.
        \end{itemize}
    \end{block}

    \begin{block}{Sample Questions to Engage Students}
        \begin{itemize}
            \item What are real-world applications of propositional logic in AI?
            \item Where might the truth of a compound statement be ambiguous?
            \item How do negations (NOT) affect interpreted meanings?
        \end{itemize}
    \end{block}

    \begin{block}{Encouragement for Questions}
        Please feel free to ask any questions, no matter how basic or complex.
    \end{block}
\end{frame}


\end{document}