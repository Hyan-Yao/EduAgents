\documentclass[aspectratio=169]{beamer}

% Theme and Color Setup
\usetheme{Madrid}
\usecolortheme{whale}
\useinnertheme{rectangles}
\useoutertheme{miniframes}

% Additional Packages
\usepackage[utf8]{inputenc}
\usepackage[T1]{fontenc}
\usepackage{graphicx}
\usepackage{booktabs}
\usepackage{listings}
\usepackage{amsmath}
\usepackage{amssymb}
\usepackage{xcolor}
\usepackage{tikz}
\usepackage{pgfplots}
\pgfplotsset{compat=1.18}
\usetikzlibrary{positioning}
\usepackage{hyperref}

% Custom Colors
\definecolor{myblue}{RGB}{31, 73, 125}
\definecolor{mygray}{RGB}{100, 100, 100}
\definecolor{mygreen}{RGB}{0, 128, 0}
\definecolor{myorange}{RGB}{230, 126, 34}
\definecolor{mycodebackground}{RGB}{245, 245, 245}

% Set Theme Colors
\setbeamercolor{structure}{fg=myblue}
\setbeamercolor{frametitle}{fg=white, bg=myblue}
\setbeamercolor{title}{fg=myblue}
\setbeamercolor{section in toc}{fg=myblue}
\setbeamercolor{item projected}{fg=white, bg=myblue}
\setbeamercolor{block title}{bg=myblue!20, fg=myblue}
\setbeamercolor{block body}{bg=myblue!10}
\setbeamercolor{alerted text}{fg=myorange}

% Set Fonts
\setbeamerfont{title}{size=\Large, series=\bfseries}
\setbeamerfont{frametitle}{size=\large, series=\bfseries}
\setbeamerfont{caption}{size=\small}
\setbeamerfont{footnote}{size=\tiny}

% Code Listing Style
\lstdefinestyle{customcode}{
  backgroundcolor=\color{mycodebackground},
  basicstyle=\footnotesize\ttfamily,
  breakatwhitespace=false,
  breaklines=true,
  commentstyle=\color{mygreen}\itshape,
  keywordstyle=\color{blue}\bfseries,
  stringstyle=\color{myorange},
  numbers=left,
  numbersep=8pt,
  numberstyle=\tiny\color{mygray},
  frame=single,
  framesep=5pt,
  rulecolor=\color{mygray},
  showspaces=false,
  showstringspaces=false,
  showtabs=false,
  tabsize=2,
  captionpos=b
}
\lstset{style=customcode}

% Custom Commands
\newcommand{\hilight}[1]{\colorbox{myorange!30}{#1}}
\newcommand{\source}[1]{\vspace{0.2cm}\hfill{\tiny\textcolor{mygray}{Source: #1}}}
\newcommand{\concept}[1]{\textcolor{myblue}{\textbf{#1}}}
\newcommand{\separator}{\begin{center}\rule{0.5\linewidth}{0.5pt}\end{center}}

% Footer and Navigation Setup
\setbeamertemplate{footline}{
  \leavevmode%
  \hbox{%
  \begin{beamercolorbox}[wd=.3\paperwidth,ht=2.25ex,dp=1ex,center]{author in head/foot}%
    \usebeamerfont{author in head/foot}\insertshortauthor
  \end{beamercolorbox}%
  \begin{beamercolorbox}[wd=.5\paperwidth,ht=2.25ex,dp=1ex,center]{title in head/foot}%
    \usebeamerfont{title in head/foot}\insertshorttitle
  \end{beamercolorbox}%
  \begin{beamercolorbox}[wd=.2\paperwidth,ht=2.25ex,dp=1ex,center]{date in head/foot}%
    \usebeamerfont{date in head/foot}
    \insertframenumber{} / \inserttotalframenumber
  \end{beamercolorbox}}%
  \vskip0pt%
}

% Turn off navigation symbols
\setbeamertemplate{navigation symbols}{}

% Title Page Information
\title[Introduction to Machine Learning]{Week 12: Introduction to Machine Learning}
\author[J. Smith]{John Smith, Ph.D.}
\institute[University Name]{
  Department of Computer Science\\
  University Name\\
  \vspace{0.3cm}
  Email: email@university.edu\\
  Website: www.university.edu
}
\date{\today}

% Document Start
\begin{document}

\frame{\titlepage}

\begin{frame}[fragile]
    \frametitle{Introduction to Machine Learning - Overview}
    \begin{block}{Overview of Machine Learning}
        Machine learning (ML) is a subset of artificial intelligence (AI) that enables computers to learn from and make predictions or decisions based on data without being explicitly programmed. 
    \end{block}
    
    \begin{itemize}
        \item Focuses on developing algorithms that identify patterns.
        \item Learns from past experiences and adapts to new information.
    \end{itemize}
\end{frame}

\begin{frame}[fragile]
    \frametitle{Introduction to Machine Learning - Significance}
    \begin{block}{Significance in Artificial Intelligence}
        \begin{itemize}
            \item \textbf{Automation:} Automates tasks that traditionally require human intelligence.
            \item \textbf{Personalization:} Enhances customer experiences with personalized recommendations.
            \item \textbf{Improved Decision-Making:} Uncovers insights and trends from vast amounts of data for better decisions.
        \end{itemize}
    \end{block}
\end{frame}

\begin{frame}[fragile]
    \frametitle{Objectives of This Week}
    \begin{enumerate}
        \item \textbf{Understanding Basic Concepts:} Explore what ML is and how it operates.
        \item \textbf{Types of Machine Learning:} Identify supervised, unsupervised, and reinforcement learning.
        \item \textbf{Applications of Machine Learning:} Discuss real-world applications and their impact across industries.
        \item \textbf{Challenges and Considerations:} Examine common challenges such as data quality and ethical considerations.
    \end{enumerate}
\end{frame}

\begin{frame}[fragile]
    \frametitle{Key Points and Examples}
    \begin{block}{Key Points to Emphasize}
        \begin{itemize}
            \item ML is integral to the development of intelligent systems.
            \item Contrasts traditional programming by learning from data rather than following explicit rules.
            \item Its relevance across various fields enhances its importance in today's technology-driven world.
        \end{itemize}
    \end{block}
    
    \begin{block}{Example of Machine Learning in Action}
        \textit{Spam Detection:} An email service classifies emails as "spam" or "not spam" based on historical data, adapting to new spam techniques over time.
    \end{block}
\end{frame}

\begin{frame}[fragile]
    \frametitle{What is Machine Learning? - Definition}
    \begin{block}{Definition of Machine Learning}
        Machine Learning (ML) is a subset of artificial intelligence (AI) that enables systems to learn from data, identify patterns, and make decisions or predictions without being explicitly programmed for every specific task.
    \end{block}
    \begin{itemize}
        \item ML systems learn from past experiences.
        \item Performance improves over time with more data.
        \item Focus shifts from coding specific tasks to training models on data.
    \end{itemize}
\end{frame}

\begin{frame}[fragile]
    \frametitle{What is Machine Learning? - Differences from Traditional Programming}
    \begin{block}{Machine Learning vs Traditional Programming}
        \begin{enumerate}
            \item \textbf{Traditional Programming:}
                \begin{itemize}
                    \item Human-defined rules dictate problem-solving.
                    \item Inputs and expected outputs are explicitly coded.
                    \item \textit{Example:} A calculator program with fixed operations.
                \end{itemize}
                
            \item \textbf{Machine Learning:}
                \begin{itemize}
                    \item Relies on data-driven learning and algorithms.
                    \item Flexibility to adapt to new, unseen data.
                    \item \textit{Example:} An email spam filter that improves as it analyzes new emails.
                \end{itemize}
        \end{enumerate}
    \end{block}
\end{frame}

\begin{frame}[fragile]
    \frametitle{What is Machine Learning? - Key Points and Illustrative Example}
    \begin{block}{Key Points to Emphasize}
        \begin{itemize}
            \item Focus on training models using data, contrasting with traditional logic-based programming.
            \item Automates decision-making processes, enabling prediction without human input.
            \item Capable of handling vast amounts of data, suitable for complex applications.
        \end{itemize}
    \end{block}
    
    \begin{block}{Illustrative Example}
        \textbf{Predicting House Prices:}
        \begin{itemize}
            \item \textbf{Traditional Approach:} Manually coded formulas based on fixed factors.
            \item \textbf{ML Approach:} Trained model learns relationships from historical data, improving with new data.
        \end{itemize}
    \end{block}
\end{frame}

\begin{frame}[fragile]
    \frametitle{Categories of Machine Learning}
    Machine Learning (ML) can be broadly classified into two main categories: 
    \begin{itemize}
        \item \textbf{Supervised Learning}
        \item \textbf{Unsupervised Learning}
    \end{itemize}
    Understanding these categories is essential for effectively applying machine learning to real-world problems.
\end{frame}

\begin{frame}[fragile]
    \frametitle{Supervised Learning}
    \begin{block}{Definition}
        In supervised learning, models are trained using a labeled dataset. Each training example is paired with an output label, and the model learns to map the input data to the correct output.
    \end{block}
    
    \begin{block}{How It Works}
        The model learns from examples, adjusting its parameters based on errors it makes during training. After sufficient training, it can predict outputs for new input data.
    \end{block}

    \begin{example}
        \textbf{Example:} 
        Email Classification, where a model is trained with labeled emails categorized as "spam" or "not spam."
    \end{example}

    \begin{itemize}
        \item Requires labeled data
        \item Involves classification or regression tasks
        \item Performance measured using accuracy, precision, and recall
    \end{itemize}
\end{frame}

\begin{frame}[fragile]
    \frametitle{Unsupervised Learning}
    \begin{block}{Definition}
        Unsupervised learning involves training on data without labeled responses. The goal is to discover inherent patterns or groupings within the data.
    \end{block}
    
    \begin{block}{How It Works}
        The model identifies structures, such as clusters or associations, without relying on labeled outputs.
    \end{block}

    \begin{example}
        \textbf{Example:} 
        Customer Segmentation, where customers are grouped based on purchasing behavior without predefined categories.
    \end{example}

    \begin{itemize}
        \item Does not require labeled data
        \item Involves clustering, association, or dimensionality reduction techniques
        \item Performance evaluated qualitatively or using measures such as silhouette score
    \end{itemize}
\end{frame}

\begin{frame}[fragile]
    \frametitle{Quick Comparison}
    \begin{center}
        \begin{tabular}{|c|c|c|}
            \hline
            \textbf{Feature} & \textbf{Supervised Learning} & \textbf{Unsupervised Learning} \\
            \hline
            Data Type & Labeled data & Unlabeled data \\
            \hline
            Goal & Predict outcomes & Discover patterns \\
            \hline
            Example Tasks & Classification, Regression & Clustering, Association \\
            \hline
        \end{tabular}
    \end{center}
\end{frame}

\begin{frame}[fragile]
    \frametitle{Summary}
    Understanding the difference between supervised and unsupervised learning is crucial for selecting the appropriate method for a given problem:
    \begin{itemize}
        \item \textbf{Supervised Learning:} Thrives on labeled datasets for informed predictions.
        \item \textbf{Unsupervised Learning:} Uncovers hidden structures within data.
    \end{itemize}
    This foundational knowledge will set the stage for diving deeper into each category in the following slides.
\end{frame}

\begin{frame}[fragile]
  \frametitle{Supervised Learning - Definition}
  \begin{block}{Definition}
    Supervised Learning is a type of machine learning where the model is trained on a dataset that includes both input features and corresponding output labels. During training, the algorithm learns to identify patterns and relationships in the data, allowing it to make predictions on new, unseen data.
  \end{block}
\end{frame}

\begin{frame}[fragile]
  \frametitle{Supervised Learning - Key Concepts}
  \begin{itemize}
    \item \textbf{Labeled Data:} The dataset must have labels indicating the correct output for each input instance.
    \item \textbf{Training Phase:} The algorithm adjusts its parameters by minimizing the difference between its predictions and the actual labels. This often uses backpropagation in neural networks.
    \item \textbf{Prediction Phase:} After training, the model predicts the output for new data points, aiming for generalization to make accurate predictions on unseen data.
  \end{itemize}
\end{frame}

\begin{frame}[fragile]
  \frametitle{Supervised Learning - Examples and Code}
  \begin{enumerate}
    \item \textbf{Classification:} Predicting discrete labels (e.g., identifying handwritten digits).
    \item \textbf{Regression:} Predicting continuous values (e.g., predicting house prices).
  \end{enumerate}
  
  \textbf{Simple Code Snippet (Python example using Scikit-learn):}
  \begin{lstlisting}[language=Python]
from sklearn.model_selection import train_test_split
from sklearn.linear_model import LinearRegression
import pandas as pd

# Load dataset
data = pd.read_csv('housing_data.csv')  # Data includes features and price label

# Preparing data
X = data.drop('price', axis=1)
y = data['price']

# Splitting data
X_train, X_test, y_train, y_test = train_test_split(X, y, test_size=0.2, random_state=42)

# Training the model
model = LinearRegression()
model.fit(X_train, y_train)

# Prediction
predictions = model.predict(X_test)
  \end{lstlisting}
\end{frame}

\begin{frame}[fragile]
  \frametitle{Supervised Learning - Summary}
  \begin{block}{Summary}
    Supervised learning empowers models to make informed predictions by leveraging labeled datasets. Mastering this method is essential for various applications across industries, including finance, healthcare, and marketing.
  \end{block}

  \textbf{Key Points to Emphasize:}
  \begin{itemize}
    \item Supervised Learning requires labeled data.
    \item Common algorithms include Linear Regression, Decision Trees, Support Vector Machines, and Neural Networks.
  \end{itemize}
\end{frame}

\begin{frame}[fragile]
    \frametitle{Applications of Supervised Learning - Introduction}
    Supervised learning is a crucial aspect of machine learning where algorithms learn from labeled data. 
    The goal is to make predictions or classify data based on learned patterns. 
    There are two primary types of tasks in supervised learning: 
    classification and regression.
\end{frame}

\begin{frame}[fragile]
    \frametitle{Applications of Supervised Learning - Classification Tasks}
    \textbf{1. Classification Tasks} 
    \begin{itemize}
        \item \textbf{Email Spam Detection:} 
        \begin{itemize}
            \item Classify emails as "spam" or "not spam" using features like content, subject line, and sender information.
            \item Labeled training set helps the model learn spam versus legitimate patterns.
        \end{itemize}
        
        \item \textbf{Medical Diagnosis:}
        \begin{itemize}
            \item Determine whether a patient has a disease based on diagnostic features (e.g., symptoms).
            \item Example: Mammograms labeled as "normal" or "malignant."
        \end{itemize}

        \item \textbf{Image Recognition:}
        \begin{itemize}
            \item Identify objects within images (e.g., recognizing faces).
            \item Classifying images into categories like "cats," "dogs," or "birds."
        \end{itemize}
    \end{itemize}
    
    \textbf{Key Points:}
    \begin{itemize}
        \item Classification categorizes data into discrete classes.
        \item Performance metrics include accuracy, precision, recall, and F1 score.
    \end{itemize}
\end{frame}

\begin{frame}[fragile]
    \frametitle{Applications of Supervised Learning - Regression Tasks}
    \textbf{2. Regression Tasks} 
    \begin{itemize}
        \item \textbf{House Price Prediction:}
        \begin{itemize}
            \item Estimate prices based on size, location, bedrooms, and age.
            \item Example Formula: 
            \begin{equation}
                \text{Price} = b_0 + b_1(\text{Size}) + b_2(\text{Location}) + b_3(\text{Bedrooms}) + b_4(\text{Age}) + \epsilon
            \end{equation}
        \end{itemize}

        \item \textbf{Stock Price Prediction:}
        \begin{itemize}
            \item Forecast future stock prices from historical data.
            \item Utilize time series regression to predict based on past performance.
        \end{itemize}

        \item \textbf{Sales Forecasting:}
        \begin{itemize}
            \item Predict future sales from historical data and trends.
            \item Helps businesses manage inventory and optimize marketing.
        \end{itemize}
    \end{itemize}
    
    \textbf{Key Points:}
    \begin{itemize}
        \item Regression estimates relationships among variables and predicts numerical values.
        \item Evaluated using metrics like mean squared error (MSE) and R-squared.
    \end{itemize}
\end{frame}

\begin{frame}[fragile]
    \frametitle{Applications of Supervised Learning - Conclusion}
    Supervised learning has a diverse range of real-world applications in both classification and regression tasks. 
    Understanding these tasks and their applications helps appreciate how machine learning impacts daily life and various industries. 
    Next, we will explore the types of algorithms used in supervised learning to effectively implement these tasks.
\end{frame}

\begin{frame}[fragile]
    \frametitle{Types of Supervised Learning Algorithms - Overview}
    Supervised learning is a type of machine learning where the model is trained on labeled data, allowing it to make predictions based on input features. In this slide, we will explore three common types of supervised learning algorithms:
    \begin{itemize}
        \item \textbf{Linear Regression}
        \item \textbf{Decision Trees}
        \item \textbf{Support Vector Machines (SVM)}
    \end{itemize}
\end{frame}

\begin{frame}[fragile]
    \frametitle{Types of Supervised Learning Algorithms - 1. Linear Regression}
    \begin{block}{Description}
        A statistical method used to model the relationship between a dependent variable (output) and one or more independent variables (inputs).
    \end{block}

    \begin{block}{Formula}
        \begin{equation}
            Y = \beta_0 + \beta_1 X_1 + \beta_2 X_2 + ... + \beta_n X_n + \epsilon
        \end{equation}
        where:
        \begin{itemize}
            \item $Y$ = Dependent variable
            \item $X_i$ = Independent variables
            \item $\beta_i$ = Coefficients
            \item $\epsilon$ = Error term
        \end{itemize}
    \end{block}

    \begin{block}{Example}
        Predicting housing prices based on features like size, number of bedrooms, and location.
    \end{block}
    
    \begin{itemize}
        \item Simple and easy to interpret.
        \item Assumes a linear relationship between variables.
    \end{itemize}
\end{frame}

\begin{frame}[fragile]
    \frametitle{Types of Supervised Learning Algorithms - 2. Decision Trees}
    \begin{block}{Description}
        A flowchart-like structure used to make decisions by splitting data into subsets based on the value of input features.
    \end{block}

    \begin{block}{Process}
        At each node, a decision is made based on the feature that best separates the data (using measures like Gini impurity or information gain).
    \end{block}

    \begin{block}{Example}
        Classifying whether an email is Spam or Not Spam based on features like the presence of certain keywords.
    \end{block}

    \begin{itemize}
        \item Visual and intuitive.
        \item Can handle both numerical and categorical data.
        \item Prone to overfitting if not managed (e.g., through pruning).
    \end{itemize}
\end{frame}

\begin{frame}[fragile]
    \frametitle{Types of Supervised Learning Algorithms - 3. Support Vector Machines (SVM)}
    \begin{block}{Description}
        A powerful classification technique that finds the hyperplane which best separates classes in high-dimensional space.
    \end{block}

    \begin{block}{Concept}
        The goal is to maximize the margin between the closest points of different classes (support vectors).
    \end{block}

    \begin{block}{Example}
        Classifying images of cats and dogs based on features extracted from pixel values.
    \end{block}

    \begin{itemize}
        \item Effective in high-dimensional spaces and with clear margins between classes.
        \item Can use kernels to handle non-linear relationships.
    \end{itemize}
\end{frame}

\begin{frame}[fragile]
    \frametitle{Types of Supervised Learning Algorithms - Summary}
    \begin{itemize}
        \item \textbf{Linear Regression}: Best for predicting continuous outcomes with a linear relation.
        \item \textbf{Decision Trees}: Useful for both classification and regression, featuring simple decision-making.
        \item \textbf{Support Vector Machines}: Ideal for complex classifications, utilizing high-dimensional spaces.
    \end{itemize}
    
    \begin{block}{Next Steps}
        In our next slide, we will delve into \textbf{Unsupervised Learning}, where models are trained on unlabeled data to discover hidden patterns.
    \end{block}
\end{frame}

\begin{frame}[fragile]
  \frametitle{Unsupervised Learning - Overview}
  \begin{block}{Definition}
    Unsupervised learning is a type of machine learning where the model learns patterns from data without labeled responses. Unlike supervised learning, unsupervised learning deals solely with input data, aiming to explore the underlying structure of the data.
  \end{block}
\end{frame}

\begin{frame}[fragile]
  \frametitle{Key Concepts of Unsupervised Learning}
  \begin{itemize}
    \item \textbf{Unlabeled Data}: Data that has not been tagged with labels. For example, a dataset of images of animals without identification of species.
    \item \textbf{Discovering Patterns}: The model identifies inherent structures within the data, leading to insights based on characteristics without predefined categories.
  \end{itemize}
\end{frame}

\begin{frame}[fragile]
  \frametitle{Examples of Unsupervised Learning}
  \begin{enumerate}
    \item \textbf{Clustering}:
      \begin{itemize}
        \item Grouping objects such that items in the same group are more similar.
        \item \textit{Example}: Clustering customers based on purchasing behavior into groups like "budget-conscious" and "premium shoppers."
      \end{itemize}
      
    \item \textbf{Anomaly Detection}:
      \begin{itemize}
        \item Identifying unusual data points, useful in applications like fraud detection.
      \end{itemize}
      
    \item \textbf{Dimensionality Reduction}:
      \begin{itemize}
        \item Reducing the number of variables while retaining essential features (e.g., PCA).
        \item \textit{Example}: Simplifying a dataset of handwritten digits while preserving critical attributes for classification.
      \end{itemize}
  \end{enumerate}
\end{frame}

\begin{frame}[fragile]
  \frametitle{Key Points to Emphasize}
  \begin{itemize}
    \item Unsupervised learning is vital when labeled data is scarce or expensive.
    \item It serves as a powerful exploratory data analysis tool, helping to understand data structure and relationships.
    \item Common algorithms include K-means, Hierarchical clustering, DBSCAN for clustering, and PCA for dimensionality reduction.
  \end{itemize}
\end{frame}

\begin{frame}[fragile]
  \frametitle{Example Algorithms}
  \begin{block}{K-means Clustering Algorithm}
    \begin{enumerate}
      \item Choose the number of clusters (k).
      \item Randomly assign data points to k clusters.
      \item Calculate the centroid for each cluster.
      \item Reassign points to the closest centroid.
      \item Repeat until assignments stabilize.
    \end{enumerate}
  \end{block}
  
  \begin{block}{PCA (Principal Component Analysis)}
     \begin{itemize}
       \item \textbf{Formula}: Projects data onto a lower-dimensional space that captures the most variance.
       \item Steps:
         \begin{enumerate}
           \item Standardization: Center the data.
           \item Compute the covariance matrix.
           \item Calculate eigenvalues and eigenvectors.
           \item Sort eigenvectors by eigenvalues for principal components.
         \end{enumerate}
     \end{itemize}
  \end{block}
\end{frame}

\begin{frame}[fragile]
  \frametitle{Conclusion}
  Unsupervised learning provides a robust framework for analyzing and interpreting data without pre-existing labels. Its applications range from customer segmentation to noise reduction. Understanding these techniques is vital for handling complex datasets and deriving actionable insights for decision-making.
\end{frame}

\begin{frame}[fragile]
  \frametitle{Applications of Unsupervised Learning - Overview}
  Unsupervised learning is a vital machine learning technique used to analyze and interpret unlabeled data. By discovering patterns and structures within the data, we can derive meaningful insights. Below, we will explore two primary categories of unsupervised learning: clustering and association tasks, along with their real-world applications.
\end{frame}

\begin{frame}[fragile]
  \frametitle{Applications of Unsupervised Learning - Clustering}
  \begin{block}{Clustering}
    Clustering refers to the task of grouping a set of objects in such a way that objects in the same group (or cluster) are more similar to each other than to those in other groups. This similarity can be based on various attributes such as distance metrics.
  \end{block}
  
  \textbf{Example Applications:}
  \begin{itemize}
    \item \textbf{Customer Segmentation:} Businesses use clustering to segment customers based on purchasing behavior, demographics, or preferences, allowing for targeted marketing strategies.
    \item \textbf{Image Segmentation:} Clustering algorithms like k-means can segment an image into regions, aiding in object recognition or scene analysis.
  \end{itemize}
\end{frame}

\begin{frame}[fragile]
  \frametitle{Applications of Unsupervised Learning - Association}
  \begin{block}{Association}
    Association involves discovering interesting relationships between variables in large datasets. This is often utilized for market basket analysis to identify sets of products frequently purchased together.
  \end{block}
  
  \textbf{Example Applications:}
  \begin{itemize}
    \item \textbf{Market Basket Analysis:} Retailers analyze transaction data to find associations between products, informing product placement and promotional strategies.
    \item \textbf{Recommendation Systems:} Many online platforms use association rules to recommend products based on user behavior patterns.
  \end{itemize}
\end{frame}

\begin{frame}[fragile]
  \frametitle{Key Points and Common Algorithms}
  \textbf{Key Points to Remember:}
  \begin{itemize}
    \item Unsupervised Learning works with unlabeled data to find hidden patterns.
    \item Clustering groups similar items, beneficial for customer behavior understanding and image analysis.
    \item Association discovers relations between variables, crucial for market strategies and product recommendations.
  \end{itemize}
  
  \textbf{Common Algorithms:}
  \begin{itemize}
    \item \textbf{K-Means Clustering:} Simple and efficient for clustering tasks.
    \item \textbf{Apriori Algorithm:} Frequently used for mining association rules.
  \end{itemize}
  
  \textbf{Closure:} Understanding unsupervised learning and its applications allows businesses and researchers to extract valuable insights from complex datasets, instrumental in making data-driven decisions.
\end{frame}

\begin{frame}[fragile]
    \frametitle{Types of Unsupervised Learning Algorithms}
    \begin{block}{Overview}
        Unsupervised learning finds hidden patterns or intrinsic structures in data without labeled outputs. 
        Common algorithms include:
        \begin{itemize}
            \item K-means Clustering
            \item Hierarchical Clustering
            \item Principal Component Analysis (PCA)
        \end{itemize}
    \end{block}
\end{frame}

\begin{frame}[fragile]
    \frametitle{1. K-Means Clustering}
    \begin{block}{Concept}
        Partitions data into K distinct groups (clusters) based on feature similarity.
    \end{block}
    \begin{block}{Process}
        \begin{enumerate}
            \item Choose the number of clusters (K).
            \item Randomly initialize K centroids.
            \item Assign each point to the nearest centroid.
            \item Update centroids as the mean of the assigned points.
            \item Repeat until centroids stabilize.
        \end{enumerate}
    \end{block}
    \begin{block}{Example}
        Grouping customer data into segments based on purchasing behavior for targeted marketing.
    \end{block}
\end{frame}

\begin{frame}[fragile]
    \frametitle{1. K-Means Clustering (cont.)}
    \begin{block}{Key Points}
        \begin{itemize}
            \item Sensitive to initial centroid placement.
            \item Requires number of clusters (K) to be predefined.
            \item Works well with spherical-shaped clusters.
        \end{itemize}
    \end{block}
    \begin{block}{Formula}
        Minimize the objective function:
        \begin{equation}
        J = \sum_{i=1}^{K} \sum_{x_j \in C_i} \| x_j - \mu_i \|^2
        \end{equation}
        where \( C_i \) is the i-th cluster and \( \mu_i \) is the mean.
    \end{block}
\end{frame}

\begin{frame}[fragile]
    \frametitle{2. Hierarchical Clustering}
    \begin{block}{Concept}
        Creates a tree of clusters; can be agglomerative (bottom-up) or divisive (top-down).
    \end{block}
    \begin{block}{Process}
        \begin{itemize}
            \item \textbf{Agglomerative}: Start with each point as its own cluster and merge iteratively.
            \item \textbf{Divisive}: Start with one cluster and split it iteratively.
        \end{itemize}
    \end{block}
    \begin{block}{Example}
        Organizing organisms in biological taxonomy based on genetic similarities using dendrograms.
    \end{block}
\end{frame}

\begin{frame}[fragile]
    \frametitle{2. Hierarchical Clustering (cont.)}
    \begin{block}{Key Points}
        \begin{itemize}
            \item No need to specify the number of clusters in advance.
            \item Produces a hierarchy for varying numbers of clusters.
            \item Computationally intensive for large datasets.
        \end{itemize}
    \end{block}
    \begin{block}{Distance Metrics}
        \begin{itemize}
            \item Euclidean, Manhattan, Cosine, etc.
        \end{itemize}
    \end{block}
\end{frame}

\begin{frame}[fragile]
    \frametitle{3. Principal Component Analysis (PCA)}
    \begin{block}{Concept}
        Reduces dimensionality while preserving variance; identifies directions (principal components) of maximum variance.
    \end{block}
    \begin{block}{Process}
        \begin{enumerate}
            \item Standardize the data.
            \item Compute the covariance matrix.
            \item Find eigenvalues and eigenvectors of the covariance matrix.
            \item Select top K eigenvectors with largest eigenvalues.
            \item Transform data into the new reduced space.
        \end{enumerate}
    \end{block}
\end{frame}

\begin{frame}[fragile]
    \frametitle{3. Principal Component Analysis (PCA) (cont.)}
    \begin{block}{Example}
        Reducing features in an image dataset from thousands of pixels to a smaller number of features.
    \end{block}
    \begin{block}{Key Points}
        \begin{itemize}
            \item Useful for noise reduction and data visualization.
            \item Sensitive to the scale of the data.
            \item Helps in avoiding the "curse of dimensionality."
        \end{itemize}
    \end{block}
    \begin{block}{Formula}
        Project data matrix \( X \) onto the first principal component:
        \begin{equation}
        Z = X \cdot w
        \end{equation}
        where \( w \) is the eigenvector associated with the largest eigenvalue.
    \end{block}
\end{frame}

\begin{frame}[fragile]
    \frametitle{Summary}
    Unsupervised learning algorithms are crucial for discovering patterns in unlabeled data. 
    \begin{itemize}
        \item **K-means** is effective for clustering.
        \item **Hierarchical** methods provide a comprehensive view of data relationships.
        \item **PCA** optimizes data representation through dimensionality reduction.
    \end{itemize}
    Understanding these methods enhances our capacity to analyze complex datasets.
\end{frame}

\begin{frame}[fragile]
    \frametitle{Overview of Supervised vs. Unsupervised Learning}
    Machine Learning (ML) is broadly categorized into two types: 
    \textbf{Supervised Learning} and \textbf{Unsupervised Learning}. Both approaches are essential for analyzing data, but they serve different purposes and have distinct requirements.
\end{frame}

\begin{frame}[fragile]
    \frametitle{1. Definition}
    \begin{itemize}
        \item \textbf{Supervised Learning}: 
        \begin{itemize}
            \item Model trained on a labeled dataset with input-output pairs.
            \item Goal: Learn mapping from inputs to outputs for predictions on unseen data.
        \end{itemize}
        
        \item \textbf{Unsupervised Learning}: 
        \begin{itemize}
            \item Model trained on unlabeled data, identifying patterns or groupings without explicit outputs.
            \item Goal: Exploratory data analysis.
        \end{itemize}
    \end{itemize}
\end{frame}

\begin{frame}[fragile]
    \frametitle{2. Data Requirements}
    \begin{itemize}
        \item \textbf{Supervised Learning}:
        \begin{itemize}
            \item Requires a comprehensive labeled dataset; can be time-consuming and costly.
            \item \textbf{Examples}: Email classification (spam vs. non-spam), house price prediction.
        \end{itemize}
        
        \item \textbf{Unsupervised Learning}:
        \begin{itemize}
            \item Does not require labeled data; works with raw data.
            \item \textbf{Examples}: Market basket analysis, customer segmentation.
        \end{itemize}
    \end{itemize}
\end{frame}

\begin{frame}[fragile]
    \frametitle{3. Model Outcomes}
    \begin{itemize}
        \item \textbf{Supervised Learning}:
        \begin{itemize}
            \item Produces predictive models for making decisions based on new data.
            \item \textbf{Outcomes}: Class labels, continuous values (regression).
        \end{itemize}
        
        \item \textbf{Unsupervised Learning}:
        \begin{itemize}
            \item Provides insights and structures from data, revealing underlying relationships.
            \item \textbf{Outcomes}: Clusters, patterns, associations.
        \end{itemize}
    \end{itemize}
\end{frame}

\begin{frame}[fragile]
    \frametitle{4. Key Points to Emphasize}
    \begin{itemize}
        \item \textbf{Training Process}:
        \begin{itemize}
            \item Supervised Learning: Adjusts model based on error feedback from known outcomes.
            \item Unsupervised Learning: Discovers structure without feedback on correctness.
        \end{itemize}
        
        \item \textbf{Common Use Cases}:
        \begin{itemize}
            \item Supervised: Predictive analytics in finance or healthcare.
            \item Unsupervised: Customer segmentation in marketing, anomaly detection in cybersecurity.
        \end{itemize}
    \end{itemize}
\end{frame}

\begin{frame}[fragile]
    \frametitle{5. Summary Comparison Table}
    \begin{center}
    \begin{tabular}{|l|l|l|}
    \hline
    \textbf{Feature} & \textbf{Supervised Learning} & \textbf{Unsupervised Learning} \\ \hline
    Data Type & Labeled data & Unlabeled data \\ \hline
    Goal & Predict outcomes from input data & Discover patterns in data \\ \hline
    Examples & Classification and regression tasks & Clustering and association tasks \\ \hline
    Feedback Mechanism & Yes (learns from labeled examples) & No (self-organizes based on data structure) \\ \hline
    \end{tabular}
    \end{center}
\end{frame}

\begin{frame}[fragile]
    \frametitle{Conclusion}
    Understanding the differences between supervised and unsupervised learning is crucial for selecting appropriate methods for data analysis and model development. Each has unique strengths and applications in the field of machine learning.
\end{frame}

\begin{frame}[fragile]
    \frametitle{Suggested Further Reading}
    \begin{itemize}
        \item "Pattern Recognition and Machine Learning" by Christopher Bishop
        \item "Deep Learning" by Ian Goodfellow, Yoshua Bengio, and Aaron Courville
    \end{itemize}
    Feel free to explore specific case studies for real-world applications of both learning types!
\end{frame}

\begin{frame}[fragile]
    \frametitle{Key Metrics for Evaluating Models}
    \begin{block}{Introduction}
        Evaluating the performance of machine learning models is crucial to understanding their effectiveness. 
        Different metrics are used for supervised and unsupervised learning models, catering to their distinct objectives.
    \end{block}

    This slide presents key metrics such as \textbf{Accuracy}, \textbf{Precision}, \textbf{Recall}, and the \textbf{Silhouette Score}.
\end{frame}

\begin{frame}[fragile]
    \frametitle{1. Supervised Learning Metrics}
    
    \begin{itemize}
        \item \textbf{Accuracy}
            \begin{itemize}
                \item \textbf{Definition}: The proportion of true results among the total number of cases examined.
                \item \textbf{Formula}:
                \begin{equation}
                \text{Accuracy} = \frac{TP + TN}{TP + TN + FP + FN}
                \end{equation}
                \item \textbf{Example}:
                \begin{equation}
                \text{Accuracy} = \frac{70 + 20}{70 + 20 + 5 + 5} = 0.90 \text{ or } 90\%
                \end{equation}
            \end{itemize}
        
        \item \textbf{Precision}
            \begin{itemize}
                \item \textbf{Definition}: The ratio of correctly predicted positive observations to the total predicted positives.
                \item \textbf{Formula}:
                \begin{equation}
                \text{Precision} = \frac{TP}{TP + FP}
                \end{equation}
                \item \textbf{Example}:
                \begin{equation}
                \text{Precision} = \frac{70}{70 + 5} \approx 0.93 \text{ or } 93\%
                \end{equation}
            \end{itemize}
            
        \item \textbf{Recall (Sensitivity)}
            \begin{itemize}
                \item \textbf{Definition}: The ratio of correctly predicted positive observations to all actual positives.
                \item \textbf{Formula}:
                \begin{equation}
                \text{Recall} = \frac{TP}{TP + FN}
                \end{equation}
                \item \textbf{Example}:
                \begin{equation}
                \text{Recall} = \frac{70}{70 + 5} \approx 0.93  \text{ or } 93\%
                \end{equation}
            \end{itemize}
            
        \item \textbf{F1 Score}
            \begin{itemize}
                \item \textbf{Definition}: The harmonic mean of Precision and Recall.
                \item \textbf{Formula}:
                \begin{equation}
                F1 = 2 \times \frac{\text{Precision} \times \text{Recall}}{\text{Precision} + \text{Recall}}
                \end{equation}
                \item \textbf{Example}:
                \begin{equation}
                F1 \approx 0.93
                \end{equation}
            \end{itemize}
    \end{itemize}
\end{frame}

\begin{frame}[fragile]
    \frametitle{2. Unsupervised Learning Metrics}
    
    \begin{itemize}
        \item \textbf{Silhouette Score}
            \begin{itemize}
                \item \textbf{Definition}: Measures how similar an object is to its own cluster compared to other clusters.
                \item \textbf{Range}: Scales from -1 to 1 
                    \begin{itemize}
                        \item \textbf{1}: Highly dense clustering
                        \item \textbf{0}: Overlapping clusters
                        \item \textbf{-1}: Misclassified data points
                    \end{itemize}
                \item \textbf{Formula}:
                \begin{equation}
                s = \frac{b - a}{\max(a, b)}
                \end{equation}
                Where:
                \begin{itemize}
                    \item \( a \) = average distance between a sample and all other points in the same cluster 
                    \item \( b \) = average distance to the nearest cluster
                \end{itemize}
                
                \item \textbf{Example}:
                \begin{equation}
                s = \frac{0.5 - 0.3}{\max(0.3, 0.5)} = 0.4
                \end{equation}
            \end{itemize}
    \end{itemize}
    
    \begin{block}{Key Points to Emphasize}
        - Different metrics are crucial for different models (supervised vs. unsupervised).
        - Accuracy can be misleading for imbalanced datasets; precision, recall, and F1 score are often more informative.
        - The Silhouette Score provides insight into clustering quality.
    \end{block}
    
    \begin{block}{Conclusion}
        Understanding these metrics will help in the evaluation and improvement of machine learning models.
    \end{block}
\end{frame}

\begin{frame}[fragile]
    \frametitle{Challenges in Machine Learning - Overview}
    \begin{itemize}
        \item Machine learning (ML) is powerful for making predictions and uncovering patterns.
        \item Key challenges include:
        \begin{itemize}
            \item Overfitting
            \item Underfitting
            \item Data Quality Issues
        \end{itemize}
    \end{itemize}
\end{frame}

\begin{frame}[fragile]
    \frametitle{Challenges in Machine Learning - Overfitting}
    \begin{block}{Definition}
        Overfitting occurs when a model learns noise in the training data, negatively impacting performance on new data.
    \end{block}
    \begin{itemize}
        \item Symptoms:
        \begin{itemize}
            \item High accuracy on training data
            \item Poor performance on validation/test data
        \end{itemize}
        \item Example:
        \begin{itemize}
            \item A model predicts housing prices by memorizing the training set, leading to struggles with new properties.
        \end{itemize}
    \end{itemize}
    \begin{block}{Key Point}
        Mitigate overfitting by:
        \begin{itemize}
            \item Using simpler models
            \item Applying regularization techniques (L1, L2)
            \item Employing cross-validation
        \end{itemize}
    \end{block}
\end{frame}

\begin{frame}[fragile]
    \frametitle{Challenges in Machine Learning - Underfitting & Data Quality Issues}
    \begin{block}{Underfitting}
        \begin{itemize}
            \item Definition: A model too simple to capture the underlying structure of the data.
            \item Symptoms: Low training and validation accuracy.
            \item Example: Using a linear model for non-linear relationships, leading to poor predictions.
        \end{itemize}
        \begin{block}{Key Point}
            Address underfitting by:
            \begin{itemize}
                \item Increasing model complexity
                \item Adding more pertinent features
            \end{itemize}
        \end{block}
    \end{block}
    \begin{block}{Data Quality Issues}
        \begin{itemize}
            \item Definition: Poor quality data (inaccurate, incomplete, inconsistent) affects model performance.
            \item Common issues:
            \begin{itemize}
                \item Missing Values
                \item Noisy Data
                \item Imbalance
            \end{itemize}
            \item Example: Insufficient examples of a rare disease in medical datasets causes poor identification.
        \end{itemize}
    \end{block}
    \begin{block}{Key Point}
        Ensure high-quality data through:
        \begin{itemize}
            \item Data cleaning
            \item Feature engineering
            \item Balancing datasets
        \end{itemize}
    \end{block}
\end{frame}

\begin{frame}[fragile]
    \frametitle{Challenges in Machine Learning - Summary}
    Understanding and addressing the challenges of ML is crucial:
    \begin{itemize}
        \item Balance model complexity and data quality.
        \item Continuous evaluation of models improves predictive performance.
    \end{itemize}
    \begin{block}{Visual Aid}
        Consider adding a diagram comparing overfitting and underfitting (model accuracy vs. data points) for better visualization.
    \end{block}
\end{frame}

\begin{frame}[fragile]
    \frametitle{Ethical Implications in Machine Learning - Introduction}
    \begin{block}{Overview}
        As machine learning (ML) technology rapidly advances, it brings forth a range of ethical considerations that must be addressed to ensure responsible use. This slide provides an overview of three key ethical implications: 
    \end{block}
    \begin{itemize}
        \item Bias
        \item Privacy concerns
        \item Accountability
    \end{itemize}
\end{frame}

\begin{frame}[fragile]
    \frametitle{Ethical Implications in Machine Learning - Bias}
    \begin{block}{Definition}
        Bias in machine learning refers to systematic errors that result from prejudiced data or algorithms. Such bias can lead to unfair treatment of individuals or groups.
    \end{block}
    \begin{block}{Example}
        An AI hiring tool trained on historical hiring data may favor male candidates over equally qualified female candidates if past company data reflects this bias.
    \end{block}
    \begin{block}{Key Point}
        It's crucial to audit datasets and algorithm outcomes to identify and mitigate bias, ensuring fair representation.
    \end{block}
\end{frame}

\begin{frame}[fragile]
    \frametitle{Ethical Implications in Machine Learning - Privacy Concerns}
    \begin{block}{Definition}
        Privacy concerns arise from the data collection practices of machine learning systems, especially involving sensitive information.
    \end{block}
    \begin{block}{Example}
        Facial recognition technologies can be used to track individuals without their consent, raising significant privacy issues.
    \end{block}
    \begin{block}{Illustration}
        Consider the implications of using social media data. While ML algorithms can analyze user behavior for marketing, they must respect users’ privacy rights and consent.
    \end{block}
    \begin{block}{Key Point}
        Organizations should prioritize data anonymization and secure handling practices to protect user privacy.
    \end{block}
\end{frame}

\begin{frame}[fragile]
    \frametitle{Ethical Implications in Machine Learning - Accountability}
    \begin{block}{Definition}
        Accountability pertains to the responsibility of developers and organizations for the outcomes produced by their machine learning systems.
    \end{block}
    \begin{block}{Example}
        If an autonomous vehicle causes an accident, determining liability (e.g., software developer, vehicle manufacturer, or owner) can be legally complex.
    \end{block}
    \begin{block}{Key Point}
        Clear governance structures and ethical guidelines should be established to hold entities accountable for the consequences of their ML systems.
    \end{block}
\end{frame}

\begin{frame}[fragile]
    \frametitle{Ethical Implications in Machine Learning - Summary and Call to Action}
    \begin{block}{Summary}
        Ethical implications in ML, including bias, privacy concerns, and accountability, are critical elements that require careful consideration and proactive measures.
    \end{block}
    \begin{block}{Call to Action}
        As future professionals in the field, engage in discussions about ethical standards and advocate for responsible AI practices within your organizations.
    \end{block}
\end{frame}

\begin{frame}[fragile]
    \frametitle{Future Trends in Machine Learning}
    \begin{block}{Overview of Emerging Trends}
        As the field of machine learning (ML) continues to evolve, several key trends are emerging. This presentation explores three of the most prominent: 
        \begin{itemize}
            \item Deep Learning
            \item Transfer Learning
            \item Reinforcement Learning
        \end{itemize}
    \end{block}
\end{frame}

\begin{frame}[fragile]
    \frametitle{1. Deep Learning}
    \begin{block}{Definition}
        Deep Learning is a subset of ML that utilizes neural networks with multiple layers (deep architectures) to model complex patterns in large datasets.
    \end{block}
    \begin{itemize}
        \item \textbf{Key Characteristics:}
        \begin{itemize}
            \item Data-Driven: Requires large amounts of data.
            \item Automatic Feature Extraction: Reduces manual feature engineering.
        \end{itemize}
        
        \item \textbf{Applications:}
        \begin{itemize}
            \item Image and speech recognition (e.g., Google Photos).
            \item Natural language processing (e.g., chatbots like GPT).
        \end{itemize}

        \item \textbf{Example:} 
        Convolutional Neural Networks (CNNs) identify objects by learning hierarchical features from image data.
    \end{itemize}
\end{frame}

\begin{frame}[fragile]
    \frametitle{2. Transfer Learning}
    \begin{block}{Definition}
        Transfer Learning is the process of adapting a pre-trained model for a new task.
    \end{block}
    \begin{itemize}
        \item \textbf{Key Characteristics:}
        \begin{itemize}
            \item Efficiency: Reduces data and training time.
            \item Leveraging Knowledge: Boosts performance using learned representations.
        \end{itemize}
        
        \item \textbf{Applications:}
        \begin{itemize}
            \item Adapting language models for specialized terminology.
            \item Fine-tuning image classifiers for specific objects.
        \end{itemize}

        \item \textbf{Example:} 
        Fine-tuning a model trained on ImageNet to classify medical images quickly.
    \end{itemize}
\end{frame}

\begin{frame}[fragile]
    \frametitle{3. Reinforcement Learning}
    \begin{block}{Definition}
        Reinforcement Learning (RL) is where an agent learns through trial and error to maximize cumulative rewards.
    \end{block}
    \begin{itemize}
        \item \textbf{Key Characteristics:}
        \begin{itemize}
            \item Trial and Error: Learns from rewards or penalties.
            \item Delayed Rewards: Immediate consequences are not always apparent.
        \end{itemize}
        
        \item \textbf{Applications:}
        \begin{itemize}
            \item Game playing (e.g., AlphaGo).
            \item Autonomous systems (e.g., self-driving cars).
        \end{itemize}

        \item \textbf{Example:} 
        An agent navigating a grid learns to reach a goal by receiving rewards for progress.
    \end{itemize}
\end{frame}

\begin{frame}[fragile]
    \frametitle{Key Points to Emphasize}
    \begin{itemize}
        \item \textbf{Interconnectedness:} These trends overlap; deep learning is often used in RL frameworks.
        \item \textbf{Impact on Industries:} Driving innovation in healthcare, finance, and robotics.
        \item \textbf{Ethical Considerations:} Rapid evolution raises important ethical questions.
    \end{itemize}
\end{frame}

\begin{frame}[fragile]
    \frametitle{Concluding Thoughts}
    Understanding these emerging trends is essential as they point to the future of machine learning. Keeping updated on these developments allows practitioners to create innovative solutions for complex problems.
\end{frame}

\begin{frame}[fragile]
    \frametitle{Conclusion and Key Takeaways}
    
    \begin{block}{Overview}
        Revisiting the foundational concepts of supervised and unsupervised learning is crucial for leveraging machine learning effectively in real-world applications.
    \end{block}
\end{frame}

\begin{frame}[fragile]
    \frametitle{Key Points: Supervised Learning}
    
    \begin{itemize}
        \item \textbf{Definition:} Learning from labeled training data to map inputs to correct outputs.
        \item \textbf{Examples:}
        \begin{itemize}
            \item \textbf{Classification:} Predicting categories (e.g., spam vs. non-spam emails).
            \item \textbf{Regression:} Predicting numerical values (e.g., housing prices based on features).
        \end{itemize}
        \item \textbf{Common Algorithms:} Linear Regression, Decision Trees, Support Vector Machines.
    \end{itemize}
\end{frame}

\begin{frame}[fragile]
    \frametitle{Key Points: Unsupervised Learning}
    
    \begin{itemize}
        \item \textbf{Definition:} Training a model on data without labeled responses, aiming to find patterns and relationships.
        \item \textbf{Examples:}
        \begin{itemize}
            \item \textbf{Clustering:} Grouping similar data points (e.g., customer segmentation).
            \item \textbf{Dimensionality Reduction:} Simplifying datasets (e.g., Principal Component Analysis).
        \end{itemize}
        \item \textbf{Common Algorithms:} K-Means Clustering, Hierarchical Clustering, t-SNE.
    \end{itemize}
\end{frame}

\begin{frame}[fragile]
    \frametitle{Applications and Importance}
    
    \begin{itemize}
        \item \textbf{Applications:} 
        \begin{itemize}
            \item Finance: Fraud detection
            \item Healthcare: Diagnostic algorithms
            \item Marketing: Recommendation systems
        \end{itemize}
        \item \textbf{Importance of Both Learning Types:} 
        \begin{itemize}
            \item Knowing when to use each can significantly influence project success.
            \item Supervised learning is suited for predictive tasks; unsupervised learning is vital for exploratory analysis.
        \end{itemize}
    \end{itemize}
\end{frame}

\begin{frame}[fragile]
    \frametitle{Illustrative Example}
    
    \begin{block}{Customer Behavior Analysis}
        \begin{itemize}
            \item \textbf{Using Supervised Learning:} Predicting purchase likelihood based on user data.
            \item \textbf{Using Unsupervised Learning:} Segmenting users by browsing patterns for targeted marketing.
        \end{itemize}
    \end{block}
\end{frame}

\begin{frame}[fragile]
    \frametitle{Final Thoughts and Next Steps}
    
    \begin{block}{Final Thoughts}
        Mastering both supervised and unsupervised learning establishes a strong foundation for future explorations in machine learning.
    \end{block}
    
    \begin{block}{Next Steps}
        Prepare for a discussion in our next session by considering the applications discussed and formulating questions regarding their implementation.
    \end{block}
\end{frame}

\begin{frame}[fragile]
    \frametitle{Formula Snippet for Supervised Learning}
    
    The relationship can often be modeled as:
    \begin{equation}
        y = mx + b
    \end{equation}
    where:
    \begin{itemize}
        \item $y$ = dependent variable (output)
        \item $m$ = slope of the line (coefficient)
        \item $x$ = independent variable (input)
        \item $b$ = y-intercept
    \end{itemize}
    
    This represents a simple linear regression model for predictive tasks.
\end{frame}

\begin{frame}[fragile]
    \frametitle{Questions and Discussion}
    \begin{block}{Overview}
        The field of Machine Learning (ML) is rapidly evolving. 
        This session is dedicated to addressing your questions and facilitating a discussion aimed at clarifying any uncertainties regarding ML and its applications.
    \end{block}
\end{frame}

\begin{frame}[fragile]
    \frametitle{Key Concepts to Reflect On}
    \begin{enumerate}
        \item \textbf{Supervised Learning}
        \begin{itemize}
            \item \textbf{Definition}: A type of ML where a model learns from labeled training data to make predictions or classify data.
            \item \textbf{Key Algorithms}: Linear Regression, Decision Trees, Support Vector Machines.
            \item \textbf{Example}: Predicting house prices based on features like size, location, and number of bedrooms.
        \end{itemize}
        
        \item \textbf{Unsupervised Learning}
        \begin{itemize}
            \item \textbf{Definition}: A type of ML used to find patterns or groupings in data without labeled outputs.
            \item \textbf{Key Algorithms}: K-means Clustering, Hierarchical Clustering, PCA.
            \item \textbf{Example}: Customer segmentation for targeted marketing approaches.
        \end{itemize}
        
        \item \textbf{Reinforcement Learning}
        \begin{itemize}
            \item \textbf{Definition}: A type of ML where an agent learns to make decisions by receiving rewards or penalties based on its actions within an environment.
            \item \textbf{Example}: Training autonomous vehicles to navigate through traffic safely.
        \end{itemize}
    \end{enumerate}
\end{frame}

\begin{frame}[fragile]
    \frametitle{Discussion Points and Interaction}
    \begin{block}{Discussion Points}
        \begin{itemize}
            \item How do different types of ML apply in real-world scenarios?
            \item Differences in data requirements for supervised vs. unsupervised learning.
            \item Challenges and ethical considerations in implementing ML models (like bias, data privacy).
            \item Common pitfalls in ML model development and techniques to mitigate errors.
        \end{itemize}
    \end{block}
    
    \begin{block}{Encouraging Interaction}
        \begin{itemize}
            \item \textbf{Open the Floor}: Please share any specific areas of confusion or topics you'd like to delve deeper into.
            \item \textbf{Examples from Practice}: Feel free to discuss any ML applications you've encountered.
        \end{itemize}
    \end{block}
\end{frame}


\end{document}