\documentclass[aspectratio=169]{beamer}

% Theme and Color Setup
\usetheme{Madrid}
\usecolortheme{whale}
\useinnertheme{rectangles}
\useoutertheme{miniframes}

% Additional Packages
\usepackage[utf8]{inputenc}
\usepackage[T1]{fontenc}
\usepackage{graphicx}
\usepackage{booktabs}
\usepackage{listings}
\usepackage{amsmath}
\usepackage{amssymb}
\usepackage{xcolor}
\usepackage{tikz}
\usepackage{pgfplots}
\pgfplotsset{compat=1.18}
\usetikzlibrary{positioning}
\usepackage{hyperref}

% Custom Colors
\definecolor{myblue}{RGB}{31, 73, 125}
\definecolor{mygray}{RGB}{100, 100, 100}
\definecolor{mygreen}{RGB}{0, 128, 0}
\definecolor{myorange}{RGB}{230, 126, 34}
\definecolor{mycodebackground}{RGB}{245, 245, 245}

% Set Theme Colors
\setbeamercolor{structure}{fg=myblue}
\setbeamercolor{frametitle}{fg=white, bg=myblue}
\setbeamercolor{title}{fg=myblue}
\setbeamercolor{section in toc}{fg=myblue}
\setbeamercolor{item projected}{fg=white, bg=myblue}
\setbeamercolor{block title}{bg=myblue!20, fg=myblue}
\setbeamercolor{block body}{bg=myblue!10}
\setbeamercolor{alerted text}{fg=myorange}

% Title Page Information
\title[Week 1: Introduction to AI]{Week 1: Introduction to Artificial Intelligence}
\author[J. Smith]{John Smith, Ph.D.}
\institute[University Name]{
  Department of Computer Science\\
  University Name\\
  \vspace{0.3cm}
  Email: email@university.edu\\
  Website: www.university.edu
}
\date{\today}

% Document Start
\begin{document}

\frame{\titlepage}

\begin{frame}[fragile]
    \frametitle{Introduction to Artificial Intelligence}
    \begin{block}{Overview of AI}
        Artificial Intelligence (AI) refers to the capability of machines to imitate intelligent human behavior, performing tasks that typically require human intelligence.
    \end{block}
\end{frame}

\begin{frame}[fragile]
    \frametitle{Significance of AI}
    \begin{itemize}
        \item \textbf{Transformational Impact:} Revolutionizing multiple industries, enhancing efficiency and effectiveness.
        \item \textbf{Data-Driven Decisions:} Facilitating decision-making through data analysis, enabling strategic planning.
    \end{itemize}
\end{frame}

\begin{frame}[fragile]
    \frametitle{Key Applications of AI}
    \begin{enumerate}
        \item \textbf{Healthcare:} AI algorithms predict patient outcomes, automate image analysis, and aid in drug discovery (e.g., IBM Watson Health).
        \item \textbf{Finance:} Fraud detection and high-frequency trading through analysis of transaction patterns.
        \item \textbf{Transportation:} Autonomous vehicles (e.g., Tesla) utilize AI for navigation and decision-making.
        \item \textbf{Customer Service:} Chatbots (e.g., Amazon’s Alexa) improve user experiences via natural language processing.
    \end{enumerate}
\end{frame}

\begin{frame}[fragile]
    \frametitle{Relevance in Today's World}
    \begin{itemize}
        \item \textbf{Job Transformation:} AI creates new opportunities while displacing jobs in some sectors.
        \item \textbf{Ethical Considerations:} Important ethical questions arise regarding privacy, bias, and future employment.
    \end{itemize}
\end{frame}

\begin{frame}[fragile]
    \frametitle{Key Points to Emphasize}
    \begin{itemize}
        \item \textbf{AI vs. Human Intelligence:} AI excels in specific tasks but lacks general reasoning and emotional understanding.
        \item \textbf{Continuous Learning:} Machine Learning (ML) allows algorithms to improve performance automatically through experience.
    \end{itemize}
\end{frame}

\begin{frame}[fragile]
    \frametitle{Recap of Key Terms}
    \begin{itemize}
        \item \textbf{Artificial Intelligence (AI):} Machines mimicking human intelligence.
        \item \textbf{Machine Learning (ML):} A field of AI focusing on how data and algorithms emulate human learning.
    \end{itemize}
\end{frame}

\begin{frame}[fragile]
    \frametitle{Conclusion}
    Understanding AI fundamentals is vital as it shapes technology's future. Exploring its applications shows how AI can be harnessed responsibly and effectively.
\end{frame}

\begin{frame}[fragile]
    \frametitle{Next Steps}
    This slide serves as a foundation for the upcoming discussion on the history of AI and the evolution of these concepts over time.
\end{frame}

\begin{frame}[fragile]
    \frametitle{History of AI - Introduction}
    \begin{block}{Overview}
        Artificial Intelligence (AI) has undergone dramatic evolution since its inception in the mid-20th century. 
        Understanding this history helps to contextualize its current applications and potential future advancements.
    \end{block}
\end{frame}

\begin{frame}[fragile]
    \frametitle{History of AI - Key Milestones}
    \begin{enumerate}
        \item \textbf{1950s - Foundations of AI}
        \begin{itemize}
            \item 1950: Alan Turing publishes "Computing Machinery and Intelligence," introducing the Turing Test.
            \item 1956: Dartmouth Conference marks the birth of AI as a formal field.
        \end{itemize}

        \item \textbf{1960s - Early Programs and Theories}
        \begin{itemize}
            \item 1966: ELIZA, a conversational program, is developed.
            \item 1969: Shakey, the first successful robot, is created.
        \end{itemize}

        \item \textbf{1970s - AI Winter}
        \begin{itemize}
            \item A reduction in funding and interest occurs due to unmet expectations.
        \end{itemize}

        \item \textbf{1980s - Revival with Expert Systems}
        \begin{itemize}
            \item 1980: Introduction of expert systems like MYCIN and XCON.
        \end{itemize}

        \item \textbf{1990s - Machine Learning and Data}
        \begin{itemize}
            \item 1997: IBM's Deep Blue defeats Garry Kasparov.
        \end{itemize}

        \item \textbf{2000s - Rise of the Internet and Big Data}
        \begin{itemize}
            \item The internet leads to enhanced machine learning techniques.
        \end{itemize}

        \item \textbf{2010s - Deep Learning Revolution}
        \begin{itemize}
            \item Introduction of deep learning models revolutionizing various applications.
            \item 2016: AlphaGo defeats a Go champion.
        \end{itemize}

        \item \textbf{2020s - Current and Future Trends}
        \begin{itemize}
            \item Advancements in natural language processing and ethical considerations in AI development.
        \end{itemize}
    \end{enumerate}
\end{frame}

\begin{frame}[fragile]
    \frametitle{History of AI - Key Takeaways}
    \begin{block}{Evolution of AI}
        - From theoretical foundations to practical applications, AI's journey has been marked by both triumphs and challenges.
    \end{block}

    \begin{block}{AI Winters}
        - Recognize the cycles of hype and disillusionment in AI's development.
    \end{block}

    \begin{block}{Current Impact}
        - Today, AI impacts various sectors such as healthcare, finance, education, and entertainment.
    \end{block}

    \begin{block}{Conclusion}
        - The history of AI not only highlights technological progress but also evolves societal implications and ethical considerations.
        - Prepare for a deeper exploration of AI definitions and applications in the next slide.
    \end{block}
\end{frame}

\begin{frame}[fragile]
    \frametitle{Defining Artificial Intelligence}
    
    \textbf{Learning Objectives:}
    \begin{itemize}
        \item Understand various definitions of Artificial Intelligence (AI).
        \item Explore how these definitions apply across different fields.
        \item Demonstrate the implications of these definitions using examples.
    \end{itemize}
    
\end{frame}

\begin{frame}[fragile]
    \frametitle{What is Artificial Intelligence (AI)?}
    
    Artificial Intelligence (AI) is a multidisciplinary field that focuses on creating systems capable of performing tasks that typically require human intelligence. These tasks include:
    \begin{itemize}
        \item Reasoning
        \item Learning
        \item Problem-solving
        \item Understanding natural language
        \item Perceiving the environment
    \end{itemize}
    
\end{frame}

\begin{frame}[fragile]
    \frametitle{Definitions Across Different Fields}
    
    \begin{enumerate}
        \item \textbf{Computer Science Perspective:}
        \begin{itemize}
            \item AI refers to the simulation of human intelligence in machines.
            \item \textit{Example:} Algorithms like neural networks mimic human brain functions to analyze complex data.
        \end{itemize}
        
        \item \textbf{Cognitive Science Perspective:}
        \begin{itemize}
            \item AI is the study of how to create computers that can think and learn like humans.
            \item \textit{Example:} Chatbots that can engage in conversation and understand context.
        \end{itemize}
        
        \item \textbf{Mathematics and Statistics Perspective:}
        \begin{itemize}
            \item AI represents computational techniques to derive patterns from data.
            \item \textit{Illustration:} Predictive model can be expressed as:
            \begin{equation}
                Y = f(X) + \epsilon
            \end{equation}
            Where $Y$ is the output, $f(X)$ is the function, and $\epsilon$ is the error term.
        \end{itemize}
        
        \item \textbf{Philosophical Perspective:}
        \begin{itemize}
            \item AI involves understanding intelligence itself—what does it mean to think, reason, or possess consciousness?
            \item \textit{Example:} Discussions on whether machines can replicate human consciousness.
        \end{itemize}
    \end{enumerate}
    
\end{frame}

\begin{frame}[fragile]
    \frametitle{Key Points and Summary}
    
    \begin{itemize}
        \item AI is \textbf{not} a monolithic term; it varies depending on context.
        \item It includes subfields like:
        \begin{itemize}
            \item Machine Learning (ML)
            \item Natural Language Processing (NLP)
            \item Computer Vision
            \item Robotics
        \end{itemize}
        \item The goal of AI is to make processes more efficient and capable of human-like tasks.
    \end{itemize}
    
    \textbf{Summary:} AI is a dynamic and evolving field with diverse definitions across disciplines, offering vast applications and implications in real-world scenarios.
    
\end{frame}

\begin{frame}[fragile]
    \frametitle{Transitioning to Next Slide}
    
    Following our exploration of AI definitions, we will dive into the \textbf{Core Concepts of AI}. This section will cover foundational principles such as:
    \begin{itemize}
        \item Search algorithms
        \item Logical reasoning
        \item Probabilistic models
    \end{itemize}
    These concepts underpin intelligent systems.
    
\end{frame}

\begin{frame}[fragile]
    \frametitle{Core Concepts of AI}
    \begin{block}{Learning Objectives}
        \begin{itemize}
            \item Understand the fundamental principles that underpin artificial intelligence.
            \item Familiarize yourself with the key methods used in AI, including search algorithms, logic reasoning, and probabilistic models.
            \item Explore practical examples that illustrate how these concepts are used in real-world applications.
        \end{itemize}
    \end{block}
\end{frame}

\begin{frame}[fragile]
    \frametitle{1. Search Algorithms}
    \begin{block}{Definition}
        Search algorithms are procedures used to navigate through a problem space to find a solution.
    \end{block}
    
    \begin{block}{Types of Search Algorithms}
        \begin{itemize}
            \item \textbf{Uninformed Search:} Lacks additional information about the goal state.
                \begin{itemize}
                    \item Example: Finding the shortest path on an unweighted graph.
                \end{itemize}
            \item \textbf{Informed Search:} Uses heuristics for efficient solutions.
                \begin{itemize}
                    \item Example: Navigating a map using GPS predicting time based on traffic conditions.
                \end{itemize}
        \end{itemize}
    \end{block}
    
    \begin{block}{Key Point}
        Search algorithms are crucial for tasks such as game playing (e.g., chess) and pathfinding in robotics.
    \end{block}
\end{frame}

\begin{frame}[fragile]
    \frametitle{2. Logic Reasoning}
    \begin{block}{Definition}
        Logic reasoning involves drawing conclusions using a formal system of logic.
    \end{block}
    
    \begin{block}{Types of Logic}
        \begin{itemize}
            \item \textbf{Propositional Logic:} Deals with propositions that can be true or false.
                \begin{itemize}
                    \item Example: "If it rains, the ground will be wet."
                \end{itemize}
            \item \textbf{Predicate Logic:} Extends propositional logic with predicates and quantifiers.
                \begin{itemize}
                    \item Example: "All humans are mortal. Socrates is human, therefore Socrates is mortal."
                \end{itemize}
        \end{itemize}
    \end{block}
    
    \begin{block}{Key Point}
        Logic reasoning is used in applications such as automated theorem proving and knowledge representation.
    \end{block}
\end{frame}

\begin{frame}[fragile]
    \frametitle{3. Probabilistic Models}
    \begin{block}{Definition}
        Probabilistic models allow AI to make predictions and decisions under uncertainty by mathematically representing uncertain information.
    \end{block}

    \begin{block}{Key Concepts}
        \begin{itemize}
            \item \textbf{Bayesian Networks:} A graphical model representing variables and their conditional dependencies. \\
            \begin{equation}
                P(A | B) = \frac{P(B | A) \cdot P(A)}{P(B)} \quad \text{(Bayes’ Theorem)}
            \end{equation}
            \item \textbf{Markov Chains:} A model describing a sequence of events where each event's probability depends solely on the previous state.
                \begin{itemize}
                    \item Example: Weather prediction based on previous days' patterns.
                \end{itemize}
        \end{itemize}
    \end{block}
    
    \begin{block}{Key Point}
        Probabilistic models are essential for applications in natural language processing, speech recognition, and predictive analytics.
    \end{block}
\end{frame}

\begin{frame}[fragile]
    \frametitle{Conclusion}
    Understanding these core concepts—search algorithms, logic reasoning, and probabilistic models—forms the foundation for more advanced topics in AI. Each plays a critical role in how machines independently think, learn, and solve complex problems, ultimately leading to the creation of intelligent systems.

    By engaging with these principles, you will be better equipped to explore more complex branches of AI in subsequent lessons.
\end{frame}

\begin{frame}[fragile]
    \frametitle{Branches of AI - Introduction}
    \begin{block}{Overview}
        Artificial Intelligence (AI) is a broad field that consists of various subfields, often referred to as branches. Understanding these branches is crucial for appreciating their diverse applications and implications in real-world scenarios.
    \end{block}
    
    \begin{itemize}
        \item Major branches include:
        \begin{itemize}
            \item Machine Learning (ML)
            \item Natural Language Processing (NLP)
            \item Computer Vision (CV)
        \end{itemize}
    \end{itemize}
\end{frame}

\begin{frame}[fragile]
    \frametitle{Branches of AI - Machine Learning}
    \begin{block}{Machine Learning (ML)}
        \textbf{Definition:} A subset of AI that focuses on building systems that learn from data and improve performance over time without explicit programming.
    \end{block}
    
    \begin{itemize}
        \item \textbf{Key Concepts:}
        \begin{itemize}
            \item Supervised Learning: Training on labeled data (e.g., spam detection).
            \item Unsupervised Learning: Identifying patterns without prior labels (e.g., customer clustering).
            \item Reinforcement Learning: Learning through rewards and penalties (e.g., training robots).
        \end{itemize}
    \end{itemize}

    \begin{block}{Example}
        \textbf{Email Filtering:} Using ML to identify spam emails based on features from previous emails.
    \end{block}
\end{frame}

\begin{frame}[fragile]
    \frametitle{Branches of AI - Natural Language Processing and Computer Vision}
    \begin{block}{Natural Language Processing (NLP)}
        \textbf{Definition:} A field of AI enabling machines to understand, interpret, and generate human language effectively.
    \end{block}
    
    \begin{itemize}
        \item \textbf{Key Concepts:}
        \begin{itemize}
            \item Text Processing: Techniques like tokenization and stemming.
            \item Sentiment Analysis: Understanding sentiments in text.
            \item Machine Translation: Automatic conversion of text between languages.
        \end{itemize}
    \end{itemize}

    \begin{block}{Example}
        \textbf{Chatbots:} NLP enhances chatbots to respond to user queries conversationally.
    \end{block}

    \begin{block}{Computer Vision (CV)}
        \textbf{Definition:} A branch of AI that helps machines interpret and make decisions based on visual data.
    \end{block}
    
    \begin{itemize}
        \item \textbf{Key Concepts:}
        \begin{itemize}
            \item Image Recognition: Object, face, and scene identification.
            \item Object Detection: Locating/classifying multiple objects in images.
            \item Image Segmentation: Dividing images for analysis.
        \end{itemize}
    \end{itemize}

    \begin{block}{Example}
        \textbf{Autonomous Vehicles:} Utilize computer vision to navigate safely based on visual data.
    \end{block}
\end{frame}

\begin{frame}[fragile]
    \frametitle{Applications of AI}
    \begin{block}{Learning Objectives}
        \begin{itemize}
            \item Understand the diverse applications of AI across various sectors.
            \item Recognize specific examples of AI technologies in use today.
            \item Analyze the impact of AI advancements in different industries.
        \end{itemize}
    \end{block}
\end{frame}

\begin{frame}[fragile]
    \frametitle{Introduction to Applications of AI}
    Artificial Intelligence (AI) has revolutionized numerous sectors by enhancing efficiency, improving decision-making, and providing innovative solutions. Here, we will explore key areas where AI is making a significant impact:
    \begin{itemize}
        \item Healthcare
        \item Finance
        \item Entertainment
    \end{itemize}
\end{frame}

\begin{frame}[fragile]
    \frametitle{AI in Healthcare}
    \begin{block}{Overview}
        AI is transforming how healthcare providers diagnose and treat patients by analyzing vast amounts of data to identify patterns and assist in clinical decision-making.
    \end{block}
    
    \begin{block}{Examples}
        \begin{itemize}
            \item \textbf{Diagnosis and Imaging:} Machine Learning algorithms like IBM Watson Health analyze medical literature and patient records for early disease detection.
            \item \textbf{Predictive Analytics:} AI can predict patient outcomes by analyzing historical data, assisting hospitals in resource allocation.
        \end{itemize}
    \end{block}

    \begin{block}{Key Point}
        AI's ability to process and analyze data at scale helps in personalizing treatment plans and improving patient care.
    \end{block}
\end{frame}

\begin{frame}[fragile]
    \frametitle{AI in Finance}
    \begin{block}{Overview}
        In the financial sector, AI enhances security, optimizes trading, and personalizes customer experiences.
    \end{block}
    
    \begin{block}{Examples}
        \begin{itemize}
            \item \textbf{Fraud Detection:} Machine Learning models identify fraudulent transactions, alerting banks to suspicious activities in real-time.
            \item \textbf{Algorithmic Trading:} AI executes trades at speeds unattainable by human traders, maximizing profit margins.
        \end{itemize}
    \end{block}

    \begin{block}{Key Point}
        AI helps financial institutions reduce risks and enhance customer satisfaction through tailored services.
    \end{block}
\end{frame}

\begin{frame}[fragile]
    \frametitle{AI in Entertainment}
    \begin{block}{Overview}
        AI reshapes the entertainment industry, from content creation to recommending viewing options.
    \end{block}
    
    \begin{block}{Examples}
        \begin{itemize}
            \item \textbf{Content Recommendation:} Platforms like Netflix utilize AI to provide personalized content recommendations based on user preferences and viewing history.
            \item \textbf{Game Development:} AI creates adaptive non-player characters (NPCs) that provide tailored experiences in video games.
        \end{itemize}
    \end{block}

    \begin{block}{Key Point}
        AI enhances user engagement by providing personalized entertainment experiences and automating production processes.
    \end{block}
\end{frame}

\begin{frame}[fragile]
    \frametitle{Summary and Discussion Questions}
    \begin{block}{Summary}
        AI applications extend across various domains, showcasing its versatility and significant impact. From improving medical diagnoses and enhancing financial security to revolutionizing entertainment, AI is becoming an essential tool in modern society.
    \end{block}
    
    \begin{block}{Questions for Discussion}
        \begin{itemize}
            \item How do you think AI will shape the future of your chosen industry?
            \item What ethical considerations arise from the integration of AI in these sectors?
        \end{itemize}
    \end{block}
\end{frame}

\begin{frame}[fragile]
    \frametitle{Types of AI - Overview}
    \begin{block}{Overview of AI Classifications}
        Artificial Intelligence (AI) can be classified into three main categories based on its capabilities:
        \begin{itemize}
            \item Narrow AI
            \item General AI
            \item Superintelligent AI
        \end{itemize}
    \end{block}
    Understanding these types helps in grasping the capabilities and limitations of AI systems.
\end{frame}

\begin{frame}[fragile]
    \frametitle{Types of AI - Narrow AI}
    \begin{block}{Narrow AI}
        \begin{itemize}
            \item \textbf{Definition}: Also known as Weak AI, it refers to AI systems designed for a specific task or set of tasks.
            \item \textbf{Characteristics}:
                \begin{itemize}
                    \item Operates under a limited context.
                    \item Does not possess consciousness or self-awareness.
                \end{itemize}
            \item \textbf{Examples}: 
                \begin{itemize}
                    \item Voice assistants (Siri, Alexa)
                    \item Recommendation systems (Netflix, Amazon)
                    \item Image recognition software
                \end{itemize}
        \end{itemize}
    \end{block}
    \textbf{Key Points}:
    \begin{itemize}
        \item Scope: Task-specific.
        \item Common Use Cases: Chatbots, virtual customer service agents, autonomous vehicles.
    \end{itemize}
\end{frame}

\begin{frame}[fragile]
    \frametitle{Types of AI - General AI and Superintelligent AI}
    \begin{block}{General AI}
        \begin{itemize}
            \item \textbf{Definition}: Also referred to as Strong AI, it has the capability to understand, learn, and apply intelligence in any domain.
            \item \textbf{Characteristics}:
                \begin{itemize}
                    \item Can perform any intellectual task that a human can do.
                    \item Ability to reason, solve problems, and comprehend complex ideas.
                \end{itemize}
            \item \textbf{Current Status}: Theoretical and not yet realized in practice. 
            \item \textbf{Example}: OpenAI's GPT-3 showcases elements of General AI by generating human-like text, but true General AI remains unachieved.
        \end{itemize}
    \end{block}
    
    \begin{block}{Superintelligent AI}
        \begin{itemize}
            \item \textbf{Definition}: A hypothetical form of AI that surpasses human intelligence in all aspects.
            \item \textbf{Characteristics}:
                \begin{itemize}
                    \item Superior intellect that can outperform the best human minds.
                    \item Potential to self-improve and evolve autonomously.
                \end{itemize}
            \item \textbf{Current Status}: Remains a conceptual topic explored in science fiction.
        \end{itemize}
    \end{block}
    \textbf{Key Points}:
    \begin{itemize}
        \item Scope: Beyond human intelligence.
        \item Note: Raises ethical considerations about control and safety in AI development.
    \end{itemize}
\end{frame}

\begin{frame}
    \frametitle{Search Strategies in AI}
    \begin{block}{Overview of Search Strategies}
        In Artificial Intelligence (AI), search strategies are essential algorithms used to traverse problem spaces and find solutions. 
        They are pivotal in applications like puzzle solving, game playing, and automated planning.
    \end{block}

    \begin{itemize}
        \item Depth-First Search (DFS)
        \item Breadth-First Search (BFS)
    \end{itemize}
\end{frame}

\begin{frame}
    \frametitle{Depth-First Search (DFS)}
    \begin{block}{Definition}
        DFS is a searching algorithm that explores as far as possible along each branch before backtracking.
    \end{block}
    
    \begin{block}{How it Works}
        \begin{enumerate}
            \item Start at the root node (initial state).
            \item Explore each branch deeply before moving to the next sibling node.
            \item Backtrack when no unvisited nodes are left.
        \end{enumerate}
    \end{block}

    \begin{block}{Example}
        Consider a maze:
        \begin{verbatim}
           A
          / \
         B   C
            / \
           D   E
        \end{verbatim}
        Search sequence: A → B (backtrack) → A → C → D (backtrack) → C → E.
    \end{block}
    
    \begin{block}{Key Points}
        \begin{itemize}
            \item Space Complexity: $O(h)$ where $h$ is maximal depth.
            \item Time Complexity: $O(b^m)$ where $b$ is the branching factor and $m$ is maximum depth.
        \end{itemize}
    \end{block}
\end{frame}

\begin{frame}
    \frametitle{Breadth-First Search (BFS)}
    \begin{block}{Definition}
        BFS is a searching algorithm that explores all nodes at the present depth prior to moving on to the nodes at the next depth level.
    \end{block}

    \begin{block}{How it Works}
        \begin{enumerate}
            \item Start at the root node (initial state).
            \item Explore nodes level by level, adding unvisited nodes to a queue.
            \item Continue until the goal node is found or all options are exhausted.
        \end{enumerate}
    \end{block}

    \begin{block}{Example}
        Using the same maze:
        \begin{verbatim}
           A
          / \
         B   C
            / \
           D   E
        \end{verbatim}
        Search sequence: A → B → C → D → E.
    \end{block}

    \begin{block}{Key Points}
        \begin{itemize}
            \item Space Complexity: $O(b^d)$ where $b$ is branching factor and $d$ is depth of shallowest solution.
            \item Time Complexity: $O(b^d)$.
        \end{itemize}
    \end{block}
\end{frame}

\begin{frame}
    \frametitle{Comparison and Practical Considerations}
    \begin{block}{Summary of Comparison}
        \begin{tabular}{|c|c|c|}
            \hline
            Feature & Depth-First Search (DFS) & Breadth-First Search (BFS) \\
            \hline
            Strategy & Goes deep into branches first & Explores breadth before going deeper \\
            \hline
            Data Structure & Stack (or Recursion) & Queue \\
            \hline
            Memory Usage & Less efficient for long paths & More memory due to level-wise exploration \\
            \hline
            Application & Pathfinding, puzzles & Shortest path, social networks \\
            \hline
        \end{tabular}
    \end{block}

    \begin{block}{Practical Considerations}
        \begin{itemize}
            \item Choose DFS when memory is a constraint and the solutions are deep.
            \item Choose BFS when the solution is likely to be closer to the root or needs to be the optimal path.
        \end{itemize}
    \end{block}
\end{frame}

\begin{frame}[fragile]
    \frametitle{Code Snippet Example}
    \begin{block}{DFS Implementation}
        \begin{lstlisting}
def dfs(node, visited):
    if node not in visited:
        print(node.value)
        visited.add(node)
        for neighbor in node.neighbors:
            dfs(neighbor, visited)
        \end{lstlisting}
    \end{block}

    \begin{block}{BFS Implementation}
        \begin{lstlisting}
from collections import deque

def bfs(start):
    visited = set()
    queue = deque([start])
    while queue:
        node = queue.popleft()
        if node not in visited:
            print(node.value)
            visited.add(node)
            queue.extend(node.neighbors)
        \end{lstlisting}
    \end{block}
\end{frame}

\begin{frame}[fragile]
    \frametitle{Logic Reasoning in AI - Introduction}
    \begin{block}{Definition}
        Logic reasoning is the process of drawing conclusions from premises known or assumed to be true. In AI, it allows machines to make deductions, solve problems, and reason about their environments.
    \end{block}
\end{frame}

\begin{frame}[fragile]
    \frametitle{Logic Reasoning in AI - Importance}
    \begin{enumerate}
        \item \textbf{Knowledge Representation:}
        \begin{itemize}
            \item Logic provides a formal framework for representing knowledge about the world.
            \item Enables AI systems to understand relationships, facts, and rules.
        \end{itemize}
        
        \item \textbf{Decision Making:}
        \begin{itemize}
            \item Helps AI systems make informed decisions based on given data.
            \item Facilitates planning and problem-solving by evaluating different scenarios.
        \end{itemize}
    \end{enumerate}
\end{frame}

\begin{frame}[fragile]
    \frametitle{Logic Reasoning in AI - Types of Logic}
    \begin{enumerate}
        \item \textbf{Propositional Logic:}
        \begin{itemize}
            \item Deals with propositions and their logical relationships.
            \item Basic operations: AND, OR, NOT.
            \item \textit{Example:} If "P" is true (it is raining) and "Q" is true (the ground is wet), then "P AND Q" may suggest a causal relationship.
        \end{itemize}
        
        \item \textbf{First-Order Logic (Predicate Logic):}
        \begin{itemize}
            \item Extends propositional logic with quantifiers and predicates.
            \item \textit{Example:} $\forall x (\text{Cat}(x) \rightarrow \text{Mammal}(x))$ states "For all x, if x is a cat, then x is a mammal".
            \item Useful for more complex statements and relationships.
        \end{itemize}
    \end{enumerate}
\end{frame}

\begin{frame}[fragile]
    \frametitle{Logic Reasoning in AI - Applications and Inference}
    \begin{itemize}
        \item \textbf{Applications:}
        \begin{itemize}
            \item Expert Systems: Use logic to mimic human expert decision-making (e.g., medical diagnoses).
            \item Natural Language Processing (NLP): Logic helps interpret sentence meaning based on structure.
            \item Automated Theorem Proving: AI systems prove mathematical theorems using logic.
        \end{itemize}
        
        \item \textbf{Example of Inference in AI:}
        \begin{itemize}
            \item \textbf{Premises:}
            \begin{enumerate}
                \item If it rains, the ground gets wet. ($P \rightarrow Q$)
                \item It is raining. ($P$)
            \end{enumerate}
            
            \item \textbf{Conclusion:} Therefore, the ground is wet. ($Q$)
        \end{itemize}
    \end{itemize}
\end{frame}

\begin{frame}[fragile]
    \frametitle{Logic Reasoning in AI - Summary}
    Logic reasoning is crucial in AI for:
    \begin{itemize}
        \item Structured knowledge representation
        \item Enabling powerful decision-making capabilities.
    \end{itemize}
    By integrating these concepts within AI systems, we can create algorithms capable of meaningful reasoning and complex decision-making—key attributes of intelligent behavior.
\end{frame}

\begin{frame}[fragile]
    \frametitle{Probabilistic Models - Introduction}
    \begin{block}{Overview}
        Probabilistic models are essential in artificial intelligence (AI) for dealing with uncertainty and reasoning under incomplete knowledge. Unlike deterministic models that predict outcomes with certainty, probabilistic models provide a framework for understanding and interpreting the variability inherent in real-world data.
    \end{block}
\end{frame}

\begin{frame}[fragile]
    \frametitle{Probabilistic Models - Key Concepts}
    \begin{enumerate}
        \item \textbf{Uncertainty in AI}:
        \begin{itemize}
            \item Real-world data is often noisy and uncertain. 
            \item Probabilistic models assist AI in making informed decisions despite this uncertainty.
            \item Applications include medical diagnosis, weather forecasting, and stock market predictions.
        \end{itemize}
        
        \item \textbf{Probabilistic Reasoning}:
        \begin{itemize}
            \item Methods to draw conclusions based on likelihood of outcomes.
            \item Incorporates prior knowledge and evidence to update beliefs about uncertain events.
        \end{itemize}
        
        \item \textbf{Bayesian Probability}:
        \begin{itemize}
            \item Updates probability estimates as new evidence becomes available.
            \item \textbf{Bayes' Theorem}:
            \begin{equation}
                P(A|B) = \frac{P(B|A) \cdot P(A)}{P(B)}
            \end{equation}
        \end{itemize}
    \end{enumerate}
\end{frame}

\begin{frame}[fragile]
    \frametitle{Probabilistic Models - Examples}
    \begin{enumerate}
        \item \textbf{Spam Email Filtering}:
        \begin{itemize}
            \item Classifies emails as spam or not based on features like word frequency and sender reputation.
            \item Each email receives a probability score indicating its likelihood of being spam.
        \end{itemize}
        
        \item \textbf{Medical Diagnosis}:
        \begin{itemize}
            \item Disease prediction systems update prior disease probabilities with new evidence (e.g., test results) to determine the likelihood of a patient having a specific condition.
        \end{itemize}
    \end{enumerate}
\end{frame}

\begin{frame}[fragile]
    \frametitle{Probabilistic Models - Conclusion}
    \begin{block}{Key Takeaways}
        \begin{itemize}
            \item Probabilistic models enable effective decision-making in uncertain environments.
            \item The ability to update beliefs with new evidence through Bayesian methods is crucial for adaptive AI.
            \item Applications range from spam detection to autonomous vehicles and predictive maintenance.
        \end{itemize}
    \end{block}
    
    \begin{block}{Final Thoughts}
        Understanding probabilistic models is vital for developing advanced AI applications capable of reasoning in uncertain scenarios, improving prediction and decision-making capabilities.
    \end{block}
\end{frame}

\begin{frame}[fragile]
    \frametitle{Reinforcement Learning}
    Reinforcement Learning (RL) is a type of machine learning where an agent learns to make decisions by taking actions in an environment to maximize cumulative rewards.
    \begin{itemize}
        \item Distinction from supervised learning: RL relies on trial and error, not labeled data.
    \end{itemize}
\end{frame}

\begin{frame}[fragile]
    \frametitle{Key Concepts}
    \begin{itemize}
        \item \textbf{Agent}: The learner or decision maker.
        \item \textbf{Environment}: The external system with which the agent interacts.
        \item \textbf{State (s)}: A representation of the current situation of the agent.
        \item \textbf{Action (a)}: A choice made by the agent that affects the state.
        \item \textbf{Reward (r)}: Feedback signal received after taking an action.
        \item \textbf{Policy ($\pi$)}: Strategy for selecting actions based on the current state.
        \item \textbf{Value Function (V)}: Prediction of future rewards from a given state.
    \end{itemize}
\end{frame}

\begin{frame}[fragile]
    \frametitle{Basic Process Loop}
    \begin{enumerate}
        \item \textbf{Observing the State}: The agent observes the current state of the environment.
        \item \textbf{Choosing an Action}: The agent selects an action based on its policy.
        \item \textbf{Receiving Reward}: The environment responds, providing a reward and transitioning to a new state.
        \item \textbf{Updating Policy}: The agent updates its knowledge and modifies the policy based on received rewards.
    \end{enumerate}
\end{frame}

\begin{frame}[fragile]
    \frametitle{Illustration: RL Process}
    \begin{enumerate}
        \item \textbf{Start in a State (s)}: Agent in a grid world.
        \item \textbf{Take Action (a)}: Moves to an adjacent state.
        \item \textbf{Receive Reward (r)}: Receives +1 for goal, 0 for hitting a wall.
        \item \textbf{Learn from Feedback}: The agent learns to maximize total rewards over time.
    \end{enumerate}
\end{frame}

\begin{frame}[fragile]
    \frametitle{Applications of Reinforcement Learning}
    \begin{itemize}
        \item \textbf{Robotics}: Training robots for tasks like walking and grasping.
        \item \textbf{Game Playing}: Used in games like chess and Go (e.g., AlphaGo).
        \item \textbf{Healthcare}: Optimizing treatment plans based on individual responses.
        \item \textbf{Finance}: Algorithmic trading strategies adapting to market conditions.
    \end{itemize}
\end{frame}

\begin{frame}[fragile]
    \frametitle{Key Points to Emphasize}
    \begin{itemize}
        \item \textbf{Exploration vs. Exploitation}: Balancing new actions and known high-reward actions.
        \item \textbf{Temporal Difference Learning}: Learning directly from raw experience without a model.
        \item \textbf{Real-World Challenges}: Computational expense and extensive training time required.
    \end{itemize}
\end{frame}

\begin{frame}[fragile]
    \frametitle{Q-Learning Example}
    \begin{block}{Pseudo-Code}
    \begin{lstlisting}
initialize Q(s, a) arbitrarily  # State-action table
for each episode:
    initialize state s
    for each step:
        choose action a from state s using policy (e.g., epsilon-greedy)
        take action a, observe reward r, next state s'
        update Q-value:
        Q(s, a) ← Q(s, a) + α[r + γ * max_a' Q(s', a') - Q(s, a)]
        s ← s'
    \end{lstlisting}
    \end{block}
\end{frame}

\begin{frame}[fragile]
    \frametitle{Deep Learning Evolution - Overview}
    \begin{block}{Learning Objectives}
        \begin{itemize}
            \item Understand the evolution and architecture of deep learning.
            \item Recognize the role of deep learning in AI advancements.
            \item Identify real-world applications of deep learning.
        \end{itemize}
    \end{block}
\end{frame}

\begin{frame}[fragile]
    \frametitle{What is Deep Learning?}
    Deep learning is a subset of machine learning (ML) that uses multi-layered neural networks to analyze various factors of data. It mimics the human brain's neural architecture, allowing computers to recognize patterns and solve complex problems.
\end{frame}

\begin{frame}[fragile]
    \frametitle{Evolution of Deep Learning}
    \begin{enumerate}
        \item \textbf{Origins}
            \begin{itemize}
                \item Neural networks concept dates back to the 1940s with the \textbf{Perceptron} by Frank Rosenblatt.
                \item \textbf{Multi-layer Perceptrons (MLPs)} developed in the 1980s laid groundwork for complex networks.
            \end{itemize}
        \item \textbf{Breakthroughs (2006-2012)}
            \begin{itemize}
                \item In 2006, \textbf{Geoffrey Hinton} introduced \textbf{Deep Belief Networks (DBNs)}.
                \item \textbf{AlexNet} won ImageNet competition in 2012, achieving a significant drop in error rates.
            \end{itemize}
        \item \textbf{Current Landscape}
            \begin{itemize}
                \item Advanced architectures like \textbf{CNNs} for images and \textbf{RNNs} for sequences.
                \item Frameworks such as \textbf{TensorFlow}, \textbf{Keras}, and \textbf{PyTorch} democratizing deep learning.
            \end{itemize}
    \end{enumerate}
\end{frame}

\begin{frame}[fragile]
    \frametitle{Architecture of Deep Learning Models}
    \begin{itemize}
        \item \textbf{Neurons and Layers}
            \begin{itemize}
                \item Neurons operate like biological neurons, processing inputs and forwarding outputs.
                \item Layers consist of an \textbf{input layer}, multiple \textbf{hidden layers}, and an \textbf{output layer}.
            \end{itemize}
        \item \textbf{Activation Functions}
            \begin{itemize}
                \item Functions such as \textbf{ReLU} (Rectified Linear Unit) and \textbf{Sigmoid} activate neurons.
                \item Example of ReLU:
                \begin{equation}
                    f(x) = \max(0, x)
                \end{equation}
            \end{itemize}
        \item \textbf{Loss Functions}
            \begin{itemize}
                \item Measure how well the network is performing; a common type is \textbf{Cross-Entropy}.
            \end{itemize}
    \end{itemize}
\end{frame}

\begin{frame}[fragile]
    \frametitle{Role of Deep Learning in AI Advancements}
    \begin{itemize}
        \item \textbf{Natural Language Processing}
            \begin{itemize}
                \item Transformed by models like \textbf{Transformers} for translation and sentiment analysis.
            \end{itemize}
        \item \textbf{Computer Vision}
            \begin{itemize}
                \item Enhancements in image classification, object detection, and facial recognition.
            \end{itemize}
        \item \textbf{Autonomous Systems}
            \begin{itemize}
                \item Applications in self-driving cars for scene understanding and decision-making.
            \end{itemize}
    \end{itemize}
\end{frame}

\begin{frame}[fragile]
    \frametitle{Example Code Snippet}
    Here is a simple neural network example using Keras:
    \begin{lstlisting}[language=Python]
import tensorflow as tf
from tensorflow import keras
from keras.models import Sequential
from keras.layers import Dense

model = Sequential()
model.add(Dense(64, activation='relu', input_shape=(input_dim,)))  # Hidden Layer
model.add(Dense(1, activation='sigmoid'))  # Output Layer
model.compile(optimizer='adam', loss='binary_crossentropy', metrics=['accuracy'])
    \end{lstlisting}
\end{frame}

\begin{frame}[fragile]
    \frametitle{Key Points and Conclusion}
    \begin{itemize}
        \item Deep learning mimics human cognitive functions and handles large datasets effectively.
        \item Its architectural flexibility allows tackling diverse problems across industries.
        \item Collaboration between academia and industry continues to drive innovation.
    \end{itemize}
    \textbf{Conclusion:} Deep learning is pivotal in AI, offering architecture and application versatility. Understanding its evolution and mechanics is critical for leveraging AI technologies.
\end{frame}

\begin{frame}[fragile]
    \frametitle{Study Questions}
    \begin{enumerate}
        \item What are the main architectural components of a deep learning model?
        \item How do deep learning and traditional machine learning differ in handling data?
        \item Can you identify a specific application in your daily life where deep learning is used?
    \end{enumerate}
\end{frame}

\begin{frame}[fragile]
    \frametitle{Ethical Considerations in AI - Introduction}
    AI technologies are revolutionizing countless fields, from healthcare to finance, but they also raise significant societal and ethical implications. 
    Conversations around AI ethics are increasingly crucial as these systems integrate deeper into our daily lives.
\end{frame}

\begin{frame}[fragile]
    \frametitle{Ethical Considerations in AI - Key Concepts}
    \begin{enumerate}
        \item \textbf{Bias in AI}
        \begin{itemize}
            \item AI systems can inherit or amplify existing biases in training data.
            \item \textit{Example:} A hiring algorithm trained on historical data might favor certain demographics over others.
        \end{itemize}
        
        \item \textbf{Privacy Concerns}
        \begin{itemize}    
            \item Data collection practices may lead to user consent issues and privacy risks.
            \item \textit{Example:} Facial recognition can track individuals without consent.
        \end{itemize}
        
        \item \textbf{Accountability}
        \begin{itemize}
            \item Determining responsibility for AI decisions is complex.
            \item \textit{Example:} In an autonomous vehicle accident, who is accountable?
        \end{itemize}

        \item \textbf{Job Displacement}
        \begin{itemize}
            \item AI-driven automation may lead to job loss in various sectors.
            \item \textit{Example:} Self-service kiosks replace cashiers in fast-food restaurants.
        \end{itemize}
    \end{enumerate}
\end{frame}

\begin{frame}[fragile]
    \frametitle{Ethical Considerations in AI - Ethical Frameworks}
    \begin{itemize}
        \item \textbf{Utilitarianism:} Aim for the greatest good for the greatest number.
        \item \textbf{Deontological Ethics:} Adhere to ethical standards regardless of consequences.
        \item \textbf{Virtue Ethics:} Promote responsible use of technology through the character of AI developers.
    \end{itemize}

    \begin{block}{Key Points to Emphasize}
        \begin{itemize}
            \item \textbf{Transparency:} AI systems should be understandable to users.
            \item \textbf{Inclusivity:} Fair representation in algorithms to prevent bias.
            \item \textbf{Collaboration:} Involve ethicists and the public in AI development.
            \item \textbf{Regulation:} Create guidelines to ensure ethical compliance.
        \end{itemize}
    \end{block}
\end{frame}

\begin{frame}[fragile]
    \frametitle{Ethical Considerations in AI - Conclusion}
    As AI continues to evolve, understanding and addressing ethical considerations is crucial to ensuring a positive societal impact. 
    Integrating ethical principles into the development process can lead to a future where AI serves humanity responsibly.
\end{frame}

\begin{frame}[fragile]
    \frametitle{Responsible AI Development}
    
    Responsible AI Development encompasses the ethical practices and considerations aimed at designing and deploying artificial intelligence technologies that are beneficial and fair for society. As AI systems are increasingly integrated into various aspects of life, it is crucial to ensure that these technologies are developed and used responsibly.
\end{frame}

\begin{frame}[fragile]
    \frametitle{Key Concepts of Responsible AI}
    
    \begin{block}{Ethical AI Principles}
        \begin{itemize}
            \item \textbf{Fairness}: AI systems should treat all individuals equitably, avoiding bias based on race, gender, or economic status.
            \item \textbf{Transparency}: Processes and data used in AI should be understandable, allowing for scrutiny and accountability.
            \item \textbf{Accountability}: Organizations must take responsibility for AI outcomes and ensure mechanisms to address harm.
            \item \textbf{Privacy}: User data should be protected, complying with regulations such as GDPR.
        \end{itemize}
    \end{block}
\end{frame}

\begin{frame}[fragile]
    \frametitle{Impact of Unethical AI}
    
    \begin{itemize}
        \item Biased algorithms can lead to:
            \begin{itemize}
                \item Unfair hiring practices
                \item Unjust legal judgments
                \item Exacerbated inequalities
            \end{itemize}
        \item Lack of transparency may erode trust in technology, causing public backlash and regulatory scrutiny.
    \end{itemize}
\end{frame}

\begin{frame}[fragile]
    \frametitle{Examples of Responsible AI Practices}
    
    \begin{enumerate}
        \item \textbf{Bias Mitigation}: Techniques like diverse data sampling and algorithmic audit frameworks.
            \begin{itemize}
                \item Example: A recruitment AI that considers under-represented groups' backgrounds for wider applicant representation.
            \end{itemize}
        
        \item \textbf{Stakeholder Engagement}: Involving diverse stakeholders can capture various perspectives.
            \begin{itemize}
                \item Illustration: Workshops that invite community feedback on using facial recognition systems for public safety.
            \end{itemize}
        
        \item \textbf{Regulatory Adherence}: Following guidelines from organizations like IEEE and ISO for ethical AI design.
    \end{enumerate}
\end{frame}

\begin{frame}[fragile]
    \frametitle{Key Points and Conclusion}
    
    \begin{itemize}
        \item Responsible AI goes beyond compliance, fostering a culture of ethical consideration.
        \item Engaging with the community enhances accountability and trustworthiness.
        \item Continuous evaluation of AI practices is essential to adapt to societal norms.
    \end{itemize}
    
    \begin{block}{Conclusion}
        The responsible development of AI technologies ensures that these tools serve humanity positively and ethically. Emphasizing fairness, transparency, accountability, and privacy enhances integrity and builds public trust for sustainable AI development.
    \end{block}
\end{frame}

\begin{frame}[fragile]
    \frametitle{Future Trends in AI}
    \begin{block}{Slide Description}
        Examination of emerging trends and future directions in the field of artificial intelligence.
    \end{block}
\end{frame}

\begin{frame}[fragile]
    \frametitle{Learning Objectives}
    \begin{itemize}
        \item Understand current and emerging trends in artificial intelligence.
        \item Identify key technologies driving AI advancements.
        \item Recognize the potential impact of these trends on various industries.
    \end{itemize}
\end{frame}

\begin{frame}[fragile]
    \frametitle{Key Emerging Trends in AI}
    \begin{enumerate}
        \item \textbf{Explainable AI (XAI)}: Understanding AI decisions increases trust in critical sectors.
        \item \textbf{AI in Edge Computing}: Data processed closer to generation point minimizes latency.
        \item \textbf{NLP Advancements}: Enhanced capabilities in understanding and generating human language.
        \item \textbf{Autonomous Systems}: AI models performing tasks independently, such as self-driving cars.
    \end{enumerate}
\end{frame}

\begin{frame}[fragile]
    \frametitle{Further Emerging Trends in AI}
    \begin{enumerate}
        \setcounter{enumi}{4}  % Continue numbering from previous frame
        \item \textbf{AI and AR/VR}: Creating immersive experiences using AI capabilities.
        \item \textbf{AI for Climate Change}: Using AI for climate modeling and optimizing renewable energy.
        \item \textbf{AI Ethics and Bias Mitigation}: Addressing ethical challenges and bias in algorithms is crucial.
    \end{enumerate}
\end{frame}

\begin{frame}[fragile]
    \frametitle{Key Points to Emphasize}
    \begin{itemize}
        \item Integration of AI reshapes various industries including finance and healthcare.
        \item Continuous AI advancement leads to ethical dilemmas requiring responsible development.
        \item Understanding trends is essential for professionals to remain relevant in technology.
    \end{itemize}
\end{frame}

\begin{frame}[fragile]
    \frametitle{Conclusion}
    \begin{block}{Important Insights}
        As we explore these trends, awareness of their impact on future applications and responsible AI development is key. It lays the foundation for advanced studies in AI and related technologies.
    \end{block}
    
    \begin{block}{Critical Thinking}
        The exploration of these trends broadens our understanding of AI's capabilities and challenges us to consider the implications of its integration into daily life.
    \end{block}
\end{frame}

\begin{frame}[fragile]
    \frametitle{Conclusion and Key Takeaways - Overview}
    \begin{block}{Overview of Chapter Concepts}
        In our introduction to Artificial Intelligence (AI), we explored foundational concepts that set the stage for understanding this transformative field. Let’s summarize the key points and the implications they carry for our future learning journey.
    \end{block}
\end{frame}

\begin{frame}[fragile]
    \frametitle{Conclusion and Key Takeaways - Key Concepts}
    \begin{enumerate}
        \item \textbf{Definition and Scope of AI}
            \begin{itemize}
                \item AI refers to computer systems that perform tasks typically requiring human intelligence, such as understanding natural language, recognizing patterns, and making decisions.
                \item \textbf{Example}: Virtual assistants like Siri or Alexa use AI to interpret voice commands and provide responses.
            \end{itemize}
        \item \textbf{Types of AI}
            \begin{itemize}
                \item \textbf{Narrow AI}: Specialized systems designed for specific tasks (e.g., spam filters).
                \item \textbf{General AI}: A theoretical form of AI that possesses the ability to understand, learn, and apply knowledge across a diverse range of tasks, similar to a human.
            \end{itemize}
        \item \textbf{Basic AI Components}
            \begin{itemize}
                \item \textbf{Data}: The fuel for AI, allowing models to learn patterns and make predictions.
                \item \textbf{Algorithms}: The set of rules or instructions that the AI system follows to process data (e.g., decision trees, neural networks).
                \item \textbf{Computational Power}: The hardware resources required to run algorithms efficiently.
            \end{itemize}
    \end{enumerate}
\end{frame}

\begin{frame}[fragile]
    \frametitle{Conclusion and Key Takeaways - Applications and Implications}
    \begin{enumerate}
        \setcounter{enumi}{3} % Continue the enumeration
        \item \textbf{Applications of AI}
            \begin{itemize}
                \item AI is reshaping industries including healthcare (diagnosis support), finance (fraud detection), and transportation (autonomous vehicles).
                \item \textbf{Illustration}: Autonomous cars utilize AI to process data from sensors to navigate roads safely.
            \end{itemize}
    \end{enumerate}
    
    \begin{block}{Implications for Future Learning}
        \begin{itemize}
            \item \textbf{Interdisciplinary Approach}: Understanding AI requires knowledge of various fields, including computer science, statistics, ethics, and social sciences. Future lessons will integrate these perspectives.
            \item \textbf{Ethical Considerations}: As AI technologies evolve, so do questions regarding ethics and bias. It's essential to examine the societal impacts of AI systems.
            \item \textbf{Lifelong Learning}: The rapidly changing nature of AI technology means ongoing education is crucial. There will always be new tools, frameworks, and applications to explore.
        \end{itemize}
    \end{block}
\end{frame}


\end{document}