\documentclass[aspectratio=169]{beamer}

% Theme and Color Setup
\usetheme{Madrid}
\usecolortheme{whale}
\useinnertheme{rectangles}
\useoutertheme{miniframes}

% Additional Packages
\usepackage[utf8]{inputenc}
\usepackage[T1]{fontenc}
\usepackage{graphicx}
\usepackage{booktabs}
\usepackage{listings}
\usepackage{amsmath}
\usepackage{amssymb}
\usepackage{xcolor}
\usepackage{tikz}
\usepackage{pgfplots}
\pgfplotsset{compat=1.18}
\usetikzlibrary{positioning}
\usepackage{hyperref}

% Custom Colors
\definecolor{myblue}{RGB}{31, 73, 125}
\definecolor{mygray}{RGB}{100, 100, 100}
\definecolor{mygreen}{RGB}{0, 128, 0}
\definecolor{myorange}{RGB}{230, 126, 34}
\definecolor{mycodebackground}{RGB}{245, 245, 245}

% Set Theme Colors
\setbeamercolor{structure}{fg=myblue}
\setbeamercolor{frametitle}{fg=white, bg=myblue}
\setbeamercolor{title}{fg=myblue}
\setbeamercolor{section in toc}{fg=myblue}
\setbeamercolor{item projected}{fg=white, bg=myblue}
\setbeamercolor{block title}{bg=myblue!20, fg=myblue}
\setbeamercolor{block body}{bg=myblue!10}
\setbeamercolor{alerted text}{fg=myorange}

% Set Fonts
\setbeamerfont{title}{size=\Large, series=\bfseries}
\setbeamerfont{frametitle}{size=\large, series=\bfseries}
\setbeamerfont{caption}{size=\small}
\setbeamerfont{footnote}{size=\tiny}

% Code Listing Style
\lstdefinestyle{customcode}{
  backgroundcolor=\color{mycodebackground},
  basicstyle=\footnotesize\ttfamily,
  breakatwhitespace=false,
  breaklines=true,
  commentstyle=\color{mygreen}\itshape,
  keywordstyle=\color{blue}\bfseries,
  stringstyle=\color{myorange},
  numbers=left,
  numbersep=8pt,
  numberstyle=\tiny\color{mygray},
  frame=single,
  framesep=5pt,
  rulecolor=\color{mygray},
  showspaces=false,
  showstringspaces=false,
  showtabs=false,
  tabsize=2,
  captionpos=b
}
\lstset{style=customcode}

% Custom Commands
\newcommand{\hilight}[1]{\colorbox{myorange!30}{#1}}
\newcommand{\source}[1]{\vspace{0.2cm}\hfill{\tiny\textcolor{mygray}{Source: #1}}}
\newcommand{\concept}[1]{\textcolor{myblue}{\textbf{#1}}}
\newcommand{\separator}{\begin{center}\rule{0.5\linewidth}{0.5pt}\end{center}}

% Footer and Navigation Setup
\setbeamertemplate{footline}{
  \leavevmode%
  \hbox{%
  \begin{beamercolorbox}[wd=.3\paperwidth,ht=2.25ex,dp=1ex,center]{author in head/foot}%
    \usebeamerfont{author in head/foot}\insertshortauthor
  \end{beamercolorbox}%
  \begin{beamercolorbox}[wd=.5\paperwidth,ht=2.25ex,dp=1ex,center]{title in head/foot}%
    \usebeamerfont{title in head/foot}\insertshorttitle
  \end{beamercolorbox}%
  \begin{beamercolorbox}[wd=.2\paperwidth,ht=2.25ex,dp=1ex,center]{date in head/foot}%
    \usebeamerfont{date in head/foot}
    \insertframenumber{} / \inserttotalframenumber
  \end{beamercolorbox}}%
  \vskip0pt%
}

% Turn off navigation symbols
\setbeamertemplate{navigation symbols}{}

% Title Page Information
\title[Week 7: First-Order Logic]{Week 7: First-Order Logic}
\author[J. Smith]{John Smith, Ph.D.}
\institute[University Name]{
  Department of Computer Science\\
  University Name\\
  \vspace{0.3cm}
  Email: email@university.edu\\
  Website: www.university.edu
}
\date{\today}

% Document Start
\begin{document}

\frame{\titlepage}

\begin{frame}[fragile]
    \frametitle{Introduction to First-Order Logic}
    \begin{block}{Overview of First-Order Logic}
        First-Order Logic (FOL) is a formal system used in mathematics, philosophy, linguistics, and artificial intelligence (AI). It allows for expression of complex statements involving objects, properties, and relationships, in contrast to propositional logic which deals with simple true or false statements.
    \end{block}
\end{frame}

\begin{frame}[fragile]
    \frametitle{Key Concepts in First-Order Logic}
    \begin{itemize}
        \item \textbf{Predicates}: Represent properties or relations. For example, $P(x)$ might denote "x is a human."
        \item \textbf{Terms}:
            \begin{itemize}
                \item \textit{Constants}: Specific objects (e.g., "Alice").
                \item \textit{Variables}: Unspecified objects (e.g., "x").
                \item \textit{Functions}: Mapping inputs to outputs (e.g., $fatherOf(x)$).
            \end{itemize}
        \item \textbf{Quantifiers}:
            \begin{itemize}
                \item \textit{Universal Quantifier ($\forall$)}: $\forall x P(x)$ means "for all x, P(x) is true."
                \item \textit{Existential Quantifier ($\exists$)}: $\exists x P(x)$ means "there exists some x such that P(x) is true."
            \end{itemize}
    \end{itemize}
\end{frame}

\begin{frame}[fragile]
    \frametitle{Importance of First-Order Logic in AI}
    \begin{enumerate}
        \item \textbf{Knowledge Representation}: FOL provides a robust framework for encoding structured information.
        \item \textbf{Inference}: Enables automated reasoning, allowing systems to infer new information from existing data.
        \item \textbf{Natural Language Processing}: Bridges human language and machine understanding through logical representation.
    \end{enumerate}
\end{frame}

\begin{frame}[fragile]
    \frametitle{Example of First-Order Logic}
    Let's express the statement "All humans are mortal" in FOL:
    \begin{equation}
        \forall x (Human(x) \rightarrow Mortal(x))
    \end{equation}
    This means that for every entity $x$, if $x$ is a human, then $x$ is mortal.
\end{frame}

\begin{frame}[fragile]{Key Components of First-Order Logic - Introduction}
    \begin{block}{Introduction to First-Order Logic}
        First-Order Logic (FOL) is a formal system used in mathematics, philosophy, linguistics, and artificial intelligence. It allows for the representation of facts about the world using structured statements involving objects, their properties, and relations.
    \end{block}
\end{frame}

\begin{frame}[fragile]{Key Components of First-Order Logic - Overview}
    \begin{block}{Key Components}
        \begin{enumerate}
            \item Predicates
            \item Terms
            \item Constants
            \item Variables
            \item Functions
        \end{enumerate}
    \end{block}
\end{frame}

\begin{frame}[fragile]{Key Components of First-Order Logic - Predicates and Terms}
    \begin{block}{1. Predicates}
        \begin{itemize}
            \item \textbf{Definition:} A statement that expresses a property or relation that can be true or false based on argument values.
            \item \textbf{Example:} Let \( P(x) \) represent "x is a student." If \( x \) is "Alice," then \( P(Alice) \) is true.
        \end{itemize}
    \end{block}

    \begin{block}{2. Terms}
        \begin{itemize}
            \item \textbf{Definition:} Terms represent objects in the domain. They can be constants, variables, or functions.
            \item \textbf{Example:} In \( P(Alice) \), "Alice" is a constant term, and \( x \) is a variable.
        \end{itemize}
    \end{block}
\end{frame}

\begin{frame}[fragile]{Key Components of First-Order Logic - Constants, Variables, and Functions}
    \begin{block}{3. Constants}
        \begin{itemize}
            \item \textbf{Definition:} Specific, fixed entities within the logic system referring to particular objects.
            \item \textbf{Example:} Constants like "Harvard" or "Alice."
        \end{itemize}
    \end{block}

    \begin{block}{4. Variables}
        \begin{itemize}
            \item \textbf{Definition:} Symbols representing any object within the domain.
            \item \textbf{Example:} In \( P(x) \), \( x \) can take any object, making the statement general.
        \end{itemize}
    \end{block}

    \begin{block}{5. Functions}
        \begin{itemize}
            \item \textbf{Definition:} Functions map objects from the domain to other objects, generating new terms.
            \item \textbf{Example:} Let \( f(x) \) denote "the mother of x," where \( f(Alice) \) refers to "Alice's mother."
        \end{itemize}
    \end{block}
\end{frame}

\begin{frame}[fragile]{Key Components of First-Order Logic - Summary and Exploration}
    \begin{block}{Summary}
        First-Order Logic is built on these components, allowing for the structure and articulation of nuanced logical statements. This framework supports powerful reasoning in various fields.
    \end{block}

    \begin{block}{Further Exploration}
        Next, we will examine the \textbf{syntax and structure} of First-Order Logic statements, focusing on how these components combine to form valid expressions.
    \end{block}
\end{frame}

\begin{frame}[fragile]
    \frametitle{Syntax and Structure of First-Order Logic}
    \begin{block}{Learning Objectives}
        \begin{itemize}
            \item Understand the syntax rules that govern first-order logic statements.
            \item Identify the main components involved in constructing valid first-order logic expressions.
            \item Apply the syntax rules to create and analyze example statements.
        \end{itemize}
    \end{block}
\end{frame}

\begin{frame}[fragile]
    \frametitle{Key Concepts of Syntax in First-Order Logic}
    \begin{enumerate}
        \item \textbf{Basic Components}
            \begin{itemize}
                \item \textbf{Predicates:} Functions that return true or false depending on the arguments. Example: \(Likes(x, y)\) denotes "x likes y."
                \item \textbf{Terms:} Objects represented in logic.
                    \begin{itemize}
                        \item \textbf{Constants:} Specific entities (e.g., Alice, 42).
                        \item \textbf{Variables:} Symbols representing any object (e.g., \(x\), \(y\)).
                        \item \textbf{Functions:} Mappings from objects to objects (e.g., \(Mother(x)\)).
                    \end{itemize}
            \end{itemize}
        
        \item \textbf{Logical Connectives}
            \begin{itemize}
                \item \textbf{Negation (\(\neg\)):} "Not." Example: \(\neg Likes(Alice, Bob)\).
                \item \textbf{Conjunction (\(\land\)):} "And." Example: \(Likes(Alice, Bob) \land Likes(Bob, Alice)\).
                \item \textbf{Disjunction (\(\lor\)):} "Or." Example: \(Likes(Alice, Bob) \lor Likes(Bob, Carol)\).
                \item \textbf{Implication (\(\Rightarrow\)):} "If...then." Example: \(Likes(Alice, Bob) \Rightarrow Happy(Alice)\).
                \item \textbf{Biconditional (\(\Leftrightarrow\)):} "If and only if." Example: \(Likes(Alice, Bob) \Leftrightarrow Likes(Bob, Alice)\).
            \end{itemize}

        \item \textbf{Quantifiers}
            \begin{itemize}
                \item \textbf{Universal Quantifier (\(\forall\)):} "For all." Example: \(\forall x, Likes(x, Bob)\).
                \item \textbf{Existential Quantifier (\(\exists\)):} "There exists." Example: \(\exists y, Likes(Alice, y)\).
            \end{itemize}
    \end{enumerate}
\end{frame}

\begin{frame}[fragile]
    \frametitle{Structure of First-Order Logic Statements}
    \begin{block}{Well-formed Formulas (WFFs)}
        Expressions in first-order logic must be valid, consisting of:
        \begin{itemize}
            \item \textbf{Atomic Formulas:} Base statements from predicates and terms. Example: \(Likes(Alice, Bob)\).
            \item \textbf{Complex Formulas:} Built using logical connectives. Example: 
            \[
            \neg Likes(Alice, Bob) \land \forall x (Likes(x, Bob) \Rightarrow Happy(x))
            \]
        \end{itemize}
    \end{block}

    \begin{block}{Example WFF Combination}
        \[
        \forall x (Student(x) \Rightarrow \exists y (Teacher(y) \land Teaches(y, x)))
        \]
        This states: "For every student \(x\), there exists a teacher \(y\) such that \(y\) teaches \(x\)."
    \end{block}
    
    \begin{block}{Key Points to Emphasize}
        \begin{itemize}
            \item Understanding the structure of first-order logic is crucial.
            \item Mastery of syntax allows for the development of complex statements.
            \item Practice constructing WFFs using various predicates, terms, and quantifiers.
        \end{itemize}
    \end{block}
\end{frame}

\begin{frame}[fragile]
    \frametitle{Semantics of First-Order Logic}
    \begin{block}{Learning Objectives}
        \begin{itemize}
            \item Understand the basic concepts of first-order logic (FOL) semantics.
            \item Learn how first-order logic statements are interpreted in models.
            \item Develop skills to analyze and evaluate statements based on their semantic meaning.
        \end{itemize}
    \end{block}
\end{frame}

\begin{frame}[fragile]
    \frametitle{Key Concepts - First-Order Logic}
    \begin{block}{1. First-Order Logic (FOL)}
        First-Order Logic extends propositional logic by allowing the use of quantifiers, variables, predicates, and functions. 
        Unlike propositional logic, which deals only with true or false values of statements, FOL can express relationships between objects.
    \end{block}
\end{frame}

\begin{frame}[fragile]
    \frametitle{Key Concepts - Interpretations and Models}
    \begin{block}{2. Interpretation and Models}
        In FOL, the meaning of statements is described through \textbf{interpretations} and \textbf{models}.
        \begin{itemize}
            \item \textbf{Interpretation (I):} Assigns meanings to variables, functions, and predicates in a logical statement, defining the universe of discourse.
            \item \textbf{Model (M):} A specific interpretation that makes a given FOL statement true.
        \end{itemize}
    \end{block}
\end{frame}

\begin{frame}[fragile]
    \frametitle{Key Concepts - Components of Interpretation}
    \begin{block}{3. Components of Interpretation}
        \begin{itemize}
            \item \textbf{Domain (D):} The set of objects that our variables can refer to.
            \item \textbf{Predicates:} Functions that return true or false based on the objects in the domain. E.g., $P(x)$ could represent "x is an even number."
            \item \textbf{Functions:} Map elements from the domain to other elements. E.g., $f(x) = x + 1$.
        \end{itemize}
    \end{block}
\end{frame}

\begin{frame}[fragile]
    \frametitle{Key Concepts - Quantifiers}
    \begin{block}{4. Quantifiers}
        \begin{itemize}
            \item \textbf{Universal Quantifier ($\forall$):} Indicates that a statement applies to all elements in the domain. 
                  E.g., $\forall x P(x)$ means "for every $x$, $P(x)$ is true."
            \item \textbf{Existential Quantifier ($\exists$):} Indicates that there exists at least one element in the domain making the statement true.
                  E.g., $\exists y Q(y)$ means "there exists a $y$ such that $Q(y)$ is true."
        \end{itemize}
    \end{block}
\end{frame}

\begin{frame}[fragile]
    \frametitle{Example of FOL Statement}
    \begin{block}{Example}
        Consider the statement: 
        \begin{equation}
            \forall x (P(x) \rightarrow Q(x))
        \end{equation}
        \begin{itemize}
            \item \textbf{Interpretation:} Let $D$ be the set of all humans.
            \item \textbf{Predicates:} Let $P(x)$ denote "x is a human" and $Q(x)$ denote "x can speak."
            \item \textbf{Model:} This statement is true in our model if, for every human $x$, if $x$ is a human ($P(x)$), then $x$ can speak ($Q(x)$).
        \end{itemize}
    \end{block}
\end{frame}

\begin{frame}[fragile]
    \frametitle{Evaluating Truth Value}
    \begin{block}{Truth Value}
        To evaluate the truth of a statement, check if it holds in all possible interpretations:
        \begin{itemize}
            \item If every element in the domain satisfies the statement, it is considered \textbf{true} under that interpretation (model).
        \end{itemize}
    \end{block}
\end{frame}

\begin{frame}[fragile]
    \frametitle{Key Points to Emphasize}
    \begin{block}{Key Points}
        \begin{itemize}
            \item Understanding semantics is critical in evaluating the truth of logical statements.
            \item Models help us concretely apply FOL to specific scenarios.
            \item Interpretation of predicates and quantifiers is vital in assessing the logic behind statements.
        \end{itemize}
    \end{block}
\end{frame}

\begin{frame}[fragile]
    \frametitle{Inference in First-Order Logic}
    \begin{block}{Understanding Inference Rules}
        Inference rules form the foundation of reasoning in First-Order Logic (FOL). They allow us to draw conclusions from premises.
    \end{block}
    \begin{itemize}
        \item Types of inference rules:
        \begin{itemize}
            \item Unary
            \item Binary
        \end{itemize}
    \end{itemize}
\end{frame}

\begin{frame}[fragile]
    \frametitle{Unary Inference Rules}
    Unary rules operate on a single premise.
    \begin{itemize}
        \item \textbf{Modus Ponens}:
        \begin{itemize}
            \item Formula: If \(P\) and \(P \rightarrow Q\), conclude \(Q\).
            \item Example: 
                \begin{itemize}
                    \item Premise 1: \(R \rightarrow W\) (If it rains, the ground will be wet.)
                    \item Premise 2: \(R\) (It is raining.)
                    \item Conclusion: \(W\) (The ground is wet.)
                \end{itemize}
        \end{itemize}
        \item \textbf{Negation Introduction}:
        \begin{itemize}
            \item If a proposition leads to a contradiction, conclude its negation.
            \item Example: Assume light is on and derive a contradiction to conclude: "The light is not on."
        \end{itemize}
    \end{itemize}
\end{frame}

\begin{frame}[fragile]
    \frametitle{Binary Inference Rules}
    Binary rules operate on two premises.
    \begin{itemize}
        \item \textbf{Modus Tollens}:
        \begin{itemize}
            \item Formula: If \(P \rightarrow Q\) and \(\neg Q\), conclude \(\neg P\).
            \item Example:
                \begin{itemize}
                    \item Premise 1: \(R \rightarrow W\) (If it rains, the ground is wet.)
                    \item Premise 2: \(\neg W\) (The ground is not wet.)
                    \item Conclusion: \(\neg R\) (It is not raining.)
                \end{itemize}
        \end{itemize}
        \item \textbf{Disjunctive Syllogism}:
        \begin{itemize}
            \item Formula: If \(P \lor Q\) and \(\neg P\), conclude \(Q\).
            \item Example:
                \begin{itemize}
                    \item Premise 1: \(S \lor R\) (It is either sunny or rainy.)
                    \item Premise 2: \(\neg S\) (It is not sunny.)
                    \item Conclusion: \(R\) (It is rainy.)
                \end{itemize}
        \end{itemize}
    \end{itemize}
\end{frame}

\begin{frame}[fragile]
    \frametitle{Quantifiers in First-Order Logic - Overview}
    \begin{block}{Learning Objectives}
        \begin{itemize}
            \item Understand the role and significance of quantifiers in First-Order Logic (FOL).
            \item Differentiate between universal and existential quantifiers.
            \item Apply quantifiers in constructing logical statements.
        \end{itemize}
    \end{block}
\end{frame}

\begin{frame}[fragile]
    \frametitle{Quantifiers - Introduction}
    \begin{block}{Introduction to Quantifiers}
        Quantifiers are symbols used in First-Order Logic to express the quantity of subjects in a statement.
        They help specify how many instances or entities a property applies to.
    \end{block}
\end{frame}

\begin{frame}[fragile]
    \frametitle{Universal Quantifier (∀)}
    \begin{block}{Definition}
        \begin{itemize}
            \item \textbf{Symbol}: $\forall$ (for all)
            \item \textbf{Definition}: Asserts that a property holds for \textit{all} elements in a particular domain.
        \end{itemize}
    \end{block}
    
    \begin{block}{Explanation}
        When we say "For all x, P(x)", it means P(x) is true for every value of x in the given domain.
    \end{block}

    \begin{block}{Example}
        \begin{itemize}
            \item Statement: $\forall x$ (Cat(x) $\rightarrow$ Mammal(x))
            \item Meaning: For every x, if x is a cat, then x is a mammal.
        \end{itemize}
    \end{block}
    
    \begin{block}{Implication}
        Proving that P(x) holds for arbitrary elements guarantees it is true for all elements in that set.
    \end{block}
\end{frame}

\begin{frame}[fragile]
    \frametitle{Existential Quantifier (∃)}
    \begin{block}{Definition}
        \begin{itemize}
            \item \textbf{Symbol}: $\exists$ (there exists)
            \item \textbf{Definition}: States that there is at least one element in the domain for which a property holds.
        \end{itemize}
    \end{block}
    
    \begin{block}{Explanation}
        When we say "There exists an x such that P(x)", it means at least one instance where P(x) is true.
    \end{block}

    \begin{block}{Example}
        \begin{itemize}
            \item Statement: $\exists y$ (Dog(y) $\land$ Barks(y))
            \item Meaning: There exists a y such that y is a dog and y barks.
        \end{itemize}
    \end{block}
    
    \begin{block}{Implication}
        Finding a single example where P holds is sufficient to satisfy $\exists$.
    \end{block}
\end{frame}

\begin{frame}[fragile]
    \frametitle{Key Points and Applications}
    \begin{block}{Key Points to Emphasize}
        \begin{itemize}
            \item Distinction:
            \begin{itemize}
                \item Universal quantifier ($\forall$) refers to \textit{all} elements.
                \item Existential quantifier ($\exists$) indicates \textit{at least one} element.
            \end{itemize}
            \item Composition: Quantifiers can be combined; e.g., $\forall x$ ($\exists y$ P(x, y)) indicates for every x, there is some y for which P holds.
        \end{itemize}
    \end{block}

    \begin{block}{Practical Applications}
        Quantifiers are fundamental in:
        \begin{itemize}
            \item Mathematical proofs
            \item Computer science (databases, programming)
            \item Formal verification systems
        \end{itemize}
    \end{block}
\end{frame}

\begin{frame}[fragile]
    \frametitle{Conclusion}
    Understanding quantifiers enables us to express and manipulate properties of objects within logical frameworks. This establishes a foundation for more complex reasoning in First-Order Logic.
    
    \begin{block}{Formula Overview}
        \begin{itemize}
            \item Universal Quantifier: $\forall x$ P(x)
            \item Existential Quantifier: $\exists x$ P(x)
        \end{itemize}
    \end{block}
    
    In summary, quantifiers are a vital component of First-Order Logic, allowing for precise expression of statements involving categories of objects, and serving as foundational building blocks for logical reasoning and proof strategies.
\end{frame}

\begin{frame}[fragile]
    \frametitle{Constructing First-Order Logic Statements}
    \begin{block}{Learning Objectives}
        \begin{itemize}
            \item Understand the components of first-order logic (FOL).
            \item Learn the step-by-step process for constructing FOL statements.
            \item Apply examples to reinforce the learning of FOL statement construction.
        \end{itemize}
    \end{block}
\end{frame}

\begin{frame}[fragile]
    \frametitle{Key Concepts in First-Order Logic}
    \begin{itemize}
        \item \textbf{First-Order Logic (FOL)}: A formal system extending propositional logic by including quantifiers and predicates for reasoning about objects.
        \item \textbf{Components of FOL Statements}:
        \begin{itemize}
            \item \textbf{Predicates}: Express properties or relationships, e.g., \( P(x) \) representing "x is a person."
            \item \textbf{Terms}: Include variables, constants, and functions. E.g., \( a \) is a constant like "Alice."
            \item \textbf{Quantifiers}:
            \begin{itemize}
                \item \textbf{Universal Quantifier} (\( \forall \)): Applies to all members of a domain.
                \item \textbf{Existential Quantifier} (\( \exists \)): Indicates at least one member of the domain satisfies the statement.
            \end{itemize}
        \end{itemize}
    \end{itemize}
\end{frame}

\begin{frame}[fragile]
    \frametitle{Step-by-Step Guide to Constructing FOL Statements}
    \begin{enumerate}
        \item \textbf{Identify the Domain}: Determine the objects referred to, e.g., "all humans."
        \item \textbf{Define Predicates}: Specify properties or relations. E.g.,
        \begin{itemize}
            \item \( Human(x) \): "x is a human."
            \item \( Loves(x, y) \): "x loves y."
        \end{itemize}
        \item \textbf{Choose the Appropriate Quantifier}: Decide on \( \forall \) (all) or \( \exists \) (at least one).
        \item \textbf{Combine Components}: Formulate the complete statement by merging predicates and quantifiers.
        \item \textbf{Ensure Clarity and Consistency}: Make sure your statements convey the intended meaning.
    \end{enumerate}
\end{frame}

\begin{frame}[fragile]
    \frametitle{Example for Practice}
    \textbf{Statement}: "Everyone who is a student is enrolled in at least one course." \\
    \textbf{FOL Construction}:
    \begin{itemize}
        \item \textbf{Domain}: All students.
        \item \textbf{Predicates}:
        \begin{itemize}
            \item \( Student(x) \)
            \item \( EnrolledIn(x, c) \) (where \( c \) is a course)
        \end{itemize}
        \item \textbf{FOL Statement}:
        \begin{equation}
            \forall x (Student(x) \rightarrow \exists c (EnrolledIn(x, c)))
        \end{equation}
    \end{itemize}
    This structured approach helps in accurately constructing first-order logic statements.
\end{frame}

\begin{frame}[fragile]
    \frametitle{Examples of First-Order Logic Statements - Overview}
    \begin{block}{Understanding First-Order Logic (FOL)}
        First-Order Logic (FOL), also known as predicate logic, extends propositional logic by allowing quantifiers and predicates. It enables the expression of statements that involve properties of objects and their relationships.
    \end{block}
    
    \begin{itemize}
        \item \textbf{Key Components of FOL:}
        \begin{itemize}
            \item \textbf{Constants:} Specific objects (e.g., Alice, Bob).
            \item \textbf{Variables:} General objects (e.g., $x$, $y$).
            \item \textbf{Predicates:} Properties or relationships (e.g., $Loves(x, y)$).
            \item \textbf{Quantifiers:} 
            \begin{itemize}
                \item \textbf{Universal quantifier (∀):} For all instances (e.g., $\forall x$).
                \item \textbf{Existential quantifier (∃):} At least one instance (e.g., $\exists x$).
            \end{itemize}
        \end{itemize}
    \end{itemize}
\end{frame}

\begin{frame}[fragile]
    \frametitle{Examples of FOL Statements - Part 1}
    \begin{enumerate}
        \item \textbf{Simple Predicates:}
        \begin{itemize}
            \item \textbf{Statement:} "Alice loves Bob."
            \item \textbf{FOL Representation:} $Loves(Alice, Bob)$
            \item \textbf{Explanation:} This statement asserts a specific relationship between two individuals.
        \end{itemize}

        \item \textbf{Universal Quantification:}
        \begin{itemize}
            \item \textbf{Statement:} "All humans are mortal."
            \item \textbf{FOL Representation:} $\forall x (Human(x) \rightarrow Mortal(x))$
            \item \textbf{Explanation:} For every $x$, if $x$ is a human, then $x$ is mortal.
        \end{itemize}
    \end{enumerate}
\end{frame}

\begin{frame}[fragile]
    \frametitle{Examples of FOL Statements - Part 2}
    \begin{enumerate}
        \setcounter{enumi}{2} % Continue numbering from previous frame
        
        \item \textbf{Existential Quantification:}
        \begin{itemize}
            \item \textbf{Statement:} "There exists a human who is a philosopher."
            \item \textbf{FOL Representation:} $\exists x (Human(x) \land Philosopher(x))$
            \item \textbf{Explanation:} At least one individual $x$ exists such that $x$ is both a human and a philosopher.
        \end{itemize}

        \item \textbf{Combining Quantifiers:}
        \begin{itemize}
            \item \textbf{Statement:} "Every student has a friend."
            \item \textbf{FOL Representation:} $\forall x (Student(x) \rightarrow \exists y Friend(x, y))$
            \item \textbf{Explanation:} For every student $x$, there exists some $y$ such that $y$ is a friend of $x$.
        \end{itemize}
    \end{enumerate}
\end{frame}

\begin{frame}[fragile]
    \frametitle{Examples of FOL Statements - Part 3}
    \begin{enumerate}
        \setcounter{enumi}{4} % Continue numbering from previous frame

        \item \textbf{Nested Quantifiers:}
        \begin{itemize}
            \item \textbf{Statement:} "For every person, there exists a pet that they own."
            \item \textbf{FOL Representation:} $\forall x (Person(x) \rightarrow \exists y (Pet(y) \land Owns(x, y)))$
            \item \textbf{Explanation:} For every individual $x$, there is at least one $y$ such that $y$ is a pet owned by $x$.
        \end{itemize}
    \end{enumerate}

    \begin{block}{Key Points to Emphasize}
        \begin{itemize}
            \item \textbf{Predicate Logic vs. Propositional Logic:} FOL adds expressiveness with predicates.
            \item \textbf{Quantifiers:} Understanding universal and existential quantifiers is essential.
            \item \textbf{Logical Structures:} FOL can depict real-world scenarios in various fields.
        \end{itemize}
    \end{block}
\end{frame}

\begin{frame}[fragile]
    \frametitle{First-Order Logic vs Propositional Logic - Overview}
    
    \begin{block}{Propositional Logic}
        \begin{itemize}
            \item \textbf{Definition}: A branch of logic dealing with propositions (statements that can be true or false) and logical connectives.
            \item \textbf{Components}: Propositions (P, Q, R), connectives (AND, OR, NOT, IMPLIES, IF AND ONLY IF).
        \end{itemize}
    \end{block}
    
    \begin{block}{First-Order Logic (FOL)}
        \begin{itemize}
            \item \textbf{Definition}: An extension of propositional logic that includes quantifiers and predicates.
            \item \textbf{Components}: Predicates (P(x), Q(x, y)), quantifiers (∀ for "for all", ∃ for "there exists"), constants (a, b), variables (x, y).
        \end{itemize}
    \end{block}
\end{frame}

\begin{frame}[fragile]
    \frametitle{First-Order Logic vs Propositional Logic - Key Differences}
    
    \begin{enumerate}
        \item \textbf{Expressiveness}: 
        \begin{itemize}
            \item Propositional Logic: Limited to simple statements (e.g., P ∧ Q).
            \item First-Order Logic: Can express relationships (e.g., ∀x (Human(x) ⇒ Mortal(x))).
        \end{itemize}
        
        \item \textbf{Quantification}: 
        \begin{itemize}
            \item Propositional Logic: No quantifiers.
            \item First-Order Logic: Uses quantifiers to generalize statements.
        \end{itemize}
        
        \item \textbf{Domain of Discourse}: 
        \begin{itemize}
            \item Propositional Logic: No specific domain; just truth values.
            \item First-Order Logic: Discusses specific domains and objects.
        \end{itemize}
        
        \item \textbf{Structure}: 
        \begin{itemize}
            \item Propositional Logic: Atomic propositions.
            \item First-Order Logic: Predicates applied to objects.
        \end{itemize}
    \end{enumerate}
\end{frame}

\begin{frame}[fragile]
    \frametitle{Examples of Propositional and First-Order Logic}
    
    \begin{block}{Propositional Logic Example}
        \begin{itemize}
            \item Statement: "It is raining (P) or it is sunny (Q)."
            \item Representation: P ∨ Q
        \end{itemize}
    \end{block}
    
    \begin{block}{First-Order Logic Example}
        \begin{itemize}
            \item Statement: "Everyone in the class is a student."
            \item Representation: ∀x (InClass(x) ⇒ Student(x))
        \end{itemize}
    \end{block}
    
    \begin{block}{Key Points to Emphasize}
        \begin{itemize}
            \item Complexity: FOL enables complex representations, beneficial for reasoning in science and AI.
            \item Applications: Crucial for knowledge representation, allowing computers to understand and infer new knowledge.
        \end{itemize}
    \end{block}
\end{frame}

\begin{frame}[fragile]
    \frametitle{Conclusion and Next Steps}
    
    \begin{block}{Conclusion}
        Understanding the differences between propositional and first-order logic is essential for logical reasoning in various fields such as computer science and artificial intelligence.
    \end{block}
    
    \begin{block}{Next Steps}
        In the following slide, we will explore the applications of first-order logic in artificial intelligence.
    \end{block}
\end{frame}

\begin{frame}[fragile]
    \frametitle{Applications of First-Order Logic in AI}
    \begin{block}{What is First-Order Logic?}
        First-Order Logic (FOL), also known as predicate logic, extends propositional logic, allowing for expressions involving objects and their relationships. It introduces quantifiers and predicates for more complex assertions.
    \end{block}
    
    \begin{itemize}
        \item \textbf{Quantifiers}:
        \begin{itemize}
            \item Universal Quantifier ( $\forall$ ): Holds for all elements in a domain.
            \item Existential Quantifier ( $\exists$ ): There is at least one element for which a property holds.
        \end{itemize}
    \end{itemize}
\end{frame}

\begin{frame}[fragile]
    \frametitle{Importance and Key Applications in AI}
    First-order logic is crucial in AI for structured representation, reasoning, and knowledge inference. Here are some key applications:
    
    \begin{enumerate}
        \item \textbf{Knowledge Representation}
            \begin{equation}
            \forall x \, (Cat(x) \rightarrow Animal(x)) 
            \end{equation}
            This states: "All cats are animals."
        
        \item \textbf{Automated Theorem Proving}
            \begin{equation}
            \forall x \, (Even(x) \rightarrow \forall y \, (y = 2 * x)) 
            \end{equation}
            This means: "If x is even, then y is twice x."
        
        \item \textbf{Natural Language Processing (NLP)}
            A statement like "Alice loves Bob." can be represented as:
            \begin{equation}
            Loves(Alice, Bob)
            \end{equation}
        
        \item \textbf{Robotics and Control Systems}
            \begin{equation}
            \forall x \, (Robot(x) \rightarrow CanMove(x)) 
            \end{equation}
            This implies: "All robots can move."
        
        \item \textbf{Semantic Web}
            Used for defining ontologies, enabling structured data sharing and reasoning over the web.
    \end{enumerate}
\end{frame}

\begin{frame}[fragile]
    \frametitle{Examples and Key Points}
    \textbf{Example of a Knowledge Base:}
    
    \begin{itemize}
        \item Facts:
            \begin{equation}
            \forall x \, (Bird(x) \rightarrow CanFly(x)) \quad \text{(Every bird can fly.)}
            \end{equation}
            \begin{equation}
            Penguin(P) 
            \end{equation}
        
        \item Query:
            \begin{equation}
            CanFly(P) 
            \end{equation}
            \textbf{Inference:} The system reasons that Penguins do not fit the "Bird" category based on additional rules.
    
        \item \textbf{Versatility and Expressiveness}: FOL adapts across AI fields, closely resembling human reasoning.
        
        \item \textbf{Inference Capability}: Enables deriving new knowledge from existing data.
    \end{itemize}
    
    \textbf{Example Formula:}
    \begin{equation}
    \forall x \, (Student(x) \rightarrow Enrolled(x))
    \end{equation}
    This means: "For every x, if x is a student, then x is enrolled."
\end{frame}

\begin{frame}[fragile]
    \frametitle{Constructing Queries in First-Order Logic}
    \begin{block}{Learning Objectives}
        \begin{itemize}
            \item Understand the fundamental components of First-Order Logic (FOL) queries.
            \item Develop the ability to craft queries that reflect real-world scenarios.
            \item Apply techniques for reasoning and inference using FOL.
        \end{itemize}
    \end{block}
\end{frame}

\begin{frame}[fragile]
    \frametitle{What is First-Order Logic?}
    \begin{itemize}
        \item First-Order Logic (FOL) extends propositional logic by using quantifiers and predicates.
        \item It allows for expressing more complex statements about objects in a domain.
        \item FOL is essential for modeling relationships and properties of data effectively.
    \end{itemize}
\end{frame}

\begin{frame}[fragile]
    \frametitle{Components of First-Order Logic Queries}
    \begin{enumerate}
        \item \textbf{Predicates:} Functions returning true or false based on object properties.
        \item \textbf{Terms:} Constants, variables, or functions denoting objects (e.g., `Alice`, `x`).
        \item \textbf{Quantifiers:} Indicate quantity:
            \begin{itemize}
                \item \textbf{Universal ($\forall$):} All elements (e.g., $\forall x (Student(x) \rightarrow Enrolled(x))$).
                \item \textbf{Existential ($\exists$):} At least one element (e.g., $\exists y (Course(y) \land Teaches(Professor, y))$).
            \end{itemize}
        \item \textbf{Logical Connectives:} Operators like AND ($\land$), OR ($\lor$), NOT ($\neg$) that combine statements.
    \end{enumerate}
\end{frame}

\begin{frame}[fragile]
    \frametitle{Example Query Construction}
    \textbf{Scenario:} Find all students enrolled in "Introduction to AI".

    \textbf{Step 1: Define Terms and Predicates} 
    \begin{itemize}
        \item \textbf{Terms:} `AI\_Course`
        \item \textbf{Predicates:} `Student(x)`, `Enrolled(x, y)` where `y` is a course.
    \end{itemize}
    
    \textbf{Step 2: Formulate Query}
    \begin{equation}
        \exists x (Student(x) \land Enrolled(x, AI\_Course))
    \end{equation}
    This means "There exists a student $x$ who is enrolled in the Introduction to AI course."
\end{frame}

\begin{frame}[fragile]
    \frametitle{Key Points to Remember}
    \begin{itemize}
        \item \textbf{Structuring Queries:} Accurately structure predicates, terms, and quantifiers.
        \item \textbf{Use Cases in AI:} Applicable in fields like natural language processing and automated reasoning.
        \item \textbf{Practice:} Proficiency in FOL requires practice in constructing relevant queries.
    \end{itemize}
\end{frame}

\begin{frame}[fragile]
    \frametitle{Practical Exercise: Building Queries}
    
    \begin{block}{Learning Objectives}
        \begin{itemize}
            \item Understand the components of first-order logic (FOL) queries
            \item Construct FOL queries based on given scenarios
            \item Apply logical reasoning to form coherent and valid queries
        \end{itemize}
    \end{block}
\end{frame}

\begin{frame}[fragile]
    \frametitle{Overview of First-Order Logic (FOL)}
    
    \begin{itemize}
        \item First-order logic extends propositional logic with:
        \begin{itemize}
            \item Predicates
            \item Quantifiers
            \item Relations
        \end{itemize}
        \item **Key Components**:
        \begin{itemize}
            \item **Predicates**: Functions returning true or false (e.g., $Loves(John, Mary)$).
            \item **Constants**: Specific objects in our domain (e.g., $John$, $Mary$).
            \item **Variables**: Symbols representing any object (e.g., $x$, $y$).
            \item **Quantifiers**:
            \begin{itemize}
                \item Universal Quantifier ($\forall$): "for all"
                \item Existential Quantifier ($\exists$): "there exists"
            \end{itemize}
        \end{itemize}
    \end{frame}

\begin{frame}[fragile]
    \frametitle{Exercise Instructions}
    
    \begin{enumerate}
        \item **Scenario Selection**: Choose or be provided with a scenario to query about.
        \begin{itemize}
            \item Example predicates:
            \begin{itemize}
                \item $Friend(A, B)$ means A is a friend of B
                \item $Loves(A, B)$ means A loves B
            \end{itemize}
        \end{itemize}
        
        \item **Formulate Queries**: Create FOL statements:
        \begin{itemize}
            \item Query 1: "Everyone who loves Mary is a friend."
            \begin{equation}
                \forall x (Loves(x, Mary) \rightarrow Friend(x, Mary))
            \end{equation}
            \item Query 2: "There exists someone who loves everyone."
            \begin{equation}
                \exists x \forall y (Loves(x, y))
            \end{equation}
            \item Query 3: "If you are a friend of John, then you are loved by him."
            \begin{equation}
                \forall z (Friend(John, z) \rightarrow Loves(John, z))
            \end{equation}
        \end{itemize}
        
        \item **Check Validity**: Evaluate each query's truth and logical soundness.
    \end{enumerate}
\end{frame}

\begin{frame}[fragile]
    \frametitle{Example Query Construction}
    
    \textbf{Scenario Context}:  
    In a class, $Teaches(Teacher, Student)$ indicates that the teacher teaches the student.

    \textbf{Example Query}: "All students are taught by at least one teacher."
    
    \textbf{FOL Representation}:
    \begin{enumerate}
        \item Step 1: Identify predicates (e.g., relation between teacher and student).
        \item Step 2: Use the existential quantifier to express the relationship.
        \begin{equation}
            \forall y (Student(y) \rightarrow \exists x (Teacher(x) \land Teaches(x, y)))
        \end{equation}
    \end{enumerate}
    
    \textbf{Key Points to Emphasize}:
    \begin{itemize}
        \item Clarity in predicates for accurate representation.
        \item Appropriate use of quantifiers to capture meanings.
        \item Revise queries as needed based on logical evaluations.
    \end{itemize}
\end{frame}

\begin{frame}[fragile]
    \frametitle{Your Turn!}
    
    Now, it's your turn to construct and refine your own queries based on provided or chosen scenarios in class. 
    
    \begin{itemize}
        \item Share your queries for peer review and discussion!
    \end{itemize}
\end{frame}

\begin{frame}[fragile]
    \frametitle{Common Challenges in First-Order Logic}
    \begin{block}{Overview}
        First-order logic (FOL) is crucial in various fields but presents several challenges:
    \end{block}
\end{frame}

\begin{frame}[fragile]
    \frametitle{Understanding the Complexity of First-Order Logic}
    \begin{itemize}
        \item FOL enables expression of complex statements.
        \item Challenges include:
        \begin{enumerate}
            \item Ambiguity in statements
            \item Incompleteness
            \item Quantifier Management
            \item Decidability Issues
            \item Expressiveness vs. Computability
        \end{enumerate}
    \end{itemize}
\end{frame}

\begin{frame}[fragile]
    \frametitle{Common Challenges - Detailed Overview}
    \begin{itemize}
        \item \textbf{Ambiguity in Statements:}
        \begin{itemize}
            \item Misinterpretation of predicates can occur.
            \item \textit{Example:} "All birds can fly."
        \end{itemize}

        \item \textbf{Incompleteness:}
        \begin{itemize}
            \item Not all truths derivable from axioms.
            \item \textit{Illustration:} The Liar Paradox challenges formal proofs.
        \end{itemize}

        \item \textbf{Quantifier Management:}
        \begin{itemize}
            \item Proper use of $\forall$ and $\exists$ is crucial.
            \item \textit{Example:} Difference between "All humans are mortal" and "Some humans are philosophers."
        \end{itemize}
    \end{itemize}
\end{frame}

\begin{frame}[fragile]
    \frametitle{Common Challenges - Continued}
    \begin{itemize}
        \item \textbf{Decidability Issues:}
        \begin{itemize}
            \item Some problems are undecidable; no algorithm can determine truth values.
            \item \textit{Example:} Challenges in constructing proofs.
        \end{itemize}

        \item \textbf{Expressiveness vs. Computability:}
        \begin{itemize}
            \item FOL is expressive but can lead to complex computational problems.
            \item \textit{Example:} Determining satisfiability.
        \end{itemize}
    \end{itemize}
\end{frame}

\begin{frame}[fragile]
    \frametitle{Conclusion and Next Steps}
    \begin{block}{Conclusion}
        Common challenges in first-order logic include:
        \begin{itemize}
            \item Ambiguity
            \item Incompleteness
            \item Quantifier management
            \item Decidability issues
            \item Trade-off between expressiveness and computability
        \end{itemize}
        Understanding these can enhance your logical reasoning skills.
    \end{block}

    \begin{block}{Next Steps}
        Next, we will explore the future of first-order logic in AI research and real-world applications.
    \end{block}
\end{frame}

\begin{frame}[fragile]
    \frametitle{Future of First-Order Logic in AI}
    \begin{block}{Overview of First-Order Logic (FOL)}
        First-Order Logic is an expressive formalism used in AI for knowledge representation and reasoning. It extends propositional logic by incorporating quantifiers and predicates, allowing for more complex statements about objects, their properties, and relationships.
    \end{block}
\end{frame}

\begin{frame}[fragile]
    \frametitle{Future of First-Order Logic in AI - Ongoing Research Areas}
    \begin{enumerate}
        \item \textbf{Automated Reasoning}
            \begin{itemize}
                \item Enhancing automated theorem proving with a focus on efficiency and completeness.
                \item Example: Integration of FOL with machine learning algorithms to improve reasoning capabilities in uncertain environments.
            \end{itemize}
        \item \textbf{Knowledge Graphs}
            \begin{itemize}
                \item Utilizing FOL to define relationships within knowledge graphs, enabling sophisticated querying and inference.
                \item Example: Using FOL to infer new relationships from existing data in healthcare knowledge bases.
            \end{itemize}
        \item \textbf{Natural Language Processing (NLP)}
            \begin{itemize}
                \item Bridging language and machine understanding through FOL’s structure.
                \item Example: Converting natural language statements into FOL to facilitate query understanding and dialogue systems.
            \end{itemize}
    \end{enumerate}
\end{frame}

\begin{frame}[fragile]
    \frametitle{Future of First-Order Logic in AI - Future Applications}
    \begin{enumerate}
        \item \textbf{Explainable AI (XAI)}
            \begin{itemize}
                \item FOL can improve transparency in AI decisions by providing formal justification for conclusions.
                \item Key Point: Elucidating the reasoning process helps build trust in AI systems.
            \end{itemize}
        \item \textbf{Multi-Agent Systems}
            \begin{itemize}
                \item FOL aids in defining the interaction protocols between agents in distributed systems.
                \item Example: Cooperative agents using FOL to reason about joint actions and shared knowledge.
            \end{itemize}
        \item \textbf{Semantic Web}
            \begin{itemize}
                \item The integration of FOL with ontologies enables richer search capabilities and reasoning over web data.
                \item Key Point: FOL supports machine-readable semantics, enhancing retrieval and data interoperability online.
            \end{itemize}
    \end{enumerate}
    \begin{block}{Challenges and Considerations}
        - Scalability of FOL reasoning systems for highly complex datasets.\\
        - Balancing expressiveness and computational tractability.\\
        - Ensuring robust handling of uncertain and incomplete information.
    \end{block}
\end{frame}

\begin{frame}[fragile]
    \frametitle{Summary of Key Takeaways - Part 1}
    \begin{enumerate}
        \item \textbf{Understanding First-Order Logic (FOL)}: 
        \begin{itemize}
            \item FOL extends propositional logic for more expressive reasoning about objects and their relationships.
            \item \textbf{Components of FOL}:
            \begin{itemize}
                \item \textbf{Predicates}: Represent properties or relations (e.g., $Loves(John, Mary)$).
                \item \textbf{Quantifiers}:
                \begin{itemize}
                    \item \textbf{Universal Quantifier ($\forall$)}: E.g., $\forall x \; Loves(x, Mary)$.
                    \item \textbf{Existential Quantifier ($\exists$)}: E.g., $\exists x \; Loves(x, Mary)$.
                \end{itemize}
            \end{itemize}
        \end{itemize}
    \end{enumerate}
\end{frame}

\begin{frame}[fragile]
    \frametitle{Summary of Key Takeaways - Part 2}
    \begin{enumerate}
        \setcounter{enumi}{1} % Continue enumeration from Part 1
        \item \textbf{Syntax and Semantics}:
        \begin{itemize}
            \item \textbf{Syntax}: Structure of FOL (terms, formulas, wffs).
            \item \textbf{Semantics}: Meaning assigned to symbols defining truth values.
        \end{itemize}
        
        \item \textbf{Inference Rules}:
        \begin{itemize}
            \item \textbf{Modus Ponens}: If $P \rightarrow Q$ and $P$ is true, then conclude $Q$.
            \item \textbf{Resolution}: Method for automated theorem proving in FOL.
        \end{itemize}
    \end{enumerate}
\end{frame}

\begin{frame}[fragile]
    \frametitle{Summary of Key Takeaways - Part 3}
    \begin{enumerate}
        \setcounter{enumi}{3} % Continue enumeration from Part 2
        \item \textbf{Applications in AI}:
        \begin{itemize}
            \item Crucial for knowledge representation and reasoning in AI.
            \item Used in natural language processing and expert systems.
        \end{itemize}
        
        \item \textbf{Limitations of FOL}:
        \begin{itemize}
            \item Struggles with vague terms and probabilistic reasoning.
        \end{itemize}
        
        \item \textbf{Engaging Concept}:
        \begin{itemize}
            \item Compare FOL with propositional logic highlighting its expressive capabilities.
        \end{itemize}
    \end{enumerate}
\end{frame}

\begin{frame}[fragile]
    \frametitle{Q\&A Session on First-Order Logic - Introduction}
    % Overview of first-order logic and its significance
    \begin{block}{What is First-Order Logic (FOL)?}
        First-Order Logic is a formal system that allows for the expression of statements about objects and their relationships using quantifiers and predicates, extending beyond propositional logic.
    \end{block}
    
    \begin{itemize}
        \item Used in mathematics, philosophy, linguistics, and computer science.
        \item Enables deeper reasoning through the incorporation of quantifiers and predicates.
    \end{itemize}
\end{frame}

\begin{frame}[fragile]
    \frametitle{Key Components of First-Order Logic}
    % Breakdown of the fundamental elements in FOL
    \begin{enumerate}
        \item \textbf{Predicates}: Functions returning binary truth values. Example: 
        \( \text{Loves(John, Mary)} \) means "John loves Mary."
        
        \item \textbf{Quantifiers}:
        \begin{itemize}
            \item \textbf{Universal Quantifier (∀)}: Example: \( \forall x (\text{Human}(x) \rightarrow \text{Mortal}(x)) \) means "All humans are mortal."
            \item \textbf{Existential Quantifier (∃)}: Example: \( \exists y (\text{Cat}(y) \land \text{Black}(y)) \) means "There exists at least one black cat."
        \end{itemize}
        
        \item \textbf{Constants and Variables}: Constants are specific objects (like John), while variables are arbitrary objects (like \( x \)).
        
        \item \textbf{Logical Connectives}: Includes AND (\(\land\)), OR (\(\lor\)), NOT (\(\neg\)), IMPLIES (\(\rightarrow\)).
    \end{enumerate}
\end{frame}

\begin{frame}[fragile]
    \frametitle{Applications and Discussion Prompts}
    % Applications of FOL and prompts for discussion
    \begin{block}{Applications of First-Order Logic}
        \begin{itemize}
            \item \textbf{Knowledge Representation}: Used in AI for knowledge bases to deduce new facts.
            \item \textbf{Automated Theorem Proving}: Systems that can automatically prove mathematical theorems.
            \item \textbf{Database Query Languages}: Underpin languages like SQL for data retrieval.
        \end{itemize}
    \end{block}

    \begin{block}{Discussion Prompts}
        \begin{itemize}
            \item What are some domains where FOL can be applied? What are its limitations?
            \item How does the expressiveness of FOL compare to other logic systems?
            \item Can FOL represent all possible truths?
        \end{itemize}
    \end{block}
\end{frame}


\end{document}