\documentclass[aspectratio=169]{beamer}

% Theme and Color Setup
\usetheme{Madrid}
\usecolortheme{whale}
\useinnertheme{rectangles}
\useoutertheme{miniframes}

% Additional Packages
\usepackage[utf8]{inputenc}
\usepackage[T1]{fontenc}
\usepackage{graphicx}
\usepackage{booktabs}
\usepackage{listings}
\usepackage{amsmath}
\usepackage{amssymb}
\usepackage{xcolor}
\usepackage{tikz}
\usepackage{pgfplots}
\pgfplotsset{compat=1.18}
\usetikzlibrary{positioning}
\usepackage{hyperref}

% Custom Colors
\definecolor{myblue}{RGB}{31, 73, 125}
\definecolor{mygray}{RGB}{100, 100, 100}
\definecolor{mygreen}{RGB}{0, 128, 0}
\definecolor{myorange}{RGB}{230, 126, 34}
\definecolor{mycodebackground}{RGB}{245, 245, 245}

% Set Theme Colors
\setbeamercolor{structure}{fg=myblue}
\setbeamercolor{frametitle}{fg=white, bg=myblue}
\setbeamercolor{title}{fg=myblue}
\setbeamercolor{section in toc}{fg=myblue}
\setbeamercolor{item projected}{fg=white, bg=myblue}
\setbeamercolor{block title}{bg=myblue!20, fg=myblue}
\setbeamercolor{block body}{bg=myblue!10}
\setbeamercolor{alerted text}{fg=myorange}

% Set Fonts
\setbeamerfont{title}{size=\Large, series=\bfseries}
\setbeamerfont{frametitle}{size=\large, series=\bfseries}
\setbeamerfont{caption}{size=\small}
\setbeamerfont{footnote}{size=\tiny}

% Code Listing Style
\lstdefinestyle{customcode}{
  backgroundcolor=\color{mycodebackground},
  basicstyle=\footnotesize\ttfamily,
  breakatwhitespace=false,
  breaklines=true,
  commentstyle=\color{mygreen}\itshape,
  keywordstyle=\color{blue}\bfseries,
  stringstyle=\color{myorange},
  numbers=left,
  numbersep=8pt,
  numberstyle=\tiny\color{mygray},
  frame=single,
  framesep=5pt,
  rulecolor=\color{mygray},
  showspaces=false,
  showstringspaces=false,
  showtabs=false,
  tabsize=2,
  captionpos=b
}
\lstset{style=customcode}

% Custom Commands
\newcommand{\hilight}[1]{\colorbox{myorange!30}{#1}}
\newcommand{\source}[1]{\vspace{0.2cm}\hfill{\tiny\textcolor{mygray}{Source: #1}}}
\newcommand{\concept}[1]{\textcolor{myblue}{\textbf{#1}}}
\newcommand{\separator}{\begin{center}\rule{0.5\linewidth}{0.5pt}\end{center}}

% Footer and Navigation Setup
\setbeamertemplate{footline}{
  \leavevmode%
  \hbox{%
  \begin{beamercolorbox}[wd=.3\paperwidth,ht=2.25ex,dp=1ex,center]{author in head/foot}%
    \usebeamerfont{author in head/foot}\insertshortauthor
  \end{beamercolorbox}%
  \begin{beamercolorbox}[wd=.5\paperwidth,ht=2.25ex,dp=1ex,center]{title in head/foot}%
    \usebeamerfont{title in head/foot}\insertshorttitle
  \end{beamercolorbox}%
  \begin{beamercolorbox}[wd=.2\paperwidth,ht=2.25ex,dp=1ex,center]{date in head/foot}%
    \usebeamerfont{date in head/foot}
    \insertframenumber{} / \inserttotalframenumber
  \end{beamercolorbox}}%
  \vskip0pt%
}

% Turn off navigation symbols
\setbeamertemplate{navigation symbols}{}

% Title Page Information
\title[Week 9: Data Visualization Techniques]{Week 9: Data Visualization Techniques}
\author[J. Smith]{John Smith, Ph.D.}
\institute[University Name]{
  Department of Computer Science\\
  University Name\\
  \vspace{0.3cm}
  Email: email@university.edu\\
  Website: www.university.edu
}
\date{\today}

% Document Start
\begin{document}

\frame{\titlepage}

\begin{frame}[fragile]
    \titlepage
\end{frame}

\begin{frame}[fragile]
    \frametitle{Overview of Data Visualization}
    \begin{itemize}
        \item Data Visualization is the graphical representation of information and data.
        \item Utilizes visual elements like charts, graphs, and maps to:
        \begin{itemize}
            \item Provide accessible ways to see and understand trends.
            \item Identify outliers and patterns in data.
        \end{itemize}
    \end{itemize}
\end{frame}

\begin{frame}[fragile]
    \frametitle{Importance of Data Visualization}
    \begin{enumerate}
        \item \textbf{Better Understanding of Data} 
            \begin{itemize}
                \item Converts complex data sets into visual formats for intuitive understanding.
                \item Example: Line graphs show trends over time more effectively than tables.
            \end{itemize}
        
        \item \textbf{Enhanced Insights and Decision Making}
            \begin{itemize}
                \item Summarizes large quantities of data in visuals for quick comprehension.
                \item Example: Heat maps illustrate geographic distribution of sales.
            \end{itemize}
        
        \item \textbf{Engagement and Retention}
            \begin{itemize}
                \item Visual content is more engaging than text-heavy reports.
                \item Illustration: Infographics combine images with data for easier understanding.
            \end{itemize}
        
        \item \textbf{Revealing Patterns and Trends}
            \begin{itemize}
                \item Visualizations reveal correlations and structures in data.
                \item Example: Scatter plots highlight relationships between variables.
            \end{itemize}
    \end{enumerate}
\end{frame}

\begin{frame}[fragile]
    \frametitle{Key Points to Emphasize}
    \begin{itemize}
        \item \textbf{User-Friendly}: Visualizations should be accessible to both experts and non-experts.
        \item \textbf{Clarity and Simplicity}: Prioritize clear, uncluttered design; every element should have a purpose.
        \item \textbf{Interactive Elements}: Modern tools offer interactivity for dynamic exploration of data.
    \end{itemize}
\end{frame}

\begin{frame}[fragile]
    \frametitle{Conclusion}
    In today's data-driven world, data visualization is essential for transforming raw data into meaningful insights. By emphasizing clarity, engagement, and effective presentation, we harness the power of data visualization across numerous domains.
\end{frame}

\begin{frame}[fragile]
    \frametitle{Example Code Snippet (Matplotlib)}
    \begin{lstlisting}[language=Python]
import matplotlib.pyplot as plt

# Sample data
categories = ['A', 'B', 'C']
values = [10, 15, 7]

# Creating the bar chart
plt.bar(categories, values)
plt.title('Sample Bar Chart')
plt.xlabel('Categories')
plt.ylabel('Values')
plt.show()
    \end{lstlisting}
    This code snippet demonstrates how to create a basic bar chart, showcasing the ease of using visual tools to interpret numerical data.
\end{frame}

\begin{frame}[fragile]
    \frametitle{What is Matplotlib? - Introduction}
    \begin{block}{Introduction to Matplotlib}
        Matplotlib is a widely-used, open-source plotting library for Python that provides an object-oriented API for embedding plots into applications. 
        It's particularly favored by data scientists and analysts for its simplicity and versatility, making it a fundamental tool in the field of data visualization.
    \end{block}
\end{frame}

\begin{frame}[fragile]
    \frametitle{What is Matplotlib? - Key Features}
    \begin{block}{Key Features of Matplotlib}
        \begin{enumerate}
            \item \textbf{Versatile Plot Types:} Supports a wide variety of plots, including line plots, bar charts, histograms, and scatter plots.
            \item \textbf{Customization:} Customize every aspect of a plot from titles, labels, and legends to colors and line styles.
            \item \textbf{Integration:} Seamlessly integrates with libraries like NumPy and Pandas.
            \item \textbf{Interactivity:} Supports interactive figures to zoom and pan.
            \item \textbf{Export Options:} Plots can be saved in formats like PNG, PDF, and SVG.
        \end{enumerate}
    \end{block}
\end{frame}

\begin{frame}[fragile]
    \frametitle{Why is Matplotlib Widely Used?}
    \begin{block}{Reasons for Popularity}
        \begin{itemize}
            \item \textbf{Accessibility:} Suitable for both beginners and advanced users with a simple syntax.
            \item \textbf{Rich Documentation:} Well-documented with numerous examples.
            \item \textbf{Active Community:} Large user community for continuous improvement and support.
        \end{itemize}
    \end{block}
\end{frame}

\begin{frame}[fragile]
    \frametitle{Matplotlib Example Code}
    \begin{block}{Example Code Snippet}
        Below is a simple example of how to create a basic line plot using Matplotlib:
        \begin{lstlisting}[language=Python]
import matplotlib.pyplot as plt

# Sample data
x = [1, 2, 3, 4, 5]
y = [2, 3, 5, 1, 4]

# Create a line plot
plt.plot(x, y, marker='o')

# Customize the plot
plt.title('Sample Line Plot')
plt.xlabel('X-axis Label')
plt.ylabel('Y-axis Label')

# Show the plot
plt.show()
        \end{lstlisting}
    \end{block}
\end{frame}

\begin{frame}[fragile]
    \frametitle{Key Points to Remember}
    \begin{block}{Key Points}
        \begin{itemize}
            \item Matplotlib is essential for data visualization in Python.
            \item Flexibility and customization support high-quality visualizations.
            \item Mastering Matplotlib leads to advanced data visualization techniques.
        \end{itemize}
    \end{block}
\end{frame}

\begin{frame}[fragile]
    \frametitle{Conclusion}
    \begin{block}{Conclusion}
        Understanding and utilizing Matplotlib is a crucial step in effectively visualizing and analyzing data. 
        In the next slide, we will dive into creating basic plots and explore the different types of visualizations offered by Matplotlib.
    \end{block}
\end{frame}

\begin{frame}
    \frametitle{Basic Plotting with Matplotlib}
    \begin{block}{Introduction}
        Matplotlib is a powerful Python library for creating a variety of visualizations. This guide covers:
        \begin{itemize}
            \item Line Plots
            \item Scatter Plots
            \item Bar Charts
        \end{itemize}
    \end{block}
\end{frame}

\begin{frame}[fragile]
    \frametitle{Line Plots}
    \begin{block}{Description}
        **Line plots** display data points over time or other continuous variables. They effectively visualize trends.
    \end{block}
    
    \begin{block}{Code Example}
    \begin{lstlisting}[language=Python]
import matplotlib.pyplot as plt

# Sample data
x = [0, 1, 2, 3, 4]
y = [0, 1, 4, 9, 16]

# Creating a line plot
plt.plot(x, y)
plt.title('Line Plot Example')
plt.xlabel('X-axis Label')
plt.ylabel('Y-axis Label')
plt.grid(True)
plt.show()
    \end{lstlisting}
    \end{block}

    \begin{block}{Key Points}
        \begin{itemize}
            \item Use \texttt{plt.plot()} for line plots.
            \item Include titles and labels for clarity.
            \item \texttt{plt.grid(True)} enhances readability.
        \end{itemize}
    \end{block}
\end{frame}

\begin{frame}[fragile]
    \frametitle{Scatter Plots}
    \begin{block}{Description}
        **Scatter plots** visualize the relationship between two variables, showing how one is affected by another.
    \end{block}

    \begin{block}{Code Example}
    \begin{lstlisting}[language=Python]
import matplotlib.pyplot as plt
import numpy as np

# Sample data
x = np.random.rand(50)
y = np.random.rand(50)

# Creating a scatter plot
plt.scatter(x, y, color='blue', alpha=0.5)
plt.title('Scatter Plot Example')
plt.xlabel('X-axis Label')
plt.ylabel('Y-axis Label')
plt.show()
    \end{lstlisting}
    \end{block}

    \begin{block}{Key Points}
        \begin{itemize}
            \item Use \texttt{plt.scatter()} for scatter plots.
            \item The \texttt{alpha} parameter controls point transparency.
            \item Customize colors for differentiation.
        \end{itemize}
    \end{block}
\end{frame}

\begin{frame}[fragile]
    \frametitle{Bar Charts}
    \begin{block}{Description}
        **Bar charts** compare different categories of data, represented with rectangular bars.
    \end{block}

    \begin{block}{Code Example}
    \begin{lstlisting}[language=Python]
import matplotlib.pyplot as plt

# Sample data
categories = ['Category A', 'Category B', 'Category C']
values = [5, 7, 3]

# Creating a bar chart
plt.bar(categories, values, color='orange')
plt.title('Bar Chart Example')
plt.xlabel('Categories')
plt.ylabel('Values')
plt.show()
    \end{lstlisting}
    \end{block}

    \begin{block}{Key Points}
        \begin{itemize}
            \item Use \texttt{plt.bar()} for bar charts.
            \item Clearly label bars to indicate categories.
            \item Color choice enhances visual appeal.
        \end{itemize}
    \end{block}
\end{frame}

\begin{frame}
    \frametitle{Conclusion}
    By mastering these basic plot types in Matplotlib, you will effectively visualize your data. These foundational techniques pave the way for further exploration and customization of plots.
\end{frame}

\begin{frame}
    \frametitle{Tips for Further Exploration}
    \begin{itemize}
        \item Experiment with different data sets to see visualization changes.
        \item Explore Matplotlib's customization options in upcoming sessions.
    \end{itemize}
\end{frame}

\begin{frame}[fragile]
    \frametitle{Customization of Matplotlib Plots - Overview}
    \begin{block}{Introduction}
        Matplotlib is a powerful library that allows for extensive customization of plots, improving clarity and visual appeal. This presentation covers various aspects of customizing Matplotlib plots including titles, labels, colors, and other stylistic elements.
    \end{block}
\end{frame}

\begin{frame}[fragile]
    \frametitle{Customization of Matplotlib Plots - Titles and Labels}
    \begin{itemize}
        \item \textbf{Title}: Clearly describe what your plot represents.
        \item \textbf{X-axis and Y-axis Labels}: Provide context for the data.
    \end{itemize}
    \begin{block}{Example Code}
    \begin{lstlisting}[language=Python]
import matplotlib.pyplot as plt

plt.plot([1, 2, 3], [1, 4, 9])
plt.title("Quadratic Function")
plt.xlabel("X Values")
plt.ylabel("Y Values")
plt.show()
    \end{lstlisting}
    \end{block}
\end{frame}

\begin{frame}[fragile]
    \frametitle{Customization of Matplotlib Plots - Colors and Styles}
    \begin{itemize}
        \item \textbf{Line Colors}: Choose distinct colors for different data series.
        \item \textbf{Markers}: Use different markers for clarity in representing data points.
    \end{itemize}
    \begin{block}{Example Code}
    \begin{lstlisting}[language=Python]
plt.plot([1, 2, 3], [1, 4, 9], color='blue', marker='o', linestyle='--')
plt.show()
    \end{lstlisting}
    \end{block}
\end{frame}

\begin{frame}[fragile]
    \frametitle{Customization of Matplotlib Plots - Font Properties}
    Customizing font size, family, and weight improves readability.
    \begin{block}{Example Code}
    \begin{lstlisting}[language=Python]
plt.title("Quadratic Function", fontsize=14, fontweight='bold', fontname='Comic Sans MS')
plt.xlabel("X Values", fontsize=12)
plt.ylabel("Y Values", fontsize=12)
    \end{lstlisting}
    \end{block}
\end{frame}

\begin{frame}[fragile]
    \frametitle{Customization of Matplotlib Plots - Grids and Limits}
    \begin{itemize}
        \item \textbf{Grids}: Adding grids can help readers interpret the plot.
        \item \textbf{Setting Limits}: Control the visible range of your axes.
    \end{itemize}
    \begin{block}{Example Code}
    \begin{lstlisting}[language=Python]
plt.grid(True)
plt.xlim(0, 4)
plt.ylim(0, 10)
    \end{lstlisting}
    \end{block}
\end{frame}

\begin{frame}[fragile]
    \frametitle{Customization of Matplotlib Plots - Legend}
    A legend is vital for identifying data series in multi-data plots.
    \begin{block}{Example Code}
    \begin{lstlisting}[language=Python]
plt.plot([1, 2, 3], [1, 4, 9], color='blue', label='y = x^2')
plt.plot([1, 2, 3], [1, 2, 3], color='green', label='y = x')
plt.legend()
    \end{lstlisting}
    \end{block}
\end{frame}

\begin{frame}[fragile]
    \frametitle{Key Points and Conclusion}
    \begin{itemize}
        \item Always label your axes and add appropriate titles.
        \item Strategically use colors, markers, and line styles.
        \item Customize properties to enhance visual aspects without clutter.
    \end{itemize}
    Customization in Matplotlib is essential for clear, informative, and visually appealing plots. Practice these techniques to communicate your findings effectively.
\end{frame}

\begin{frame}[fragile]
    \frametitle{What is Seaborn? - Overview}
    \begin{block}{Overview of Seaborn}
        Seaborn is a powerful Python visualization library built on top of Matplotlib, designed for creating informative and attractive statistical graphics. It simplifies the process of drawing visually appealing statistical graphics without extensive customization.
    \end{block}
\end{frame}

\begin{frame}[fragile]
    \frametitle{What is Seaborn? - Advantages}
    \begin{block}{Advantages of Seaborn Over Matplotlib}
        \begin{itemize}
            \item \textbf{High-level Interface}: Abstraction allows creation of complex visualizations with simple commands.
            \item \textbf{Statistical Functions}: Built-in support for heatmaps, box plots, and violin plots for better data representation.
            \item \textbf{Enhanced Aesthetics}: Improved default styles and color palettes enhance visualization appeal.
            \item \textbf{Integration with Pandas}: Directly works with Pandas DataFrames for easier data handling and visualization.
        \end{itemize}
    \end{block}
\end{frame}

\begin{frame}[fragile]
    \frametitle{What is Seaborn? - Key Features}
    \begin{block}{Key Features of Seaborn}
        \begin{enumerate}
            \item \textbf{Data Handling}: Seamless integration with Pandas DataFrames for direct manipulation and visualization.
            \item \textbf{Built-in Themes}: Various themes can be applied to plots with commands like \texttt{sns.set\_theme(style="darkgrid")}.
            \item \textbf{Multiple Plot Types}: Supports a variety of statistical plots: 
                \begin{itemize}
                    \item Distribution plots: histograms and KDE.
                    \item Categorical plots: box plots and violin plots.
                    \item Matrix plots: heatmaps and pair plots.
                \end{itemize}
            \item \textbf{Color Palettes}: Selection of customizable color palettes according to the data.
        \end{enumerate}
    \end{block}
\end{frame}

\begin{frame}[fragile]
    \frametitle{What is Seaborn? - Example Usage}
    \begin{block}{Example Usage}
        Here’s a basic example of creating a box plot using Seaborn with Pandas:
        \begin{lstlisting}[language=Python]
import seaborn as sns
import pandas as pd
import matplotlib.pyplot as plt

# Load an example dataset
tips = sns.load_dataset("tips")

# Create a box plot
sns.boxplot(x="day", y="total_bill", data=tips)
plt.title('Total Bill Amount by Day of the Week')
plt.show()
        \end{lstlisting}
        In this example:
        \begin{itemize}
            \item The \texttt{load\_dataset} function fetches a preloaded dataset.
            \item The command \texttt{sns.boxplot()} creates a box plot to compare total bills across different days.
        \end{itemize}
    \end{block}
\end{frame}

\begin{frame}[fragile]
    \frametitle{What is Seaborn? - Key Points}
    \begin{block}{Key Points to Emphasize}
        \begin{itemize}
            \item Seaborn simplifies the creation of complex visualizations.
            \item Enhances data visualization aesthetics with better integration into the Pandas ecosystem.
            \item Its ease of use and strong statistical functionality make it ideal for data analysis and exploration.
        \end{itemize}
        \end{block}
        As we continue, we will explore specific statistical plots using Seaborn to enhance our analytical capabilities!
\end{frame}

\begin{frame}[fragile]
    \frametitle{Creating Statistical Plots with Seaborn}
    \begin{block}{Overview of Statistical Plots}
        Statistical plots are vital for visualizing data distributions, relationships, and comparisons. 
        Seaborn, built on top of Matplotlib, excels in creating aesthetically pleasing and informative statistical graphics with minimal coding effort.
    \end{block}
\end{frame}

\begin{frame}[fragile]
    \frametitle{Types of Statistical Plots - Box Plots}
    \begin{itemize}
        \item \textbf{Definition}:
            Box plots represent the distribution of data based on five summary statistics: minimum, first quartile (Q1), median (Q2), third quartile (Q3), and maximum.
        \item \textbf{Uses}:
            Ideal for comparing distributions across different categories.
        \item \textbf{Key Points}:
            \begin{itemize}
                \item Visualization shows median and interquartile range.
                \item Outliers are indicated with points.
            \end{itemize}
    \end{itemize}
    
    \begin{block}{Code Example}
    \begin{lstlisting}[language=Python]
import seaborn as sns
import matplotlib.pyplot as plt
import pandas as pd

# Load dataset
tips = sns.load_dataset("tips")

# Create box plot
sns.boxplot(x="day", y="total_bill", data=tips)
plt.title("Box Plot of Total Bill by Day")
plt.show()
    \end{lstlisting}
    \end{block}
\end{frame}

\begin{frame}[fragile]
    \frametitle{Types of Statistical Plots - Violin and Pair Plots}
    \begin{itemize}
        \item \textbf{Violin Plots}:
            \begin{itemize}
                \item \textbf{Definition}: Combine the features of box plots and density plots.
                \item \textbf{Uses}: Useful for visualizing the density of the data at different values.
                \item \textbf{Key Points}:
                    \begin{itemize}
                        \item Displays the probability density of the data.
                        \item Helps to identify multimodal distributions.
                    \end{itemize}
                
                \begin{block}{Code Example}
                \begin{lstlisting}[language=Python]
sns.violinplot(x="day", y="total_bill", data=tips)
plt.title("Violin Plot of Total Bill by Day")
plt.show()
                \end{lstlisting}
                \end{block}
            \end{itemize}
        
        \item \textbf{Pair Plots}:
            \begin{itemize}
                \item \textbf{Definition}: Create a matrix of scatter plots for each pair of variables.
                \item \textbf{Uses}: Effective for understanding relationships between multiple variables.
                \item \textbf{Key Points}:
                    \begin{itemize}
                        \item Color coding by category (e.g., "sex").
                        \item Useful for preliminary data analysis in EDA.
                    \end{itemize}
                
                \begin{block}{Code Example}
                \begin{lstlisting}[language=Python]
sns.pairplot(tips, hue="sex")
plt.title("Pair Plot of Tips Dataset")
plt.show()
                \end{lstlisting}
                \end{block}
            \end{itemize}
    \end{itemize}
\end{frame}

\begin{frame}[fragile]
    \frametitle{Conclusion}
    \begin{block}{Summary}
        Seaborn simplifies the creation of statistical plots, enabling clearer insights into data through intuitive visualizations. 
        Each plot type serves a unique purpose, making it easier to explore data distributions and relationships efficiently.
    \end{block}
    
    \begin{alertblock}{Tip}
        Experiment with different parameters and options in Seaborn to refine your visualizations and adjust as per your data's story!
    \end{alertblock}
\end{frame}

\begin{frame}[t]
  \frametitle{Comparing Matplotlib and Seaborn - Introduction}
  \begin{block}{Introduction}
    Data visualization is a crucial step in data analysis. Python's two most popular libraries for creating visualizations are \textbf{Matplotlib} and \textbf{Seaborn}. While both can produce high-quality plots, they serve different purposes and possess unique features.
  \end{block}
\end{frame}

\begin{frame}[t]
  \frametitle{Comparing Matplotlib and Seaborn - Key Differences}
  \begin{itemize}
    \item \textbf{Purpose and Philosophy}
      \begin{itemize}
        \item \textbf{Matplotlib:} Comprehensive library for static, animated, and interactive plots; high level of customization but more code required.
        \item \textbf{Seaborn:} Built on Matplotlib, simplifies creation of attractive statistical graphics; focuses on statistical data visualization.
      \end{itemize}
    \item \textbf{Syntax and Ease of Use}
      \begin{itemize}
        \item \textbf{Matplotlib:} More detailed code, steep learning curve for complex plots.
        \item \textbf{Seaborn:} High-level interface, user-friendly for statistical plots.
      \end{itemize}
  \end{itemize}
\end{frame}

\begin{frame}[fragile]
  \frametitle{Comparing Matplotlib and Seaborn - Code Examples}
  \begin{block}{Example Code: Matplotlib}
    \begin{lstlisting}[language=Python]
import matplotlib.pyplot as plt
plt.plot(x, y)
plt.title('Matplotlib Plot')
plt.show()
    \end{lstlisting}
  \end{block}
  
  \begin{block}{Example Code: Seaborn}
    \begin{lstlisting}[language=Python]
import seaborn as sns
sns.scatterplot(data=df, x='column_x', y='column_y')
plt.title('Seaborn Scatter Plot')
plt.show()
    \end{lstlisting}
  \end{block}
\end{frame}

\begin{frame}[t]
  \frametitle{Comparing Matplotlib and Seaborn - Aesthetic Appeal and Statistical Plots}
  \begin{itemize}
    \item \textbf{Aesthetic Appeal}
      \begin{itemize}
        \item \textbf{Matplotlib:} Full control over style but requires manual adjustments.
        \item \textbf{Seaborn:} Beautiful default themes and color palettes, easy for visually appealing graphs.
      \end{itemize}
    \item \textbf{Statistical Plots}
      \begin{itemize}
        \item \textbf{Matplotlib:} Good for general plotting, needs extra libraries for statistical visualizations.
        \item \textbf{Seaborn:} Designed for statistical plots with built-in functions for complex techniques.
      \end{itemize}
  \end{itemize}
\end{frame}

\begin{frame}[t]
  \frametitle{When to Use Each Library}
  \begin{itemize}
    \item \textbf{Use Matplotlib when:}
      \begin{itemize}
        \item Precise control over plot appearance is needed.
        \item Creating simple plots or custom visualizations.
        \item Working with animations or 3D plots.
      \end{itemize}
    \item \textbf{Use Seaborn when:}
      \begin{itemize}
        \item Easily visualizing statistical relationships is desired.
        \item Sophisticated plots with minimal coding are required.
        \item Conducting exploratory data analysis (EDA) with datasets.
      \end{itemize}
\end{itemize}
\end{frame}

\begin{frame}[t]
  \frametitle{Summary}
  Understanding the distinctions between Matplotlib and Seaborn is essential for effective data visualization. 
  \begin{itemize}
    \item Use Matplotlib for intricate, customizable plots.
    \item Leverage Seaborn for statistical insights with appealing aesthetics.
  \end{itemize}
  \begin{block}{Reminder}
    In data visualization, clarity and insight are paramount. Choose the tool that best fits the story your data needs to tell!
  \end{block}
\end{frame}

\begin{frame}[fragile]
  \frametitle{Practical Examples in Data Visualization Techniques}
  
  \begin{block}{Introduction to Data Visualization}
    Data visualization is an essential skill that transforms raw data into a visual format, allowing for easier interpretation and analysis. Effective data visualization conveys insights clearly and efficiently, making information accessible to both technical and non-technical audiences.
  \end{block}
\end{frame}

\begin{frame}[fragile]
  \frametitle{Best Practices for Effective Data Visualization}
  
  \begin{enumerate}
    \item \textbf{Clear Purpose}: Define the story you want to tell with your data.
    \item \textbf{Appropriate Charts}:
      \begin{itemize}
        \item \textbf{Bar Charts}: Great for comparing quantities across categories.
        \item \textbf{Line Graphs}: Ideal for showing trends over time.
        \item \textbf{Pie Charts}: Useful for illustrating proportions, though often overused.
        \item \textbf{Scatter Plots}: Best for showing relationships between two continuous variables.
      \end{itemize}
    \item \textbf{Labeling and Legends}: Include titles, axis labels, and legends for quick interpretation.
    \item \textbf{Color Use}: Stick to a coherent color palette and be mindful of colorblind-friendly options.
    \item \textbf{Avoid Clutter}: Simplify visuals by removing unnecessary elements for clarity.
  \end{enumerate}
\end{frame}

\begin{frame}[fragile]
  \frametitle{Real-World Examples of Data Visualization}

  \textbf{Example 1: Sales Performance Dashboard}
  \begin{itemize}
    \item A dashboard illustrating monthly sales by product category can include:
    \begin{itemize}
      \item A \textbf{bar chart} for total sales per category.
      \item A \textbf{line graph} showing sales growth over the last year.
    \end{itemize}
  \end{itemize}

  \begin{lstlisting}[language=Python]
  import matplotlib.pyplot as plt

  categories = ['Electronics', 'Furniture', 'Clothing']
  sales = [15000, 12000, 30000]

  plt.bar(categories, sales, color=['blue', 'green', 'orange'])
  plt.title('Monthly Sales by Category')
  plt.xlabel('Product Categories')
  plt.ylabel('Sales in USD')
  plt.show()
  \end{lstlisting}

\end{frame}

\begin{frame}[fragile]
  \frametitle{Real-World Examples of Data Visualization (Continued)}

  \textbf{Example 2: COVID-19 Trend Analysis}
  \begin{itemize}
    \item A \textbf{line chart} effectively depicts COVID-19 case trends over time:
    \begin{itemize}
      \item X-axis: Time (days or months)
      \item Y-axis: Number of cases reported
      \item Use different colored lines for different regions or variants.
    \end{itemize}
  \end{itemize}

  \begin{lstlisting}[language=Python]
  import matplotlib.pyplot as plt

  days = ['Jan', 'Feb', 'Mar', 'Apr', 'May']
  cases = [100, 900, 3000, 5000, 7000]

  plt.plot(days, cases, marker='o')
  plt.title('COVID-19 Cases Over Time')
  plt.xlabel('Months')
  plt.ylabel('Number of Cases')
  plt.grid()
  plt.show()
  \end{lstlisting}
  
\end{frame}

\begin{frame}[fragile]
  \frametitle{Real-World Examples of Data Visualization (Continued)}

  \textbf{Example 3: Customer Segmentation via Scatter Plot}
  \begin{itemize}
    \item Visualizing customer data based on income and spending score:
    \begin{itemize}
      \item X-axis: Income
      \item Y-axis: Spending Score
    \end{itemize}
  \end{itemize}

  \begin{lstlisting}[language=Python]
  import matplotlib.pyplot as plt

  income = [15, 30, 45, 60, 75]
  spending_score = [55, 70, 30, 80, 60]

  plt.scatter(income, spending_score, color='purple')
  plt.title('Customer Segmentation')
  plt.xlabel('Annual Income (K$)')
  plt.ylabel('Spending Score (1-100)')
  plt.show()
  \end{lstlisting}
  
\end{frame}

\begin{frame}[fragile]
  \frametitle{Conclusion}
  By adhering to best practices and employing the appropriate visualization techniques, we can turn complex datasets into insightful visual narratives. Keep experimenting with styles and tools, like Matplotlib and Seaborn, to enhance data presentation skills!
\end{frame}

\begin{frame}
    \frametitle{Hands-On Exercise}
    \textbf{Interactive Session: Create Your Own Visualizations using Matplotlib and Seaborn}

    \begin{block}{Learning Objectives}
        \begin{itemize}
            \item Understand the basics of Matplotlib and Seaborn for data visualization.
            \item Create various types of visualizations to represent datasets effectively.
            \item Enhance your ability to choose appropriate visuals based on data characteristics.
        \end{itemize}
    \end{block}
\end{frame}

\begin{frame}
    \frametitle{Requirements for the Exercise}
    \begin{block}{What You'll Need}
        \begin{itemize}
            \item \textbf{Python Environment:} Ensure you have Jupyter Notebook or any IDE set up.
            \item \textbf{Libraries:} Make sure you have Matplotlib and Seaborn installed. You can install these with the following command:
            \begin{lstlisting}[language=bash]
pip install matplotlib seaborn
            \end{lstlisting}
        \end{itemize}
    \end{block}
\end{frame}

\begin{frame}[fragile]
    \frametitle{Key Concepts}
    \begin{itemize}
        \item \textbf{Matplotlib:} A versatile plotting library for creating static, animated, and interactive visualizations in Python.
        \item \textbf{Seaborn:} Built on top of Matplotlib, provides a high-level interface for drawing attractive statistical graphics.
    \end{itemize}
\end{frame}

\begin{frame}[fragile]
    \frametitle{Exercise Instructions}
    \begin{enumerate}
        \item \textbf{Load Your Dataset:}
        \begin{lstlisting}[language=python]
import pandas as pd
import seaborn as sns
import matplotlib.pyplot as plt

# Load sample dataset
data = sns.load_dataset('tips')  # Example dataset
        \end{lstlisting}

        \item \textbf{Basic Visualization (Bar Plot):}
        \begin{lstlisting}[language=python]
sns.barplot(x='day', y='total_bill', data=data)
plt.title('Total Bill Amount by Day')
plt.show()
        \end{lstlisting}

        \item \textbf{Enhance Your Plot (Style and Colors):}
        \begin{lstlisting}[language=python]
sns.set(style="whitegrid")
sns.barplot(x='day', y='total_bill', data=data, palette='pastel')
plt.title('Total Bill Amount by Day with Enhanced Style')
plt.show()
        \end{lstlisting}
    \end{enumerate}
\end{frame}

\begin{frame}[fragile]
    \frametitle{Exploring Further Visualizations}
    \begin{enumerate}[resume]
        \item \textbf{Create a Heatmap:}
        \begin{lstlisting}[language=python]
correlation_matrix = data.corr()
sns.heatmap(correlation_matrix, annot=True, cmap='coolwarm')
plt.title('Correlation Heatmap')
plt.show()
        \end{lstlisting}

        \item \textbf{Group Challenge:}
        Collaborate with a partner to explore different visualizations like Scatter Plots, Box Plots, or Pair Plots. Share your findings and discuss which visualizations work best for your data!
    \end{enumerate}
\end{frame}

\begin{frame}
    \frametitle{Key Points to Emphasize}
    \begin{itemize}
        \item \textbf{Choosing the right visualization:} Understand the story behind your data to select an appropriate visualization method.
        \item \textbf{Aesthetics matter:} A well-designed plot not only conveys information but also engages the audience.
        \item \textbf{Iterate and Experiment:} Don't hesitate to tweak parameters and styles to see how they affect your visuals.
    \end{itemize}
\end{frame}

\begin{frame}
    \frametitle{Wrap-Up}
    In this exercise, you have enhanced your skills in using Matplotlib and Seaborn to create informative and visually appealing graphics. Keep practicing, and remember: the best way to master data visualization is through continuous exploration and experimentation!

    \textbf{Next Steps:} Now that you have hands-on experience, let’s move to the \textbf{Summary and Best Practices} section, where we will consolidate what we’ve learned.
\end{frame}

\begin{frame}[fragile]
    \frametitle{Summary and Best Practices - Introduction}
    \begin{block}{Introduction to Data Visualization}
        Data visualization is the graphical representation of information and data. It employs visual elements like charts, graphs, and maps to provide an accessible way to understand trends, outliers, and patterns in data.
    \end{block}
\end{frame}

\begin{frame}[fragile]
    \frametitle{Summary and Best Practices - Key Points}
    \begin{block}{Key Points to Remember}
        \begin{enumerate}
            \item \textbf{Purpose of Data Visualization:}
                \begin{itemize}
                    \item To convey complex data clearly and effectively.
                    \item To provide insights that inform decisions.
                \end{itemize}
            \item \textbf{Types of Data Visualizations:}
                \begin{itemize}
                    \item Bar Charts: Great for comparing quantities across categories.
                    \item Line Charts: Ideal for showing trends over time.
                    \item Pie Charts: Useful for illustrating proportions; use sparingly.
                    \item Scatter Plots: Excellent for showing relationships between two numerical variables.
                \end{itemize}
            \item \textbf{Interactivity:}
                \begin{itemize}
                    \item Interactive visualizations engage users through filtering, zooming, and tooltips.
                \end{itemize}
        \end{enumerate}
    \end{block}
\end{frame}

\begin{frame}[fragile]
    \frametitle{Summary and Best Practices - Best Practices}
    \begin{block}{Best Practices for Effective Data Visualization}
        \begin{enumerate}
            \item Know Your Audience: Tailor visuals to the audience's knowledge and preferences.
            \item Tell a Story: Structure your visuals to guide viewers through the data.
            \item Keep it Simple: Avoid clutter and limit elements to prevent overwhelming viewers.
            \item Use Appropriate Scales: Choose scales carefully to accurately represent data.
            \item Choose Colors Wisely: Use a consistent color palette and limit the number of colors.
            \item Label Clearly: Ensure all axes and legends are well-labeled for clarity.
            \item Highlight Key Insights: Use annotations to draw attention to important data parts.
        \end{enumerate}
    \end{block}
\end{frame}

\begin{frame}[fragile]
    \frametitle{Summary and Best Practices - Resources}
    \begin{block}{Resources for Further Learning}
        \begin{itemize}
            \item \textbf{Books:}
                \begin{itemize}
                    \item "The Visual Display of Quantitative Information" by Edward Tufte
                    \item "Storytelling with Data" by Cole Nussbaumer Knaflic
                \end{itemize}
            \item \textbf{Online Courses:}
                \begin{itemize}
                    \item Courses on platforms like Coursera and edX covering tools like Tableau, D3.js, and Python libraries.
                \end{itemize}
            \item \textbf{Documentation \& Blogs:}
                \begin{itemize}
                    \item Official documentation for tools like \textbf{Matplotlib} and \textbf{Seaborn}.
                    \item Websites like \textbf{DataVizWatch} for inspiration and examples.
                \end{itemize}
        \end{itemize}
    \end{block}
\end{frame}

\begin{frame}[fragile]
    \frametitle{Summary and Best Practices - Conclusion}
    \begin{block}{Conclusion}
        Effective data visualization is essential for interpreting data in today's data-driven world. By following the best practices outlined and utilizing the resources provided, you can enhance your visualization skills, ensuring that your data communicates clear and impactful stories.
    \end{block}
\end{frame}


\end{document}