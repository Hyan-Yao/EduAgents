\documentclass[aspectratio=169]{beamer}

% Theme and Color Setup
\usetheme{Madrid}
\usecolortheme{whale}
\useinnertheme{rectangles}
\useoutertheme{miniframes}

% Additional Packages
\usepackage[utf8]{inputenc}
\usepackage[T1]{fontenc}
\usepackage{graphicx}
\usepackage{booktabs}
\usepackage{listings}
\usepackage{amsmath}
\usepackage{amssymb}
\usepackage{xcolor}
\usepackage{tikz}
\usepackage{pgfplots}
\pgfplotsset{compat=1.18}
\usetikzlibrary{positioning}
\usepackage{hyperref}

% Custom Colors
\definecolor{myblue}{RGB}{31, 73, 125}
\definecolor{mygray}{RGB}{100, 100, 100}
\definecolor{mygreen}{RGB}{0, 128, 0}
\definecolor{myorange}{RGB}{230, 126, 34}
\definecolor{mycodebackground}{RGB}{245, 245, 245}

% Set Theme Colors
\setbeamercolor{structure}{fg=myblue}
\setbeamercolor{frametitle}{fg=white, bg=myblue}
\setbeamercolor{title}{fg=myblue}
\setbeamercolor{section in toc}{fg=myblue}
\setbeamercolor{item projected}{fg=white, bg=myblue}
\setbeamercolor{block title}{bg=myblue!20, fg=myblue}
\setbeamercolor{block body}{bg=myblue!10}
\setbeamercolor{alerted text}{fg=myorange}

% Set Fonts
\setbeamerfont{title}{size=\Large, series=\bfseries}
\setbeamerfont{frametitle}{size=\large, series=\bfseries}
\setbeamerfont{caption}{size=\small}
\setbeamerfont{footnote}{size=\tiny}

% Document Start
\begin{document}

\frame{\titlepage}

\begin{frame}[fragile]
    \frametitle{Introduction to Data Analysis Techniques}
    Data analysis is the systematic approach of inspecting, cleansing, transforming, and modeling data to discover useful information, inform conclusions, and support decision-making. 
\end{frame}

\begin{frame}[fragile]
    \frametitle{Importance of Data Analysis Techniques}
    \begin{enumerate}
        \item \textbf{Insight Discovery}: Allows uncovering patterns and trends that might not be immediately visible.
        \item \textbf{Improved Decision-Making}: Facilitates informed decisions that enhance operational efficiency.
        \item \textbf{Problem-Solving}: Identifies problems and develops solutions through evidence-based strategies.
        \item \textbf{Predictive Capabilities}: Supports forecasting future trends based on historical data.
    \end{enumerate}
\end{frame}

\begin{frame}[fragile]
    \frametitle{Types of Data Analysis Techniques}
    \begin{itemize}
        \item \textbf{Descriptive Analysis}: Summarizes past data to provide insights.
        \item \textbf{Inferential Analysis}: Makes inferences from a sample to a broader population.
        \item \textbf{Exploratory Data Analysis (EDA)}: Visual methods to analyze datasets for patterns.
        \item \textbf{Statistical Modelling}: Uses mathematical equations to predict outcomes.
    \end{itemize}
\end{frame}

\begin{frame}[fragile]
    \frametitle{Key Points and Conclusion}
    \begin{block}{Key Points to Emphasize}
        \begin{itemize}
            \item Right techniques turn raw data into actionable insights.
            \item EDA and statistical analysis are foundational for data work.
            \item Data-driven decisions outperform gut feelings.
        \end{itemize}
    \end{block}
    \newline
    Understanding data analysis techniques empowers you to derive meaningful insights and enhance strategic decisions.
\end{frame}

\begin{frame}[fragile]
    \frametitle{Example Code Snippet for Exploratory Data Analysis}
    \begin{lstlisting}[language=Python]
import pandas as pd
import seaborn as sns
import matplotlib.pyplot as plt

# Load a dataset
data = pd.read_csv('data.csv')

# Basic descriptive statistics
print(data.describe())

# Visualizing data distribution
sns.histplot(data['age'], kde=True)
plt.title('Distribution of Age')
plt.xlabel('Age')
plt.ylabel('Frequency')
plt.show()
    \end{lstlisting}
\end{frame}

\begin{frame}[fragile]
    \frametitle{Learning Objectives - Overview}
    By the end of this week, students will be able to:
    \begin{enumerate}
        \item \textbf{Understand Statistical Analysis Basics}: Grasp fundamental statistical concepts like measures of central tendency and dispersion.
        \item \textbf{Conduct Exploratory Data Analysis (EDA)}: Implement techniques to summarize key characteristics of datasets and visualize them effectively.
    \end{enumerate}
\end{frame}

\begin{frame}[fragile]
    \frametitle{Learning Objectives - Statistical Analysis Basics}
    \begin{block}{Measures of Central Tendency}
        \begin{itemize}
            \item \textbf{Mean}: The average value of a dataset.
                \begin{equation}
                \text{Mean} = \frac{\sum{x_i}}{n}
                \end{equation}
                \begin{itemize}
                    \item Example: For the dataset [2, 4, 6], \( \text{Mean} = \frac{2 + 4 + 6}{3} = 4 \)
                \end{itemize}
            \item \textbf{Median}: The middle value when the dataset is ordered.
                \begin{itemize}
                    \item Example: For [3, 1, 2], the ordered version is [1, 2, 3]. The Median is 2.
                \end{itemize}
            \item \textbf{Mode}: The value that occurs most frequently.
                \begin{itemize}
                    \item Example: In [1, 1, 2, 3], the Mode is 1.
                \end{itemize}
        \end{itemize}
    \end{block}
    
    \begin{block}{Measures of Dispersion}
        \begin{itemize}
            \item \textbf{Standard Deviation (SD)}: Indicates how spread out the data points are from the mean.
                \begin{equation}
                \text{SD} = \sqrt{\frac{\sum{(x_i - \text{Mean})^2}}{n}}
                \end{equation}
                \begin{itemize}
                    \item Example: For dataset [2, 4, 4, 4, 5, 5, 7, 9], calculate the mean first, then follow the formula to find SD.
                \end{itemize}
        \end{itemize}
    \end{block}
\end{frame}

\begin{frame}[fragile]
    \frametitle{Learning Objectives - Exploratory Data Analysis (EDA)}
    \begin{block}{What is EDA?}
        A critical step in data analysis where the main goal is to summarize the main characteristics of a dataset, often with visual methods.
    \end{block}
    
    \begin{block}{Key EDA Techniques}
        \begin{itemize}
            \item \textbf{Data Visualization}: Use of graphs and plots (e.g., histograms, boxplots, scatter plots) to explore patterns, trends, and outliers.
                \begin{itemize}
                    \item Example: A histogram can show the distribution of exam scores among students.
                \end{itemize}
            \item \textbf{Summary Statistics}: Generating insights from descriptive statistics such as counts, percentages, and range.
        \end{itemize}
    \end{block}
    
    \begin{block}{Key Points to Emphasize}
        \begin{itemize}
            \item Grasping basic statistical concepts is crucial for effective data analysis.
            \item EDA serves as the foundation for more complex analysis, guiding data cleaning and hypothesis formulation.
        \end{itemize}
    \end{block}
\end{frame}

\begin{frame}[fragile]
    \frametitle{Learning Objectives - Conclusion}
    By mastering these learning objectives, you will enhance your analytical skills and be better equipped to handle real-world data. Prepare for hands-on practice that will deepen your understanding and application of these techniques!
\end{frame}

\begin{frame}[fragile]
    \frametitle{Statistical Analysis Basics}
    \begin{block}{Introduction}
        Statistical analysis is essential in data analysis as it allows us to summarize and interpret data effectively. 
        This presentation will explore four key concepts: 
        \begin{itemize}
            \item Mean 
            \item Median 
            \item Mode 
            \item Standard Deviation
        \end{itemize}
    \end{block}
\end{frame}

\begin{frame}[fragile]
    \frametitle{Mean (Average)}
    \begin{block}{Definition}
        The mean is the sum of all numerical values in a dataset divided by the total number of values.
    \end{block}
    \begin{block}{Formula}
        \[
        \text{Mean} = \frac{\sum_{i=1}^{n} x_i}{n}
        \]
        where \( x_i \) represents each value in the dataset and \( n \) is the total number of values.
    \end{block}
    \begin{block}{Example}
        For the dataset: {4, 8, 6, 5, 3}
        \begin{itemize}
            \item Calculation: 
            \[
            \text{Mean} = \frac{4 + 8 + 6 + 5 + 3}{5} = \frac{26}{5} = 5.2
            \end{itemize}
        \end{block}
\end{frame}

\begin{frame}[fragile]
    \frametitle{Median}
    \begin{block}{Definition}
        The median is the middle value in a dataset when the values are arranged in ascending order.
    \end{block}
    \begin{block}{How to Find the Median}
        \begin{itemize}
            \item If \( n \) is odd, the median is the middle value.
            \item If \( n \) is even, the median is the average of the two middle values.
        \end{itemize}
    \end{block}
    \begin{block}{Example}
        \begin{itemize}
            \item For the dataset {3, 4, 5, 6, 8} (n=5 - odd): Median = 5.
            \item For {3, 4, 5, 6} (n=4 - even): Median = \(\frac{4 + 5}{2} = 4.5\).
        \end{itemize}
    \end{block}
\end{frame}

\begin{frame}[fragile]
    \frametitle{Mode and Standard Deviation}
    \begin{block}{Mode}
        \begin{itemize}
            \item The mode is the value that appears most frequently in a dataset.
            \item A dataset can be unimodal, bimodal, or multimodal.
        \end{itemize}
        \begin{block}{Example}
            \begin{itemize}
                \item In {1, 2, 2, 3, 4}, the mode is 2.
                \item In {1, 1, 2, 2, 3}, both 1 and 2 are modes (bimodal).
            \end{itemize}
        \end{block}
    \end{block}
    
    \begin{block}{Standard Deviation}
        \begin{itemize}
            \item Standard Deviation (SD) measures the amount of variation or dispersion in a set of values.
            \item Formula:
            \[
            \text{SD} = \sqrt{\frac{\sum_{i=1}^{n} (x_i - \text{Mean})^2}{n}}
            \]
        \end{itemize}
    \end{block}
\end{frame}

\begin{frame}[fragile]
    \frametitle{Calculating Standard Deviation}
    \begin{block}{Example}
        Using the dataset {4, 8, 6, 5, 3}, with Mean = 5.2:
        \begin{itemize}
            \item Calculate individual terms:
            \begin{itemize}
                \item \( (4 - 5.2)^2 = 1.44 \)
                \item \( (8 - 5.2)^2 = 7.84 \)
                \item \( (6 - 5.2)^2 = 0.64 \)
                \item \( (5 - 5.2)^2 = 0.04 \)
                \item \( (3 - 5.2)^2 = 4.84 \)
            \end{itemize}
            \item Sum = 14.8
            \item \[
            \text{SD} = \sqrt{\frac{14.8}{5}} \approx 1.72
            \]
        \end{itemize}
    \end{block}  
\end{frame}

\begin{frame}[fragile]
    \frametitle{Key Points and Engagement}
    \begin{block}{Key Points}
        \begin{itemize}
            \item Mean, Median, and Mode provide insights into central tendency.
            \item Standard Deviation informs about variability and consistency.
            \item Understanding these concepts is critical for advanced data analysis.
        \end{itemize}
    \end{block}

    \begin{block}{Engagement Tip}
        Consider creating a visual representation of these concepts, such as a graph, to show how mean, median, and mode differ, especially in skewed distributions.
    \end{block}
\end{frame}

\begin{frame}[fragile]
    \frametitle{Descriptive Statistics}
    \begin{block}{What are Descriptive Statistics?}
        Descriptive statistics are mathematical techniques used to summarize and describe the main features of a dataset. They provide a quick overview of the data characteristics, enabling better understanding and interpretation of the information at hand. Unlike inferential statistics, which draws conclusions from a sample to a larger population, descriptive statistics focus solely on describing the data collected.
    \end{block}
\end{frame}

\begin{frame}[fragile]
    \frametitle{Key Components of Descriptive Statistics}
    \begin{enumerate}
        \item \textbf{Measures of Central Tendency}:
        \begin{itemize}
            \item \textbf{Mean (Average)}:
            \begin{equation}
            \text{Mean} = \frac{\sum_{i=1}^n x_i}{n}
            \end{equation}
            \item \textbf{Median}: The middle value when the data points are arranged in ascending order.
            \item \textbf{Mode}: The most frequently occurring value in the dataset.
        \end{itemize}
        
        \item \textbf{Measures of Dispersion}:
        \begin{itemize}
            \item \textbf{Range}:
            \begin{equation}
            \text{Range} = \text{Max} - \text{Min}
            \end{equation}
            \item \textbf{Variance}:
            \begin{equation}
            \text{Variance} = \frac{\sum_{i=1}^n (x_i - \text{Mean})^2}{n}
            \end{equation}
            \item \textbf{Standard Deviation}:
            \begin{equation}
            \text{Standard Deviation} = \sqrt{\text{Variance}}
            \end{equation}
        \end{itemize}
        
        \item \textbf{Data Visualization}:
        \begin{itemize}
            \item Histograms
            \item Box Plots
            \item Scatter Plots
        \end{itemize}
    \end{enumerate}
\end{frame}

\begin{frame}[fragile]
    \frametitle{Example of Descriptive Statistics}
    Consider a dataset representing the test scores of 10 students: 
    \[
    78, 85, 92, 75, 88, 95, 80, 78, 85, 90
    \]
    
    \begin{itemize}
        \item \textbf{Mean}:
        \[
        \frac{78 + 85 + 92 + 75 + 88 + 95 + 80 + 78 + 85 + 90}{10} = 85.1
        \]
        
        \item \textbf{Median}: (Arranged order: 75, 78, 78, 80, 85, 85, 88, 90, 92, 95) 
        \begin{itemize}
            \item Median = (85 + 85) / 2 = 85
        \end{itemize}
        
        \item \textbf{Mode}: 78 and 85 (both appear twice)
        
        \item \textbf{Range}:
        \[
        95 - 75 = 20
        \]
        
        \item \textbf{Standard Deviation}: Calculate variance first, then take the square root.
    \end{itemize}
\end{frame}

\begin{frame}[fragile]
    \frametitle{Key Takeaways}
    \begin{itemize}
        \item Descriptive statistics help summarize large datasets effectively.
        \item Uses measures of central tendency and measures of dispersion to provide clear insights.
        \item Visual data presentations enhance understanding of data distribution and trends.
    \end{itemize}
    
    By using descriptive statistics, researchers and analysts can create a solid foundation for further data exploration and hypothesis testing, paving the way towards more complex analyses.
\end{frame}

\begin{frame}[fragile]
    \frametitle{Inferential Statistics - Introduction}
    \begin{block}{Introduction to Inferential Statistics}
        Inferential statistics allows us to draw conclusions and make predictions about a population based on a sample of data. Unlike descriptive statistics, which merely summarizes the data, inferential statistics goes a step further to help us make educated guesses about a larger group.
    \end{block}
\end{frame}

\begin{frame}[fragile]
    \frametitle{Inferential Statistics - Key Concepts}
    \begin{itemize}
        \item \textbf{Population vs. Sample}
            \begin{itemize}
                \item \textbf{Population}: The whole group you want to study (e.g., all students at a university).
                \item \textbf{Sample}: A subset of the population (e.g., 100 students at that university).
            \end{itemize}
        
        \item \textbf{Hypothesis Testing}
            \begin{itemize}
                \item A method used to determine if there is enough evidence to support a specific claim about a population.
                \item Steps:
                    \begin{enumerate}
                        \item \textbf{Formulate Hypotheses}:
                            \begin{itemize}
                                \item Null Hypothesis (H₀): Assumes no effect or difference.
                                \item Alternative Hypothesis (H₁): Assumes there is an effect or a difference.
                            \end{itemize}
                        \item \textbf{Choose a Significance Level (α)}: Commonly set at 0.05.
                        \item \textbf{Collect Data and Calculate a Test Statistic}: Use methodologies suitable for your data (e.g., t-tests, chi-square tests).
                        \item \textbf{Make a Decision}: If the p-value is less than α, reject the null hypothesis.
                    \end{enumerate}
            \end{itemize}
    \end{itemize}
\end{frame}

\begin{frame}[fragile]
    \frametitle{Inferential Statistics - Confidence Intervals}
    \begin{block}{Confidence Intervals}
        A range of values used to estimate the true parameter of a population.
        
        \begin{equation}
        \text{Confidence Interval} = \bar{x} \pm z^* \left(\frac{\sigma}{\sqrt{n}}\right)
        \end{equation}

        Where:
        \begin{itemize}
            \item $\bar{x}$ = sample mean  
            \item $z^*$ = z-score corresponding to the desired confidence level (e.g., 1.96 for 95\% confidence)  
            \item $\sigma$ = population standard deviation (or sample standard deviation if population is unknown)  
            \item $n$ = sample size  
        \end{itemize}
        
        \textbf{Example:} If a sample of 30 students has a mean score of 80 with a standard deviation of 10, the 95\% confidence interval would be calculated as:
        
        \( CI: 80 \pm 1.96 \left(\frac{10}{\sqrt{30}}\right) \)
    \end{block}
\end{frame}

\begin{frame}[fragile]
    \frametitle{Inferential Statistics - Key Takeaways}
    \begin{itemize}
        \item Inferential statistics enable predictions and conclusions beyond the data.
        \item Understanding hypothesis testing and confidence intervals is crucial for making data-driven decisions.
        \item Properly interpreting results minimizes errors in conclusions and enhances decision-making.
    \end{itemize}

    \begin{block}{Summary}
        By mastering these concepts, we can analyze data more effectively and communicate our findings with confidence.
    \end{block}
\end{frame}

\begin{frame}[fragile]
    \frametitle{Introduction to Exploratory Data Analysis (EDA)}
    \begin{block}{What is Exploratory Data Analysis (EDA)?}
        Exploratory Data Analysis (EDA) is a critical process in data analysis that involves summarizing and visualizing datasets to uncover underlying patterns, spot anomalies, and test assumptions.
        EDA is often the first step in the data analysis pipeline, guiding the subsequent modeling and analysis.
    \end{block}
\end{frame}

\begin{frame}[fragile]
    \frametitle{Significance of EDA in Data Analysis}
    \begin{enumerate}
        \item \textbf{Understanding Data Structure:} Comprehending features, types, distributions, and relationships among variables.
        \item \textbf{Spotting Anomalies:} Identifying outliers and unusual observations through visualization techniques like box plots.
        \item \textbf{Informing Data Cleaning:} Recognizing missing values and duplicates; understanding missing data patterns (e.g., Missing At Random).
        \item \textbf{Formulating Hypotheses:} Generating hypotheses about data relationships before formal testing.
        \item \textbf{Guiding Feature Selection:} Selecting variables for modeling based on relationships and distributions.
    \end{enumerate}
\end{frame}

\begin{frame}[fragile]
    \frametitle{Key Techniques in EDA}
    \begin{itemize}
        \item \textbf{Visualization:} Graphs such as histograms and scatter plots to visualize distributions and relationships.
        \item \textbf{Descriptive Statistics:} Computing measures like mean, median, mode, variance, and standard deviation.
        \item \textbf{Data Summarization:} Creating summary tables with counts and percentages for categorical data insights.
        \item \textbf{Distribution Analysis:} Investigating data distribution types to inform modeling decisions.
    \end{itemize}
\end{frame}

\begin{frame}[fragile]
    \frametitle{Example of EDA}
    Imagine you have a dataset of housing prices. EDA would involve:
    \begin{itemize}
        \item \textbf{Visualizing Prices:} Creating a histogram or box plot to assess the distribution of house prices.
        \item \textbf{Correlation Analysis:} Using a scatter plot to explore the relationship between house size and price.
        \item \textbf{Handling Missing Data:} Identifying houses with missing bedrooms and analyzing their price differences from the average.
    \end{itemize}
\end{frame}

\begin{frame}[fragile]
    \frametitle{Conclusion and Key Takeaways}
    \begin{block}{Conclusion}
        EDA is a foundational step in data analysis. It allows analysts to understand datasets and derive actionable insights, ensuring data readiness for formal analyses.
    \end{block}
    \begin{block}{Remember:}
        \begin{itemize}
            \item Always visualize your data before diving into statistical analysis.
            \item Utilize both graphical and numerical summaries.
            \item Allocate time for EDA to save time during later stages of analysis.
        \end{itemize}
    \end{block}
\end{frame}

\begin{frame}
  \frametitle{Techniques Used in EDA}
  \begin{block}{Overview of EDA Techniques}
    Exploratory Data Analysis (EDA) is a crucial step that allows data scientists to understand the underlying patterns within data before applying complex modeling techniques.
  \end{block}
  \begin{itemize}
    \item Data Visualization
    \item Data Cleaning
    \item Transformation
  \end{itemize}
\end{frame}

\begin{frame}[fragile]
  \frametitle{Data Visualization}
  \begin{block}{Definition}
    The graphical representation of data helps to identify patterns, trends, and outliers visually.
  \end{block}
  \begin{itemize}
    \item \textbf{Purpose:} To facilitate a deeper understanding of data distributions and relationships.
    \item \textbf{Common Tools:}
      \begin{itemize}
        \item Matplotlib
        \item Seaborn
        \item Plotly
      \end{itemize}
  \end{itemize}
  \begin{block}{Example}
    A histogram displays the distribution of a variable, revealing the frequency of data points in various intervals:
  \end{block}
  \begin{lstlisting}[language=Python]
import matplotlib.pyplot as plt

data = [10, 2, 3, 5, 7, 9, 2, 10, 12, 15]  # Sample data
plt.hist(data, bins=5, edgecolor='black')
plt.title('Histogram of Sample Data')
plt.xlabel('Value')
plt.ylabel('Frequency')
plt.show()
  \end{lstlisting}
\end{frame}

\begin{frame}
  \frametitle{Data Cleaning and Transformation}
  \begin{block}{Data Cleaning}
    \begin{itemize}
      \item \textbf{Definition:} The process of identifying and correcting inaccuracies and inconsistencies in data.
      \item \textbf{Importance:} High-quality data is crucial for reliable analysis and ensures valid results.
      \item \textbf{Key Steps:}
        \begin{itemize}
          \item Handle missing values (imputation or removal)
          \item Remove duplicate records
          \item Correct data types and formats
        \end{itemize}
    \end{itemize}
  \end{block}

  \begin{block}{Transformation}
    \begin{itemize}
      \item \textbf{Definition:} Adjusting the format or structure of data to make it suitable for analysis.
      \item \textbf{Common Transformations:}
        \begin{itemize}
          \item Normalization
          \item Encoding categorical variables
        \end{itemize}
      \item \textbf{Example:} Using Min-Max Normalization formula:
        \begin{equation}
        x' = \frac{x - \text{min}(X)}{\text{max}(X) - \text{min}(X)}
        \end{equation}
      \item \textbf{Implementation Example:}
    \end{itemize}
    
    \begin{lstlisting}[language=Python]
from sklearn.preprocessing import MinMaxScaler
import numpy as np

data = np.array([[100], [200], [300], [400]])
scaler = MinMaxScaler()
normalized_data = scaler.fit_transform(data)
print(normalized_data)
    \end{lstlisting}
  \end{block}
\end{frame}

\begin{frame}
  \frametitle{Key Points to Emphasize}
  \begin{itemize}
    \item EDA employs visualization, cleaning, and transformation to ensure a strong understanding of data.
    \item Each technique plays a crucial role in preparing data for further analysis and modeling.
    \item The quality of insights drawn in EDA directly impacts the success of predictive modeling efforts.
  \end{itemize}
  By mastering these techniques in EDA, you lay a solid foundation for making informed decisions based on data.
\end{frame}

\begin{frame}[fragile]
    \frametitle{Data Visualization Tools - Introduction}
    Data visualization is a crucial component of Exploratory Data Analysis (EDA) that helps analysts identify patterns, trends, and insights in data through visual representation. 
    We will discuss three popular tools used widely in the data science community:
    
    \begin{itemize}
        \item \textbf{Matplotlib}
        \item \textbf{Seaborn}
        \item \textbf{Plotly}
    \end{itemize}
    
    Each tool has unique strengths that cater to different visualization needs.
\end{frame}

\begin{frame}[fragile]
    \frametitle{Data Visualization Tools - Matplotlib}
    
    \begin{block}{Overview}
        Matplotlib is a versatile and powerful Python 2D plotting library.
    \end{block}
    
    \begin{itemize}
        \item Produces static, animated, and interactive visualizations.
        \item Publication-quality figures with extensive customization options (e.g., colors, fonts, sizes).
        \item Supports various backends for rendering in different formats (PDF, SVG, etc.).
    \end{itemize}
    
    \begin{block}{Basic Example Code}
        \begin{lstlisting}[language=Python]
import matplotlib.pyplot as plt

# Sample data
x = [1, 2, 3, 4]
y = [10, 15, 7, 10]

# Create a line plot
plt.plot(x, y, marker='o')
plt.title('Sample Line Plot')
plt.xlabel('X-axis Label')
plt.ylabel('Y-axis Label')
plt.show()
        \end{lstlisting}
    \end{block}
\end{frame}

\begin{frame}[fragile]
    \frametitle{Data Visualization Tools - Seaborn and Plotly}
    
    \begin{block}{Seaborn}
        Built on top of Matplotlib, Seaborn simplifies creating attractive statistical visualizations.
    \end{block}

    \begin{itemize}
        \item Built-in themes for styling.
        \item Functions to visualize complex data relationships (e.g., heat maps, pair plots).
        \item Direct integration with pandas dataframes.
    \end{itemize}
    
    \begin{block}{Basic Example Code}
        \begin{lstlisting}[language=Python]
import seaborn as sns
import matplotlib.pyplot as plt

# Load example dataset
tips = sns.load_dataset('tips')

# Create a scatter plot with a regression line
sns.regplot(x='total_bill', y='tip', data=tips)
plt.title('Total Bill vs Tip')
plt.show()
        \end{lstlisting}
    \end{block}
    
    \begin{block}{Plotly}
        Plotly excels at creating interactive and web-based visualizations.
    \end{block}

    \begin{itemize}
        \item Interactive plots (zoom, hover information).
        \item Integrated with web applications and dashboards.
        \item Supports 3D visualizations and geographical mapping.
    \end{itemize}

    \begin{block}{Basic Example Code}
        \begin{lstlisting}[language=Python]
import plotly.express as px

# Sample dataset
df = px.data.iris()

# Create an interactive scatter plot
fig = px.scatter(df, x='sepal_width', y='sepal_length', color='species', title='Iris Sepal Dimensions')
fig.show()
        \end{lstlisting}
    \end{block}
\end{frame}

\begin{frame}
    \frametitle{Key Points to Emphasize}
    \begin{itemize}
        \item Choosing the right tool depends on the analysis requirements:
        \begin{itemize}
            \item \textbf{Matplotlib} for detailed customization of static plots.
            \item \textbf{Seaborn} for statistical plotting with improved aesthetics.
            \item \textbf{Plotly} for creating interactive visualizations that enhance user engagement.
        \end{itemize}
        \item These tools enable data scientists to explore data visually, identifying anomalies and trends guiding further analysis.
        \item Incorporating these tools empowers clear communication of findings and facilitates a better understanding of complex datasets.
    \end{itemize}
\end{frame}

\begin{frame}[fragile]
    \frametitle{Case Study: EDA in Action}
    \begin{block}{What is Exploratory Data Analysis (EDA)?}
        EDA is a fundamental step in data analysis that involves:
        \begin{itemize}
            \item Visually exploring datasets 
            \item Discovering insights 
            \item Spotting anomalies 
            \item Testing hypotheses 
            \item Checking assumptions
        \end{itemize}
    \end{block}
\end{frame}

\begin{frame}[fragile]
    \frametitle{Case Study Example: Titanic Survival Dataset}
    \begin{block}{Context}
        The Titanic dataset is famous for data scientists, comprising:
        \begin{itemize}
            \item Information about passengers on the Titanic
            \item Survival status: survived or perished
            \item Rich basis for EDA exploration
        \end{itemize}
    \end{block}
\end{frame}

\begin{frame}[fragile]
    \frametitle{Step-by-Step EDA Process}
    \begin{enumerate}
        \item \textbf{Data Collection:}
            \begin{lstlisting}
            import pandas as pd
            df = pd.read_csv('titanic.csv')
            \end{lstlisting}
        
        \item \textbf{Data Overview:}
            \begin{lstlisting}
            print(df.head())
            \end{lstlisting}
        
        \item \textbf{Descriptive Statistics:}
            \begin{lstlisting}
            print(df.describe())
            \end{lstlisting}
        
        \item \textbf{Visualizations:}
            \begin{itemize}
                \item Distribution of Ages
                \begin{lstlisting}
                import seaborn as sns
                sns.histplot(df['Age'], bins=30)
                \end{lstlisting}
                
                \item Survival Rates by Gender
                \begin{lstlisting}
                sns.barplot(x='Sex', y='Survived', data=df)
                \end{lstlisting}
            \end{itemize}
        
        \item \textbf{Correlation Analysis:}
            \begin{lstlisting}
            sns.heatmap(df.corr(), annot=True)
            \end{lstlisting}
    \end{enumerate}
\end{frame}

\begin{frame}[fragile]
    \frametitle{Key Insights Gained}
    \begin{itemize}
        \item \textbf{Survival by Gender:} Approximately 74\% of females survived vs 20\% of males.
        \item \textbf{Age Influence:} Younger passengers (especially children) had a higher chance of survival.
        \item \textbf{Class Influence:} First-class passengers had significantly greater survival rates than those in second and third class.
    \end{itemize}
\end{frame}

\begin{frame}[fragile]
    \frametitle{Conclusion and Key Points}
    \begin{block}{Key Points to Remember}
        \begin{itemize}
            \item EDA is essential for understanding data.
            \item Visualizations are powerful tools for uncovering patterns.
            \item Insights from EDA guide further analysis and decisions.
        \end{itemize}
    \end{block}
    \begin{block}{Conclusion}
        Through EDA, critical insights from datasets can be revealed, supporting data-driven decisions.
    \end{block}
\end{frame}

\begin{frame}[fragile]
    \frametitle{Summary and Key Takeaways - Overview}
    \begin{block}{Overview of Data Analysis Techniques}
        This chapter focused on essential data analysis techniques that empower analysts to glean insights from datasets effectively. We explored various methodologies, which include:
    \end{block}
\end{frame}

\begin{frame}[fragile]
    \frametitle{Summary and Key Takeaways - Techniques}
    \begin{enumerate}
        \item \textbf{Exploratory Data Analysis (EDA)}
            \begin{itemize}
                \item \textbf{Definition:} Analyzing datasets to summarize main characteristics, often using visual methods.
                \item \textbf{Importance:} A preliminary step to discover patterns and test hypotheses.
                \item \textbf{Example:} EDA revealed trends in customer purchasing behaviors, informing marketing strategies.
            \end{itemize}

        \item \textbf{Statistical Summary Techniques}
            \begin{itemize}
                \item \textbf{Key Concepts:} Measures of central tendency and variability.
                \item \textbf{Application:} Overview of data characteristics, essential for understanding distribution.
                \item \textbf{Example:} Assessing sales performance using mean and variance.
            \end{itemize}

        \item \textbf{Data Visualization}
            \begin{itemize}
                \item \textbf{Definition:} Graphical representation of information and data.
                \item \textbf{Importance:} Simplifies complex findings and enhances understanding.
                \item \textbf{Example:} Scatter plots showing correlation between advertising spend and sales.
            \end{itemize}
    \end{enumerate}
\end{frame}

\begin{frame}[fragile]
    \frametitle{Summary and Key Takeaways - Conclusion and Key Points}
    \begin{enumerate}
        \item \textbf{Hypothesis Testing}
            \begin{itemize}
                \item \textbf{Definition:} Statistical method to determine evidence against a null hypothesis.
                \item \textbf{Importance:} Helps identify statistically significant findings.
                \item \textbf{Example:} Testing if a new strategy increases sales compared to traditional methods.
            \end{itemize}

        \item \textbf{Correlation and Causation}
            \begin{itemize}
                \item \textbf{Definition:} Correlation indicates a relationship; causation indicates direct effects.
                \item \textbf{Key Point:} "Correlation does not imply causation."
            \end{itemize}

        \item \textbf{Data Cleaning and Preprocessing}
            \begin{itemize}
                \item \textbf{Significance:} Prepares raw data for analysis.
                \item \textbf{Techniques:} Imputation for missing values, normalization for scaling.
                \item \textbf{Example:} Cleaning sales data ensures accurate insights.
            \end{itemize}
    \end{enumerate}

    \begin{block}{Key Takeaways}
        - EDA is foundational in identifying patterns.\\
        - Statistical summaries provide insights into data properties.\\
        - Visualizations enhance understanding.\\
        - Recognizing correlation vs. causation is critical.\\
        - Thorough data cleaning ensures integrity.
    \end{block}

    \begin{block}{Conclusion}
        Understanding these techniques is crucial for informed decision-making in data science and analytics.
    \end{block}
\end{frame}

\begin{frame}[fragile]
  \frametitle{Questions and Discussion - Introduction}
  As we conclude Week 4's exploration of Data Analysis Techniques, this slide is dedicated to fostering an interactive discussion and reflecting on the concepts we've covered. Engaging in dialogue is crucial to deepen our understanding and address any uncertainties.
\end{frame}

\begin{frame}[fragile]
  \frametitle{Questions and Topics for Discussion}
  \begin{enumerate}
    \item \textbf{Understanding Data Analysis Techniques}
      \begin{itemize}
        \item What specific techniques resonated with you the most? 
        \item How can these techniques be applied in real-world scenarios?
      \end{itemize}
      
    \item \textbf{Importance of Data Visualization}
      \begin{itemize}
        \item What are some key benefits of visualizing data, and how does it enhance analytical understanding? 
        \item Can you think of a situation where poor data visualization led to misinterpretation?
      \end{itemize}

    \item \textbf{Challenges in Data Interpretation}
      \begin{itemize}
        \item Have you encountered challenges in analyzing or interpreting data? 
        \item What strategies can be employed to overcome these challenges?
      \end{itemize}

    \item \textbf{Ethics in Data Analysis}
      \begin{itemize}
        \item Why is it vital to consider ethical implications in data analysis?
        \item Can anyone share an example of ethical dilemmas in data handling?
      \end{itemize}
  \end{enumerate}
\end{frame}

\begin{frame}[fragile]
  \frametitle{Examples and Key Points}
  \begin{block}{Examples to Illustrate Discussions}
    \begin{itemize}
      \item \textbf{Example of Data Visualization}: Imagine two bar graphs showing sales data. One graph uses vibrant colors and clear labels, while the other uses dull colors and lacks a legend. Which graph would be more effective in conveying the sales trends? 
      \item \textbf{Contextualizing Challenges}: Reflect on a recent project where data misinterpretation occurred. How did this impact the decision-making process and what could have been improved?
    \end{itemize}
  \end{block}

  \begin{block}{Key Points to Emphasize}
    \begin{itemize}
      \item Encourage participation by creating a safe environment for questions and sharing experiences.
      \item Emphasize the importance of critical thinking and adaptability when applying various data analysis techniques.
      \item Highlight that discussions can lead to a better understanding of complex topics and innovative ideas.
    \end{itemize}
  \end{block}
\end{frame}

\begin{frame}[fragile]
  \frametitle{Open Floor for Questions}
  \begin{itemize}
    \item Use this time to ask questions or share insights on any related topics.
    \item Reflect on how your understanding has evolved through this week's content and what areas you feel need further clarification.
  \end{itemize}
\end{frame}

\begin{frame}[fragile]
  \frametitle{Conclusion}
  Engaging in dialogue enriches our learning experience. Your contributions are valuable in shaping a multidimensional understanding of data analysis techniques. Let’s open the floor for questions, reflections, and insightful discussions!
\end{frame}


\end{document}