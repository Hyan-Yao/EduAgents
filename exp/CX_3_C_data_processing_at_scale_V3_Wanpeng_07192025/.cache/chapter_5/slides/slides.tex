\documentclass[aspectratio=169]{beamer}

% Theme and Color Setup
\usetheme{Madrid}
\usecolortheme{whale}
\useinnertheme{rectangles}
\useoutertheme{miniframes}

% Additional Packages
\usepackage[utf8]{inputenc}
\usepackage[T1]{fontenc}
\usepackage{graphicx}
\usepackage{booktabs}
\usepackage{listings}
\usepackage{amsmath}
\usepackage{amssymb}
\usepackage{xcolor}
\usepackage{tikz}
\usepackage{pgfplots}
\pgfplotsset{compat=1.18}
\usetikzlibrary{positioning}
\usepackage{hyperref}

% Custom Colors
\definecolor{myblue}{RGB}{31, 73, 125}
\definecolor{mygray}{RGB}{100, 100, 100}
\definecolor{mygreen}{RGB}{0, 128, 0}
\definecolor{myorange}{RGB}{230, 126, 34}
\definecolor{mycodebackground}{RGB}{245, 245, 245}

% Set Theme Colors
\setbeamercolor{structure}{fg=myblue}
\setbeamercolor{frametitle}{fg=white, bg=myblue}
\setbeamercolor{title}{fg=myblue}
\setbeamercolor{section in toc}{fg=myblue}
\setbeamercolor{item projected}{fg=white, bg=myblue}
\setbeamercolor{block title}{bg=myblue!20, fg=myblue}
\setbeamercolor{block body}{bg=myblue!10}
\setbeamercolor{alerted text}{fg=myorange}

% Set Fonts
\setbeamerfont{title}{size=\Large, series=\bfseries}
\setbeamerfont{frametitle}{size=\large, series=\bfseries}
\setbeamerfont{caption}{size=\small}
\setbeamerfont{footnote}{size=\tiny}

% Footer and Navigation Setup
\setbeamertemplate{footline}{
  \leavevmode%
  \hbox{%
  \begin{beamercolorbox}[wd=.3\paperwidth,ht=2.25ex,dp=1ex,center]{author in head/foot}%
    \usebeamerfont{author in head/foot}\insertshortauthor
  \end{beamercolorbox}%
  \begin{beamercolorbox}[wd=.5\paperwidth,ht=2.25ex,dp=1ex,center]{title in head/foot}%
    \usebeamerfont{title in head/foot}\insertshorttitle
  \end{beamercolorbox}%
  \begin{beamercolorbox}[wd=.2\paperwidth,ht=2.25ex,dp=1ex,center]{date in head/foot}%
    \usebeamerfont{date in head/foot}
    \insertframenumber{} / \inserttotalframenumber
  \end{beamercolorbox}}%
  \vskip0pt%
}

% Title Page Information
\title[SQL and Databases]{Week 5: Introduction to SQL and Databases}
\author[J. Smith]{John Smith, Ph.D.}
\institute[University Name]{
  Department of Computer Science\\
  University Name\\
  \vspace{0.3cm}
  Email: email@university.edu\\
  Website: www.university.edu
}
\date{\today}

% Document Start
\begin{document}

\frame{\titlepage}

\begin{frame}[fragile]
    \frametitle{Introduction to SQL and Databases - Overview}
    \begin{block}{Overview}
        SQL (Structured Query Language) is the standard programming language used for managing and manipulating relational databases. Databases are structured collections of data that allow for efficient data storage, retrieval, and management, which is crucial to modern data-centric applications.
    \end{block}
\end{frame}

\begin{frame}[fragile]
    \frametitle{What is SQL?}
    \begin{itemize}
        \item \textbf{Definition:} SQL is a domain-specific language designed for interacting with relational database management systems (RDBMS). 
        \item \textbf{Common SQL Tasks:}
        \begin{itemize}
            \item Data Manipulation: Inserting, updating, and deleting data.
            \item Data Querying: Retrieving specific data through queries.
            \item Data Definition: Defining the structure of the database (e.g., creating tables).
        \end{itemize}
    \end{itemize}
    
    \begin{block}{Example: Basic SQL Query}
        \begin{lstlisting}[language=SQL]
SELECT * FROM Customers WHERE Country = 'USA';
        \end{lstlisting}
        \textbf{Explanation:} This command retrieves all records from the "Customers" table where the country is 'USA'.
    \end{block}
\end{frame}

\begin{frame}[fragile]
    \frametitle{What is a Database?}
    \begin{itemize}
        \item \textbf{Definition:} A database is an organized collection of structured information, typically stored electronically in a computer system. It is managed by a Database Management System (DBMS).
        
        \item \textbf{Types of Databases:}
        \begin{itemize}
            \item Relational Databases: Organizes data into tables (e.g., MySQL, PostgreSQL).
            \item Non-relational Databases: Often document or key-value based (e.g., MongoDB, Redis).
        \end{itemize}
        
        \item \textbf{Key Characteristics of Relational Databases:}
        \begin{itemize}
            \item Data is stored in tables with defined relationships.
            \item Data integrity and consistency are maintained through constraints.
        \end{itemize}
    \end{itemize}
\end{frame}

\begin{frame}[fragile]
    \frametitle{Importance of SQL and Databases in Data Management}
    \begin{enumerate}
        \item \textbf{Efficiency:} SQL allows for efficient data access and management through powerful querying capabilities.
        \item \textbf{Consistency:} Ensures data integrity across different applications using the same database.
        \item \textbf{Scalability:} Databases can handle large volumes of data and support many users without performance loss.
        \item \textbf{Data Security:} Provides mechanisms to restrict unauthorized access and ensure data safety.
    \end{enumerate}
    
    \begin{block}{Key Points to Emphasize}
        \begin{itemize}
            \item SQL is essential for anyone looking to work with relational databases.
            \item Understanding database structures and types is foundational for effective data management.
            \item SQL's versatility makes it a valuable skill in various fields, including data science, web development, and business analytics.
        \end{itemize}
    \end{block}
\end{frame}

\begin{frame}[fragile]
    \frametitle{What is a Database?}
    \begin{block}{Definition of a Database}
        A \textbf{database} is an organized collection of structured information or data, typically stored electronically in a computer system. Databases can be easily accessed, managed, and updated, allowing for efficient data retrieval and manipulation.
    \end{block}
    \begin{block}{Importance of Databases}
        \begin{itemize}
            \item \textbf{Data Management}: Provides a systematic way to capture and retrieve data, ensuring accuracy and consistency.
            \item \textbf{Data Relationships}: Allows for connections and relationships among data entities, making complex data accessible.
        \end{itemize}
    \end{block}
\end{frame}

\begin{frame}[fragile]
    \frametitle{Types of Databases}
    \begin{enumerate}
        \item \textbf{Relational Databases (RDBMS)}
        \begin{itemize}
            \item \textbf{Definition}: Stores data in tables (relations) linked to each other with a fixed schema.
            \item \textbf{Key Characteristics}:
            \begin{itemize}
                \item Structured data organized into rows and columns.
                \item SQL (Structured Query Language) as standard for querying.
                \item ACID Properties: Ensures reliability (Atomicity, Consistency, Isolation, Durability).
            \end{itemize}
            \item \textbf{Examples}: MySQL, PostgreSQL, Oracle Database.
        \end{itemize}
    \end{enumerate}
\end{frame}

\begin{frame}[fragile]
    \frametitle{Relational Database Example}
    \begin{lstlisting}[language=SQL]
    -- Customers Table
    CREATE TABLE Customers (
        CustomerID INT,
        Name VARCHAR(50),
        Email VARCHAR(50)
    );

    -- Sample Data
    INSERT INTO Customers (CustomerID, Name, Email) VALUES
        (1, 'Alice', 'alice@example.com'),
        (2, 'Bob', 'bob@example.com');
    \end{lstlisting}
    \begin{block}{Non-Relational Databases (NoSQL)}
        \begin{itemize}
            \item \textbf{Definition}: Designed to store unstructured data with no fixed schema.
            \item \textbf{Key Characteristics}:
            \begin{itemize}
                \item Flexible schema for varying data formats.
                \item Scalability for large volumes of data.
                \item Eventual consistency.
            \end{itemize}
            \item \textbf{Examples}: MongoDB (Document Store), Redis (Key-Value Store), Cassandra (Wide-Column Store).
        \end{itemize}
    \end{block}
\end{frame}

\begin{frame}[fragile]
    \frametitle{Database Design Fundamentals - Overview}
    \begin{block}{Importance of Database Design}
        \begin{itemize}
            \item Minimizes redundancy and inconsistency.
            \item Enhances data integrity and security.
            \item Facilitates easier querying and reporting.
        \end{itemize}
    \end{block}
    
    \begin{block}{Key Concepts}
        \begin{itemize}
            \item \textbf{Data Model}: A framework for organizing data elements and relationships.
            \begin{itemize}
                \item \textbf{Entity-Relationship (ER) Model}: Illustrates relationships between entities.
                \item \textbf{Relational Model}: Represents data in tables; foreign keys define relationships.
            \end{itemize}
        \end{itemize}
    \end{block}
\end{frame}

\begin{frame}[fragile]
    \frametitle{Database Design Fundamentals - Normalization}
    \begin{block}{What is Normalization?}
        Normalization reduces redundancy and improves data integrity by organizing data into smaller, related tables.
    \end{block}
    
    \begin{block}{Normalization Forms}
        \begin{enumerate}
            \item \textbf{First Normal Form (1NF)}:
                \begin{itemize}
                    \item Attributes must contain atomic values.
                    \item Example: Instead of storing multiple phone numbers in one cell, create a separate table for them.
                \end{itemize}
            \item \textbf{Second Normal Form (2NF)}:
                \begin{itemize}
                    \item All non-key attributes must be fully functionally dependent on the primary key.
                    \item Example: Separate grades into another table if they depend only on courses.
                \end{itemize}
            \item \textbf{Third Normal Form (3NF)}:
                \begin{itemize}
                    \item No transitive dependencies; non-key attributes must not depend on other non-key attributes.
                    \item Example: Store major advisors in a separate table from students.
                \end{itemize}
        \end{enumerate}
    \end{block}
\end{frame}

\begin{frame}[fragile]
    \frametitle{Database Design Fundamentals - Schema}
    \begin{block}{What is a Schema?}
        A schema is a blueprint for organizing data within a database. It defines:
        \begin{itemize}
            \item \textbf{Tables}: Structures for storing data.
            \item \textbf{Fields/Columns}: Attributes of tables.
            \item \textbf{Relationships}: Connections via primary and foreign keys.
        \end{itemize}
    \end{block}
    
    \begin{block}{Example of a Simple Database Schema}
    \begin{lstlisting}
    Students Table:
    ---------------------------------
    | StudentID (PK) | Name       | MajorID (FK) |
    ---------------------------------
    | 1               | John Doe  | 101           |
    | 2               | Jane Smith| 102           |
    ---------------------------------
    
    Majors Table:
    --------------------------
    | MajorID (PK) | Major   |
    --------------------------
    | 101           | CS      |
    | 102           | Math    |
    --------------------------
    
    Courses Table:
    ---------------------------------
    | CourseID (PK) | CourseName | MajorID (FK) |
    ---------------------------------
    | 201            | Database    | 101          |
    | 202            | Calculus    | 102          |
    ---------------------------------
    \end{lstlisting}
    \end{block}
\end{frame}

\begin{frame}[fragile]
    \frametitle{Database Design Fundamentals - Key Points}
    \begin{itemize}
        \item \textbf{Effective Database Design}: Essential for maintaining performance, usability, and data integrity.
        \item \textbf{Normalization}: A structured technique to avoid redundancy and ensure proper data dependencies.
        \item \textbf{Schema Understanding}: Crucial for implementing a relational database, serving as a guide for data structure and relationships.
    \end{itemize}
\end{frame}

\begin{frame}[fragile]
    \frametitle{Database Design Fundamentals - Summary}
    Understanding and applying these database design fundamentals is crucial for creating effective databases that can accommodate growing data while maintaining accuracy and efficiency. In the next topic, we will discuss how SQL is employed to manipulate and interact with these database schemas effectively.
\end{frame}

\begin{frame}[fragile]
    \frametitle{Introduction to SQL}
    \begin{block}{Overview of Structured Query Language (SQL)}
        SQL (Structured Query Language) is a standardized programming language designed for managing and manipulating relational databases.
    \end{block}
\end{frame}

\begin{frame}[fragile]
    \frametitle{What is SQL?}
    \begin{itemize}
        \item SQL enables users to perform various operations such as querying data, updating records, and managing database structures.
        \item SQL is the cornerstone of database management systems (DBMS) and serves as the primary interface for interacting with databases.
    \end{itemize}
\end{frame}

\begin{frame}[fragile]
    \frametitle{Role of SQL in Database Management}
    \begin{itemize}
        \item SQL enables communication with the database to execute various tasks:
        \begin{itemize}
            \item Data retrieval (querying)
            \item Data manipulation (inserting, updating, and deleting)
            \item Database schema creation and modifications (DDL)
        \end{itemize}
    \end{itemize}
\end{frame}

\begin{frame}[fragile]
    \frametitle{SQL Components}
    \begin{enumerate}
        \item \textbf{DML (Data Manipulation Language)}
            \begin{itemize}
                \item Commands for manipulating data.
                \item Examples:
                \begin{itemize}
                    \item \texttt{SELECT}: query data
                    \item \texttt{INSERT}: add new records
                    \item \texttt{UPDATE}: modify existing records
                    \item \texttt{DELETE}: remove records
                \end{itemize}
            \end{itemize}
        \item \textbf{DDL (Data Definition Language)}
            \begin{itemize}
                \item Commands that define or alter the database structure.
                \item Examples:
                \begin{itemize}
                    \item \texttt{CREATE TABLE}: create a new table
                    \item \texttt{ALTER TABLE}: modify an existing table
                    \item \texttt{DROP TABLE}: delete a table
                \end{itemize}
            \end{itemize}
        \item \textbf{DCL (Data Control Language)}
            \begin{itemize}
                \item Commands for permissions and access controls.
                \item Examples:
                \begin{itemize}
                    \item \texttt{GRANT}: gives user access privileges
                    \item \texttt{REVOKE}: removes user access privileges
                \end{itemize}
            \end{itemize}
    \end{enumerate}
\end{frame}

\begin{frame}[fragile]
    \frametitle{Example SQL Command}
    \begin{lstlisting}[language=SQL]
SELECT first_name, last_name 
FROM employees 
WHERE department = 'Sales';
    \end{lstlisting}
    \begin{block}{Explanation}
        This command retrieves the first and last names of employees who work in the Sales department.
    \end{block}
\end{frame}

\begin{frame}[fragile]
    \frametitle{Key Points to Emphasize}
    \begin{itemize}
        \item SQL is essential for data management in various applications, from small-scale databases to enterprise systems.
        \item Proficiency in SQL allows developers, data analysts, and data scientists to derive meaningful insights from data.
        \item Understanding its components helps design and implement robust database solutions.
    \end{itemize}
\end{frame}

\begin{frame}[fragile]
    \frametitle{Conclusion}
    \begin{itemize}
        \item As we delve into this chapter, we will explore foundational SQL commands and their applications in managing and utilizing databases effectively.
    \end{itemize}
\end{frame}

\begin{frame}[fragile]
    \frametitle{Basic SQL Commands - Introduction}
    \begin{block}{Introduction}
        Structured Query Language (SQL) is essential for interacting with databases. In this slide, we will explore the fundamental SQL commands that are crucial for managing and manipulating data within a database: 
        \textbf{SELECT}, \textbf{INSERT}, \textbf{UPDATE}, and \textbf{DELETE}. 
        These commands are often referred to as the \textbf{CRUD} operations: Create, Read, Update, and Delete.
    \end{block}
\end{frame}

\begin{frame}[fragile]
    \frametitle{Basic SQL Commands - SELECT}
    \begin{block}{1. SELECT}
        The \textbf{SELECT} command is used to retrieve data from a database. It allows you to specify which columns you want to see from which table.
    \end{block}
    
    \begin{itemize}
        \item \textbf{Syntax:}
        \begin{lstlisting}
SELECT column1, column2 
FROM table_name 
WHERE condition;
        \end{lstlisting}
        
        \item \textbf{Example:}
        \begin{lstlisting}
SELECT name, age 
FROM employees 
WHERE department = 'Sales';
        \end{lstlisting}
        *This example retrieves the names and ages of employees who work in the Sales department.*
    \end{itemize}
\end{frame}

\begin{frame}[fragile]
    \frametitle{Basic SQL Commands - INSERT, UPDATE, DELETE}
    \begin{block}{2. INSERT}
        The \textbf{INSERT} command adds new records to a table.
    \end{block}
    
    \begin{itemize}
        \item \textbf{Syntax:}
        \begin{lstlisting}
INSERT INTO table_name (column1, column2) 
VALUES (value1, value2);
        \end{lstlisting}
        
        \item \textbf{Example:}
        \begin{lstlisting}
INSERT INTO employees (name, age, department) 
VALUES ('Alice', 30, 'HR');
        \end{lstlisting}
        *This adds a new employee named Alice, aged 30, to the HR department.*
    \end{itemize}

    \begin{block}{3. UPDATE}
        The \textbf{UPDATE} command modifies existing records.
    \end{block}
    
    \begin{itemize}
        \item \textbf{Syntax:}
        \begin{lstlisting}
UPDATE table_name 
SET column1 = value1, column2 = value2 
WHERE condition;
        \end{lstlisting}
        
        \item \textbf{Example:}
        \begin{lstlisting}
UPDATE employees 
SET age = 31 
WHERE name = 'Alice';
        \end{lstlisting}
        *This updates Alice's age to 31 in the employees table.*
    \end{itemize}
    
    \begin{block}{4. DELETE}
        The \textbf{DELETE} command removes records from a table.
    \end{block}
    
    \begin{itemize}
        \item \textbf{Syntax:}
        \begin{lstlisting}
DELETE FROM table_name 
WHERE condition;
        \end{lstlisting}
        
        \item \textbf{Example:}
        \begin{lstlisting}
DELETE FROM employees 
WHERE name = 'Alice';
        \end{lstlisting}
        *This deletes the employee named Alice from the employees table.*
    \end{itemize}
\end{frame}

\begin{frame}[fragile]
    \frametitle{Basic SQL Commands - Key Points and Conclusion}
    \begin{block}{Key Points}
        \begin{itemize}
            \item \textbf{SELECT:} Used for retrieving data.
            \item \textbf{INSERT:} Adds new entries.
            \item \textbf{UPDATE:} Changes existing data.
            \item \textbf{DELETE:} Removes data.
        \end{itemize}
        When using these commands, always be cautious, particularly with \textbf{UPDATE} and \textbf{DELETE}, to avoid unintended changes to your data.
    \end{block}
    
    \begin{block}{Conclusion}
        Understanding these basic SQL commands forms the foundation for effectively managing databases and performing data manipulation tasks. Mastering these commands will enable you to write powerful queries and fully harness SQL for data management.
    \end{block}
\end{frame}

\begin{frame}[fragile]
    \frametitle{Creating and Modifying Tables}
    
    \begin{block}{Introduction to Tables in SQL}
        \begin{itemize}
            \item \textbf{Tables}: A table is a collection of related data organized in rows and columns within a database.
            \item Each table consists of \textit{records} (rows) and \textit{fields} (columns).
        \end{itemize}
    \end{block}
\end{frame}

\begin{frame}[fragile]
    \frametitle{SQL Commands for Tables}
    
    Two primary SQL commands are used to manage tables:
    \begin{enumerate}
        \item \textbf{CREATE TABLE}: Used to create a new table in the database.
        \item \textbf{ALTER TABLE}: Used to modify an existing table.
    \end{enumerate}
\end{frame}

\begin{frame}[fragile]
    \frametitle{1. Creating a New Table}
    
    \textbf{Syntax}:
    \begin{lstlisting}[language=SQL]
CREATE TABLE table_name (
    column1_name column1_datatype constraints,
    column2_name column2_datatype constraints,
    ...
);
    \end{lstlisting}

    \textbf{Example}:
    Creating a table named \texttt{employees}:
    \begin{lstlisting}[language=SQL]
CREATE TABLE employees (
    employee_id INT PRIMARY KEY,
    first_name VARCHAR(50),
    last_name VARCHAR(50),
    hire_date DATE,
    salary DECIMAL(10, 2)
);
    \end{lstlisting}
    
    \textbf{Explanation}:
    \begin{itemize}
        \item \texttt{employee_id}: An integer that uniquely identifies an employee (primary key).
        \item \texttt{first_name}, \texttt{last_name}: Strings storing the employee's first and last names.
        \item \texttt{hire_date}: A date field for when the employee was hired.
        \item \texttt{salary}: A decimal field for the employee's salary.
    \end{itemize}
\end{frame}

\begin{frame}[fragile]
    \frametitle{2. Modifying an Existing Table}
    
    \textbf{Syntax}:
    \begin{lstlisting}[language=SQL]
ALTER TABLE table_name
ADD column_name column_datatype constraints;
    \end{lstlisting}

    \textbf{Example}: Adding a column \texttt{email} to the \texttt{employees} table:
    \begin{lstlisting}[language=SQL]
ALTER TABLE employees
ADD email VARCHAR(100);
    \end{lstlisting}
    
    \textbf{Removing a Column}:
    \begin{lstlisting}[language=SQL]
ALTER TABLE employees
DROP COLUMN email;
    \end{lstlisting}
    
    \textbf{Modifying a Column} (e.g., changing the data type of \texttt{salary}):
    \begin{lstlisting}[language=SQL]
ALTER TABLE employees
MODIFY salary FLOAT;
    \end{lstlisting}
\end{frame}

\begin{frame}[fragile]
    \frametitle{Key Points to Remember}
    
    \begin{itemize}
        \item \textbf{Data Types}: Common data types include \texttt{INT}, \texttt{VARCHAR}, \texttt{DATE}, \texttt{DECIMAL}, etc.
        \item \textbf{Constraints}: Used to enforce guidelines on the data in a table (e.g., \texttt{PRIMARY KEY}, \texttt{NOT NULL}, \texttt{UNIQUE}).
        \item \textbf{Modifications}: Always back up data before structural changes, as some changes (like dropping a column) can lead to data loss.
    \end{itemize}
\end{frame}

\begin{frame}[fragile]
    \frametitle{Querying Data - Introduction to SQL}
    SQL (Structured Query Language) is the standard language for interacting with relational databases, allowing you to retrieve and manipulate data.

    \begin{block}{Key Concept: Query}
        A SQL query is a request for data from one or more tables in a database. The results can be filtered, sorted, and displayed in various formats.
    \end{block}
\end{frame}

\begin{frame}[fragile]
    \frametitle{Querying Data - The SELECT Statement}
    The `SELECT` statement is central to SQL queries, specifying the fields to be retrieved.

    \begin{block}{Basic Syntax}
        \begin{lstlisting}
SELECT column1, column2
FROM table_name;
        \end{lstlisting}
    \end{block}

    \begin{itemize}
        \item \textbf{SELECT:} Indicates which columns to include in the result set.
        \item \textbf{FROM:} Specifies the table from which to retrieve data.
    \end{itemize}
\end{frame}

\begin{frame}[fragile]
    \frametitle{Querying Data - Filtering with the WHERE Clause}
    The `WHERE` clause filters results based on specific conditions that the rows must meet.

    \begin{block}{Basic Syntax with WHERE}
        \begin{lstlisting}
SELECT column1, column2
FROM table_name
WHERE condition;
        \end{lstlisting}
    \end{block}

    \begin{block}{Example}
        \begin{lstlisting}
SELECT Title, Author 
FROM Books 
WHERE Price < 20.00;
        \end{lstlisting}
    \end{block}

    \textbf{Explanation:} This query retrieves titles and authors of books priced below \$20.
\end{frame}

\begin{frame}[fragile]
    \frametitle{Querying Data - Additional Operators for Filtering}
    The `WHERE` clause can utilize various operators for complex conditions:
    
    \begin{itemize}
        \item \textbf{Comparison Operators:} =, != or <>, <, >, <=, >=
        \item \textbf{Logical Operators:}
            \begin{itemize}
                \item \textbf{AND:} Combines multiple conditions (both must be true).
                \item \textbf{OR:} At least one condition must be true.
                \item \textbf{NOT:} Negates a condition.
            \end{itemize}
    \end{itemize}

    \begin{block}{Example with Logical Operators}
        \begin{lstlisting}
SELECT Title, Author 
FROM Books 
WHERE Price < 20.00 AND Author = 'J.K. Rowling';
        \end{lstlisting}
    \end{block}

    This query finds all books authored by J.K. Rowling priced below \$20.
\end{frame}

\begin{frame}[fragile]
    \frametitle{Querying Data - Key Points}
    \begin{enumerate}
        \item Understand the basic syntax of the `SELECT` statement.
        \item Use the `WHERE` clause for precise data retrieval based on criteria.
        \item Combine conditions with logical operators for complex queries.
    \end{enumerate}
\end{frame}

\begin{frame}[fragile]
    \frametitle{Querying Data - Conclusion}
    Querying data with SQL allows extraction of valuable information from a database. The `SELECT` statement and the `WHERE` clause enable precise control over the data retrieved.

    This foundational knowledge sets the stage for advanced SQL concepts, including JOIN operations, in subsequent discussions.
\end{frame}

\begin{frame}
    \frametitle{Joins and Relationships}
    \begin{block}{Understanding Joins between Tables}
        In SQL, a \textbf{join} is used to combine records from two or more tables in a database based on related columns. Joins facilitate the retrieval of related data stored across multiple tables. The two primary types of joins are:
    \end{block}
    \begin{itemize}
        \item \textbf{INNER JOIN}
        \item \textbf{OUTER JOIN} (which includes LEFT, RIGHT, and FULL OUTER JOINs)
    \end{itemize}
\end{frame}

\begin{frame}[fragile]
    \frametitle{Joins and Relationships - INNER JOIN}
    \begin{block}{Definition}
        An \textbf{INNER JOIN} returns records that have matching values in both tables involved in the join.
    \end{block}
    \textbf{Syntax:}
    \begin{lstlisting}
SELECT columns
FROM table1
INNER JOIN table2
ON table1.common_column = table2.common_column;
    \end{lstlisting}

    \textbf{Example:} Consider two tables: \texttt{Students} and \texttt{Enrollments}.
    \begin{itemize}
        \item \texttt{Students Table}
        \begin{itemize}
            \item StudentID | Name
            \item 1 | Alice
            \item 2 | Bob
            \item 3 | Charlie
        \end{itemize}

        \item \texttt{Enrollments Table}
        \begin{itemize}
            \item EnrollmentID | StudentID | Course
            \item 101 | 1 | Math
            \item 102 | 2 | Science
            \item 103 | 2 | History
        \end{itemize}
    \end{itemize}
\end{frame}

\begin{frame}[fragile]
    \frametitle{Joins and Relationships - INNER JOIN Example}
    \textbf{Query:}
    \begin{lstlisting}
SELECT Students.Name, Enrollments.Course
FROM Students
INNER JOIN Enrollments
ON Students.StudentID = Enrollments.StudentID;
    \end{lstlisting}

    \textbf{Result:}
    \begin{itemize}
        \item Name | Course
        \item Alice | Math
        \item Bob | Science
        \item Bob | History
    \end{itemize}
\end{frame}

\begin{frame}[fragile]
    \frametitle{Joins and Relationships - OUTER JOIN}
    \begin{block}{Definition}
        An \textbf{OUTER JOIN} returns all records from one table and the matched records from the other table. If there is no match, NULL values are displayed for columns from the table that doesn't have a match.
    \end{block}

    \textbf{Types of OUTER JOIN:}
    \begin{itemize}
        \item \textbf{LEFT JOIN:} Returns all rows from the left table and matched rows from the right table.
        \item \textbf{RIGHT JOIN:} Returns all rows from the right table and matched rows from the left table.
        \item \textbf{FULL OUTER JOIN:} Returns all rows when there is a match in either left or right table records.
    \end{itemize}
    
    \textbf{LEFT JOIN Syntax:}
    \begin{lstlisting}
SELECT columns
FROM table1
LEFT JOIN table2
ON table1.common_column = table2.common_column;
    \end{lstlisting}
\end{frame}

\begin{frame}[fragile]
    \frametitle{Joins and Relationships - LEFT JOIN Example}
    \textbf{Example of LEFT JOIN Query:}
    \begin{lstlisting}
SELECT Students.Name, Enrollments.Course
FROM Students
LEFT JOIN Enrollments
ON Students.StudentID = Enrollments.StudentID;
    \end{lstlisting}

    \textbf{Result:}
    \begin{itemize}
        \item Name | Course
        \item Alice | Math
        \item Bob | Science
        \item Bob | History
        \item Charlie | NULL
    \end{itemize}

    \begin{block}{Key Points}
    \begin{itemize}
        \item Use \textbf{INNER JOIN} when you only want to see matched rows.
        \item Use \textbf{OUTER JOIN} to see all records from one table, regardless of matches.
        \item Joins are essential for efficient queries needing related data.
    \end{itemize}
    \end{block}
\end{frame}

\begin{frame}[fragile]
    \frametitle{Data Aggregation - Overview}
    \begin{block}{Understanding Data Aggregation in SQL}
        Data Aggregation is a fundamental concept in SQL that allows us to summarize and analyze large datasets efficiently. 
        Using SQL functions together with the \texttt{GROUP BY} clause enables us to derive meaningful insights.
    \end{block}
    
    \begin{itemize}
        \item Aggregation functions: SUM, COUNT, AVG.
        \item The importance of the \texttt{GROUP BY} clause.
        \item Transforming raw data into valuable information.
    \end{itemize}
\end{frame}

\begin{frame}[fragile]
    \frametitle{Data Aggregation - Key SQL Functions}
    \begin{enumerate}
        \item \textbf{SUM()}
        \begin{itemize}
            \item \textbf{Purpose}: Adds values in a specified column.
            \item \textbf{Example}:
            \begin{lstlisting}[language=SQL]
SELECT SUM(sales_amount) AS TotalSales 
FROM sales;
            \end{lstlisting}
            \item \textbf{Usage}: Used for total figures like total revenue.
        \end{itemize}
        
        \item \textbf{COUNT()}
        \begin{itemize}
            \item \textbf{Purpose}: Counts the number of rows.
            \item \textbf{Example}:
            \begin{lstlisting}[language=SQL]
SELECT COUNT(*) AS TotalRecords 
FROM customers;
            \end{lstlisting}
            \item \textbf{Usage}: Ideal for counting entries or records.
        \end{itemize}
        
        \item \textbf{AVG()}
        \begin{itemize}
            \item \textbf{Purpose}: Computes the average value.
            \item \textbf{Example}:
            \begin{lstlisting}[language=SQL]
SELECT AVG(rating) AS AverageRating 
FROM products;
            \end{lstlisting}
            \item \textbf{Usage}: Used for finding average metrics.
        \end{itemize}
    \end{enumerate}
\end{frame}

\begin{frame}[fragile]
    \frametitle{Data Aggregation - GROUP BY Clause}
    \begin{block}{Using GROUP BY Clause}
        The \texttt{GROUP BY} clause organizes rows that have the same values into summary rows.
    \end{block}
    
    \begin{itemize}
        \item \textbf{Structure}:
        \begin{lstlisting}[language=SQL]
SELECT column1, aggregate_function(column2) 
FROM table_name 
GROUP BY column1;
        \end{lstlisting}
        \item \textbf{Example}:
        To calculate total sales by each salesperson:
        \begin{lstlisting}[language=SQL]
SELECT salesperson_id, SUM(sales_amount) AS TotalSales 
FROM sales 
GROUP BY salesperson_id;
        \end{lstlisting}
        \item This query groups the records by \texttt{salesperson_id} and calculates total sales.
    \end{itemize}
\end{frame}

\begin{frame}[fragile]
    \frametitle{Introduction to Database Management Systems (DBMS)}
    \begin{block}{What is a DBMS?}
        A Database Management System (DBMS) is software that interacts with end users, applications, and the database itself to capture and analyze data. 
        It allows users to create, read, update, and delete data in a structured way. 
        DBMS serves as an intermediary between users and the database, ensuring data is managed efficiently and securely.
    \end{block}
\end{frame}

\begin{frame}[fragile]
    \frametitle{Key Functions of a DBMS}
    \begin{enumerate}
        \item \textbf{Data Definition}:
        \begin{itemize}
            \item Defines the structure of the database using a Data Definition Language (DDL).
            \item \texttt{CREATE TABLE} example:
            \begin{lstlisting}[language=SQL]
CREATE TABLE Employees (
    EmployeeID INT PRIMARY KEY,
    Name VARCHAR(100),
    Department VARCHAR(50),
    Salary DECIMAL(10, 2)
);
            \end{lstlisting}
        \end{itemize}
        
        \item \textbf{Data Manipulation}:
        \begin{itemize}
            \item Allows users to interact with the data using Data Manipulation Language (DML).
            \item \texttt{INSERT INTO} example:
            \begin{lstlisting}[language=SQL]
INSERT INTO Employees (EmployeeID, Name, Department, Salary)
VALUES (1, 'Alice Johnson', 'Sales', 50000.00);
            \end{lstlisting}
        \end{itemize}
    \end{enumerate}
\end{frame}

\begin{frame}[fragile]
    \frametitle{More Functions and Types of DBMS}
    \begin{enumerate}
        \setcounter{enumi}{2}
        \item \textbf{Data Security}: Protects data from unauthorized access.
        \item \textbf{Data Integrity}: Maintains accuracy through constraints like primary keys.
        \item \textbf{Data Backup and Recovery}: Enables data protection against loss.
        \item \textbf{Concurrency Control}: Manages simultaneous data access by multiple users.
    \end{enumerate}

    \begin{block}{Types of DBMS}
        \begin{itemize}
            \item \textbf{Hierarchical DBMS}: Organizes data in a tree-like structure.
            \item \textbf{Network DBMS}: Allows complex relationships among data.
            \item \textbf{Relational DBMS (RDBMS)}: Stores data in tables (e.g., MySQL).
            \item \textbf{Object-oriented DBMS}: Stores data as objects, aligning with object-oriented programming.
        \end{itemize}
    \end{block}
\end{frame}

\begin{frame}
    \frametitle{SQL Best Practices}
    \begin{block}{Introduction}
        Writing efficient and maintainable SQL queries is essential for optimizing database performance and ensuring clarity in your code. This presentation covers best practices to help you craft better SQL focusing on readability, efficiency, and maintainability.
    \end{block}
\end{frame}

\begin{frame}[fragile]
    \frametitle{Key Best Practices - Part 1}
    \begin{enumerate}
        \item \textbf{Use Meaningful Naming Conventions}
        \begin{itemize}
            \item Tables and Columns: Choose descriptive names that reflect their contents.
            \item Example: Instead of naming a table \texttt{tbl1}, name it \texttt{customers} or \texttt{orders}.
        \end{itemize}
        
        \item \textbf{Organize Your SQL Code}
        \begin{itemize}
            \item Indentation and Line Breaks: Use consistent formatting to improve readability.
            \item Example:
            \begin{lstlisting}
SELECT first_name, last_name
FROM customers
WHERE city = 'New York'
ORDER BY last_name;
            \end{lstlisting}
        \end{itemize}
        
        \item \textbf{Use Proper Joins}
        \begin{itemize}
            \item JOIN Types: Understand INNER JOIN, LEFT JOIN, RIGHT JOIN, and FULL OUTER JOIN.
            \item Example:
            \begin{lstlisting}
SELECT c.first_name, o.order_id
FROM customers c
LEFT JOIN orders o ON c.customer_id = o.customer_id;
            \end{lstlisting}
        \end{itemize}
    \end{enumerate}
\end{frame}

\begin{frame}[fragile]
    \frametitle{Key Best Practices - Part 2}
    \begin{enumerate}
        \setcounter{enumi}{3}
        \item \textbf{Limit Results with WHERE Clauses}
        \begin{itemize}
            \item Use the WHERE clause to filter data early.
            \item Example:
            \begin{lstlisting}
SELECT * FROM orders
WHERE order_date >= '2023-01-01';
            \end{lstlisting}
        \end{itemize}
        
        \item \textbf{Avoid SELECT *}
        \begin{itemize}
            \item Specify columns to return only the necessary data.
            \item Example:
            \begin{lstlisting}
SELECT order_id, order_total FROM orders;
            \end{lstlisting}
        \end{itemize}

        \item \textbf{Use Indexes Wisely}
        \begin{itemize}
            \item Use indexes on frequently searched columns.
            \item Example:
            \begin{lstlisting}
CREATE INDEX idx_customer_city ON customers(city);
            \end{lstlisting}
        \end{itemize}
    \end{enumerate}
\end{frame}

\begin{frame}[fragile]
    \frametitle{Key Best Practices - Part 3}
    \begin{enumerate}
        \setcounter{enumi}{6}
        \item \textbf{Use Transactions for Integrity}
        \begin{itemize}
            \item Wrap data modifications in transactions for integrity.
            \item Example:
            \begin{lstlisting}
BEGIN TRANSACTION;
UPDATE accounts SET balance = balance - 100 WHERE account_id = 10;
UPDATE accounts SET balance = balance + 100 WHERE account_id = 20;
COMMIT;
            \end{lstlisting}
        \end{itemize}

        \item \textbf{Comment Your Code}
        \begin{itemize}
            \item Explain complex logic with comments for clarity.
            \item Example:
            \begin{lstlisting}
-- Get the total sales for each customer
SELECT customer_id, SUM(total_amount) 
FROM sales
GROUP BY customer_id;
            \end{lstlisting}
        \end{itemize}
    \end{enumerate}
\end{frame}

\begin{frame}
    \frametitle{Conclusion}
    Adhering to these SQL best practices not only improves performance and efficiency but also enhances maintainability and readability of your SQL code. Incorporating these practices will lead to more robust database applications and help you leverage the full power of SQL.
\end{frame}

\begin{frame}[fragile]
    \frametitle{Conclusion and Q\&A - Recap of SQL and Databases}
    
    \begin{block}{Understanding Databases}
        \begin{itemize}
            \item A database is an organized collection of data that can be easily accessed, managed, and updated.
            \item \textbf{Types of Databases:}
            \begin{itemize}
                \item \textbf{Relational Databases:} Use tables to store data. \textit{Examples: MySQL, PostgreSQL.}
                \item \textbf{NoSQL Databases:} Use various data models such as key-value pairs, documents, or graphs. \textit{Examples: MongoDB, Cassandra.}
            \end{itemize}
        \end{itemize}
    \end{block}
\end{frame}

\begin{frame}[fragile]
    \frametitle{Conclusion and Q\&A - Introduction to SQL}
    
    \begin{block}{Introduction to SQL}
        \begin{itemize}
            \item SQL (Structured Query Language) is the standard language for interacting with relational databases.
            \item \textbf{Key SQL Functions:}
            \begin{itemize}
                \item \textbf{Data Retrieval:} Use \texttt{SELECT} to fetch data from databases.
                \item \textbf{Data Manipulation:} Use \texttt{INSERT}, \texttt{UPDATE}, and \texttt{DELETE} to modify data.
                \item \textbf{Data Definition:} Use \texttt{CREATE}, \texttt{ALTER}, and \texttt{DROP} to manage database structure.
            \end{itemize}
        \end{itemize}
    \end{block}
\end{frame}

\begin{frame}[fragile]
    \frametitle{Conclusion and Q\&A - Best Practices and Examples}
    
    \begin{block}{SQL Best Practices Recap}
        \begin{itemize}
            \item Always use meaningful names for tables and columns.
            \item Normalize your database to reduce redundancy.
            \item Use comments in your SQL code to explain complex logic.
            \item Limit your \texttt{SELECT} statements to fetch only necessary columns.
        \end{itemize}
    \end{block}

    \begin{block}{Examples}
        \textbf{Basic SELECT Query:}
        \begin{lstlisting}[language=SQL]
        SELECT first_name, last_name 
        FROM employees 
        WHERE department = 'Sales';
        \end{lstlisting}
        
        \textbf{Inserting Data:}
        \begin{lstlisting}[language=SQL]
        INSERT INTO employees (first_name, last_name, department) 
        VALUES ('Jane', 'Doe', 'Marketing');
        \end{lstlisting}
    \end{block}
\end{frame}

\begin{frame}[fragile]
    \frametitle{Conclusion and Q\&A - Q\&A Session & Closing Thoughts}
    
    \begin{block}{Q\&A Session}
        \begin{itemize}
            \item Open the floor for students to ask questions regarding SQL, databases, or specific challenges they might be facing.
            \item Encourage discussion on real-world applications of SQL in various industries.
        \end{itemize}
    \end{block}
    
    \begin{block}{Closing Thoughts}
        \begin{itemize}
            \item Mastering SQL and database management is key in today's data-driven environment.
            \item Continued practice and exploration of advanced SQL features will further enhance your skills.
        \end{itemize}
    \end{block}
\end{frame}


\end{document}