\documentclass[aspectratio=169]{beamer}

% Theme and Color Setup
\usetheme{Madrid}
\usecolortheme{whale}
\useinnertheme{rectangles}
\useoutertheme{miniframes}

% Additional Packages
\usepackage[utf8]{inputenc}
\usepackage[T1]{fontenc}
\usepackage{graphicx}
\usepackage{booktabs}
\usepackage{listings}
\usepackage{amsmath}
\usepackage{amssymb}
\usepackage{xcolor}
\usepackage{tikz}
\usepackage{pgfplots}
\pgfplotsset{compat=1.18}
\usetikzlibrary{positioning}
\usepackage{hyperref}

% Custom Colors
\definecolor{myblue}{RGB}{31, 73, 125}
\definecolor{mygray}{RGB}{100, 100, 100}
\definecolor{mygreen}{RGB}{0, 128, 0}
\definecolor{myorange}{RGB}{230, 126, 34}
\definecolor{mycodebackground}{RGB}{245, 245, 245}

% Set Theme Colors
\setbeamercolor{structure}{fg=myblue}
\setbeamercolor{frametitle}{fg=white, bg=myblue}
\setbeamercolor{title}{fg=myblue}
\setbeamercolor{section in toc}{fg=myblue}
\setbeamercolor{item projected}{fg=white, bg=myblue}
\setbeamercolor{block title}{bg=myblue!20, fg=myblue}
\setbeamercolor{block body}{bg=myblue!10}
\setbeamercolor{alerted text}{fg=myorange}

% Set Fonts
\setbeamerfont{title}{size=\Large, series=\bfseries}
\setbeamerfont{frametitle}{size=\large, series=\bfseries}
\setbeamerfont{caption}{size=\small}
\setbeamerfont{footnote}{size=\tiny}

% Code Listing Style
\lstdefinestyle{customcode}{
  backgroundcolor=\color{mycodebackground},
  basicstyle=\footnotesize\ttfamily,
  breakatwhitespace=false,
  breaklines=true,
  commentstyle=\color{mygreen}\itshape,
  keywordstyle=\color{blue}\bfseries,
  stringstyle=\color{myorange},
  numbers=left,
  numbersep=8pt,
  numberstyle=\tiny\color{mygray},
  frame=single,
  framesep=5pt,
  rulecolor=\color{mygray},
  showspaces=false,
  showstringspaces=false,
  showtabs=false,
  tabsize=2,
  captionpos=b
}
\lstset{style=customcode}

% Custom Commands
\newcommand{\hilight}[1]{\colorbox{myorange!30}{#1}}
\newcommand{\source}[1]{\vspace{0.2cm}\hfill{\tiny\textcolor{mygray}{Source: #1}}}
\newcommand{\concept}[1]{\textcolor{myblue}{\textbf{#1}}}
\newcommand{\separator}{\begin{center}\rule{0.5\linewidth}{0.5pt}\end{center}}

% Footer and Navigation Setup
\setbeamertemplate{footline}{
  \leavevmode%
  \hbox{%
  \begin{beamercolorbox}[wd=.3\paperwidth,ht=2.25ex,dp=1ex,center]{author in head/foot}%
    \usebeamerfont{author in head/foot}\insertshortauthor
  \end{beamercolorbox}%
  \begin{beamercolorbox}[wd=.5\paperwidth,ht=2.25ex,dp=1ex,center]{title in head/foot}%
    \usebeamerfont{title in head/foot}\insertshorttitle
  \end{beamercolorbox}%
  \begin{beamercolorbox}[wd=.2\paperwidth,ht=2.25ex,dp=1ex,center]{date in head/foot}%
    \usebeamerfont{date in head/foot}
    \insertframenumber{} / \inserttotalframenumber
  \end{beamercolorbox}}%
  \vskip0pt%
}

% Turn off navigation symbols
\setbeamertemplate{navigation symbols}{}

% Title Page Information
\title[Working with Python Libraries]{Week 8: Working with Python Libraries}
\author[J. Smith]{John Smith, Ph.D.}
\institute[University Name]{
  Department of Computer Science\\
  University Name\\
  \vspace{0.3cm}
  Email: email@university.edu\\
  Website: www.university.edu
}
\date{\today}

% Document Start
\begin{document}

\frame{\titlepage}

\begin{frame}[fragile]
    \frametitle{Introduction to Working with Python Libraries - Overview}
    \begin{block}{What are Python Libraries?}
        Python libraries are collections of pre-written code that extend the functionalities of Python programming. 
        They allow developers to perform complex tasks without writing code from scratch, speeding up the development process.
    \end{block}
\end{frame}

\begin{frame}[fragile]
    \frametitle{Importance of Python Libraries in Data Processing}
    \begin{enumerate}
        \item \textbf{Efficiency}: Libraries contain optimized functions for data manipulation and analysis, significantly reducing development time and resource consumption.
        \item \textbf{Simplicity}: They abstract complex functionalities into easy-to-use methods, making coding more accessible.
        \item \textbf{Community Support}: Popular libraries are often maintained by a large community, ensuring timely updates and a wealth of shared knowledge.
        \item \textbf{Interoperability}: Many libraries work together seamlessly, allowing for versatile workflows in data science, machine learning, and web development.
    \end{enumerate}
\end{frame}

\begin{frame}[fragile]
    \frametitle{Key Applications of Python Libraries}
    \begin{itemize}
        \item \textbf{Data Analysis}: Libraries like Pandas facilitate easy handling of data.
        \begin{lstlisting}[language=Python]
import pandas as pd

# Load a CSV file
data = pd.read_csv('data.csv')

# Display the first few rows
print(data.head())
        \end{lstlisting}

        \item \textbf{Numerical Computing}: NumPy supports large, multi-dimensional arrays and matrices.
        \begin{lstlisting}[language=Python]
import numpy as np

# Create a NumPy array
array = np.array([1, 2, 3, 4, 5])

# Calculate the mean
mean_value = np.mean(array)
print(f'Mean: {mean_value}')  # Output: Mean: 3.0
        \end{lstlisting}

        \item \textbf{Data Visualization}: Libraries such as Matplotlib and Seaborn enable the creation of compelling graphics to effectively convey data insights.
    \end{itemize}
\end{frame}

\begin{frame}[fragile]
    \frametitle{Key Points to Emphasize}
    \begin{itemize}
        \item Python libraries enhance productivity and provide robust solutions for data processing.
        \item Familiarity with popular libraries is essential for data-related tasks in various sectors.
        \item Learning to utilize these libraries saves time and leverages the power of community-driven code.
    \end{itemize}
\end{frame}

\begin{frame}[fragile]
    \frametitle{Conclusion}
    Working with Python libraries is crucial for anyone seeking to process and analyze data efficiently. As we move through the upcoming slides, we will dive deeper into specific libraries like Pandas and NumPy, exploring their unique features and applications in the data analytics landscape.
\end{frame}

\begin{frame}{What are Pandas and NumPy?}
    \begin{block}{Introduction to Pandas and NumPy}
        Pandas and NumPy are key libraries in Python for data analysis and manipulation. 
        They provide essential tools for efficiently managing structured data, making them crucial for anyone working with data.
    \end{block}
\end{frame}

\begin{frame}{Key Concepts - NumPy}
    \begin{block}{NumPy (Numerical Python)}
        \begin{itemize}
            \item \textbf{Purpose}: Designed for numerical computations, enabling support for large, multi-dimensional arrays and matrices.
            \item \textbf{Key Features}:
                \begin{itemize}
                    \item \textbf{Arrays}: Core feature is the ndarray (N-dimensional array) for fast data storage.
                    \item \textbf{Mathematical Functions}: Allows element-wise operations on arrays (e.g., addition, multiplication).
                \end{itemize}
        \end{itemize}
    \end{block}
    \begin{block}{Example}
        \begin{lstlisting}[language=Python]
import numpy as np
a = np.array([1, 2, 3])
b = np.array([4, 5, 6])
c = np.add(a, b)  # c will be array([5, 7, 9])
        \end{lstlisting}
    \end{block}
\end{frame}

\begin{frame}{Key Concepts - Pandas}
    \begin{block}{Pandas}
        \begin{itemize}
            \item \textbf{Purpose}: Focused on data manipulation and analysis with data structures such as Series and DataFrame.
            \item \textbf{Key Features}:
                \begin{itemize}
                    \item \textbf{DataFrame}: Essential for handling datasets with rows and columns, enabling easy data selection and manipulation.
                    \item \textbf{Data Analysis Functions}: Extensive functions for data cleaning, filtering, grouping, and aggregating.
                \end{itemize}
        \end{itemize}
    \end{block}
    \begin{block}{Example}
        \begin{lstlisting}[language=Python]
import pandas as pd
data = {'Name': ['Alice', 'Bob', 'Charlie'], 'Age': [25, 30, 35]}
df = pd.DataFrame(data)
print(df)
        \end{lstlisting}
        Output:
        \begin{verbatim}
   Name  Age
0  Alice   25
1    Bob   30
2 Charlie  35
        \end{verbatim}
    \end{block}
\end{frame}

\begin{frame}{Relevance in Data Analysis}
    \begin{block}{Why Pandas and NumPy Matter}
        \begin{itemize}
            \item \textbf{Efficiency}: Handle large datasets quickly and easily.
            \item \textbf{Functionality}: Powerful methods for data cleaning, exploration, and visualization.
            \item \textbf{Integration}: Work seamlessly together for better data handling.
        \end{itemize}
    \end{block}
\end{frame}

\begin{frame}{Key Takeaways}
    \begin{itemize}
        \item \textbf{NumPy}: Essential for numerical computations and efficient dataset handling.
        \item \textbf{Pandas}: Simplifies data manipulation and analysis with its DataFrame structure.
        \item Mastery of these libraries is crucial for any data-related tasks in Python.
    \end{itemize}
\end{frame}

\begin{frame}[fragile]
    \frametitle{Key Features of Pandas - Overview}
    Pandas is a powerful and flexible library in Python designed for data manipulation and analysis. 
    It provides rich data structures that simplify data handling and operations. Below are the primary features of Pandas:
\end{frame}

\begin{frame}[fragile]
    \frametitle{Key Features of Pandas - Data Structures}
    \begin{enumerate}
        \item \textbf{DataFrame}
        \begin{itemize}
            \item A two-dimensional, size-mutable, and potentially heterogeneous tabular data structure with labeled axes (rows and columns).
            \item \textbf{Example}:
            \begin{lstlisting}[language=Python]
import pandas as pd

data = {
    'Name': ['Alice', 'Bob', 'Charlie'],
    'Age': [25, 30, 35],
    'City': ['New York', 'Los Angeles', 'Chicago']
}

df = pd.DataFrame(data)
print(df)
            \end{lstlisting}
            \item \textbf{Output}:
            \begin{lstlisting}
     Name  Age         City
0  Alice   25     New York
1    Bob   30  Los Angeles
2 Charlie   35      Chicago
            \end{lstlisting}
        \end{itemize}
        
        \item \textbf{Series}
        \begin{itemize}
            \item A one-dimensional labeled array capable of holding any data type. It can be seen as a single column of a DataFrame.
            \item \textbf{Example}:
            \begin{lstlisting}[language=Python]
ages = pd.Series([25, 30, 35], index=['Alice', 'Bob', 'Charlie'])
print(ages)
            \end{lstlisting}
            \item \textbf{Output}:
            \begin{lstlisting}
Alice      25
Bob        30
Charlie    35
dtype: int64
            \end{lstlisting}
        \end{itemize}
    \end{enumerate}
\end{frame}

\begin{frame}[fragile]
    \frametitle{Key Features of Pandas - Data Manipulation Functionalities}
    \begin{itemize}
        \item \textbf{Data Cleaning}:
        \begin{lstlisting}[language=Python]
df.dropna()   # Removes rows with missing values
df.fillna(0)  # Replaces missing values with 0
        \end{lstlisting}

        \item \textbf{Filtering and Selection}:
        \begin{lstlisting}[language=Python]
adults = df[df['Age'] > 30]  # Select rows where Age > 30
        \end{lstlisting}

        \item \textbf{Grouping Data}:
        \begin{lstlisting}[language=Python]
grouped = df.groupby('City')['Age'].mean()  # Average age by city
        \end{lstlisting}

        \item \textbf{Merging and Joining}:
        \begin{lstlisting}[language=Python]
df2 = pd.DataFrame({'City': ['New York', 'Chicago', 'Los Angeles'],
                    'Population': [8419600, 2716000, 3979576]})
merged_df = pd.merge(df, df2, on='City')  # Merging on 'City'
        \end{lstlisting}
    \end{itemize}
\end{frame}

\begin{frame}[fragile]
    \frametitle{Key Features of Pandas - Key Points}
    \begin{itemize}
        \item \textbf{Versatility}: Supports both labeled and unlabeled data, adaptable for different analysis tasks.
        \item \textbf{Integration}: Well integrated with libraries like NumPy and Matplotlib for scientific computing and visualization.
        \item \textbf{Efficiency}: Highly optimized for operations on large datasets, enhancing data analysis capabilities in Python.
    \end{itemize}
    
    Utilizing Pandas significantly enhances data analysis, making it a fundamental tool for data scientists and analysts.
\end{frame}

\begin{frame}[fragile]
    \frametitle{Key Features of NumPy - Overview}
    \begin{block}{Overview of NumPy}
        NumPy (Numerical Python) is a powerful open-source library utilized for numerical computing in Python. 
        It offers a wide array of features for handling large, multidimensional arrays and matrices, complemented by a collection of mathematical functions to operate on these data structures.
    \end{block}
\end{frame}

\begin{frame}[fragile]
    \frametitle{Key Features of NumPy - Array Structure}
    \begin{itemize}
        \item \textbf{N-Dimensional Arrays:}
        \begin{itemize}
            \item Core component: \texttt{ndarray} for efficient storage and manipulation of large datasets.
        \end{itemize}
        \item \textbf{Array Creation:}
        \begin{itemize}
            \item From a list: \texttt{array = np.array([1, 2, 3])}
            \item Using functions:
            \begin{itemize}
                \item \texttt{zeros\_array = np.zeros((2, 3))}
                \item \texttt{ones\_array = np.ones((2, 3))}
                \item \texttt{arange\_array = np.arange(1, 10)}
            \end{itemize}
        \end{itemize}
        \item \textbf{Shape and Reshaping:}
        \begin{itemize}
            \item \texttt{array.shape} indicates dimensions.
            \item Reshape using:
            \begin{lstlisting}
array = np.array([1, 2, 3, 4, 5, 6])
reshaped_array = array.reshape((2, 3))
            \end{lstlisting}
        \end{itemize}
    \end{itemize}
\end{frame}

\begin{frame}[fragile]
    \frametitle{Key Features of NumPy - Numerical Capabilities}
    \begin{itemize}
        \item \textbf{Data Types:}
        \begin{itemize}
            \item Supports: \texttt{int}, \texttt{float}, \texttt{complex}, \texttt{bool}, \texttt{object}, \texttt{string}.
            \item Specify type: 
            \begin{lstlisting}
float_array = np.array([1, 2, 3], dtype='float')
            \end{lstlisting}
        \end{itemize}
        \item \textbf{Mathematical Operations:}
        \begin{itemize}
            \item Element-wise operations:
            \begin{lstlisting}
a = np.array([1, 2, 3])
b = np.array([4, 5, 6])
sum_array = a + b  # Results in array([5, 7, 9])
            \end{lstlisting}
        \end{itemize}
        \item \textbf{Universal Functions (ufuncs):}
        \begin{itemize}
            \item Built-in functions for element-wise operation, e.g., \texttt{np.sqrt()}, \texttt{np.sin()}.
            \item Example:
            \begin{lstlisting}
square_root = np.sqrt(np.array([1, 4, 9]))
            \end{lstlisting}
        \end{itemize}
    \end{itemize}
\begin{block}{Key Points to Emphasize}
    \begin{itemize}
        \item Performance: More efficient memory usage and performance compared to Python lists.
        \item Broadcasting: Enables operations between arrays of different shapes.
        \item Interoperability: Works well with libraries like Pandas, SciPy, and Matplotlib.
    \end{itemize}
\end{block}
\end{frame}

\begin{frame}[fragile]
    \frametitle{Data Structures in Pandas - Overview}
    \begin{itemize}
        \item Pandas is a powerful library for data manipulation in Python.
        \item The two primary data structures in Pandas:
        \begin{itemize}
            \item \textbf{Series}: One-dimensional labeled array.
            \item \textbf{DataFrame}: Two-dimensional labeled data structure.
        \end{itemize}
    \end{itemize}
\end{frame}

\begin{frame}[fragile]
    \frametitle{Data Structures in Pandas - Series}
    \begin{block}{Series}
        A \texttt{Series} is a one-dimensional labeled array capable of holding any data type.
    \end{block}
    \begin{itemize}
        \item \textbf{Key Characteristics}:
        \begin{itemize}
            \item Unique labels (index) for each item.
            \item Suitable for a single column of data.
        \end{itemize}
        \item \textbf{Creating a Series}:
        \begin{lstlisting}[language=Python]
import pandas as pd

# Creating a Series
data = [10, 20, 30, 40]
s = pd.Series(data)
print(s)
        \end{lstlisting}
        \item \textbf{Accessing Data}:
        \begin{lstlisting}[language=Python]
print(s[2])  # Output: 30
        \end{lstlisting}
    \end{itemize}
\end{frame}

\begin{frame}[fragile]
    \frametitle{Data Structures in Pandas - DataFrame}
    \begin{block}{DataFrame}
        A \texttt{DataFrame} is a two-dimensional labeled data structure with columns of potentially different types.
    \end{block}
    \begin{itemize}
        \item \textbf{Key Characteristics}:
        \begin{itemize}
            \item Rows and columns can hold different data types.
            \item Labeled indexing for both rows and columns.
        \end{itemize}
        \item \textbf{Creating a DataFrame}:
        \begin{lstlisting}[language=Python]
# Creating a DataFrame
data = {
    'Name': ['Alice', 'Bob', 'Charlie'],
    'Age': [25, 30, 35],
    'City': ['New York', 'Los Angeles', 'Chicago']
}
df = pd.DataFrame(data)
print(df)
        \end{lstlisting}
        \item \textbf{Accessing Data}:
        \begin{itemize}
            \item Selecting a column: \texttt{print(df['Age'])}
            \item Selecting a row: \texttt{print(df.loc[1])}
        \end{itemize}
    \end{itemize}
\end{frame}

\begin{frame}[fragile]
    \frametitle{Pandas Data Structures - Key Points and Summary}
    \begin{itemize}
        \item \textbf{Indexing}:
        \begin{itemize}
            \item Series and DataFrames support various indexing methods.
        \end{itemize}
        \item \textbf{Flexibility}:
        \begin{itemize}
            \item Use Series for one-dimensional data.
            \item Use DataFrames for tabular data.
        \end{itemize}
        \item \textbf{Operations}:
        \begin{itemize}
            \item Filtering, aggregation, and merging capabilities.
        \end{itemize}
        \item \textbf{Illustrative Summary}:
        \begin{itemize}
            \item Series: 1D data structure.
            \item DataFrame: Collection of Series (2D).
        \end{itemize}
    \end{itemize}    
    \begin{block}{Up Next}
        Explore \textbf{Basic Operations with Pandas} in the next slide!
    \end{block}
\end{frame}

\begin{frame}
    \frametitle{Basic Operations with Pandas}
    \begin{block}{Introduction to Pandas}
        Pandas is a powerful Python library used for data manipulation and analysis. It offers data structures and functions needed to work efficiently with structured data. This session will explore fundamental operations such as \textbf{data selection}, \textbf{indexing}, and \textbf{filtering}.
    \end{block}
\end{frame}

\begin{frame}[fragile]
    \frametitle{Data Selection}
    Data selection in Pandas allows you to access individual data points, rows, or columns in a DataFrame.
    
    \begin{itemize}
        \item \textbf{Selecting Columns}: You can select a column by its label.
        \begin{lstlisting}[language=Python]
import pandas as pd

# Creating a sample DataFrame
data = {
    'Name': ['Alice', 'Bob', 'Charlie'],
    'Age': [25, 30, 35],
    'City': ['New York', 'Los Angeles', 'Chicago']
}
df = pd.DataFrame(data)

# Selecting the 'Age' column
ages = df['Age']
print(ages)
        \end{lstlisting}
        
        \item \textbf{Selecting Rows}: Use `loc` and `iloc` for label-based or position-based indexing.
        \begin{itemize}
            \item \textbf{Label-based}:
            \begin{lstlisting}[language=Python]
first_row = df.loc[0]  # Selects the first row
print(first_row)
            \end{lstlisting}
            \item \textbf{Position-based}:
            \begin{lstlisting}[language=Python]
first_row = df.iloc[0]  # Selects the first row
print(first_row)
            \end{lstlisting}
        \end{itemize}
    \end{itemize}
\end{frame}

\begin{frame}[fragile]
    \frametitle{Indexing and Filtering Data}
    \begin{block}{Indexing}
        Indexing helps to define how data is organized. The default index is numerical (0, 1, 2,...), but you can set a custom index using any column.
        
        \begin{lstlisting}[language=Python]
df.set_index('Name', inplace=True)
print(df)
        \end{lstlisting}
    \end{block}

    \begin{block}{Filtering Data}
        Filtering retrieves rows based on certain conditions.
        
        \begin{itemize}
            \item \textbf{Condition-Based Filtering}:
            \begin{lstlisting}[language=Python]
# Filtering rows where Age is greater than 28
filtered_data = df[df['Age'] > 28]
print(filtered_data)
            \end{lstlisting}
            \item \textbf{Multiple Conditions}:
            \begin{lstlisting}[language=Python]
# Filtering rows where Age > 25 and City is 'Chicago'
filtered_data = df[(df['Age'] > 25) & (df['City'] == 'Chicago')]
print(filtered_data)
            \end{lstlisting}
        \end{itemize}
    \end{block}
\end{frame}

\begin{frame}
    \frametitle{Key Points and Conclusion}
    \begin{itemize}
        \item \textbf{Selection}: Done using column names or indices; `loc` and `iloc` provide label and position access respectively.
        \item \textbf{Indexing}: Enhances data retrieval ease and can be customized.
        \item \textbf{Filtering}: Enables detailed data analysis based on conditions.
    \end{itemize}

    \begin{block}{Conclusion}
        Understanding basic operations in Pandas is crucial for efficient data handling. This knowledge simplifies data analysis tasks and enhances your ability to extract meaningful insights from datasets. By mastering these foundational skills, you are well on your way to becoming proficient in data analysis with Python and Pandas!
    \end{block}
\end{frame}

\begin{frame}[fragile]
    \frametitle{Basic Operations with NumPy}
    \begin{block}{Introduction to NumPy}
        NumPy (Numerical Python) is a fundamental package for scientific computing in Python, providing support for handling arrays, matrices, and a large collection of mathematical functions.
    \end{block}
\end{frame}

\begin{frame}[fragile]
    \frametitle{Array Creation}
    \begin{itemize}
        \item **From a List**:
        \begin{lstlisting}[language=Python]
import numpy as np

# Creating a 1D array from a list
array_1d = np.array([1, 2, 3, 4])
print(array_1d)  # Output: [1 2 3 4]
        \end{lstlisting}
        
        \item **Multi-dimensional Arrays**:
        \begin{lstlisting}[language=Python]
# Creating a 2D array (matrix) from a list of lists
array_2d = np.array([[1, 2, 3], [4, 5, 6]])
print(array_2d)
        \end{lstlisting}
        
        \item **Using Built-in Functions**:
        \begin{itemize}
            \item Zeros Array:
            \begin{lstlisting}[language=Python]
zeros_array = np.zeros((2, 3))  # 2x3 array of zeros
print(zeros_array)
            \end{lstlisting}
            \item Ones Array:
            \begin{lstlisting}[language=Python]
ones_array = np.ones((3, 3))  # 3x3 array of ones
print(ones_array)
            \end{lstlisting}
            \item Random Array:
            \begin{lstlisting}[language=Python]
random_array = np.random.rand(2, 2)  # 2x2 array of random floats
print(random_array)
            \end{lstlisting}
        \end{itemize}
    \end{itemize}
\end{frame}

\begin{frame}[fragile]
    \frametitle{Indexing and Slicing}
    \begin{itemize}
        \item **Indexing**:
        \begin{lstlisting}[language=Python]
# Accessing element in a 1D array
print(array_1d[0])  # Output: 1

# Accessing element in a 2D array
print(array_2d[1, 2])  # Output: 6
        \end{lstlisting}
        
        \item **Slicing**:
        \begin{lstlisting}[language=Python]
# Slicing elements
print(array_1d[1:3])  # Output: [2 3]

# Slicing 2D array
print(array_2d[:, 1])  # Output: [2 5]
        \end{lstlisting}
    \end{itemize}
\end{frame}

\begin{frame}[fragile]
    \frametitle{Array Manipulation}
    \begin{itemize}
        \item **Reshaping Arrays**:
        \begin{lstlisting}[language=Python]
reshaped_array = np.reshape(array_1d, (2, 2))
print(reshaped_array)
        \end{lstlisting}
        
        \item **Array Concatenation**:
        \begin{lstlisting}[language=Python]
array_a = np.array([1, 2])
array_b = np.array([3, 4])
concatenated_array = np.concatenate((array_a, array_b))
print(concatenated_array)  # Output: [1 2 3 4]
        \end{lstlisting}
        
        \item **Element-wise Operations**:
        \begin{lstlisting}[language=Python]
array_c = np.array([1, 2, 3])
array_d = np.array([4, 5, 6])
result = array_c + array_d
print(result)  # Output: [5 7 9]
        \end{lstlisting}
    \end{itemize}
\end{frame}

\begin{frame}
    \frametitle{Key Points and Conclusion}
    \begin{itemize}
        \item NumPy is vital for numerical tasks in Python and is widely used in data analysis and scientific computing.
        \item Mastery of array creation and manipulation is essential for working effectively with data.
        \item Indexing and slicing allow you to extract and modify the data efficiently.
    \end{itemize}
    \begin{block}{Next Steps}
        In the next slide, we will discuss data cleaning techniques using Pandas, including handling missing values and duplicates.
    \end{block}
\end{frame}

\begin{frame}
    \titlepage
\end{frame}

\begin{frame}
    \frametitle{Introduction to Data Cleaning}
    \begin{itemize}
        \item Data cleaning is a critical preprocessing step in data analysis.
        \item Involves identifying and correcting errors or inconsistencies in data.
        \item Ensures accuracy and quality of data before analysis.
    \end{itemize}
\end{frame}

\begin{frame}[fragile]
    \frametitle{1. Handling Missing Values}
    Missing values can significantly impact your analysis. Pandas offers several techniques to handle them effectively:
    
    \begin{itemize}
        \item \textbf{Identifying Missing Values}
        \begin{lstlisting}
import pandas as pd
df = pd.read_csv('data.csv')
print(df.isnull().sum())
        \end{lstlisting}
        
        \item \textbf{Removing Missing Values}
        \begin{itemize}
            \item \textbf{Drop Rows:}
            \begin{lstlisting}
df_cleaned = df.dropna()
            \end{lstlisting}
            \item \textbf{Drop Columns:}
            \begin{lstlisting}
df_cleaned = df.dropna(axis=1)
            \end{lstlisting}
        \end{itemize}
        
        \item \textbf{Filling Missing Values}
        \begin{itemize}
            \item With a Static Value:
            \begin{lstlisting}
df['column_name'].fillna(0, inplace=True)
            \end{lstlisting}
            \item With Statistical Values:
            \begin{lstlisting}
df['numeric_column'].fillna(df['numeric_column'].mean(), inplace=True)
            \end{lstlisting}
        \end{itemize}
    \end{itemize}
\end{frame}

\begin{frame}[fragile]
    \frametitle{2. Handling Duplicates}
    Duplicate rows can skew your analysis and must be dealt with appropriately.

    \begin{itemize}
        \item \textbf{Detecting Duplicates}
        \begin{lstlisting}
duplicate_rows = df[df.duplicated()]
print(duplicate_rows)
        \end{lstlisting}
        
        \item \textbf{Removing Duplicates}
        \begin{lstlisting}
df_cleaned = df.drop_duplicates()
        \end{lstlisting}
        
        \item \textbf{Keeping Specific Duplicates:}
        To keep the first or last occurrence of the duplicates:
        \begin{lstlisting}
df_cleaned = df.drop_duplicates(keep='first')  # or keep='last'
        \end{lstlisting}
    \end{itemize}
\end{frame}

\begin{frame}
    \frametitle{Key Points to Remember}
    \begin{block}{}
        \begin{itemize}
            \item Always explore your dataset for missing values and duplicates before analysis.
            \item Choose an appropriate strategy for handling missing values based on the data context.
            \item Use \texttt{.dropna()} to remove or \texttt{.fillna()} to replace missing values, and \texttt{.drop_duplicates()} to clean duplicates.
        \end{itemize}
    \end{block}
\end{frame}

\begin{frame}
    \frametitle{Conclusion}
    Cleaning your data with Pandas is fundamental for ensuring accurate analysis. Mastering these techniques will help maintain data integrity and reliability in your analyses.
\end{frame}

\begin{frame}
    \frametitle{Data Analysis with NumPy}
    \begin{block}{Overview}
        Using NumPy for performing various statistical analyses including mean, median, and standard deviation.
    \end{block}
\end{frame}

\begin{frame}
    \frametitle{Understanding NumPy}
    \begin{itemize}
        \item NumPy (Numerical Python) is a fundamental package for scientific computing in Python.
        \item It provides support for arrays, matrices, and a wide range of mathematical functions.
    \end{itemize}
    \begin{block}{Key Concepts}
        \begin{enumerate}
            \item \textbf{Arrays:} Optimized for numerical computation, enabling faster operations on large datasets.
            \item \textbf{Statistical Functions:} Built-in functions to calculate mean, median, and standard deviation.
        \end{enumerate}
    \end{block}
\end{frame}

\begin{frame}
    \frametitle{Core Statistical Analyses}
    \textbf{1. Mean}
    \begin{itemize}
        \item The mean is the average, computed as:
        \begin{equation}
            \text{Mean} = \frac{\sum_{i=1}^{n} x_i}{n}
        \end{equation}
        \item Example:
        \begin{lstlisting}[language=Python]
import numpy as np

data = np.array([1, 2, 3, 4, 5])
mean_value = np.mean(data)
print("Mean:", mean_value)
        \end{lstlisting}
        \item \textbf{Output:} Mean: 3.0
    \end{itemize}
\end{frame}

\begin{frame}
    \frametitle{Core Statistical Analyses continued}
    \textbf{2. Median}
    \begin{itemize}
        \item The median is the middle value:
        \begin{itemize}
            \item If n is odd: Median = \(x_{\frac{(n+1)}{2}}\)
            \item If n is even: Median = \(\frac{x_{\frac{n}{2}} + x_{\frac{n}{2} + 1}}{2}\)
        \end{itemize}
        \item Example:
        \begin{lstlisting}[language=Python]
data = np.array([1, 3, 3, 6, 7, 8, 9])
median_value = np.median(data)
print("Median:", median_value)
        \end{lstlisting}
        \item \textbf{Output:} Median: 6.0
    \end{itemize}
\end{frame}

\begin{frame}
    \frametitle{Core Statistical Analyses continued}
    \textbf{3. Standard Deviation}
    \begin{itemize}
        \item Measures the dispersion of data points:
        \begin{equation}
            \text{Standard Deviation} = \sqrt{\frac{\sum_{i=1}^{n} (x_i - \text{Mean})^2}{n-1}}
        \end{equation}
        \item Example:
        \begin{lstlisting}[language=Python]
data = np.array([1, 2, 3, 4, 5])
std_dev_value = np.std(data, ddof=0)  # Population standard deviation
print("Standard Deviation:", std_dev_value)
        \end{lstlisting}
        \item \textbf{Output:} Standard Deviation: 1.4142135623730951
    \end{itemize}
\end{frame}

\begin{frame}
    \frametitle{Key Points to Remember}
    \begin{itemize}
        \item NumPy is efficient for handling large datasets.
        \item Use \texttt{np.mean()}, \texttt{np.median()}, and \texttt{np.std()} for quick statistical analyses.
        \item Understanding these statistics is crucial for data analysis and interpretation.
    \end{itemize}
    \begin{block}{Next Steps}
        Explore how NumPy and Pandas can work together to enhance data processing capabilities.
    \end{block}
\end{frame}

\begin{frame}[fragile]
    \frametitle{Combining Pandas and NumPy - Overview}
    \begin{itemize}
        \item Pandas and NumPy are powerful Python libraries for data manipulation and analysis.
        \item Working together enhances data processing capabilities.
        \item They simplify complex tasks, making data analysis more efficient.
    \end{itemize}
\end{frame}

\begin{frame}[fragile]
    \frametitle{Combining Pandas and NumPy - Key Concepts}
    \begin{block}{1. Understanding Pandas and NumPy}
        \begin{itemize}
            \item \textbf{Pandas}: Data structures (Series, DataFrames) designed for data analysis.
            \item \textbf{NumPy}: Supports arrays, matrices, and mathematical functions for numerical computing.
        \end{itemize}
    \end{block}
    
    \begin{block}{2. Advantages of Combining}
        \begin{itemize}
            \item \textbf{Performance}: Faster numerical computations with NumPy's C-based implementation.
            \item \textbf{Functionality}: Pandas for data manipulation, NumPy for efficient mathematical operations.
        \end{itemize}
    \end{block}
\end{frame}

\begin{frame}[fragile]
    \frametitle{How They Work Together - Example}
    Here’s an example to illustrate the combination of Pandas and NumPy:

    \begin{lstlisting}[language=Python]
import pandas as pd
import numpy as np

# Sample DataFrame creation
data = {
    'Product': ['A', 'B', 'C', 'D'],
    'Sales': [100, 200, 150, 300],
    'Category': ['Electronics', 'Electronics', 'Clothing', 'Clothing']
}
df = pd.DataFrame(data)

# Calculating the total sales using NumPy
total_sales = np.sum(df['Sales'])
print(f'Total Sales: {total_sales}')

# Calculating mean sales per category
mean_sales_per_category = df.groupby('Category')['Sales'].mean()
print(mean_sales_per_category)
    \end{lstlisting}
\end{frame}

\begin{frame}[fragile]
    \frametitle{Key Points and Conclusion}
    \begin{itemize}
        \item Pandas and NumPy are essential for efficient data analysis in Python.
        \item Their synergy allows for quick data manipulation through method chaining.
        \item Using NumPy functions within Pandas harnesses performance benefits.
    \end{itemize}
    
    \textbf{Conclusion:} Combining Pandas and NumPy enables efficient, complex data analyses, optimizing data processing tasks.
    
    \textbf{Next Steps:} Explore practical applications of this combination in real-world scenarios to deepen your understanding of data analysis.
\end{frame}

\begin{frame}[fragile]
    \frametitle{Practical Applications - Overview}
    \begin{block}{Overview}
        Pandas and NumPy are indispensable libraries in Python for data analysis and processing.
        They are widely used across various industries to manipulate data efficiently,
        perform calculations, and generate insights. Below are real-world scenarios that showcase their application.
    \end{block}
\end{frame}

\begin{frame}[fragile]
    \frametitle{Practical Applications - Case Study 1: Finance}
    
    \begin{itemize}
        \item \textbf{Context:}
        Financial analysts use Pandas to analyze historical stock prices and predict future trends.
        
        \item \textbf{Application:}
        \begin{enumerate}
            \item \textbf{Data Retrieval:}
            \begin{lstlisting}[language=Python]
import pandas as pd
stock_data = pd.read_csv("stock_prices.csv")
            \end{lstlisting}
            
            \item \textbf{Data Processing:}
            \begin{lstlisting}[language=Python]
stock_data.fillna(method='ffill', inplace=True)  # Forward fill
            \end{lstlisting}

            \item \textbf{Data Analysis:}
            \begin{lstlisting}[language=Python]
stock_data['Returns'] = stock_data['Close'].pct_change()  # Daily returns
stock_data['Moving Average'] = stock_data['Close'].rolling(window=30).mean()  # 30-day MA
            \end{lstlisting}
        \end{enumerate}
        
        \item \textbf{Outcome:} Identify trends for better investment decisions.
    \end{itemize}
\end{frame}

\begin{frame}[fragile]
    \frametitle{Practical Applications - Case Study 2: Healthcare}
    
    \begin{itemize}
        \item \textbf{Context:}
        Healthcare professionals analyze patient data to improve treatment plans.
        
        \item \textbf{Application:}
        \begin{enumerate}
            \item \textbf{Data Loading:}
            \begin{lstlisting}[language=Python]
patient_data = pd.read_csv("patient_records.csv")
            \end{lstlisting}

            \item \textbf{Statistics:}
            \begin{lstlisting}[language=Python]
average_age = np.mean(patient_data['Age'])  # Average age
mean_blood_pressure = np.mean(patient_data['BloodPressure'])  # Mean blood pressure
            \end{lstlisting}

            \item \textbf{Visualization:}
            \begin{lstlisting}[language=Python]
patient_data['Age'].hist()  # Histogram of patient ages
            \end{lstlisting}
        \end{enumerate}
        
        \item \textbf{Outcome:} Personalizes medicine by understanding patient demographics and outcomes.
    \end{itemize}
\end{frame}

\begin{frame}[fragile]
    \frametitle{Key Points and Conclusion}
    
    \begin{block}{Key Points}
        \begin{itemize}
            \item \textbf{Integration:} 
            Pandas excels in data manipulation while NumPy provides fast mathematical functions.
            \item \textbf{Efficiency:} 
            Streamlines handling large datasets for effective data analysis.
            \item \textbf{Real-World Implications:} 
            Better decision-making in finance, healthcare, and various sectors.
        \end{itemize}
    \end{block}
    
    \begin{block}{Conclusion}
        Understanding the practical applications of Pandas and NumPy equips you with valuable skills for data analysis,
        preparing you for real-world problems. By leveraging these libraries, data scientists can draw actionable insights
        from complex datasets enabling informed decision-making.
    \end{block}
\end{frame}

\begin{frame}[fragile]
    \frametitle{Resources for Further Learning - Overview}
    \begin{block}{Enhance Your Skills in Pandas and NumPy}
        As you delve deeper into data analysis and manipulation with Python, leveraging external resources is vital for refining your skills and expanding your knowledge. Below is a curated list of recommended resources for both Pandas and NumPy.
    \end{block}
\end{frame}

\begin{frame}[fragile]
    \frametitle{Resources for Pandas and NumPy - Documentation}
    \begin{enumerate}
        \item \textbf{Official Documentation}
        \begin{itemize}
            \item \textbf{Pandas Documentation:} 
            The best starting point is the \href{https://pandas.pydata.org/docs/}{Pandas official documentation}. It includes comprehensive guides, API references, and tutorials.
            \item \textbf{NumPy Documentation:} 
            Equip yourself with the foundational knowledge from the \href{https://numpy.org/doc/stable/}{NumPy official documentation}. This resource covers the library's functionality and array operations in detail.
        \end{itemize}
    \end{enumerate}
\end{frame}

\begin{frame}[fragile]
    \frametitle{Resources for Pandas and NumPy - Courses and Books}
    \begin{enumerate}
        \setcounter{enumi}{1}
        \item \textbf{Online Courses and Tutorials}
        \begin{itemize}
            \item \textbf{Courses:}
            \begin{itemize}
                \item \underline{Coursera - "Data Analysis with Python"}: Covers the entire data analysis process using Pandas.
                \item \underline{edX - "Introduction to Data Science using Python"}: Focuses on data manipulation and visualization using both Pandas and NumPy.
            \end{itemize}
            \item \textbf{Tutorials:}
            \begin{itemize}
                \item \underline{Kaggle - "Pandas Course"}: A hands-on experience with real datasets.
                \item \underline{W3Schools - "NumPy Tutorial"}: Quick references and basic examples of NumPy functionalities.
            \end{itemize}
        \end{itemize}
        
        \item \textbf{Books}
        \begin{itemize}
            \item \underline{"Python for Data Analysis" by Wes McKinney}: Excellent resource with practical examples.
            \item \underline{"Python Data Science Handbook" by Jake VanderPlas}: Comprehensive overview of using NumPy and Pandas in data science tasks.
        \end{itemize}
    \end{enumerate}
\end{frame}


\end{document}