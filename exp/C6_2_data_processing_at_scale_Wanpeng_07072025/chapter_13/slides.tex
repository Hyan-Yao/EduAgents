\documentclass[aspectratio=169]{beamer}

% Theme and Color Setup
\usetheme{Madrid}
\usecolortheme{whale}
\useinnertheme{rectangles}
\useoutertheme{miniframes}

% Additional Packages
\usepackage[utf8]{inputenc}
\usepackage[T1]{fontenc}
\usepackage{graphicx}
\usepackage{booktabs}
\usepackage{listings}
\usepackage{amsmath}
\usepackage{amssymb}
\usepackage{xcolor}
\usepackage{tikz}
\usepackage{pgfplots}
\pgfplotsset{compat=1.18}
\usetikzlibrary{positioning}
\usepackage{hyperref}

% Custom Colors
\definecolor{myblue}{RGB}{31, 73, 125}
\definecolor{mygray}{RGB}{100, 100, 100}
\definecolor{mygreen}{RGB}{0, 128, 0}
\definecolor{myorange}{RGB}{230, 126, 34}
\definecolor{mycodebackground}{RGB}{245, 245, 245}

% Set Theme Colors
\setbeamercolor{structure}{fg=myblue}
\setbeamercolor{frametitle}{fg=white, bg=myblue}
\setbeamercolor{title}{fg=myblue}
\setbeamercolor{section in toc}{fg=myblue}
\setbeamercolor{item projected}{fg=white, bg=myblue}
\setbeamercolor{block title}{bg=myblue!20, fg=myblue}
\setbeamercolor{block body}{bg=myblue!10}
\setbeamercolor{alerted text}{fg=myorange}

% Set Fonts
\setbeamerfont{title}{size=\Large, series=\bfseries}
\setbeamerfont{frametitle}{size=\large, series=\bfseries}
\setbeamerfont{caption}{size=\small}
\setbeamerfont{footnote}{size=\tiny}

% Code Listing Style
\lstdefinestyle{customcode}{
  backgroundcolor=\color{mycodebackground},
  basicstyle=\footnotesize\ttfamily,
  breakatwhitespace=false,
  breaklines=true,
  commentstyle=\color{mygreen}\itshape,
  keywordstyle=\color{blue}\bfseries,
  stringstyle=\color{myorange},
  numbers=left,
  numbersep=8pt,
  numberstyle=\tiny\color{mygray},
  frame=single,
  framesep=5pt,
  rulecolor=\color{mygray},
  showspaces=false,
  showstringspaces=false,
  showtabs=false,
  tabsize=2,
  captionpos=b
}
\lstset{style=customcode}

% Footer and Navigation Setup
\setbeamertemplate{footline}{
  \leavevmode%
  \hbox{%
  \begin{beamercolorbox}[wd=.3\paperwidth,ht=2.25ex,dp=1ex,center]{author in head/foot}%
    \usebeamerfont{author in head/foot}\insertshortauthor
  \end{beamercolorbox}%
  \begin{beamercolorbox}[wd=.5\paperwidth,ht=2.25ex,dp=1ex,center]{title in head/foot}%
    \usebeamerfont{title in head/foot}\insertshorttitle
  \end{beamercolorbox}%
  \begin{beamercolorbox}[wd=.2\paperwidth,ht=2.25ex,dp=1ex,center]{date in head/foot}%
    \usebeamerfont{date in head/foot}
    \insertframenumber{} / \inserttotalframenumber
  \end{beamercolorbox}}%
  \vskip0pt%
}

% Turn off navigation symbols
\setbeamertemplate{navigation symbols}{}

% Title Page Information
\title[Project Development]{Week 13: Project Development \& Troubleshooting}
\author[J. Smith]{John Smith, Ph.D.}
\institute[University Name]{
  Department of Computer Science\\
  University Name\\
  \vspace{0.3cm}
  Email: email@university.edu\\
  Website: www.university.edu
}
\date{\today}

% Document Start
\begin{document}

\frame{\titlepage}

\begin{frame}[fragile]
    \frametitle{Introduction to Project Development \& Troubleshooting}
    \begin{block}{Overview}
        Project development and troubleshooting are vital components of the software development lifecycle. 
        Understanding these processes enhances coding skills and fosters the ability to manage and resolve issues effectively.
    \end{block}
\end{frame}

\begin{frame}[fragile]
    \frametitle{Key Concepts}
    \begin{enumerate}
        \item \textbf{Project Development}
            \begin{itemize}
                \item The entire process of creating software: planning, coding, testing, and deployment.
                \item \textbf{Importance}: Ensures efficient project completion and meeting user requirements.
            \end{itemize}
        \item \textbf{Troubleshooting}
            \begin{itemize}
                \item Diagnosing and fixing issues during development or post-deployment.
                \item \textbf{Importance}: Minimizes downtime and improves application performance.
            \end{itemize}
    \end{enumerate}
\end{frame}

\begin{frame}[fragile]
    \frametitle{Practical Coding Assistance \& Performance Optimization}
    \begin{itemize}
        \item \textbf{Practical Coding Assistance}
            \begin{itemize}
                \item Developers face challenges such as bugs and performance issues.
                \item Tools like IDEs, linters, and debugging tools improve accuracy and efficiency.
            \end{itemize}
        \item \textbf{Performance Optimization}
            \begin{itemize}
                \item \textbf{Definition}: Improving software efficiency and speed.
                \item \textbf{Key Techniques}:
                    \begin{itemize}
                        \item Code Refactoring: Optimizing existing code for readability without changing functionality.
                        \item Algorithm Optimization: Using efficient algorithms to reduce execution time.
                    \end{itemize}
            \end{itemize}
    \end{itemize}
\end{frame}

\begin{frame}[fragile]
    \frametitle{Example: Code Optimization}
    \begin{block}{Initial Approach}
        \begin{lstlisting}[language=Python]
def find_items(data):
    for i in data:
        for j in data:
            if i == j:
                print(i)
        \end{lstlisting}
        \textbf{Time Complexity}: O(n²)
    \end{block}
    
    \begin{block}{Optimized Approach}
        \begin{lstlisting}[language=Python]
def find_items(data):
    seen = set(data)
    for item in seen:
        print(item)
        \end{lstlisting}
        \textbf{Time Complexity}: O(n)
    \end{block}
\end{frame}

\begin{frame}[fragile]
    \frametitle{Conclusion}
    Mastering project development and troubleshooting is essential for every software developer. 
    It increases efficiency and ensures high-quality software delivery.
    \begin{itemize}
        \item Importance of early issue identification and resolution.
        \item Value of collaborative tools: code reviews and pair programming.
        \item Balance between new feature development and maintaining performance and reliability.
    \end{itemize}
    
    \textit{In the next slide, we will outline specific learning objectives for the week to enhance your coding skills and troubleshooting strategies.}
\end{frame}

\begin{frame}{Learning Objectives}
    \begin{block}{Overview}
        In this week's module on \textbf{Project Development \& Troubleshooting}, we aim to strengthen your coding skills and enhance your ability to troubleshoot effectively.
    \end{block}
\end{frame}

\begin{frame}{Learning Objectives - Part 1}
    \begin{enumerate}
        \item \textbf{Enhance Coding Skills}
            \begin{itemize}
                \item Develop a deeper understanding of best coding practices.
                \item Write clean, maintainable, and efficient code.
                \item \textbf{Example}: Refactoring a poorly written function:
                \begin{lstlisting}[language=Python]
# Poorly Written Function
def calculate_sum(n):
    total = 0
    for i in range(n):
        total += i
    return total

# Refactored Version
def calculate_sum(n):
    return sum(range(n))
                \end{lstlisting}
            \end{itemize}
    \end{enumerate}
\end{frame}

\begin{frame}{Learning Objectives - Part 2}
    \begin{enumerate}[start=2]
        \item \textbf{Master Debugging Techniques}
            \begin{itemize}
                \item Identify common programming errors and learn how to fix them.
                \item Use debugging tools and techniques like breakpoints and logging.
                \item \textbf{Example}: Utilizing print statements for debugging:
                \begin{lstlisting}[language=Python]
def divide(x, y):
    print(f"Dividing {x} by {y}")
    return x / y

# Error Handling
try:
    result = divide(10, 0)
except ZeroDivisionError:
    print("Cannot divide by zero!")
                \end{lstlisting}
            \end{itemize}

        \item \textbf{Implement Troubleshooting Strategies}
            \begin{itemize}
                \item Develop a systematic approach to isolate and fix bugs.
                \item Understand how to analyze error messages and warnings effectively.
                \item \textbf{Key Point}: A structured approach helps reduce frustration.
            \end{itemize}
    \end{enumerate}
\end{frame}

\begin{frame}{Learning Objectives - Part 3}
    \begin{enumerate}[start=4]
        \item \textbf{Optimize Code Performance}
            \begin{itemize}
                \item Recognize the importance of performance metrics in coding.
                \item Learn to analyze and improve the efficiency of your code.
                \item \textbf{Illustration}: Understanding Big O notation:
                    \begin{itemize}
                        \item Linear Search: $O(n)$
                        \item Binary Search: $O(\log n)$
                    \end{itemize}
            \end{itemize}
    \end{enumerate}
    
    \begin{block}{Conclusion}
        By the end of this week, you will be equipped to write, refine, and diagnose code effectively.
    \end{block}
\end{frame}

\begin{frame}{Next Steps}
    Ready to dive into hands-on coding assistance in the next section!
\end{frame}

\begin{frame}[fragile]
  \frametitle{Hands-On Coding Assistance}
  Focuses on providing coding support during lab sessions, including debugging techniques.
\end{frame}

\begin{frame}[fragile]
  \frametitle{Coding Support in Lab Sessions}
  \begin{itemize}
    \item Hands-on experience helps students apply theoretical knowledge.
    \item Immediate assistance with coding challenges for real-time progress.
    \item Constructive feedback is provided throughout lab sessions.
  \end{itemize}
\end{frame}

\begin{frame}[fragile]
  \frametitle{Debugging Techniques}
  \begin{block}{Common Debugging Techniques}
    \begin{enumerate}
      \item \textbf{Print Statements}:
        \begin{lstlisting}[language=python]
def factorial(n):
    print(f"Calculating factorial of {n}")
    if n == 1:
        return 1
    else:
        return n * factorial(n - 1)
        \end{lstlisting}
      
      \item \textbf{Using a Debugger}:
        \begin{itemize}
          \item Built-in debuggers in IDEs allow step-by-step code execution.
          \item Enables inspection of variable values.
        \end{itemize}
        
      \item \textbf{Code Reviews}:
        \begin{itemize}
          \item Collaborate with peers to catch mistakes or suggest improvements.
        \end{itemize}
        
      \item \textbf{Console Logs and Error Messages}:
        \begin{itemize}
          \item Examining error messages provides clues about bugs.
        \end{itemize}
    \end{enumerate}
  \end{block}
\end{frame}

\begin{frame}[fragile]
  \frametitle{Tips for Effective Debugging}
  \begin{itemize}
    \item \textbf{Replicate the Error}: Understand error conditions to gain insights.
    \item \textbf{Simplify Your Code}: Reduce complexity to isolate the problem.
    \item \textbf{Document Findings}: Keep a debug log of errors and resolutions.
  \end{itemize}
\end{frame}

\begin{frame}[fragile]
  \frametitle{Coding Example: Division}
  \begin{lstlisting}[language=python]
def divide_numbers(a, b):
    print(f"Dividing {a} by {b}")
    return a / b

result = divide_numbers(10, 0)  # This will cause an error
  \end{lstlisting}
  
  \begin{block}{Debugging}
    \begin{itemize}
      \item The above code will raise a \texttt{ZeroDivisionError}.
      \item To fix, implement exception handling:
      \begin{lstlisting}[language=python]
def divide_numbers(a, b):
    try:
        return a / b
    except ZeroDivisionError:
        print("Error: Division by zero is not allowed.")
        return None
      \end{lstlisting}
    \end{itemize}
  \end{block}
\end{frame}

\begin{frame}[fragile]
  \frametitle{Key Points to Emphasize}
  \begin{itemize}
    \item Utilize coding support during labs to enhance problem-solving skills.
    \item Master debugging techniques for effective troubleshooting.
    \item Collaborate and learn from peers for skill enhancement.
  \end{itemize}
\end{frame}

\begin{frame}[fragile]
  \frametitle{Conclusion}
  Hands-on coding assistance is essential for mastering coding. Employing debugging techniques enables students to overcome obstacles and complete projects successfully.
\end{frame}

\begin{frame}[fragile]
    \frametitle{Common Coding Challenges - Overview}
    \begin{block}{Overview}
        When developing projects using big data frameworks like Hadoop and Spark, programmers often encounter specific coding challenges. Understanding these pitfalls can significantly enhance the efficiency and success of your projects.
    \end{block}
\end{frame}

\begin{frame}[fragile]
    \frametitle{Common Coding Challenges - Part 1}
    \begin{enumerate}
        \item \textbf{Data Skew}
            \begin{itemize}
                \item \textbf{Explanation}: Data skew occurs when certain keys have significantly more records than others, leading to uneven load distribution among nodes.
                \item \textbf{Example}: In a sales dataset, if most sales are attributed to a few popular products, processing those can take much longer than handling other items.
                \item \textbf{Mitigation}: Use techniques like salting (adding random values to keys) to distribute data more evenly.
            \end{itemize}
        
        \item \textbf{Inefficient Use of Resources}
            \begin{itemize}
                \item \textbf{Explanation}: Not utilizing cluster resources effectively (e.g., CPU, memory) can lead to underperformance.
                \item \textbf{Example}: Running Spark jobs without controlling the level of parallelism can overwhelm certain nodes while leaving others underutilized.
                \item \textbf{Mitigation}: Set the appropriate parallelism level using Spark's \texttt{setParallelism()} method.
                \begin{lstlisting}[language=python]
spark.conf.set("spark.default.parallelism", 4)
                \end{lstlisting}
            \end{itemize}
    \end{enumerate}
\end{frame}

\begin{frame}[fragile]
    \frametitle{Common Coding Challenges - Part 2}
    \begin{enumerate}
        \setcounter{enumi}{2}
        
        \item \textbf{Ignoring Data Locality}
            \begin{itemize}
                \item \textbf{Explanation}: Data locality refers to processing data on the node where it is stored to reduce network I/O.
                \item \textbf{Example}: If a job processes data stored on HDFS without consideration of data locality, it may incur higher latency.
                \item \textbf{Mitigation}: Leverage Hadoop's resource manager to schedule tasks closer to data locations.
            \end{itemize}
        
        \item \textbf{Misconfigurations in Cluster Settings}
            \begin{itemize}
                \item \textbf{Explanation}: Incorrectly configured settings like memory limits can lead to job failures.
                \item \textbf{Example}: Setting the executor memory too low may cause OutOfMemory errors.
                \item \textbf{Mitigation}: Review and adjust configurations in \texttt{spark-defaults.conf} based on workload demands.
                \begin{lstlisting}
spark.executor.memory=4g
                \end{lstlisting}
            \end{itemize}
        
        \item \textbf{Neglecting Data Quality Issues}
            \begin{itemize}
                \item \textbf{Explanation}: Poor data quality can lead to inaccuracies in analyses.
                \item \textbf{Example}: Inconsistent timestamp formats can lead to incorrect aggregations.
                \item \textbf{Mitigation}: Implement data validation and cleansing steps before processing.
            \end{itemize}
    \end{enumerate}
\end{frame}

\begin{frame}[fragile]
    \frametitle{Common Coding Challenges - Part 3}
    \begin{enumerate}
        \setcounter{enumi}{5}
        
        \item \textbf{Overlooking Exception Handling}
            \begin{itemize}
                \item \textbf{Explanation}: Failing to account for exceptions can lead to job crashes without informative error messages.
                \item \textbf{Example}: Not catching exceptions when accessing external data can halt a job unexpectedly.
                \item \textbf{Mitigation}: Use proper exception handling with try-catch blocks.
                \begin{lstlisting}[language=python]
try:
    df = spark.read.csv("data.csv")
except Exception as e:
    print(f"An error occurred: {e}")
                \end{lstlisting}
            \end{itemize}
    \end{enumerate}
        
    \begin{block}{Key Points to Emphasize}
        \begin{itemize}
            \item Awareness of common pitfalls enables proactive measures.
            \item Use best practices for data locality, resource utilization, and ensuring data quality.
            \item Flexibility in adjusting configurations and applying exception handling as necessary.
        \end{itemize}
    \end{block}
\end{frame}

\begin{frame}[fragile]
    \frametitle{Performance Optimization Techniques - Introduction}
    \begin{block}{Overview}
        In distributed systems, optimizing performance is crucial for:
        \begin{itemize}
            \item Efficiency
            \item Scalability
            \item Speed
        \end{itemize}
        Optimization ensures that code runs effectively, minimizing resource usage while maximizing output.
    \end{block}
\end{frame}

\begin{frame}[fragile]
    \frametitle{Performance Optimization Techniques - Key Concepts}
    \begin{enumerate}
        \item \textbf{Code Efficiency}:
        \begin{itemize}
            \item Efficient code uses resources (CPU, memory, I/O) effectively.
            \item Faster execution and reduced resource consumption.
        \end{itemize}

        \item \textbf{Distributed Computing}:
        \begin{itemize}
            \item Tasks are divided across multiple machines for handling large datasets.
            \item Optimization is vital due to variability in network speeds and capabilities.
        \end{itemize}
    \end{enumerate}
\end{frame}

\begin{frame}[fragile]
    \frametitle{Performance Optimization Techniques - Strategies}
    \begin{enumerate}
        \item \textbf{Algorithm Optimization}:
        \begin{itemize}
            \item Choose efficient algorithms (e.g., QuickSort vs. Bubble Sort).
        \end{itemize}

        \item \textbf{Data Serialization}:
        \begin{itemize}
            \item Use efficient serialization formats (e.g., Protocol Buffers).
        \end{itemize}

        \item \textbf{Data Partitioning}:
        \begin{itemize}
            \item Divide data for parallel processing (e.g., Apache Spark's \texttt{repartition()}).
        \end{itemize}

        \item \textbf{Caching and Persistence}:
        \begin{itemize}
            \item Cache frequently accessed data (e.g., using Spark's \texttt{cache()} or \texttt{persist()}).
        \end{itemize}

        \item \textbf{Network Optimization}:
        \begin{itemize}
            \item Process data locally to minimize data transfer; apply network compression.
        \end{itemize}
        
        \item \textbf{Resource Management}:
        \begin{itemize}
            \item Monitor and allocate resources using tools like Apache Mesos.
        \end{itemize}
    \end{enumerate}
\end{frame}

\begin{frame}[fragile]
    \frametitle{Performance Optimization Techniques - Best Practices}
    \begin{itemize}
        \item \textbf{Benchmarking}: Measure performance using relevant benchmarks.
        \item \textbf{Iterative Improvement}: Implement changes, measure impacts, and refine based on metrics.
        \item \textbf{Documentation}: Maintain clear documentation for reference and collaboration.
    \end{itemize}
\end{frame}

\begin{frame}[fragile]
    \frametitle{Performance Optimization Techniques - Conclusion}
    \begin{block}{Summary}
        Implementing performance optimization techniques is essential for efficient distributed systems. Focus on:
        \begin{itemize}
            \item Algorithm efficiency
            \item Data handling
            \item Resource management
        \end{itemize}
        Stay alert for new optimizations based on evolving technologies.
    \end{block}
\end{frame}

\begin{frame}[fragile]
    \frametitle{Next Steps}
    \begin{block}{Upcoming Topic}
        In our next slide, we will delve into \textbf{Identifying Performance Bottlenecks} to recognize where improvements can yield the most significant benefits.
    \end{block}
\end{frame}

\begin{frame}[fragile]
    \frametitle{Identifying Performance Bottlenecks}
    \begin{block}{Understanding Performance Bottlenecks}
        Performance bottlenecks are specific components that limit the throughput or efficiency of a system, particularly in big data applications. Identifying these issues is essential as they can lead to increased latency and reduced performance.
    \end{block}
\end{frame}

\begin{frame}[fragile]
    \frametitle{Key Areas to Analyze}
    \begin{enumerate}
        \item \textbf{Data I/O Operations}
            \begin{itemize}
                \item Example: Slow disk reads can lead to increased latency.
            \end{itemize}
        \item \textbf{Network Latency}
            \begin{itemize}
                \item Example: Delays in data transfer impact application responsiveness.
            \end{itemize}
        \item \textbf{CPU Utilization}
            \begin{itemize}
                \item Example: Expensive computations can monopolize CPU cycles.
            \end{itemize}
        \item \textbf{Memory Usage}
            \begin{itemize}
                \item Example: Insufficient memory causing excessive swapping or pauses.
            \end{itemize}
    \end{enumerate}
\end{frame}

\begin{frame}[fragile]
    \frametitle{Methods to Analyze Performance}
    \begin{enumerate}
        \item \textbf{Profiling Tools}
            \begin{itemize}
                \item Use tools like \textbf{Apache Spark UI} and \textbf{Java VisualVM} to monitor and visualize performance.
            \end{itemize}
        \item \textbf{Monitoring Metrics}
            \begin{itemize}
                \item Systems like \textbf{Prometheus} and \textbf{Grafana} can be used to track performance metrics.
            \end{itemize}
        \item \textbf{Logging}
            \begin{itemize}
                \item Use logging frameworks to monitor execution times.
                \begin{lstlisting}[language=Java]
long startTime = System.currentTimeMillis();
// Code to execute
long endTime = System.currentTimeMillis();
System.out.println("Execution Time: " + (endTime - startTime) + " ms");
                \end{lstlisting}
            \end{itemize}
        \item \textbf{Testing Under Load}
            \begin{itemize}
                \item Tools like \textbf{JMeter} or \textbf{Gatling} can simulate high demand on applications.
            \end{itemize}
    \end{enumerate}
\end{frame}

\begin{frame}[fragile]
    \frametitle{Key Points to Emphasize}
    \begin{itemize}
        \item \textbf{Identify and Isolate:} Monitor components individually to pinpoint bottleneck origins.
        \item \textbf{Iterative Improvement:} Optimization should be ongoing; re-test after modifications.
        \item \textbf{Holistic View:} Consider all components, as interactions can create unexpected bottlenecks.
    \end{itemize}
\end{frame}

\begin{frame}[fragile]
    \frametitle{Profiling and Benchmarking Code - Introduction}
    \begin{block}{Introduction}
        Profiling and benchmarking are crucial techniques for evaluating the performance of your code, especially in big data environments. Understanding how to measure the time and resources used helps in making informed decisions about performance improvements.
    \end{block}
\end{frame}

\begin{frame}[fragile]
    \frametitle{Profiling and Its Tools}
    \begin{block}{Profiling}
        Profiling measures space (memory) and time complexity of your code to identify performance bottlenecks.
    \end{block}
    \begin{itemize}
        \item \textbf{Tools}:
        \begin{itemize}
            \item \texttt{cProfile}, \texttt{line\_profiler} for Python
            \item VisualVM, YourKit for Java
            \item Apache Spark's UI and resource managers like YARN
        \end{itemize}
    \end{itemize}
    \begin{block}{Example}
        \begin{lstlisting}[language=Python]
import cProfile

def main():
    # Your code to execute
    pass

if __name__ == "__main__":
    cProfile.run('main()')
        \end{lstlisting}
        This code will show time spent in each function call for performance analysis.
    \end{block}
\end{frame}

\begin{frame}[fragile]
    \frametitle{Benchmarking and Key Points}
    \begin{block}{Benchmarking}
        Benchmarking involves running tests under controlled conditions to measure performance metrics.
    \end{block}
    \begin{itemize}
        \item \textbf{Types of Benchmarks}:
        \begin{itemize}
            \item Micro-benchmarks: Small and isolated units of code
            \item Macro-benchmarks: Performance of entire applications
        \end{itemize}
    \end{itemize}
    \begin{block}{Example}
        \begin{lstlisting}[language=Python]
import time

start_time = time.time()
# Code block to benchmark
execution_time = time.time() - start_time
print(f"Execution Time: {execution_time} seconds")
        \end{lstlisting}
    \end{block}
    
    \begin{block}{Key Points}
        \begin{itemize}
            \item Profiling and benchmarking are iterative processes.
            \item Performance varies based on dataset size.
            \item Leverage libraries like \texttt{timeit} and \texttt{pytest-benchmark}.
        \end{itemize}
    \end{block}
\end{frame}

\begin{frame}[fragile]
    \frametitle{Hands-On Lab Session}
    \begin{block}{Description}
        Interactive session where students troubleshoot their projects with direct support from instructors and TAs.
    \end{block}
\end{frame}

\begin{frame}[fragile]
    \frametitle{Objectives of the Lab Session}
    \begin{itemize}
        \item Provide an opportunity for hands-on troubleshooting of projects.
        \item Enhance understanding of project development processes through real-time support and feedback.
        \item Foster collaboration amongst peers and encourage group problem-solving.
    \end{itemize}
\end{frame}

\begin{frame}[fragile]
    \frametitle{Key Concepts}
    \begin{enumerate}
        \item \textbf{Troubleshooting Methodology}
        \begin{itemize}
            \item Identify the Issue
            \item Analyze the Problem
            \item Develop Hypotheses
            \item Test Hypotheses
        \end{itemize}
        
        \item \textbf{Collaborative Problem-Solving}
        \begin{itemize}
            \item Encourage peer discussions.
            \item Utilize instructors and TAs for expertise.
        \end{itemize}
    \end{enumerate}
\end{frame}

\begin{frame}[fragile]
    \frametitle{Example Scenario}
    \textbf{Problem:} A student is unable to process a large dataset due to memory errors.\\
    \textbf{Troubleshooting Steps:}
    \begin{enumerate}
        \item Identify: The error message indicates memory overflow.
        \item Analyze: Review the data handling section. Is the dataset too large?
        \item Hypotheses: 
            \begin{itemize}
                \item Hypothesis 1: Data is not filtered correctly.
                \item Hypothesis 2: Inefficient data structures are used (e.g., using lists instead of data frames).
            \end{itemize}
        \item Test: 
            \begin{itemize}
                \item Modify code to filter data to a smaller size.
                \item Test performance with smaller datasets first.
            \end{itemize}
    \end{enumerate}
\end{frame}

\begin{frame}[fragile]
    \frametitle{Key Points to Emphasize}
    \begin{itemize}
        \item Importance of isolating each issue before diving into the code.
        \item Value of asking questions and engaging in dialogue with peers and instructors.
        \item Encourage the use of debugging tools (e.g., loggers, print statements).
    \end{itemize}
\end{frame}

\begin{frame}[fragile]
    \frametitle{Troubleshooting Tools and Techniques}
    \begin{itemize}
        \item \textbf{Debugging Tools:} Use built-in features in IDEs for step-by-step execution.
        \item \textbf{Logging:} Implement logging to track execution flow and catch errors.
        \item \textbf{Peer Review:} Regularly review code for potential improvements.
    \end{itemize}
\end{frame}

\begin{frame}[fragile]
    \frametitle{Conclusion and Next Steps}
    \begin{block}{Conclusion}
        Utilize this hands-on lab session to enhance your project through collaborative troubleshooting.
        Remember the goal is to learn how to identify and prevent future coding problems.
    \end{block}
    
    \textbf{Next Steps:} Prepare for the upcoming session on Best Practices in Coding for Big Data.
\end{frame}

\begin{frame}[fragile]
    \frametitle{Best Practices in Coding for Big Data - Introduction}
    \begin{block}{Introduction}
        In the realm of big data projects, writing clean and maintainable code is vital for success. 
        Implementing best practices enhances efficiency and reduces the likelihood of errors. Below are key coding conventions and practices to adopt.
    \end{block}
\end{frame}

\begin{frame}[fragile]
    \frametitle{Best Practices in Coding for Big Data - Key Practices}
    \begin{enumerate}
        \item \textbf{Code Readability and Documentation}
        \begin{itemize}
            \item Write clear, descriptive code with meaningful names. 
            \item Document your code with comments and summaries.
            \item Use code style tools for consistency.
        \end{itemize}

        \item \textbf{Modular Code Structure}
        \begin{itemize}
            \item Break down code into functions/modules for reusability.
            \item Encapsulate logic with classes and objects.
        \end{itemize}

        \item \textbf{Error Handling and Validation}
        \begin{itemize}
            \item Implement robust error handling using try and except blocks.
            \item Validate input data before processing.
        \end{itemize}

        \item \textbf{Version Control and Collaboration}
        \begin{itemize}
            \item Use version control systems like Git to track changes and facilitate collaboration.
            \item Employ branching strategies to manage development work.
        \end{itemize}
    \end{enumerate}
\end{frame}

\begin{frame}[fragile]
    \frametitle{Best Practices in Coding for Big Data - Continued}
    \begin{enumerate}[resume]
        \item \textbf{Performance Optimization}
        \begin{itemize}
            \item Leverage built-in functions and libraries like Pandas or Dask.
            \item Profile and monitor code performance to identify bottlenecks.
        \end{itemize}

        \item \textbf{Testing and Validation}
        \begin{itemize}
            \item Write unit tests using frameworks like pytest to ensure correctness.
            \begin{lstlisting}[language=Python]
def test_calculate_average_salary():
    assert calculate_average_salary([60000, 70000]) == 65000
            \end{lstlisting}
            \item Conduct integration testing to ensure modules work well together.
        \end{itemize}
    \end{enumerate}

    \begin{block}{Key Points to Emphasize}
        - Clear and maintainable code facilitates easier future troubleshooting. \\
        - Modular structures enhance project organization. \\
        - Robust error handling reduces runtime errors. \\
        - Version control fosters collaboration and responsibility.
    \end{block}
\end{frame}

\begin{frame}
    \frametitle{Collaborative Troubleshooting}
    \begin{block}{Explanation}
        Collaborative troubleshooting is crucial in coding, especially in big data projects. It involves teamwork to identify, analyze, and solve issues, fostering diverse perspectives and enhancing problem-solving abilities.
    \end{block}
    
    \begin{block}{Why Collaborate?}
        \begin{itemize}
            \item \textbf{Diverse Skill Sets:} Unique strengths from team members lead to more innovative solutions.
            \item \textbf{Increased Efficiency:} Teamwork speeds up troubleshooting via workload distribution.
            \item \textbf{Learning Opportunities:} Individuals learn from each other, enhancing overall competence.
        \end{itemize}
    \end{block}
\end{frame}

\begin{frame}
    \frametitle{Collaborative Troubleshooting - Example Scenario}
    \begin{block}{Project Context}
        A team develops a big data processing application and faces a bug that causes crashes during data input.
    \end{block}
    
    \begin{block}{Collaborative Approach}
        \begin{enumerate}
            \item \textbf{Identify the Problem:} Document symptoms (error messages, logs).
            \item \textbf{Brainstorm Solutions:} Suggest potential causes based on experience.
            \item \textbf{Distribute the Investigation:} Assign tasks to check various aspects.
            \item \textbf{Share Findings:} Convene to discuss findings and determine fixes.
        \end{enumerate}
    \end{block}
\end{frame}

\begin{frame}[fragile]
    \frametitle{Collaborative Troubleshooting - Code Example}
    Here’s a simple example of how collaborative troubleshooting might appear in code:

    \begin{lstlisting}[language=Python]
# Initial Function to Process Data (Problematic)
def process_data(input_data):
    result = []
    for item in input_data:
        if not isinstance(item, int):  # Check for data type
            raise ValueError("Invalid input type.")   # Potential source of error
        result.append(item * 2)
    return result

# Revised Function after Collaboration & Troubleshooting
def process_data(input_data):
    if not input_data:  # Check for empty input
        return []
    
    result = []
    for item in input_data:
        try:
            if not isinstance(item, int):
                raise TypeError("All items must be integers.")
            result.append(item * 2)
        except Exception as e:
            print(f"Error processing item {item}: {e}")
    return result
    \end{lstlisting}
\end{frame}

\begin{frame}
    \frametitle{Real-Time Q\&A Session}
    \begin{block}{Purpose of the Q\&A Session}
        The Real-Time Q\&A session provides an interactive platform for students to ask questions, share coding experiences, and seek clarification on coding challenges related to their projects. This is an opportunity to engage with peers and instructors in a collaborative learning environment.
    \end{block}
\end{frame}

\begin{frame}
    \frametitle{Key Concepts to Explore}
    \begin{enumerate}
        \item \textbf{Clarification of Coding Challenges:}
            \begin{itemize}
                \item Identify specific issues students may be encountering in their coding projects.
                \item Encourage students to formulate questions that target their areas of confusion or difficulty.
            \end{itemize}
        
        \item \textbf{Creative Problem-Solving:}
            \begin{itemize}
                \item Discuss different approaches to overcome obstacles in coding.
                \item Share alternative solutions that students may not have considered.
            \end{itemize}
        
        \item \textbf{Peer Knowledge Sharing:}
            \begin{itemize}
                \item Utilize the collective knowledge of the group to enhance understanding.
                \item Encourage students to share their experiences and solutions to similar issues.
            \end{itemize}
    \end{enumerate}
\end{frame}

\begin{frame}
    \frametitle{Examples of Typical Questions}
    \begin{itemize}
        \item "I'm encountering a syntax error in my code. Can you help me identify what's wrong?"
        \item "How can I optimize my code for better performance?"
        \item "What are some strategies for debugging effectively?"
    \end{itemize}

    \begin{block}{Key Points to Emphasize}
        \begin{itemize}
            \item \textbf{Active Participation:} Encourage all students to participate by asking questions or sharing insights.
            \item \textbf{Respectful Communication:} Remind students to be respectful when discussing solutions or offering feedback.
            \item \textbf{Documentation Reference:} Guide students to refer to official documentation for more comprehensive answers.
        \end{itemize}
    \end{block}
\end{frame}

\begin{frame}[fragile]
    \frametitle{Framework for Asking Questions}
    \begin{enumerate}
        \item \textbf{Describe the Problem:}
            Clearly articulate the specific coding challenge or error message.
        
        \item \textbf{Share Your Code:}
            Provide snippets of code where the problem occurs to give context.
            \begin{lstlisting}[language=Python]
# Example: Function that may not be returning the expected output
def calculate_area(radius):
    return 3.14 * radius * radius

# Invoking the function incorrectly
area = calculate_area("five")  # This will cause a TypeError
            \end{lstlisting}
        
        \item \textbf{Specify What You've Tried:}
            Mention any troubleshooting steps already undertaken to resolve the issue.
    \end{enumerate}
\end{frame}

\begin{frame}
    \frametitle{Conclusion}
    The Real-Time Q\&A aims to harness the collective knowledge of the class to enhance understanding of project development and troubleshoot effectively. This session is an integral part of the learning process, allowing for real-time guidance and support.
    
    \begin{block}{Reminder}
        Maintain a friendly and supportive atmosphere where no question is too simple. Your inquiry could help a classmate as well!
    \end{block}
\end{frame}

\begin{frame}[fragile]
  \frametitle{Resources for Further Learning - Introduction}
  \begin{block}{Introduction}
    As you embark on your project development journey, having access to the right resources can greatly enhance your effectiveness and efficiency. Below is a curated list of invaluable resources that cater to different aspects of project development, specifically focusing on documentation, online forums, and community support.
  \end{block}
\end{frame}

\begin{frame}[fragile]
  \frametitle{Resources for Further Learning - Documentation}
  \begin{block}{Documentation}
    Understanding official documentation is crucial for mastering programming languages, tools, and frameworks. Here are key examples:
  \end{block}
  
  \begin{enumerate}
    \item \textbf{Programming Languages}
      \begin{itemize}
        \item \textbf{Python Documentation}: \texttt{https://docs.python.org/3/}
        \item \textbf{Java Documentation}: \texttt{https://docs.oracle.com/en/java/javase/11/docs/api/index.html}
      \end{itemize}
    
    \item \textbf{Frameworks and Libraries}
      \begin{itemize}
        \item \textbf{Django Documentation}: \texttt{https://docs.djangoproject.com/}
        \item \textbf{React Documentation}: \texttt{https://reactjs.org/docs/getting-started.html}
      \end{itemize}
    
    \item \textbf{Data Science Libraries}
      \begin{itemize}
        \item \textbf{Pandas Documentation}: \texttt{https://pandas.pydata.org/docs/}
        \item \textbf{TensorFlow Documentation}: \texttt{https://www.tensorflow.org/learn}
      \end{itemize}
  \end{enumerate}
\end{frame}

\begin{frame}[fragile]
  \frametitle{Resources for Further Learning - Online Forums}
  \begin{block}{Online Forums}
    Engaging with communities can provide you with insights, problem-solving help, and collaboration opportunities:
  \end{block}
  
  \begin{enumerate}
    \item \textbf{Stack Overflow}
      \begin{itemize}
        \item A huge Q\&A platform for developers. Remember to search for similar issues before posting! \texttt{(https://stackoverflow.com)}
      \end{itemize}
  
    \item \textbf{GitHub Discussions}
      \begin{itemize}
        \item Many open-source projects use GitHub Discussions for community questions and feedback.
      \end{itemize}
  
    \item \textbf{Reddit}
      \begin{itemize}
        \item Subreddits like r/learnprogramming and r/datascience offer great advice and learning resources.
      \end{itemize}
  \end{enumerate}
\end{frame}

\begin{frame}[fragile]
  \frametitle{Resources for Further Learning - Learning Platforms}
  \begin{block}{Learning Platforms}
    Consider using the following platforms to supplement your learning:
  \end{block}
  
  \begin{enumerate}
    \item \textbf{Coursera}
      \begin{itemize}
        \item Offers courses from universities and companies on various topics, including project management and software development.
      \end{itemize}
  
    \item \textbf{edX}
      \begin{itemize}
        \item Provides a variety of courses on software engineering, data analysis, and more from renowned institutions.
      \end{itemize}

    \item \textbf{Udacity}
      \begin{itemize}
        \item Focused on tech skills, particularly data science and web development, through Nanodegree programs offering projects and mentorship.
      \end{itemize}
  \end{enumerate}
\end{frame}

\begin{frame}[fragile]
  \frametitle{Resources for Further Learning - Key Takeaways}
  \begin{block}{Key Takeaways}
    \begin{itemize}
      \item Utilize official documentation to understand tools and languages thoroughly.
      \item Engage in online forums for community support and to clarify concepts.
      \item Explore learning platforms for structured courses to deepen your knowledge.
    \end{itemize}
  \end{block}
  
  \begin{block}{Conclusion}
    These resources are designed to empower you in your project development endeavors. Don't hesitate to leverage them to overcome challenges and expand your skill set effectively!
  \end{block}
\end{frame}

\begin{frame}[fragile]
    \frametitle{Project Development Guidelines - Overview}
    \begin{block}{Key Components for Successfully Developing a Large-Scale Project in a Big Data Context}
        \begin{enumerate}
            \item Defining Project Scope and Objectives
            \item Data Collection and Preparation
            \item Choosing Appropriate Tools and Technologies
            \item Creating a Development Timeline
            \item Testing and Validation Processes
            \item Documentation and Reporting
            \item Continuous Feedback and Iteration
            \item Deployment and Maintenance
        \end{enumerate}
    \end{block}
\end{frame}

\begin{frame}[fragile]
    \frametitle{Project Development Guidelines - Details 1}
    \begin{enumerate}
        \item \textbf{Defining Project Scope and Objectives}
            \begin{itemize}
                \item Clearly outline project aims and establish measurable goals.
                \item \textbf{Example:} Predicting customer behavior such as product purchase likelihood.
            \end{itemize}

        \item \textbf{Data Collection and Preparation}
            \begin{itemize}
                \item Identify relevant data sources for quality data.
                \item Implement data cleaning and transformation processes.
                \item \textbf{Key Point:} A well-prepared dataset is crucial for reliable outcomes.
            \end{itemize}
    \end{enumerate}
\end{frame}

\begin{frame}[fragile]
    \frametitle{Project Development Guidelines - Details 2}
    \begin{enumerate}
        \setcounter{enumi}{2} % Start from the third item
        \item \textbf{Choosing Appropriate Tools and Technologies}
            \begin{itemize}
                \item Select tools based on project requirements and team expertise.
                \item \textbf{Example:} Use Apache Kafka for real-time analytics.
            \end{itemize}

        \item \textbf{Creating a Development Timeline}
            \begin{itemize}
                \item Develop a timeline with milestones using iterative development cycles like Agile.
            \end{itemize}
            \begin{lstlisting}[language=Python]
# Tracking project timeline with milestones:
milestones = {
    'Phase 1': 'Data Collection',
    'Phase 2': 'Data Processing',
    'Phase 3': 'Model Development',
    'Phase 4': 'Testing & Optimization',
    'Phase 5': 'Final Review & Deployment'
}
            \end{lstlisting}
    \end{enumerate}
\end{frame}

\begin{frame}[fragile]
    \frametitle{Feedback Mechanisms - Importance}
    \begin{block}{Importance of Feedback in Project Development}
        Feedback is essential in the project development process, as it enables continuous improvement and innovation. By sharing their experiences, students can:
        \begin{itemize}
            \item Contribute to refining project guidelines
            \item Enhance peer collaboration
            \item Inform future iterations of course content
        \end{itemize}
    \end{block}
\end{frame}

\begin{frame}[fragile]
    \frametitle{Feedback Mechanisms - Types}
    \begin{block}{Types of Feedback Mechanisms}
        \begin{enumerate}
            \item \textbf{Surveys \& Questionnaires}
                \begin{itemize}
                    \item \textit{What?} Structured tools with specific questions regarding various aspects of the project.
                    \item \textit{Example:} Rating the clarity of project guidelines.
                    \item \textit{Key Point:} Use Likert scales (1-5 rating) for quantifiable insights.
                \end{itemize}

            \item \textbf{Mid-Project Reviews}
                \begin{itemize}
                    \item \textit{What?} Scheduled feedback sessions for presenting progress.
                    \item \textit{Example:} Peer review session for project status.
                    \item \textit{Key Point:} Encourages collaboration and diverse perspectives.
                \end{itemize}

            \item \textbf{One-on-One Meetings}
                \begin{itemize}
                    \item \textit{What?} Discussions with instructors for specific challenges.
                    \item \textit{Example:} Addressing technical hurdles or group dynamics.
                    \item \textit{Key Point:} Supports personalized guidance.
                \end{itemize}

            \item \textbf{Discussion Boards \& Forums}
                \begin{itemize}
                    \item \textit{What?} Online platforms for posting feedback, questions, and suggestions.
                    \item \textit{Example:} Dedicated thread for data analysis difficulties.
                    \item \textit{Key Point:} Fosters a community of support.
                \end{itemize}

            \item \textbf{Reflective Journals}
                \begin{itemize}
                    \item \textit{What?} Regular entries documenting experiences and insights.
                    \item \textit{Example:} Journaling teamwork dynamics.
                    \item \textit{Key Point:} Enhances self-awareness.
                \end{itemize}
        \end{enumerate}
    \end{block}
\end{frame}

\begin{frame}[fragile]
    \frametitle{Feedback Mechanisms - Best Practices and Conclusion}
    \begin{block}{Best Practices for Providing Feedback}
        \begin{itemize}
            \item \textbf{Be Specific:} Offer detailed and actionable suggestions.
            \item \textbf{Be Constructive:} Focus on solutions, not just problems.
            \item \textbf{Anonymous Options:} Encourage honest feedback.
            \item \textbf{Regular Feedback Loops:} Implement feedback mechanisms at various stages.
        \end{itemize}
    \end{block}

    \begin{block}{Conclusion}
        Establishing robust feedback mechanisms is vital for successful project development. By participating in these processes, students enhance their learning experiences and prepare for future projects. Encourage openness to foster a collaborative learning environment.
    \end{block}

    \begin{quote}
        Effective feedback not only hones project outcomes but also strengthens your skills as a reflective practitioner!
    \end{quote}
\end{frame}

\begin{frame}[fragile]
    \frametitle{Reflection and Next Steps - Introduction}
    \begin{block}{Importance of Reflection}
        Reflecting on your learning is crucial after a challenging phase of project development.
        \begin{itemize}
            \item Assess accomplishments
            \item Identify areas for improvement
            \item Consolidate insights
        \end{itemize}
    \end{block}
\end{frame}

\begin{frame}[fragile]
    \frametitle{Reflection and Next Steps - Key Concepts}
    \begin{enumerate}
        \item \textbf{Learning from Experience}
            \begin{itemize}
                \item What worked well and what did not
                \item Techniques and methodologies used
            \end{itemize}
        
        \item \textbf{Problem-Solving Insights}
            \begin{itemize}
                \item Troubleshooting experiences
                \item Changes in approach to resolve issues
            \end{itemize}
        
        \item \textbf{Feedback Application}
            \begin{itemize}
                \item Mechanisms to integrate feedback from earlier stages
            \end{itemize}
    \end{enumerate}
\end{frame}

\begin{frame}[fragile]
    \frametitle{Reflection and Next Steps - Planning for Next Steps}
    \begin{enumerate}
        \item \textbf{Identifying Improvements}
            \begin{itemize}
                \item Outline three aspects for improvement
            \end{itemize}
        
        \item \textbf{Actionable Goals}
            \begin{itemize}
                \item Create specific and achievable goals for final project
                \item Example:
                    \begin{itemize}
                        \item Improve documentation clarity
                        \item Incorporate more user testing sessions
                    \end{itemize}
            \end{itemize}
        
        \item \textbf{Implementing Feedback}
            \begin{itemize}
                \item Plan concrete application of feedback
            \end{itemize}
    \end{enumerate}
    \begin{block}{Conclusion}
        Reflection enhances learning and project quality.
    \end{block}
\end{frame}

\begin{frame}[fragile]
    \frametitle{Conclusion \& Summary - Key Takeaways}
    \begin{enumerate}
        \item \textbf{Understanding Project Development Phases}
        \begin{itemize}
            \item \textbf{Planning:} Establish clear goals and define the project scope.
            \item \textbf{Execution:} Implement the project plan and coordinate tasks.
            \item \textbf{Monitoring \& Control:} Use metrics to evaluate performance.
        \end{itemize}
        
        \item \textbf{Effective Troubleshooting Strategies}
        \begin{itemize}
            \item \textbf{Identify the Problem:} Gather and analyze data.
            \item \textbf{Develop Solutions:} Brainstorm and evaluate options.
            \item \textbf{Implement and Review:} Apply solutions and monitor results.
        \end{itemize}
        
        \item \textbf{Collaboration and Communication}
        \begin{itemize}
            \item Foster open communication among team members.
            \item Regularly engage stakeholders and keep them informed.
        \end{itemize}
    \end{enumerate}
\end{frame}

\begin{frame}[fragile]
    \frametitle{Conclusion \& Summary - Documentation \& Final Thoughts}
    \begin{block}{Documentation Matters}
        Maintaining thorough documentation throughout all phases is crucial for clarity and continuity. Detailed records aid in future troubleshooting and project planning.
    \end{block}

    \textbf{Final Thoughts:} Project development and troubleshooting are iterative processes requiring planning, flexibility, and communication. Every project is a learning opportunity!
\end{frame}

\begin{frame}[fragile]
    \frametitle{Conclusion \& Summary - Quick Tips for Success}
    \begin{itemize}
        \item Review your project plan regularly to ensure alignment with goals.
        \item Be proactive in identifying potential issues before they escalate into significant problems.
        \item Encourage feedback from team members to foster a collaborative environment.
    \end{itemize}

    By applying these principles, you enhance both project outcomes and problem-solving abilities in future endeavors!
\end{frame}


\end{document}