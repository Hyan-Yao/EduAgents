\documentclass[aspectratio=169]{beamer}

% Theme and Color Setup
\usetheme{Madrid}
\usecolortheme{whale}
\useinnertheme{rectangles}
\useoutertheme{miniframes}

% Additional Packages
\usepackage[utf8]{inputenc}
\usepackage[T1]{fontenc}
\usepackage{graphicx}
\usepackage{booktabs}
\usepackage{listings}
\usepackage{amsmath}
\usepackage{amssymb}
\usepackage{xcolor}
\usepackage{tikz}
\usepackage{pgfplots}
\pgfplotsset{compat=1.18}
\usetikzlibrary{positioning}
\usepackage{hyperref}

% Custom Colors
\definecolor{myblue}{RGB}{31, 73, 125}
\definecolor{mygray}{RGB}{100, 100, 100}
\definecolor{mygreen}{RGB}{0, 128, 0}
\definecolor{myorange}{RGB}{230, 126, 34}
\definecolor{mycodebackground}{RGB}{245, 245, 245}

% Set Theme Colors
\setbeamercolor{structure}{fg=myblue}
\setbeamercolor{frametitle}{fg=white, bg=myblue}
\setbeamercolor{title}{fg=myblue}
\setbeamercolor{section in toc}{fg=myblue}
\setbeamercolor{item projected}{fg=white, bg=myblue}
\setbeamercolor{block title}{bg=myblue!20, fg=myblue}
\setbeamercolor{block body}{bg=myblue!10}
\setbeamercolor{alerted text}{fg=myorange}

% Set Fonts
\setbeamerfont{title}{size=\Large, series=\bfseries}
\setbeamerfont{frametitle}{size=\large, series=\bfseries}
\setbeamerfont{caption}{size=\small}
\setbeamerfont{footnote}{size=\tiny}

% Document Start
\begin{document}

\frame{\titlepage}

\begin{frame}[fragile]
    \frametitle{Introduction to Course Reflection - Overview}
    \begin{block}{Objective of Course Reflection}
        Course reflection provides an opportunity for students to critically evaluate their learning process, assess their understanding of content, and identify areas for improvement. By engaging in this reflective practice, students can synthesize their knowledge, recognize their growth, and articulate their experiences throughout the course.
    \end{block}
\end{frame}

\begin{frame}[fragile]
    \frametitle{Introduction to Course Reflection - Importance}
    \begin{itemize}
        \item \textbf{Enhances Learning Retention:}
        \begin{itemize}
            \item Reflecting on learned material helps reinforce knowledge and connect concepts.
        \end{itemize}

        \item \textbf{Promotes Self-Assessment:}
        \begin{itemize}
            \item Students can gauge their understanding with questions such as:
            \begin{itemize}
                \item What concepts did I grasp well?
                \item Which topics need further review?
            \end{itemize}
        \end{itemize}

        \item \textbf{Encourages Critical Thinking:}
        \begin{itemize}
            \item Reflection fosters analysis of materials, e.g., evaluating Hadoop vs. Spark.
        \end{itemize}

        \item \textbf{Facilitates Feedback for Improvement:}
        \begin{itemize}
            \item Leads to constructive feedback through peer discussions or evaluations.
        \end{itemize}
    \end{itemize}
\end{frame}

\begin{frame}[fragile]
    \frametitle{Introduction to Course Reflection - Key Points}
    \begin{itemize}
        \item \textbf{Reflection is a Continuous Process:}
        \begin{itemize}
            \item Learning doesn’t end with a final exam; regular reflection supports lifelong learning.
        \end{itemize}

        \item \textbf{Create a Reflective Journal:}
        \begin{itemize}
            \item Tracking progress and thoughts over time can enhance understanding.
        \end{itemize}

        \item \textbf{Ask Thought-Provoking Questions:}
        \begin{itemize}
            \item Consider questions like:
            \begin{itemize}
                \item How has my understanding of data analytics evolved?
                \item What real-world applications can I draw from the theories learned?
            \end{itemize}
        \end{itemize}
    \end{itemize}
\end{frame}

\begin{frame}[fragile]{Course Highlights - Big Data Architecture}
  \frametitle{Key Topics Covered}
  
  \begin{enumerate}
    \item \textbf{Big Data Architecture}
    \begin{itemize}
      \item \textbf{Definition}: Design framework for large data collection and analysis.
      \item \textbf{Components}: Data sources, storage, processing layer, analytics applications.
      \item \textbf{Example}: Retail company analyzing customer buying patterns to optimize stock levels.
    \end{itemize}
  \end{enumerate}
\end{frame}

\begin{frame}[fragile]{Course Highlights - Hadoop and Spark}
  \frametitle{Key Topics Covered (Continued)}
  
  \begin{enumerate}
    \setcounter{enumi}{1} % Start from the second point
    \item \textbf{Hadoop}
    \begin{itemize}
      \item \textbf{Overview}: Open-source framework for distributed data processing.
      \item \textbf{Key Components}:
      \begin{itemize}
        \item HDFS: Distributed file system.
        \item MapReduce: Programming model for processing data across clusters.
      \end{itemize}
      \item \textbf{Example}: Financial institution using Hadoop for fraud detection by processing transaction data.
    \end{itemize}

    \item \textbf{Spark}
    \begin{itemize}
      \item \textbf{Overview}: Unified analytics engine for big data with built-in modules.
      \item \textbf{Advantages}: Faster in-memory processing compared to Hadoop.
      \item \textbf{Example}: E-commerce platform using Spark Streaming for real-time recommendations.
    \end{itemize}
  \end{enumerate}
\end{frame}

\begin{frame}[fragile]{Course Highlights - Machine Learning and Conclusion}
  \frametitle{Key Topics Covered (Continued)}
  
  \begin{enumerate}
    \setcounter{enumi}{3} % Continue from previous list
    \item \textbf{Machine Learning}
    \begin{itemize}
      \item \textbf{Definition}: AI subset enabling systems to learn from data patterns.
      \item \textbf{Key Algorithms}:
      \begin{itemize}
        \item Supervised: Linear Regression, Decision Trees.
        \item Unsupervised: K-Means Clustering, PCA.
      \end{itemize}
      \item \textbf{Example}: Health tech company predicting effective treatments using patient records.
    \end{itemize}
    
    \item \textbf{Key Points to Emphasize}
    \begin{itemize}
      \item Integration enhances data processing capabilities.
      \item Hadoop and Spark serve complementary purposes.
      \item Machine learning enables predictive analytics and automation.
    \end{itemize}

    \item \textbf{Conclusion}
    \begin{itemize}
      \item Understanding these concepts prepares you for real-world applications in data science.
    \end{itemize}
  \end{enumerate}
\end{frame}

\begin{frame}[fragile]
    \frametitle{Lessons Learned - Overview}
    \begin{block}{Overview}
        As we wrap up this course, it’s essential to reflect on the key lessons learned that have shaped our understanding 
        of big data, its architecture, and the applications of data science. This reflection covers both technical proficiencies 
        and theoretical insights that will serve as a foundation for your future studies and professional work.
    \end{block}
\end{frame}

\begin{frame}[fragile]
    \frametitle{Lessons Learned - Technical Aspects}
    \begin{block}{1. Technical Aspects}
        \begin{itemize}
            \item \textbf{A. Understanding Big Data Architecture:}
                \begin{itemize}
                    \item \textbf{Concept:} Design of systems and tools for efficient data processing.
                    \item \textbf{Key Components:}
                    \begin{itemize}
                        \item Data Sources: IoT devices, social media, etc.
                        \item Data Storage Solutions: SQL/NoSQL databases, data lakes.
                        \item Processing Engines: Hadoop (batch), Spark (stream).
                    \end{itemize}
                \end{itemize}

            \item \textbf{B. Data Processing Frameworks:}
                \begin{itemize}
                    \item \textbf{Concept:} Streamlining data handling from ingestion to transformation.
                    \item \textbf{Key Tools:} 
                    \begin{itemize}
                        \item Hadoop: Efficient for large datasets; sharding for task distribution.
                        \item Apache Spark: In-memory processing speeds up analysis tasks.
                    \end{itemize}
                \end{itemize}
        \end{itemize}
    \end{block}
\end{frame}

\begin{frame}[fragile]
    \frametitle{Lessons Learned - Theoretical Aspects}
    \begin{block}{2. Theoretical Aspects}
        \begin{itemize}
            \item \textbf{A. Data Science Fundamentals:}
                \begin{itemize}
                    \item \textbf{Concept:} Principles of statistics, probability, and algorithms in data analysis and machine learning.
                    \item \textbf{Key Insight:} Difference between supervised and unsupervised learning is crucial.
                \end{itemize}
                \begin{itemize}
                    \item \textbf{Examples:}
                    \begin{itemize}
                        \item Supervised Learning: Predicting house prices based on features (size, location).
                        \item Unsupervised Learning: Clustering customers based on purchasing behavior.
                    \end{itemize}
                \end{itemize}
            \item \textbf{B. Ethical Considerations in Data Usage:}
                \begin{itemize}
                    \item \textbf{Concept:} Responsibility to use data ethically.
                    \item \textbf{Key Point:} Awareness of data privacy laws (e.g., GDPR) and algorithmic biases.
                \end{itemize}
        \end{itemize}
    \end{block}
\end{frame}

\begin{frame}[fragile]
    \frametitle{Lessons Learned - Key Takeaways}
    \begin{block}{3. Key Takeaways}
        \begin{itemize}
            \item Interoperability of Tools: Familiarity with various data processing tools enables seamless integration across platforms.
            \item Importance of Data Quality: Clean, well-structured data is paramount for accurate analysis and insights.
            \item Real-World Application: Translating theoretical knowledge into practical solutions reinforces the relevance of continuous learning in technology.
        \end{itemize}
    \end{block}
\end{frame}

\begin{frame}[fragile]
    \frametitle{Lessons Learned - Conclusion}
    \begin{block}{Conclusion}
        Reflecting on these lessons underscores the necessity of a balanced approach—technical skills complemented by theoretical 
        understanding. As you step into the real world or advanced studies, grasping these concepts will equip you to tackle 
        complex data challenges while ensuring ethical practices. Feel free to reach out with questions as we transition into 
        applying this knowledge in real-world scenarios in the next slide!
    \end{block}
\end{frame}

\begin{frame}[fragile]
    \frametitle{Application of Knowledge - Overview}
    This slide explores how the knowledge gained throughout this course can be applied in various real-world scenarios. 

    \begin{itemize}
        \item Emphasizes translating theoretical understanding into effective problem-solving in professional contexts.
        \item Focuses on practical applications of theories and tools discussed in the course.
    \end{itemize}
\end{frame}

\begin{frame}[fragile]
    \frametitle{Application of Knowledge - Concepts Explained}
    \begin{itemize}
        \item \textbf{Real-World Relevance}:
            \begin{itemize}
                \item Knowledge acquired can help identify, analyze, and solve real-world problems across diverse domains such as healthcare, finance, marketing, and engineering.
            \end{itemize}
        
        \item \textbf{Transferability of Skills}:
            \begin{itemize}
                \item Skills developed can be applied in various job roles, including Data Analyst, Data Scientist, Big Data Engineer, and IT Consultant.
            \end{itemize}
    \end{itemize}
\end{frame}

\begin{frame}[fragile]
    \frametitle{Application of Knowledge - Examples}
    \begin{itemize}
        \item \textbf{Healthcare Analytics}:
            \begin{itemize}
                \item Utilizing statistical models and machine learning techniques to predict patient outcomes.

                \begin{lstlisting}[language=Python]
import pandas as pd
from sklearn.model_selection import train_test_split
from sklearn.linear_model import LogisticRegression

# Example code for predicting hospital readmissions
data = pd.read_csv('hospital_data.csv')
X = data[['age', 'number_of_visits', 'diagnosis']]
y = data['readmission']

X_train, X_test, y_train, y_test = train_test_split(X, y, test_size=0.2, random_state=42)
model = LogisticRegression()
model.fit(X_train, y_train)
predictions = model.predict(X_test)
                \end{lstlisting}
            \end{itemize}
        \item \textbf{Financial Modeling}:
            \begin{itemize}
                \item Knowledge of time series analysis to forecast stock prices or economic indicators.
            \end{itemize}
        \item \textbf{Marketing Strategies}:
            \begin{itemize}
                \item Using data mining techniques to analyze customer behavior and preferences, leading to targeted advertising campaigns.
            \end{itemize}
    \end{itemize}
\end{frame}

\begin{frame}[fragile]
    \frametitle{Application of Knowledge - Key Points}
    \begin{itemize}
        \item \textbf{Integration of Learning}: 
            \begin{itemize}
                \item Ability to integrate technical tools and theoretical knowledge for innovative solutions.
            \end{itemize}
        \item \textbf{Skill Adaptation}:
            \begin{itemize}
                \item Knowledge gained is adaptable to various industries, enhancing employability and career advancement.
            \end{itemize}
        \item \textbf{Critical Thinking}:
            \begin{itemize}
                \item Application requires critical thinking to choose the right approach for varying situations.
            \end{itemize}
    \end{itemize}
\end{frame}

\begin{frame}[fragile]
    \frametitle{Application of Knowledge - Conclusion}
    The application of knowledge gained from this course is not limited to theoretical exercises; it is about making impactful, data-driven decisions in your respective fields. 

    \begin{itemize}
        \item Recognizing real-world applications prepares you for your future career.
        \item Contributes to your professional growth.
    \end{itemize}
\end{frame}

\begin{frame}[fragile]
    \frametitle{Big Data Architectures Review}
    \begin{block}{Overview of Big Data Architectures}
        Big data architectures are frameworks that facilitate the processing, storage, and analysis of vast amounts of data. In this course, we explored several key architectures, each with unique strengths and weaknesses.
    \end{block}
\end{frame}

\begin{frame}[fragile]
    \frametitle{Batch Processing Architectures}
    \begin{itemize}
        \item \textbf{Definition}: Processing of data in large sets at periodic intervals.
        \item \textbf{Examples}: 
            \begin{itemize}
                \item Apache Hadoop
                \item Apache Spark (when used for batch processing)
            \end{itemize}
        \item \textbf{Strengths}:
            \begin{itemize}
                \item Efficient for processing large volumes of predefined data.
                \item Effective for complex computations due to processing in bulk.
            \end{itemize}
        \item \textbf{Weaknesses}:
            \begin{itemize}
                \item Higher latency; results are not available in real-time.
                \item Requires extensive setup and maintenance.
            \end{itemize}
    \end{itemize}
\end{frame}

\begin{frame}[fragile]
    \frametitle{Stream Processing Architectures}
    \begin{itemize}
        \item \textbf{Definition}: Real-time processing of data as it is generated.
        \item \textbf{Examples}: 
            \begin{itemize}
                \item Apache Kafka
                \item Apache Flink
            \end{itemize}
        \item \textbf{Strengths}:
            \begin{itemize}
                \item Real-time analytics; suitable for applications needing immediate insights.
                \item Can handle varying data input rates.
            \end{itemize}
        \item \textbf{Weaknesses}:
            \begin{itemize}
                \item More complexity in system design and management.
                \item Potentially less robust for complex queries compared to batch processing.
            \end{itemize}
    \end{itemize}
\end{frame}

\begin{frame}[fragile]
    \frametitle{Lambda and Kappa Architectures}
    \begin{block}{Lambda Architecture}
        \begin{itemize}
            \item \textbf{Definition}: Combines batch and stream processing for comprehensive data analysis.
            \item \textbf{Structure}:
                \begin{itemize}
                    \item Batch Layer: Manages the master dataset and pre-computes results.
                    \item Speed Layer: Handles real-time data processing.
                    \item Serving Layer: Merges outputs for end-user queries.
                \end{itemize}
            \item \textbf{Strengths}:
                \begin{itemize}
                    \item Offers speed and thoroughness.
                    \item High availability and fault tolerance.
                \end{itemize}
            \item \textbf{Weaknesses}:
                \begin{itemize}
                    \item Increased complexity in architecture and maintenance.
                    \item Higher resource consumption due to dual processing layers.
                \end{itemize}
        \end{itemize}
    \end{block}
    
    \begin{block}{Kappa Architecture}
        \begin{itemize}
            \item \textbf{Definition}: Focuses solely on stream processing, eliminating the batch layer.
            \item \textbf{Strengths}:
                \begin{itemize}
                    \item Reduces complexity—fewer components to manage.
                    \item Real-time processing and updates.
                \end{itemize}
            \item \textbf{Weaknesses}:
                \begin{itemize}
                    \item Requires robust stream processing capabilities.
                    \item May not be suitable for applications needing batch data analysis.
                \end{itemize}
        \end{itemize}
    \end{block}
\end{frame}

\begin{frame}[fragile]
    \frametitle{Summary of Key Points}
    \begin{itemize}
        \item \textbf{Batch Processing}: Best for large volumes and complex computations; higher latency.
        \item \textbf{Stream Processing}: Ideal for real-time demands; more complex setups.
        \item \textbf{Lambda Architecture}: Merges strengths of both; higher complexity and resource usage.
        \item \textbf{Kappa Architecture}: Simplified approach focusing on streaming; may lack batch capability.
    \end{itemize}
\end{frame}

\begin{frame}[fragile]
    \frametitle{Closing Thoughts}
    Understanding the strengths and weaknesses of each big data architecture is crucial for selecting the right approach for specific business needs and technological environments. This knowledge equips you to make informed decisions in real-world applications.
\end{frame}

\begin{frame}[fragile]
    \frametitle{Next Steps}
    As we prepare to dive into developing distributed applications in the next session, reflect on how these architectures can shape your approaches to data processing and analytics!
\end{frame}

\begin{frame}[fragile]
    \frametitle{Distributed Applications Development}
    
    \begin{block}{Understanding Distributed Applications}
        Distributed applications are systems that run on multiple computers (or nodes) across a network, enabling processing of large datasets and facilitating tasks that require high availability and scalability.
    \end{block}
\end{frame}

\begin{frame}[fragile]
    \frametitle{Key Distributed Computing Models}

    \begin{enumerate}
        \item \textbf{MapReduce}
        \item \textbf{Apache Spark}
    \end{enumerate}
\end{frame}

\begin{frame}[fragile]
    \frametitle{MapReduce}

    \begin{block}{Concept}
        A programming model for processing large data sets with a distributed algorithm.
    \end{block}
    
    \begin{block}{Process}
        \begin{itemize}
            \item \textbf{Map Function:}
            \begin{lstlisting}[language=Python]
def map_function(document):
    for word in document.split():
        emit(word, 1)
            \end{lstlisting}
            \item \textbf{Reduce Function:}
            \begin{lstlisting}[language=Python]
def reduce_function(word, counts):
    return sum(counts)
            \end{lstlisting}
        \end{itemize}
    \end{block}
\end{frame}

\begin{frame}[fragile]
    \frametitle{Apache Spark}

    \begin{block}{Concept}
        A unified analytics engine for big data processing, with built-in modules for streaming, SQL, machine learning, and graph processing.
    \end{block}

    \begin{block}{Advantages}
        \begin{itemize}
            \item In-memory computation: Faster data processing compared to MapReduce.
            \item Language Compatibility: Supports Java, Scala, Python, and R.
        \end{itemize}
    \end{block}

    \begin{block}{Example}
        \begin{lstlisting}[language=Python]
from pyspark import SparkContext

sc = SparkContext("local", "Word Count")
text_file = sc.textFile("path/to/file.txt")
counts = text_file.flatMap(lambda line: line.split()) \
                   .map(lambda word: (word, 1)) \
                   .reduceByKey(lambda a, b: a + b)
counts.saveAsTextFile("output/path")
        \end{lstlisting}
    \end{block}  
\end{frame}

\begin{frame}[fragile]
    \frametitle{Lessons Learned and Challenges}

    \begin{block}{Lessons Learned}
        \begin{itemize}
            \item Scalability: Distributed applications manage large data efficiently.
            \item Fault Tolerance: Mechanisms to handle errors enhance reliability.
            \item Performance: Trade-offs between disk-based (MapReduce) and in-memory (Spark) processing.
        \end{itemize}
    \end{block}
    
    \begin{block}{Challenges Faced}
        \begin{itemize}
            \item Debugging distributed systems can be complex.
            \item Managing data consistency in real-time applications.
        \end{itemize}
    \end{block}
\end{frame}

\begin{frame}[fragile]
    \frametitle{Key Points and Conclusion}

    \begin{block}{Key Points to Emphasize}
        \begin{itemize}
            \item Architecture Understanding: Grasping architectural differences is crucial.
            \item Practical Experience: Hands-on experience helps overcome challenges.
            \item Community Support: Leveraging resources aids navigation of the learning curve.
        \end{itemize}
    \end{block}

    \begin{block}{Conclusion}
        Reflecting on the development journey reveals the evolution of big data technologies. Gaining proficiency with MapReduce and Spark prepares students for real-world data science challenges.
    \end{block}
\end{frame}

\begin{frame}[fragile]
    \frametitle{Streaming Data Analytics - Overview}
    \begin{block}{Introduction to Streaming Data}
        Streaming data refers to continuously generated data flows transmitted in real-time. Applications include:
        \begin{itemize}
            \item Social media feeds
            \item Financial transactions
            \item IoT sensors
            \item Live sports updates
        \end{itemize}
        Unlike static data, streaming data must be processed instantly to provide timely insights.
    \end{block}
\end{frame}

\begin{frame}[fragile]
    \frametitle{Streaming Data Analytics - Key Concepts}
    \begin{enumerate}
        \item \textbf{Real-Time Processing}
            \begin{itemize}
                \item Analyze data as it arrives, allowing immediate insights.
                \item Example: Financial markets track stock prices using real-time data.
            \end{itemize}

        \item \textbf{Event-Driven Architecture}
            \begin{itemize}
                \item Systems react to real-time events. 
                \item Example: Online shopping platforms recommend products based on user behavior.
            \end{itemize}

        \item \textbf{Stream vs. Batch Processing}
            \begin{itemize}
                \item Stream Processing: Continuously processes data (e.g., Apache Kafka, Apache Flink).
                \item Batch Processing: Processes large volumes of accumulated data (e.g., Hadoop MapReduce).
            \end{itemize}
    \end{enumerate}
\end{frame}

\begin{frame}[fragile]
    \frametitle{Tools for Streaming Data Analytics}
    \begin{block}{Apache Kafka}
        A widely-used distributed event streaming platform that acts as a message broker. 
        \begin{itemize}
            \item \textbf{High Throughput}: Handles millions of messages per second.
            \item \textbf{Scalability}: Easily scales up as data volumes increase.
            \item \textbf{Fault Tolerance}: Ensures data durability even in case of failures.
        \end{itemize}

        \textbf{Example in Action:}
        \begin{itemize}
            \item Real-Time Analysis of Clickstream Data:
            \begin{itemize}
                \item Websites capture user interactions in real-time using Kafka.
                \item Enhances user experience by adapting site presentation based on interactions.
            \end{itemize}
        \end{itemize}
    \end{block}
\end{frame}

\begin{frame}[fragile]
    \frametitle{Insights & Conclusion}
    \begin{block}{Insights Gained from Streaming Data Analytics}
        \begin{itemize}
            \item \textbf{Enhanced Decision Making}: Faster data-driven decisions.
            \item \textbf{Predictive Analytics}: Analyzing trends in real-time for better customer engagement.
            \item \textbf{Operational Efficiency}: Responding in real-time reduces latency in operations.
        \end{itemize}
    \end{block}

    \begin{block}{Conclusion}
        Mastering streaming data analytics tools and techniques is crucial for leveraging real-time insights to optimize business performance and drive innovation.
    \end{block}
\end{frame}

\begin{frame}[fragile]
    \frametitle{Code Snippet Example}
    \begin{lstlisting}[language=Java, basicstyle=\footnotesize]
        // Simple Apache Kafka producer code example
        import org.apache.kafka.clients.producer.KafkaProducer;
        import org.apache.kafka.clients.producer.ProducerRecord;

        import java.util.Properties;

        public class MyProducer {
            public static void main(String[] args) {
                Properties props = new Properties();
                props.put("bootstrap.servers", "localhost:9092");
                props.put("key.serializer", "org.apache.kafka.common.serialization.StringSerializer");
                props.put("value.serializer", "org.apache.kafka.common.serialization.StringSerializer");

                KafkaProducer<String, String> producer = new KafkaProducer<>(props);
                producer.send(new ProducerRecord<>("my-topic", "key", "value"));
                producer.close();
            }
        }
    \end{lstlisting}
\end{frame}

\begin{frame}[fragile]
    \frametitle{Large-Scale Machine Learning Techniques}
    \begin{block}{Introduction}
        Large-scale machine learning techniques are essential in today's data-driven world where massive datasets require efficient processing and analysis. 
    \end{block}
\end{frame}

\begin{frame}[fragile]
    \frametitle{Key Concepts}
    \begin{enumerate}
        \item \textbf{Scalability}
            \begin{itemize}
                \item Definition: The ability of an algorithm to handle increasing amounts of data without significant performance drops.
                \item Example: Linear regression on datasets exceeding millions of records.
            \end{itemize}
        
        \item \textbf{Distributed Computing}
            \begin{itemize}
                \item Definition: Using multiple machines to store data and perform computations.
                \item Example: Apache Spark for large-scale data processing across clusters.
            \end{itemize}
        
        \item \textbf{Batch vs. Online Learning}
            \begin{itemize}
                \item Batch Learning: Trained on complete datasets; suitable for static data.
                \item Online Learning: Continuously updated with real-time data; suitable for dynamic scenarios.
            \end{itemize}
    \end{enumerate}
\end{frame}

\begin{frame}[fragile]
    \frametitle{Techniques in Large-Scale ML}
    \begin{enumerate}
        \item \textbf{Gradient Descent Variants}
            \begin{itemize}
                \item Stochastic Gradient Descent (SGD): Updates using individual data points.
                \item Mini-Batch Gradient Descent: A balance between batch and SGD. 
                \begin{equation}
                    \theta = \theta - \alpha \nabla J(\theta)
                \end{equation}
            \end{itemize}
            
        \item \textbf{Distributed Data Storage}
            \begin{itemize}
                \item Technologies like Hadoop and Google BigQuery improve access to large datasets.
            \end{itemize}
            
        \item \textbf{Model Parallelism}
            \begin{itemize}
                \item Techniques to split a model across multiple processing units for faster training.
            \end{itemize}
    \end{enumerate}
\end{frame}

\begin{frame}[fragile]
    \frametitle{Learning Outcomes and Example Code}
    \begin{block}{Learning Outcomes}
        \begin{itemize}
            \item Identify and implement large-scale ML algorithms in real-world scenarios.
            \item Analyze case studies showcasing efficiency improvements.
            \item Evaluate techniques against traditional methods in speed, accuracy, and resource utilization.
        \end{itemize}
    \end{block}

    \begin{block}{Example Code Snippet}
        \begin{lstlisting}[language=Python]
from pyspark.ml.regression import LinearRegression
from pyspark.sql import SparkSession

# Initializing Spark session
spark = SparkSession.builder.appName("LargeScaleML").getOrCreate()

# Loading large dataset
data = spark.read.csv("hdfs://path_to_large_dataset.csv", header=True, inferSchema=True)

# Training a linear regression model
lr = LinearRegression(featuresCol='features', labelCol='label')
model = lr.fit(data)

# Making predictions
predictions = model.transform(data)
        \end{lstlisting}
    \end{block}
\end{frame}

\begin{frame}[fragile]
    \frametitle{Conclusion}
    \begin{block}{Conclusion}
        Mastering large-scale machine learning techniques involves understanding key concepts of scalability, distributed computing, and adopting appropriate training methods. These skills will be vital for future data scientists and machine learning engineers as datasets continue to grow.
    \end{block}
\end{frame}

\begin{frame}[fragile]
  \frametitle{Evaluating Big Data Solutions - Introduction}
  \begin{block}{Introduction}
    Evaluating big data solutions involves examining their scalability and performance. 
    These assessments are crucial for ensuring that the selected tools can handle increasing 
    volumes of data efficiently while maintaining optimal performance.
  \end{block}
\end{frame}

\begin{frame}[fragile]
  \frametitle{Evaluating Big Data Solutions - Key Concepts}
  \begin{block}{Key Concepts}
    \begin{enumerate}
      \item \textbf{Scalability}
        \begin{itemize}
          \item \textbf{Definition}: Ability of a system to grow and manage increased load without sacrificing performance.
          \item \textbf{Types of Scalability}:
            \begin{itemize}
              \item \textbf{Vertical Scaling (Scaling Up)}: Adding resources to an existing system (e.g., upgrading hardware).
              \item \textbf{Horizontal Scaling (Scaling Out)}: Adding more machines to distribute the load.
            \end{itemize}
        \end{itemize}
      \item \textbf{Performance}
        \begin{itemize}
          \item \textbf{Definition}: Measures how quickly and efficiently a system processes data.
          \item \textbf{Key Metrics}:
            \begin{itemize}
              \item \textbf{Throughput}: Number of operations or transactions processed per unit of time (e.g., requests/second).
              \item \textbf{Latency}: Time taken to process a single request or transaction.
            \end{itemize}
        \end{itemize}
    \end{enumerate}
  \end{block}
\end{frame}

\begin{frame}[fragile]
  \frametitle{Evaluating Big Data Solutions - Practical Example}
  \begin{block}{Practical Example}
    \textbf{Scenario}: Comparing two big data frameworks: Apache Hadoop and Apache Spark.
    
    \begin{itemize}
      \item \textbf{Hadoop}:
        \begin{itemize}
          \item \textbf{Scalability}: Scales well horizontally; added nodes can improve performance significantly.
          \item \textbf{Performance}: Higher latency due to disk-based storage; suited for batch processing.
        \end{itemize}
      \item \textbf{Spark}:
        \begin{itemize}
          \item \textbf{Scalability}: Also scales horizontally, but utilizes memory for faster data processing.
          \item \textbf{Performance}: Lower latency; excels in real-time analytics and iterative algorithms.
        \end{itemize}
    \end{itemize}
  \end{block}

  \begin{block}{Key Takeaways}
    \begin{itemize}
      \item Understanding scalability and performance is fundamental in choosing the right big data solution.
      \item Conduct thorough evaluations through workload testing, benchmarking, and resource utilization.
      \item The choice between frameworks often depends on specific project requirements (real-time vs. batch processing).
    \end{itemize}
  \end{block}
\end{frame}

\begin{frame}[fragile]
    \frametitle{Final Project Experience - Overview}
    \begin{block}{Overview}
        The final project serves as a capstone experience that encapsulates the knowledge and skills acquired throughout the course. 
        It challenges students to apply theoretical concepts in a practical setting, encouraging a deeper understanding and integration of their learning.
    \end{block}
\end{frame}

\begin{frame}[fragile]
    \frametitle{Final Project Experience - Key Concepts}
    \begin{enumerate}
        \item \textbf{Integration of Skills}
            \begin{itemize}
                \item \textbf{Data Analysis:} Applying statistical methods to analyze datasets.
                \item \textbf{Coding:} Utilizing programming languages (e.g., Python, R) for data manipulation and visualization.
                \item \textbf{Big Data Solutions:} Evaluating and implementing scalable solutions for data processing.
                \item \textbf{Team Collaboration:} Working effectively in groups to achieve common goals.
            \end{itemize}
        
        \item \textbf{Problem-Solving}
            \begin{itemize}
                \item Define the problem clearly.
                \item Gather relevant data.
                \item Analyze and derive insights.
                \item Propose actionable solutions based on findings.
            \end{itemize}
    \end{enumerate}
\end{frame}

\begin{frame}[fragile]
    \frametitle{Final Project Experience - Examples and Reflection Points}
    \begin{block}{Examples of Capstone Projects}
        \begin{itemize}
            \item \textbf{Predictive Analytics in Retail:} Analyze consumer behavior data to forecast sales trends.
            \item \textbf{Social Media Sentiment Analysis:} Evaluate public opinion on current topics using Twitter data.
            \item \textbf{Optimizing Logistics Operations:} Use big data analytics to enhance supply chain efficiency.
        \end{itemize}
    \end{block}
    
    \begin{block}{Reflection Points}
        \begin{itemize}
            \item \textbf{Learning Outcomes:} What specific skills did you develop?
            \item \textbf{Challenges Faced:} Identify hurdles encountered during the project.
            \item \textbf{Future Endeavors:} How can the skills gained facilitate your career development?
        \end{itemize}
    \end{block}
\end{frame}

\begin{frame}[fragile]
    \frametitle{Final Project Experience - Conclusion and Key Takeaways}
    \begin{block}{Conclusion}
        The final project is an essential part of your learning journey; it validates your ability to apply theoretical knowledge to practical scenarios. 
        Use this experience to build confidence in your skills and to pave the way for future academic or professional opportunities.
    \end{block}
    
    \begin{block}{Key Takeaways}
        \begin{itemize}
            \item Synthesize learned skills.
            \item Apply problem-solving techniques.
            \item Reflect on your journey and future goals.
        \end{itemize}
    \end{block}
\end{frame}

\begin{frame}[fragile]
    \frametitle{Student Feedback Summary}
    \begin{block}{Overview of Collected Feedback}
        The Student Feedback Summary captures the insights and experiences shared by students over the course. Understanding their perspectives is crucial for evaluating the effectiveness of the learning environment, course structure, and instructional methods.
    \end{block}
\end{frame}

\begin{frame}[fragile]
    \frametitle{Key Concepts Explained}
    \begin{enumerate}
        \item \textbf{Importance of Student Feedback:}
        \begin{itemize}
            \item Feedback provides valuable insights into successful aspects and areas needing improvement.
            \item Enables instructors to adapt teaching strategies and course content to better meet students' needs.
        \end{itemize}
        
        \item \textbf{Types of Feedback Collected:}
        \begin{itemize}
            \item \textbf{Qualitative Feedback:} Open-ended responses about learning experiences, group projects, and peer and instructor interactions.
            \item \textbf{Quantitative Feedback:} Rating scales evaluating aspects such as clarity of instructions, relevance of materials, and faculty support.
        \end{itemize}
    \end{enumerate}
\end{frame}

\begin{frame}[fragile]
    \frametitle{Feedback Categories and Key Points}
    \begin{block}{Example Feedback Categories}
        \begin{itemize}
            \item \textbf{Content Relevance:} Strong engagement with practical applications of course theories.
            \item \textbf{Instruction Clarity:} Majority rated clarity of explanations 4 out of 5; generally well communicated, with suggestions for enhancement.
            \item \textbf{Peer Interaction:} Value in collaborative projects, though noted challenges with group dynamics.
        \end{itemize}
    \end{block}

    \begin{block}{Key Points to Emphasize}
        \begin{itemize}
            \item \textbf{Recommendations for Enhancement:} Students valued interactive sessions and requested more one-on-one instructor feedback.
            \item \textbf{Common Themes:} While students appreciated the course structure, many desired more hands-on activities for deeper understanding.
        \end{itemize}
    \end{block}
\end{frame}

\begin{frame}[fragile]
    \frametitle{Conclusion}
    Collecting student feedback is an essential part of the reflective teaching process. By synthesizing their insights, we aim to enhance the learning experience, cultivate a more effective educational environment, and prepare for future iterations of the course.
\end{frame}

\begin{frame}[fragile]
    \frametitle{Next Steps}
    In the following slide, we will discuss suggested improvements based on this feedback and address common challenges highlighted by the students.
\end{frame}

\begin{frame}[fragile]
    \frametitle{Suggestions for Course Improvement}
    
    \begin{block}{Introduction}
        In examining the feedback received from students, it is imperative to translate their insights into actionable improvements. The following suggestions are rooted in understanding common challenges faced during the course and aim at enhancing the overall learning experience.
    \end{block}
\end{frame}

\begin{frame}[fragile]
    \frametitle{Key Areas for Improvement - Part 1}
    
    \begin{enumerate}
        \item \textbf{Clarity of Course Materials}
        \begin{itemize}
            \item \textbf{Explanation:} Confusion regarding the clarity and accessibility of lecture materials and readings.
            \item \textbf{Suggestion:} Streamline and simplify instructional materials; utilize bullet points, visuals, and summaries.
            \item \textbf{Example:} Provide annotated slides highlighting key concepts.
        \end{itemize}
        
        \item \textbf{Enhanced Engagement Strategies}
        \begin{itemize}
            \item \textbf{Explanation:} Passive learning identified as a barrier to effective knowledge absorption.
            \item \textbf{Suggestion:} Integrate more interactive elements (polls, discussion forums).
            \item \textbf{Example:} Use breakout sessions for peer discussions on complex topics.
        \end{itemize}
    \end{enumerate}
\end{frame}

\begin{frame}[fragile]
    \frametitle{Key Areas for Improvement - Part 2}
    
    \begin{enumerate}[resume]
        \item \textbf{Assessment and Feedback Frequency}
        \begin{itemize}
            \item \textbf{Explanation:} Feedback on assessments viewed as too infrequent.
            \item \textbf{Suggestion:} Increase formative assessments and provide timely feedback.
            \item \textbf{Example:} Weekly quizzes with instant feedback on answers.
        \end{itemize}
        
        \item \textbf{Diverse Learning Resources}
        \begin{itemize}
            \item \textbf{Explanation:} Desire for a variety of learning materials.
            \item \textbf{Suggestion:} Supplement core materials with videos and articles.
            \item \textbf{Example:} Curate a list of recommended TED talks or documentaries.
        \end{itemize}
        
        \item \textbf{Flexible Course Pace}
        \begin{itemize}
            \item \textbf{Explanation:} Some learners felt pressured by the pace.
            \item \textbf{Suggestion:} More flexible timelines and optional remediation sessions.
            \item \textbf{Example:} "Office hours" or review sessions before assessment deadlines.
        \end{itemize}
        
        \item \textbf{Networking Opportunities}
        \begin{itemize}
            \item \textbf{Explanation:} Building connections can enhance learning.
            \item \textbf{Suggestion:} Organize networking events or virtual meet-and-greets.
            \item \textbf{Example:} A panel discussion featuring alumni or industry experts.
        \end{itemize}
    \end{enumerate}
\end{frame}

\begin{frame}[fragile]
    \frametitle{Conclusion and Key Points}
    
    \begin{block}{Conclusion}
        These suggestions reflect our commitment to adapting the course to meet the diverse needs of students. Implementing these adjustments will not only address current challenges but also foster a more enriching learning environment.
    \end{block}
    
    \begin{itemize}
        \item \textbf{Student-Centric Approach:} Prioritize adjustments based on feedback.
        \item \textbf{Continuous Improvement:} Regular evaluation is crucial for enhancement.
        \item \textbf{Engagement:} Boosting interactivity can enhance understanding and retention.
    \end{itemize}
\end{frame}

\begin{frame}[fragile]
    \frametitle{Future Learning Paths}
    \begin{block}{Suggestions for Further Studies}
        As we conclude this course, it's essential to consider the educational journeys that lie ahead. Here are several paths and topics to deepen your understanding and expand your skills:
    \end{block}
\end{frame}

\begin{frame}[fragile]
    \frametitle{Future Learning Paths - Further Studies}
    \begin{enumerate}
        \item \textbf{Advanced Topics in [Course Subject]}
        \begin{itemize}
            \item Gain in-depth knowledge of advanced theories and techniques.
            \item \textit{Examples:} Specialized courses in machine learning or big data analytics.
        \end{itemize}
        
        \item \textbf{Interdisciplinary Approaches}
        \begin{itemize}
            \item Explore intersections with other fields.
            \item \textit{Examples:} 
            \begin{itemize}
                \item Data Science \& Psychology
                \item Environmental Science \& Engineering
            \end{itemize}
        \end{itemize}
    \end{enumerate}
\end{frame}

\begin{frame}[fragile]
    \frametitle{Future Learning Paths - Practical Applications and Networking}
    \begin{enumerate}
        \setcounter{enumi}{2} % To continue the enumeration
        \item \textbf{Practical Applications and Certifications}
        \begin{itemize}
            \item Gain practical skills and industry-recognized certificates.
            \item \textit{Examples:} Google Analytics, AWS Certified Solutions Architect.
        \end{itemize}
        
        \item \textbf{Research and Academic Pursuits}
        \begin{itemize}
            \item Engage in research or pursue advanced degrees.
            \item \textit{Examples:} Participate in research projects or consider a Master's or PhD.
        \end{itemize}
        
        \item \textbf{Professional Development and Networking}
        \begin{itemize}
            \item Connect with professionals to broaden your horizons.
            \item \textit{Examples:} Attend workshops, join forums and professional organizations.
        \end{itemize}
    \end{enumerate}
\end{frame}

\begin{frame}[fragile]
    \frametitle{Q\&A Session - Overview}
    \begin{block}{Slide Description}
        Open floor for questions and discussion to clarify concepts and connections to future applications.
    \end{block}
    \begin{itemize}
        \item Providing an opportunity to ask questions 
        \item Engaging in discussions on course material
        \item Connecting concepts to future applications
    \end{itemize}
\end{frame}

\begin{frame}[fragile]
    \frametitle{Q\&A Session - Key Points to Clarify}
    \begin{enumerate}
        \item \textbf{Clarification of Concepts}
        \begin{itemize}
            \item Ask about challenging topics
            \item Focus on core theories and frameworks
            \item Real-world application of models
        \end{itemize}
        
        \item \textbf{Connecting Concepts to Future Applications}
        \begin{itemize}
            \item Discuss applications in various fields
            \item Examples include statistical analysis in business and behavioral theories in marketing
        \end{itemize}
        
        \item \textbf{Reflect on Personal Experiences}
        \begin{itemize}
            \item Share experiences applying course content
            \item Discuss implications of theoretical knowledge
        \end{itemize}
    \end{enumerate}
\end{frame}

\begin{frame}[fragile]
    \frametitle{Q\&A Session - Engaging Questions}
    \begin{block}{Engaging Questions to Encourage Discussion}
        \begin{itemize}
            \item What concepts do you feel were most applicable to your future career?
            \item Can you think of situations where analytical skills gained could apply?
            \item How has your understanding of course topics evolved?
        \end{itemize}
    \end{block}
    
    \begin{block}{Examples to Illustrate Concept Application}
        \begin{itemize}
            \item \textbf{Case Study Example:} Recent marketing campaign utilizing data analytics.
            \item \textbf{Real-World Application:} Behavioral economics in policy-making.
        \end{itemize}
    \end{block}
\end{frame}

\begin{frame}[fragile]
    \frametitle{Conclusion and Takeaways - Overall Learning Experience}
    \begin{block}{Reflection on the Journey}
        As we conclude this course, it’s essential to reflect on the journey we embarked on together. 
        Throughout these weeks, we have engaged with a variety of concepts, tools, and frameworks that are 
        pivotal to our subject area and applicable to real-world scenarios. 
    \end{block}
    
    \begin{itemize}
        \item \textbf{Understanding Fundamental Principles:} 
        Established a framework for exploration, covering foundational concepts such as descriptive 
        and inferential statistics.
        
        \item \textbf{Application of Knowledge:} 
        Actively applied theoretical knowledge through case studies and projects, bridging the gap 
        between theory and practice.
        
        \item \textbf{Critical Thinking and Problem Solving:} 
        Engaged in discussions and collaborative projects, valuing critical thinking and enhancing 
        analytical skills.
    \end{itemize}
\end{frame}

\begin{frame}[fragile]
    \frametitle{Conclusion and Takeaways - Key Takeaways}
    \begin{enumerate}
        \item \textbf{Real-World Application:}
        Ability to apply learned concepts to real-life situations strengthens connections between 
        theory and practice.
        
        \item \textbf{Collaborative Learning:}
        Discussions, team projects, and peer feedback enhance understanding and encourage innovative 
        solutions.
        
        \item \textbf{Lifelong Learning:}
        Embrace curiosity and seek knowledge beyond the classroom to stay updated in your field.
        
        \item \textbf{Constructive Feedback:}
        Importance of giving and receiving constructive criticism as a tool for growth.
        
        \item \textbf{Reflection and Self-Assessment:}
        Regularly reflecting on learning experiences aids in assessing progress and identifying areas 
        for improvement.
    \end{enumerate}
\end{frame}

\begin{frame}[fragile]
    \frametitle{Conclusion and Takeaways - Final Thoughts}
    \begin{block}{Moving Forward}
        Carry not just the knowledge gained but also the skills and habits necessary for continuous 
        improvement. Use this reflection to identify strengths and areas for further development.
    \end{block}

    \begin{block}{Reminder}
        The next slide will discuss the Course Evaluation Feedback, emphasizing its importance for 
        continuous improvement in future iterations of the course. Your insights are invaluable!
    \end{block}
    
    Feel encouraged to continue engaging with the material, sharing ideas, and pursuing your interests. 
    Thank you for your dedication and enthusiasm throughout the course!
\end{frame}

\begin{frame}[fragile]
  \frametitle{Course Evaluation Feedback - Introduction}
  
  \begin{block}{Importance of Feedback}
    Your feedback is invaluable as we conclude this course. 
    Completing the course evaluation survey reflects your personal experience and shapes the future of this course, enhancing the learning environment for upcoming students.
  \end{block}
  
\end{frame}

\begin{frame}[fragile]
  \frametitle{Course Evaluation Feedback - Why Feedback Matters}
  
  \begin{itemize}
    \item \textbf{Improvement:} Evaluations help instructors identify strengths and areas for growth, ensuring the course evolves to meet learner needs.
    
    \item \textbf{Student Voice:} Your insights influence curricular choices, teaching methods, and overall course design.
    
    \item \textbf{Quality Assurance:} Surveys contribute to institutional quality assurance, ensuring educational standards are upheld for the best possible education.
  \end{itemize}

\end{frame}

\begin{frame}[fragile]
  \frametitle{Course Evaluation Feedback - Instructions and Key Points}
  
  \begin{block}{Instructions for Completing the Survey}
    \begin{enumerate}
      \item \textbf{Accessing the Survey:} Log into your LMS account and navigate to 'Course Evaluation'.
      \item \textbf{Survey Structure:} Expect multiple-choice and open-ended questions covering course content and teaching effectiveness.
      \item \textbf{Providing Constructive Feedback:} Use specific examples and be honest but respectful.
      \item \textbf{Time Commitment:} Allocate 10-15 minutes for thoughtful responses.
    \end{enumerate}
  \end{block}

  \begin{block}{Key Points to Emphasize}
    - \textbf{Deadline Awareness:} Complete the survey before the deadline, usually a week before the course ends.
    - \textbf{Confidentiality Assurance:} Your responses are anonymous, promoting honest feedback.
    - \textbf{Impact on Future Iterations:} Feedback will directly inform future course modifications for the benefit of students and instructors alike.
  \end{block}

\end{frame}

\begin{frame}[fragile]
  \frametitle{Course Evaluation Feedback - Example Questions}
  
  \begin{itemize}
    \item Rate your overall satisfaction with the course on a scale from 1 to 5.
    \item Which topics did you find most valuable, and why?
    \item What improvements would you suggest for future iterations of this course?
  \end{itemize}

\end{frame}

\begin{frame}[fragile]
  \frametitle{Course Evaluation Feedback - Conclusion}
  
  \begin{block}{Final Thoughts}
    Your feedback is critical, and we encourage all students to participate in the evaluation survey. 
    By doing so, you contribute to an ongoing cycle of improvement that enhances the educational experience at our institution.
  \end{block}
  
  Thank you for your time and insights throughout this course!

\end{frame}


\end{document}