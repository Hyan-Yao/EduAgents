\documentclass[aspectratio=169]{beamer}

% Theme and Color Setup
\usetheme{Madrid}
\usecolortheme{whale}
\useinnertheme{rectangles}
\useoutertheme{miniframes}

% Additional Packages
\usepackage[utf8]{inputenc}
\usepackage[T1]{fontenc}
\usepackage{graphicx}
\usepackage{booktabs}
\usepackage{listings}
\usepackage{amsmath}
\usepackage{amssymb}
\usepackage{xcolor}
\usepackage{tikz}
\usepackage{pgfplots}
\pgfplotsset{compat=1.18}
\usetikzlibrary{positioning}
\usepackage{hyperref}

% Custom Colors
\definecolor{myblue}{RGB}{31, 73, 125}
\definecolor{mygray}{RGB}{100, 100, 100}
\definecolor{mygreen}{RGB}{0, 128, 0}
\definecolor{myorange}{RGB}{230, 126, 34}
\definecolor{mycodebackground}{RGB}{245, 245, 245}

% Set Theme Colors
\setbeamercolor{structure}{fg=myblue}
\setbeamercolor{frametitle}{fg=white, bg=myblue}
\setbeamercolor{title}{fg=myblue}
\setbeamercolor{section in toc}{fg=myblue}
\setbeamercolor{item projected}{fg=white, bg=myblue}
\setbeamercolor{block title}{bg=myblue!20, fg=myblue}
\setbeamercolor{block body}{bg=myblue!10}
\setbeamercolor{alerted text}{fg=myorange}

% Set Fonts
\setbeamerfont{title}{size=\Large, series=\bfseries}
\setbeamerfont{frametitle}{size=\large, series=\bfseries}
\setbeamerfont{caption}{size=\small}
\setbeamerfont{footnote}{size=\tiny}

% Code Listing Style
\lstdefinestyle{customcode}{
  backgroundcolor=\color{mycodebackground},
  basicstyle=\footnotesize\ttfamily,
  breakatwhitespace=false,
  breaklines=true,
  commentstyle=\color{mygreen}\itshape,
  keywordstyle=\color{blue}\bfseries,
  stringstyle=\color{myorange},
  numbers=left,
  numbersep=8pt,
  numberstyle=\tiny\color{mygray},
  frame=single,
  framesep=5pt,
  rulecolor=\color{mygray},
  showspaces=false,
  showstringspaces=false,
  showtabs=false,
  tabsize=2,
  captionpos=b
}
\lstset{style=customcode}

% Custom Commands
\newcommand{\hilight}[1]{\colorbox{myorange!30}{#1}}
\newcommand{\source}[1]{\vspace{0.2cm}\hfill{\tiny\textcolor{mygray}{Source: #1}}}
\newcommand{\concept}[1]{\textcolor{myblue}{\textbf{#1}}}
\newcommand{\separator}{\begin{center}\rule{0.5\linewidth}{0.5pt}\end{center}}

% Footer and Navigation Setup
\setbeamertemplate{footline}{
  \leavevmode%
  \hbox{%
  \begin{beamercolorbox}[wd=.3\paperwidth,ht=2.25ex,dp=1ex,center]{author in head/foot}%
    \usebeamerfont{author in head/foot}\insertshortauthor
  \end{beamercolorbox}%
  \begin{beamercolorbox}[wd=.5\paperwidth,ht=2.25ex,dp=1ex,center]{title in head/foot}%
    \usebeamerfont{title in head/foot}\insertshorttitle
  \end{beamercolorbox}%
  \begin{beamercolorbox}[wd=.2\paperwidth,ht=2.25ex,dp=1ex,center]{date in head/foot}%
    \usebeamerfont{date in head/foot}
    \insertframenumber{} / \inserttotalframenumber
  \end{beamercolorbox}}%
  \vskip0pt%
}

% Turn off navigation symbols
\setbeamertemplate{navigation symbols}{}

% Title Page Information
\title[Reinforcement Learning]{Week 1: Introduction to Reinforcement Learning}
\author[J. Smith]{John Smith, Ph.D.}
\institute[University Name]{Department of Computer Science\\University Name\\Email: email@university.edu\\Website: www.university.edu}
\date{\today}

% Document Start
\begin{document}

\frame{\titlepage}

\begin{frame}[fragile]
    \frametitle{Introduction to Reinforcement Learning - Course Objectives}
    \begin{itemize}
        \item \textbf{Understand Reinforcement Learning (RL):} Comprehensive understanding of RL as a significant paradigm in machine learning and AI.
        \item \textbf{Explore Applications in AI:} Discover applications of RL in various fields including robotics, gaming, finance, and healthcare.
        \item \textbf{Develop Practical Skills:} Hands-on projects to learn how to implement RL algorithms and apply them to real-world problems.
    \end{itemize}
\end{frame}

\begin{frame}[fragile]
    \frametitle{Introduction to Reinforcement Learning - Importance}
    \begin{block}{What is Reinforcement Learning?}
        \begin{itemize}
            \item \textbf{Definition:} A type of machine learning where an agent learns to make decisions to maximize cumulative rewards.
            \item \textbf{Key Components:}
            \begin{itemize}
                \item \textbf{Agent:} The learner or decision-maker.
                \item \textbf{Environment:} The context in which the agent operates.
                \item \textbf{Actions:} The choices available to the agent.
                \item \textbf{States:} The current situation of the agent in the environment.
                \item \textbf{Rewards:} Feedback from the environment based on actions taken.
            \end{itemize}
        \end{itemize}
    \end{block}
\end{frame}

\begin{frame}[fragile]
    \frametitle{Introduction to Reinforcement Learning - Relevance}
    \begin{itemize}
        \item \textbf{Versatile Applications:} Enables training models for complex sequential decision-making tasks.
        \begin{itemize}
            \item \textbf{Game playing:} Example - AlphaGo achieved victory over a world champion in Go.
            \item \textbf{Robotics:} Important for real-time decision-making in dynamic environments.
            \item \textbf{Autonomous Vehicles:} RL allows safe and efficient navigation.
        \end{itemize}

        \item \textbf{Key Points to Emphasize:}
        \begin{enumerate}
            \item Learning by Interaction: RL learns from the consequences of actions, unlike supervised learning.
            \item Exploration vs. Exploitation: Balancing new actions with known rewarding actions.
            \item Long-term vs. Short-term Rewards: Focus on maximizing long-term rewards is crucial.
        \end{enumerate}

        \item \textbf{Mathematical Objective:}
        \begin{equation}
            \text{Objective: } \max \sum_{t=0}^{T} \gamma^t r_t
        \end{equation}
        \begin{itemize}
            \item \( r_t \): Reward received at time \( t \)
            \item \( \gamma \): Discount factor (0 < \( \gamma \) < 1)
            \item \( T \): Total time steps
        \end{itemize}
    \end{itemize}
\end{frame}

\begin{frame}[fragile]
    \frametitle{What is Reinforcement Learning? - Definition}
    \begin{block}{Definition}
        \textbf{Reinforcement Learning (RL)} is a subfield of machine learning where an agent learns to make decisions by taking actions in an environment to maximize cumulative rewards over time.
    \end{block}
    \begin{itemize}
        \item Unlike supervised learning, RL relies on trial-and-error interactions to learn optimal behaviors.
    \end{itemize}
\end{frame}

\begin{frame}[fragile]
    \frametitle{What is Reinforcement Learning? - Key Components}
    \begin{itemize}
        \item \textbf{Agent:} The learner or decision-maker that interacts with the environment.
        \item \textbf{Environment:} All that the agent interacts with, providing feedback based on actions.
        \item \textbf{State:} A specific configuration of the environment at a given time.
        \item \textbf{Action:} A choice made by the agent that changes the state of the environment.
        \item \textbf{Reward:} A signal received after taking an action indicating its immediate benefit.
    \end{itemize}
\end{frame}

\begin{frame}[fragile]
    \frametitle{Reinforcement Learning Process}
    \begin{itemize}
        \item The agent observes the current state \(S_t\) of the environment.
        \item It selects an action \(A_t\) based on a policy.
        \item The environment provides a new state \(S_{t+1}\) and a reward \(R_t\).
        \item The agent evaluates the action taken using the reward to adjust strategy for future actions.
    \end{itemize}
    \begin{center}
        \begin{tikzpicture}
            \node (agent) [draw, rectangle] {Agent};
            \node (action) [draw, rectangle, right=2cm of agent] {Action ($A_t$)};
            \node (env) [draw, rectangle, below=2cm of action] {Environment};
            \node (state) [draw, rectangle, right=2cm of env] {State ($S_t$)};
            \node (reward) [draw, rectangle, below=1cm of state] {Reward ($R_t$)};
            \node (newstate) [draw, rectangle, below=1cm of reward] {New State ($S_{t+1}$)};
            
            \draw [->] (agent) -- (action);
            \draw [->] (action) -- (env);
            \draw [->] (env) -- (state);
            \draw [->] (state) -- (reward);
            \draw [->] (reward) -- (newstate);
        \end{tikzpicture}
    \end{center}
\end{frame}

\begin{frame}[fragile]
    \frametitle{Key Concepts in Reinforcement Learning}
    \begin{itemize}
        \item \textbf{Exploration vs. Exploitation:} Balancing new actions vs. known rewards.
        \item \textbf{Cumulative Reward:} Focus on maximizing total reward rather than immediate rewards.
    \end{itemize}
\end{frame}

\begin{frame}[fragile]
    \frametitle{Conclusion}
    \begin{block}{Summary}
        Reinforcement Learning empowers agents to learn optimal behaviors through interactions with environments and guided feedback of rewards, functioning effectively in dynamic situations. This foundational knowledge sets the stage for exploring advanced techniques and algorithms in upcoming slides.
    \end{block}
\end{frame}

\begin{frame}[fragile]
    \frametitle{Historical Background - Evolution of Reinforcement Learning}
    \begin{itemize}
        \item Key milestones in the development of reinforcement learning:
        \begin{enumerate}
            \item Early Beginnings (1950s)
            \item Formalization of Concepts (1980s)
            \item Establishing Foundations (1990s)
            \item Rise of Practical Applications (2000s)
            \item Present Day
        \end{enumerate}
    \end{itemize}
\end{frame}

\begin{frame}[fragile]
    \frametitle{Historical Background - Key Milestones}
    \begin{enumerate}
        \item \textbf{1950s - Early Beginnings}
            \begin{itemize}
                \item Hebbian Learning (1949): Laid groundwork for neural networks.
                \item Trial and Error Learning: Theoretical foundation by B.F. Skinner.
            \end{itemize}
        
        \item \textbf{1980s - Formalization of Concepts}
            \begin{itemize}
                \item Temporal-Difference Learning (1988): Introduced by Richard Sutton.
                \item Q-Learning (1989): Proposed by Christopher Watkins.
            \end{itemize}
        
        \item \textbf{1990s - Establishing Foundations}
            \begin{itemize}
                \item Actor-Critic Methods: Strategy involving both policy and value function.
                \item The Reinforcement Learning Framework (1998): Sutton and Barto's comprehensive outline.
            \end{itemize}
    \end{enumerate}
\end{frame}

\begin{frame}[fragile]
    \frametitle{Historical Background - Recent Advances}
    \begin{enumerate}
        \setcounter{enumi}{3}
        \item \textbf{2000s - Rise of Practical Applications}
            \begin{itemize}
                \item Deep Reinforcement Learning (2013): Combination of deep learning and RL.
                \item AlphaGo (2016): Demonstrated RL's power in complex games.
            \end{itemize}
        
        \item \textbf{Present Day}
            \begin{itemize}
                \item Real-World Applications: Robotics, autonomous vehicles, and game development.
            \end{itemize}
    \end{enumerate}
    \begin{block}{Key Points to Emphasize}
        \begin{itemize}
            \item Interdisciplinary Influence: RL's roots in psychology and neuroscience.
            \item Foundational Algorithms: Importance of Q-learning for understanding RL.
            \item Evolutionary Catalyst: Breakthroughs spurring rapid advancements.
        \end{itemize}
    \end{block}
\end{frame}

\begin{frame}[fragile]
    \frametitle{Historical Background - Example and Conclusion}
    \textbf{Example: Q-learning}
    \begin{itemize}
        \item An agent navigating a maze maximizes rewards (exit) and minimizes penalties (steps).
        \begin{equation}
            Q(s, a) = Q(s, a) + \alpha [r + \gamma \max Q(s', a') - Q(s, a)]
        \end{equation}
    \end{itemize}
    \textbf{Conclusion:}
    \begin{itemize}
        \item Understanding RL's history informs current methodologies and future research.
        \item Each milestone is crucial for RL's robust application across various domains.
    \end{itemize}
\end{frame}

\begin{frame}[fragile]
    \frametitle{Core Concepts of Reinforcement Learning}
    In this section, we will cover the foundational concepts of Reinforcement Learning, including:
    \begin{itemize}
        \item Markov Decision Processes (MDPs)
        \item Value Functions
        \item Policies
    \end{itemize}
\end{frame}

\begin{frame}[fragile]
    \frametitle{1. Markov Decision Processes (MDPs)}
    \begin{block}{Definition}
        MDPs provide a mathematical framework for modeling decision-making situations where outcomes are partly random and partly under the control of a decision-maker.
    \end{block}
    An MDP is defined by:
    \begin{itemize}
        \item **States (S):** A set of all possible situations the agent can be in.
        \item **Actions (A):** A set of all possible actions the agent can take.
        \item **Transition Probability (P):** Probability of transitioning from one state to another given an action, denoted as \(P(s'|s, a)\).
        \item **Reward Function (R):** The immediate reward received after transitioning from state \(s\) to state \(s'\) using action \(a\), denoted as \(R(s, a, s')\).
        \item **Discount Factor ($\gamma$):** A factor between 0 and 1 that determines the present value of future rewards.
    \end{itemize}
    \begin{block}{Example}
        Consider a simple grid world where:
        \begin{itemize}
            \item Each position (cell) represents a state.
            \item Possible movements (up, down, left, right) are the actions.
            \item Reward could be +1 for reaching a goal and -1 for falling into a trap.
        \end{itemize}
    \end{block}
\end{frame}

\begin{frame}[fragile]
    \frametitle{2. Value Functions}
    \begin{block}{Definition}
        Value functions estimate how good it is for an agent to be in a given state or to perform a certain action.
    \end{block}
    There are two primary types:
    \begin{itemize}
        \item **State Value Function ($V(s)$):** 
        \[
        V(s) = \mathbb{E}_{\pi} \left[\sum_{t=0}^{\infty} \gamma^t R_t \mid S_0 = s\right]
        \]
        \item **Action Value Function ($Q(s, a)$):** 
        \[
        Q(s, a) = \mathbb{E}_{\pi} \left[\sum_{t=0}^{\infty} \gamma^t R_t \mid S_0 = s, A_0 = a\right]
        \]
    \end{itemize}
    \begin{block}{Key Point}
        Value functions guide the agent in selecting actions that maximize expected rewards.
    \end{block}
    \begin{block}{Example}
        In the grid world, the value of a state could be calculated based on the rewards from reaching the goal minus the penalties from traps.
    \end{block}
\end{frame}

\begin{frame}[fragile]
    \frametitle{3. Policies}
    \begin{block}{Definition}
        A policy \( \pi \) defines the agent's behavior, mapping states to actions.
    \end{block}
    Policies can be:
    \begin{itemize}
        \item **Deterministic Policy:** Prescribes a specific action for each state, \( \pi(s) = a \).
        \item **Stochastic Policy:** A probability distribution over actions for each state, \( \pi(a|s) \).
    \end{itemize}
    \begin{block}{Key Point}
        The goal of reinforcement learning is to find the optimal policy that maximizes the total expected reward over time.
    \end{block}
    \begin{block}{Example}
        In the grid world:
        \begin{itemize}
            \item A deterministic policy might always move right at the beginning.
            \item A stochastic policy might choose left with a 30% probability and right with a 70% probability when in the same state.
        \end{itemize}
    \end{block}
\end{frame}

\begin{frame}[fragile]
    \frametitle{Conclusion}
    Understanding MDPs, value functions, and policies is crucial for implementing reinforcement learning algorithms effectively. These concepts form the foundational building blocks of how agents learn to make decisions and optimize their actions in uncertain environments. 

    By integrating these core concepts into your knowledge base, you will be better prepared for exploring more complex RL algorithms and their applications in the following slides.
\end{frame}

\begin{frame}[fragile]
    \frametitle{Applications of Reinforcement Learning - Overview}
    Reinforcement Learning (RL) is a powerful paradigm with myriad applications across various fields. 
    The ability to learn optimal policies through environmental interaction is pivotal to its success.
\end{frame}

\begin{frame}[fragile]
    \frametitle{Applications of Reinforcement Learning - Gaming}
    \begin{itemize}
        \item \textbf{Example}: AlphaGo by DeepMind
        \begin{itemize}
            \item RL systems have achieved superhuman performance in complex games like Go and Chess.
            \item AlphaGo utilized both deep learning and RL, learning strategies from millions of game simulations.
        \end{itemize}
        \item \textbf{Key Point}: RL in gaming shows how agents can master intricate strategies through trial and error.
    \end{itemize}
\end{frame}

\begin{frame}[fragile]
    \frametitle{Applications of Reinforcement Learning - Robotics}
    \begin{itemize}
        \item \textbf{Example}: Robotic Hand Manipulation
        \begin{itemize}
            \item RL enables robots to learn tasks such as grasping and manipulation by improving actions based on environmental feedback.
        \end{itemize}
        \item \textbf{Key Point}: Robots learn in simulations and transfer skills to real-world applications, improving efficiency in complex and unpredictable environments.
    \end{itemize}
\end{frame}

\begin{frame}[fragile]
    \frametitle{Applications of Reinforcement Learning - Healthcare}
    \begin{itemize}
        \item \textbf{Example}: Treatment Recommendations
        \begin{itemize}
            \item RL algorithms can personalize treatment plans by learning from patient data over time, optimizing drug administration and outcomes.
        \end{itemize}
        \item \textbf{Key Point}: RL supports dynamic decision-making in healthcare, adjusting treatments based on real-time patient responses.
    \end{itemize}
\end{frame}

\begin{frame}[fragile]
    \frametitle{Applications of Reinforcement Learning - Finance}
    \begin{itemize}
        \item \textbf{Example}: Algorithmic Trading
        \begin{itemize}
            \item RL is used to develop trading strategies that maximize return on investment by learning from historical market data.
        \end{itemize}
        \item \textbf{Key Point}: The ability to adapt to market changes in real-time makes RL a valuable tool for investors and financial analysts.
    \end{itemize}
\end{frame}

\begin{frame}[fragile]
    \frametitle{Additional Insights}
    \begin{itemize}
        \item \textbf{Cross-Domain Potential}: RL's versatility allows it to be applied in other domains like autonomous vehicles, natural language processing, and recommendation systems.
        \item \textbf{Real-time Learning}: One of the strengths of RL is its capability to improve continuously through ongoing interaction with the environment.
    \end{itemize}
\end{frame}

\begin{frame}[fragile]
    \frametitle{Conclusion}
    Reinforcement Learning's capability to learn from experience and optimize outcomes is transforming industries by:
    \begin{itemize}
        \item Enhancing automation
        \item Personalizing experiences
        \item Contributing to more intelligent decision systems
    \end{itemize}
    Understanding these applications highlights RL's transformative impact on real-world problem-solving.
\end{frame}

\begin{frame}[fragile]
    \frametitle{Challenges in Reinforcement Learning - Overview}
    \begin{itemize}
        \item Discussion of key challenges in reinforcement learning:
        \begin{itemize}
            \item Sample Efficiency
            \item Exploration vs. Exploitation
            \item Scalability
        \end{itemize}
    \end{itemize}
\end{frame}

\begin{frame}[fragile]
    \frametitle{Challenges in Reinforcement Learning - Sample Efficiency}
    \begin{block}{1. Sample Efficiency}
        \begin{itemize}
            \item \textbf{Definition}: The ability of an RL algorithm to learn an effective policy with minimal interactions with the environment.
            \item \textbf{Challenge}: Many RL methods require extensive data to converge, which can be costly and time-consuming.
            \item \textbf{Example}: Training a robot for navigation could require thousands of trials to master proper movement.
        \end{itemize}
    \end{block}
\end{frame}

\begin{frame}[fragile]
    \frametitle{Challenges in Reinforcement Learning - Exploration vs. Exploitation}
    \begin{block}{2. Exploration vs. Exploitation}
        \begin{itemize}
            \item \textbf{Definition}: The trade-off between:
            \begin{itemize}
                \item \textbf{Exploration}: Trying new actions to gain knowledge.
                \item \textbf{Exploitation}: Maximizing immediate rewards based on known information.
            \end{itemize}
            \item \textbf{Challenge}: Balancing exploration and exploitation is crucial. Too much exploration may lead to suboptimal performance, while excessive exploitation could overlook better strategies.
            \item \textbf{Example}: In games, relying solely on known strategies may miss winning tactics, whereas too much exploration can result in inefficient play.
        \end{itemize}
    \end{block}
\end{frame}

\begin{frame}[fragile]
    \frametitle{Challenges in Reinforcement Learning - Scalability}
    \begin{block}{3. Scalability}
        \begin{itemize}
            \item \textbf{Definition}: How well an RL algorithm can handle increased complexity or size of the problem.
            \item \textbf{Challenge}: As the state or action space grows, the number of interactions required can increase exponentially, making optimal policy learning impractical.
            \item \textbf{Example}: In Chess, the vast state and action spaces pose significant learning challenges for RL agents.
        \end{itemize}
    \end{block}
\end{frame}

\begin{frame}[fragile]
    \frametitle{Key Points and Additional Resources}
    \begin{itemize}
        \item \textbf{Key Points to Emphasize}:
        \begin{itemize}
            \item Importance of sample efficiency to minimize costs in real-world applications.
            \item The significance of balancing exploration and exploitation for optimal learning.
            \item Scalability issues as problem sizes increase, necessitating advanced algorithms.
        \end{itemize}
        \item \textbf{Additional Resources}:
        \begin{itemize}
            \item Literature on “Efficient Exploration Methods” and “Scalable Reinforcement Learning Algorithms”.
            \item Real-world scenario discussion: strategies for improving sample efficiency in RL.
        \end{itemize}
    \end{itemize}
\end{frame}

\begin{frame}[fragile]
    \frametitle{Learning Outcomes for this Course - Introduction}
    This course is designed to provide a comprehensive understanding of Reinforcement Learning (RL). By the end of this course, students are expected to achieve:
    \begin{itemize}
        \item Proficiency in key RL concepts
        \item Ability to apply various algorithms
        \item Effective communication of findings
    \end{itemize}
\end{frame}

\begin{frame}[fragile]
    \frametitle{Learning Outcomes for this Course - Key Learning Outcomes}
    \begin{enumerate}
        \item Proficiency in Key RL Concepts
        \begin{itemize}
            \item \textbf{Definition of Reinforcement Learning}: Understand the framework in which agents learn to make decisions by interacting with an environment.
                \begin{itemize}
                    \item \textbf{Agent}: The learner or decision-maker.
                    \item \textbf{Environment}: Everything that the agent interacts with.
                    \item \textbf{Actions, States, Rewards}: Core components of the RL framework.
                \end{itemize}
            \item \textbf{Understanding of Terminologies}: Familiarize with terms such as:
                \begin{itemize}
                    \item \textbf{Policy}: A strategy that the agent employs to determine actions based on the current state.
                    \item \textbf{Value Function}: A function that estimates the expected return or total reward expected from a given state.
                \end{itemize}
        \end{itemize}
    \end{enumerate}
\end{frame}

\begin{frame}[fragile]
    \frametitle{Learning Outcomes for this Course - Application of Algorithms and Communication}
    \begin{enumerate}
        \setcounter{enumi}{1} % Continue numbering from previous frame
        \item Ability to Apply Algorithms
        \begin{itemize}
            \item \textbf{Implementation of RL Algorithms}: Gain skills to implement various RL algorithms such as:
                \begin{itemize}
                    \item \textbf{Q-Learning}: A value-based off-policy algorithm. 
                    \begin{block}{Example}
                    \begin{lstlisting}[language=Python]
                    Q(s, a) = Q(s, a) + α[r + γ max_a Q(s', a) - Q(s, a)]
                    \end{lstlisting}
                    \end{block}
                    \item \textbf{Policy Gradients}: A method that optimizes the policy directly.
                \end{itemize}
            \item \textbf{Hands-on Projects}: Engage in projects to apply learned algorithms in real-world scenarios.
        \end{itemize}
        
        \item Effective Communication of Findings
        \begin{itemize}
            \item \textbf{Data Visualization}: Techniques to visualize results effectively.
            \item \textbf{Report Writing and Presentation Skills}: Develop skills to write reports and present findings clearly.
            \item \textbf{Group Discussions}: Participate in discussions to articulate thoughts on RL topics.
        \end{itemize}
    \end{enumerate}
\end{frame}

\begin{frame}[fragile]
    \frametitle{Course Structure and Schedule - Overview}
    \begin{block}{Course Overview}
        This course on Reinforcement Learning (RL) is structured into weekly sessions that incrementally build your understanding and proficiency. 
        Each week will cover specific topics, paired with practical exercises and assessments to enhance your learning experience.
    \end{block}
\end{frame}

\begin{frame}[fragile]
    \frametitle{Course Structure and Schedule - Weekly Schedule}
    \begin{table}[ht]
        \centering
        \begin{tabular}{|c|l|l|l|}
            \hline
            Week & Topic & Key Activities & Assessment Methods \\ \hline
            1 & \textbf{Introduction to Reinforcement Learning} & - Overview of RL \\ 
              & & - Interactive examples & - Participation in discussions \\ 
              & &  & - Quiz on basic concepts \\ \hline
            2 & \textbf{Key Concepts in RL} & - Exploration vs. Exploitation \\ 
              & & - Markov Decision Processes (MDPs) & - Homework assignment \\ 
              & &  & - Problem-solving tasks \\ \hline
            3 & \textbf{Dynamic Programming in RL} & - Policy Iteration \\ 
              & & - Value Iteration & - Group project presentation \\ 
              & &  & - Code implementation \\ \hline
            4 & \textbf{Monte Carlo Methods} & - Monte Carlo Prediction \\ 
              & & - First/Every Visit methods & - Quizzes \\ 
              & &  & - Peer review of assignments \\ \hline
            5 & \textbf{Temporal-Difference Learning} & - Q-Learning \\ 
              & & - SARSA & - Case study analysis \\ 
              & &  & - Practical coding tasks \\ \hline
            6 & \textbf{Function Approximation} & - Introduction to Neural Networks \\ 
              & & - Using function approximators & - Ongoing coding projects \\ 
              & &  & - Midterm examination \\ \hline
            7 & \textbf{Deep Reinforcement Learning} & - Deep Q-Networks (DQN) \\ 
              & & - Policy Gradient methods & - Practical experimentation \\ 
              & &  & - Project proposal submission \\ \hline
            8 & \textbf{Advanced Topics in RL} & - Multi-Agent Systems \\ 
              & & - Applications of RL & - Final project submission \\ 
              & &  & - Presentation of final projects \\ \hline
        \end{tabular}
        \caption{Weekly schedule of the course.}
    \end{table}
\end{frame}

\begin{frame}[fragile]
    \frametitle{Course Structure and Schedule - Key Points}
    \begin{itemize}
        \item \textbf{Interactive Learning:} Engage with weekly topics through hands-on exercises and discussions to solidify your understanding.
        \item \textbf{Continuous Assessment:} Assessments are diverse, including quizzes, homework, projects, and presentations to accommodate different learning styles.
        \item \textbf{Collaborative Efforts:} Group projects allow for knowledge sharing and teamwork, enhancing engagement and practical learning experiences.
    \end{itemize}
\end{frame}

\begin{frame}[fragile]
    \frametitle{Course Structure and Schedule - Examples}
    \begin{itemize}
        \item \textbf{Reinforcement Learning Example:} 
        Consider a simple grid world where an agent learns to navigate to a goal while avoiding obstacles. This visual can be simulated in your assignments.
        \item \textbf{Algorithm Spotlights:}
        Each major RL algorithm will be covered with examples; for instance, Q-Learning will involve updating value functions based on agent actions in a defined environment.
    \end{itemize}
\end{frame}

\begin{frame}[fragile]
    \frametitle{Transition to Next Slide}
    Next, we will explore the essential resources and requirements for this course, ensuring you are fully equipped to engage effectively.
\end{frame}

\begin{frame}[fragile]
    \frametitle{Introduction to Reinforcement Learning: Resources and Requirements}
    To ensure a successful and engaging experience in this reinforcement learning course, the following resources, prerequisites, and technological requirements should be considered:
\end{frame}

\begin{frame}[fragile]
    \frametitle{Required Resources}
    \begin{enumerate}
        \item \textbf{Textbooks and Reading Material:}
            \begin{itemize}
                \item "Reinforcement Learning: An Introduction" by Richard S. Sutton and Andrew G. Barto serves as the primary reference for the course concepts.
                \item Additional research papers and articles will be assigned during the course for contemporary insights and applications.
            \end{itemize}
        \item \textbf{Online Resources:}
            \begin{itemize}
                \item Access video lectures or tutorials on platforms like Coursera, edX, or YouTube for enhanced understanding.
            \end{itemize}
        \item \textbf{Discussion Forums:}
            \begin{itemize}
                \item Participate in course-specific forums or platforms like Piazza or Slack to encourage collaboration and discussion.
            \end{itemize}
    \end{enumerate}
\end{frame}

\begin{frame}[fragile]
    \frametitle{Prerequisites and Technological Requirements}
    \textbf{Prerequisites:}
    \begin{enumerate}
        \item \textbf{Mathematics Background:}
            \begin{itemize}
                \item Strong understanding of linear algebra (e.g., matrices and vectors).
                \item Proficiency in calculus (e.g., differentiation and integration).
                \item Familiarity with probability and statistics, especially concepts like Markov Chains.
            \end{itemize}
        \item \textbf{Programming Skills:}
            \begin{itemize}
                \item Proficiency in Python programming, the primary language for coding assignments and projects.
                \item Familiarity with libraries such as NumPy, Pandas, and Matplotlib for data manipulation and visualization.
            \end{itemize}
    \end{enumerate}

    \vspace{1em}
    
    \textbf{Technological Requirements:}
    \begin{itemize}
        \item \textbf{Software:} Install Python (latest version) with essential libraries:
        \begin{lstlisting}
        pip install numpy pandas matplotlib gym
        \end{lstlisting}
        \item \textbf{Development Environment:} Use an IDE such as Jupyter Notebook, PyCharm, or Visual Studio Code.
        \item \textbf{Hardware:} A computer with at least 8GB of RAM (16GB preferred), multi-core processor (Intel i5/Ryzen 5 or higher), and a dedicated GPU (NVIDIA GTX 1060 or equivalent).
    \end{itemize}
\end{frame}

\begin{frame}[fragile]
    \frametitle{Conclusion and Next Steps - Summary of Key Concepts}
    
    \begin{itemize}
        \item \textbf{What We Learned in Week 1:}
        \begin{itemize}
            \item \textbf{Definition of Reinforcement Learning (RL):} 
            A type of machine learning where an agent learns to make decisions by taking actions in an environment to maximize a cumulative reward.
            \item \textbf{Components of RL:}
            \begin{itemize}
                \item \textbf{Agent:} The learner or decision-maker.
                \item \textbf{Environment:} The setting where the agent operates.
                \item \textbf{Actions:} Choices made by the agent.
                \item \textbf{States:} Situations in which the agent finds itself.
                \item \textbf{Rewards:} Feedback from the environment based on actions taken.
            \end{itemize}
        \end{itemize}
    \end{itemize}
\end{frame}

\begin{frame}[fragile]
    \frametitle{Conclusion and Next Steps - Importance of RL and Key Takeaways}

    \begin{itemize}
        \item \textbf{Importance of RL:}
        \begin{itemize}
            \item \textbf{Applications:} Reinforcement Learning is widely applied in various fields, including:
            \begin{itemize}
                \item Robotics
                \item Games (Like Chess and Go)
                \item Autonomous vehicles
                \item Healthcare decision-making
            \end{itemize}
        \end{itemize}
        
        \item \textbf{Key Takeaways:}
        \begin{itemize}
            \item \textbf{Exploration vs. Exploitation:} A fundamental trade-off in RL where an agent must balance trying new actions to discover their effects (exploration) and choosing known actions that yield high rewards (exploitation).
            \item \textbf{Learning Strategies:} We discussed various approaches, including Model-Free vs. Model-Based learning, and how these principles guide the development of effective RL agents.
        \end{itemize}
    \end{itemize}
\end{frame}

\begin{frame}[fragile]
    \frametitle{Conclusion and Next Steps - Upcoming Topics}
    
    \begin{enumerate}
        \item \textbf{Dive Deeper into Core Algorithms:}
        \begin{itemize}
            \item Expect to explore popular RL algorithms, including:
            \begin{itemize}
                \item \textbf{Q-Learning:} 
                It uses value-based learning where an agent updates its action value (Q-value) based on received rewards.
                \item \textbf{Policy Gradients:} 
                A method to optimize the policy directly rather than iteratively improving value functions.
            \end{itemize}
        \end{itemize}
        
        \item \textbf{Hands-On Programming Assignments:}
        \begin{itemize}
            \item Engage in practical exercises using Python libraries (like OpenAI Gym) to implement basic RL algorithms.
            \item Apply your learning by developing simple agents that can play games or solve mazes using RL techniques.
        \end{itemize}
        
        \item \textbf{Collaborative Learning:}
        \begin{itemize}
            \item Pair up with classmates for discussions and collaborative projects to enhance your understanding and engagement.
        \end{itemize}
    \end{enumerate}
\end{frame}


\end{document}