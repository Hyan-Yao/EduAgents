\documentclass[aspectratio=169]{beamer}

% Theme and Color Setup
\usetheme{Madrid}
\usecolortheme{whale}
\useinnertheme{rectangles}
\useoutertheme{miniframes}

% Additional Packages
\usepackage[utf8]{inputenc}
\usepackage[T1]{fontenc}
\usepackage{graphicx}
\usepackage{booktabs}
\usepackage{listings}
\usepackage{amsmath}
\usepackage{amssymb}
\usepackage{xcolor}
\usepackage{tikz}
\usepackage{pgfplots}
\pgfplotsset{compat=1.18}
\usetikzlibrary{positioning}
\usepackage{hyperref}

% Custom Colors
\definecolor{myblue}{RGB}{31, 73, 125}
\definecolor{mygray}{RGB}{100, 100, 100}
\definecolor{mygreen}{RGB}{0, 128, 0}
\definecolor{myorange}{RGB}{230, 126, 34}
\definecolor{mycodebackground}{RGB}{245, 245, 245}

% Set Theme Colors
\setbeamercolor{structure}{fg=myblue}
\setbeamercolor{frametitle}{fg=white, bg=myblue}
\setbeamercolor{title}{fg=myblue}
\setbeamercolor{section in toc}{fg=myblue}
\setbeamercolor{item projected}{fg=white, bg=myblue}
\setbeamercolor{block title}{bg=myblue!20, fg=myblue}
\setbeamercolor{block body}{bg=myblue!10}
\setbeamercolor{alerted text}{fg=myorange}

% Set Fonts
\setbeamerfont{title}{size=\Large, series=\bfseries}
\setbeamerfont{frametitle}{size=\large, series=\bfseries}
\setbeamerfont{caption}{size=\small}
\setbeamerfont{footnote}{size=\tiny}

% Document Start
\begin{document}

\frame{\titlepage}

\begin{frame}[fragile]
    \frametitle{Introduction to Dynamic Programming}
    \begin{block}{What is Dynamic Programming?}
        Dynamic programming (DP) is a problem-solving approach that breaks down complex problems into simpler subproblems, particularly in optimization contexts. It is crucial in reinforcement learning for managing and computing state values and policies.
    \end{block}

    \begin{block}{Importance in Reinforcement Learning}
        \begin{itemize}
            \item \textbf{Value Function Estimation:} Estimates value functions for various states to determine the best policy.
            \item \textbf{Policy Improvement:} Refines policies to enhance decision-making and secure optimal rewards.
            \item \textbf{Temporal-Difference Learning:} Utilizes DP to learn value functions through bootstrapping.
        \end{itemize}
    \end{block}
\end{frame}

\begin{frame}[fragile]
    \frametitle{Key Components of Dynamic Programming}
    \begin{enumerate}
        \item \textbf{States:} Different situations the agent can encounter.
        \item \textbf{Actions:} Choices available to the agent in each state.
        \item \textbf{Rewards:} Feedback received upon taking actions, indicating their success.
        \item \textbf{Transition model:} The probability of moving from one state to another after executing an action.
    \end{enumerate}
    
    \begin{block}{Core Techniques}
        The main DP algorithms in reinforcement learning are:
        \begin{itemize}
            \item \textbf{Policy Evaluation:} Determines the value function for a given policy.
            \item \textbf{Policy Improvement:} Updates the policy based on current value function.
        \end{itemize}
    \end{block}
\end{frame}

\begin{frame}[fragile]
    \frametitle{Example Application}
    Consider a gridworld where an agent navigates to a goal while avoiding obstacles:
    \begin{itemize}
        \item \textbf{States:} Each position in the grid.
        \item \textbf{Actions:} Move up, down, left, right.
        \item \textbf{Rewards:} Positive for reaching the goal, negative for hitting an obstacle.
    \end{itemize}
    
    Using DP, the agent iteratively evaluates and improves its policy, transitioning from random movements to a near-optimal path.

    \begin{block}{Key Formula}
        \textbf{Bellman Equation:}
        \begin{equation}
            V(s) = R(s) + \gamma \sum_{s'} P(s' | s, a) V(s')
        \end{equation}
        where \( V(s) \) is the value of state \( s \), \( R(s) \) is the immediate reward, 
        \( \gamma \) is the discount factor, and \( P(s' | s, a) \) is the transition probability.
    \end{block}

    \begin{block}{Conclusion}
        DP is essential in reinforcement learning, enriching the evaluation and updating of policies efficiently. Understanding its principles is key to developing intelligent agents.
    \end{block}
\end{frame}

\begin{frame}[fragile]
    \frametitle{Policy Evaluation - Definition}
    \begin{block}{Definition}
        Policy evaluation is a fundamental concept in dynamic programming and reinforcement learning. It assesses how well a given policy performs in achieving objectives within an environment.
    \end{block}
    
    \begin{itemize}
        \item A **policy** is a strategy defining actions an agent takes in each state.
        \item Evaluating a policy involves calculating the expected returns (or value) it generates across all states.
    \end{itemize}
\end{frame}

\begin{frame}[fragile]
    \frametitle{Policy Evaluation - Importance}
    \begin{block}{Importance of Policy Evaluation}
        Policy evaluation plays several key roles:
    \end{block}
    
    \begin{itemize}
        \item **Performance Assessment:** Understands the effectiveness of a policy in maximizing cumulative rewards.
        \item **Informed Decision-Making:** Identifies superior policies and aids in selection and refinement.
        \item **Feedback for Improvement:** Provides feedback mechanisms to enhance policies based on identified strengths and weaknesses.
    \end{itemize}
\end{frame}

\begin{frame}[fragile]
    \frametitle{Policy Evaluation - Bellman Equation}
    \begin{block}{How Policy Evaluation Works}
        The most common method of evaluating a policy is using the **Bellman Equation**. This equation defines the relationship between state value and expected values of subsequent states.
    \end{block}

    \begin{enumerate}
        \item **State Value Function:**
        \begin{equation}
            V^\pi(s) = \mathbb{E} \left[ G_t | S_t = s, \pi \right]
        \end{equation}
        
        \item **Bellman Equation for Policy Evaluation:**
        \begin{equation}
            V^\pi(s) = \sum_{a \in A} \pi(a|s) \sum_{s', r} p(s', r | s, a) \left[ r + \gamma V^\pi(s') \right]
        \end{equation}
    \end{enumerate}
    
    \begin{itemize}
        \item Where:
        \begin{itemize}
            \item $\pi(a|s)$ = probability of taking action $a$ in state $s$.
            \item $p(s', r | s, a)$ = transition probability of moving to state $s'$ and receiving reward $r$.
            \item $\gamma$ = discount factor balancing immediate and future rewards.
        \end{itemize}
    \end{itemize}
\end{frame}

\begin{frame}[fragile]
    \frametitle{Policy Improvement - Concept Explanation}
    \begin{block}{Definition}
        \textbf{Policy Improvement} refers to the process of enhancing a policy in reinforcement learning or dynamic programming based on the results obtained from the evaluation of that policy. 
    \end{block}
    After assessing how well a policy performs (as covered in our previous slide on Policy Evaluation), we utilize that information to refine our decision-making strategies, aiming to achieve greater rewards or significantly better outcomes.
\end{frame}

\begin{frame}[fragile]
    \frametitle{Policy Improvement - Importance}
    \begin{itemize}
        \item \textbf{Enhances Decision Quality:} Systematically improving policies can optimize strategies for better performance in various environments.
        \item \textbf{Convergence to Optimal Solutions:} Continuous improvement helps in gradually reaching the optimal policy, providing the best possible action in every state.
        \item \textbf{Dynamic Adaptation:} Policies can adjust to environmental changes, ensuring strategies remain effective over time.
    \end{itemize}
\end{frame}

\begin{frame}[fragile]
    \frametitle{Policy Improvement - Techniques}
    \begin{enumerate}
        \item \textbf{Greedy Improvement:}
            \begin{itemize}
                \item At each state, choose the action that maximizes the expected reward based on the current policy.
                \item \textbf{Formula:} 
                \[
                \pi' (s) = \arg\max_a Q(s, a)
                \]
                Where \(\pi'\) is the improved policy, \(s\) is the state, and \(Q(s, a)\) is the action-value function.
            \end{itemize}
        
        \item \textbf{Policy Gradient Methods:}
            \begin{itemize}
                \item Utilize gradients to optimize the policy parameters directly.
                \item \textbf{Basic Formula:}
                \[
                \nabla J(\theta) = \mathbb{E}[\nabla \log(\pi_\theta(a|s)) \cdot G]
                \]
                Where \(J(\theta)\) is the objective function, \(G\) is the cumulative reward, and \(\theta\) are the policy parameters.
            \end{itemize}
        
        \item \textbf{Value Iteration and Policy Iteration:}
            \begin{itemize}
                \item Re-evaluate state values and adjust policies iteratively until no further improvement can be made.
            \end{itemize}
    \end{enumerate}
\end{frame}

\begin{frame}[fragile]
    \frametitle{Policy Improvement - Example Illustration}
    Suppose we are training a robot to navigate a maze: 
    \begin{itemize}
        \item \textbf{Initial Policy (\(\pi_0\)):} The robot makes random moves.
        \item \textbf{Policy Evaluation:} Assess how well the robot navigates via simulations, with average steps taken to reach the goal.
        \item \textbf{Policy Improvement:} If evaluation shows the robot frequently hits obstacles, modify the policy to avoid certain actions (e.g., moving forward when in proximity to walls).
    \end{itemize}
    
    \textbf{Key Takeaway:} By leveraging evaluation results to guide action selection within our policy, we systematically guide the learning agent toward not just better performance but optimal behavior.
\end{frame}

\begin{frame}[fragile]
    \frametitle{Policy Improvement - Conclusion}
    As we move toward our next topic on \textbf{Policy Iteration}, remember:
    \begin{itemize}
        \item Policy Improvement is a critical step that relies on evaluating current strategies and systematically enhancing decision-making opportunities.
        \item Through constant refinement using models and principles outlined, we ensure our policies are robust and yield optimal results.
    \end{itemize}
\end{frame}

\begin{frame}[fragile]
    \frametitle{Policy Iteration - Introduction}
    \begin{block}{Introduction to Policy Iteration}
        Policy Iteration is a fundamental algorithm in dynamic programming, especially within reinforcement learning and Markov Decision Processes (MDPs). It systematically derives the optimal policy, which defines the best action to take in each state.
    \end{block}
\end{frame}

\begin{frame}[fragile]
    \frametitle{Policy Iteration - How it Works}
    Policy Iteration consists of two main steps, repeated iteratively:
    
    \begin{enumerate}
        \item \textbf{Policy Evaluation}:
        \begin{itemize}
            \item Assess the value of the current policy to calculate expected returns for each state.
            \item Value function \( V(s) \) calculated using the Bellman equation:
            \begin{equation}
                V^\pi(s) = R(s) + \gamma \sum_{s'} P(s' | s, \pi(s)) V^\pi(s')
            \end{equation}
            where:
            \begin{itemize}
                \item \( R(s) \) is the immediate reward for state \( s \),
                \item \( \gamma \) is the discount factor,
                \item \( P(s' | s, \pi(s)) \) is the transition probability to state \( s' \) from state \( s \) under policy \( \pi \).
            \end{itemize}
        \end{itemize}
        
        \item \textbf{Policy Improvement}:
        \begin{itemize}
            \item Update the policy to a new policy that yields a higher or equal value using:
            \begin{equation}
                \pi'(s) = \text{argmax}_a \left( R(s) + \gamma \sum_{s'} P(s' | s, a) V^\pi(s') \right)
            \end{equation}
            \item If \( \pi' = \pi \), the process ends, indicating the optimal policy is found.
        \end{itemize}
    \end{enumerate}
\end{frame}

\begin{frame}[fragile]
    \frametitle{Policy Iteration - Key Points and Example}
    \begin{block}{Key Points}
        \begin{itemize}
            \item \textbf{Iterative Nature}: Continues until policy stabilizes.
            \item \textbf{Convergence}: Guaranteed to reach the optimal policy for finite states and actions.
            \item \textbf{Efficiency}: Generally faster convergence compared to Value Iteration, especially in smaller spaces.
        \end{itemize}
    \end{block}

    \begin{block}{Example}
        Consider an MDP with three states (S1, S2, S3) and two actions (A1, A2):
        \begin{itemize}
            \item Start with an arbitrary policy.
            \item Evaluate using the Bellman equation.
            \item Update the policy based on evaluated values.
            \item Repeat until the policy no longer changes.
        \end{itemize}
    \end{block}
    
    \begin{block}{Takeaway}
        Policy Iteration is essential for deriving optimal policies, crucial for decision-making in uncertain environments.
    \end{block}
\end{frame}

\begin{frame}[fragile]
  \frametitle{Dynamic Programming Algorithms - Overview}
  \begin{itemize}
    \item Dynamic Programming (DP) is a technique used for optimization by breaking problems into subproblems.
    \item Key Algorithms:
    \begin{itemize}
      \item Value Iteration
      \item Policy Iteration
    \end{itemize}
  \end{itemize}
\end{frame}

\begin{frame}[fragile]
  \frametitle{Dynamic Programming Algorithms - Value Iteration}
  \begin{block}{Concept}
    \begin{itemize}
      \item Computes the optimal value function for Markov Decision Processes (MDPs).
      \item Updates state values using the Bellman equation:
      \end{itemize}
      \begin{equation}
        V(s) \leftarrow \max_a \sum_{s'} P(s'|s,a) [R(s,a,s') + \gamma V(s')]
      \end{equation}
      \begin{itemize}
        \item Where:
        \begin{itemize}
          \item \( V(s) \): Value of state \( s \)
          \item \( P(s'|s,a) \): Transition probability
          \item \( R(s,a,s') \): Reward for the transition
          \item \( \gamma \): Discount factor (0 < $\gamma$ < 1)
        \end{itemize}
    \end{itemize}
  \end{block}
\end{frame}

\begin{frame}[fragile]
  \frametitle{Dynamic Programming Algorithms - Policy Iteration}
  \begin{block}{Concept}
    \begin{itemize}
      \item Composed of two main steps: Policy Evaluation and Policy Improvement.
      \item Evaluates the current policy to determine the value of each state.
    \end{itemize}
  \end{block}
  
  \begin{block}{Steps}
    \begin{enumerate}
      \item \textbf{Policy Evaluation}:
        \begin{equation}
          V^\pi(s) = \sum_{s'} P(s'|s,\pi(s)) [R(s,\pi(s),s') + \gamma V^\pi(s')]
        \end{equation}
      \item \textbf{Policy Improvement}:
        \begin{equation}
          \pi_{new}(s) = \arg\max_a \sum_{s'} P(s'|s,a) [R(s,a,s') + \gamma V^\pi(s')]
        \end{equation}
    \end{enumerate}
  \end{block}
\end{frame}

\begin{frame}[fragile]
    \frametitle{Implementation in Simulated Environments - Introduction}
    % Introduction to dynamic programming in simulated environments
    Dynamic programming (DP) is a technique used in reinforcement learning (RL) to solve complex decision-making problems.
    
    \begin{itemize}
        \item Breaks down problems into simpler subproblems.
        \item Facilitates training and evaluation in controlled settings.
        \item Models real-world scenarios effectively.
    \end{itemize}
\end{frame}

\begin{frame}[fragile]
    \frametitle{Key Concepts in RL}
    % Key concepts related to state representation, actions, rewards, and transition dynamics
    \begin{enumerate}
        \item **State Representation (S):**
            \begin{itemize}
                \item Represents specific situations an agent encounters.
                \item Example: Chess configurations as states.
            \end{itemize}
        
        \item **Action Set (A):**
            \begin{itemize}
                \item Actions influence transitions between states.
                \item Example: Movements in a grid world (up, down, left, right).
            \end{itemize}

        \item **Rewards (R):**
            \begin{itemize}
                \item Feedback received from the environment.
                \item Guides the agent's learning and behavior reinforcement.
            \end{itemize}
        
        \item **Transition Dynamics:**
            \begin{itemize}
                \item State transitions can be deterministic or stochastic.
                \item Understanding dynamics is crucial for policy evaluation.
            \end{itemize}
    \end{enumerate}
\end{frame}

\begin{frame}[fragile]
    \frametitle{Implementation Strategies in DP}
    % Overview of implementation strategies including value and policy iteration
    \begin{block}{Value Iteration}
        \begin{itemize}
            \item Computes optimal policy and value function iteratively.
            \item Example: Each cell in a grid world holds a value representing expected utility.
            \item **Updating Formula:**
            \begin{equation}
                V(s) \leftarrow \max_a \sum_{s'} P(s'|s,a) [R(s,a,s') + \gamma V(s')]
            \end{equation}
            where $\gamma$ is the discount factor.
        \end{itemize}
    \end{block}
    
    \begin{block}{Policy Iteration}
        \begin{itemize}
            \item Alternates between policy evaluation and policy improvement.
            \item Example: Start with a random policy, evaluate its value, and update greedily.
        \end{itemize}
    \end{block}
\end{frame}

\begin{frame}[fragile]
  \frametitle{Comparative Analysis - Introduction to Dynamic Programming}
  \begin{block}{Dynamic Programming (DP)}
      DP is a model-based reinforcement learning approach that:
      \begin{itemize}
          \item Leverages the principles of optimality.
          \item Breaks problems down into simpler subproblems.
          \item Requires knowledge of the environment's dynamics (transition probabilities and rewards).  
      \end{itemize}
  \end{block}

  \begin{block}{Key DP Methods}
      \begin{enumerate}
          \item \textbf{Value Iteration}: Updates the value function iteratively until convergence.
          \item \textbf{Policy Iteration}: Alternates between policy evaluation and policy improvement until the optimal policy is found.
      \end{enumerate}
  \end{block}
\end{frame}

\begin{frame}[fragile]
  \frametitle{Comparative Analysis - Dynamic Programming vs. Monte Carlo Methods}
  \begin{block}{Comparison Criteria}
      \begin{itemize}
          \item \textbf{Model Requirement}:
              \begin{itemize}
                  \item DP: Requires a complete model of the environment.
                  \item Monte Carlo: Does not require a model; relies on sampled experiences.
              \end{itemize}

          \item \textbf{Learning Method}:
              \begin{itemize}
                  \item DP: Systematically updates value function.
                  \item Monte Carlo: Averages returns from complete episodes.
              \end{itemize}
      \end{itemize}
  \end{block}

  \begin{block}{Example Scenario}
      In a grid world:
      \begin{itemize}
          \item DP calculates expected utility using transition probabilities.
          \item Monte Carlo simulates episodes and updates value function from experiences.
      \end{itemize}
  \end{block}
\end{frame}

\begin{frame}[fragile]
  \frametitle{Comparative Analysis - Dynamic Programming vs. Temporal-Difference Learning}
  \begin{block}{Key Comparisons}
      \begin{itemize}
          \item \textbf{Model Requirement}:
              \begin{itemize}
                  \item DP: Needs a model.
                  \item TD: Learns directly from experience without a model.
              \end{itemize}

          \item \textbf{Learning Mechanism}:
              \begin{itemize}
                  \item DP: Updates estimates using Bellman equation.
                  \item TD: Uses bootstrapping to update estimates from previous state values.
              \end{itemize}
      \end{itemize}
  \end{block}

  \begin{block}{Example Scenario}
      Using TD(0):
      \begin{itemize}
          \item An agent updates value function immediately after observing the next state and reward.
          \item Faster learning loop compared to DP's multi-step updates.
      \end{itemize}
  \end{block}
\end{frame}

\begin{frame}
    \titlepage
\end{frame}

\begin{frame}
    \frametitle{Introduction to Challenges}
    \begin{itemize}
        \item Dynamic Programming (DP) is a powerful technique in Reinforcement Learning (RL).
        \item Its application can present several challenges and limitations.
        \item Understanding these challenges is crucial for optimizing DP algorithms in RL.
    \end{itemize}
\end{frame}

\begin{frame}
    \frametitle{Common Challenges and Limitations - Part 1}
    \begin{enumerate}
        \item \textbf{Curse of Dimensionality}
            \begin{itemize}
                \item As the state or action space grows, computations increase exponentially.
                \item Example: A 10x10 grid is manageable; a 100x100 grid becomes impractical.
            \end{itemize}

        \item \textbf{Computationally Intensive}
            \begin{itemize}
                \item DP methods require repeated updates, which are computationally expensive.
                \item Example: Policy evaluation can require thousands of iterations for convergence.
            \end{itemize} 
    \end{enumerate}
\end{frame}

\begin{frame}
    \frametitle{Common Challenges and Limitations - Part 2}
    \begin{enumerate}[resume]
        \item \textbf{Model Dependency}
            \begin{itemize}
                \item DP requires a complete model (transition probabilities, reward functions).
                \item Example: A robotic agent in a partially observable environment may struggle.
            \end{itemize}

        \item \textbf{Convergence Issues}
            \begin{itemize}
                \item Problems like numerical stability can hinder convergence.
                \item Example: Poorly chosen function approximators may fail to converge to true values.
            \end{itemize}

        \item \textbf{Exploration vs. Exploitation Trade-off}
            \begin{itemize}
                \item DP assumes model knowledge; suboptimal selections may occur if exploration is insufficient.
                \item Example: An agent may miss more rewarding strategies by only exploiting known rewards.
            \end{itemize}
    \end{enumerate}
\end{frame}

\begin{frame}
    \frametitle{Key Points to Emphasize}
    \begin{itemize}
        \item Address the \textbf{curse of dimensionality} with state aggregation or function approximation.
        \item Balance the \textbf{computational load} using efficient approaches or parallel processing.
        \item Recognize \textbf{model uncertainty} as a significant barrier in dynamic environments.
        \item Be aware of \textbf{convergence problems} and choose suitable algorithms.
        \item Understand the importance of the \textbf{exploration-exploitation trade-off}.
    \end{itemize}
\end{frame}

\begin{frame}
    \frametitle{Conclusion}
    \begin{itemize}
        \item Dynamic Programming is foundational in Reinforcement Learning.
        \item Practitioners must handle inherent challenges to benefit fully from DP methods.
        \item Awareness of limitations leads to better algorithm design and implementation strategies.
    \end{itemize}
\end{frame}

\begin{frame}[fragile]
    \frametitle{Illustration of the Concept}
    \begin{block}{Diagram Idea}
        Show a flowchart of the DP process with annotations for challenges like:
        \begin{itemize}
            \item Curse of Dimensionality
            \item Exploration Issues
            \item Model Dependency
        \end{itemize}
    \end{block}

    \begin{block}{Code Snippet Example}
        \begin{lstlisting}
def value_iteration(states, actions, transition_model, rewards, discount_factor, theta):
    V = {s: 0 for s in states}  # Initialize value function
    while True:
        delta = 0
        for s in states:
            v = V[s]
            V[s] = max(sum(transition_model[s][a][s_next] * (rewards[s, a, s_next] + discount_factor * V[s_next]) 
                            for s_next in states) for a in actions)
            delta = max(delta, abs(v - V[s]))
        if delta < theta:
            break
    return V
        \end{lstlisting}
    \end{block}
\end{frame}

\begin{frame}[fragile]
    \frametitle{Future Directions in Dynamic Programming - Overview}
    \begin{block}{Overview}
        Dynamic Programming (DP) has long been a cornerstone in reinforcement learning (RL), solving complex decision-making problems by breaking them into simpler subproblems. Recent advancements in DP methodologies have opened new avenues for improving learning efficiency and tackling previously unsolved challenges.
    \end{block}
\end{frame}

\begin{frame}[fragile]
    \frametitle{Future Directions in Dynamic Programming - Recent Advancements}
    \begin{enumerate}
        \item \textbf{Approximate Dynamic Programming (ADP)}:
            \begin{itemize}
                \item Utilizes function approximation for value functions and policies, improving scalability in high-dimensional spaces.
                \item Key: Reduces computational burden and memory requirements.
                \item Example: Neural networks approximate the value function in complex state spaces.
            \end{itemize}
        
        \item \textbf{Model-Free and Model-Based Approaches}:
            \begin{itemize}
                \item Hybrid models enhance efficiency by integrating strengths of both approaches.
                \item Key: Agents can plan actions using a learned model, reducing exploration in unknown environments.
                \item Example: AlphaZero combines deep learning with Monte Carlo Tree Search.
            \end{itemize}
        
        \item \textbf{Incorporation of Deep Learning}:
            \begin{itemize}
                \item Deep Reinforcement Learning (DRL) enables handling vast state spaces with neural networks.
                \item Key: Enables application of DP techniques in complex environments.
                \item Example: AlphaGo's evaluation function improves DP-based algorithms.
            \end{itemize}
        
        \item \textbf{Generalized Policy Iteration (GPI)}:
            \begin{itemize}
                \item Expanded GPI framework facilitates better policy evaluation and improvement interactions.
                \item Key: Enhances robustness and adaptability to changing environments.
                \item Example: Modern adaptations like SARSA(λ) improve convergence and efficiency.
            \end{itemize}
    \end{enumerate}
\end{frame}

\begin{frame}[fragile]
    \frametitle{Future Directions in Dynamic Programming - Implications}
    \begin{block}{Implications for Reinforcement Learning}
        \begin{itemize}
            \item \textbf{Scalability:} Enhanced efficiency allows RL agents to tackle larger, complex tasks.
            \item \textbf{Practicality:} Real-world applications like self-driving cars and robotics benefit significantly.
            \item \textbf{New Opportunities:} Continuous enhancements in DP could provide innovative solutions to emerging problems in AI.
        \end{itemize}
    \end{block}

    \begin{block}{Concluding Remarks}
        Advancements in dynamic programming are transforming reinforcement learning by improving performance and integrating deep learning techniques, promising to shape AI’s future across various domains.
    \end{block}
\end{frame}


\end{document}