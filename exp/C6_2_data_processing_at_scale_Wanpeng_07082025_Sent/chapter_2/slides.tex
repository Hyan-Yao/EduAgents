\documentclass[aspectratio=169]{beamer}

% Theme and Color Setup
\usetheme{Madrid}
\usecolortheme{whale}
\useinnertheme{rectangles}
\useoutertheme{miniframes}

% Additional Packages
\usepackage[utf8]{inputenc}
\usepackage[T1]{fontenc}
\usepackage{graphicx}
\usepackage{booktabs}
\usepackage{listings}
\usepackage{amsmath}
\usepackage{amssymb}
\usepackage{xcolor}
\usepackage{tikz}
\usepackage{pgfplots}
\pgfplotsset{compat=1.18}
\usetikzlibrary{positioning}
\usepackage{hyperref}

% Custom Colors
\definecolor{myblue}{RGB}{31, 73, 125}
\definecolor{mygray}{RGB}{100, 100, 100}
\definecolor{mygreen}{RGB}{0, 128, 0}
\definecolor{myorange}{RGB}{230, 126, 34}
\definecolor{mycodebackground}{RGB}{245, 245, 245}

% Set Theme Colors
\setbeamercolor{structure}{fg=myblue}
\setbeamercolor{frametitle}{fg=white, bg=myblue}
\setbeamercolor{title}{fg=myblue}
\setbeamercolor{section in toc}{fg=myblue}
\setbeamercolor{item projected}{fg=white, bg=myblue}
\setbeamercolor{block title}{bg=myblue!20, fg=myblue}
\setbeamercolor{block body}{bg=myblue!10}
\setbeamercolor{alerted text}{fg=myorange}

% Set Fonts
\setbeamerfont{title}{size=\Large, series=\bfseries}
\setbeamerfont{frametitle}{size=\large, series=\bfseries}
\setbeamerfont{caption}{size=\small}
\setbeamerfont{footnote}{size=\tiny}

% Code Listing Style
\lstdefinestyle{customcode}{
  backgroundcolor=\color{mycodebackground},
  basicstyle=\footnotesize\ttfamily,
  breakatwhitespace=false,
  breaklines=true,
  commentstyle=\color{mygreen}\itshape,
  keywordstyle=\color{blue}\bfseries,
  stringstyle=\color{myorange},
  numbers=left,
  numbersep=8pt,
  numberstyle=\tiny\color{mygray},
  frame=single,
  framesep=5pt,
  rulecolor=\color{mygray},
  showspaces=false,
  showstringspaces=false,
  showtabs=false,
  tabsize=2,
  captionpos=b
}
\lstset{style=customcode}

% Custom Commands
\newcommand{\hilight}[1]{\colorbox{myorange!30}{#1}}
\newcommand{\source}[1]{\vspace{0.2cm}\hfill{\tiny\textcolor{mygray}{Source: #1}}}
\newcommand{\concept}[1]{\textcolor{myblue}{\textbf{#1}}}
\newcommand{\separator}{\begin{center}\rule{0.5\linewidth}{0.5pt}\end{center}}

% Footer and Navigation Setup
\setbeamertemplate{footline}{
  \leavevmode%
  \hbox{%
  \begin{beamercolorbox}[wd=.3\paperwidth,ht=2.25ex,dp=1ex,center]{author in head/foot}%
    \usebeamerfont{author in head/foot}\insertshortauthor
  \end{beamercolorbox}%
  \begin{beamercolorbox}[wd=.5\paperwidth,ht=2.25ex,dp=1ex,center]{title in head/foot}%
    \usebeamerfont{title in head/foot}\insertshorttitle
  \end{beamercolorbox}%
  \begin{beamercolorbox}[wd=.2\paperwidth,ht=2.25ex,dp=1ex,center]{date in head/foot}%
    \usebeamerfont{date in head/foot}
    \insertframenumber{} / \inserttotalframenumber
  \end{beamercolorbox}}%
  \vskip0pt%
}

% Turn off navigation symbols
\setbeamertemplate{navigation symbols}{}

% Title Page Information
\title[Chapter: Hadoop Ecosystem]{Week 2: Hadoop Ecosystem}
\author[J. Smith]{John Smith, Ph.D.}
\institute[University Name]{
  Department of Computer Science\\
  University Name\\
  \vspace{0.3cm}
  Email: email@university.edu\\
  Website: www.university.edu
}
\date{\today}

% Document Start
\begin{document}

\frame{\titlepage}

\begin{frame}[fragile]
    \frametitle{Introduction to Hadoop Ecosystem}
    \begin{block}{Overview}
        An overview of Hadoop as a framework for distributed storage and processing of large datasets.
    \end{block}
\end{frame}

\begin{frame}[fragile]
    \frametitle{What is Hadoop?}
    \begin{itemize}
        \item Hadoop is an open-source framework for distributed storage and processing.
        \item It allows scaling from a single server to thousands of machines.
        \item Designed to enable local computation and storage.
    \end{itemize}
\end{frame}

\begin{frame}[fragile]
    \frametitle{Key Components of Hadoop}
    \begin{enumerate}
        \item \textbf{Hadoop Common}: Utilities and libraries for other modules.
        \item \textbf{Hadoop Distributed File System (HDFS)}: Distributed storage system for large files.
        \item \textbf{Yet Another Resource Negotiator (YARN)}: Resource management and scheduling layer.
        \item \textbf{Hadoop MapReduce}: Programming model for processing large data sets.
    \end{enumerate}
\end{frame}

\begin{frame}[fragile]
    \frametitle{How Does Hadoop Work?}
    \begin{itemize}
        \item \textbf{Data Storage}: Data is partitioned into blocks (default size: 128 MB) and stored across the cluster in HDFS.
        \item \textbf{Data Processing}: Tasks are defined as MapReduce jobs to process data in parallel.
    \end{itemize}
\end{frame}

\begin{frame}[fragile]
    \frametitle{Example Use Case: Log Analysis}
    \begin{itemize}
        \item Imagine a website generating millions of log entries daily.
        \item Using Hadoop, businesses can:
        \begin{itemize}
            \item Store logs in HDFS.
            \item Analyze access patterns, user behavior, and detect anomalies using MapReduce.
        \end{itemize}
    \end{itemize}
\end{frame}

\begin{frame}[fragile]
    \frametitle{Advantages of Hadoop}
    \begin{itemize}
        \item \textbf{Scalability}: Easily add more nodes as data volume grows.
        \item \textbf{Cost-Effectiveness}: Uses commodity hardware to lower operational costs.
        \item \textbf{Fault Tolerance}: Data replication across nodes ensures reliability.
        \item \textbf{Flexibility}: Handles various data types (structured, semi-structured, unstructured).
    \end{itemize}
\end{frame}

\begin{frame}[fragile]
    \frametitle{Key Points to Remember}
    \begin{itemize}
        \item Hadoop enables quick and efficient processing of vast data amounts.
        \item Its architecture provides scalability and fault tolerance crucial for big data.
        \item Understanding the Hadoop ecosystem is vital for leveraging big data analytics.
    \end{itemize}
\end{frame}

\begin{frame}[fragile]
    \frametitle{Diagram: Hadoop Ecosystem Overview}
    \begin{block}{Visual Representation}
        A diagram illustrating the relationship between the components: HDFS, MapReduce, and YARN, showing how data moves through the system.
    \end{block}
\end{frame}

\begin{frame}[fragile]
    \frametitle{Conclusion}
    \begin{itemize}
        \item The Hadoop ecosystem is essential for organizations leveraging big data.
        \item It facilitates distributed storage and processing for real-time data-driven decisions.
    \end{itemize}
\end{frame}

\begin{frame}[fragile]
    \frametitle{Next Steps}
    \begin{block}{Looking Ahead}
        Next, we will discuss the concept of Big Data, including its definition, characteristics, and management challenges.
    \end{block}
\end{frame}

\begin{frame}[fragile]
    \frametitle{Understanding Big Data - Definition}
    \begin{block}{Definition of Big Data}
        Big Data refers to extremely large datasets that may be analyzed computationally to reveal patterns, trends, and associations, particularly concerning human behavior and interactions. It encompasses data generated from various sources such as social media, sensors, devices, videos, and online transactions.
    \end{block}
\end{frame}

\begin{frame}[fragile]
    \frametitle{Understanding Big Data - Characteristics}
    \begin{block}{Characteristics of Big Data}
        Big Data can be characterized by the ``3 Vs'' which encapsulate its key attributes:
    \end{block}
    
    \begin{enumerate}
        \item \textbf{Volume:} 
        \begin{itemize}
            \item Refers to the massive amount of data generated every second. 
            \item Example: Global internet traffic was projected to reach 4.8 zettabytes in 2022.
        \end{itemize}

        \item \textbf{Velocity:} 
        \begin{itemize}
            \item Describes the speed at which data is generated and processed. 
            \item Example: Thousands of tweets are sent every second on Twitter.
        \end{itemize}

        \item \textbf{Variety:} 
        \begin{itemize}
            \item Indicates different types of data formats, including structured (databases), semi-structured (XML, JSON), and unstructured (text, images, videos).
            \item Example: A single customer interaction may include various formats of data.
        \end{itemize}
    \end{enumerate}
\end{frame}

\begin{frame}[fragile]
    \frametitle{Understanding Big Data - Significance and Challenges}
    \begin{block}{Significance of Big Data}
        \begin{itemize}
            \item Data-Driven Decision Making: Enhances business strategies and operations.
            \item Innovation: Drives advancements in various sectors such as healthcare and finance.
        \end{itemize}
    \end{block}
    
    \begin{block}{Challenges of Big Data}
        \begin{enumerate}
            \item \textbf{Storage and Management:} Requires scalable storage solutions.
            \item \textbf{Processing Complexity:} Necessitates advanced tools like Hadoop.
            \item \textbf{Data Security and Privacy:} Protecting sensitive information is critical.
            \item \textbf{Skill Gaps:} Shortage of skilled data scientists affects analysis capabilities.
        \end{enumerate}
    \end{block}
    
    \begin{block}{Conclusion}
        Understanding Big Data's definition, characteristics, significance, and challenges is crucial for navigating its potential and exploring tools such as the Hadoop Ecosystem.
    \end{block}
\end{frame}

\begin{frame}[fragile]
    \frametitle{Hadoop Architecture - Overview}
    \begin{block}{Understanding Hadoop Architecture}
        Hadoop is designed to manage large datasets in a distributed computing environment, which is critical for efficiently processing Big Data across computer clusters.
    \end{block}
\end{frame}

\begin{frame}[fragile]
    \frametitle{Key Components of Hadoop Architecture}
    \begin{enumerate}
        \item \textbf{Hadoop Common}
        \begin{itemize}
            \item Common utilities and libraries for Hadoop modules.
            \item Includes necessary Java files and scripts.
        \end{itemize}
        
        \item \textbf{Hadoop Distributed File System (HDFS)}
        \begin{itemize}
            \item Stores large datasets across multiple machines.
            \item Splits data into blocks (default size: 128 MB or 256 MB).
            \item Replicates each block (default replication factor: 3) for fault tolerance.
        \end{itemize}
        
        \item \textbf{YARN (Yet Another Resource Negotiator)}
        \begin{itemize}
            \item Resource management layer for scheduling and resource allocation.
            \item Decouples resource management from data processing.
        \end{itemize}
        
        \item \textbf{MapReduce}
        \begin{itemize}
            \item Data processing model enabling parallel data processing.
            \item Consists of two main functions: Map and Reduce.
        \end{itemize}
    \end{enumerate}
\end{frame}

\begin{frame}[fragile]
    \frametitle{How It Works: A Simplified Example}
    \begin{block}{Example Workflow}
        Consider analyzing a massive dataset of customer transactions:
        \begin{enumerate}
            \item \textbf{Step 1}: Break the dataset into smaller blocks.
            \item \textbf{Step 2}: HDFS distributes the blocks across the cluster.
            \item \textbf{Step 3}: YARN allocates resources for processing.
            \item \textbf{Step 4}: MapReduce processes data in parallel.
            \begin{itemize}
                \item In the \textbf{Map} phase, count the number of purchases by each customer.
                \item In the \textbf{Reduce} phase, sum up these counts.
            \end{itemize}
        \end{enumerate}
    \end{block}
\end{frame}

\begin{frame}[fragile]
    \frametitle{Hadoop Distributed File System (HDFS)}
    \begin{block}{What is HDFS?}
        The Hadoop Distributed File System (HDFS) is designed for storing large datasets across multiple machines.
        \begin{itemize}
            \item Enables high-throughput access to application data.
            \item Highly fault-tolerant, ideal for big data applications.
        \end{itemize}
    \end{block}
\end{frame}

\begin{frame}[fragile]
    \frametitle{Key Architecture Components of HDFS}
    \begin{enumerate}
        \item \textbf{NameNode}:
        \begin{itemize}
            \item The master server managing metadata.
            \item Tracks file names, directory structure, and data block mapping.
            \item A single point of failure, though can be replicated.
        \end{itemize}
        
        \item \textbf{DataNode}:
        \begin{itemize}
            \item Slave servers storing data blocks.
            \item Handles read/write requests and sends heartbeat signals to NameNode.
            \item Manages block creation, deletion, and replication.
        \end{itemize}
        
        \item \textbf{Block}:
        \begin{itemize}
            \item Basic unit of storage in HDFS.
            \item Files are divided into blocks, typically 128 MB in size.
            \item Blocks are replicated for fault tolerance (default factor: 3).
        \end{itemize}
    \end{enumerate}
\end{frame}

\begin{frame}[fragile]
    \frametitle{Data Storage and Example Workflow}
    \begin{block}{Data Storage in HDFS}
        \begin{itemize}
            \item \textbf{Distributed Storage:} Files are split into blocks across a clustered environment enabling parallel processing.
            \item \textbf{Fault Tolerance:} Automatically reroutes requests in case of DataNode failure.
            \item \textbf{Data Locality Optimization:} Runs computations where data is stored to reduce network overhead.
        \end{itemize}
    \end{block}
    
    \begin{block}{Example Workflow}
        1. User uploads a large file (1 GB).
        2. HDFS splits the file into eight blocks (128 MB each).
        3. NameNode records metadata and placement of each block.
        4. Blocks are distributed across DataNodes.
        5. Users can access the file quickly due to its distributed nature.
    \end{block}
\end{frame}

\begin{frame}[fragile]
    \frametitle{YARN: Yet Another Resource Negotiator}
    \begin{block}{Overview}
        YARN is a critical component of the Hadoop ecosystem that serves as the resource management layer. It decouples resource management and job scheduling from data processing, enabling multiple data processing engines to run and share resources dynamically on Hadoop clusters.
    \end{block}
\end{frame}

\begin{frame}[fragile]
    \frametitle{YARN: Key Concepts}
    \begin{itemize}
        \item \textbf{Resource Manager (RM)}: 
            \begin{itemize}
                \item The master daemon managing resource allocation across applications.
                \item Composed of the Scheduler and the Application Manager.
            \end{itemize}
        \item \textbf{Node Manager (NM)}: 
            \begin{itemize}
                \item A per-node daemon monitoring resource usage of containers.
                \item Reports metrics to the Resource Manager.
            \end{itemize}
        \item \textbf{Application Master (AM)}: 
            \begin{itemize}
                \item Unique instance per application that negotiates resources with the Resource Manager and interacts with Node Managers.
            \end{itemize}
    \end{itemize}
\end{frame}

\begin{frame}[fragile]
    \frametitle{YARN: Key Functions}
    \begin{itemize}
        \item \textbf{Resource Management}: 
            \begin{itemize}
                \item Ensures efficient utilization of CPU, memory, and storage across applications.
            \end{itemize}
        \item \textbf{Job Scheduling}: 
            \begin{itemize}
                \item Assigns resources to applications using pluggable scheduler implementations.
            \end{itemize}
    \end{itemize}
    
    \begin{block}{Example Scenario}
        Consider a big data application requiring processing large datasets:
        \begin{itemize}
            \item \textbf{Real-time Analytics}: 4 GB memory, 2 CPUs.
            \item \textbf{Batch Processing}: 8 GB memory, 4 CPUs.
        \end{itemize}
        Both can run simultaneously with YARN's dynamic resource allocation.
    \end{block}
\end{frame}

\begin{frame}[fragile]
    \frametitle{YARN: Conclusion and Next Steps}
    \begin{itemize}
        \item YARN decouples resource management from processing frameworks, allowing for scalability and flexibility.
        \item It enables dynamic scheduling for concurrent application execution without interference.
        \item Improved resource utilization maximizes cluster efficiency.
    \end{itemize}
    
    \begin{block}{Transition}
        In the next slide, we will explore \textbf{Data Processing with MapReduce}, examining how YARN orchestrates resources for efficient data processing.
    \end{block}
\end{frame}

\begin{frame}[fragile]
    \frametitle{Data Processing with MapReduce}
    \begin{block}{Introduction to MapReduce}
        MapReduce is a programming model and processing technique designed for handling large datasets across distributed systems. 
        Developed by Google, it simplifies data processing by breaking tasks into manageable pieces, allowing for parallel processing.
    \end{block}
\end{frame}

\begin{frame}[fragile]
    \frametitle{How MapReduce Works}
    \begin{enumerate}
        \item \textbf{Map Phase:}
        \begin{itemize}
            \item Input data is split into smaller sub-problems processed in parallel.
            \item Each mapper applies a function to key-value pairs to produce intermediate results.
            \item \textbf{Example:} Word count task emits pairs of words and the number 1 for each occurrence. 
            \begin{verbatim}
Input: "apple banana apple"
Output: [("apple", 1), ("banana", 1), ("apple", 1)]
            \end{verbatim}
        \end{itemize}
        
        \item \textbf{Shuffle and Sort Phase:}
        \begin{itemize}
            \item Sorts and categorizes intermediate pairs to group values by key.
            \item Crucial for Reducers to aggregate data correctly.
        \end{itemize}
        
        \item \textbf{Reduce Phase:}
        \begin{itemize}
            \item Aggregates results from sorted intermediate data.
            \item \textbf{Example:} Reducer sums values for each word.
            \begin{verbatim}
Input: [("apple", 1), ("apple", 1), ("banana", 1)]
Output: [("apple", 2), ("banana", 1)]
            \end{verbatim}
        \end{itemize}
    \end{enumerate}
\end{frame}

\begin{frame}[fragile]
    \frametitle{Significance of MapReduce}
    \begin{itemize}
        \item \textbf{Scalability:} Efficiently scales across thousands of machines.
        \item \textbf{Fault Tolerance:} Automatically manages failures by redirecting tasks.
        \item \textbf{Cost-Effectiveness:} Utilizes commodity hardware to reduce infrastructure costs.
        \item \textbf{Parallel Processing:} Allows simultaneous task execution for faster data processing.
    \end{itemize}
\end{frame}

\begin{frame}[fragile]
    \frametitle{Key Points to Remember}
    \begin{itemize}
        \item Integral to the Hadoop ecosystem with distributed storage (HDFS).
        \item The combination of Map and Reduce functions is powerful for large dataset analysis.
        \item Understand the flow from Map to Shuffle to Reduce for efficient application development.
    \end{itemize}
\end{frame}

\begin{frame}[fragile]
    \frametitle{Example Code Snippet (Pseudocode)}
    \begin{lstlisting}[language=Python]
# Example of a simple Map function in Python
def mapper(line):
    for word in line.split():
        emit(word, 1)

# Example of a simple Reduce function in Python
def reducer(word, counts):
    total_count = sum(counts)
    emit(word, total_count)
    \end{lstlisting}
\end{frame}

\begin{frame}[fragile]
    \frametitle{Conclusion}
    The MapReduce paradigm is foundational in big data processing. 
    It streamlines computation for large datasets, ensuring tasks are efficient, reliable, and scalable. 
    Understanding this model is crucial for harnessing Hadoop's full capabilities.
\end{frame}

\begin{frame}[fragile]
    \frametitle{Components of the Hadoop Ecosystem - Overview}
    \begin{block}{Overview of the Hadoop Ecosystem}
        The Hadoop ecosystem is a collection of various tools and frameworks that allow for the efficient management and analysis of vast amounts of data. Each component serves a unique purpose, working together to provide a robust data processing solution.
    \end{block}
\end{frame}

\begin{frame}[fragile]
    \frametitle{Components of the Hadoop Ecosystem - Key Components}
    \begin{enumerate}
        \item \textbf{Hadoop Distributed File System (HDFS)}:
            \begin{itemize}
                \item \textbf{Description}: A scalable file system that stores data across multiple machines.
                \item \textbf{Key Functions}:
                    \begin{itemize}
                        \item Fault tolerance: Data is automatically replicated across different nodes.
                        \item High throughput: Designed for applications with large data sets.
                    \end{itemize}
            \end{itemize}
        
        \item \textbf{MapReduce}:
            \begin{itemize}
                \item \textbf{Description}: A programming model for processing large data sets with a distributed algorithm.
                \item \textbf{Key Functions}:
                    \begin{itemize}
                        \item \textbf{Map Phase}: Processes input data into key-value pairs.
                        \item \textbf{Reduce Phase}: Merges and aggregates results from the map phase.
                    \end{itemize}
                \item \textbf{Example}: Counting occurrences of each word in a large document.
            \end{itemize}
    \end{enumerate}
\end{frame}

\begin{frame}[fragile]
    \frametitle{Components of the Hadoop Ecosystem - Continued}
    \begin{enumerate}[resume]
        \item \textbf{Apache Pig}:
            \begin{itemize}
                \item \textbf{Description}: A high-level platform for creating programs that run on Hadoop, using a script-like language called Pig Latin.
                \item \textbf{Key Functions}:
                    \begin{itemize}
                        \item Simplifies complex MapReduce tasks.
                    \end{itemize}
                \item \textbf{Example}: Writing a Pig script to filter data from large datasets with minimal effort.
            \end{itemize}
        
        \item \textbf{Apache Hive}:
            \begin{itemize}
                \item \textbf{Description}: A data warehouse infrastructure built on Hadoop for providing data summarization, query, and analysis.
                \item \textbf{Key Functions}:
                    \begin{itemize}
                        \item SQL-like querying: Use HiveQL to run queries on data.
                    \end{itemize}
                \item \textbf{Example}: Querying data to compute total sales over a period using simple SQL-like syntax.
            \end{itemize}
        
        \item \textbf{HBase}:
            \begin{itemize}
                \item \textbf{Description}: A NoSQL database that runs on top of HDFS, suitable for real-time read/write access to big data.
                \item \textbf{Key Functions}:
                    \begin{itemize}
                        \item Column-oriented storage: Efficient for querying large amounts of sparse data.
                    \end{itemize}
                \item \textbf{Example}: Storing user profiles in a scalable way, where each profile can contain varied attributes.
            \end{itemize}
    \end{enumerate}
\end{frame}

\begin{frame}[fragile]
    \frametitle{Summary and Conclusion}
    \begin{block}{Summary of Key Points}
        \begin{itemize}
            \item The Hadoop ecosystem provides essential tools for managing and processing big data efficiently.
            \item Each component works together to handle diverse data processing needs.
            \item Understanding each component helps leverage Hadoop’s full potential for analytics.
        \end{itemize}
    \end{block}
    
    \begin{block}{Conclusion}
        By familiarizing yourself with these components, you will be better equipped to utilize the Hadoop ecosystem effectively for data processing challenges. 
    \end{block}
\end{frame}

\begin{frame}
    \frametitle{Apache Pig}
    \begin{block}{Understanding Apache Pig}
        Apache Pig is a high-level platform designed for creating programs that run on Apache Hadoop. It simplifies the process of analyzing large data sets, making it accessible to non-programmers.
    \end{block}
    \begin{block}{Key Features}
        \begin{itemize}
            \item \textbf{Pig Latin}: A user-friendly scripting language for data transformations, easier than Java MapReduce.
            \item \textbf{UDFs}: User Defined Functions allow customization and extension.
        \end{itemize}
    \end{block}
\end{frame}

\begin{frame}[fragile]
    \frametitle{How Does Apache Pig Work?}
    \begin{enumerate}
        \item \textbf{Execution Framework}: Pig scripts are compiled into Map-Reduce jobs executed on the Hadoop cluster.
        \item \textbf{UDFs (User Defined Functions)}: Enhance functionality with custom functions in Java or other languages.
    \end{enumerate}
\end{frame}

\begin{frame}[fragile]
    \frametitle{Example of Pig Latin}
    \begin{lstlisting}
        -- Load data from a file
        data = LOAD 'mydata.txt' USING PigStorage(',') AS (name:chararray, age:int);
        
        -- Filter records where age is greater than 25
        filtered_data = FILTER data BY age > 25;
        
        -- Group the filtered data by name
        grouped_data = GROUP filtered_data BY name;
        
        -- Count the number of records for each group
        result = FOREACH grouped_data GENERATE group, COUNT(filtered_data);
        
        -- Store the result
        STORE result INTO 'output.txt' USING PigStorage(',');
    \end{lstlisting}
    
    \begin{block}{Code Explanation}
        \begin{itemize}
            \item \texttt{LOAD}: Imports data.
            \item \texttt{FILTER}, \texttt{GROUP}, \texttt{FOREACH}: Manipulate the data.
            \item \texttt{STORE}: Saves the output to a specified location.
        \end{itemize}
    \end{block}
\end{frame}

\begin{frame}[fragile]
    \frametitle{Apache Hive - Overview}
    \begin{block}{Overview of Apache Hive}
        Apache Hive is a data warehousing tool built on top of Hadoop that provides an SQL-like interface to query and manage large datasets stored in Hadoop's HDFS (Hadoop Distributed File System). 
        \begin{itemize}
            \item Facilitates ETL processes
            \item Allows data analysts to perform queries without complex MapReduce programming
        \end{itemize}
    \end{block}
\end{frame}

\begin{frame}[fragile]
    \frametitle{Apache Hive - Key Features}
    \begin{block}{Key Features of Apache Hive}
        \begin{enumerate}
            \item \textbf{SQL-Like Query Language:} Uses HiveQL (HQL), familiar to SQL users.
            \item \textbf{Schema-on-Read:} Applies schema while reading data, allowing diverse data types.
            \item \textbf{High Scalability:} Can process petabytes of data using Hadoop's distributed computing power.
            \item \textbf{Extensibility:} Supports user-defined functions (UDFs) to enhance capabilities.
        \end{enumerate}
    \end{block}
\end{frame}

\begin{frame}[fragile]
    \frametitle{Apache Hive - Example Usage}
    \begin{block}{Example Usage}
        Consider a retail company that logs customer transactions in HDFS. An analyst can run the following HQL query to summarize total sales per product:
        \begin{lstlisting}
SELECT product_id, SUM(sale_amount) AS total_sales
FROM transactions
GROUP BY product_id;
        \end{lstlisting}
        This query extracts insights without needing low-level MapReduce coding.
    \end{block}
\end{frame}

\begin{frame}[fragile]
    \frametitle{Apache Hive - Internals and Key Points}
    \begin{block}{Internals of Hive}
        \begin{itemize}
            \item \textbf{Metastore:} Stores metadata about tables such as schema and HDFS locations.
            \item \textbf{Execution Engine:} Translates HiveQL into MapReduce jobs for efficient processing.
        \end{itemize}
    \end{block}
    
    \begin{block}{Key Points to Emphasize}
        \begin{itemize}
            \item \textbf{Accessibility:} Simplifies data access for non-programmers.
            \item \textbf{Performance Optimization:} Compiled queries are optimized by the execution engine.
            \item \textbf{Batch Processing:} Ideal for high-latency operations rather than real-time querying.
        \end{itemize}
    \end{block}
\end{frame}

\begin{frame}[fragile]
    \frametitle{Apache Hive - Conclusion}
    \begin{block}{Conclusion}
        Apache Hive is a powerful tool within the Hadoop ecosystem that simplifies interaction with large datasets through a SQL-like interface. It empowers organizations to unlock valuable insights from Big Data.
        \begin{itemize}
            \item Familiarizing with Hive enables efficient querying and data manipulation.
        \end{itemize}
    \end{block}
\end{frame}

\begin{frame}[fragile]
    \frametitle{Apache HBase - Overview}
    \begin{block}{Overview of HBase}
        Apache HBase is a scalable and distributed NoSQL database modeled after Google's Bigtable. 
        Built on top of HDFS, it provides real-time read/write access to large amounts of structured and semi-structured data.
        HBase is particularly useful for applications requiring high-speed transactions, 
        allowing both random access to data and batch processing.
    \end{block}
\end{frame}

\begin{frame}[fragile]
    \frametitle{Apache HBase - Key Features}
    \begin{itemize}
        \item \textbf{Real-time Data Access:}
        \begin{itemize}
            \item Designed to handle real-time queries.
            \item Example applications include e-commerce platforms tracking user activity and social media analytics.
        \end{itemize}
        
        \item \textbf{Scalable Architecture:}
        \begin{itemize}
            \item Can scale horizontally by adding more nodes to the cluster.
            \item Efficiently handles billions of rows and millions of columns.
        \end{itemize}
        
        \item \textbf{Data Storage:}
        \begin{itemize}
            \item Data stored in tables with flexible structures.
            \item Organized into column families to group related data.
        \end{itemize}
        
        \item \textbf{High Availability:}
        \begin{itemize}
            \item Built-in support for data replication ensures availability during hardware failures.
        \end{itemize}
    \end{itemize}
\end{frame}

\begin{frame}[fragile]
    \frametitle{Apache HBase - Core Concepts}
    \begin{itemize}
        \item \textbf{Column Families:}
        \begin{itemize}
            \item Tables consist of rows and column families.
            \item Column families group related data for performance.
        \end{itemize}
        
        \item \textbf{Regions:}
        \begin{itemize}
            \item Tables split into regions hosted by RegionServers allow distributed management.
        \end{itemize}
        
        \item \textbf{Row Key:}
        \begin{itemize}
            \item Unique identifier for each row, crucial for performance.
        \end{itemize}
    \end{itemize}
\end{frame}

\begin{frame}[fragile]
    \frametitle{Apache HBase - Example Use Case}
    \begin{block}{Use Case: Weather Data}
        A company gathering real-time sensor data from weather stations employs HBase for storage:
        \begin{itemize}
            \item \textbf{Schema Design:}
            \begin{itemize}
                \item Table: \texttt{WeatherData}
                \item Column Families: \texttt{Temperature}, \texttt{Humidity}, \texttt{WindSpeed}
            \end{itemize}
            
            \item \textbf{Row Example:}
            \begin{itemize}
                \item Row Key: \texttt{StationID123}
                \item Columns:
                \begin{itemize}
                    \item Temperature: 72°F
                    \item Humidity: 50\%
                    \item WindSpeed: 10 mph
                \end{itemize}
            \end{itemize}
        \end{itemize}
        This allows for rapid retrieval and analysis of weather data.
    \end{block}
\end{frame}

\begin{frame}[fragile]
    \frametitle{Apache HBase - Summary and Next Steps}
    \begin{itemize}
        \item HBase is ideal for applications requiring rapid data access.
        \item Its architecture supports scalable storage and high availability.
        \item Utilizing column families and efficient row keys can boost performance.
    \end{itemize}
    
    \begin{block}{Next Steps}
        Explore Apache Spark to learn how it complements HBase by providing advanced data processing capabilities, 
        such as in-memory computing for faster analytics.
    \end{block}
\end{frame}

\begin{frame}[fragile]
    \frametitle{Apache Spark - Overview}
    % Content goes here
    \begin{block}{Introduction to Apache Spark}
        Apache Spark is an open-source, fast, and general-purpose cluster computing system developed at UC Berkeley's AMPLab in 2009. 
        It significantly enhances the speed and efficiency of big data processing tasks across distributed computing environments.
    \end{block}
\end{frame}

\begin{frame}[fragile]
    \frametitle{Apache Spark - Key Features}
    % Content goes here
    \begin{enumerate}
        \item \textbf{Speed:}
        \begin{itemize}
            \item Up to 100 times faster than traditional MapReduce in memory.
            \item Up to 10 times faster when running on disk.
            \item In-memory processing reduces I/O operations.
        \end{itemize}
        
        \item \textbf{Ease of Use:}
        \begin{itemize}
            \item High-level APIs in Java, Scala, Python, and R.
            \item Spark shell for interactive data analysis.
        \end{itemize}
        
        \item \textbf{Unified Engine:}
        \begin{itemize}
            \item Supports batch processing, stream processing, machine learning, and graph processing.
        \end{itemize}
        
        \item \textbf{Extensibility:}
        \begin{itemize}
            \item Runs on various cluster managers (YARN, Mesos, Kubernetes).
            \item Access to diverse storage systems (HDFS, Apache Cassandra, S3).
        \end{itemize}
    \end{enumerate}
\end{frame}

\begin{frame}[fragile]
    \frametitle{Apache Spark - Ecosystem Components}
    % Content goes here
    \begin{itemize}
        \item \textbf{Spark Core:} The foundation providing task scheduling, memory management, and fault recovery.
        
        \item \textbf{Spark SQL:} Module for processing structured data with SQL queries.
        
        \item \textbf{Spark Streaming:} Enables processing of real-time data streams.
        
        \item \textbf{MLlib:} Machine learning library for classification, regression, clustering, etc.
        
        \item \textbf{GraphX:} Library for graph processing and analysis of graph data.
    \end{itemize}
\end{frame}

\begin{frame}[fragile]
    \frametitle{Apache Spark - Example}
    % Content goes here
    \begin{block}{Word Count Example}
        Here is a popular Spark application example — the Word Count program.
        
        \begin{lstlisting}[language=Python]
from pyspark import SparkContext

# Initialize SparkContext
sc = SparkContext("local", "Word Count")

# Load text file
text_file = sc.textFile("input.txt")

# Count words
word_count = text_file.flatMap(lambda line: line.split()) \
                       .map(lambda word: (word, 1)) \
                       .reduceByKey(lambda a, b: a + b)

# Collect and print results
for word, count in word_count.collect():
    print(f"{word}: {count}")
        \end{lstlisting}
    \end{block}
    
    \begin{itemize}
        \item \textbf{flatMap:} Splits each line into words.
        \item \textbf{map:} Creates a tuple (word, 1) for each word.
        \item \textbf{reduceByKey:} Aggregates counts for each word.
    \end{itemize}
\end{frame}

\begin{frame}[fragile]
    \frametitle{Apache Spark - Key Points}
    % Content goes here
    \begin{itemize}
        \item Designed for fast data processing with low latency.
        \item Versatile framework for various data processing tasks.
        \item Strong community support contributing to its robustness and flexibility.
    \end{itemize}
\end{frame}

\begin{frame}[fragile]
    \frametitle{Next Topic}
    % Content goes here
    \begin{block}{Transition to Apache Kafka}
        This introduction to Apache Spark sets the stage for understanding real-time data processing capabilities, as explored in the next slide on frameworks like Apache Kafka.
    \end{block}
\end{frame}

\begin{frame}[fragile]
    \frametitle{Real-Time Data Processing - Introduction}
    \begin{block}{Introduction to Real-Time Data Processing}
        - Real-time data processing refers to the ability to ingest, analyze, and act on data instantly or with minimal delay.
        - Essential in applications such as:
        \begin{itemize}
            \item Fraud detection
            \item Social media monitoring
            \item IoT device management
        \end{itemize}
    \end{block}
\end{frame}

\begin{frame}[fragile]
    \frametitle{Real-Time Data Processing - Hadoop Ecosystem}
    \begin{block}{Importance in the Hadoop Ecosystem}
        - Traditionally known for batch processing, Hadoop supports real-time data processing via:
        \begin{itemize}
            \item Apache Kafka
            \item Apache Spark Streaming
        \end{itemize}
        - Utilizes Hadoop's storage system (HDFS) for efficient processing of large-scale, streaming data.
    \end{block}
\end{frame}

\begin{frame}[fragile]
    \frametitle{Apache Kafka - Core Component}
    \begin{block}{What is Apache Kafka?}
        - A distributed event streaming platform capable of handling trillions of events per day.
        - Designed for high-throughput, fault-tolerance, and scalability.
    \end{block}
    \begin{block}{Key Features of Kafka}
        \begin{itemize}
            \item \textbf{Publish-Subscribe Model}: Producers write data to topics, consumers read from topics.
            \item \textbf{Scalability}: Horizontally scales by adding more brokers.
            \item \textbf{Durability}: Data replication across multiple nodes ensures no data loss.
        \end{itemize}
    \end{block}
\end{frame}

\begin{frame}[fragile]
    \frametitle{Apache Kafka - Integration with Hadoop}
    \begin{block}{How Kafka Fits in the Hadoop Ecosystem}
        - Acts as a buffer for incoming data, processed by Spark Streaming or other Hadoop technologies.
    \end{block}
\end{frame}

\begin{frame}[fragile]
    \frametitle{Real-Time Data Processing - Use Case}
    \begin{block}{Example Use Case: Fraud Detection}
        - \textbf{Scenario}: Monitor transactions in real-time to detect fraud.
        \begin{enumerate}
            \item \textbf{Data Ingestion}: Transactions streamed into Kafka topics.
            \item \textbf{Stream Processing}: Spark Streaming consumes messages, applies ML models, triggers alerts.
            \item \textbf{Storage}: Valid transactions stored in HDFS or NoSQL for further analysis if needed.
        \end{enumerate}
    \end{block}
\end{frame}

\begin{frame}[fragile]
    \frametitle{Real-Time Data Processing - Key Points}
    \begin{block}{Key Points to Emphasize}
        - \textbf{Complementary Frameworks}: Kafka enhances both real-time and batch processing in Hadoop.
        - \textbf{Speed and Efficiency}: Kafka and Spark provide rapid data analysis with Hadoop's robustness.
        - \textbf{Real-World Applications}: Utilized in finance, retail, telecommunications to improve efficiency and user experience.
    \end{block}
\end{frame}

\begin{frame}[fragile]
    \frametitle{Real-Time Data Processing - Conclusion}
    \begin{block}{Conclusion}
        - Real-time data processing is critical for timely decision-making.
        - The integration of Apache Kafka with Hadoop enables effective streaming data utilization.
    \end{block}
\end{frame}

\begin{frame}[fragile]
    \frametitle{Real-Time Data Processing - Code Snippet}
    \begin{lstlisting}[language=Python]
from pyspark.sql import SparkSession

# Initialize Spark Session
spark = SparkSession.builder \
    .appName("KafkaSparkStreaming") \
    .getOrCreate()

# Read Data from Kafka
kafkaStreamDF = spark \
    .readStream \
    .format("kafka") \
    .option("kafka.bootstrap.servers", "localhost:9092") \
    .option("subscribe", "transaction_topic") \
    .load()

# Process the stream (e.g., filtering fraud)
processedStreamDF = kafkaStreamDF.filter("is_fraud = true")

# Start Query to Write Back to Kafka
query = processedStreamDF.writeStream \
    .outputMode("append") \
    .format("kafka") \
    .option("kafka.bootstrap.servers", "localhost:9092") \
    .option("topic", "alert_topic") \
    .start()
    \end{lstlisting}
\end{frame}

\begin{frame}
    \frametitle{Integrating Hadoop and Spark}
    Discussion on the integration of Hadoop and Spark to enhance data processing and analytics.
\end{frame}

\begin{frame}
    \frametitle{1. Understanding Hadoop and Spark}
    \begin{itemize}
        \item \textbf{Hadoop}:
        \begin{itemize}
            \item An open-source framework for distributed storage and processing of large data sets using the MapReduce programming model.
            \item Components include HDFS (Hadoop Distributed File System) for storage and YARN (Yet Another Resource Negotiator) for resource management.
        \end{itemize}
        \item \textbf{Spark}:
        \begin{itemize}
            \item A fast and general-purpose cluster-computing system, offering an interface for programming entire clusters with implicit data parallelism and fault tolerance.
            \item Supports in-memory computation, making it significantly faster for certain workloads compared to traditional disk-based processing.
        \end{itemize}
    \end{itemize}
\end{frame}

\begin{frame}
    \frametitle{2. Why Integrate Hadoop and Spark?}
    \begin{itemize}
        \item \textbf{Enhanced Performance}: 
        \begin{itemize}
            \item Spark uses in-memory processing which allows quicker data access compared to the disk-based model of MapReduce.
        \end{itemize}
        \item \textbf{Diverse Workloads}:
        \begin{itemize}
            \item Suitable for a wide range of applications, from batch processing (Hadoop) to interactive queries and real-time analytics (Spark).
        \end{itemize}
        \item \textbf{Complement Business Needs}:
        \begin{itemize}
            \item Serve both historical data processing (Hadoop) and real-time insights (Spark) tailored to business analytics requirements.
        \end{itemize}
    \end{itemize}
\end{frame}

\begin{frame}[fragile]
    \frametitle{3. Integration Points}
    \begin{itemize}
        \item \textbf{Data Storage}:
        \begin{itemize}
            \item Spark can read data directly from HDFS, allowing for processing of large data sets stored in Hadoop.
        \end{itemize}
        \item \textbf{Job Scheduling}:
        \begin{itemize}
            \item Spark can run on top of YARN, enabling resource sharing and job scheduling across the Hadoop ecosystem.
        \end{itemize}
    \end{itemize}
    
    \textbf{Example}:
    \begin{lstlisting}[language=Python]
from pyspark import SparkContext

# Initialize a SparkContext
sc = SparkContext(master="yarn", appName="Hadoop-Spark Integration")

# Load data from HDFS
data = sc.textFile("hdfs:///user/hadoop/data.txt")

# Perform transformations
result = data.flatMap(lambda line: line.split(" ")).map(lambda word: (word, 1)).reduceByKey(lambda a, b: a + b)

# Save output back to HDFS
result.saveAsTextFile("hdfs:///user/hadoop/output/")
    \end{lstlisting}
\end{frame}

\begin{frame}
    \frametitle{4. Key Benefits of Integration}
    \begin{itemize}
        \item \textbf{Flexibility}: Choose the right processing tool based on specific use cases.
        \item \textbf{Scalability}: Easily scale to process petabytes of data.
        \item \textbf{Streamlined Analytics}: Combine batch and real-time data analysis seamlessly.
    \end{itemize}
\end{frame}

\begin{frame}
    \frametitle{5. Summary}
    The integration of Hadoop and Spark provides a powerful and efficient means of handling diverse data processing needs, allowing businesses to leverage the strengths of both frameworks for better analytics and insights.
    
    \textbf{Remember:} The combination of Hadoop’s vast storage capabilities with Spark’s swift processing provides a robust environment for large-scale data analytics.
\end{frame}

\begin{frame}[fragile]
    \frametitle{Machine Learning in the Hadoop Ecosystem}
    \begin{block}{Overview}
        Overview of how machine learning can be implemented in the Hadoop ecosystem using tools like MLlib in Spark.
    \end{block}
\end{frame}

\begin{frame}[fragile]
    \frametitle{Understanding Machine Learning in Hadoop}
    \begin{itemize}
        \item Machine Learning (ML) benefits from Hadoop's massive data processing capabilities.
        \item Apache Spark and its MLlib library simplify implementation of scalable and efficient ML algorithms.
    \end{itemize}
\end{frame}

\begin{frame}[fragile]
    \frametitle{Key Concepts}
    \begin{itemize}
        \item \textbf{Hadoop Ecosystem}: Framework for distributed storage and processing of large datasets.
            \begin{itemize}
                \item Tools: HDFS (Hadoop Distributed File System), YARN (Yet Another Resource Negotiator).
            \end{itemize}
        \item \textbf{Apache Spark}: In-memory processing makes it faster than Hadoop MapReduce. 
            \begin{itemize}
                \item MLlib: Scalable machine learning library facilitating ML pipeline construction.
            \end{itemize}
    \end{itemize}
\end{frame}

\begin{frame}[fragile]
    \frametitle{Implementing Machine Learning with MLlib}
    \begin{itemize}
        \item \textbf{Data Handling}: Use Hadoop's HDFS for storing large datasets.
        \item \textbf{Transformations and Actions}: MLlib uses RDDs (Resilient Distributed Datasets) and DataFrames for data preparation.
        \item \textbf{Algorithm Usage}: Available algorithms in MLlib include:
            \begin{itemize}
                \item Classification: Decision Trees, Logistic Regression
                \item Regression: Linear Regression
                \item Clustering: K-means
                \item Collaborative Filtering: Alternating Least Squares
            \end{itemize}
    \end{itemize}
\end{frame}

\begin{frame}[fragile]
    \frametitle{Example: Logistic Regression}
    Here's how logistic regression can be implemented using MLlib:
    \begin{lstlisting}[language=Python]
from pyspark.sql import SparkSession
from pyspark.ml.classification import LogisticRegression

# Create Spark session
spark = SparkSession.builder.appName("MLlibExample").getOrCreate()

# Load data
data = spark.read.format("libsvm").load("data/mllib/sample_libsvm_data.txt")

# Create a Logistic Regression model
lr = LogisticRegression(maxIter=10, regParam=0.01)

# Fit the model
model = lr.fit(data)

# Print the coefficients and intercept
print("Coefficients: " + str(model.coefficients))
print("Intercept: " + str(model.intercept))
    \end{lstlisting}
\end{frame}

\begin{frame}[fragile]
    \frametitle{Advantages of Using MLlib in Hadoop}
    \begin{itemize}
        \item \textbf{Scalability}: Easily handles large datasets via distributed computations.
        \item \textbf{Performance}: In-memory processing increases the speed of ML algorithms.
        \item \textbf{Integration}: Works seamlessly with other Hadoop tools (HDFS, Hive).
    \end{itemize}
\end{frame}

\begin{frame}[fragile]
    \frametitle{Key Points to Remember}
    \begin{itemize}
        \item Combination of Hadoop and Spark enables advanced ML implementations.
        \item MLlib offers a diverse suite of algorithms for various tasks.
        \item The synergy enhances data processing speed and efficiency.
    \end{itemize}
\end{frame}

\begin{frame}
    \frametitle{Performance and Scalability}
    \begin{block}{Understanding Performance in the Hadoop Ecosystem}
        \begin{itemize}
            \item Performance: Efficiency of data processing tasks across clusters.
            \item Influencing Factors:
                \begin{itemize}
                    \item Data locality
                    \item CPU and memory usage
                    \item Network bandwidth
                    \item Algorithm efficiencies
                \end{itemize}
        \end{itemize}
    \end{block}
\end{frame}

\begin{frame}
    \frametitle{Key Performance Tools}
    \begin{block}{Hadoop vs. Apache Spark}
        \begin{itemize}
            \item \textbf{Hadoop MapReduce}:
                \begin{itemize}
                    \item Core processing engine
                    \item Efficient for batch processing, less so for iterative algorithms
                \end{itemize}
            \item \textbf{Apache Spark}:
                \begin{itemize}
                    \item In-memory processing
                    \item Up to 100 times faster than Hadoop MapReduce for certain tasks
                \end{itemize}
        \end{itemize}
    \end{block}
\end{frame}

\begin{frame}
    \frametitle{Comparative Analysis of Tools}
    \begin{block}{Latency and Ease of Use}
        \begin{itemize}
            \item \textbf{Latency}:
                \begin{itemize}
                    \item Hadoop: High latency due to disk I/O
                    \item Spark: Low latency due to in-memory computation
                \end{itemize}
            \item \textbf{Ease of Use}:
                \begin{itemize}
                    \item Hadoop: Requires detailed coding in Java
                    \item Spark: Offers APIs in Java, Scala, Python, and R
                \end{itemize}
        \end{itemize}
        \begin{block}{Example}
            A word count operation could take minutes on Hadoop MapReduce while taking mere seconds on Spark.
        \end{block}
    \end{block}
\end{frame}

\begin{frame}
    \frametitle{Techniques for Enhancing Scalability}
    \begin{itemize}
        \item \textbf{Data Partitioning}:
            \begin{itemize}
                \item Distributing data across multiple nodes improves processing speed
            \end{itemize}
        \item \textbf{Horizontal Scaling}:
            \begin{itemize}
                \item Adding more nodes allows for linear growth in processing capacity
                \item Example: Doubling nodes can double processing power
            \end{itemize}
        \item \textbf{Resource Management with YARN}:
            \begin{itemize}
                \item YARN manages resources across the Hadoop cluster
                \item Allows applications to scale efficiently based on demand
            \end{itemize}
    \end{itemize}
\end{frame}

\begin{frame}[fragile]
    \frametitle{Conclusion and Code Example}
    \begin{block}{Key Points to Emphasize}
        \begin{itemize}
            \item Monitor performance metrics (job execution time, resource utilization, network latency).
            \item Choose the right tool based on task requirements, data size, and execution time sensitivity.
        \end{itemize}
    \end{block}

    \begin{block}{Code Snippet Example (Simple Spark Word Count)}
        \begin{lstlisting}[language=Python]
from pyspark import SparkContext

sc = SparkContext("local", "Word Count")
text_file = sc.textFile("hdfs:///path/to/file.txt")
word_counts = text_file.flatMap(lambda line: line.split(" ")) \
                       .map(lambda word: (word, 1)) \
                       .reduceByKey(lambda a, b: a + b)

word_counts.saveAsTextFile("hdfs:///path/to/output")
        \end{lstlisting}
    \end{block}
\end{frame}

\begin{frame}[fragile]
    \frametitle{Conclusion and Q\&A - Summary of Key Points}
    \begin{block}{Understanding the Hadoop Ecosystem}
        The Hadoop ecosystem consists of diverse tools and frameworks designed for managing and processing vast amounts of data. Understanding these components is crucial for leveraging their capabilities effectively.
    \end{block}

    \begin{enumerate}
        \item \textbf{Core Components of Hadoop:}
        \begin{itemize}
            \item \textbf{Hadoop Distributed File System (HDFS):}
            \begin{itemize}
                \item A scalable and fault-tolerant file system that stores data across multiple machines.
                \item Example: HDFS can handle large datasets, such as logs from a web application, facilitating efficient data retrieval.
            \end{itemize}

            \item \textbf{MapReduce:}
            \begin{itemize}
                \item A programming model for processing large data sets with a parallel, distributed algorithm.
                \item Example: Word count program where the input is a text file, and the output is the count of occurrences of each word.
            \end{itemize}
        \end{itemize}
    \end{enumerate}
\end{frame}

\begin{frame}[fragile]
    \frametitle{Conclusion and Q\&A - Higher-Level Tools and Performance}
    \begin{block}{Higher-Level Tools}
        \begin{itemize}
            \item \textbf{Pig:} A high-level platform for creating programs that run on Hadoop. Pig's scripting language, Pig Latin, simplifies data manipulation.
            \item \textbf{Hive:} A data warehousing solution for Hadoop, facilitating SQL-like queries for analyzing large datasets.
        \end{itemize}
    \end{block}

    \begin{block}{Performance and Scalability}
        Techniques for enhancing Hadoop application performance include:
        \begin{itemize}
            \item Data locality optimization
            \item Using optimized file formats (like Parquet)
            \item Choosing the right parallel execution strategies
        \end{itemize}
    \end{block}
\end{frame}

\begin{frame}[fragile]
    \frametitle{Conclusion and Q\&A - Ecosystem Integration and Discussion}
    \begin{block}{Ecosystem Integration}
        The interaction between Hadoop and other tools, such as Spark for real-time processing and HBase for NoSQL storage, has greatly enhanced its data processing capabilities.
    \end{block}

    \begin{block}{Interactive Q\&A Session}
        \textbf{Prompt for Students:}
        \begin{itemize}
            \item What areas of the Hadoop ecosystem would you like to explore further?
            \item Do you have specific challenges you've encountered or anticipate in handling big data with Hadoop?
        \end{itemize}

        \textbf{Encourage Discussion:}
        \begin{itemize}
            \item Share your thoughts or questions about integrating Apache Spark into your current data pipelines.
            \item Discuss experiences with using Hive or Pig for data querying and transformation.
        \end{itemize}
    \end{block}
\end{frame}


\end{document}