\documentclass[aspectratio=169]{beamer}

% Theme and Color Setup
\usetheme{Madrid}
\usecolortheme{whale}
\useinnertheme{rectangles}
\useoutertheme{miniframes}

% Additional Packages
\usepackage[utf8]{inputenc}
\usepackage[T1]{fontenc}
\usepackage{graphicx}
\usepackage{booktabs}
\usepackage{listings}
\usepackage{amsmath}
\usepackage{amssymb}
\usepackage{xcolor}
\usepackage{tikz}
\usepackage{pgfplots}
\pgfplotsset{compat=1.18}
\usetikzlibrary{positioning}
\usepackage{hyperref}

% Custom Colors
\definecolor{myblue}{RGB}{31, 73, 125}
\definecolor{mygray}{RGB}{100, 100, 100}
\definecolor{mygreen}{RGB}{0, 128, 0}
\definecolor{myorange}{RGB}{230, 126, 34}
\definecolor{mycodebackground}{RGB}{245, 245, 245}

% Set Theme Colors
\setbeamercolor{structure}{fg=myblue}
\setbeamercolor{frametitle}{fg=white, bg=myblue}
\setbeamercolor{title}{fg=myblue}
\setbeamercolor{section in toc}{fg=myblue}
\setbeamercolor{item projected}{fg=white, bg=myblue}
\setbeamercolor{block title}{bg=myblue!20, fg=myblue}
\setbeamercolor{block body}{bg=myblue!10}
\setbeamercolor{alerted text}{fg=myorange}

% Set Fonts
\setbeamerfont{title}{size=\Large, series=\bfseries}
\setbeamerfont{frametitle}{size=\large, series=\bfseries}
\setbeamerfont{caption}{size=\small}
\setbeamerfont{footnote}{size=\tiny}

% Title Page Information
\title[Final Project Presentations]{Week 13: Final Project Presentations}
\author[J. Smith]{John Smith, Ph.D.}
\institute[University Name]{
  Department of Computer Science\\
  University Name\\
  \vspace{0.3cm}
  Email: email@university.edu\\
  Website: www.university.edu
}
\date{\today}

% Document Start
\begin{document}

\frame{\titlepage}

\begin{frame}[fragile]
  \titlepage
\end{frame}

\begin{frame}[fragile]
  \frametitle{Introduction to Final Project Presentations}
  \begin{block}{Overview of the Final Project}
    The final project serves as a capstone experience in the Data Mining course, integrating and applying the skills learned throughout the semester.
  \end{block}
\end{frame}

\begin{frame}[fragile]
  \frametitle{Purpose and Significance of Data Mining}
  \begin{itemize}
    \item \textbf{Purpose}:
      \begin{itemize}
        \item Demonstrate proficiency in data mining techniques and algorithms.
        \item Extract meaningful insights from data.
      \end{itemize}
    \item \textbf{Significance}:
      \begin{itemize}
        \item Data Mining discovers patterns from large datasets, crucial in industries like:
          \begin{itemize}
            \item \textbf{Healthcare}: Predictive analytics for patient outcomes.
            \item \textbf{Finance}: Fraud detection via transaction analysis.
            \item \textbf{Marketing}: Customer segmentation for personalized campaigns.
          \end{itemize}
      \end{itemize}
  \end{itemize}
\end{frame}

\begin{frame}[fragile]
  \frametitle{Project Expectations and Presentation Outline}
  \begin{enumerate}
    \item \textbf{Expectations}:
      \begin{itemize}
        \item Originality: Unique dataset and research question.
        \item Methodology: Appropriate data mining techniques (e.g., classification, clustering).
        \item Presentation: 10-15 minute talk with visual aids.
      \end{itemize}
    \item \textbf{Presentation Outline}:
      \begin{itemize}
        \item Introduction to problem and dataset.
        \item Description of data mining techniques.
        \item Results and interpretations.
        \item Conclusions and future directions.
      \end{itemize}
  \end{enumerate}
\end{frame}

\begin{frame}[fragile]
  \frametitle{Closing Thoughts}
  \begin{block}{}
    The final project prepares students for the professional world, showcasing analytical skills and understanding the significance of data-driven decision-making.
  \end{block}
  Embrace this chance to contribute to the evolving field of data mining!
\end{frame}

\begin{frame}[fragile]{Objectives of the Final Project}
  The Final Project is a crucial part of your learning journey. It enables you to showcase your skills in data mining while achieving specific objectives. 
\end{frame}

\begin{frame}[fragile]{Overview of Objectives - Part 1}
  \begin{itemize}
    \item \textbf{Enhance Analytical Skills}

      \begin{itemize}
        \item \textit{Explanation:} Enhance your ability to interpret data, identify patterns, and draw logical conclusions. 
        \item \textit{Example:} Analyzing customer purchase data to identify which products are frequently bought together.
      \end{itemize}
      
    \item \textbf{Demonstrate Data Mining Techniques}
    
      \begin{itemize}
        \item \textit{Explanation:} Apply various data mining methods such as clustering, classification, regression, and association rules to solve real-world problems.
        \item \textit{Example:} Predicting sales using regression analysis to understand the relationship between advertising spend and sales figures.
      \end{itemize}
  \end{itemize}
\end{frame}

\begin{frame}[fragile]{Overview of Objectives - Part 2}
  \begin{itemize}
    \item \textbf{Collaborative Work}

      \begin{itemize}
        \item \textit{Explanation:} Show your capability to work effectively within a team to pool diverse skills toward a common goal.
        \item \textit{Example:} Assigning roles in a group project based on each member's strengths, such as data analysis and presentation design.
      \end{itemize}

    \item \textbf{Effective Communication of Findings}
    
      \begin{itemize}
        \item \textit{Explanation:} Convey complex ideas simply and effectively, utilizing data visualizations and clear summaries.
        \item \textit{Example:} Translating findings from jargon into simple language and using visuals to enhance understanding.
      \end{itemize}
  \end{itemize}
\end{frame}

\begin{frame}[fragile]{Key Points to Emphasize}
  \begin{itemize}
    \item \textbf{Real-World Application:} Skills from this project prepare you for practical challenges in applying data mining techniques in real scenarios.
    
    \item \textbf{Critical Thinking:} Engage in a critical evaluation of findings and their implications for decision-making.
    
    \item \textbf{Feedback Gathering:} Use presentations to gather feedback, which can provide new perspectives for improvements.
  \end{itemize}
\end{frame}

\begin{frame}[fragile]{Example Structure of the Final Project Presentation}
  \begin{enumerate}
    \item \textbf{Introduction:} Brief overview of the problem statement and objectives.
    \item \textbf{Methodology:} Explanation of the data mining techniques employed.
    \item \textbf{Analysis and Findings:} Detailed presentation of data analysis with relevant visualizations.
    \item \textbf{Conclusion and Recommendations:} Summary of insights and actionable recommendations based on findings.
  \end{enumerate}
\end{frame}

\begin{frame}[fragile]
    \frametitle{Project Presentation Format - Overview}
    \begin{block}{Overview}
        The format of your final project presentations is crucial for effectively communicating your findings. This slide outlines key elements including:
        \begin{itemize}
            \item Time limits
            \item Presentation structure
            \item Submission requirements
        \end{itemize}
    \end{block}
\end{frame}

\begin{frame}[fragile]
    \frametitle{Project Presentation Format - Time Limits}
    \begin{block}{1. Time Limits}
        \begin{itemize}
            \item \textbf{Total Duration}: Each presentation must be \textbf{10 minutes} long.
            \item \textbf{Q\&A Session}: Follow each presentation with a \textbf{5-minute} question and answer period.
            \item \textbf{Timing Enforcement}: A timer will be used; practice to stay within allotted time to respect your peers' schedules.
        \end{itemize}
    \end{block}
\end{frame}

\begin{frame}[fragile]
    \frametitle{Project Presentation Format - Structure}
    \begin{block}{2. Structure of Presentation}
        Follow this structured format to ensure clarity and engagement:
        \begin{itemize}
            \item \textbf{Introduction (1-2 minutes)}:
                \begin{itemize}
                    \item Briefly introduce your project topic.
                    \item State the objectives and significance of your work.
                \end{itemize}
            \item \textbf{Literature Review (2-3 minutes)}:
                \begin{itemize}
                    \item Summarize relevant background research.
                    \item Highlight the gap your project addresses.
                \end{itemize}
            \item \textbf{Methodology (2 minutes)}:
                \begin{itemize}
                    \item Explain the methods used clearly and concisely.
                    \item Include any tools or frameworks you applied.
                \end{itemize}
            \item \textbf{Findings (2-3 minutes)}:
                \begin{itemize}
                    \item Present key results; use visual aids to enhance understanding.
                    \item Discuss relevance of your findings.
                \end{itemize}
            \item \textbf{Conclusion (1 minute)}:
                \begin{itemize}
                    \item Recap main points.
                    \item Suggest implications or future directions.
                \end{itemize}
        \end{itemize}
    \end{block}
\end{frame}

\begin{frame}[fragile]
    \frametitle{Project Presentation Format - Submission Requirements}
    \begin{block}{3. Submission Requirements}
        \begin{itemize}
            \item \textbf{Presentation Slides}:
                \begin{itemize}
                    \item Submit a \textbf{PowerPoint or PDF} version of your slides.
                    \item \textbf{Deadline}: 24 hours before your scheduled presentation time.
                    \item \textbf{Format}: Ensure slides are clear and visually appealing (max 6 bullet points per slide).
                \end{itemize}
            \item \textbf{Backup}: Bring a backup copy of your presentation on a USB drive or cloud storage.
            \item \textbf{Peer Review}: Your peers will provide feedback. Include their input in your final submission for improvement.
        \end{itemize}
    \end{block}
\end{frame}

\begin{frame}[fragile]
    \frametitle{Key Points to Emphasize}
    \begin{itemize}
        \item Adhere to time limits for a well-structured presentation.
        \item Clearly communicate your project's significance, methods, and findings.
        \item Use visual aids to maintain audience interest and understanding.
        \item Submission must be timely and according to the specified format.
    \end{itemize}
    \begin{block}{Conclusion}
        By following these guidelines, you will be prepared to deliver an impactful and engaging presentation that meets the course expectations. Good luck!
    \end{block}
\end{frame}

\begin{frame}[fragile]
    \frametitle{Evaluation Criteria - Overview}
    \begin{itemize}
        \item Presentations are assessed on:
        \begin{itemize}
            \item Clarity (25 points)
            \item Content (35 points)
            \item Technical Execution (25 points)
            \item Teamwork (15 points)
        \end{itemize}
        \item Importance: Effectiveness of communication and audience perception.
    \end{itemize}
\end{frame}

\begin{frame}[fragile]
    \frametitle{Evaluation Criteria - Clarity and Content}
    \begin{block}{1. Clarity (25 points)}
        \begin{itemize}
            \item \textbf{Explanation}: Present ideas effectively for easy audience comprehension.
            \item \textbf{Key Aspects}:
            \begin{itemize}
                \item Clear language and terminology.
                \item Logical information flow.
                \item Engaging visual aids.
            \end{itemize}
            \item \textbf{Example}: Summarizing data with simple charts rather than jargon.
        \end{itemize}
    \end{block}
    
    \begin{block}{2. Content (35 points)}
        \begin{itemize}
            \item \textbf{Explanation}: Rich, relevant information demonstrating subject knowledge.
            \item \textbf{Key Aspects}:
            \begin{itemize}
                \item Comprehensive topic coverage.
                \item Inclusion of examples or case studies.
                \item Addressing counterpoints.
            \end{itemize}
            \item \textbf{Example}: Analyzing AI applications like ChatGPT and data mining's role.
        \end{itemize}
    \end{block}
\end{frame}

\begin{frame}[fragile]
    \frametitle{Evaluation Criteria - Technical Execution and Teamwork}
    \begin{block}{3. Technical Execution (25 points)}
        \begin{itemize}
            \item \textbf{Explanation}: Assesses audio-visual quality and format adherence.
            \item \textbf{Key Aspects}:
            \begin{itemize}
                \item Use of technology effectively.
                \item Time management.
                \item Professionalism in delivery.
            \end{itemize}
            \item \textbf{Example}: Using quality visuals and ensuring media works seamlessly.
        \end{itemize}
    \end{block}
    
    \begin{block}{4. Teamwork (15 points)}
        \begin{itemize}
            \item \textbf{Explanation}: Effective collaboration enhances presentation quality.
            \item \textbf{Key Aspects}:
            \begin{itemize}
                \item Clear role division.
                \item Cohesion in style and delivery.
                \item Evidence of joint preparation.
            \end{itemize}
            \item \textbf{Example}: Seamless transitions between group members' speaking points.
        \end{itemize}
    \end{block}
\end{frame}

\begin{frame}[fragile]
    \frametitle{Group Dynamics and Collaboration}
    \begin{block}{Importance of Collaboration}
        Collaboration is essential in group projects as it fosters an environment where team members can share their strengths and perspectives. Successful projects hinge on effective teamwork, enhancing creativity, improving problem-solving, and leading to better outcomes.
    \end{block}
\end{frame}

\begin{frame}[fragile]
    \frametitle{Key Concepts - Part 1}
    \begin{enumerate}
        \item \textbf{Synergy}: The whole is greater than the sum of its parts.
            \begin{itemize}
                \item \textit{Example}: In design projects, brainstorming can lead to innovative ideas not possible individually.
            \end{itemize}
        \item \textbf{Role Distribution}: Assigning roles based on individual strengths optimizes productivity.
            \begin{itemize}
                \item \textit{Example}: A member skilled in graphics can focus on design while others handle research.
            \end{itemize}
    \end{enumerate}
\end{frame}

\begin{frame}[fragile]
    \frametitle{Key Concepts - Part 2}
    \begin{enumerate}
        \setcounter{enumi}{2}
        \item \textbf{Open Communication}: Regular updates and feedback are crucial for alignment.
            \begin{itemize}
                \item \textit{Illustration}: Weekly check-ins clarify goals and track progress.
            \end{itemize}
    \end{enumerate}
    
    \begin{block}{Benefits of Effective Collaboration}
        \begin{itemize}
            \item \textbf{Diverse Perspectives}: Integrating various viewpoints enriches project outcomes.
                \begin{itemize}
                    \item \textit{Example}: Incorporating feedback from diverse team members can create comprehensive strategies.
                \end{itemize}
            \item \textbf{Enhanced Problem Solving}: Teams can collectively tackle complex problems.
            \item \textbf{Skill Development}: Exposure to new ideas facilitates personal growth.
        \end{itemize}
    \end{block}
\end{frame}

\begin{frame}[fragile]
    \frametitle{Key Takeaways and Conclusion}
    \begin{itemize}
        \item Understand team dynamics to improve collaboration.
        \item Leverage technology (e.g., Trello, Slack) for better communication and project management.
        \item Foster a positive team culture to motivate members.
    \end{itemize}
    
    \begin{block}{Conclusion}
        In summary, collaboration is a cornerstone of successful project completion. Embracing teamwork enables high-quality work, mutual learning, and more engaging presentations.
    \end{block}
    
    \begin{block}{Next Slide Preview}
        We will address common challenges in project presentations and share effective tips for overcoming them.
    \end{block}
\end{frame}

\begin{frame}[fragile]
  \frametitle{Common Challenges in Project Presentations}
  \begin{block}{Introduction}
    During any group project, challenges can arise that hinder effective presentation and successful completion. Recognizing these challenges and knowing how to address them is key to ensuring your project is not only completed on time but also presented in an engaging and coherent manner.
  \end{block}
\end{frame}

\begin{frame}[fragile]
  \frametitle{Common Challenges}
  \begin{enumerate}
    \item \textbf{Lack of Preparation}
      \begin{itemize}
        \item \textit{Explanation}: Insufficient practice can lead to confusion and disorganization during the presentation.
        \item \textit{Tip}: Schedule multiple practice sessions, assign roles, and simulate the presentation environment.
      \end{itemize}
    
    \item \textbf{Communication Breakdowns}
      \begin{itemize}
        \item \textit{Explanation}: Miscommunication can lead to conflicting messages or topics being overlooked.
        \item \textit{Tip}: Maintain clear communication channels and use collaborative tools.
      \end{itemize}
    
    \item \textbf{Disparity in Participation}
      \begin{itemize}
        \item \textit{Explanation}: Some members may contribute more, creating an imbalance.
        \item \textit{Tip}: Clearly define each member's role at the project's start.
      \end{itemize}
  \end{enumerate}
\end{frame}

\begin{frame}[fragile]
  \frametitle{Common Challenges (Continued)}
  \begin{enumerate}
    \setcounter{enumi}{3} % Continue to enumerate from where we left off
    \item \textbf{Technical Difficulties}
      \begin{itemize}
        \item \textit{Explanation}: Issues with equipment or software can disrupt the flow.
        \item \textit{Tip}: Test all technology in advance and prepare for potential malfunctions.
      \end{itemize}
    
    \item \textbf{Time Management Issues}
      \begin{itemize}
        \item \textit{Explanation}: Running out of time can affect the overall effectiveness.
        \item \textit{Tip}: Create an agenda with timed segments for each part of the presentation.
      \end{itemize}

    \item \textbf{Audience Engagement}
      \begin{itemize}
        \item \textit{Explanation}: Failing to keep the audience interested can lead to poor retention.
        \item \textit{Tip}: Use engaging visuals, interactive elements, and storytelling techniques.
      \end{itemize}
  \end{enumerate}
\end{frame}

\begin{frame}[fragile]
  \frametitle{Key Points to Emphasize}
  \begin{itemize}
    \item \textbf{Preparation is Crucial}: Spend adequate time practicing and refining your presentation together.
    \item \textbf{Clear Roles}: Everyone should know their part to contribute effectively and confidently.
    \item \textbf{Backup Plans}: Always have a contingency plan for technology-related issues.
    \item \textbf{Engagement Techniques}: Use visuals, ask questions, and interact with the audience.
  \end{itemize}
\end{frame}

\begin{frame}[fragile]
  \frametitle{Conclusion}
  By anticipating these common challenges and employing proactive strategies, your group can enhance not only the quality of your project presentation but ensure a more cohesive and successful project journey. Remember, teamwork and planning are essential components of a successful presentation!
\end{frame}

\begin{frame}[fragile]
    \frametitle{Ethical Considerations in Data Mining Projects - Introduction}
    \begin{itemize}
        \item \textbf{Definition:} Data mining involves extracting patterns from large datasets, raising significant ethical issues.
        \item \textbf{Importance:} Responsible data use is essential for maintaining trust and credibility among all stakeholders.
    \end{itemize}
\end{frame}

\begin{frame}[fragile]
    \frametitle{Ethical Considerations in Data Mining Projects - Ethical Implications}
    \begin{enumerate}
        \item \textbf{Privacy Concerns:}
            \begin{itemize}
                \item Access to sensitive personal information (e.g., health records).
                \item \textit{Example:} Improper handling of healthcare data risks breaching patient confidentiality.
            \end{itemize}
        \item \textbf{Informed Consent:}
            \begin{itemize}
                \item Individuals must be informed about data usage and give explicit consent.
                \item \textit{Illustration:} Social media platforms must disclose data usage for targeted advertising.
            \end{itemize}
    \end{enumerate}
\end{frame}

\begin{frame}[fragile]
    \frametitle{Ethical Considerations in Data Mining Projects - Integrity, Fairness, and Regulation}
    \begin{enumerate}
        \item \textbf{Maintaining Integrity:}
            \begin{itemize}
                \item \textbf{Transparency:} Data sources, methodologies, and findings must be open for scrutiny.
                \item \textbf{Accountability:} Organizations must be responsible for data misuse (e.g., Cambridge Analytica).
            \end{itemize}
        \item \textbf{Fairness and Bias:}
            \begin{itemize}
                \item Algorithms can inherit biases, affecting fairness.
                \item \textit{Example:} Predictive policing tools may target specific communities unfairly.
            \end{itemize}
        \item \textbf{Legal and Regulatory Frameworks:}
            \begin{itemize}
                \item \textbf{Data Protection Laws:} Regulations like GDPR enforce rules on personal data usage.
                \item \textbf{Importance:} Compliance enhances legitimacy and safeguards privacy.
            \end{itemize}
    \end{enumerate}
\end{frame}

\begin{frame}[fragile]
    \frametitle{Ethical Considerations in Data Mining Projects - Conclusion and Best Practices}
    \begin{itemize}
        \item \textbf{Develop Ethical Guidelines:} Start with a clear ethical framework for every project.
        \item \textbf{Impact Assessment:} Conduct assessments on potential ethical implications before starting.
        \item \textbf{Regular Training:} Engage teams in ongoing training about ethical data practices.
    \end{itemize}
    \textbf{Key Points to Emphasize:}
    \begin{itemize}
        \item Ethics is fundamental in data mining.
        \item Adhering to ethical principles promotes trust and sustainability.
        \item Continuous evaluation of practices is vital in the evolving data landscape.
    \end{itemize}
\end{frame}

\begin{frame}[fragile]
    \frametitle{Feedback and Continuous Improvement}
    \begin{block}{Understanding Feedback}
        Feedback is essential in any learning process as it identifies strengths and areas for growth, enhancing skills.
        \begin{itemize}
            \item \textbf{Peer Feedback}: Offers insights from classmates that highlight aspects you may overlook.
            \item \textbf{Instructor Feedback}: Brings expertise, providing constructive criticism based on established standards.
        \end{itemize}
    \end{block}
\end{frame}

\begin{frame}[fragile]
    \frametitle{Importance of Feedback}
    \begin{block}{Why is Feedback Important?}
        \begin{enumerate}
            \item \textbf{Enhancing Project Quality}
                \begin{itemize}
                    \item Identifies weaknesses in methodology, analysis, or presentation.
                    \item Example: A peer may suggest integrating data visualization techniques for a more robust analysis.
                \end{itemize}
            \item \textbf{Improving Presentation Skills}
                \begin{itemize}
                    \item Refines communication techniques, body language, and visual aids.
                    \item Example: An instructor might recommend varying tone to engage the audience better.
                \end{itemize}
        \end{enumerate}
    \end{block}
\end{frame}

\begin{frame}[fragile]
    \frametitle{Continuous Improvement through Feedback}
    \begin{block}{Iterative Process}
        Improvement involves cycles of feedback and adaptation. After feedback:
        \begin{itemize}
            \item Make revisions, present again, and solicit additional thoughts.
        \end{itemize}
    \end{block}
    
    \begin{block}{Actionable Steps}
        \begin{enumerate}
            \item \textbf{Solicit Feedback}: Actively ask for input before and after presentations.
            \item \textbf{Reflect on Feedback}: Analyze received feedback to identify common themes.
            \item \textbf{Implement Changes}: Adapt your project and presentation styles accordingly.
        \end{enumerate}
    \end{block}
\end{frame}

\begin{frame}[fragile]
    \frametitle{Practical Example}
    Consider a student presenting a data mining project analyzing sentiment toward a product launch:
    \begin{block}{Feedback Received}
        \begin{itemize}
            \item \textbf{Content}: “You need to explain how data was collected more clearly.”
            \item \textbf{Presentation}: “Use fewer text-heavy slides; include more visuals.”
        \end{itemize}
    \end{block}
    
    \begin{block}{Actions Taken}
        \begin{enumerate}
            \item Clarified data collection methodology in the next presentation.
            \item Simplified slides, highlighting key points and adding graphs.
        \end{enumerate}
    \end{block}
\end{frame}

\begin{frame}[fragile]
    \frametitle{Key Points and Conclusion}
    \begin{block}{Key Points to Emphasize}
        \begin{itemize}
            \item Continuous engagement with feedback is crucial for mastery.
            \item Peer and instructor feedback both hold unique value.
            \item Effective presentations are built on iterative cycles of practice, feedback, and improvement.
        \end{itemize}
    \end{block}
    
    \begin{block}{Conclusion}
        Embracing feedback is pivotal for academic and professional growth. Utilize feedback to enhance your projects and presentation skills moving forward.
    \end{block}
    
    \begin{block}{Remember}
        \textbf{Feedback is a gift – embrace it to foster your personal and academic growth!}
    \end{block}
\end{frame}

\begin{frame}[fragile]
    \frametitle{Conclusion and Reflection - Overview}
    As we conclude our final project presentations, it's vital to take a step back and reflect on the entire learning journey. This is not only about reviewing the projects but also understanding how these experiences have contributed to your personal and academic growth.
\end{frame}

\begin{frame}[fragile]
    \frametitle{Why Reflect?}
    Reflection is a key component of the learning process. It allows you to:
    \begin{itemize}
        \item \textbf{Identify Strengths and Weaknesses:} Recognize what you did well and areas that require more effort.
        \item \textbf{Enhance Retention:} Integrating new knowledge with existing understanding helps solidify your learning.
        \item \textbf{Foster Continuous Improvement:} Encourages a mindset geared toward ongoing personal and professional development.
    \end{itemize}
\end{frame}

\begin{frame}[fragile]
    \frametitle{Questions for Reflection}
    To guide your reflection, consider the following questions:
    \begin{enumerate}
        \item \textbf{What were the main goals of your project?}  
            Think about whether you achieved these goals and how they align with your expectations.
        \item \textbf{What challenges did you face during the project?}  
            Reflect on these difficulties and how they impacted your approach or outcomes.
        \item \textbf{How did your understanding of the subject evolve?}  
            Evaluate how your perceptions or beliefs about the topic changed throughout the project.
        \item \textbf{What skills did you develop?}  
            Identify both technical skills (like data analysis or project management) and soft skills (like teamwork or communication).
    \end{enumerate}
\end{frame}


\end{document}