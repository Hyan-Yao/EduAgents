\documentclass{beamer}

% Theme choice
\usetheme{Madrid} % You can change to e.g., Warsaw, Berlin, CambridgeUS, etc.

% Encoding and font
\usepackage[utf8]{inputenc}
\usepackage[T1]{fontenc}

% Graphics and tables
\usepackage{graphicx}
\usepackage{booktabs}

% Code listings
\usepackage{listings}
\lstset{
    basicstyle=\ttfamily\small,
    keywordstyle=\color{blue},
    commentstyle=\color{gray},
    stringstyle=\color{red},
    breaklines=true,
    frame=single
}

% Math packages
\usepackage{amsmath}
\usepackage{amssymb}

% Colors
\usepackage{xcolor}

% TikZ and PGFPlots
\usepackage{tikz}
\usepackage{pgfplots}
\pgfplotsset{compat=1.18}
\usetikzlibrary{positioning}

% Hyperlinks
\usepackage{hyperref}

% Title information
\title{Deep Q-Networks (DQN)}
\author{Your Name}
\institute{Your Institution}
\date{\today}

\begin{document}

\frame{\titlepage}

\begin{frame}[fragile]
    \frametitle{Introduction to Deep Q-Networks (DQN)}
    \begin{block}{Overview}
        Deep Q-Networks (DQN) are a groundbreaking approach in reinforcement learning that combines Q-learning with deep neural networks. This architecture allows agents to make decisions based on high-dimensional input data, like images.
    \end{block}
\end{frame}

\begin{frame}[fragile]
    \frametitle{Key Concepts}
    \begin{itemize}
        \item \textbf{Q-Learning:} Off-policy reinforcement learning algorithm that estimates the action-value function \( Q(s, a) \) to learn the value of action choices in various states.
        \item \textbf{Deep Learning:} Utilizes neural networks with multiple layers to extract features automatically; in DQN, it approximates the Q-function, learning policies from visual inputs.
    \end{itemize}
\end{frame}

\begin{frame}[fragile]
    \frametitle{Architecture of DQN}
    \begin{enumerate}
        \item \textbf{Input Layer:} Takes pre-processed state representations (e.g., frames from a game).
        \item \textbf{Hidden Layers:} One or more fully connected layers with nonlinear activation functions to capture complex features.
        \item \textbf{Output Layer:} Outputs Q-values for each possible action, selecting the one with the highest Q-value.
    \end{enumerate}
\end{frame}

\begin{frame}[fragile]
    \frametitle{Significance of DQN}
    \begin{itemize}
        \item \textbf{Handling High Dimensionality:} Successfully processes visual input, applicable for games.
        \item \textbf{Experience Replay:} Stores past experiences in a memory buffer, allowing efficient learning.
        \item \textbf{Target Network:} A separate network providing a stable target Q-value for main network updates.
    \end{itemize}
\end{frame}

\begin{frame}[fragile]
    \frametitle{Example of DQN in Action}
    \begin{block}{Example}
        An agent plays a video game by feeding frames into the DQN, processing visual information to predict Q-values for actions (e.g., move left, jump, shoot), and updating knowledge based on rewards from actions taken.
    \end{block}
\end{frame}

\begin{frame}[fragile]
    \frametitle{Key Points to Emphasize}
    \begin{itemize}
        \item DQN is an advancement over traditional Q-learning utilizing deep learning for enhanced performance.
        \item The combination of experience replay and target networks is vital for stable learning.
        \item DQN has demonstrated success in various domains, particularly in Atari games.
    \end{itemize}
\end{frame}

\begin{frame}[fragile]
    \frametitle{Q-Learning Update Rule}
    \begin{equation}
        Q(s, a) \leftarrow Q(s, a) + \alpha \left[ r + \gamma \max_{a'} Q(s', a') - Q(s, a) \right]
    \end{equation}
    \begin{itemize}
        \item Where \( \alpha \) is the learning rate, \( r \) is the reward, \( \gamma \) is the discount factor, and \( s' \) is the next state.
    \end{itemize}
\end{frame}

\begin{frame}[fragile]
    \frametitle{Conclusion}
    DQN marks a pivotal moment in reinforcement learning, effectively using deep neural networks to tackle complex decision-making tasks and advancing AI in high-dimensional observation environments.
\end{frame}

\begin{frame}[fragile]
    \frametitle{Fundamentals of Q-Learning - Overview}
    \begin{itemize}
        \item Q-Learning is a model-free reinforcement learning algorithm.
        \item It helps agents learn to select optimal actions based on their experiences.
        \item This slide reviews Q-Learning, its functions, limitations, and how DQNs propose to overcome these challenges.
    \end{itemize}
\end{frame}

\begin{frame}[fragile]
    \frametitle{What is Q-Learning?}
    \begin{itemize}
        \item Q-Learning is a model-free reinforcement learning algorithm.
        \item It learns the value of actions in states, enabling decision-making.
    \end{itemize}
\end{frame}

\begin{frame}[fragile]
    \frametitle{Key Concepts of Q-Learning}
    \begin{block}{Q-Value (Action-Value Function)}
        Represents the expected future rewards of taking action \( a \) in state \( s \), denoted as \( Q(s, a) \).
    \end{block}

    \begin{itemize}
        \item \textbf{Learning Rate (\( \alpha \)):} Value between 0 and 1 that determines how much of the new Q-value overwrites the old one.
        \item \textbf{Discount Factor (\( \gamma \)):} Value between 0 and 1 that balances long-term and short-term rewards.
    \end{itemize}
\end{frame}

\begin{frame}[fragile]
    \frametitle{Q-Learning Update Rule}
    The Q-value is updated using the formula:
    \begin{equation}
        Q(s, a) \leftarrow Q(s, a) + \alpha \left( r + \gamma \max_{a'} Q(s', a') - Q(s, a) \right)
    \end{equation}
    \begin{itemize}
        \item \( s' \): next state after taking action \( a \).
        \item \( r \): reward received after state transition from \( s \) to \( s' \).
    \end{itemize}
\end{frame}

\begin{frame}[fragile]
    \frametitle{Example of Q-Learning}
    \begin{itemize}
        \item Imagine an agent navigating a simple maze:
        \begin{itemize}
            \item \textbf{States (s):} Each position in the maze.
            \item \textbf{Actions (a):} Move Up, Down, Left, Right.
        \end{itemize}
        \item The agent explores different actions, updates its Q-values based on rewards, learning the best path to the goal.
    \end{itemize}
\end{frame}

\begin{frame}[fragile]
    \frametitle{Limitations of Q-Learning}
    \begin{itemize}
        \item \textbf{Scalability:} Q-table grows exponentially with the number of states and actions, making it impractical in large state spaces.
        \item \textbf{Exploration vs. Exploitation:} Finding the right balance can be challenging.
        \item \textbf{Convergence:} Requires significant training data to reach the optimal policy, which can be slow.
    \end{itemize}
\end{frame}

\begin{frame}[fragile]
    \frametitle{How DQN Addresses Limitations}
    \begin{itemize}
        \item Uses deep neural networks to approximate Q-values instead of storing them, handling large state spaces.
        \item Introduces \textbf{experience replay} to reuse past experiences, improving learning efficiency.
        \item Utilizes \textbf{target networks} for stable training, reducing issues from correlated updates.
    \end{itemize}
\end{frame}

\begin{frame}[fragile]
    \frametitle{Introduction to Deep Q-Networks (DQN)}
    \begin{itemize}
        \item Definition: Combines Q-learning with deep learning techniques.
        \item Purpose: Enables effective approximation of Q-value functions using deep neural networks.
    \end{itemize}
\end{frame}

\begin{frame}[fragile]
    \frametitle{Key Concepts in DQN}
    \begin{itemize}
        \item \textbf{Q-Learning}: 
        \begin{itemize}
            \item Model-free reinforcement learning algorithm.
            \item Learns value of actions in given states.
        \end{itemize}
        
        \item \textbf{Deep Learning}:
        \begin{itemize}
            \item Utilizes deep neural networks to learn patterns from complex data.
        \end{itemize}
        
        \item \textbf{Function Approximation}:
        \begin{itemize}
            \item DQNs use deep networks to estimate Q-values, solving issues with vast state spaces.
        \end{itemize}
    \end{itemize}
\end{frame}

\begin{frame}[fragile]
    \frametitle{Example Illustration and Key Points}
    \begin{itemize}
        \item \textbf{Example}: A robot navigating a maze
        \begin{itemize}
            \item Traditional Q-learning: Uses a fixed Q-table for specific positions.
            \item DQN: Employs a neural network for Q-value approximation from maze images.
        \end{itemize}
        
        \item \textbf{Key Points}:
        \begin{enumerate}
            \item Bridges Q-learning and deep learning for efficient complex data learning.
            \item Simplifies high-dimensional problems by approximating Q-values.
            \item Uses \textbf{Experience Replay} for stabilizing training.
        \end{enumerate}
    \end{itemize}
\end{frame}

\begin{frame}[fragile]
    \frametitle{Summary and Next Steps}
    \begin{itemize}
        \item DQNs represent a significant advancement in reinforcement learning.
        \item They blend Q-learning's strengths with deep learning's capacity for complex inputs.
        \item \textbf{Next Step}: Explore the architecture of DQNs.
    \end{itemize}
\end{frame}

\begin{frame}[fragile]
    \frametitle{Relevant Formula}
    \begin{block}{Q-Learning Update Rule}
        \begin{equation}
        Q(s_t, a_t) \leftarrow Q(s_t, a_t) + \alpha \left( r_t + \gamma \max_{a} Q(s_{t+1}, a) - Q(s_t, a_t) \right)
        \end{equation}
        Where:
        \begin{itemize}
            \item $s_t$: Current state
            \item $a_t$: Action taken
            \item $r_t$: Reward received
            \item $\alpha$: Learning rate
            \item $\gamma$: Discount factor
        \end{itemize}
    \end{block}
\end{frame}

\begin{frame}[fragile]
    \frametitle{Architecture of DQN - Overview}
    \begin{block}{Overview of DQN}
        Deep Q-Networks (DQN) merge traditional Q-learning with deep learning, enabling AI to learn effective policies from high-dimensional input spaces (like images). Understanding the architecture is essential for grasping how DQNs operate.
    \end{block}
\end{frame}

\begin{frame}[fragile]
    \frametitle{Architecture of DQN - Input and Hidden Layers}
    \begin{enumerate}
        \item \textbf{Input Layer:}
        \begin{itemize}
            \item \textbf{What it Accepts:} Current state representation of the environment (e.g., raw pixels in a game).
            \item \textbf{Data Preprocessing:} Transformations (e.g., grayscale conversion, resizing) reduce computational overhead.
        \end{itemize}

        \item \textbf{Hidden Layers:}
        \begin{itemize}
            \item \textbf{Convolutional Layers:}
            \begin{itemize}
                \item \textbf{Purpose:} Detect features from the input state (e.g., edges, shapes).
                \item \textbf{Functionality:} Filters generate feature maps highlighting important spatial hierarchies.
            \end{itemize}

            \item \textbf{Fully Connected Layers:}
            \begin{itemize}
                \item \textbf{Purpose:} Combine features learned by convolutional layers.
                \item \textbf{Activation Functions:} Typically, Rectified Linear Unit (ReLU) for non-linear transformations.
            \end{itemize}
        \end{itemize}
    \end{enumerate}
\end{frame}

\begin{frame}[fragile]
    \frametitle{Architecture of DQN - Output Layer and Key Points}
    \begin{itemize}
        \item \textbf{Output Layer:}
        \begin{itemize}
            \item \textbf{What it Produces:} Q-values for each possible action in the current state.
            \item \textbf{Size of Output:} If there are 4 actions, the output layer has 4 nodes.
            \item \textbf{Final Output:} The Q-value estimates expected future rewards for each action.
        \end{itemize}
    \end{itemize}

    \begin{block}{Key Points to Emphasize}
        \begin{itemize}
            \item Efficient learning from high-dimensional inputs.
            \item Convolutional layers are critical for feature extraction.
            \item Experience replay enhances learning efficiency.
        \end{itemize}
    \end{block}
\end{frame}

\begin{frame}[fragile]
    \frametitle{Architecture of DQN - Code Snippet Example}
    \begin{lstlisting}[language=Python]
class DQNModel(nn.Module):
    def __init__(self, action_size):
        super(DQNModel, self).__init__()
        self.conv1 = nn.Conv2d(in_channels=1, out_channels=32, kernel_size=8, stride=4)
        self.conv2 = nn.Conv2d(in_channels=32, out_channels=64, kernel_size=4, stride=2)
        self.fc1 = nn.Linear(64 * 7 * 7, 1024)
        self.output = nn.Linear(1024, action_size)

    def forward(self, x):
        x = F.relu(self.conv1(x))
        x = F.relu(self.conv2(x))
        x = x.view(x.size(0), -1)  # Flatten the output
        x = F.relu(self.fc1(x))
        return self.output(x)
    \end{lstlisting}
\end{frame}

\begin{frame}[fragile]
    \frametitle{Experience Replay - Definition}
    \begin{block}{Definition}
        \textbf{Experience Replay} is a technique used in Reinforcement Learning (RL), particularly in Deep Q-Networks (DQN), where an agent stores past experiences to train the model. By reusing previous experiences, the agent learns more efficiently.
    \end{block}
\end{frame}

\begin{frame}[fragile]
    \frametitle{Experience Replay - How It Works}
    \begin{enumerate}
        \item \textbf{Storage}:
        \begin{itemize}
            \item The agent collects experiences as tuples: \((s_t, a_t, r_t, s_{t+1})\)
            \item Where:
            \begin{itemize}
                \item \(s_t\) = state at time \(t\)
                \item \(a_t\) = action taken at state \(s_t\)
                \item \(r_t\) = reward received for action \(a_t\)
                \item \(s_{t+1}\) = new state after taking action \(a_t\)
            \end{itemize}
        \end{itemize}
        
        \item \textbf{Replay Buffer}:
        \begin{itemize}
            \item Experiences are stored in a **replay buffer** with a fixed size.
            \item Older experiences are discarded when new experiences fill it up.
        \end{itemize}
        
        \item \textbf{Sample \& Train}:
        \begin{itemize}
            \item Batches of experiences are randomly sampled to update the DQN.
            \item This helps break the correlation between consecutive experiences and stabilizes training.
        \end{itemize}
        
    \end{enumerate}
\end{frame}

\begin{frame}[fragile]
    \frametitle{Experience Replay - Advantages and Example}
    \begin{block}{Advantages of Experience Replay}
        \begin{enumerate}
            \item \textbf{Efficiency in Learning}:
            \begin{itemize}
                \item Reusing experiences allows the agent to learn from past actions multiple times, accelerating the learning process.
            \end{itemize}
            
            \item \textbf{Increased Stability}:
            \begin{itemize}
                \item Breaking correlation reduces update variance, leading to more stable training.
            \end{itemize}
            
            \item \textbf{Diverse Training}:
            \begin{itemize}
                \item Sampling from varied experiences helps the model generalize better.
            \end{itemize}
        \end{enumerate}
    \end{block}

    \begin{block}{Illustrative Example}
        Consider an agent playing a game:
        \begin{itemize}
            \item Experience: 
            \((s_t, a_t, r_t, s_{t+1})\): 
            \begin{itemize}
                \item \(s_t\) = "Player at position (3, 4)"
                \item \(a_t\) = "Move Right"
                \item \(r_t\) = "+10 points"
                \item \(s_{t+1}\) = "Player at position (3, 5)"
            \end{itemize}
            \item This experience is stored in the replay buffer.
        \end{itemize}
    \end{block}
\end{frame}

\begin{frame}[fragile]
    \frametitle{Target Network Mechanism in DQN - Overview}
    \begin{block}{Overview of the Target Network}
        In Deep Q-Networks (DQN), the target network stabilizes training by reducing correlations among Q-value updates. 
        This mechanism mitigates oscillations and divergence that can occur when learning from rapidly changing targets.
    \end{block}
\end{frame}

\begin{frame}[fragile]
    \frametitle{Target Network Mechanism in DQN - Functionality}
    \begin{block}{How the Target Network Works}
        \begin{enumerate}
            \item \textbf{Architecture}: 
                \begin{itemize}
                    \item DQN uses two neural networks: the online and the target network, having the same architecture but updated at different intervals.
                \end{itemize}
            \item \textbf{Q-value Updates}:
                \begin{itemize}
                    \item The online network generates Q-values based on the current state while the target network provides stable targets.
                \end{itemize}
            \item \textbf{Updating the Target Network}:
                \begin{itemize}
                    \item The target network's weights are updated to match the online network every fixed number of steps (e.g., every 1000 steps), ensuring stable targets.
                \end{itemize}
        \end{enumerate}
    \end{block}
\end{frame}

\begin{frame}[fragile]
    \frametitle{Target Network Mechanism in DQN - Impact and Example}
    \begin{block}{Key Points to Emphasize}
        \begin{itemize}
            \item \textbf{Stability}: Provides a consistent target over multiple updates.
            \item \textbf{Delay in Updates}: Acts as a buffer against sudden Q-value changes.
            \item \textbf{Effect on Learning}: Separate networks enhance convergence and performance.
        \end{itemize}
    \end{block}

    \begin{block}{Example Scenario}
        Suppose the online network predicts Q-values for actions A, B, and C. The target Q-values from the target network guide the updates of the online network.
    \end{block}

    \begin{block}{Code Snippet}
        \begin{lstlisting}[language=Python]
# Pseudo code for updating target network
if step % TARGET_UPDATE_FREQ == 0:
    target_network.load_state_dict(online_network.state_dict())
        \end{lstlisting}
    \end{block}
\end{frame}

\begin{frame}[fragile]
    \frametitle{Target Network Mechanism in DQN - Summary}
    \begin{block}{Summary}
        The target network in DQN significantly enhances training stability by decoupling the learning process of Q-values from their targets.
        This leads to more reliable updates and improved performance in reinforcement learning tasks.
    \end{block}
\end{frame}

\begin{frame}[fragile]
    \frametitle{Loss Function in DQN - Overview}
    In Deep Q-Networks (DQN), the loss function is crucial for updating Q-values and improving decision-making. 
    Understanding this loss function is essential for grasping how DQN optimizes its learning process.
\end{frame}

\begin{frame}[fragile]
    \frametitle{Key Concepts - Q-Learning and Loss Function}
    \begin{enumerate}
        \item \textbf{Q-Learning}
        \begin{itemize}
            \item A reinforcement learning algorithm to learn the value of actions in states.
            \item Goal: Approximate the optimal action-value function, \(Q^*(s, a)\).
        \end{itemize}
        
        \item \textbf{Loss Function}
        \begin{itemize}
            \item Used to measure the difference between predicted and target Q-values.
            \item Defined as Mean Squared Error (MSE):
            \begin{equation}
                L(\theta) = \mathbb{E}_{(s, a, r, s')}\left[(r + \gamma \max_{a'}Q(s', a'; \theta^-) - Q(s, a; \theta))^2\right]
            \end{equation}
            \item Where:
            \begin{itemize}
                \item \( r \): Reward after taking action \( a \) in state \( s \)
                \item \( \gamma \): Discount factor (importance of future rewards)
                \item \( Q(s, a; \theta) \): Predicted Q-value
                \item \( Q(s', a'; \theta^-) \): Target Q-value from target network
            \end{itemize}
        \end{itemize}
    \end{enumerate}
\end{frame}

\begin{frame}[fragile]
    \frametitle{Key Concepts - Target Network and How DQN Minimizes Loss}
    \begin{enumerate}
        \setcounter{enumi}{2}
        \item \textbf{Target Network}
        \begin{itemize}
            \item A secondary neural network updated less frequently for stability.
        \end{itemize}
        
        \item \textbf{How DQN Minimizes Loss}
        \begin{itemize}
            \item \textbf{Experience Replay}
            \begin{itemize}
                \item Stores previous experiences to sample during training for stability.
            \end{itemize}
            \item \textbf{Stochastic Gradient Descent (SGD)}
            \begin{equation}
                \theta \leftarrow \theta - \alpha \nabla L(\theta)
            \end{equation}
            \begin{itemize}
                \item \( \alpha \): Learning rate
            \end{itemize}
            \item \textbf{Convergence}
            \begin{itemize}
                \item Adjusts Q-values to minimize loss for optimal policy decisions.
            \end{itemize}
        \end{itemize}
    \end{enumerate}
\end{frame}

\begin{frame}[fragile]
    \frametitle{Example and Key Points}
    \textbf{Example:} Suppose an agent receives a reward of \( +1 \) after choosing action \( a \) in state \( s \) and transitioning to state \( s' \).
    
    \begin{itemize}
        \item Predicted \( Q(s, a; \theta) = 0.5 \) and target \( Q(s', a'; \theta^-) = 0.8 \).
        \item Target calculation:
        \begin{equation}
            target = r + \gamma \max_{a'}Q(s', a'; \theta^-) = 1 + 0.9 \times 0.8 = 1.72
        \end{equation}
        \item Loss calculation:
        \begin{equation}
            L(\theta) = (1.72 - 0.5)^2 = 1.2976
        \end{equation}
    \end{itemize}
    
    \textbf{Key Points:}
    \begin{itemize}
        \item The loss function is central to the learning process in DQN.
        \item Target networks and experience replay are vital for stable learning.
    \end{itemize}
\end{frame}

\begin{frame}[fragile]
    \frametitle{Training Process - Overview}
    In this slide, we will explore the steps involved in training a Deep Q-Network (DQN), focusing on:
    \begin{itemize}
        \item Epochs
        \item Batch Updates
        \item Convergence
    \end{itemize}
\end{frame}

\begin{frame}[fragile]
    \frametitle{Training Process - Epochs}
    \begin{block}{1. Epochs}
        \begin{itemize}
            \item \textbf{Definition}: An epoch refers to one complete pass through the entire training dataset. In the context of DQN, it involves using episodes of interaction with the environment.
            \item \textbf{Example}: For example, if training a DQN to play a video game, an epoch could involve the agent playing through 100 games, experiencing different states, taking actions, and receiving rewards.
        \end{itemize}
    \end{block}
\end{frame}

\begin{frame}[fragile]
    \frametitle{Training Process - Batch Updates}
    \begin{block}{2. Batch Updates}
        \begin{itemize}
            \item \textbf{Experience Replay}: This technique helps break the correlation between consecutive experiences by storing experiences in a replay buffer.
            \item \textbf{Mini-Batch Size}: Typically, a mini-batch size of 32 or 64 is used to stabilize training and improve convergence.
            \item \textbf{Loss Function Calculation}:
            \begin{equation}
            L(\theta) = \mathbb{E}_{(s,a,r,s') \sim D} \left[ \left( y - Q(s, a; \theta) \right)^2 \right]
            \end{equation}
            where:
            \begin{itemize}
                \item \( y = r + \gamma \max_{a'} Q(s', a'; \theta^{-}) \)  
                \item \( D \): replay buffer
                \item \( \theta \): parameters of the current Q-network
                \item \( \theta^{-} \): parameters of the target network
            \end{itemize}
        \end{itemize}
    \end{block}
\end{frame}

\begin{frame}[fragile]
    \frametitle{Training Process - Convergence}
    \begin{block}{3. Convergence}
        \begin{itemize}
            \item \textbf{Goal}: The goal of training is to converge the Q-values to the optimal Q-values that reflect the expected returns of actions taken in given states.
            \item \textbf{Monitoring Convergence}: This is gauged by observing if the loss stabilizes and if the Q-values approach a fixed point over time.
            \item A common practice is to plot the average loss over several epochs to visualize convergence.
        \end{itemize}
    \end{block}
\end{frame}

\begin{frame}[fragile]
    \frametitle{Training Process - Summary}
    The training process of DQN includes:
    \begin{itemize}
        \item \textbf{Epochs}: Interaction with the environment
        \item \textbf{Batch Updates}: Utilizing experience replay for stable learning
        \item \textbf{Convergence}: Indicating successful learning
    \end{itemize}
    Understanding these components is essential for properly training DQNs and applying them effectively in various scenarios.
\end{frame}

\begin{frame}[fragile]
    \frametitle{Applications of Deep Q-Networks (DQN)}
    \begin{block}{Introduction to DQN Applications}
        Deep Q-Networks (DQN) integrate deep learning into reinforcement learning, effectively handling high-dimensional state spaces and learning optimal policies from raw sensory data.
    \end{block}
\end{frame}

\begin{frame}[fragile]
    \frametitle{Applications of DQN - Gaming}
    \begin{itemize}
        \item \textbf{Industry Impact}: DQN outperformed human players in complex video games.
        \item \textbf{Example: Atari Games}
        \begin{itemize}
            \item \textit{Game: Breakout}
                \begin{itemize}
                    \item DQN learned to play and exceeded human-level performance from pixel data.
                    \item Optimized strategies for ball trajectory and paddle positioning using experience replay.
                \end{itemize}
        \end{itemize}
        \item \textbf{Key Highlights}
        \begin{itemize}
            \item Achieved superhuman performance in several games.
            \item Utilizes frame stacking to capture temporal dynamics.
        \end{itemize}
    \end{itemize}
\end{frame}

\begin{frame}[fragile]
    \frametitle{Applications of DQN - Robotics, Autonomous Vehicles, and Finance}
    \begin{itemize}
        \item \textbf{Robotics}
        \begin{itemize}
            \item DQNs control robots in complex environments.
            \item \textit{Example: Robotic Manipulation}
            \begin{itemize}
                \item Task: Control a robotic arm to pick up objects.
                \item Adapts well to new configurations and object types.
            \end{itemize}
        \end{itemize}
        
        \item \textbf{Autonomous Vehicles}
        \begin{itemize}
            \item Application in self-driving cars for decision-making.
            \item \textit{Example: Path Planning}
            \begin{itemize}
                \item Helps in lane changing and obstacle avoidance using past experiences.
            \end{itemize}
        \end{itemize}
        
        \item \textbf{Finance}
        \begin{itemize}
            \item Aids in algorithmic trading strategies.
            \item \textit{Example: Stock Trading}
            \begin{itemize}
                \item Makes buy/sell decisions based on historical market conditions.
            \end{itemize}
        \end{itemize}
    \end{itemize}
\end{frame}

\begin{frame}[fragile]
    \frametitle{Conclusion and Key Takeaway}
    \begin{itemize}
        \item DQNs show versatility across gaming, robotics, autonomous vehicles, and finance.
        \item They effectively tackle complex decision-making tasks.
        \item \textbf{Key Takeaway}: DQNs represent a leap in artificial intelligence, showcasing the power of combining reinforcement learning with deep learning.
    \end{itemize}
\end{frame}

\begin{frame}[fragile]
    \frametitle{Challenges Faced by DQN - Part 1}
    \begin{enumerate}
        \item \textbf{Overestimation Bias}:
        \begin{itemize}
            \item Description: DQNs often tend to overestimate action values due to function approximation.
            \item Illustration: Q-values for certain actions get exaggerated, leading to suboptimal policy decisions.
            \item Example: In a game, a high Q-value for an unfavorable action may cause the agent to prefer it over better options.
        \end{itemize}

        \item \textbf{Instability and Divergence}:
        \begin{itemize}
            \item Description: Training can become unstable from correlations in data and continuous updates to the Q-network.
            \item Example: Small perturbations in input may lead to drastic output changes if the model is not converged.
        \end{itemize}
    \end{enumerate}
\end{frame}

\begin{frame}[fragile]
    \frametitle{Challenges Faced by DQN - Part 2}
    \begin{enumerate}
        \setcounter{enumi}{2}
        \item \textbf{Sample Inefficiency}:
        \begin{itemize}
            \item Description: DQNs require significant experience for training due to high dimensional state spaces.
            \item Example: In complex environments, learning optimal policies may take thousands of episodes.
        \end{itemize}

        \item \textbf{Need for Hyperparameter Tuning}:
        \begin{itemize}
            \item Description: Multiple hyperparameters (learning rate, discount factor) heavily influence training success.
            \item Key Point: Finding optimal hyperparameters necessitates extensive trial and error, which is time-consuming.
        \end{itemize}
    \end{enumerate}
\end{frame}

\begin{frame}[fragile]
    \frametitle{Future Research Opportunities in DQN}
    \begin{enumerate}
        \item \textbf{Improved Value Function Estimation}:
        \begin{itemize}
            \item Goal: Reduce overestimation bias using techniques like Double DQN.
            \item Future Directions: Explore ensemble approaches for multiple value estimators to improve accuracy.
        \end{itemize}

        \item \textbf{Algorithmic Enhancements}:
        \begin{itemize}
            \item Ideas: Investigate methods like Dueling DQN for better training efficiency using separate state value and advantage estimates.
            \item Focus: Enhancing DQN architecture for more robust policies with lesser data.
        \end{itemize}

        \item \textbf{Transfer Learning and Meta-Reinforcement Learning}:
        \begin{itemize}
            \item Description: Integrate knowledge from previous tasks to expedite new task training.
            \item Future Directions: Apply DQNs in dynamic real-world settings, learning from past experiences.
        \end{itemize}

        \item \textbf{Multi-Agent Systems}:
        \begin{itemize}
            \item Description: Extend DQNs for cooperation among multiple agents learning concurrently.
            \item Future Directions: Explore communication strategies and collaboration to improve learning in complex environments.
        \end{itemize}
    \end{enumerate}
\end{frame}


\end{document}