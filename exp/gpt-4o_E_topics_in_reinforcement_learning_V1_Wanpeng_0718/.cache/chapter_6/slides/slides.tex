\documentclass{beamer}

% Theme choice
\usetheme{Madrid} % You can change to e.g., Warsaw, Berlin, CambridgeUS, etc.

% Encoding and font
\usepackage[utf8]{inputenc}
\usepackage[T1]{fontenc}

% Graphics and tables
\usepackage{graphicx}
\usepackage{booktabs}

% Code listings
\usepackage{listings}
\lstset{
basicstyle=\ttfamily\small,
keywordstyle=\color{blue},
commentstyle=\color{gray},
stringstyle=\color{red},
breaklines=true,
frame=single
}

% Math packages
\usepackage{amsmath}
\usepackage{amssymb}

% Colors
\usepackage{xcolor}

% TikZ and PGFPlots
\usepackage{tikz}
\usepackage{pgfplots}
\pgfplotsset{compat=1.18}
\usetikzlibrary{positioning}

% Hyperlinks
\usepackage{hyperref}

% Title information
\title{Week 6: Exploring SARSA}
\author{Your Name}
\institute{Your Institution}
\date{\today}

\begin{document}

\frame{\titlepage}

\begin{frame}[fragile]
    \frametitle{Introduction to SARSA}
    \begin{block}{What is SARSA?}
        SARSA (State-Action-Reward-State-Action) is an on-policy reinforcement learning algorithm used to estimate the action-value function. It helps an agent learn how to behave optimally in a given environment.
    \end{block}
    
    \begin{block}{Key Components of SARSA:}
        \begin{itemize}
            \item \textbf{State (S)}: The current situation in which the agent finds itself.
            \item \textbf{Action (A)}: The choice made by the agent to interact with the environment.
            \item \textbf{Reward (R)}: Feedback from the environment after taking action A in state S.
            \item \textbf{Next State (S')} : The new situation resulting from the action taken.
            \item \textbf{Next Action (A')} : The action taken in the new state S'.
        \end{itemize}
    \end{block}
\end{frame}

\begin{frame}[fragile]
    \frametitle{SARSA Learning Process}
    \begin{enumerate}
        \item \textbf{Initialize}: Set initial values for Q(S,A) arbitrarily.
        \item \textbf{Policy Selection}: Choose an action A based on the current policy (e.g., $\epsilon$-greedy).
        \item \textbf{Execute Action}: Perform action A, receive reward R, and observe new state S'.
        \item \textbf{Next Action}: Choose the next action A' using the current policy.
        \item \textbf{Update Q-value}:
        \begin{equation}
        Q(S, A) \leftarrow Q(S, A) + \alpha \big( R + \gamma Q(S', A') - Q(S, A) \big)
        \end{equation}
        where:
        \begin{itemize}
            \item $\alpha$: Learning rate.
            \item $\gamma$: Discount factor.
        \end{itemize}
        \item \textbf{Iteration}: Repeat until the policy converges.
    \end{enumerate} 
\end{frame}

\begin{frame}[fragile]
    \frametitle{Example of SARSA}
    Consider a simple grid world where an agent navigates from a start to a goal position:
    \begin{itemize}
        \item \textbf{State}: (2,2)
        \item \textbf{Action}: Move right
        \item \textbf{Reward}: +10 (goal)
        \item \textbf{Next State}: (2,3)
        \item \textbf{Next Action}: Explore possible moves according to the policy.
    \end{itemize}

    \begin{block}{Why is SARSA Relevant?}
        \begin{itemize}
            \item \textbf{On-Policy Learning}: Adjusts updates based on the current policy.
            \item \textbf{Exploration vs. Exploitation}: Utilizes $\epsilon$-greedy action selection, balancing exploration and optimal policies.
            \item \textbf{Applicability}: Useful in various reinforcement learning problems, from simple grid-worlds to complex navigation tasks.
        \end{itemize}
    \end{block}
\end{frame}

\begin{frame}[fragile]
    \frametitle{Reinforcement Learning Basics - Introduction}
    \begin{block}{Definition}
        Reinforcement Learning (RL) is a subfield of machine learning where an agent learns to make decisions by interacting with its environment. This learning process is driven by feedback in the form of rewards or penalties.
    \end{block}
\end{frame}

\begin{frame}[fragile]
    \frametitle{Reinforcement Learning Basics - Key Components}
    \begin{enumerate}
        \item \textbf{Agent}
            \begin{itemize}
                \item Definition: The learner/decision-maker interacting with the environment.
                \item Example: A robot navigating a maze.
                \item Role: Learn a policy to maximize cumulative reward.
            \end{itemize}
        
        \item \textbf{Environment}
            \begin{itemize}
                \item Definition: Everything the agent interacts with.
                \item Example: Walls and pathways in a maze.
                \item Dynamics: Evolves in response to agent's actions, defining a Markov Decision Process (MDP).
            \end{itemize}
        
        \item \textbf{State}
            \begin{itemize}
                \item Definition: A specific configuration of the environment at a given time.
                \item Example: The robot's location in the maze.
            \end{itemize}
    \end{enumerate}
\end{frame}

\begin{frame}[fragile]
    \frametitle{Reinforcement Learning Basics - Actions and Rewards}
    \begin{enumerate}
        \setcounter{enumi}{3}
        \item \textbf{Action}
            \begin{itemize}
                \item Definition: A choice made by the agent that affects the state.
                \item Example: Moving left in the maze.
                \item Policy: The strategy for choosing actions based on the current state.
            \end{itemize}
        
        \item \textbf{Reward}
            \begin{itemize}
                \item Definition: Feedback signal received after an action.
                \item Example: Positive reward for reaching the exit.
                \item Objective: Maximize total reward over time.
            \end{itemize}
    \end{enumerate}
\end{frame}

\begin{frame}[fragile]
    \frametitle{Reinforcement Learning Basics - Learning Objective}
    \begin{block}{Objective}
        The agent seeks to learn an optimal policy (\(\pi\)) that maximizes the expected cumulative reward, represented as:
        \begin{equation}
            R = r_1 + \gamma r_2 + \gamma^2 r_3 + ... + \gamma^{t-1} r_t
        \end{equation}
        where:
        \begin{itemize}
            \item \(r_t\) is the reward at time \(t\),
            \item \(\gamma\) (0 ≤ \(\gamma\) < 1) is the discount factor prioritizing immediate rewards.
        \end{itemize}
    \end{block}
\end{frame}

\begin{frame}[fragile]
    \frametitle{Reinforcement Learning Basics - Conclusion}
    \begin{block}{Summary}
        Understanding these foundational concepts is crucial for diving deeper into reinforcement learning algorithms like SARSA. The interplay between the agent, environment, states, actions, and rewards creates a framework for effective learning and decision-making strategies.
    \end{block}
\end{frame}

\begin{frame}[fragile]
    \frametitle{What is SARSA?}
    \begin{block}{Introduction}
        SARSA stands for State-Action-Reward-State-Action. It is an on-policy reinforcement learning algorithm that trains agents to make a series of decisions aimed at maximizing cumulative reward.
    \end{block}
\end{frame}

\begin{frame}[fragile]
    \frametitle{Key Concepts}
    \begin{itemize}
        \item \textbf{State (S)}: Representation of the environment at a given time.
        \item \textbf{Action (A)}: Decision made by the agent that influences its state.
        \item \textbf{Reward (R)}: Numerical feedback signal received after an action in a specific state.
    \end{itemize}
\end{frame}

\begin{frame}[fragile]
    \frametitle{The SARSA Process}
    \begin{enumerate}
        \item \textbf{Initialization}: Initialize Q-values for all state-action pairs.
        \item \textbf{Select Action}: Choose an action (A) using the $\epsilon$-greedy strategy.
        \item \textbf{Take Action}: Execute action A, transition to new state (S'), and receive reward (R).
        \item \textbf{Select Next Action}: Choose next action (A') in new state (S') with the same $\epsilon$-greedy strategy.
        \item \textbf{Update Q-value}:
        \begin{equation}
            Q(S, A) \leftarrow Q(S, A) + \alpha \left[ R + \gamma Q(S', A') - Q(S, A) \right]
        \end{equation}
        \item \textbf{Repeat}: Transition back to step 2 for the next episode.
    \end{enumerate}
\end{frame}

\begin{frame}[fragile]
    \frametitle{Example of SARSA}
    \begin{block}{Scenario}
        Imagine a robot navigating a maze. At each position (state), it can go left, right, up, or down (actions). 
    \end{block}
    \begin{itemize}
        \item Rewards: +10 for finding the exit, -5 for hitting a wall.
        \item Using SARSA, the robot learns from received rewards while exploring paths, updating Q-values accordingly.
    \end{itemize}
\end{frame}

\begin{frame}[fragile]
    \frametitle{Key Points to Emphasize}
    \begin{itemize}
        \item SARSA is an \textbf{on-policy} method, valuing the actions executed by the agent.
        \item Balances \textbf{exploration} with \textbf{exploitation}.
        \item Learning rate ($\alpha$) and discount factor ($\gamma$) are crucial for blending immediate and future rewards.
    \end{itemize}
\end{frame}

\begin{frame}[fragile]
    \frametitle{SARSA Algorithm Steps - Overview}
    The SARSA (State-Action-Reward-State-Action) algorithm is a model-free reinforcement learning technique. It:
    \begin{itemize}
        \item Is an on-policy algorithm, improving the policy used to make decisions.
        \item Works in stochastic environments.
    \end{itemize}
\end{frame}

\begin{frame}[fragile]
    \frametitle{SARSA Algorithm Steps - Step-by-Step Process}
    \begin{enumerate}
        \item \textbf{Initialize the Q-values:}
            \begin{itemize}
                \item Create a table or function for Q-values of all state-action pairs.
                \item Initial values are typically set to zeros.
                \begin{lstlisting}
Q(s, a) = 0 for all (s, a)
                \end{lstlisting}
            \end{itemize}
        \item \textbf{Choose an action using an epsilon-greedy policy:}
            \begin{itemize}
                \item Select action $a$ based on state $s$ with probability $\epsilon$ for exploration.
                \begin{lstlisting}
a = {
    random_action with probability \epsilon
    greedy_action with probability (1 - \epsilon)
}
                \end{lstlisting}
            \end{itemize}
    \end{enumerate}
\end{frame}

\begin{frame}[fragile]
    \frametitle{SARSA Algorithm Steps - Continued}
    \begin{enumerate}[resume]
        \item \textbf{Perform the action and observe reward and next state:}
            \begin{itemize}
                \item Take action $a$, observe reward $r$, and move to state $s_{next}$.
            \end{itemize}
        \item \textbf{Choose the next action from the new state:}
            \begin{itemize}
                \item Select action $a_{next}$ from state $s_{next}$ using the same epsilon-greedy policy.
            \end{itemize}
        \item \textbf{Update the Q-value:}
            \begin{itemize}
                \item Using the Bellman equation:
                \begin{equation}
Q(s, a) \leftarrow Q(s, a) + \alpha \left[ r + \gamma \cdot Q(s_{next}, a_{next}) - Q(s, a) \right]
                \end{equation}
                Where:
                \begin{itemize}
                    \item $\alpha$ = learning rate
                    \item $\gamma$ = discount factor
                \end{itemize}
            \end{itemize}
    \end{enumerate}
\end{frame}

\begin{frame}[fragile]
    \frametitle{Exploration vs. Exploitation in SARSA - Key Concept}
    \begin{block}{The Exploration-Exploitation Dilemma}
        \begin{itemize}
            \item \textbf{Definition}: The challenge of balancing between exploring new actions for potential rewards and exploiting known actions for maximum reward based on current knowledge.
            \item \textbf{Importance}: Finding the right balance between gaining new experience (exploration) and leveraging current knowledge (exploitation) is crucial for effective learning in environments.
        \end{itemize}
    \end{block}
\end{frame}

\begin{frame}[fragile]
    \frametitle{How SARSA Addresses the Dilemma}
    \begin{block}{SARSA (State-Action-Reward-State-Action)}
        An on-policy RL algorithm that updates Q-values based on the actions taken under the current policy.
    \end{block}
    
    \begin{block}{Policy Use}
        SARSA employs an $\epsilon$-greedy strategy:
        \begin{itemize}
            \item With probability $\epsilon$, the agent explores by choosing a random action.
            \item With probability $1 - \epsilon$, it exploits the best-known action based on its Q-values.
        \end{itemize}
    \end{block}
\end{frame}

\begin{frame}[fragile]
    \frametitle{Example Scenario}
    \begin{block}{Grid World Navigation}
        \begin{itemize}
            \item \textbf{Exploration}: If the agent is in cell (2,2) and randomly moves to (2,3), it may discover a new path leading to a reward.
            \item \textbf{Exploitation}: If the agent knows that moving to (2,1) yields a high reward, it will prefer that action.
        \end{itemize}
    \end{block}
\end{frame}

\begin{frame}[fragile]
    \frametitle{SARSA Update Rule}
    The Q-value update for search and exploitation is formulated as follows:
    \begin{equation}
        Q(s, a) \leftarrow Q(s, a) + \alpha \left[ r + \gamma Q(s', a') - Q(s, a) \right]
    \end{equation}
    
    \begin{itemize}
        \item Where:
        \begin{itemize}
            \item $s$: Current state
            \item $a$: Current action
            \item $r$: Reward received after taking action $a$
            \item $s'$: Next state
            \item $a'$: Next action (determined by the $\epsilon$-greedy policy)
            \item $\alpha$: Learning rate
            \item $\gamma$: Discount factor
        \end{itemize}
    \end{itemize}
\end{frame}

\begin{frame}[fragile]
    \frametitle{Key Points to Emphasize}
    \begin{itemize}
        \item \textbf{Dynamic Balance}: The balance between exploration and exploitation changes over time; exploration is critical at the start of training, while exploitation is vital as the agent learns more.
        \item \textbf{Impact of $\epsilon$}: The value of $\epsilon$ significantly affects learning; too high may lead to inefficient learning, while too low can cause suboptimal performance if the agent gets stuck in local optima.
    \end{itemize}
\end{frame}

\begin{frame}[fragile]
    \frametitle{Conclusion}
    Exploration and exploitation are fundamental aspects of the SARSA algorithm. They shape the agent's learning strategy and determine its efficiency in optimizing performance in its environment.
\end{frame}

\begin{frame}[fragile]
    \frametitle{Comparison with Q-learning - Overview}
    \begin{block}{Understanding the Algorithms}
        Both SARSA (State-Action-Reward-State-Action) and Q-learning are popular reinforcement learning algorithms used to learn optimal action-selection policies.
        While they share similarities, they exhibit key differences that affect their applications and performance.
    \end{block}
\end{frame}

\begin{frame}[fragile]
    \frametitle{Comparison with Q-learning - Similarities and Differences}
    \begin{block}{Key Similarities}
        \begin{itemize}
            \item Both aim to find the optimal policy that maximizes cumulative reward.
            \item Both use Q-values (action-value functions) to evaluate the expected future rewards of actions taken in given states.
            \item Both algorithms iteratively improve their policies by updating Q-values based on experiences.
        \end{itemize}
    \end{block}

    \begin{block}{Key Differences}
        \begin{tabular}{|l|l|l|}
            \hline
            Feature                     & SARSA                            & Q-learning                      \\ \hline
            Update Rule                 & On-policy; updates the Q-value for the action actually taken. & Off-policy; updates the Q-value based on the maximum estimated action-value. \\ \hline
            Exploration Method          & Uses current policy for action selection (i.e. $\epsilon$-greedy from current policy). & Selects the greedy action (argmax) in the update, promoting $\epsilon$-greedy sampling. \\ \hline
            Learning Stability           & More stable in high variability environments due to on-policy nature. & Can be more aggressive in learning, potentially leading to instability. \\ \hline
            Convergence                 & Converges under certain conditions but may require more exploration. & Guarantees convergence if all state-action pairs are sufficiently explored. \\ \hline
        \end{tabular}
    \end{block}
\end{frame}

\begin{frame}[fragile]
    \frametitle{Comparison with Q-learning - Update Equations}
    \begin{block}{Visual Representation}
        \begin{equation}
            Q(s, a) \leftarrow Q(s, a) + \alpha \left[ r + \gamma \max_{a'} Q(s', a') - Q(s, a) \right]
        \end{equation}
        \begin{equation}
            Q(s, a) \leftarrow Q(s, a) + \alpha \left[ r + \gamma Q(s', a') - Q(s, a) \right]
        \end{equation}
    \end{block}
    
    \begin{block}{Example Scenario}
        Consider a robotic agent navigating a grid to reach a goal:
        \begin{itemize}
            \item SARSA: The agent chooses actions based on its current policy and learns the Q-value accordingly.
            \item Q-learning: The agent selects actions by considering the highest future rewards, updating Q-values based on the best potential action.
        \end{itemize}
    \end{block}
\end{frame}

\begin{frame}[fragile]
    \frametitle{Comparison with Q-learning - Key Takeaways}
    \begin{block}{Key Takeaways}
        \begin{itemize}
            \item SARSA learns the value of actions taken under the current policy; Q-learning aims to learn the best possible actions.
            \item Applications differ based on the need for stability (SARSA) or aggressive learning (Q-learning).
        \end{itemize}
    \end{block}

    \begin{block}{Summary}
        Understanding the differences and similarities helps in selecting the most suitable algorithm, leading to effective learning strategies.
    \end{block}
\end{frame}

\begin{frame}[fragile]
    \frametitle{Next Steps}
    In the upcoming slide, we will explore various variations of the SARSA algorithm, including SARSA($\lambda$) and Deep SARSA, to understand their enhancements and use cases.
\end{frame}

\begin{frame}[fragile]
    \frametitle{SARSA Variations - Overview}
    This section delves into notable variations of the SARSA algorithm, specifically SARSA(λ) and Deep SARSA. 
    \begin{itemize}
        \item Aim to improve efficiency and effectiveness of the original SARSA algorithm 
        \item Enhance applicability to more complex environments
    \end{itemize}
\end{frame}

\begin{frame}[fragile]
    \frametitle{SARSA(λ)}
    \begin{block}{Definition}
        SARSA(λ) is an extension of the standard SARSA algorithm that incorporates eligibility traces for enhanced learning speed and performance.
    \end{block}
    
    \begin{itemize}
        \item **Eligibility Traces:** Track states that have been visited, allowing for efficient Q-value updates.
        \item **Lambda (λ):** Decay rate parameter (0 ≤ λ ≤ 1) blending TD learning and Monte Carlo methods.
    \end{itemize}
    
    \begin{block}{Update Formula}
        The update rule can be represented as:
        \[
        Q(s, a) \leftarrow Q(s, a) + \alpha \delta \cdot E(s, a)
        \]
        where \( \delta = r + \gamma Q(s', a') - Q(s, a) \).
    \end{block}
\end{frame}

\begin{frame}[fragile]
    \frametitle{SARSA(λ) - Example}
    \begin{block}{Example}
        Consider a robot navigating a maze:
        \begin{itemize}
            \item If it receives a reward for reaching a goal, updates propagate back to all previously visited states.
            \item Updates are proportionate to their trace values, allowing for faster learning compared to updating only the last state-action pair.
        \end{itemize}
    \end{block}
\end{frame}

\begin{frame}[fragile]
    \frametitle{Deep SARSA}
    \begin{block}{Definition}
        Deep SARSA integrates Deep Learning with the SARSA algorithm by utilizing neural networks to approximate Q-values.
    \end{block}

    \begin{itemize}
        \item **Neural Networks:** Replace the Q-value table with a deep neural network predicting Q-values for any state-action pair.
        \item **Experience Replay:** Store and sample experiences to break correlation in learning data, enhancing stability.
    \end{itemize}
    
    \begin{block}{Q-Value Update Formula}
        Update involves adjusting the weights of the neural network:
        \[
        \theta \leftarrow \theta + \beta \nabla_{\theta} L(\theta)
        \]
        where \( L(\theta) \) is the mean squared error between predicted and target Q-values.
    \end{block}
\end{frame}

\begin{frame}[fragile]
    \frametitle{Deep SARSA - Example}
    \begin{block}{Example}
        In a complex video game:
        \begin{itemize}
            \item A deep neural network is trained to predict the expected utility of various actions based on pixel inputs (the game state).
            \item Allows for nuanced decision-making that traditional tabular SARSA cannot provide.
        \end{itemize}
    \end{block}
\end{frame}

\begin{frame}[fragile]
    \frametitle{Key Points to Emphasize}
    \begin{itemize}
        \item SARSA(λ) enhances learning through eligibility traces, aiding environments with delayed rewards.
        \item Deep SARSA leverages neural networks, making it effective in high-dimensional environments, facilitating advancements in robotics and gaming.
    \end{itemize}
    
    \textbf{Additional Notes:}
    \begin{itemize}
        \item Experimenting with parameters in SARSA(λ) or neural network architecture in Deep SARSA significantly impacts performance.
        \item Both variations demonstrate the adaptability of SARSA to tackle real-world challenges effectively.
    \end{itemize}
\end{frame}

\begin{frame}[fragile]
    \frametitle{Practical Applications of SARSA - Overview}
    \begin{block}{What is SARSA?}
        SARSA (State-Action-Reward-State-Action) is an on-policy reinforcement learning algorithm that updates action-value functions. It learns from the actions that the agent takes, making it suitable for environments with known action policies.
    \end{block}
\end{frame}

\begin{frame}[fragile]
    \frametitle{Practical Applications of SARSA - Part 1}
    \begin{enumerate}
        \item \textbf{Robotics and Autonomous Systems}  
            \begin{itemize}
                \item \textit{Example:} Robot Navigation
                \item SARSA enables robots to learn optimal navigation paths while avoiding obstacles.
            \end{itemize}
        
        \item \textbf{Game AI Development}  
            \begin{itemize}
                \item \textit{Example:} Video Game Agents
                \item AI-controlled characters learn optimal strategies based on player actions.
            \end{itemize}
    \end{enumerate}
\end{frame}

\begin{frame}[fragile]
    \frametitle{Practical Applications of SARSA - Part 2}
    \begin{enumerate}[resume]
        \item \textbf{Finance and Trading Systems}  
            \begin{itemize}
                \item \textit{Example:} Portfolio Management
                \item SARSA is used in stock trading to optimize buying and selling decisions.
            \end{itemize}

        \item \textbf{Health Care and Treatment Planning}  
            \begin{itemize}
                \item \textit{Example:} Personalized Treatment Decisions
                \item Optimize treatment paths based on patient responses to previous treatments.
            \end{itemize}
    \end{enumerate}

    \begin{block}{Key Points}
        \begin{itemize}
            \item SARSA learns from actions taken, refining strategies through experience.
            \item Versatile across various fields, effective in dynamic environments.
        \end{itemize}
    \end{block}
\end{frame}

\begin{frame}[fragile]
    \frametitle{Practical Applications of SARSA - Conclusion}
    \begin{block}{Conclusion}
        SARSA provides effective solutions in numerous real-world scenarios. Its adaptability to evolving situations makes it a valuable reinforcement learning tool.
    \end{block}
    
    \begin{block}{Further Study}
        Interested individuals should explore frameworks such as OpenAI Gym to implement SARSA in simulated environments.
    \end{block}
\end{frame}

\begin{frame}[fragile]
    \frametitle{Challenges in SARSA Implementation - Overview}
    \begin{itemize}
        \item Implementing the SARSA (State-Action-Reward-State-Action) algorithm presents several challenges.
        \item Key challenges include:
            \begin{itemize}
                \item Exploration vs. Exploitation Dilemma
                \item Learning Rate Selection
                \item Reward Structure
                \item State and Action Space Size
                \item Improper Initialization
            \end{itemize}
    \end{itemize}
\end{frame}

\begin{frame}[fragile]
    \frametitle{Challenges in SARSA Implementation - Details}
    \begin{enumerate}
        \item \textbf{Exploration vs. Exploitation Dilemma}
            \begin{itemize}
                \item Balancing exploration of new actions and exploiting known rewards.
                \item \textbf{Strategy:} Implement $\epsilon$-greedy or softmax action selection.
            \end{itemize}
        \item \textbf{Learning Rate Selection}
            \begin{itemize}
                \item Choosing a suitable learning rate $\alpha$ is critical for convergence.
                \item \textbf{Strategy:} Utilize adaptive learning rates or techniques like RMSProp or Adam.
            \end{itemize}
        \item \textbf{Reward Structure}
            \begin{itemize}
                \item Sparse or poorly designed rewards hinder learning.
                \item \textbf{Strategy:} Design consistent and guiding reward functions.
            \end{itemize}
    \end{enumerate}
\end{frame}

\begin{frame}[fragile]
    \frametitle{Challenges in SARSA Implementation - Further Considerations}
    \begin{enumerate}[resume]
        \item \textbf{State and Action Space Size}
            \begin{itemize}
                \item Large spaces increase computational load and slow convergence.
                \item \textbf{Strategy:} Use function approximation or state abstraction techniques.
            \end{itemize}
        \item \textbf{Improper Initialization}
            \begin{itemize}
                \item Poor Q-value initialization can bias learning.
                \item \textbf{Strategy:} Initialize Q-values optimally to encourage exploration.
            \end{itemize}
    \end{enumerate}
    
    \begin{block}{Key Points to Emphasize}
        \begin{itemize}
            \item Address the exploration-exploitation dilemma effectively.
            \item Tune the learning rate for better convergence.
            \item A well-designed reward structure is essential for efficient learning.
            \item Function approximation is useful in high-dimensional spaces.
        \end{itemize}
    \end{block}
    
    \begin{lstlisting}[language=Python, caption={Example: $\epsilon$-greedy Policy}]
import numpy as np

def select_action(Q, epsilon):
    if np.random.rand() < epsilon:  # Exploration
        return np.random.choice(action_space)  # Random action
    else:  # Exploitation
        return np.argmax(Q)  # Best known action
    \end{lstlisting}
\end{frame}

\begin{frame}[fragile]
    \frametitle{Conclusion and Future Directions - Overview}
    \begin{block}{Overview of SARSA in Reinforcement Learning}
        \begin{itemize}
            \item SARSA (State-Action-Reward-State-Action) is an on-policy reinforcement learning algorithm.
            \item It learns the value of actions taken in a given state, optimizing the action-selection strategy based on the current policy.
        \end{itemize}
    \end{block}
\end{frame}

\begin{frame}[fragile]
    \frametitle{Conclusion and Future Directions - Key Takeaways}
    \begin{block}{Key Takeaways}
        \begin{enumerate}
            \item \textbf{On-Policy Learning}:
            \begin{itemize}
                \item Updates Q-values based on actions taken by the policy being learned.
                \item This leads to more accurate estimates of expected future rewards.
            \end{itemize}
            \item \textbf{Exploration vs. Exploitation}:
            \begin{itemize}
                \item Uses strategies like $\epsilon$-greedy exploration to balance exploring new actions with exploiting known good actions.
            \end{itemize}
            \item \textbf{Challenges Addressed}:
            \begin{itemize}
                \item Includes convergence issues, balancing exploration, and fine-tuning hyperparameters.
            \end{itemize}
        \end{enumerate}
    \end{block}
\end{frame}

\begin{frame}[fragile]
    \frametitle{Conclusion and Future Directions - Future Trends}
    \begin{block}{Future Directions}
        \begin{enumerate}
            \item \textbf{Integration with Deep Learning}:
            \begin{itemize}
                \item Combining SARSA with deep learning may enhance learning efficiency in complex environments.
            \end{itemize}
            \item \textbf{Adaptive Exploration Strategies}:
            \begin{itemize}
                \item Research on advanced exploration techniques could improve the balance of exploration and exploitation.
            \end{itemize}
            \item \textbf{Real-World Applications}:
            \begin{itemize}
                \item Continuous exploration in fields like robotics, healthcare, and finance can reveal practical utilities.
            \end{itemize}
            \item \textbf{Hybrid Approaches}:
            \begin{itemize}
                \item Future work may explore hybrid algorithms combining SARSA with other reinforcement learning methods.
            \end{itemize}
        \end{enumerate}
    \end{block}
\end{frame}


\end{document}