\documentclass[aspectratio=169]{beamer}

% Theme and Color Setup
\usetheme{Madrid}
\usecolortheme{whale}
\useinnertheme{rectangles}
\useoutertheme{miniframes}

% Additional Packages
\usepackage[utf8]{inputenc}
\usepackage[T1]{fontenc}
\usepackage{graphicx}
\usepackage{booktabs}
\usepackage{listings}
\usepackage{amsmath}
\usepackage{amssymb}
\usepackage{xcolor}
\usepackage{tikz}
\usepackage{pgfplots}
\pgfplotsset{compat=1.18}
\usetikzlibrary{positioning}
\usepackage{hyperref}

% Custom Colors
\definecolor{myblue}{RGB}{31, 73, 125}
\definecolor{mygray}{RGB}{100, 100, 100}
\definecolor{mygreen}{RGB}{0, 128, 0}
\definecolor{myorange}{RGB}{230, 126, 34}
\definecolor{mycodebackground}{RGB}{245, 245, 245}

% Set Theme Colors
\setbeamercolor{structure}{fg=myblue}
\setbeamercolor{frametitle}{fg=white, bg=myblue}
\setbeamercolor{title}{fg=myblue}
\setbeamercolor{section in toc}{fg=myblue}
\setbeamercolor{item projected}{fg=white, bg=myblue}
\setbeamercolor{block title}{bg=myblue!20, fg=myblue}
\setbeamercolor{block body}{bg=myblue!10}
\setbeamercolor{alerted text}{fg=myorange}

% Set Fonts
\setbeamerfont{title}{size=\Large, series=\bfseries}
\setbeamerfont{frametitle}{size=\large, series=\bfseries}
\setbeamerfont{caption}{size=\small}
\setbeamerfont{footnote}{size=\tiny}

% Custom Commands
\newcommand{\hilight}[1]{\colorbox{myorange!30}{#1}}
\newcommand{\concept}[1]{\textcolor{myblue}{\textbf{#1}}}
\newcommand{\separator}{\begin{center}\rule{0.5\linewidth}{0.5pt}\end{center}}

% Title Page Information
\title[AI Case Studies]{Week 10: Case Studies in AI}
\author[J. Smith]{John Smith, Ph.D.}
\institute[University Name]{
  Department of Computer Science\\
  University Name\\
  \vspace{0.3cm}
  Email: email@university.edu\\
  Website: www.university.edu
}
\date{\today}

% Document Start
\begin{document}

\frame{\titlepage}

\begin{frame}[fragile]
    \frametitle{Introduction to Case Studies in AI}
    \begin{block}{Description}
        This chapter explores the significance of case studies in artificial intelligence (AI) applications, emphasizing the ethical dilemmas that arise as AI technologies evolve and become more integrated into various sectors.
    \end{block}
\end{frame}

\begin{frame}[fragile]
    \frametitle{Importance of Case Studies in AI}
    \begin{itemize}
        \item \textbf{Real-World Applications:} 
            \begin{itemize}
                \item Provide concrete examples of AI implementation across industries, offering insights into successes and challenges.
            \end{itemize}
        \item \textbf{Learning from Experience:} 
            \begin{itemize}
                \item Allows stakeholders—researchers, practitioners, and policymakers—to derive lessons from effective strategies and mistakes.
            \end{itemize}
        \item \textbf{Ethical Considerations:} 
            \begin{itemize}
                \item Understanding the moral implications of AI's use is crucial, as case studies reveal ethical dilemmas that challenge existing norms.
            \end{itemize}
    \end{itemize}
\end{frame}

\begin{frame}[fragile]
    \frametitle{Understanding Ethical Dilemmas}
    \begin{itemize}
        \item \textbf{Definition:} 
            \begin{itemize}
                \item Conflicts between societal values and technological capabilities in the deployment of AI technologies.
            \end{itemize}
        \item \textbf{Examples of Ethical Dilemmas:}
            \begin{itemize}
                \item \textbf{Autonomous Vehicles:} Should a self-driving car prioritize the life of the passenger or pedestrians in a potential accident?
                \item \textbf{Data Privacy:} How can organizations balance the benefits of AI-driven insights with the need for individual privacy and informed consent?
            \end{itemize}
    \end{itemize}
\end{frame}

\begin{frame}[fragile]
    \frametitle{Discussion Points and Illustrative Example}
    \begin{itemize}
        \item \textbf{Key Points to Emphasize:}
            \begin{itemize}
                \item Interdisciplinary learning from technology, ethics, sociology, and law is essential to understand AI implications.
                \item AI systems often operate as "black boxes," complicating the traceability of their decision-making processes and leading to ethical challenges.
                \item Continuous evolution of AI technology necessitates ongoing study of case studies to adapt to emerging technologies and shifting ethical perspectives.
            \end{itemize}
        \item \textbf{Illustrative Example: AI in Healthcare}
            \begin{itemize}
                \item \textbf{Case Study:} IBM Watson for Oncology
                    \begin{itemize}
                        \item \textbf{Success:} Achieved high accuracy in identifying treatment options based on large datasets.
                        \item \textbf{Ethical Dilemma:} Concerns about reliance on AI over human judgment, data privacy issues, and transparency in decision-making processes.
                    \end{itemize}
            \end{itemize}
    \end{itemize}
\end{frame}

\begin{frame}[fragile]
    \frametitle{Learning Objectives}
    This week, we aim to deepen our understanding of \textbf{ethical dilemmas} related to AI and explore various \textbf{industry applications}. 
    The following key learning objectives will guide our study in this chapter:
\end{frame}

\begin{frame}[fragile]
    \frametitle{Learning Objectives - Part 1}
    \begin{enumerate}
        \item Identify and Analyze Ethical Dilemmas in AI
            \begin{itemize}
                \item \textbf{Concept Explanation}: Ethical dilemmas in AI arise when actions based on AI decisions conflict with moral principles.
                \item \textbf{Example}: The use of facial recognition technology by law enforcement raises concerns about privacy, consent, and potential racial bias.
                \item \textbf{Key Points to Emphasize}: 
                    \begin{itemize}
                        \item Understand stakeholder perspectives (e.g., individuals, communities, and organizations).
                        \item Explore frameworks for ethical decision-making (utilitarianism, deontology).
                    \end{itemize}
            \end{itemize}
    \end{enumerate}
\end{frame}

\begin{frame}[fragile]
    \frametitle{Learning Objectives - Part 2}
    \begin{enumerate}
        \setcounter{enumi}{1} % resume enumeration from the previous frame
        \item Understand Industry Applications of AI
            \begin{itemize}
                \item \textbf{Concept Explanation}: AI is transforming various industries, creating opportunities and challenges.
                \item \textbf{Example}: In healthcare, AI systems analyze medical images to detect diseases faster than human radiologists, resulting in life-saving early interventions.
                \item \textbf{Key Points to Emphasize}: 
                    \begin{itemize}
                        \item Recognize how AI applications improve efficiency and accuracy.
                        \item Consider industry-specific ethical concerns (e.g., patient data security in healthcare).
                    \end{itemize}
            \end{itemize}
    \end{enumerate}
\end{frame}

\begin{frame}[fragile]
    \frametitle{Learning Objectives - Part 3}
    \begin{enumerate}
        \setcounter{enumi}{2} % resume enumeration from the previous frame
        \item Evaluate the Impact of AI on Society
            \begin{itemize}
                \item \textbf{Concept Explanation}: Assessing the societal impact of AI helps identify both positive outcomes and potential drawbacks.
                \item \textbf{Example}: Automation in manufacturing increases productivity but may lead to job displacement for workers.
                \item \textbf{Key Points to Emphasize}:
                    \begin{itemize}
                        \item Analyze both short-term and long-term effects of AI deployment.
                        \item Discuss the balance between technological advancement and social responsibility.
                    \end{itemize}
            \end{itemize}
            
        \item Formulate Strategies for Ethical AI Development
            \begin{itemize}
                \item \textbf{Concept Explanation}: Developing strategies that prioritize ethics in AI initiatives is crucial for sustainable progress.
                \item \textbf{Example}: Implementing fairness audits in algorithm design ensures that AI systems do not perpetuate biases.
                \item \textbf{Key Points to Emphasize}: 
                    \begin{itemize}
                        \item Importance of interdisciplinary collaboration (ethicists, engineers, policymakers).
                        \item Practical steps: establish ethics committees, conduct impact assessments.
                    \end{itemize}
            \end{itemize}
    \end{enumerate}
\end{frame}

\begin{frame}[fragile]
    \frametitle{Learning Objectives - Summary}
    By the end of this chapter, students should be able to critically evaluate ethical dilemmas, comprehend AI applications across different sectors, assess their societal implications, and contribute to the development of responsible technologies. 
    Emphasizing these objectives will foster deeper discussions and a nuanced understanding of the intersection of AI and ethics.
\end{frame}

\begin{frame}[fragile]
    \frametitle{Case Study Selection Criteria - Overview}
    \begin{block}{Key Criteria for Selecting Case Studies in AI}
        Selecting case studies in the field of AI is crucial for comprehensive understanding and analysis. The main criteria to consider include:
        \begin{itemize}
            \item Relevance
            \item Industry Impact
            \item Ethical Implications
            \item Generalizability
            \item Available Data and Research Quality
        \end{itemize}
    \end{block}
\end{frame}

\begin{frame}[fragile]
    \frametitle{Case Study Selection Criteria - Relevance and Impact}
    \begin{enumerate}
        \item \textbf{Relevance}
        \begin{itemize}
            \item Definition: Must address current topics, technologies, or challenges in AI.
            \item Example: AI in autonomous vehicles due to public interest.
            \item Key Point: Align with course objectives and contemporary issues.
        \end{itemize}

        \item \textbf{Industry Impact}
        \begin{itemize}
            \item Definition: Evaluate significance and adoption of AI applications.
            \item Example: AI in supply chain optimization can reduce costs across sectors.
            \item Key Point: Choose studies showcasing tangible outcomes.
        \end{itemize}
    \end{enumerate}
\end{frame}

\begin{frame}[fragile]
    \frametitle{Case Study Selection Criteria - Ethics and Generalization}
    \begin{enumerate}
        \setcounter{enumi}{2}
        \item \textbf{Ethical Implications}
        \begin{itemize}
            \item Definition: Consider biases, privacy concerns, and societal impacts.
            \item Example: Facial recognition technology's benefits versus ethical drawbacks.
            \item Key Point: Ensure ethical considerations are embedded in development.
        \end{itemize}

        \item \textbf{Generalizability}
        \begin{itemize}
            \item Definition: The ability to apply findings to other contexts or sectors.
            \item Example: Insights from AI fraud detection relevant to both banking and e-commerce.
            \item Key Point: Select studies with broader themes for enhanced learning.
        \end{itemize}

        \item \textbf{Available Data and Research Quality}
        \begin{itemize}
            \item Definition: Availability of empirical data and research rigor.
            \item Example: Rich qualitative and quantitative data enhances effectiveness assessment.
            \item Key Point: Prioritize studies offering robust datasets and transparent methods.
        \end{itemize}
    \end{enumerate}
\end{frame}

\begin{frame}[fragile]
    \frametitle{Case Study Selection Criteria - Conclusion}
    Selecting the right case studies in AI is pivotal for understanding the technology's potential and challenges. By considering:
    \begin{itemize}
        \item Relevance
        \item Industry impact
        \item Ethical implications
        \item Generalizability
        \item Research quality
    \end{itemize}
    Students can engage in meaningful analysis and discussions.
\end{frame}

\begin{frame}[fragile]
    \frametitle{AI in Healthcare - Overview}
    \begin{block}{Overview of AI in Healthcare}
        Artificial Intelligence (AI) is increasingly becoming a vital component in the healthcare industry, significantly impacting diagnosis, treatment, and patient care. 
        This presentation will explore a case study that illustrates the transformative potential of AI in healthcare, along with its benefits, challenges, and ethical concerns.
    \end{block}
\end{frame}

\begin{frame}[fragile]
    \frametitle{AI in Healthcare - Case Study}
    \begin{block}{Case Study: AI-driven Diagnostics}
        \textbf{Example: IBM Watson for Oncology}
        \begin{itemize}
            \item IBM Watson for Oncology is an AI system that analyzes medical data to assist healthcare providers in diagnosing and recommending treatment plans for cancer patients.
            \item Utilizes natural language processing and machine learning to synthesize vast amounts of medical literature, clinical trial data, and patient records.
        \end{itemize}
    \end{block}
\end{frame}

\begin{frame}[fragile]
    \frametitle{AI in Healthcare - Benefits}
    \begin{block}{Benefits of AI in Healthcare}
        \begin{enumerate}
            \item \textbf{Enhanced Diagnostic Accuracy}
                \begin{itemize}
                    \item Watson evaluates patient data against millions of medical publications rapidly.
                    \item Research shows a 93\% concordance rate with expert oncologists.
                \end{itemize}
            \item \textbf{Timely Treatment Decisions}
                \begin{itemize}
                    \item AI processes information faster, reducing time from diagnosis to treatment.
                    \item Leads to better outcomes in critical conditions like cancer.
                \end{itemize}
            \item \textbf{Personalized Medicine}
                \begin{itemize}
                    \item Analyzes genetic information to tailor treatments based on individual tumor profiles.
                \end{itemize}
        \end{enumerate}
    \end{block}
\end{frame}

\begin{frame}[fragile]
    \frametitle{AI in Healthcare - Challenges}
    \begin{block}{Challenges of AI in Healthcare}
        \begin{enumerate}
            \item \textbf{Data Quality and Availability}
                \begin{itemize}
                    \item Effective models require large datasets; biased data can skew results.
                \end{itemize}
            \item \textbf{Integration into Clinical Workflow}
                \begin{itemize}
                    \item Resistance from medical staff and training needs complicate integration.
                \end{itemize}
            \item \textbf{High Costs}
                \begin{itemize}
                    \item Significant investment is needed for infrastructure and maintenance.
                \end{itemize}
        \end{enumerate}
    \end{block}
\end{frame}

\begin{frame}[fragile]
    \frametitle{AI in Healthcare - Ethical Concerns}
    \begin{block}{Ethical Concerns in AI}
        \begin{enumerate}
            \item \textbf{Patient Privacy}
                \begin{itemize}
                    \item Raises questions regarding data security and patient consent.
                \end{itemize}
            \item \textbf{Bias and Fairness}
                \begin{itemize}
                    \item AI may perpetuate biases if trained on non-representative data.
                \end{itemize}
            \item \textbf{Accountability}
                \begin{itemize}
                    \item Concerns about who is responsible for errors made by AI systems.
                \end{itemize}
        \end{enumerate}
    \end{block}
\end{frame}

\begin{frame}[fragile]
    \frametitle{AI in Healthcare - Conclusion}
    \begin{block}{Conclusion}
        AI stands to revolutionize healthcare through improved precision and personalization in patient care. 
        However, successfully navigating the associated challenges and ethical implications is critical for effective implementation and acceptance within the medical community.
    \end{block}
\end{frame}

\begin{frame}[fragile]
    \frametitle{AI in Finance - Introduction}
    \begin{block}{Overview}
        Artificial Intelligence (AI) significantly reshapes the finance sector by enhancing decision-making, optimizing efficiency, and mitigating risks. This presentation focuses on:
        \begin{itemize}
            \item Risk Assessment
            \item Model Transparency
            \item Ethical Dilemmas
        \end{itemize}
    \end{block}
\end{frame}

\begin{frame}[fragile]
    \frametitle{AI in Finance - Risk Assessment}
    \begin{block}{Definition}
        Risk assessment involves identifying and analyzing factors that could negatively affect an asset, investment, or project.
    \end{block}
    
    \begin{block}{AI Applications}
        \begin{itemize}
            \item \textbf{Credit Scoring:} Analyzing data points to predict creditworthiness.
            \item \textbf{Fraud Detection:} Identifying unusual patterns in real-time.
        \end{itemize}
    \end{block}
    
    \begin{block}{Example}
        A bank implemented a machine learning model that reduced false positives in fraud detection by 30\%, allowing for faster processing of legitimate transactions.
    \end{block}
\end{frame}

\begin{frame}[fragile]
    \frametitle{AI in Finance - Model Transparency and Ethical Dilemmas}
    \begin{block}{Model Transparency}
        Transparency is crucial for trust and accountability. Key aspects include:
        \begin{itemize}
            \item \textbf{Explainability:} Understanding model decision processes (e.g., using LIME).
            \item \textbf{Regulatory Compliance:} Adhering to regulations like GDPR for explainability.
        \end{itemize}
        \begin{itemize}
            \item \textbf{Example:} A firm that used an explainable AI model to gain regulatory approval while maintaining customer trust.
        \end{itemize}
    \end{block}

    \begin{block}{Ethical Dilemmas}
        Considerations include:
        \begin{itemize}
            \item \textbf{Bias in Algorithms:} Risk of unfair treatment based on training data biases.
            \item \textbf{Data Privacy Concerns:} Issues regarding consent and misuse of personal data.
        \end{itemize}
        \begin{itemize}
            \item \textbf{Ethical Dilemma:} A lending platform faced backlash for discrimination against certain demographics.
        \end{itemize}
    \end{block}
\end{frame}

\begin{frame}[fragile]
    \frametitle{AI in Autonomous Vehicles}
    In this presentation, we will review a case study on autonomous vehicles, exploring safety, accountability, and ethical questions surrounding AI decision-making.
\end{frame}

\begin{frame}[fragile]
    \frametitle{Overview of Autonomous Vehicles}
    \begin{block}{Definition}
        Autonomous vehicles (AVs), also known as self-driving cars, are equipped with AI technology to navigate and drive without human intervention.
    \end{block}
    
    \begin{itemize}
        \item **Components**: AVs rely on a combination of sensors (cameras, LiDAR), machine learning, and data analysis.
        \item **Functionality**: They operate safely in real-time using complex algorithms.
    \end{itemize}
\end{frame}

\begin{frame}[fragile]
    \frametitle{Key Concepts - Safety}
    \begin{itemize}
        \item **Importance**: Primary concern is the safety of passengers, pedestrians, and other vehicles.
        \item **Case Study**: The Uber self-driving car incident in 2018 raised questions about safety measures.
        \item **AI Training**: AVs are trained with vast datasets to recognize hazards but face challenges with unexpected scenarios (e.g., a child running into the street).
    \end{itemize}
\end{frame}

\begin{frame}[fragile]
    \frametitle{Key Concepts - Accountability}
    \begin{itemize}
        \item **Responsibility**: Questions arise regarding liability when AVs are involved in accidents (manufacturer, software developer, car owner).
        \item **Legal Framework**: Current laws are unclear about accountability for AV-related incidents.
        \item **Determining Blame**: Essential to establish whether the fault is with the programming or human oversight in cases of malfunction.
    \end{itemize}
\end{frame}

\begin{frame}[fragile]
    \frametitle{Key Concepts - Ethical Questions}
    \begin{itemize}
        \item **Decision-Making**: AVs must weigh outcomes during critical situations (e.g., avoiding pedestrians vs. passenger safety).
        \item **The Trolley Problem**: A dilemma illustrating the programming of moral choices into AI.
        \item **Public Acceptance**: Ethical issues significantly influence public perception of AV technology.
    \end{itemize}
\end{frame}

\begin{frame}[fragile]
    \frametitle{Key Points to Emphasize}
    \begin{itemize}
        \item **Safety and Ethics are intertwined**: AVs must prioritize safety while navigating ethical dilemmas.
        \item **Need for Clear Accountability**: Laws governing liability in AV accidents are necessary.
        \item **Public Trust is Essential**: Transparency and ethical programming are vital for public confidence in AV technology.
    \end{itemize}
\end{frame}

\begin{frame}[fragile]
    \frametitle{AI Decision-Making Process in AVs}
    \begin{block}{Diagram}
        Input Data (Sensors) $\rightarrow$ Perception (Environment Detection) $\rightarrow$ Decision-Making (AI Algorithms) $\rightarrow$ Action (Vehicle Control)
    \end{block}
\end{frame}

\begin{frame}[fragile]
    \frametitle{Conclusion}
    \begin{itemize}
        \item The integration of AI in autonomous vehicles brings advancements and also crucial questions involving safety, accountability, and ethics.
        \item Addressing these issues is essential for the responsible deployment of AVs in society.
    \end{itemize}
\end{frame}

\begin{frame}[fragile]
    \frametitle{Ethical Frameworks in AI - Introduction}
    \begin{block}{Introduction to Ethical Frameworks}
        As AI technologies increasingly influence decision-making across various fields, understanding and implementing ethical frameworks becomes essential. These frameworks help evaluate the implications of AI applications, ensuring they align with societal values and ethical standards.
    \end{block}
    
    \begin{itemize}
        \item \textbf{Trustworthiness}: Foster trust in AI technologies by ensuring responsible use.
        \item \textbf{Accountability}: Clear guidelines help determine accountability for AI decisions.
        \item \textbf{Risk Mitigation}: Frameworks identify and address risks associated with AI systems.
    \end{itemize}
\end{frame}

\begin{frame}[fragile]
    \frametitle{Ethical Frameworks in AI - Key Ethical Frameworks}
    \begin{enumerate}
        \item \textbf{IEEE Global Initiative for Ethical Considerations}
            \begin{itemize}
                \item \textbf{Overview}: An initiative by IEEE aimed at ensuring AI is designed to uplift humanity.
                \item \textbf{Principles}:
                    \begin{itemize}
                        \item Transparency
                        \item Fairness
                        \item Accountability
                        \item Privacy
                    \end{itemize}
            \end{itemize}
        
        \item \textbf{Asilomar AI Principles}
            \begin{itemize}
                \item \textbf{Focus}: Promoting AI's societal impact since 2017.
                \item \textbf{Key Points}:
                    \begin{itemize}
                        \item Safety
                        \item Value Alignment
                    \end{itemize}
            \end{itemize}
        
        \item \textbf{European Commission Guidelines}
            \begin{itemize}
                \item \textbf{Purpose}: Fosters trust in AI through ethical and robust frameworks.
                \item \textbf{Key Elements}:
                    \begin{itemize}
                        \item Human Agency and Oversight
                        \item Technical Robustness and Safety
                    \end{itemize}
            \end{itemize}
    \end{enumerate}
\end{frame}

\begin{frame}[fragile]
    \frametitle{Ethical Frameworks in AI - Application & Conclusion}
    \begin{block}{Application of Ethical Frameworks}
        \textbf{Example: Autonomous Vehicles}
        \begin{itemize}
            \item Ethical frameworks can guide decision-making during emergencies.
            \item Guidelines can protect user data collection practices.
        \end{itemize}
    \end{block}

    \begin{block}{Conclusion}
        Understanding and implementing ethical frameworks in AI applications is critical for developing technologies that are responsible and aligned with human values. 
        \begin{itemize}
            \item Ethical frameworks ensure trust, accountability, and risk mitigation.
            \item Familiarity with these frameworks contributes to a future where AI enhances societal values.
        \end{itemize}
    \end{block}
\end{frame}

\begin{frame}[fragile]
    \frametitle{Common Ethical Dilemmas}
    
    \begin{block}{Introduction to Ethical Dilemmas in AI}
        As Artificial Intelligence (AI) systems become increasingly pervasive in various sectors, they bring about significant ethical concerns. 
        It’s crucial to evaluate these dilemmas thoughtfully to harness AI responsibly and effectively.
    \end{block}
\end{frame}

\begin{frame}[fragile]
    \frametitle{1. Bias in AI}
    
    \begin{block}{Explanation}
        AI systems learn from historical data. If that data embodies societal biases, the algorithms can perpetuate or even amplify those biases.
    \end{block}

    \begin{itemize}
        \item \textbf{Types of Bias:}
            \begin{itemize}
                \item \textbf{Data Bias:} Originates from unrepresentative training datasets.
                \item \textbf{Algorithmic Bias:} Arises from the algorithms themselves, misinterpreting learned data.
            \end{itemize}
    \end{itemize}

    \begin{exampleblock}{Example}
        In 2018, an AI recruiting tool developed by Amazon was scrapped after it was found to favor male candidates, 
        illustrating the harmful impacts of bias in hiring algorithms.
    \end{exampleblock}
\end{frame}

\begin{frame}[fragile]
    \frametitle{2. Privacy Concerns}
    
    \begin{block}{Explanation}
        AI applications often require vast amounts of personal data to function effectively. 
        The collection, storage, and analysis of this data raise significant privacy concerns.
    \end{block}

    \begin{itemize}
        \item \textbf{Key Points:}
            \begin{itemize}
                \item \textbf{Data Collection:} Users often unknowingly consent to extensive data harvesting.
                \item \textbf{Data Usage:} Responsible usage requires robust privacy protections.
            \end{itemize}
    \end{itemize}

    \begin{exampleblock}{Example}
        Smart home devices, like voice assistants, collect data on user habits and preferences. 
        Without strict privacy regulations, this data could be misused or exposed in breaches.
    \end{exampleblock}
\end{frame}

\begin{frame}[fragile]
    \frametitle{3. Transparency and Accountability}
    
    \begin{block}{Explanation}
        Many AI systems operate as "black boxes," making it difficult to perceive how decisions are made. 
        This poses challenges for accountability.
    \end{block}

    \begin{itemize}
        \item \textbf{Key Points:}
            \begin{itemize}
                \item \textbf{Transparency:} Users and stakeholders should understand AI decision-making processes.
                \item \textbf{Accountability:} Clear guidelines must be established regarding responsibility for harmful AI decisions.
            \end{itemize}
    \end{itemize}

    \begin{exampleblock}{Example}
        In healthcare, if an AI system inaccurately predicts a patient's risk for a disease, 
        understanding how the decision was made is essential for correcting errors and ensuring patient safety.
    \end{exampleblock}
\end{frame}

\begin{frame}[fragile]
    \frametitle{Conclusion}
    
    Addressing ethical dilemmas in AI is crucial for responsible innovation. A proactive approach is necessary to:
    \begin{itemize}
        \item Mitigate bias
        \item Protect privacy
        \item Ensure AI systems are transparent and accountable
    \end{itemize}

    This understanding is essential as we explore real-world case studies in the following sections. By engaging with these concepts, you’ll be equipped to critically analyze AI applications and contribute to discussions on ethical practices in technology development.
\end{frame}

\begin{frame}[fragile]
    \frametitle{Group Case Study Presentations - Overview}
    The Group Case Study Presentations are a culminating activity where students apply their understanding of ethical dilemmas in AI through real-world scenario analyses. The process consists of:

    \begin{enumerate}
        \item \textbf{Group Formation}: Students are divided into small groups (3-5 members) to encourage diverse perspectives.
        \item \textbf{Case Study Selection}: Each group selects an AI case study for analysis, with topics including:
        \begin{itemize}
            \item AI in healthcare (e.g., diagnostic tools)
            \item AI in hiring processes (e.g., resume screening)
            \item AI in law enforcement (e.g., predictive policing)
        \end{itemize}
        \item \textbf{Research and Analysis}: Groups conduct comprehensive research focusing on background, ethical dilemmas, stakeholders, and impact.
    \end{enumerate}
\end{frame}

\begin{frame}[fragile]
    \frametitle{Group Case Study Presentations - Preparation}
    Following the research phase, groups will focus on preparation and delivery:

    \begin{enumerate}
        \setcounter{enumi}{3}
        \item \textbf{Presentation Preparation}: Create a 15-20 minute presentation including:
        \begin{itemize}
            \item Introduction to the case study
            \item Ethical dilemmas identified
            \item Proposed solutions or recommendations
            \item Audience engagement through Q\&A
        \end{itemize}
        \item \textbf{Rehearsal}: Practice is essential to ensure smooth delivery, clarity, and timing for the actual presentation.
    \end{enumerate}
\end{frame}

\begin{frame}[fragile]
    \frametitle{Expected Outcomes and Evaluation Criteria}
    \textbf{Expected Outcomes}:
    By the end of the presentations, students should be able to:
    \begin{itemize}
        \item Articulate ethical challenges in AI use cases
        \item Engage in critical thinking about societal impacts
        \item Demonstrate collaborative communication skills
    \end{itemize}

    \textbf{Evaluation Criteria}:
    Presentations will be assessed based on:
    \begin{itemize}
        \item Content Knowledge (30\%)
        \item Clarity and Organization (25\%)
        \item Engagement \& Team Collaboration (20\%)
        \item Visual Aids and Delivery (15\%)
        \item Reflection on Ethical Considerations (10\%)
    \end{itemize}
\end{frame}

\begin{frame}[fragile]
    \frametitle{Conclusion and Reflection - Insights Gained}
    \begin{enumerate}
        \item \textbf{Understanding AI Applications} 
            \begin{itemize}
                \item Case studies illustrate varied applications of AI across industries (healthcare, finance, transportation).
                \item Example: In healthcare, AI assists in diagnosing diseases and optimizing treatment plans, enhancing patient care.
            \end{itemize}
        \item \textbf{Real-World Impact} 
            \begin{itemize}
                \item AI implementations have transformed industries, shaping economic structures and employment patterns.
                \item Example: AI-driven chatbots improve customer service efficiency, changing traditional workforce dynamics.
            \end{itemize}
    \end{enumerate}
\end{frame}

\begin{frame}[fragile]
    \frametitle{Conclusion and Reflection - Ethical Considerations}
    \begin{enumerate}
        \item \textbf{Definition of Ethics in AI}
            \begin{itemize}
                \item Focuses on fairness, accountability, transparency, and the societal impact of automated decisions.
            \end{itemize}
        \item \textbf{Importance of Ethical AI}
            \begin{itemize}
                \item Promotes responsible innovation prioritizing human welfare and mitigates harm.
                \item Encourages public trust in AI systems, crucial for widespread adoption.
            \end{itemize}
        \item \textbf{Key Ethical Issues}
            \begin{itemize}
                \item \textit{Bias and Discrimination}: Algorithms may perpetuate existing biases.
                \item \textit{Privacy Concerns}: Data collection can infringe on individual privacy rights.
            \end{itemize}
    \end{enumerate}
\end{frame}

\begin{frame}[fragile]
    \frametitle{Conclusion and Reflection - Frameworks and Key Points}
    \begin{enumerate}
        \item \textbf{Frameworks for Ethical AI}
            \begin{itemize}
                \item Organizations develop ethical guidelines (e.g., IEEE Ethically Aligned Design, EU AI Act) for responsible AI.
            \end{itemize}
        \item \textbf{Key Points to Emphasize}
            \begin{itemize}
                \item \textbf{Reflective Practice}: Reflect on ethical implications after each case study.
                \item \textbf{Collaborative Responsibility}: Ethics is a shared duty among developers, policymakers, and the public.
            \end{itemize}
        \item \textbf{Conclusion}
            \begin{itemize}
                \item Case studies show the significance of ethical considerations in AI development for equitable innovation.
            \end{itemize}
    \end{enumerate}
\end{frame}


\end{document}