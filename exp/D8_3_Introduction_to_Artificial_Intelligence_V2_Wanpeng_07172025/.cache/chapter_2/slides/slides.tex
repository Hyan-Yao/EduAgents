\documentclass[aspectratio=169]{beamer}

% Theme and Color Setup
\usetheme{Madrid}
\usecolortheme{whale}
\useinnertheme{rectangles}
\useoutertheme{miniframes}

% Additional Packages
\usepackage[utf8]{inputenc}
\usepackage[T1]{fontenc}
\usepackage{graphicx}
\usepackage{booktabs}
\usepackage{listings}
\usepackage{amsmath}
\usepackage{amssymb}
\usepackage{xcolor}
\usepackage{tikz}
\usepackage{pgfplots}
\pgfplotsset{compat=1.18}
\usetikzlibrary{positioning}
\usepackage{hyperref}

% Custom Colors
\definecolor{myblue}{RGB}{31, 73, 125}
\definecolor{mygray}{RGB}{100, 100, 100}
\definecolor{mygreen}{RGB}{0, 128, 0}
\definecolor{myorange}{RGB}{230, 126, 34}
\definecolor{mycodebackground}{RGB}{245, 245, 245}

% Set Theme Colors
\setbeamercolor{structure}{fg=myblue}
\setbeamercolor{frametitle}{fg=white, bg=myblue}
\setbeamercolor{title}{fg=myblue}
\setbeamercolor{section in toc}{fg=myblue}

% Set Fonts
\setbeamerfont{title}{size=\Large, series=\bfseries}
\setbeamerfont{frametitle}{size=\large, series=\bfseries}
\setbeamerfont{caption}{size=\small}
\setbeamerfont{footnote}{size=\tiny}

% Footer and Navigation Setup
\setbeamertemplate{footline}{
  \leavevmode%
  \hbox{%
  \begin{beamercolorbox}[wd=.3\paperwidth,ht=2.25ex,dp=1ex,center]{author in head/foot}%
    \usebeamerfont{author in head/foot}\insertshortauthor
  \end{beamercolorbox}%
  \begin{beamercolorbox}[wd=.5\paperwidth,ht=2.25ex,dp=1ex,center]{title in head/foot}%
    \usebeamerfont{title in head/foot}\insertshorttitle
  \end{beamercolorbox}%
  \begin{beamercolorbox}[wd=.2\paperwidth,ht=2.25ex,dp=1ex,center]{date in head/foot}%
    \usebeamerfont{date in head/foot}
    \insertframenumber{} / \inserttotalframenumber
  \end{beamercolorbox}}%
  \vskip0pt%
}

% Document Start
\begin{document}

\frame{\titlepage}

\begin{frame}[fragile]
    \titlepage
\end{frame}

\begin{frame}[fragile]
    \frametitle{Introduction to AI, Machine Learning, and Deep Learning}

    \begin{block}{Overview}
        This segment covers:
        \begin{itemize}
            \item Understanding Artificial Intelligence (AI)
            \item Key distinctions between AI, Machine Learning (ML), and Deep Learning (DL)
            \item Learning objectives for the week
        \end{itemize}
    \end{block}
\end{frame}

\begin{frame}[fragile]
    \frametitle{Understanding Artificial Intelligence (AI)}

    \begin{block}{Definition}
        Artificial Intelligence (AI) refers to the simulation of human intelligence processes by machines, particularly computer systems. This includes:
        \begin{itemize}
            \item Learning (acquisition of information)
            \item Reasoning (using rules to reach conclusions)
            \item Self-correction
        \end{itemize}
    \end{block}

    \begin{block}{Significance}
        \begin{itemize}
            \item Transforms industries by automating processes
            \item Enhances decision-making and improves efficiency
            \item Applications include healthcare, finance, and transportation
        \end{itemize}
    \end{block}
\end{frame}

\begin{frame}[fragile]
    \frametitle{Distinguishing Key Concepts}

    \begin{block}{Machine Learning (ML)}
        \begin{itemize}
            \item \textbf{Definition:} A subset of AI focusing on developing algorithms allowing computers to learn from data.
            \item \textbf{Example:} Email filtering systems distinguishing between spam and legitimate emails.
        \end{itemize}
    \end{block}

    \begin{block}{Deep Learning (DL)}
        \begin{itemize}
            \item \textbf{Definition:} A subfield of ML that uses deep neural networks to analyze data at multiple levels of abstraction.
            \item \textbf{Example:} Image recognition technologies identifying objects through layers of neurons.
        \end{itemize}
    \end{block}
\end{frame}

\begin{frame}[fragile]
    \frametitle{Key Points to Emphasize}

    \begin{itemize}
        \item \textbf{Hierarchy of Concepts:}
        \begin{itemize}
            \item AI encompasses both ML and DL.
            \item ML serves as a bridge between traditional programming and advanced AI.
            \item DL requires significant computational resources and large datasets.
        \end{itemize}

        \item \textbf{Real-world Applications:}
        \begin{itemize}
            \item Natural Language Processing (NLP) in virtual assistants (e.g., Siri, Alexa).
            \item Autonomous vehicles utilizing DL for visual recognition.
        \end{itemize}
    \end{itemize}
\end{frame}

\begin{frame}[fragile]
    \frametitle{Learning Objectives for the Week}

    \begin{itemize}
        \item Differentiate clearly between AI, ML, and DL.
        \item Understand applications and implications in various industries.
        \item Explore ethical considerations in AI and ML decision-making processes.
    \end{itemize}
\end{frame}

\begin{frame}[fragile]
    \frametitle{Illustration: Hierarchical Representation of AI Concepts}

    \begin{center}
    \texttt{AI}\\
    \hspace{1cm} $\downarrow$\\
    \texttt{Machine Learning}\\
    \hspace{1cm} $\downarrow$\\
    \texttt{Deep Learning}
    \end{center}
\end{frame}

\begin{frame}[fragile]
    \frametitle{Sample Code for Machine Learning: Linear Regression}

    \begin{lstlisting}[language=Python]
# Sample ML Algorithm: Linear Regression in Python
from sklearn.linear_model import LinearRegression
model = LinearRegression()
model.fit(X_train, y_train)  # Train the model
predictions = model.predict(X_test)  # Make predictions
    \end{lstlisting}
\end{frame}

\begin{frame}[fragile]{Definitions and Distinctions - Part 1}
    \begin{block}{Artificial Intelligence (AI)}
        \textbf{Definition:} AI refers to the simulation of human intelligence processes by machines, especially computer systems. It encompasses reasoning, learning, and self-correction.  
    \end{block}

    \begin{itemize}
        \item \textbf{Key Concepts:}
        \begin{itemize}
            \item \textbf{Narrow AI:} AI systems designed to perform a narrow task (e.g., facial recognition, internet searches).
            \item \textbf{General AI:} Hypothetical AI that possesses the ability to perform any intellectual task that a human can do.
        \end{itemize}
        
        \item \textbf{Example:} Voice assistants like Siri and Alexa are examples of narrow AI that can process natural language commands and perform specific tasks.
    \end{itemize}

    \begin{block}{Visual Aid}
        [Diagram illustrating Narrow AI vs. General AI, labeled with examples.]
    \end{block}
\end{frame}

\begin{frame}[fragile]{Definitions and Distinctions - Part 2}
    \begin{block}{Machine Learning (ML)}
        \textbf{Definition:} ML is a subset of AI that enables systems to learn from data and improve their performance over time without being explicitly programmed.
    \end{block}

    \begin{itemize}
        \item \textbf{Key Concepts:}
        \begin{itemize}
            \item \textbf{Supervised Learning:} The model is trained on labeled data (e.g., predicting house prices based on features).
            \item \textbf{Unsupervised Learning:} The model identifies patterns in unlabeled data (e.g., clustering customers based on purchasing behavior).
            \item \textbf{Reinforcement Learning:} Learning through trial and error to achieve a goal (e.g., training a robot to navigate a maze).
        \end{itemize}

        \item \textbf{Example:} A spam filter uses supervised ML to classify emails as "spam" or "not spam" based on features derived from labeled examples.
    \end{itemize}

    \begin{block}{Visual Aid}
        [Flowchart depicting supervised vs. unsupervised learning.]
    \end{block}
\end{frame}

\begin{frame}[fragile]{Definitions and Distinctions - Part 3}
    \begin{block}{Deep Learning (DL)}
        \textbf{Definition:} DL is a specialized subset of ML that utilizes neural networks with multiple layers (deep architectures) to analyze and learn from large amounts of data.
    \end{block}

    \begin{itemize}
        \item \textbf{Key Concepts:}
        \begin{itemize}
            \item \textbf{Neural Networks:} Computational models inspired by human brain networks, consisting of interconnected nodes (neurons) designed to recognize patterns.
            \item \textbf{Convolutional Neural Networks (CNNs):} Primarily used for image processing tasks by automatically detecting features.
            \item \textbf{Recurrent Neural Networks (RNNs):} Suitable for sequential data, such as time series or natural language.
        \end{itemize}

        \item \textbf{Example:} Image recognition software, which distinguishes different objects in photos, usually relies on convolutional neural networks.
    \end{itemize}

    \begin{block}{Visual Aid}
        [Diagram of a simple neural network architecture, highlighting layers and connections.]
    \end{block}
\end{frame}

\begin{frame}[fragile]
    \frametitle{Key Concepts in AI - Algorithms}
    \begin{itemize}
        \item \textbf{Definition}: An algorithm is a set of rules or instructions for solving a problem or performing a task. In AI, algorithms drive the decision-making processes.
        \item \textbf{Example}: 
        \begin{itemize}
            \item \textit{Decision Trees}: Algorithms that split data into branches to represent decisions and their possible consequences in a tree-like structure. For instance, they can be used for predicting customer churn by analyzing behaviors and traits of customers.
        \end{itemize}
    \end{itemize}
\end{frame}

\begin{frame}[fragile]
    \frametitle{Key Concepts in AI - Data Processing}
    \begin{itemize}
        \item \textbf{Definition}: Refers to the collection, transformation, and analysis of raw data into a meaningful format usable by AI algorithms.
        \item \textbf{Key Steps}:
        \begin{itemize}
            \item \textit{Data Collection}: Gathering data from various sources, such as databases, APIs, or user inputs.
            \item \textit{Data Cleaning}: Removing inaccuracies or inconsistencies to ensure high-quality input.
            \item \textit{Feature Extraction}: Selecting and transforming variables (features) that will be used in model training.
        \end{itemize}
        \item \textbf{Example}: In image recognition, raw pixel data is processed to highlight features like edges, colors, and shapes that enhance recognition accuracy.
    \end{itemize}
\end{frame}

\begin{frame}[fragile]
    \frametitle{Key Concepts in AI - Model Training}
    \begin{itemize}
        \item \textbf{Definition}: The process of teaching an AI model to make predictions or classifications based on historical data by adjusting the model's internal parameters.
        \item \textbf{Procedure}:
        \begin{itemize}
            \item \textit{Supervised Learning}: Models trained using labeled data (e.g., training a spam filter using emails marked as "spam" or "not spam").
            \item \textit{Unsupervised Learning}: Models uncover patterns in unlabeled data (e.g., clustering customers into segments based on purchasing behavior).
        \end{itemize}
        \item \textbf{Example}: A neural network for image classification learns distinct features of cats and dogs from thousands of labeled images using backpropagation.
    \end{itemize}
\end{frame}

\begin{frame}[fragile]
    \frametitle{Key Concepts in AI - Key Points and Illustration}
    \begin{itemize}
        \item \textbf{Interconnected Concepts}: Algorithms rely on processed data to train models; data quality directly impacts model efficacy.
        \item \textbf{Real-world Applications}: Technologies like recommendation systems (e.g., Netflix, Amazon) and autonomous vehicles demonstrate effective application of these concepts.
        \item \textbf{Continuous Improvement}: Model training is iterative; model performance is constantly evaluated and refined.
    \end{itemize}
    
    \begin{block}{Illustration: Basic Algorithm Flow}
    \begin{verbatim}
        Start
          |
        Gather Data
          |
        Clean Data  --> Feature Extraction
          |              
      Choose Algorithm
          |
      Train Model  (Supervised/Unsupervised)
          |
    Evaluate Model Performance
          |
        End
    \end{verbatim}
    \end{block}

    \textit{Understanding these core concepts is foundational to working effectively with AI technologies and will prepare us for discussing ethical considerations in AI.}
\end{frame}

\begin{frame}[fragile]
    \frametitle{Ethical Considerations in AI - Introduction}
    As AI technologies advance and become integrated into various aspects of society, 
    ethical considerations become increasingly crucial. The effectiveness and acceptance 
    of AI depend not just on technological capabilities but also on the ethical deployment 
    and governance of these systems. 
    \begin{itemize}
        \item Fundamental ethical principles:
        \begin{itemize}
            \item Fairness
            \item Accountability
            \item Transparency
        \end{itemize}
    \end{itemize}
\end{frame}

\begin{frame}[fragile]
    \frametitle{Ethical Considerations in AI - Fairness}
    \begin{block}{Fairness}
        \begin{itemize}
            \item \textbf{Definition}: Fairness in AI refers to the principle that AI systems should not favor or discriminate against any group or individual.
            \item \textbf{Example}: In recruiting algorithms, if the AI is trained on historical data reflecting biases (e.g., hiring predominantly male candidates), it may unintentionally perpetuate these biases in future hiring practices.
            \item \textbf{Key Point}: Ensuring fairness requires rigorous testing and an understanding of societal biases that may be embedded in training datasets.
        \end{itemize}
    \end{block}
\end{frame}

\begin{frame}[fragile]
    \frametitle{Ethical Considerations in AI - Accountability and Transparency}
    \begin{block}{Accountability}
        \begin{itemize}
            \item \textbf{Definition}: Accountability relates to the responsibility of AI developers and deployers to ensure their systems operate ethically and comply with legal standards.
            \item \textbf{Example}: If an autonomous vehicle is involved in an accident, determining who is accountable – the manufacturer, software developer, or the user – initiates complex ethical and legal discussions.
            \item \textbf{Key Point}: Clear accountability frameworks must be established to ensure someone can be held responsible for AI decisions, particularly in high-stakes situations.
        \end{itemize}
    \end{block}

    \begin{block}{Transparency}
        \begin{itemize}
            \item \textbf{Definition}: Transparency involves making the processes and decision-making paths of AI systems understandable and accessible to users and stakeholders.
            \item \textbf{Example}: In healthcare, if an algorithm suggests a treatment plan, patients and doctors should understand how decisions were reached, including the data used and the rationale behind recommendations.
            \item \textbf{Key Point}: Enhancing transparency fosters trust among users and helps in identifying and correcting errors or biases within AI systems.
        \end{itemize}
    \end{block}
\end{frame}

\begin{frame}[fragile]
    \frametitle{Frameworks for Ethical Analysis - Overview}
    As AI technologies evolve, ethical considerations are crucial. Several established frameworks guide ethical analysis, aiding stakeholders in navigating complex moral dilemmas.

    \begin{itemize}
        \item Importance of ethical frameworks in AI design and implementation
        \item Considerations include human rights, transparency, and accountability
    \end{itemize}
\end{frame}

\begin{frame}[fragile]
    \frametitle{Key Ethical Frameworks}
    \begin{enumerate}
        \item **IEEE Global Initiative for Ethical Considerations**
            \begin{itemize}
                \item \textbf{Purpose:} Establish ethical standards for AI design.
                \item \textbf{Key Principles:}
                    \begin{itemize}
                        \item Human Rights
                        \item Transparency
                        \item Accountability
                    \end{itemize}
            \end{itemize}
        \item **EU AI Ethics Guidelines**
            \begin{itemize}
                \item \textbf{Core Values:}
                    \begin{itemize}
                        \item Respect for Human Autonomy
                        \item Prevention of Harm
                        \item Privacy and Data Governance
                    \end{itemize}
            \end{itemize}
        \item **OECD Principles on AI**
            \begin{itemize}
                \item \textbf{Guidelines Include:}
                    \begin{itemize}
                        \item Inclusive Growth
                        \item Sustainable Development
                        \item Robustness and Safety
                    \end{itemize}
            \end{itemize}
    \end{enumerate}
\end{frame}

\begin{frame}[fragile]
    \frametitle{Practical Examples and Conclusions}
    \begin{itemize}
        \item **Self-driving Cars:** Ethical frameworks help inform decision-making in critical scenarios.
        \item **Facial Recognition Technology:** Guidelines address bias and privacy risks.
    \end{itemize}
    
    \begin{block}{Key Points}
        \begin{itemize}
            \item Ethical frameworks guide responsible AI development.
            \item Combining insights from various frameworks can lead to better solutions.
            \item Stakeholder engagement is crucial for viable frameworks.
        \end{itemize}
    \end{block}
    
    The understanding of these frameworks empowers students to assess and engage with ethical challenges in AI.
\end{frame}

\begin{frame}[fragile]
    \frametitle{Case Studies on Ethical Dilemmas in AI}
    \begin{block}{Introduction}
        Ethical dilemmas in AI arise when the deployment of technology conflicts with moral principles or societal values. Understanding these dilemmas requires employing ethical frameworks to analyze the implications of AI technologies on individuals and communities.
    \end{block}
\end{frame}

\begin{frame}[fragile]
    \frametitle{Case Study 1: Facial Recognition Technology}
    
    \begin{itemize}
        \item \textbf{Scenario:} A city employs facial recognition technology for surveillance to enhance public safety, but it disproportionately misidentifies people of color, leading to wrongful accusations.
        
        \item \textbf{Ethical Dilemma:}
        \begin{itemize}
            \item Privacy vs. Safety
            \item Bias and Fairness
        \end{itemize}
        
        \item \textbf{Ethical Framework Analysis:}
        \begin{enumerate}
            \item Utilitarianism: Weighing overall safety benefits against negative impacts on individuals.
            \item Deontological Ethics: Violations of privacy and fairness regardless of technology benefits.
        \end{enumerate}
        
        \item \textbf{Key Points:}
        \begin{itemize}
            \item Addressing bias in AI systems is crucial for fairness and justice.
            \item Ethical considerations must accompany technological advancements.
        \end{itemize}
    \end{itemize}
\end{frame}

\begin{frame}[fragile]
    \frametitle{Case Study 2: Autonomous Vehicles}
    
    \begin{itemize}
        \item \textbf{Scenario:} An AV faces an accident decision: swerving to hit a pedestrian versus harming its passengers.
        
        \item \textbf{Ethical Dilemma:}
        \begin{itemize}
            \item Consequentialism vs. Moral Absolutism
            \item Responsibility and Accountability
        \end{itemize}
        
        \item \textbf{Ethical Framework Analysis:}
        \begin{enumerate}
            \item Virtue Ethics: Emphasizing moral character in design and decision-making.
            \item Social Contract Theory: Meeting societal expectations and safety responsibilities.
        \end{enumerate}
        
        \item \textbf{Key Points:}
        \begin{itemize}
            \item Ethical AI must align with human-centered values.
            \item Engaging in ethical programming discussions is essential for societal trust.
        \end{itemize}
    \end{itemize}
\end{frame}

\begin{frame}[fragile]
    \frametitle{Concluding Thoughts and Q\&A - Summary of Key Points}
    \begin{enumerate}
        \item \textbf{Understanding AI and Its Implications}:
        \begin{itemize}
            \item Integration of AI in various sectors raises ethical responsibilities.
            \item Responsible AI development enhances societal well-being.
        \end{itemize}
        
        \item \textbf{Ethical Frameworks Discussed}:
        \begin{itemize}
            \item Utilitarianism
            \item Deontological Ethics
            \item Virtue Ethics
        \end{itemize}
        
        \item \textbf{Case Study Insights}:
        \begin{itemize}
            \item Contemporary case studies reveal ethical dilemmas in AI deployment.
        \end{itemize}

        \item \textbf{Role of Stakeholders}:
        \begin{itemize}
            \item Engagement among technologists, ethicists, policymakers, and the public is essential.
        \end{itemize}

        \item \textbf{Future Directions}:
        \begin{itemize}
            \item Continuous evaluation and adaptation of ethical AI.
            \item Importance of regulatory frameworks.
        \end{itemize}
    \end{enumerate}
\end{frame}

\begin{frame}[fragile]
    \frametitle{Concluding Thoughts and Q\&A - Key Points to Emphasize}
    \begin{itemize}
        \item Ethics in AI is a collective responsibility across diverse stakeholders.
        \item Ethical frameworks help analyze dilemmas but require nuanced interpretations due to complex real-world scenarios.
    \end{itemize}
\end{frame}

\begin{frame}[fragile]
    \frametitle{Concluding Thoughts and Q\&A - Engagement and Discussion}
    \begin{block}{Q\&A Session}
        \begin{itemize}
            \item Open floor for questions and reflections.
            \item Encourage discussion on case studies and ethical frameworks.
            \item Invite queries on practical applications in various fields.
        \end{itemize}
    \end{block}
    
    \begin{block}{Decision-Making Formula}
        \begin{enumerate}
            \item Identify the ethical dilemma.
            \item Analyze using ethical frameworks:
            \begin{itemize}
                \item Utilitarian: What are the potential outcomes?
                \item Deontological: What obligations do we have?
                \item Virtue: What virtues should guide our decision?
            \end{itemize}
            \item Make an informed decision and adapt as new information arises.
        \end{enumerate}
    \end{block}
\end{frame}


\end{document}