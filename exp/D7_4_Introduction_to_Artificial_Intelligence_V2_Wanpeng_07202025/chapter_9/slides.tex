\documentclass[aspectratio=169]{beamer}

% Theme and Color Setup
\usetheme{Madrid}
\usecolortheme{whale}
\useinnertheme{rectangles}
\useoutertheme{miniframes}

% Additional Packages
\usepackage[utf8]{inputenc}
\usepackage[T1]{fontenc}
\usepackage{graphicx}
\usepackage{booktabs}
\usepackage{listings}
\usepackage{amsmath}
\usepackage{amssymb}
\usepackage{xcolor}
\usepackage{tikz}
\usepackage{pgfplots}
\pgfplotsset{compat=1.18}
\usetikzlibrary{positioning}
\usepackage{hyperref}

% Custom Colors
\definecolor{myblue}{RGB}{31, 73, 125}
\definecolor{mygray}{RGB}{100, 100, 100}
\definecolor{mygreen}{RGB}{0, 128, 0}
\definecolor{myorange}{RGB}{230, 126, 34}
\definecolor{mycodebackground}{RGB}{245, 245, 245}

% Set Theme Colors
\setbeamercolor{structure}{fg=myblue}
\setbeamercolor{frametitle}{fg=white, bg=myblue}
\setbeamercolor{title}{fg=myblue}
\setbeamercolor{section in toc}{fg=myblue}
\setbeamercolor{item projected}{fg=white, bg=myblue}
\setbeamercolor{block title}{bg=myblue!20, fg=myblue}
\setbeamercolor{block body}{bg=myblue!10}
\setbeamercolor{alerted text}{fg=myorange}

% Set Fonts
\setbeamerfont{title}{size=\Large, series=\bfseries}
\setbeamerfont{frametitle}{size=\large, series=\bfseries}
\setbeamerfont{caption}{size=\small}
\setbeamerfont{footnote}{size=\tiny}

% Document Start
\begin{document}

\frame{\titlepage}

\begin{frame}[fragile]
    \frametitle{Introduction to First-Order Logic (FOL)}
    \begin{block}{What is First-Order Logic (FOL)?}
        First-Order Logic (FOL), also known as Predicate Logic or Quantified Logic, is an extension of Propositional Logic that allows for more expressive statements involving complex relationships and quantification of variables. This makes it a powerful tool for reasoning in various domains, particularly in Artificial Intelligence (AI).
    \end{block}
\end{frame}

\begin{frame}[fragile]
    \frametitle{Key Features of First-Order Logic}
    \begin{enumerate}
        \item \textbf{Predicates:} Represent properties or relationships.\\
        Example: $Loves(John, Mary)$ means "John loves Mary."
        
        \item \textbf{Quantifiers:} Express statements about some or all elements.
        \begin{itemize}
            \item \textbf{Universal Quantifier ($\forall$):} $\forall x (Human(x) \rightarrow Mortal(x))$ means "All humans are mortal."
            \item \textbf{Existential Quantifier ($\exists$):} $\exists y (Cat(y) \land Black(y))$ means "There exists a cat that is black."
        \end{itemize}
        
        \item \textbf{Variables:} Symbols that represent objects.\\
        Example: In $\forall x (Bird(x) \rightarrow CanFly(x))$, $x$ is a variable.
        
        \item \textbf{Functions:} Maps elements to other elements.\\
        Example: $ParentOf(John)$ could denote "the parent of John."
    \end{enumerate}
\end{frame}

\begin{frame}[fragile]
    \frametitle{Significance of FOL in AI}
    \begin{enumerate}
        \item \textbf{Knowledge Representation:} Structures knowledge for effective reasoning in AI.
        
        \item \textbf{Automated Reasoning:} Forms the basis for many reasoning systems that draw conclusions from a knowledge base.
        
        \item \textbf{Natural Language Understanding:} Aids in parsing human language, crucial for intelligent conversational agents.
        
        \item \textbf{Formal Verification:} Ensures that algorithms behave as intended through formal techniques.
    \end{enumerate}
    
    \begin{block}{Key Points to Remember}
        - FOL extends Propositional Logic by allowing quantification and complex relationships.\\
        - It plays a critical role in various AI domains, from natural language processing to automated reasoning.
    \end{block}
\end{frame}

\begin{frame}[fragile]
    \frametitle{Foundations of First-Order Logic}
    \begin{block}{Learning Objectives}
        \begin{itemize}
            \item Understand the basic concepts of syntax and semantics in First-Order Logic (FOL).
            \item Recognize the importance of these foundations in formal reasoning.
            \item Differentiate between syntax and semantics in logic.
        \end{itemize}
    \end{block}
\end{frame}

\begin{frame}[fragile]
    \frametitle{What is First-Order Logic?}
    \begin{block}{Definition}
        First-Order Logic (FOL), also known as Predicate Logic, is a powerful framework used to represent and reason about relationships between objects in a domain.
    \end{block}
    \begin{itemize}
        \item Extends propositional logic by allowing the use of quantifiers and predicates.
    \end{itemize}
\end{frame}

\begin{frame}[fragile]
    \frametitle{1. Syntax of First-Order Logic}
    \begin{block}{Definition}
        Syntax refers to the formal rules governing the structure of expressions in FOL.
    \end{block}
    \begin{itemize}
        \item \textbf{Key Components of FOL Syntax:}
            \begin{itemize}
                \item \textbf{Constants:} e.g., `a`, `b`, `John`
                \item \textbf{Variables:} e.g., `x`, `y`, `z`
                \item \textbf{Predicates:} e.g., `Likes(John, IceCream)`
                \item \textbf{Functions:} e.g., `MotherOf(John)`
                \item \textbf{Logical Connectives:} `∧`, `∨`, `¬`, and `→`
                \item \textbf{Quantifiers:}
                    \begin{itemize}
                        \item \textbf{Universal Quantifier ($\forall$):} For all elements (e.g., $\forall x\ Loves(x, IceCream)$)
                        \item \textbf{Existential Quantifier ($\exists$):} There exists at least one element (e.g., $\exists x\ Loves(x, IceCream)$)
                    \end{itemize}
            \end{itemize}
        \item \textbf{Example Syntax:}
            \begin{equation}
            \forall x (Human(x) \rightarrow Mortal(x))
            \end{equation}
    \end{itemize}
\end{frame}

\begin{frame}[fragile]
    \frametitle{2. Semantics of First-Order Logic}
    \begin{block}{Definition}
        Semantics provides meaning to the expressions formed according to the syntax of FOL.
    \end{block}
    \begin{itemize}
        \item \textbf{Key Concepts in FOL Semantics:}
            \begin{itemize}
                \item \textbf{Interpretation:} Assigns meanings to constants, predicates, and functions.
                \item \textbf{Domain:} The set of objects over which variables can range (e.g., the set of all people).
                \item \textbf{Truth Values:} Assignments of "true" or "false" based on interpretation.
            \end{itemize}
        \item \textbf{Example of Semantics:}
            \begin{itemize}
                \item For $\forall x (Human(x) \rightarrow Mortal(x))$ to be true, whenever `x` is a human, it must also be true for being mortal.
            \end{itemize}
    \end{itemize}
\end{frame}

\begin{frame}[fragile]
    \frametitle{Key Points to Emphasize}
    \begin{itemize}
        \item Understanding both syntax and semantics is crucial for effectively using First-Order Logic.
        \item Syntax provides the structure that allows us to form logical statements, while semantics gives those statements meaning.
        \item The interplay between syntax and semantics enables reasoning and deduction within the framework of FOL.
    \end{itemize}
    \begin{block}{Conclusion}
        By grasping these foundational concepts, students will be equipped to explore more complex topics in First-Order Logic, such as predicates and their applications in artificial intelligence and automated reasoning.
    \end{block}
\end{frame}

\begin{frame}[fragile]
    \frametitle{Predicates - Definition in FOL}
    \begin{block}{Definition of Predicates in First-Order Logic (FOL)}
        A predicate is a fundamental component in first-order logic that expresses a property or a relation about one or more entities in a specific domain.
        In simpler terms, predicates help us make assertions about objects.
    \end{block}
    
    \begin{block}{Structure of a Predicate}
        Predicates typically have the following structure:
        \begin{itemize}
            \item \textbf{Predicate Symbol}: A capital letter (e.g., \( P \), \( Q \)).
            \item \textbf{Arguments}: Variables or constants that represent objects from the domain (e.g., \( x \), \( y \), \( a \)).
        \end{itemize}
        \textbf{Notation}: A predicate \( P(x) \) indicates that \( x \) possesses a certain property or satisfies a specific relation.
    \end{block}
\end{frame}

\begin{frame}[fragile]
    \frametitle{Predicates - Examples}
    \begin{block}{Examples of Predicates}
        \begin{enumerate}
            \item \textbf{Unary Predicate}:
                \begin{itemize}
                    \item \textbf{Predicate}: \( \text{isEven}(x) \)
                    \item \textbf{Interpretation}: "x is an even number."
                    \item \textbf{Example Instances}:
                        \begin{itemize}
                            \item \( \text{isEven}(2) \) is \textbf{True}.
                            \item \( \text{isEven}(3) \) is \textbf{False}.
                        \end{itemize}
                \end{itemize}

            \item \textbf{Binary Predicate}:
                \begin{itemize}
                    \item \textbf{Predicate}: \( \text{Loves}(x, y) \)
                    \item \textbf{Interpretation}: "x loves y."
                    \item \textbf{Example Instances}:
                        \begin{itemize}
                            \item \( \text{Loves}(\text{Alice}, \text{Bob}) \) is \textbf{True}.
                            \item \( \text{Loves}(\text{Bob}, \text{Alice}) \) is \textbf{False}.
                        \end{itemize}
                \end{itemize}

            \item \textbf{Ternary Predicate}:
                \begin{itemize}
                    \item \textbf{Predicate}: \( \text{Between}(x, y, z) \)
                    \item \textbf{Interpretation}: "x is between y and z."
                    \item \textbf{Example Instance}: 
                        \begin{itemize}
                            \item \( \text{Between}(5, 3, 7) \) is \textbf{True}.
                        \end{itemize}
                \end{itemize}
        \end{enumerate}
    \end{block}
\end{frame}

\begin{frame}[fragile]
    \frametitle{Predicates - Key Points and Conclusion}
    \begin{block}{Key Points to Emphasize}
        \begin{itemize}
            \item \textbf{Quantification}: Predicates can be used with quantifiers to express general claims.
            \begin{equation}
                \forall x (Human(x) \rightarrow Mortal(x))
            \end{equation}
            \item \textbf{Domain of Discourse}: The meaning of a predicate depends on the domain over which the variables range.
            \item \textbf{Subjectivity}: Predicate truth values may vary depending on the interpretation of the objects involved.
        \end{itemize}
    \end{block}

    \begin{block}{Conclusion}
        Predicates are essential for constructing meaningful statements in first-order logic. 
        They allow us to talk about objects and their properties systematically. 
        Understanding predicates sets the foundation for more complex logical expressions and reasoning processes in FOL.
    \end{block}
\end{frame}

\begin{frame}[fragile]
    \frametitle{Quantifiers - Introduction}
    \begin{block}{Introduction to Quantifiers in First-Order Logic (FOL)}
        In First-Order Logic, \textbf{quantifiers} are essential for expressing the properties of collections of objects within a domain. 
        They help us convey statements that involve "all," "some," or "none" of the elements in a particular set.
    \end{block}
    
    This slide focuses on two primary types of quantifiers:
    \begin{enumerate}
        \item \textbf{Universal Quantifier (∀)}
        \item \textbf{Existential Quantifier (∃)}
    \end{enumerate}
\end{frame}

\begin{frame}[fragile]
    \frametitle{Quantifiers - Universal Quantifier (∀)}
    \begin{block}{Definition}
        The universal quantifier asserts that a property holds for all elements in a given domain. It is denoted by the symbol \textbf{∀}.
    \end{block}

    \begin{block}{Expression}
        \begin{itemize}
            \item \textbf{Formal:} ∀x P(x)
            \item \textbf{Read as:} "For every x, P of x is true."
        \end{itemize}
    \end{block}

    \begin{block}{Example}
        \begin{itemize}
            \item \textbf{Statement:} ∀x (Dog(x) → Mammal(x))
            \item \textbf{Interpretation:} "For every object x, if x is a dog, then x is also a mammal."
        \end{itemize}
    \end{block}

    \begin{block}{Key Points}
        \begin{itemize}
            \item Used to generalize statements across a set.
            \item Often employed in mathematical proofs to establish a property that is true for all instances.
        \end{itemize}
    \end{block}
\end{frame}

\begin{frame}[fragile]
    \frametitle{Quantifiers - Existential Quantifier (∃)}
    \begin{block}{Definition}
        The existential quantifier claims that there exists at least one element in the domain for which a property holds true. It is denoted by the symbol \textbf{∃}.
    \end{block}

    \begin{block}{Expression}
        \begin{itemize}
            \item \textbf{Formal:} ∃x P(x)
            \item \textbf{Read as:} "There exists an x such that P of x is true."
        \end{itemize}
    \end{block}

    \begin{block}{Example}
        \begin{itemize}
            \item \textbf{Statement:} ∃x (Mammal(x) ∧ Flees(x))
            \item \textbf{Interpretation:} "There exists at least one object x such that x is a mammal and x flees."
        \end{itemize}
    \end{block}

    \begin{block}{Key Points}
        \begin{itemize}
            \item Used to assert the existence of an example or case that satisfies certain conditions.
            \item Crucial in contexts where proof of existence is required.
        \end{itemize}
    \end{block}
\end{frame}

\begin{frame}[fragile]
    \frametitle{Quantifiers - Summary and Visual Representation}
    \begin{block}{Summary}
        \begin{itemize}
            \item \textbf{Quantifiers} play a vital role in FOL by allowing us to express logical relations succinctly.
            \item \textbf{Universal quantifier (∀):} Indicates that a property applies to all elements.
            \item \textbf{Existential quantifier (∃):} Indicates that at least one element exists that satisfies the property.
        \end{itemize}
    \end{block}

    \begin{block}{Visual Representation}
        \begin{itemize}
            \item For Universal Quantifier (∀): [All Dogs] → [All Mammals]
            \item For Existential Quantifier (∃): [Some Mammals] ⟶ [Dogs, Cats, etc.]
        \end{itemize}
    \end{block}

    \begin{block}{Essential Takeaway}
        Understanding quantifiers is fundamental in constructing logical statements in First-Order Logic. They empower us to express complex ideas succinctly and allow for rigorous reasoning in mathematics and computer science.
    \end{block}
\end{frame}

\begin{frame}[fragile]
    \frametitle{Syntax of First-Order Logic (FOL)}
    \begin{block}{Learning Objectives}
        \begin{itemize}
            \item Understand and articulate the components of First-Order Logic (FOL) syntax.
            \item Identify the structure of FOL expressions and their meanings.
            \item Use FOL syntax to construct simple logical statements.
        \end{itemize}
    \end{block}
\end{frame}

\begin{frame}[fragile]
    \frametitle{Components of FOL Syntax}
    First-Order Logic (FOL) consists of several key elements:
    \begin{itemize}
        \item \textbf{Terms}: Represent objects.
        \begin{itemize}
            \item \textbf{Constants}: Specific objects (e.g., $a$, $b$, $c$).
            \item \textbf{Variables}: Arbitrary objects (e.g., $x$, $y$, $z$).
            \item \textbf{Functions}: Map terms to terms (e.g., $f(x)$).
        \end{itemize}
        \item \textbf{Predicates}: Properties or relations between objects (e.g., $P(a)$, $R(x, y)$).
        \item \textbf{Atomic Formulas}: Simplest logical statements (e.g., $P(a)$, $R(x, y)$).
        \item \textbf{Logical Connectives}: Combine expressions.
        \begin{itemize}
            \item Conjunction ($\land$): "and"
            \item Disjunction ($\lor$): "or"
            \item Negation ($\lnot$): "not"
            \item Implication ($\rightarrow$): "if...then"
            \item Biconditional ($\leftrightarrow$): "if and only if"
        \end{itemize}
    \end{itemize}
\end{frame}

\begin{frame}[fragile]
    \frametitle{Quantifiers in FOL}
    FOL uses quantifiers for domain statements:
    \begin{itemize}
        \item \textbf{Universal Quantifier} ($\forall$): For all elements.
        \begin{itemize}
            \item Example: $\forall x P(x)$ means "For every $x$, $P$ holds."
        \end{itemize}
        \item \textbf{Existential Quantifier} ($\exists$): At least one element exists.
        \begin{itemize}
            \item Example: $\exists x P(x)$ means "There exists an $x$ such that $P$ holds."
        \end{itemize}
    \end{itemize}
\end{frame}

\begin{frame}[fragile]
    \frametitle{Structure and Examples of FOL Expressions}
    \begin{block}{FOL Expression Structure}
        \begin{equation}
            \text{Expression: } (\text{Quantifier (Variable) (Predicate | Logical Expression)})
        \end{equation}
    \end{block}
    \begin{block}{Examples}
        \begin{enumerate}
            \item \textbf{Universal Statement}: 
                \begin{itemize}
                    \item Expression: $\forall x (Human(x) \rightarrow Mortal(x))$
                    \item Meaning: "For all $x$, if $x$ is human, then $x$ is mortal."
                \end{itemize}
            \item \textbf{Existential Statement}: 
                \begin{itemize}
                    \item Expression: $\exists y (Bird(y) \land CanFly(y))$
                    \item Meaning: "There exists a $y$ such that $y$ is a bird and $y$ can fly."
                \end{itemize}
        \end{enumerate}
    \end{block}
\end{frame}

\begin{frame}[fragile]
    \frametitle{Key Points to Remember}
    \begin{itemize}
        \item FOL allows quantification over variables, enabling more expressive statements than propositional logic.
        \item Understanding the structure and syntax is essential for constructing valid logical arguments.
        \item Mastery of FOL syntax lays the foundation for later concepts in logic, including semantics.
    \end{itemize}
    By combining these components of FOL syntax, we can articulate complex logical statements that convey precise relationships in the logical world.
\end{frame}

\begin{frame}[fragile]
    \frametitle{Introduction to Semantics in FOL}
    The semantics of First-Order Logic (FOL) refers to:
    \begin{itemize}
        \item The interpretation of expressions in FOL.
        \item The structures that give meaning to them, contrasting with syntax which defines construction.
    \end{itemize}
\end{frame}

\begin{frame}[fragile]
    \frametitle{Key Concepts of FOL Semantics}
    \begin{enumerate}
        \item \textbf{Interpretation}:
        \begin{itemize}
            \item Assigns meanings to symbols.
            \item Includes:
            \begin{itemize}
                \item \textbf{Domain}: Set of entities.
                \item \textbf{Predicate Interpretation}: Each predicate is associated with tuples from the domain satisfying it.
            \end{itemize}
        \end{itemize}
        \item \textbf{Models}:
        \begin{itemize}
            \item A specific interpretation that makes certain sentences true.
            \item Defined by:
            \begin{itemize}
                \item Domain of discourse (D).
                \item An assignment to predicates and functions in logic.
            \end{itemize}
        \end{itemize}
    \end{enumerate}
\end{frame}

\begin{frame}[fragile]
    \frametitle{Example Interpretation in FOL}
    Consider the expression:
    \begin{equation}
    \forall x (Person(x) \rightarrow Mortal(x))
    \end{equation}
    
    Components:
    \begin{itemize}
        \item \textbf{Domain (D)}: \{Socrates, Plato\}
        \item \textbf{Predicates}:
        \begin{itemize}
            \item Person(x) = True for \{Socrates, Plato\}
            \item Mortal(x) = True for \{Socrates, Plato\}
        \end{itemize}
    \end{itemize}
    
    Interpretation:
    All individuals in our domain are both persons and mortals.
\end{frame}

\begin{frame}[fragile]
    \frametitle{Truth and Satisfaction in FOL}
    \begin{itemize}
        \item A formula is \textbf{true} in a model if it holds under the interpretation.
        \item A formula is \textbf{satisfied} by a model if the statements it makes are correct under that interpretation.
        \item Truth values can be assigned to FOL statements based on the domain. 
    \end{itemize}
    
    Important Points to Emphasize:
    \begin{itemize}
        \item Semantics in FOL is crucial for logical reasoning and understanding real-world relations.
        \item Different interpretations lead to different truths, showcasing FOL's flexibility.
        \item A syntactic sentence can have multiple interpretations and thus lead to different models.
    \end{itemize}
\end{frame}

\begin{frame}[fragile]
    \frametitle{Example Model Summary}
    \begin{tabular}{|l|l|}
        \hline
        \textbf{Element} & \textbf{Meaning} \\
        \hline
        Domain (D) & \{Socrates, Plato\} \\
        \hline
        Person(x) & True for x = Socrates, Plato \\
        \hline
        Mortal(x) & True for x = Socrates, Plato \\
        \hline
    \end{tabular}
    
    Conclusion:
    \begin{itemize}
        \item Understanding semantics in FOL enriches skills in mathematics, computer science, and philosophy.
    \end{itemize}
\end{frame}

\begin{frame}[fragile]
    \frametitle{Using Predicates in First-Order Logic (FOL)}
    \begin{block}{Understanding Predicates}
        \begin{itemize}
            \item \textbf{Definition:} A predicate is a function that returns a truth value (True or False) based on its input. 
            \item \textbf{Notation:} Predicates are represented as capital letters followed by variables. 
        \end{itemize}
    \end{block}
    Example: $P(x)$ could represent "x is a prime number."
\end{frame}

\begin{frame}[fragile]
    \frametitle{Types of Predicates}
    \begin{enumerate}
        \item \textbf{Unary Predicate:} A predicate that takes a single argument.
            \begin{itemize}
                \item Example: $IsStudent(x)$ – True if $x$ is a student.
            \end{itemize}
        \item \textbf{Binary Predicate:} A predicate that takes two arguments.
            \begin{itemize}
                \item Example: $Loves(x, y)$ – True if $x$ loves $y$.
            \end{itemize}
        \item \textbf{N-ary Predicate:} A predicate that can take multiple arguments.
            \begin{itemize}
                \item Example: $Siblings(x, y, z)$ – True if $x, y,$ and $z$ are siblings.
            \end{itemize}
    \end{enumerate}
\end{frame}

\begin{frame}[fragile]
    \frametitle{Example Statements Using Predicates}
    \begin{block}{Expressing Properties}
        \begin{itemize}
            \item \textbf{Statement:} "John is a student."
            \item \textbf{Predicate Representation:} $Student(John)$
            \item \textbf{Interpretation:} This tells us that the predicate $Student$ applied to John evaluates to True.
        \end{itemize}
    \end{block}

    \begin{block}{Expressing Relationships}
        \begin{itemize}
            \item \textbf{Statement:} "Alice loves Bob."
            \item \textbf{Predicate Representation:} $Loves(Alice, Bob)$
            \item \textbf{Interpretation:} This captures the relationship between Alice and Bob with respect to love.
        \end{itemize}
    \end{block}

    \begin{block}{Complex Statements}
        \begin{itemize}
            \item \textbf{Statement:} "All students study."
            \item \textbf{Predicate Representation:} $\forall x (Student(x) \rightarrow Studies(x))$
            \item \textbf{Interpretation:} This universal quantification states that for all $x$, if $x$ is a student, then $x$ studies.
        \end{itemize}
    \end{block}
\end{frame}

\begin{frame}[fragile]
    \frametitle{Key Points and Formulas}
    \begin{block}{Key Points to Emphasize}
        \begin{itemize}
            \item Predicates are fundamental in First-Order Logic, allowing us to articulate specific properties and relationships among objects.
            \item Structure helps us build logical frameworks for reasoning about statements.
            \item Understanding predicates enables more complex logical relationships using quantifiers (e.g., $\forall$, $\exists$).
        \end{itemize}
    \end{block}

    \begin{block}{Formulas}
        \begin{itemize}
            \item \textbf{Universal Quantification:} $\forall x P(x)$ means "For all $x$, $P(x)$ is true."
            \item \textbf{Existential Quantification:} $\exists x P(x)$ means "There exists an $x$ such that $P(x)$ is true."
        \end{itemize}
    \end{block}
\end{frame}

\begin{frame}[fragile]
    \frametitle{Recap}
    \begin{itemize}
        \item Predicates are versatile tools in FOL that allow for precise expressions of logical statements.
        \item Mastery of predicates enables students to create models to represent real-world scenarios and reason about them logically.
    \end{itemize}
\end{frame}

\begin{frame}[fragile]
    \frametitle{Applications of First-Order Logic}
    \begin{block}{Introduction to First-Order Logic (FOL)}
        First-Order Logic is a powerful formal system that allows us to represent relationships and make logical assertions about objects. It extends propositional logic by incorporating predicates, quantifiers, and more complex structures.
    \end{block}
\end{frame}

\begin{frame}[fragile]
    \frametitle{Key Applications of FOL - 1}
    
    \begin{enumerate}
        \item \textbf{Knowledge Representation}
        \begin{itemize}
            \item Definition: Encoding information about the world into a computable format.
            \item Example: Using predicates to express facts: 
            \begin{lstlisting}
            Likes(John, IceCream)
            \end{lstlisting}
            \item Ontologies: Structuring knowledge in specific domains (e.g., biology, AI).
        \end{itemize}
        
        \item \textbf{Automated Theorem Proving}
        \begin{itemize}
            \item Definition: Computers proving mathematical theorems automatically.
            \item Example: 
            \begin{equation}
            \forall x (Human(x) \Rightarrow Mortal(x)), \; Human(Socrates) \vdash Mortal(Socrates)
            \end{equation}
            \item Application: Used in software and hardware verification.
        \end{itemize}
    \end{enumerate}
\end{frame}

\begin{frame}[fragile]
    \frametitle{Key Applications of FOL - 2}
    
    \begin{enumerate}
        \setcounter{enumi}{2}
        \item \textbf{Natural Language Processing (NLP)}
        \begin{itemize}
            \item Definition: Interaction between computers and human language.
            \item Example: Encoding semantic relationships:
            \begin{equation}
            \forall x (Bird(x) \Rightarrow CanFly(x))
            \end{equation}
        \end{itemize}
        
        \item \textbf{Artificial Intelligence & Expert Systems}
        \begin{itemize}
            \item Definition: Emulating human decision-making.
            \item Example: Medical diagnosis system using FOL to represent diseases and symptoms.
        \end{itemize}
        
        \item \textbf{Database Querying}
        \begin{itemize}
            \item Definition: Requesting information from databases.
            \item Example: FOL-based query to find students:
            \begin{lstlisting}
            FindAll(x) where Taken(x, Mathematics) and Score(x) > 80
            \end{lstlisting}
        \end{itemize}
    \end{enumerate}
\end{frame}

\begin{frame}[fragile]
    \frametitle{Conclusion and Key Points}
    \begin{block}{Key Points to Emphasize}
        \begin{itemize}
            \item \textbf{Expressiveness:} FOL provides a nuanced representation compared to propositional logic.
            \item \textbf{Flexibility:} Applicable in various domains, including technology and linguistics.
            \item \textbf{Reasoning Capabilities:} Enables derivation of new knowledge through inference rules.
        \end{itemize}
    \end{block}
    
    \begin{block}{Conclusion}
        First-Order Logic serves as a critical foundation in various fields, enhancing our ability to represent complex information and reason about relationships among different elements.
    \end{block}
\end{frame}

\begin{frame}[fragile]
    \frametitle{Inference in First-Order Logic (FOL)}
    \begin{block}{Learning Objectives}
        \begin{itemize}
            \item Understand key inference methods used in FOL, such as resolution and unification.
            \item Apply these methods to derive conclusions from given premises.
        \end{itemize}
    \end{block}
\end{frame}

\begin{frame}[fragile]
    \frametitle{Inference in First-Order Logic}
    \begin{block}{Key Concepts}
        \begin{itemize}
            \item Inference involves deriving new information from known facts or statements in FOL.
            \item It is crucial for manipulating predicates, quantifiers, and logical relationships.
        \end{itemize}
    \end{block}
\end{frame}

\begin{frame}[fragile]
    \frametitle{Resolution}
    \begin{block}{Definition}
        A rule of inference that allows one to derive a conclusion from a set of premises by refuting a negated conclusion.
    \end{block}
    \begin{block}{Procedure}
        \begin{enumerate}
            \item Convert all premises to Conjunctive Normal Form (CNF).
            \item Identify pairs of clauses with complementary literals.
            \item Apply the resolution rule to derive new clauses.
        \end{enumerate}
    \end{block}
    \begin{example}
        \begin{lstlisting}
        Premise 1: ∀x (Cat(x) → Mammal(x))
        Premise 2: Cat(Tom)
        Goal: Mammal(Tom)

        Steps:
        1. Convert to CNF:
           - Clause 1: ¬Cat(x) ∨ Mammal(x)
           - Clause 2: Cat(Tom)
        2. Resolve:
           - Resolving ¬Cat(Tom) ∨ Mammal(Tom) with Cat(Tom) gives: Mammal(Tom)
        \end{lstlisting}
    \end{example}
\end{frame}

\begin{frame}[fragile]
    \frametitle{Unification}
    \begin{block}{Definition}
        A process of making different logical expressions identical by substituting variables.
    \end{block}
    \begin{block}{Procedure}
        \begin{enumerate}
            \item Identify variables in expressions.
            \item Substitute variables with terms or other variables to create identical structures.
        \end{enumerate}
    \end{block}
    \begin{example}
        \begin{lstlisting}
        Expression 1: Loves(John, x)
        Expression 2: Loves(y, Mary)

        Unification:
        - Substitute y with John
        - Result: Loves(John, Mary)
        \end{lstlisting}
    \end{example}
\end{frame}

\begin{frame}[fragile]
    \frametitle{Key Points and Conclusion}
    \begin{block}{Key Points}
        \begin{itemize}
            \item Importance of Resolution: A powerful method for proving theorems in FOL.
            \item Role of Unification: Essential for matching predicates to apply inference rules.
            \item Applications: Vital in automated theorem proving, artificial intelligence, and knowledge representation.
        \end{itemize}
    \end{block}
    \begin{block}{Conclusion}
        Understanding inference methods like resolution and unification is crucial for effectively working with First-Order Logic. 
    \end{block}
\end{frame}

\begin{frame}[fragile]
    \frametitle{Limitations of Propositional Logic}
    
    \begin{block}{Introduction to Propositional Logic (PL)}
        Propositional Logic (PL) is a formal system that uses propositions as its basic units. Propositions are declarative statements that can either be true or false, such as:
        
        \begin{itemize}
            \item "It is raining." (True or False)
            \item "The sky is blue." (True or False)
        \end{itemize}
        
        While powerful in certain contexts, PL has inherent limitations compared to First-Order Logic (FOL).
    \end{block}
\end{frame}

\begin{frame}[fragile]
    \frametitle{Key Limitations of Propositional Logic}
    
    \begin{enumerate}
        \item \textbf{Lacks Expressiveness}:
            \begin{itemize}
                \item PL cannot express quantifiers such as "all" or "some."
                \item Example: "All humans are mortal" in FOL: $\forall x \ (Human(x) \rightarrow Mortal(x))$.
                \item In PL, it only represents specific cases such as "Socrates is mortal."
            \end{itemize}
        
        \item \textbf{Inability to Handle Relations}:
            \begin{itemize}
                \item PL treats propositions as isolated statements.
                \item In FOL: "Alice is taller than Bob" is expressed as $Taller(Alice, Bob)$.
                \item In PL, this requires separate propositions losing the relational aspect.
            \end{itemize}
        
        \item \textbf{Propositional Complexity}:
            \begin{itemize}
                \item PL becomes cumbersome with nested propositions.
                \item Combining propositions increases complexity of truth assignments.
            \end{itemize}
        
        \item \textbf{Static Nature}:
            \begin{itemize}
                \item PL lacks dynamic variables and cannot express concepts such as "there exists."
                \item Example in FOL: "There exists a person who is a teacher" as $\exists x \ (Teacher(x))$.
            \end{itemize}
    \end{enumerate}
\end{frame}

\begin{frame}[fragile]
    \frametitle{Contrasting with First-Order Logic}
    
    FOL enhances capabilities by introducing:
    
    \begin{itemize}
        \item \textbf{Quantifiers}: $\forall$ (for all), $\exists$ (there exists) for statements about groups.
        \item \textbf{Predicates}: Capture properties or relations among objects, enabling more sophisticated expressions.
    \end{itemize}
    
    \begin{block}{Summary of Key Points}
        \begin{itemize}
            \item PL is limited in expressing general truths, relationships, and dynamic content.
            \item FOL expands upon these limitations, allowing for richer formal expressions.
        \end{itemize}
    \end{block}
\end{frame}

\begin{frame}[fragile]
    \frametitle{Conclusion and Example}
    
    Understanding the limitations of PL is essential for further exploration into logical reasoning and the realm of FOL, which is crucial in areas like:
    
    \begin{itemize}
        \item Mathematics
        \item Computer Science
        \item Artificial Intelligence
    \end{itemize}
    
    \begin{block}{Example of Transition from PL to FOL}
        \textbf{PL Statement}: “It is raining or it is sunny.” \\
        \textbf{FOL Equivalent}: "For all weather conditions, if it is not raining, then it must be sunny." \\
        Expressed as: $\forall x \ (Weather(x) \rightarrow (\neg Raining(x) \rightarrow Sunny(x)))$
    \end{block}
    
    This comparison empowers learners to appreciate the necessity of FOL and prepares them for upcoming examples of FOL in use.
\end{frame}

\begin{frame}[fragile]
    \frametitle{Examples of First-Order Logic Statements}
    
    \begin{block}{Learning Objectives}
        \begin{itemize}
            \item Understand the structure of First-Order Logic (FOL) statements
            \item Differentiate between types of FOL statements
            \item Interpret and analyze FOL statements in practical contexts
        \end{itemize}
    \end{block}

    \begin{block}{Concept Overview}
        First-Order Logic (FOL) extends propositional logic by introducing quantifiers
        and predicates, allowing for more expressive statements about properties and
        relationships in a domain. FOL consists of terms (constants, variables, functions),
        predicates (representing properties or relationships), and logical connectives.
    \end{block}
\end{frame}

\begin{frame}[fragile]
    \frametitle{Structure of FOL Statements}
    
    \begin{itemize}
        \item \textbf{Predicates}: Represent properties or relations (e.g., \(Loves(x, y)\), \(Human(x)\))
        \item \textbf{Quantifiers}: Indicate the scope of the statement:
        \begin{itemize}
            \item \textbf{Universal Quantifier (\(\forall\))}: True for all elements (e.g., \(\forall x: Human(x) \rightarrow Mortal(x)\))
            \item \textbf{Existential Quantifier (\(\exists\))}: At least one element is true (e.g., \(\exists y: Loves(John, y)\))
        \end{itemize}
    \end{itemize}
\end{frame}

\begin{frame}[fragile]
    \frametitle{Examples of FOL Statements}

    \begin{enumerate}
        \item \textbf{Universal Statement}
            \begin{itemize}
                \item \textit{Statement}: "All humans are mortal."
                \item \textit{FOL Representation}: 
                \[
                \forall x (Human(x) \rightarrow Mortal(x))
                \]
                \item \textit{Explanation}: For every element \(x\), if \(x\) is a human, then \(x\) is mortal.
            \end{itemize}
        
        \item \textbf{Existential Statement}
            \begin{itemize}
                \item \textit{Statement}: "There exists a human who loves."
                \item \textit{FOL Representation}: 
                \[
                \exists x (Human(x) \land \exists y (Loves(x, y)))
                \]
                \item \textit{Explanation}: There is at least one \(x\) that is a human, and there exists some \(y\) such that \(x\) loves \(y\).
            \end{itemize}   
    \end{enumerate}
\end{frame}

\begin{frame}[fragile]
    \frametitle{Equivalence and Validity - Overview}
    
    \begin{block}{Key Concepts}
        \begin{itemize}
            \item \textbf{First-Order Logic (FOL)}: A formal system to express statements about objects and their relations with predicates, quantifiers, and logical connectives.
            \item \textbf{Equivalence}: Two statements are equivalent if they yield the same truth value in every model, denoted as \( A \equiv B \).
            \item \textbf{Validity}: A formula is valid if it is true in every interpretation, denoted as \( ⊨ A \).
        \end{itemize}
    \end{block}
    
\end{frame}

\begin{frame}[fragile]
    \frametitle{Importance of Equivalence and Validity}
    
    \begin{itemize}
        \item \textbf{Logical Reasoning}: Understanding equivalence streamlines logical expressions and validity ensures consistency across situations.
        \item \textbf{Proof Construction}: Recognizing equivalent forms can simplify proofs and facilitate the derivation of new statements.
    \end{itemize}
    
\end{frame}

\begin{frame}[fragile]
    \frametitle{Illustrative Examples}
    
    \begin{block}{Equivalence Example}
        Statements:
        \begin{enumerate}
            \item \( \forall x (P(x) \lor Q) \)
            \item \( \forall x P(x) \lor Q \)
        \end{enumerate}
        These are equivalent if \( Q \) is a constant statement, leading to the same truth conditions.
    \end{block}
    
    \begin{block}{Validity Example}
        Consider the formula: 
        \[
        \forall x (P(x) \rightarrow P(x))
        \]
        This is valid, as it holds true regardless of the predicate \( P(x) \).
    \end{block}
    
\end{frame}

\begin{frame}[fragile]
    \frametitle{Key Points and Summary}
    
    \begin{itemize}
        \item Equivalence simplifies logic statements and provides efficient reasoning.
        \item Validity ensures logical arguments are universally sound.
        \item Mastery of these concepts is fundamental for reasoning in First-Order Logic.
    \end{itemize}
    
    \begin{block}{Next Steps}
        Prepare to explore \textbf{Logical Consequences and Derivations} in the next slide.
    \end{block}
    
\end{frame}

\begin{frame}[fragile]
    \frametitle{Logical Consequences and Derivations - Overview}
    \begin{block}{Understanding Logical Consequences in First-Order Logic (FOL)}
        \begin{itemize}
            \item \textbf{Definition}: A statement \(C\) is a logical consequence of premises \(P_1, P_2, \ldots, P_n\) if \( P_1, P_2, \ldots, P_n \models C \).
            \item \textbf{Importance}: Enables inferences and deriving new knowledge in FOL, essential for automated reasoning and theorem proving.
        \end{itemize}
    \end{block}
\end{frame}

\begin{frame}[fragile]
    \frametitle{Logical Consequences and Derivations - Derivation}
    \begin{block}{Derivation in FOL}
        \begin{itemize}
            \item \textbf{Definition}: The process of demonstrating that a statement can be inferred from premises using inference rules.
            \item \textbf{Example of Derivation}:
                \begin{enumerate}
                    \item \(\forall x (P(x) \rightarrow Q(x))\)
                    \item \(P(a)\)
                    \item \textbf{Derived Conclusion}: \(Q(a)\) via Modus Ponens.
                    \begin{itemize}
                        \item Substitute \(a\) into \(\forall x (P(x) \rightarrow Q(x))\) to get \(P(a) \rightarrow Q(a)\).
                        \item Applying Modus Ponens gives \(Q(a)\).
                    \end{itemize}
                \end{enumerate}
        \end{itemize}
    \end{block}
\end{frame}

\begin{frame}[fragile]
    \frametitle{Logical Consequences and Derivations - Rules of Inference}
    \begin{block}{Rules of Inference}
        \begin{itemize}
            \item \textbf{Modus Ponens}: If \(P \rightarrow Q\) and \(P\) are true, then \(Q\) must be true.
            \item \textbf{Modus Tollens}: If \(P \rightarrow Q\) and \(\neg Q\) are true, then \(\neg P\) must be true.
            \item \textbf{Universal Instantiation}: From \(\forall x P(x)\), conclude \(P(a)\) for any specific \(a\).
        \end{itemize}
    \end{block}
    \begin{block}{Key Takeaways}
        \begin{itemize}
            \item Logical consequences are vital for systematic reasoning.
            \item Understanding inference rules is fundamental for proofs.
            \item Derivation skills are essential in mathematics, computer science (AI), and philosophy.
        \end{itemize}
    \end{block}
\end{frame}

\begin{frame}[fragile]
    \frametitle{Real-world Use Cases of FOL - Introduction}
    \begin{block}{First-Order Logic (FOL)}
        First-Order Logic (FOL) is a powerful system of formal logic for representing facts and relationships through quantified variables and predicates.
    \end{block}
    \begin{itemize}
        \item Extends propositional logic with quantifiers.
        \item Suitable for reasoning about properties and relations in a more expressive way.
    \end{itemize}
\end{frame}

\begin{frame}[fragile]
    \frametitle{Real-world Use Cases of FOL - Applications}
    \begin{enumerate}
        \item \textbf{Automated Reasoning Systems}
        \begin{itemize}
            \item Use FOL for deriving conclusions from premises.
            \item \textbf{Example:} Theorem Provers (e.g., Prover9, Vampire) validate mathematical statements and verify software correctness.
        \end{itemize}
        
        \item \textbf{Databases and Query Languages}
        \begin{itemize}
            \item FOL underpins many database systems, enabling complex queries.
            \item \textbf{Example:}
            \begin{equation}
                \forall x \left( Employee(x) \land Salary(x) > 50000 \rightarrow Output(x) \right)
            \end{equation}
            Similar to query: \text{SELECT * FROM Employees WHERE Salary > 50000}
        \end{itemize}
        
        \item \textbf{Artificial Intelligence}
        \begin{itemize}
            \item FOL is used for knowledge representation and reasoning.
            \item \textbf{Example:}
            \begin{equation}
                Likes(Alice, Pizza) \land Likes(Alice, Pasta) \rightarrow Suggest(Alice, Italian Restaurant)
            \end{equation}
        \end{itemize}
    \end{enumerate}
\end{frame}

\begin{frame}[fragile]
    \frametitle{Key Points and Conclusion}
    \begin{itemize}
        \item \textbf{Expressiveness of FOL:} Allows detailed modeling of complex relationships.
        \item \textbf{Automation of Reasoning:} Increases efficiency, reducing human error.
        \item \textbf{Foundation for Technologies:} Essential in AI, databases, and various technologies.
    \end{itemize}
    \begin{block}{Conclusion}
        FOL is invaluable in automated reasoning systems, databases, and AI, enabling efficient modeling and reasoning in real-world applications.
    \end{block}
\end{frame}

\begin{frame}[fragile]
    \frametitle{Summary of Key Takeaways - Part 1}
    
    \begin{block}{1. Introduction to First-Order Logic (FOL)}
        \begin{itemize}
            \item \textbf{Definition}: FOL is a formal system enabling reasoning about objects and their relationships, extending propositional logic with quantifiers and predicates.
        \end{itemize}
    \end{block}

    \begin{block}{2. Key Concepts in FOL}
        \begin{itemize}
            \item \textbf{Predicates}: Functions that express properties/relationships. Example: $Loves(John, Mary)$.
            \item \textbf{Quantifiers}:
                \begin{itemize}
                    \item \textbf{Universal Quantifier ($\forall$)}: True for all elements. Example: $\forall x (Human(x) \rightarrow Mortal(x))$.
                    \item \textbf{Existential Quantifier ($\exists$)}: At least one element. Example: $\exists y (Cat(y) \land Black(y))$.
                \end{itemize}
            \item \textbf{Syntax and Semantics}:
                \begin{itemize}
                    \item \textbf{Syntax}: Structure of sentences in FOL.
                    \item \textbf{Semantics}: Interpretation of formulas based on truth conditions.
                \end{itemize}
        \end{itemize}
    \end{block}
\end{frame}

\begin{frame}[fragile]
    \frametitle{Summary of Key Takeaways - Part 2}

    \begin{block}{3. Importance of FOL in AI}
        \begin{itemize}
            \item \textbf{Automated Reasoning}: 
                Crucial for machines to perform reasoning tasks. 
                Example: Theorem provers utilize FOL to derive conclusions.
            \item \textbf{Knowledge Representation}: 
                Facilitates representing complex structures in AI, enhancing capabilities in natural language processing and expert systems.
        \end{itemize}
    \end{block}

    \begin{block}{4. Applications of FOL}
        \begin{itemize}
            \item \textbf{Databases}: FOL principles underpin SQL for structured querying.
            \item \textbf{Natural Language Understanding}: 
                FOL frameworks help translate natural language into formal representations.
        \end{itemize}
    \end{block}
\end{frame}

\begin{frame}[fragile]
    \frametitle{Summary of Key Takeaways - Part 3}

    \begin{block}{5. Summary Key Points}
        \begin{itemize}
            \item FOL extends beyond true/false evaluations to reasoning about properties and relationships.
            \item The integration of quantifiers supports expressing general truths and structured relationships.
            \item Its applications in AI, such as reasoning and knowledge representation, demonstrate its significance in intelligent systems.
        \end{itemize}
    \end{block}

    \begin{block}{6. Further Exploration}
        \begin{itemize}
            \item Explore FOL implementations in specific AI applications like expert systems and knowledge graphs for deeper insights.
        \end{itemize}
    \end{block}
\end{frame}

\begin{frame}[fragile]{Questions and Further Discussion}
    \begin{block}{Overview of First-Order Logic (FOL)}
        First-Order Logic (FOL) is a powerful formal system used in many areas of computer science, particularly artificial intelligence. It extends propositional logic by introducing quantifiers, predicates, and variables, allowing for more nuanced expressions of knowledge.
    \end{block}
\end{frame}

\begin{frame}[fragile]{Key Concepts Recap (Part 1)}
    \begin{enumerate}
        \item \textbf{Syntax of FOL}:
            \begin{itemize}
                \item \textbf{Predicates:} Statements about objects (e.g., $Human(Socrates)$).
                \item \textbf{Quantifiers:}
                    \begin{itemize}
                        \item Universal $(\forall)$: "For all" (e.g., $\forall x (Human(x) \rightarrow Mortal(x))$).
                        \item Existential $(\exists)$: "There exists" (e.g., $\exists x (Human(x) \land Mortal(x))$).
                    \end{itemize}
            \end{itemize}

        \item \textbf{Semantics of FOL}:
            \begin{itemize}
                \item Interpretation of terms and relationships in a model.
                \item World representations that FOL can describe, e.g., objects, their properties, and relationships.
            \end{itemize}
    \end{enumerate}
\end{frame}

\begin{frame}[fragile]{Key Concepts Recap (Part 2)}
    \begin{enumerate}
        \setcounter{enumi}{2}
        \item \textbf{Inference Rules}:
            \begin{itemize}
                \item Methods to derive new statements from given ones, essential for reasoning and proving in AI systems.
            \end{itemize}
    \end{enumerate}
\end{frame}

\begin{frame}[fragile]{Engaging Discussion Points}
    \begin{itemize}
        \item \textbf{Application of FOL in AI}:
            \begin{itemize}
                \item Enhancing knowledge representation systems.
                \item Utilizing FOL in various machine learning frameworks.
            \end{itemize}
        
        \item \textbf{Challenges of FOL}:
            \begin{itemize}
                \item Limitations such as undecidability and computational complexity.
            \end{itemize}
        
        \item \textbf{Practical Examples}:
            \begin{itemize}
                \item Using FOL for database queries and natural language processing.
            \end{itemize}
    \end{itemize}
\end{frame}

\begin{frame}[fragile]{Discussion Questions}
    \begin{enumerate}
        \item How does the concept of quantifiers in FOL differ in practical application compared to natural language quantifiers?
        \item What are effective strategies to automate reasoning with FOL in AI applications?
        \item How can FOL representation assist in the development of knowledge bases and expert systems?
    \end{enumerate}
\end{frame}

\begin{frame}[fragile]{Conclusion}
    This slide serves as an open floor for questions and further insights into First-Order Logic. Encouraging collaborative discussion can deepen understanding and foster critical thinking about how FOL applies within artificial intelligence contexts.

    \textbf{Let’s discuss your thoughts, questions, and any real-life connections you can make with First-Order Logic!}
\end{frame}


\end{document}