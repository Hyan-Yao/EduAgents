\documentclass[aspectratio=169]{beamer}

% Theme and Color Setup
\usetheme{Madrid}
\usecolortheme{whale}
\useinnertheme{rectangles}
\useoutertheme{miniframes}

% Additional Packages
\usepackage[utf8]{inputenc}
\usepackage[T1]{fontenc}
\usepackage{graphicx}
\usepackage{booktabs}
\usepackage{listings}
\usepackage{amsmath}
\usepackage{amssymb}
\usepackage{xcolor}
\usepackage{tikz}
\usepackage{pgfplots}
\pgfplotsset{compat=1.18}
\usetikzlibrary{positioning}
\usepackage{hyperref}

% Custom Colors
\definecolor{myblue}{RGB}{31, 73, 125}
\definecolor{mygray}{RGB}{100, 100, 100}
\definecolor{mygreen}{RGB}{0, 128, 0}
\definecolor{myorange}{RGB}{230, 126, 34}
\definecolor{mycodebackground}{RGB}{245, 245, 245}

% Set Theme Colors
\setbeamercolor{structure}{fg=myblue}
\setbeamercolor{frametitle}{fg=white, bg=myblue}
\setbeamercolor{title}{fg=myblue}
\setbeamercolor{section in toc}{fg=myblue}
\setbeamercolor{item projected}{fg=white, bg=myblue}
\setbeamercolor{block title}{bg=myblue!20, fg=myblue}
\setbeamercolor{block body}{bg=myblue!10}
\setbeamercolor{alerted text}{fg=myorange}

% Set Fonts
\setbeamerfont{title}{size=\Large, series=\bfseries}
\setbeamerfont{frametitle}{size=\large, series=\bfseries}
\setbeamerfont{caption}{size=\small}
\setbeamerfont{footnote}{size=\tiny}

% Footer and Navigation Setup
\setbeamertemplate{footline}{
  \leavevmode%
  \hbox{%
  \begin{beamercolorbox}[wd=.3\paperwidth,ht=2.25ex,dp=1ex,center]{author in head/foot}%
    \usebeamerfont{author in head/foot}\insertshortauthor
  \end{beamercolorbox}%
  \begin{beamercolorbox}[wd=.5\paperwidth,ht=2.25ex,dp=1ex,center]{title in head/foot}%
    \usebeamerfont{title in head/foot}\insertshorttitle
  \end{beamercolorbox}%
  \begin{beamercolorbox}[wd=.2\paperwidth,ht=2.25ex,dp=1ex,center]{date in head/foot}%
    \usebeamerfont{date in head/foot}
    \insertframenumber{} / \inserttotalframenumber
  \end{beamercolorbox}}%
  \vskip0pt%
}

% Turn off navigation symbols
\setbeamertemplate{navigation symbols}{}

% Title Page Information
\title[Constraint Satisfaction Problems]{Week 5: Constraint Satisfaction Problems}
\author[Your Name]{Your Name}
\institute[Your Institution]{Your Institution}
\date{\today}

% Document Start
\begin{document}

\frame{\titlepage}

\begin{frame}[fragile]
    \title{Introduction to Constraint Satisfaction Problems (CSPs)}
    \author{Your Name}
    \date{\today}
    \maketitle
\end{frame}

\begin{frame}[fragile]
    \frametitle{What are Constraint Satisfaction Problems (CSPs)?}
    A Constraint Satisfaction Problem is a mathematical problem defined as a set of objects whose state must satisfy several constraints and restrictions. The CSP consists of:
    
    \begin{itemize}
        \item \textbf{Variables}: Unknowns that need to be solved for (e.g., times assigned to tasks in a scheduling problem).
        \item \textbf{Domains}: Sets of possible values that each variable can take (e.g., range of times for tasks).
        \item \textbf{Constraints}: Rules that dictate valid combinations of variable assignments (e.g., task A cannot start before task B finishes).
    \end{itemize}
\end{frame}

\begin{frame}[fragile]
    \frametitle{Why are CSPs Important in AI?}
    CSPs are significant in AI due to their structured representation of complex problems involving multiple interrelated decisions:
    
    \begin{itemize}
        \item \textbf{Modeling Real-World Problems}: Useful in scheduling, planning, resource allocation, etc.
        \item \textbf{Underlying Algorithms}: Many efficient algorithms exist, including backtracking, constraint propagation, and heuristics.
        \item \textbf{Foundation for Advanced AI}: CSPs are crucial for more complex AI tasks like planning and robotics.
    \end{itemize}
    
    \textbf{Example of a CSP: Sudoku}
    \begin{itemize}
        \item \textbf{Variables}: Each cell in a 9x9 grid.
        \item \textbf{Domains}: Numbers 1 to 9.
        \item \textbf{Constraints}: Each number appears once per row, column, and 3x3 sub-grid.
    \end{itemize}
\end{frame}

\begin{frame}[fragile]
    \frametitle{Understanding Constraint Satisfaction Problems (CSPs)}
    \begin{block}{Definition}
        Constraint Satisfaction Problems (CSPs) are mathematical problems defined by a set of objects whose state must satisfy several constraints and restrictions. These problems are prevalent in various real-world scenarios.
    \end{block}
    \begin{itemize}
        \item Vital in fields such as:
        \begin{itemize}
            \item Artificial Intelligence
            \item Operations Research
            \item Computer Science
        \end{itemize}
        \item Aim to model real-world problems systematically.
    \end{itemize}
\end{frame}

\begin{frame}[fragile]
    \frametitle{Key Areas Where CSPs are Applied}
    \begin{enumerate}
        \item \textbf{Scheduling Problems}
        \begin{itemize}
            \item \textbf{Example: Employee Scheduling}
            \begin{itemize}
                \item Goal: Assign shifts while covering all shifts and respecting employee preferences.
                \item \textbf{Model as CSP:}
                \begin{itemize}
                    \item \textbf{Variables:} Employee schedules
                    \item \textbf{Domains:} Possible shifts
                    \item \textbf{Constraints:} Shift overlap, max hours, required staff per shift.
                \end{itemize}
            \end{itemize}
        \end{itemize}
        \item \textbf{Resource Allocation}
        \begin{itemize}
            \item \textbf{Example: Project Management}
            \begin{itemize}
                \item Goal: Allocate resources effectively to tasks.
                \item \textbf{Model as CSP:}
                \begin{itemize}
                    \item \textbf{Variables:} Tasks needing resources
                    \item \textbf{Domains:} Available resources
                    \item \textbf{Constraints:} Capacities, dependencies, deadlines.
                \end{itemize}
            \end{itemize}
        \end{itemize}
    \end{enumerate}
\end{frame}

\begin{frame}[fragile]
    \frametitle{CSP Applications Continued}
    \begin{enumerate}
        \setcounter{enumi}{2}
        \item \textbf{Graph Coloring}
        \begin{itemize}
            \item \textbf{Example: Register Allocation in Compilers}
            \begin{itemize}
                \item Goal: Assign variables to registers without conflicts.
                \item \textbf{Model as CSP:}
                \begin{itemize}
                    \item \textbf{Variables:} Variables needing registers
                    \item \textbf{Domains:} Available registers
                    \item \textbf{Constraints:} No shared registers for conflicting variables.
                \end{itemize}
            \end{itemize}
        \end{itemize}
        \item \textbf{Sudoku Puzzles}
        \begin{itemize}
            \item \textbf{Example: Sudoku}
            \begin{itemize}
                \item Goal: Fill a 9x9 grid satisfying row, column, and subgrid constraints.
                \item \textbf{Model as CSP:}
                \begin{itemize}
                    \item \textbf{Variables:} Empty cells
                    \item \textbf{Domains:} Possible values (1-9)
                    \item \textbf{Constraints:} Unique values in each row, column, and subgrid.
                \end{itemize}
            \end{itemize}
        \end{itemize}
    \end{enumerate}
\end{frame}

\begin{frame}[fragile]
    \frametitle{Key Points and Conclusion}
    \begin{block}{Key Points to Emphasize}
        \begin{itemize}
            \item CSPs structure problem-solving for various domains.
            \item Identifying variables, domains, and constraints is critical.
            \item Algorithms (e.g., backtracking, constraint propagation) aid in efficient solutions.
        \end{itemize}
    \end{block}
    \begin{block}{Conclusion}
        CSPs are essential in solving practical problems across multiple domains. The structured framework facilitates effective modeling and strategic solutions. Understanding these applications equips students for systematically addressing complex challenges in their careers.
    \end{block}
\end{frame}

\begin{frame}[fragile]
    \frametitle{Additional Resources}
    \begin{itemize}
        \item Explore programming languages and libraries for CSP solving, such as:
        \begin{itemize}
            \item Python with Google’s OR-Tools
        \end{itemize}
    \end{itemize}
\end{frame}

\begin{frame}[fragile]{Components of CSPs - Learning Objectives}
    \begin{itemize}
        \item Understand the main components of Constraint Satisfaction Problems (CSPs).
        \item Identify how variables, domains, and constraints interact within CSPs.
    \end{itemize}
\end{frame}

\begin{frame}[fragile]{Components of CSPs - Variables}
    \begin{block}{1. Variables}
        \begin{itemize}
            \item \textbf{Definition:} Variables are the unknowns we want to solve for in a CSP. They represent the different aspects of the problem we need to determine.
            \item \textbf{Example:} In a scheduling problem, the variables could be the times assigned to meetings: MeetingA, MeetingB, and MeetingC.
        \end{itemize}
    \end{block}
\end{frame}

\begin{frame}[fragile]{Components of CSPs - Domains}
    \begin{block}{2. Domains}
        \begin{itemize}
            \item \textbf{Definition:} The domain of a variable is the set of possible values that the variable can take. Each variable in a CSP has its own domain.
            \item \textbf{Example:} Continuing with the scheduling example:
            \begin{itemize}
                \item Domain of MeetingA: \{9 AM, 10 AM, 11 AM\}
                \item Domain of MeetingB: \{9 AM, 10 AM\}
                \item Domain of MeetingC: \{10 AM, 11 AM, 12 PM\}
            \end{itemize}
        \end{itemize}
    \end{block}
\end{frame}

\begin{frame}[fragile]{Components of CSPs - Constraints}
    \begin{block}{3. Constraints}
        \begin{itemize}
            \item \textbf{Definition:} Constraints are the restrictions or conditions that the variables must satisfy. They define the relationships between variables, limiting the possible combinations of values.
            \item \textbf{Types of Constraints:}
            \begin{itemize}
                \item \textbf{Unary Constraints:} Involve a single variable (e.g., MeetingC can only occur after 10 AM).
                \item \textbf{Binary Constraints:} Involve two variables (e.g., MeetingA and MeetingB cannot be scheduled at the same time).
                \item \textbf{Global Constraints:} Involve a set of variables and express a more complex relationship (e.g., All meetings must occur in different time slots).
            \end{itemize}
            \item \textbf{Example:} If MeetingA and MeetingB cannot overlap, a binary constraint could be: \( MeetingA \neq MeetingB \).
        \end{itemize}
    \end{block}
\end{frame}

\begin{frame}[fragile]{Key Points and Conclusion}
    \begin{block}{Key Points to Emphasize}
        \begin{itemize}
            \item \textbf{Interrelationships:} The values chosen for variables are constrained by their domains and constraints. A valid solution to a CSP is one where all variables have values from their respective domains that satisfy all constraints.
            \item \textbf{Real-World Relevance:} Understanding these components enables the modeling of complex problems, such as project planning or resource allocation, in an efficient and manageable way.
        \end{itemize}
    \end{block}
    
    \begin{block}{Conclusion}
        CSPs provide a structured framework for solving problems involving multiple variables and constraints. Grasping the components of variables, domains, and constraints is crucial for successful problem formulation and solution in various applications.
    \end{block}
\end{frame}

\begin{frame}[fragile]
    \frametitle{Types of Constraints}
    \begin{block}{Learning Objectives}
        \begin{itemize}
            \item To understand the different types of constraints in Constraint Satisfaction Problems (CSPs).
            \item To identify and differentiate between unary, binary, and global constraints.
            \item To apply these concepts in formulating and solving CSPs.
        \end{itemize}
    \end{block}
\end{frame}

\begin{frame}[fragile]
    \frametitle{Introduction to Constraints}
    \begin{block}{Overview}
        Constraints are essential components of Constraint Satisfaction Problems (CSPs), which dictate the permissible values for variables. 
        Understanding the types of constraints is key to effectively modeling and solving CSPs.
    \end{block}
\end{frame}

\begin{frame}[fragile]
    \frametitle{Types of Constraints - Unary}
    \begin{block}{Unary Constraints}
        \begin{itemize}
            \item \textbf{Definition:} Involves a single variable and restricts its possible values based on a specific condition.
            \item \textbf{Example:} Suppose we have a variable \( X \) representing the age of a person with a unary constraint \( X > 18 \) (the person must be older than 18).
            \item \textbf{Purpose:} Filters the domain of individual variables, simplifying problem solving.
        \end{itemize}
        \textbf{Key Point:} Unary constraints apply to one variable only and help narrow down its potential values.
    \end{block}
\end{frame}

\begin{frame}[fragile]
    \frametitle{Types of Constraints - Binary}
    \begin{block}{Binary Constraints}
        \begin{itemize}
            \item \textbf{Definition:} Involves two variables and specifies the allowable combinations of their values.
            \item \textbf{Example:} For variables \( A \) (color of a shirt) and \( B \) (color of pants), a binary constraint could state \( A \neq B \) (both cannot be the same color).
            \item \textbf{Purpose:} Focuses on the relationships between pairs of variables, ensuring conditions involving them are met.
        \end{itemize}
        \textbf{Key Point:} Essential for establishing relationships between variables in applications like scheduling and resource allocation.
    \end{block}
\end{frame}

\begin{frame}[fragile]
    \frametitle{Types of Constraints - Global}
    \begin{block}{Global Constraints}
        \begin{itemize}
            \item \textbf{Definition:} Involves multiple variables and expresses a complex relationship within a CSP.
            \item \textbf{Example:} A global constraint like "all-different" states that every variable \( X_1, X_2, \ldots, X_n \) must have a different value:
            \[
            \text{all-different}(X_1, X_2, \ldots, X_n) \implies X_i \neq X_j \quad \forall i,j, \; 1 \leq i < j \leq n
            \]
            \item \textbf{Purpose:} Efficiently encapsulates common requirements, reducing the need for multiple binary constraints.
        \end{itemize}
        \textbf{Key Point:} Global constraints enhance efficiency in CSPs by minimizing the number of constraints needed.
    \end{block}
\end{frame}

\begin{frame}[fragile]
    \frametitle{Summary}
    \begin{block}{Conclusion}
        Understanding the different types of constraints—unary, binary, and global—is foundational in modeling CSPs. Each type serves a unique purpose in defining how variables can interact, guiding towards efficient solutions.
    \end{block}
    \begin{block}{Next Steps}
        By mastering these concepts, students will be better equipped to formulate CSP problems effectively, paving the way for further exploration in solving CSPs.
    \end{block}
\end{frame}

\begin{frame}[fragile]
    \frametitle{CSP Formulation}
    \begin{block}{Learning Objectives}
        \begin{itemize}
            \item Understand the fundamental components of a Constraint Satisfaction Problem (CSP).
            \item Learn to identify variables, domains, and constraints within a problem.
            \item Gain practical skills in formulating a problem as a CSP using a structured approach.
        \end{itemize}
    \end{block}
\end{frame}

\begin{frame}[fragile]
    \frametitle{Understanding CSP Basics}
    A \textbf{Constraint Satisfaction Problem (CSP)} consists of:
    \begin{itemize}
        \item \textbf{Variables}: Unknown values that need to be determined.
        \item \textbf{Domains}: Sets of possible values for each variable.
        \item \textbf{Constraints}: Rules that restrict the values that the variables can take.
    \end{itemize}
\end{frame}

\begin{frame}[fragile]
    \frametitle{Steps to Formulate a Problem as a CSP}
    \begin{enumerate}
        \item \textbf{Define Variables}
            \begin{itemize}
                \item Identify key elements to solve.
                \item \textit{Example}: In a Sudoku puzzle, variables are the empty cells.
            \end{itemize}
        
        \item \textbf{Define Domains}
            \begin{itemize}
                \item Assign permissible values for each variable.
                \item \textit{Example}: Each Sudoku cell can take values from \{1, 2, ..., 9\}.
            \end{itemize}
        
        \item \textbf{Define Constraints}
            \begin{itemize}
                \item Identify rules governing relationships between variables.
                \item Types of constraints:
                    \begin{itemize}
                        \item \textbf{Unary}: Conditions on a single variable.
                        \item \textbf{Binary}: Conditions between pairs of variables.
                        \item \textbf{Global}: Conditions involving multiple variables.
                    \end{itemize}
            \end{itemize}
    \end{enumerate}
\end{frame}

\begin{frame}[fragile]
    \frametitle{Example of CSP Formulation}
    \textbf{Problem Statement:} Solve a simple Sudoku puzzle.
    \begin{itemize}
        \item \textbf{Variables:} Let \(X_{i,j}\) represent the cell in the \(i^{th}\) row and \(j^{th}\) column.
        \item \textbf{Domains:} Each \(X_{i,j}\) can take values from \{1, 2, ..., 9\}.
        \item \textbf{Constraints:} 
            \begin{itemize}
                \item \(X_{i,j} \neq X_{k,l}\) for all cells in the same row, column, or 3x3 block.
            \end{itemize}
    \end{itemize}
\end{frame}

\begin{frame}[fragile]
    \frametitle{Key Points to Emphasize}
    \begin{itemize}
        \item Proper formulation of a CSP is crucial for effective problem-solving.
        \item Each component (variables, domains, constraints) defines the search space for solutions.
        \item Recognizing the types of constraints (unary, binary, global) aids in navigating the solution space.
    \end{itemize}

    \begin{block}{Helpful Notations}
        \begin{itemize}
            \item Let \(D(X)\) denote the domain of variable \(X\).
            \item A constraint can be noted as \(C(X_1, X_2, \ldots, X_n)\).
        \end{itemize}
    \end{block}
\end{frame}

\begin{frame}[fragile]
    \frametitle{Search Strategies for Solving CSPs - Overview}
    \begin{block}{Learning Objectives}
        \begin{itemize}
            \item Understand the main types of search strategies used in Constraint Satisfaction Problems (CSPs).
            \item Identify the backtracking algorithm as a fundamental approach for solving CSPs.
            \item Recognize other search techniques that complement or enhance backtracking.
        \end{itemize}
    \end{block}
\end{frame}

\begin{frame}[fragile]
    \frametitle{Search Strategies for Solving CSPs - Backtracking}
    \begin{block}{Backtracking}
        \begin{itemize}
            \item \textbf{Definition}: A systematic method for exploring possible configurations of variable assignments, undoing decisions that lead to conflicts.
            \item \textbf{Process}:
            \begin{enumerate}
                \item Start with an empty assignment.
                \item Assign values to variables sequentially, checking for violations of constraints.
                \item If a violation occurs, backtrack to the previous assignment and try the next value.
                \item Repeat until a solution is found or all possibilities are exhausted.
            \end{enumerate}
            \item \textbf{Example}:
            \begin{lstlisting}
                Assign Class A to Time Slot 1
                Assign Class B to Time Slot 2 (check for overlap)
                If no overlap, assign Class C, else backtrack and try a new time slot for Class A
            \end{lstlisting}
        \end{itemize}
    \end{block}
\end{frame}

\begin{frame}[fragile]
    \frametitle{Search Strategies for Solving CSPs - Other Techniques}
    \begin{block}{Other Search Strategies}
        \begin{itemize}
            \item \textbf{Forward Checking}
                \begin{itemize}
                    \item \textbf{Definition}: Extends backtracking by checking all future variables’ domains as assignments are made.
                    \item \textbf{Example}:
                    \begin{lstlisting}
                        Color Vertex U = Red 
                        Thus, remove Red from Vertices V and W
                    \end{lstlisting}
                \end{itemize}
            \item \textbf{Constraint Propagation}
                \begin{itemize}
                    \item \textbf{Definition}: Enforces constraints to reduce variable domains before search begins.
                    \item \textbf{Example}: Reduces possible values in a Sudoku puzzle through predefined rules.
                \end{itemize}
            \item \textbf{Heuristic Search}
                \begin{itemize}
                    \item \textbf{Definition}: Uses heuristics to guide the search process towards promising areas of the solution space.
                    \item \textbf{Examples of Heuristics}:
                    \begin{itemize}
                        \item Minimum Remaining Values (MRV)
                        \item Degree Heuristic
                    \end{itemize}
                    \item \textbf{Application}: In the n-queens problem, selecting the position that threatens the most other queens.
                \end{itemize}
        \end{itemize}
    \end{block}
\end{frame}

\begin{frame}[fragile]
    \frametitle{Backtracking Algorithms - Overview}
    \begin{block}{What is Backtracking?}
        Backtracking is a systematic method for exploring all possible configurations in a problem space, 
        particularly useful for solving Constraint Satisfaction Problems (CSPs). 
        It incrementally builds candidates to solutions and abandons candidates that cannot lead to a valid solution.
    \end{block}

    \begin{block}{Key Features}
        \begin{itemize}
            \item Depth-first search approach.
            \item Candidates are built incrementally.
            \item Solutions are explored until constraints are violated.
        \end{itemize}
    \end{block}
\end{frame}

\begin{frame}[fragile]
    \frametitle{Backtracking Algorithms - Process}
    \begin{block}{How Backtracking Works}
        \begin{enumerate}
            \item \textbf{Choosing a Variable:} Start with a variable.
            \item \textbf{Selecting a Value:} Assign a value from its domain.
            \item \textbf{Checking Constraints:}
                \begin{itemize}
                    \item If valid, proceed to the next variable.
                    \item If invalid, backtrack and try the next value.
                \end{itemize}
            \item \textbf{Repeating the Process:} Until all variables are assigned or options are exhausted.
        \end{enumerate}
    \end{block}
\end{frame}

\begin{frame}[fragile]
    \frametitle{Backtracking Algorithms - Example and Complexity}
    \begin{block}{Example: 8-Queens Problem}
        \begin{itemize}
            \item \textbf{Variables:} Columns of the chessboard (1 to 8).
            \item \textbf{Domains:} Integer values (1 to 8 for rows).
            \item \textbf{Constraints:} No two queens can threaten each other.
        \end{itemize}
    \end{block}

    \begin{block}{Backtracking Steps}
        \begin{itemize}
            \item Start at the first column.
            \item Attempt to place queens in permissible rows.
            \item Backtrack if queens threaten each other until a solution or impossibility is reached.
        \end{itemize}
    \end{block}

    \begin{block}{Key Points}
        \begin{enumerate}
            \item Backtracking can be computationally expensive.
            \item Techniques like pruning (constraint propagation) are vital.
            \item Worst-case time complexity is exponential.
        \end{enumerate}
    \end{block}
\end{frame}

\begin{frame}[fragile]
    \frametitle{Backtracking Algorithms - Pseudocode}
    \begin{block}{Illustrative Pseudocode}
    \begin{lstlisting}[language=Python]
def backtrack(assignment):
    if is_complete(assignment):
        return assignment
    variable = select_unassigned_variable()
    for value in variable.domain:
        if is_consistent(variable, value, assignment):
            assignment[variable] = value
            result = backtrack(assignment)
            if result:
                return result
            del assignment[variable]  # undo assignment
    return None  # No solution
    \end{lstlisting}
    \end{block}
\end{frame}

\begin{frame}[fragile]
    \frametitle{Backtracking Algorithms - Conclusion}
    \begin{block}{Conclusion}
        Backtracking algorithms are powerful tools for solving CSPs. They efficiently explore potential solutions while navigating constraints. 
        Practice with various CSPs using backtracking will enhance understanding and proficiency in algorithm design.
    \end{block}
\end{frame}

\begin{frame}[fragile]
    \frametitle{Heuristics in CSPs - Introduction}
    \begin{block}{Definition of Heuristics}
        Heuristics are strategies or rules of thumb that aim to provide efficient problem-solving methods for complex issues like CSPs.
    \end{block}

    \begin{block}{Why Use Heuristics?}
        The search space in CSPs can grow exponentially with the number of variables and constraints. Heuristics help optimize the search for solutions by making informed decisions.
    \end{block}
\end{frame}

\begin{frame}[fragile]
    \frametitle{Heuristics in CSPs - Key Strategies}
    \begin{enumerate}
        \item \textbf{Variable Ordering Heuristics}
            \begin{itemize}
                \item \textbf{Most Constrained Variable (MCV)}: Selects the variable with the fewest legal values left.
                \item \textbf{Most Constraining Variable (ACV)}: Chooses the variable that rules out the largest number of values for remaining variables.
            \end{itemize}
        
        \item \textbf{Value Ordering Heuristics}
            \begin{itemize}
                \item \textbf{Least Constraining Value (LCV)}: Selects the value that rules out the fewest values in neighboring variables.
                \item \textbf{Most Preferred Value}: Pre-prioritizes values based on criteria like historical success.
            \end{itemize}
    \end{enumerate}
\end{frame}

\begin{frame}[fragile]
    \frametitle{Heuristics in CSPs - Example and Summary}
    \begin{block}{Illustrative Example}
        \textbf{CSP Example: Scheduling Classes}
        \begin{itemize}
            \item \textbf{Variables}: Classes A, B, C, D
            \item \textbf{Domains}: Each class can be at Morning (M), Afternoon (A), or Evening (E).
            \item \textbf{Constraints}: 
                \begin{itemize}
                    \item Classes A and B cannot be at the same time.
                    \item Class C must be at a different time than Class D.
                \end{itemize}
        \end{itemize}
        \textbf{Heuristic Application:} Use MCV to select Class A then LCV to assign A to Morning.
    \end{block}

    \begin{block}{Summary}
        Heuristics in CSPs significantly reduce search space and improve solving efficiency. Intelligent selection of variables and values leads to advanced problem-solving capabilities in AI.
    \end{block}
\end{frame}

\begin{frame}{AC-3 Algorithm}
    \frametitle{Learning Objectives}
    \begin{itemize}
        \item Understand the purpose of the AC-3 algorithm in Constraint Satisfaction Problems (CSPs).
        \item Learn how the algorithm operates step-by-step.
        \item Recognize the significance of arc consistency and its impact on solving CSPs.
    \end{itemize}
\end{frame}

\begin{frame}{What is the AC-3 Algorithm?}
    The AC-3 (Arc-Consistency 3) algorithm is a fundamental method used to achieve arc consistency in Constraint Satisfaction Problems (CSPs). Arc consistency ensures that for every variable, all the values in its domain have at least one valid support from neighboring variables.
    \begin{block}{Key Concepts}
        \begin{itemize}
            \item \textbf{CSP}: A mathematical problem defined by a set of variables, domains, and constraints.
            \item \textbf{Arc Consistency}: A state where for each value of a variable, there exists a value in the connected variable that satisfies the constraints.
        \end{itemize}
    \end{block}
\end{frame}

\begin{frame}{How the AC-3 Algorithm Works}
    \begin{enumerate}
        \item \textbf{Initialization}:
        \begin{itemize}
            \item Begin with a queue containing all arcs in the CSP, e.g., \( (X_i, X_j) \).
        \end{itemize}
        
        \item \textbf{Main Loop}:
        \begin{itemize}
            \item While the queue is not empty:
                \begin{itemize}
                    \item Remove an arc \( (X_i, X_j) \).
                    \item Check each value \( a \) in the domain of \( X_i \):
                        \begin{itemize}
                            \item Remove \( a \) if it lacks support from \( X_j \)'s domain.
                        \end{itemize}
                    \item If \( X_i \)'s domain is modified, add arcs back for its neighbors.
                \end{itemize}
        \end{itemize}
        
        \item \textbf{Termination}: Continues until the queue is empty or a domain is empty.
    \end{enumerate}
\end{frame}

\begin{frame}{Example of AC-3 Algorithm}
    Consider a simple CSP with variables \( X_1 \) and \( X_2 \) having:
    \begin{itemize}
        \item \( D(X_1) = \{1, 2\} \)
        \item \( D(X_2) = \{2, 3\} \)
    \end{itemize}
    with the constraint \( X_1 \neq X_2 \).

    \textbf{AC-3 Steps}:
    \begin{enumerate}
        \item Initialize with arcs: \( (X_1, X_2) \), \( (X_2, X_1) \).
        \item Dequeue \( (X_1, X_2) \): No changes.
        \item Dequeue \( (X_2, X_1) \):
        \begin{itemize}
            \item Remove \( 3 \) from \( D(X_2) \), leading to \( D(X_2) = \{2\} \).
        \end{itemize}
        \item Termination with no further changes.
    \end{enumerate}
\end{frame}

\begin{frame}[fragile]{Code Snippet for AC-3}
    \begin{lstlisting}[language=Python]
def ac_3(csp):
    queue = [(xi, xj) for (xi, xj) in csp.constraints]
    while queue:
        (X_i, X_j) = queue.pop(0)
        if revise(csp, X_i, X_j):
            if not csp.domains[X_i]:  # Domain wiped
                return False
            for X_k in csp.neighbors[X_i]:
                if X_k != X_j:
                    queue.append((X_k, X_i))
    return True

def revise(csp, X_i, X_j):
    revised = False
    for value_a in csp.domains[X_i][:]:
        if not any(satisfaction(value_a, value_b) for value_b in csp.domains[X_j]):
            csp.domains[X_i].remove(value_a)
            revised = True
    return revised
    \end{lstlisting}
\end{frame}

\begin{frame}{Summary}
    The AC-3 algorithm is a powerful tool in constraint satisfaction, improving efficiency by enforcing arc consistency. It aids in pruning the search space and is essential for understanding and solving CSPs effectively.
\end{frame}

\begin{frame}[fragile]
    \frametitle{Constraint Satisfaction vs. Optimization Problems}
    
    \textbf{Learning Objectives:}
    \begin{itemize}
        \item Understand the distinction between Constraint Satisfaction Problems (CSPs) and optimization problems.
        \item Recognize how CSPs can be viewed as a subset of optimization problems.
        \item Identify real-world applications for both types of problems.
    \end{itemize}
\end{frame}

\begin{frame}[fragile]
    \frametitle{Definitions}
    
    \textbf{Constraint Satisfaction Problems (CSPs):}
    \begin{itemize}
        \item A CSP involves finding values for a set of variables that satisfy specific constraints.
        \item Each variable has a finite domain, and the solution must meet defined constraints.
    \end{itemize}
    
    \textbf{Example:}
    \begin{itemize}
        \item \textbf{Variables:} $X_1$, $X_2$
        \item \textbf{Domains:} $D(X_1) = \{1, 2\}$, $D(X_2) = \{1, 2\}$
        \item \textbf{Constraints:} $X_1 \neq X_2$
        \item \textbf{Solution:} $\{X_1: 1, X_2: 2\}$ (or vice versa)
    \end{itemize}
    
    \textbf{Optimization Problems:}
    \begin{itemize}
        \item An optimization problem seeks to find the best solution defined by an objective function that needs to be maximized or minimized.
    \end{itemize}
    
    \textbf{Example:}
    \begin{itemize}
        \item \textbf{Objective Function:} Minimize Cost $= 3X_1 + 5X_2$
        \item \textbf{Constraints:} $X_1 + X_2 \leq 10$, $X_1 \geq 0$, $X_2 \geq 0$
        \item \textbf{Optimal Solution:} Find values for $X_1$ and $X_2$ that minimize cost while satisfying constraints.
    \end{itemize}
\end{frame}

\begin{frame}[fragile]
    \frametitle{Key Differences}
    
    \begin{block}{Comparison Table}
        \begin{center}
            \begin{tabular}{|l|l|l|}
                \hline
                \textbf{Feature} & \textbf{CSPs} & \textbf{Optimization Problems} \\
                \hline
                Goal & Find feasible solutions within constraints & Optimize objective function \\
                \hline
                Solution Type & Single or multiple valid assignments & Unique or multiple optimal values \\
                \hline
                Output & Values that meet all constraints & Optimal value(s) and associated variable values \\
                \hline
                Complexity & Typically polynomial time & May involve NP-hard problems \\
                \hline
                Subset Relation & Every CSP can be framed as an optimization problem & Not every optimization problem is a CSP \\
                \hline
            \end{tabular}
        \end{center}
    \end{block}
\end{frame}

\begin{frame}[fragile]
    \frametitle{CSP as a Subset}
    
    \begin{itemize}
        \item \textbf{Optimization as an Extension:} 
        Each CSP can be transformed into an optimization problem by defining an objective function that measures constraint satisfaction (e.g., minimizing the number of constraint violations).
    \end{itemize}
    
    \textbf{Example of Transformation:}
    \begin{enumerate}
        \item A CSP: Assign colors to a map such that no adjacent regions share the same color.
        \item An Optimization Problem: Minimize the number of adjacent regions sharing the same color.
    \end{enumerate}
\end{frame}

\begin{frame}[fragile]
    \frametitle{Applications}
    
    \begin{itemize}
        \item \textbf{CSP Applications:}
        \begin{itemize}
            \item Scheduling
            \item Resource allocation
            \item Configuration problems
        \end{itemize}
        
        \item \textbf{Optimization Applications:}
        \begin{itemize}
            \item Financial portfolio management
            \item Transportation routing
            \item Production planning
        \end{itemize}
    \end{itemize}
\end{frame}

\begin{frame}[fragile]
    \frametitle{Summary of Key Points}
    
    \begin{itemize}
        \item CSPs focus on satisfying constraints; any solution must meet all conditions.
        \item Optimization Problems concentrate on maximizing or minimizing an objective.
        \item CSPs can serve as the basis for optimization problems when an objective function is added.
    \end{itemize}
\end{frame}

\begin{frame}[fragile]
    \frametitle{Conclusion}
    
    Understanding the differences and relationships between CSPs and optimization problems is crucial for effectively modeling and solving complex real-world issues. 
    As we proceed to the next slide on the N-Queens Problem, we will apply these concepts in practical scenarios.
\end{frame}

\begin{frame}[fragile]
    \frametitle{Example Problem: N-Queens Problem}
    \begin{block}{Overview}
        An overview of the N-Queens Problem, a classic example of a constraint satisfaction problem (CSP) where the goal is to place N queens on an N x N chessboard such that no two queens threaten each other.
    \end{block}
\end{frame}

\begin{frame}[fragile]
    \frametitle{Understanding the N-Queens Problem}
    \begin{itemize}
        \item \textbf{Definition:} The N-Queens Problem requires placing N queens on an NxN board without them threatening each other.
        \item \textbf{Threat condition:} Queens threaten each other if they are in the same row, column, or diagonal.
    \end{itemize}
\end{frame}

\begin{frame}[fragile]
    \frametitle{CSP Formulation}
    \begin{itemize}
        \item \textbf{Variables:} Each queen is represented as a variable \( Q_i \), \( i = 1, 2, \ldots, N \).
        \item \textbf{Domains:} The domain for each \( Q_i \) is the set of columns \( \{1, 2, \ldots, N\} \).
        \item \textbf{Constraints:}
        \begin{enumerate}
            \item Row Constraints: Each queen in a different row.
            \item Column Constraints: \( Q_i \neq Q_j \) for \( i \neq j \).
            \item Diagonal Constraints: \( |Q_i - Q_j| \neq |i - j| \) for \( i \neq j \).
        \end{enumerate}
    \end{itemize}
\end{frame}

\begin{frame}[fragile]
    \frametitle{Example: Solving the 4-Queens Problem}
    \begin{block}{Approaches}
        \textbf{Brute Force Approach:}
        \begin{itemize}
            \item Generate all possible arrangements of queens.
            \item Verify each arrangement against constraints.
        \end{itemize}

        \textbf{Backtracking Algorithm:}
        \begin{itemize}
            \item Place the first queen and attempt valid placements recursively.
            \item Backtrack if no valid position is found.
        \end{itemize}
    \end{block}
\end{frame}

\begin{frame}[fragile]{Code Snippet for Backtracking}
    \begin{lstlisting}[language=Python]
def is_safe(board, row, col):
    for i in range(row):
        if board[i] == col or abs(board[i] - col) == abs(i - row):
            return False
    return True

def solve_n_queens(board, row, N):
    if row == N:
        return [board[:]]  # Found a solution
    solutions = []
    for col in range(N):
        if is_safe(board, row, col):
            board[row] = col
            solutions += solve_n_queens(board, row + 1, N)
    return solutions

N = 4
board = [-1] * N  # Initialize board
solutions = solve_n_queens(board, 0, N)
print(solutions)
    \end{lstlisting}
\end{frame}

\begin{frame}[fragile]
    \frametitle{Key Points to Emphasize}
    \begin{itemize}
        \item The N-Queens Problem demonstrates modeling of real-world issues via CSP.
        \item Variables, domains, and constraints define the problem space clearly.
        \item Backtracking serves as an effective method for solving CSPs by exploring potential solutions and pruning inviable paths.
    \end{itemize}
\end{frame}

\begin{frame}[fragile]
    \frametitle{Example Problem: Sudoku}
    \begin{block}{Modeling Sudoku as a CSP}
        \begin{itemize}
            \item Sudoku is a number-placement puzzle on a 9x9 grid.
            \item Goal: Each row, column, and 3x3 subgrid contains digits 1-9 without repetition.
        \end{itemize}
    \end{block}
\end{frame}

\begin{frame}[fragile]
    \frametitle{Sudoku as a CSP}
    \begin{block}{CSP Framework}
        \begin{itemize}
            \item \textbf{Variables:} 81 variables for each cell in the grid.
            \item \textbf{Domains:} 
                \begin{itemize}
                    \item For empty cells: all numbers {1, 2, ..., 9}.
                    \item For filled cells: limited to the number already present.
                \end{itemize}
            \item \textbf{Constraints:} 
                \begin{itemize}
                    \item Row Constraints: each number must be unique in rows.
                    \item Column Constraints: each number must be unique in columns.
                    \item Box Constraints: each number must be unique in 3x3 subgrids.
                \end{itemize}
        \end{itemize}
    \end{block}
\end{frame}

\begin{frame}[fragile]
    \frametitle{Backtracking Solution Approach}
    \begin{block}{Steps in Backtracking}
        \begin{enumerate}
            \item Start with an empty cell and choose a value from its domain.
            \item Check for constraints:
                \begin{itemize}
                    \item If satisfied, proceed to the next empty cell.
                \end{itemize}
            \item Continue until:
                \begin{itemize}
                    \item A solution is found (all cells filled correctly).
                    \item An assignment fails (backtrack and try next value).
                \end{itemize}
            \item Visualize backtracking as a tree structure.
        \end{enumerate}
    \end{block}

    \begin{block}{Key Points}
        \begin{itemize}
            \item Understand the structure of CSPs.
            \item Backtracking's recursive nature is crucial for efficiency.
            \item Real-world applications extend beyond Sudoku.
        \end{itemize}
    \end{block}
\end{frame}

\begin{frame}[fragile]
    \frametitle{Conclusion}
    \begin{block}{Significance of CSP in Sudoku}
        \begin{itemize}
            \item Designing and solving Sudoku enriches the problem-solving toolkit.
            \item Shows how backtracking navigates through constraints.
            \item Foundational in computer science and AI, applicable to other domains.
        \end{itemize}
    \end{block}
\end{frame}

\begin{frame}[fragile]
    \frametitle{Complexity in CSPs - Overview}
    \begin{block}{Understanding Computational Complexity}
        \textbf{Constraint Satisfaction Problems (CSPs)} involve finding assignments to a set of variables under specific constraints. They are relevant in various real-world scenarios such as scheduling, Sudoku, and map coloring.
    \end{block}
\end{frame}

\begin{frame}[fragile]
    \frametitle{CSP Definition and Complexity Classes}
    \begin{itemize}
        \item \textbf{CSP Definition}:
        \begin{itemize}
            \item A CSP is characterized by:
            \begin{itemize}
                \item A set of variables \(X = \{x_1, x_2, \ldots, x_n\}\)
                \item A set of domains \(D = \{D_1, D_2, \ldots, D_n\}\)
                \item A set of constraints \(C\) defining valid value combinations.
            \end{itemize}
        \end{itemize}
        
        \item \textbf{Complexity Classes}:
        \begin{itemize}
            \item **Polynomial Time**: Some CSPs can be solved efficiently when the problem structure is simple.
            \item **NP-Hard**: Many CSPs are NP-hard, where no algorithm efficiently solves all instances (e.g., Sudoku).
        \end{itemize}
    \end{itemize}
\end{frame}

\begin{frame}[fragile]
    \frametitle{Impact of Constraints on Complexity}
    \begin{itemize}
        \item \textbf{Tightness of Constraints}:
        \begin{itemize}
            \item Tighter constraints can simplify the search space but may lead to unsolvable problems.
        \end{itemize}

        \item \textbf{Constraint Types}:
        \begin{itemize}
            \item \textbf{Binary Constraints}: Involve two variables (e.g., adjacent nodes in graph coloring).
            \item \textbf{Global Constraints}: Involve multiple variables (e.g., all-different constraints).
        \end{itemize}
    \end{itemize}
\end{frame}

\begin{frame}[fragile]
    \frametitle{Examples of CSP Complexity}
    \begin{enumerate}
        \item \textbf{Bin Packing Problem (NP-Hard)}:
        \begin{itemize}
            \item Pack a set of items into bins without exceeding a maximum weight limit.
        \end{itemize}
        
        \item \textbf{N-Queens Problem}:
        \begin{itemize}
            \item Place \(N\) queens on an \(N \times N\) chessboard such that no queens threaten each other. Solvable via backtracking in \(O(N!)\).
        \end{itemize}
    \end{enumerate}
\end{frame}

\begin{frame}[fragile]
    \frametitle{Key Points and Conclusion}
    \begin{itemize}
        \item CSPs represent various real-world problems characterized by constraints.
        \item The complexity of CSPs is determined by variable interdependencies and constraint types.
        \item Understanding the complexity classes (P vs. NP-hard) is crucial for algorithm design.
    \end{itemize}
    
    \textbf{Conclusion:} CSPs are foundational in AI and operational research, influencing applications from logistics to game development.
\end{frame}

\begin{frame}[fragile]
    \frametitle{Real-Time CSP Applications - Overview}
    \begin{block}{Introduction}
        Constraint Satisfaction Problems (CSPs) are mathematical problems defined as a set of objects whose states must satisfy specific constraints. Real-time applications of CSPs are vital in many domains, particularly in robotics and game AI, where rapid decision-making is required.
    \end{block}
    
    \begin{block}{Key Points}
        \begin{itemize}
            \item CSPs consist of variables, domains, and constraints.
            \item Real-time applications need solutions within predefined time constraints.
            \item Quick response is crucial for success in dynamic environments.
        \end{itemize}
    \end{block}
\end{frame}

\begin{frame}[fragile]
    \frametitle{Real-Time CSP Applications - Robotics}
    \begin{block}{Applications in Robotics}
        \begin{itemize}
            \item \textbf{Path Planning:} Robots navigate through environments avoiding obstacles.
            \item \textbf{Multi-Robot Coordination:} Assigning tasks/paths to multiple robots while preventing interference.
        \end{itemize}
    \end{block}
    
    \begin{exampleblock}{Example - Path Planning}
        \begin{itemize}
            \item Variables: \((x, y)\) coordinates of the robot
            \item Domain: Possible coordinates within the grid
            \item Constraints: No overlap with obstacles, path continuity
        \end{itemize}
    \end{exampleblock}
\end{frame}

\begin{frame}[fragile]
    \frametitle{Real-Time CSP Applications - Game AI}
    \begin{block}{Applications in Game AI}
        \begin{itemize}
            \item \textbf{Character Behavior:} AI characters decide on paths and actions based on environmental constraints.
            \item \textbf{Resource Management:} Allocating resources effectively under constraints.
        \end{itemize}
    \end{block}
    
    \begin{exampleblock}{Example - Resource Management}
        \begin{itemize}
            \item Troop allocation in a strategy game while minimizing travel time and ensuring defense.
        \end{itemize}
    \end{exampleblock}

    \begin{block}{Conclusion}
        Real-time CSP applications illustrate the versatility of CSPs. Understanding CSP formulation is essential for developing intelligent systems that adapt to dynamic environments.
    \end{block}
\end{frame}

\begin{frame}[fragile]
    \frametitle{Challenges and Limitations of CSPs - Introduction}
    \begin{block}{Overview}
        Constraint Satisfaction Problems (CSPs) are powerful tools for modeling and solving complex issues. However, several challenges and limitations can result in oversimplifications. Understanding these challenges is crucial for effective application.
    \end{block}
\end{frame}

\begin{frame}[fragile]
    \frametitle{Challenges in CSPs - Complexity and Scalability}
    \begin{itemize}
        \item \textbf{Exponential Growth:} The solution space can grow exponentially with more variables and constraints.
        \item \textbf{Example:} 
        \begin{itemize}
            \item A scheduling problem with 10 tasks, each having 2 possible time slots, results in $2^{10} = 1,024$ combinations.
        \end{itemize}
    \end{itemize}
\end{frame}

\begin{frame}[fragile]
    \frametitle{Challenges in CSPs - Inherent Incompleteness}
    \begin{itemize}
        \item \textbf{Undecidable Problems:} Some CSPs can't guarantee a solution in finite time due to inherent complexities.
        \item \textbf{Example:} 
        \begin{itemize}
            \item The “Three-Coloring Problem” measures if a graph can be colored using three colors without two adjacent nodes sharing the same color; for certain graphs, this is undecidable.
        \end{itemize}
    \end{itemize}
\end{frame}

\begin{frame}[fragile]
    \frametitle{Challenges in CSPs - Over-Simplification Risks}
    \begin{itemize}
        \item \textbf{Real-World Dynamics:} Simplifying real-world problems can overlook essential factors like time and resource limits.
        \item \textbf{Illustration:} 
        \begin{itemize}
            \item In robotic navigation, a CSP ignoring dynamic obstacles results in theoretical pathways that are not practical.
        \end{itemize}
    \end{itemize}
\end{frame}

\begin{frame}[fragile]
    \frametitle{Challenges in CSPs - Quality of Constraints}
    \begin{itemize}
        \item \textbf{Poor Constraint Design:} Effectiveness relies on the quality of constraints; weak or overly restrictive constraints hinder finding feasible solutions.
        \item \textbf{Example:} 
        \begin{itemize}
            \item In job allocation, inadequate constraints may result in tasks that are unassignable, causing redundancy.
        \end{itemize}
    \end{itemize}
\end{frame}

\begin{frame}[fragile]
    \frametitle{Challenges in CSPs - Interaction of Constraints}
    \begin{itemize}
        \item \textbf{Constraint Interaction Bottleneck:} Conflicts can complicate the search for solutions as constraints interact in unpredictable ways.
        \item \textbf{Example:} 
        \begin{itemize}
            \item In chess, managing movement constraints appropriately is essential; otherwise, conflicts may make solving more complex.
        \end{itemize}
    \end{itemize}
\end{frame}

\begin{frame}[fragile]
    \frametitle{Key Points and Conclusion}
    \begin{itemize}
        \item Recognize exponential complexity in CSPs, especially as models scale.
        \item Understand the risks of oversimplification when modeling real-world problems.
        \item Thoughtfully design constraints for effective and practical solutions.
    \end{itemize}
    \begin{block}{Conclusion}
        CSPs are powerful but must be approached with an understanding of their limitations. Recognizing these challenges leads to improved strategies for developing CSP applications.
    \end{block}
\end{frame}

\begin{frame}[fragile]
    \frametitle{Call to Action}
    \begin{itemize}
        \item Explore real-world CSP examples and analyze their associated challenges.
        \item Consider the implications of constraint design in your projects.
        \item Be mindful of oversimplifications, continuously refining constraints for enhanced solutions.
    \end{itemize}
\end{frame}

\begin{frame}[fragile]
    \frametitle{Conclusion and Summary - Overview of CSPs}
    \begin{block}{What are CSPs?}
        Constraint Satisfaction Problems (CSPs) are a fundamental concept in artificial intelligence and operations research. The objective is to find a solution that satisfies a number of constraints.
    \end{block}
    
    \begin{itemize}
        \item \textbf{Key Components:}
        \begin{enumerate}
            \item Variables: The unknowns we want to solve for.
            \item Domains: The possible values each variable can take.
            \item Constraints: The rules that define valid combinations of values.
        \end{enumerate}
    \end{itemize}
\end{frame}

\begin{frame}[fragile]
    \frametitle{Conclusion and Summary - Significance of CSPs}
    \begin{itemize}
        \item CSPs are prevalent in various fields such as:
        \begin{itemize}
            \item Scheduling
            \item Resource allocation
            \item Circuit design
            \item Puzzle-solving (e.g., Sudoku)
        \end{itemize}

        \item \textbf{Real-World Examples:}
        \begin{enumerate}
            \item \textbf{Scheduling:} Assigning time slots to classes in a school while considering constraints such as instructor availability and classroom capacity.
            \item \textbf{Puzzle Solving:} Solving Sudoku puzzles, where each number must satisfy constraints of uniqueness in rows, columns, and grids.
        \end{enumerate}
    \end{itemize}
\end{frame}

\begin{frame}[fragile]
    \frametitle{Conclusion and Summary - Challenges and Takeaways}
    \begin{block}{Challenges Addressed}
        The previous slide detailed challenges in CSPs such as:
        \begin{itemize}
            \item Combinatorial explosion
            \item Incomplete information
            \item Potential oversimplifications in modeling real-world problems
        \end{itemize}
    \end{block}

    \begin{block}{Key Takeaway}
        Understanding these limitations is critical for effectively applying CSP techniques to complex problems.
    \end{block}

    \begin{itemize}
        \item \textbf{Key Points Recap:}
        \begin{enumerate}
            \item CSP Definition: A problem framing approach that involves finding assignments to variables under specific constraints.
            \item Applications: Widely used in AI for optimization problems, logistics, and decision-making processes.
            \item Problem-Solving Techniques:
            \begin{itemize}
                \item Backtracking Algorithms
                \item Constraint Propagation
            \end{itemize}
        \end{enumerate}
    \end{itemize}
\end{frame}


\end{document}