\documentclass[aspectratio=169]{beamer}

% Theme and Color Setup
\usetheme{Madrid}
\usecolortheme{whale}
\useinnertheme{rectangles}
\useoutertheme{miniframes}

% Additional Packages
\usepackage[utf8]{inputenc}
\usepackage[T1]{fontenc}
\usepackage{graphicx}
\usepackage{booktabs}
\usepackage{listings}
\usepackage{amsmath}
\usepackage{amssymb}
\usepackage{xcolor}
\usepackage{tikz}
\usepackage{pgfplots}
\pgfplotsset{compat=1.18}
\usetikzlibrary{positioning}
\usepackage{hyperref}

% Custom Colors
\definecolor{myblue}{RGB}{31, 73, 125}
\definecolor{mygray}{RGB}{100, 100, 100}
\definecolor{mygreen}{RGB}{0, 128, 0}
\definecolor{myorange}{RGB}{230, 126, 34}
\definecolor{mycodebackground}{RGB}{245, 245, 245}

% Set Theme Colors
\setbeamercolor{structure}{fg=myblue}
\setbeamercolor{frametitle}{fg=white, bg=myblue}
\setbeamercolor{title}{fg=myblue}
\setbeamercolor{section in toc}{fg=myblue}
\setbeamercolor{item projected}{fg=white, bg=myblue}
\setbeamercolor{block title}{bg=myblue!20, fg=myblue}
\setbeamercolor{block body}{bg=myblue!10}
\setbeamercolor{alerted text}{fg=myorange}

% Set Fonts
\setbeamerfont{title}{size=\Large, series=\bfseries}
\setbeamerfont{frametitle}{size=\large, series=\bfseries}
\setbeamerfont{caption}{size=\small}
\setbeamerfont{footnote}{size=\tiny}

% Custom Commands
\newcommand{\hilight}[1]{\colorbox{myorange!30}{#1}}
\newcommand{\concept}[1]{\textcolor{myblue}{\textbf{#1}}}

% Title Page Information
\title[Heuristic Search Methods]{Week 4: Heuristic Search Methods}
\author[J. Smith]{John Smith, Ph.D.}
\institute[University Name]{
  Department of Computer Science\\
  University Name\\
  \vspace{0.3cm}
  Email: email@university.edu\\
  Website: www.university.edu
}
\date{\today}

% Document Start
\begin{document}

\frame{\titlepage}

\begin{frame}[fragile]
    \frametitle{Introduction to Heuristic Search Methods}
    \begin{block}{Overview}
        Heuristic search methods are strategies used to solve optimization problems efficiently, especially in large search spaces where traditional algorithms may fail.
    \end{block}
\end{frame}

\begin{frame}[fragile]
    \frametitle{Key Concepts}
    \begin{enumerate}
        \item \textbf{Definition}: A heuristic is a rule of thumb that helps guide the search process towards a solution more quickly than a brute-force search method.
        \item \textbf{Importance}: Heuristic methods are vital in artificial intelligence, operations research, and computer science for complex problems (pathfinding, scheduling, etc.).
    \end{enumerate}
\end{frame}

\begin{frame}[fragile]
    \frametitle{Examples of Heuristic Search Methods}
    \begin{itemize}
        \item \textbf{Greedy Algorithms}: Choose the best option at each step; e.g., Kruskal’s or Prim’s Algorithm for minimum spanning trees.
        \item \textbf{A* Search Algorithm}: Combines actual cost to reach a node and estimated cost to reach the goal:
        \begin{equation}
            f(n) = g(n) + h(n)
        \end{equation}
        \begin{itemize}
            \item \( f(n) \): total estimated cost of the cheapest solution through node \( n \).
            \item \( g(n) \): cost from the start node to node \( n \).
            \item \( h(n) \): estimated cost from node \( n \) to the goal (heuristic).
        \end{itemize}
        \item \textbf{Genetic Algorithms}: Evolve solutions using natural selection principles—crossover, mutation, selection.
    \end{itemize}
\end{frame}

\begin{frame}[fragile]
    \frametitle{Key Points to Emphasize}
    \begin{itemize}
        \item \textbf{Efficiency Over Perfection}: Heuristic searches prioritize speed and resource utilization, suited for real-time applications.
        \item \textbf{Flexibility}: They adapt to different problem structures by altering guiding strategies.
        \item \textbf{Application Domains}: Commonly used in AI for game playing, robotics navigation, and network routing.
    \end{itemize}
\end{frame}

\begin{frame}[fragile]{What is Heuristic Search? - Definition}
    \begin{block}{Definition}
        Heuristic search is a strategy used to efficiently solve complex problems by employing "rules of thumb" or educated guesses. These methods incorporate knowledge about the problem domain to streamline the search process and find satisfactory solutions in a practical timeframe, rather than exhaustively evaluating every possible option.
    \end{block}
\end{frame}

\begin{frame}[fragile]{What is Heuristic Search? - Purpose in AI}
    \begin{block}{Purpose in AI}
        In the realm of Artificial Intelligence (AI), heuristic searches are crucial for:
        \begin{enumerate}
            \item \textbf{Optimization}: Navigating vast search spaces, particularly in optimization problems where finding the best solution is computationally expensive.
            \item \textbf{Problem Solving}: Providing approximate solutions to problems that may not be solvable in a reasonable time using traditional search techniques.
            \item \textbf{Efficiency}: Improving the speed and performance of algorithms by reducing the number of evaluations required to reach a solution.
        \end{enumerate}
    \end{block}
\end{frame}

\begin{frame}[fragile]{What is Heuristic Search? - Key Concepts and Example}
    \begin{block}{Key Concepts}
        \begin{itemize}
            \item \textbf{Heuristic Function}: A function estimating the cost of the cheapest path from a given state to the goal state.
            \item \textbf{Search Space}: The set of all possible states or configurations that can be explored during the search process.
        \end{itemize}
    \end{block}
    
    \begin{block}{Example - 8-Puzzle Problem}
        In this problem, a 3x3 grid contains 8 numbered tiles and one empty space. A heuristic such as the \textbf{Manhattan Distance} can be applied:
        \begin{itemize}
            \item If tile '1' is currently at (1, 2) but needs to be at (0, 0), its contribution to the heuristic would be:
            \[
            |1-0| + |2-0| = 3.
            \]
        \end{itemize}
    \end{block}
\end{frame}

\begin{frame}[fragile]
    \frametitle{Types of Heuristic Search Methods}
    % An overview of heuristic search methods and their importance in optimization problems.
    \begin{block}{Overview of Heuristic Search}
        Heuristic search methods leverage problem-specific knowledge to find efficient solutions. They are particularly useful for problems where traditional search methods are computationally expensive.
    \end{block}
    \begin{block}{Methods Discussed}
        \begin{itemize}
            \item Greedy Search
            \item A* Algorithm
            \item Hill Climbing
        \end{itemize}
    \end{block}
\end{frame}

\begin{frame}[fragile]
    \frametitle{Greedy Search}
    \begin{block}{Definition}
        Greedy Search is a heuristic search algorithm that makes the most immediate beneficial choice at each step, hoping to find a global optimum.
    \end{block}
    \begin{block}{Characteristics}
        \begin{itemize}
            \item \textbf{Local Optimum}: Attempts to optimize the objective by choosing the best local option.
            \item \textbf{Efficiency}: Often faster as it doesn't explore all possible solutions exhaustively.
        \end{itemize}
    \end{block}
    \begin{block}{Example}
        In a pathfinding problem, Greedy Search might choose the next node based on the shortest distance to the target node, ignoring the overall path cost. For instance, if at vertex A wanting to reach vertex C, it would select vertex B if the distance from B to C is shorter than from A to C.
    \end{block}
\end{frame}

\begin{frame}[fragile]
    \frametitle{A* Algorithm}
    \begin{block}{Definition}
        The A* algorithm is a pathfinding and graph traversal algorithm that combines features of Greedy Best-First Search and Dijkstra's algorithm.
    \end{block}
    \begin{block}{Characteristics}
        \begin{itemize}
            \item \textbf{Heuristic + Cost Function}: It uses a cost function \( f(n) = g(n) + h(n) \):
            \begin{itemize}
                \item \( g(n) \): cost from the start node to node \( n \).
                \item \( h(n) \): heuristic estimate of the cost from node \( n \) to the goal.
            \end{itemize}
            \item \textbf{Optimal}: A* guarantees the least cost path if the heuristic is admissible.
        \end{itemize}
    \end{block}
    \begin{block}{Example}
        For navigating a city, A* might use straight-line distance (Euclidean distance) as a heuristic while considering the actual distance traveled.
    \end{block}
\end{frame}

\begin{frame}[fragile]
    \frametitle{Hill Climbing}
    \begin{block}{Definition}
        Hill Climbing is a local search algorithm that continually moves in the direction of increasing elevation (or value) to find the peak.
    \end{block}
    \begin{block}{Characteristics}
        \begin{itemize}
            \item \textbf{Local Search}: Works on single solutions and explores their neighbors.
            \item \textbf{Greedy Nature}: Makes decisions based on local information.
            \item \textbf{Stuck at Local Maxima}: There is a risk of getting stuck in local maxima.
        \end{itemize}
    \end{block}
    \begin{block}{Example}
        For optimizing a function such as \( f(x) = -x^{2} + 10x \), it starts at an initial point, checks neighbors, and moves to the neighboring point with the highest function value.
    \end{block}
\end{frame}

\begin{frame}[fragile]
    \frametitle{Key Points to Emphasize}
    \begin{itemize}
        \item \textbf{Efficiency vs. Optimality}: Greedy Search and Hill Climbing focus on speed but risk missing the global optimum, while A* optimizes for both.
        \item \textbf{Choice of Heuristic}: The effectiveness of each algorithm heavily depends on the quality of heuristics used.
        \item \textbf{Applications}: These methods are crucial in AI for routing, game development, and other optimization problems.
    \end{itemize}
\end{frame}

\begin{frame}[fragile]{Problem Solving with Heuristic Search}
    \begin{block}{Understanding Heuristic Search}
        Heuristic search methods are strategies that improve the efficiency of problem-solving in complex optimization and decision-making scenarios. They provide "rule-of-thumb" approaches that help us find satisfactory solutions faster than classical methods while dealing with vast search spaces.
    \end{block}
\end{frame}

\begin{frame}[fragile]{Key Concepts}
    \begin{enumerate}
        \item \textbf{Heuristic Function (h(n))}:
        \begin{itemize}
            \item Estimates the cost from a given node (n) to the goal.
            \item Example: In a pathfinding scenario, h(n) could represent the straight-line distance between the current position and the destination.
        \end{itemize}

        \item \textbf{Search Space}:
        \begin{itemize}
            \item The set of all possible states or configurations to be explored to reach the solution.
            \item A small heuristic search can dramatically reduce this space.
        \end{itemize}

        \item \textbf{Optimal and Suboptimal Solutions}:
        \begin{itemize}
            \item Heuristic methods often yield suboptimal solutions (close but not perfect) quickly to problems that may be computationally intensive to solve fully.
        \end{itemize}
    \end{enumerate}
\end{frame}

\begin{frame}[fragile]{Application of Heuristic Search Methods}
    \begin{enumerate}
        \item \textbf{Greedy Search}:
        \begin{itemize}
            \item Chooses the neighbor with the lowest cost (or highest reward) at each step.
            \item Common in scenarios like traveling salesman problems where immediate gratification might lead to a viable solution path.
            \item \textbf{Example}: If a traveler aims to visit several cities, a greedy algorithm would take the nearest unvisited city at every step.
        \end{itemize}

        \item \textbf{Hill Climbing}:
        \begin{itemize}
            \item An iterative algorithm that starts with an arbitrary solution and makes incremental improvements.
            \item Continues until there are no further improvements.
            \item \textbf{Illustration}: Imagine navigating a mountainous region where your goal is to reach the highest peak.
        \end{itemize}
        
        \item \textbf{A* Algorithm}:
        \begin{itemize}
            \item Combines features of both Dijkstra’s algorithm and Greedy Best-First-Search.
            \item Uses a cost function \(f(n) = g(n) + h(n)\), where:
            \begin{itemize}
                \item \(g(n)\) is the cost from the start node to \(n\),
                \item \(h(n)\) is the heuristic estimate from \(n\) to the goal.
                \item \(f(n)\) is the total estimated cost of the cheapest solution through \(n\).
            \end{itemize}
        \end{itemize}
    \end{enumerate}
\end{frame}

\begin{frame}[fragile]
    \frametitle{A* Algorithm Explained - Overview}
    \begin{block}{Overview}
        The A* (A-star) algorithm is a popular and powerful pathfinding and graph traversal algorithm used in computer science and artificial intelligence. It combines features of Dijkstra's algorithm and Greedy Best-First Search to efficiently find the least-cost path from a start node to a goal node.
    \end{block}
\end{frame}

\begin{frame}[fragile]
    \frametitle{A* Algorithm Explained - Key Components}
    \begin{itemize}
        \item \textbf{Nodes}: Represent states or positions in the search space.
        \item \textbf{Start Node}: The initial node from which the algorithm begins searching.
        \item \textbf{Goal Node}: The target node or state that the algorithm aims to reach.
        \item \textbf{Cost Function (g)}: The actual cost from the start node to the current node.
        \item \textbf{Heuristic Function (h)}: Estimates the cost to reach the goal from the current node (admissible).
        \item \textbf{Evaluation Function (f)}: Combines \( g \) and \( h \) as follows:
        \begin{equation}
            f(n) = g(n) + h(n)
        \end{equation}
    \end{itemize}
\end{frame}

\begin{frame}[fragile]
    \frametitle{A* Algorithm Explained - How It Works}
    \begin{enumerate}
        \item \textbf{Initialization}: Start with the open list (containing the start node) and the closed list (initially empty).
        \item \textbf{Loop}:
        \begin{itemize}
            \item select the node with the lowest \( f \) score from the open list.
            \item \textbf{Check Goal}: If this node is the goal node, the search is done; backtrack to extract the path.
            \item \textbf{Generate Successors}: For each neighboring node:
            \begin{itemize}
                \item Calculate \( g \), \( h \), and \( f \) values.
                \item If already in the closed list with a lower \( f \), disregard it.
                \item If not in the open list, add it.
            \end{itemize}
            \item Move the current node to the closed list.
        \end{itemize}
    \end{enumerate}
\end{frame}

\begin{frame}[fragile]
    \frametitle{A* Algorithm Explained - Example}
    \begin{block}{Example Scenario}
        Suppose we have a grid-based world with a starting point (A) and goal point (B). Each step costs a fixed amount (e.g., 1). 
    \end{block}
    \begin{itemize}
        \item Start from A, calculate direct neighbors and their costs.
        \item Use heuristics like Manhattan distance or Euclidean distance to estimate \( h \).
    \end{itemize}
    \begin{block}{Example Calculation}
        At node C (neighbor of A):
        \begin{itemize}
            \item \( g(C) = 1 \) (cost to reach C from A)
            \item \( h(C) = 3 \) (estimated distance to B)
            \item \( f(C) = g(C) + h(C) = 1 + 3 = 4 \)
        \end{itemize}
    \end{block}
\end{frame}

\begin{frame}[fragile]
    \frametitle{A* Algorithm Explained - Key Points and Wrap-Up}
    \begin{itemize}
        \item \textbf{Admissibility}: A* is optimal if the heuristic is admissible.
        \item \textbf{Efficiency}: Often more efficient than Dijkstra’s algorithm due to heuristic guidance.
        \item \textbf{Applications}: Used in AI for games, robotics, and network routing.
    \end{itemize}
    \begin{block}{Wrap-Up}
        The A* algorithm's combination of actual costs and heuristic estimates makes it a flexible and efficient choice for many pathfinding applications. Understanding its components and operation is fundamental to leveraging its potential in practical scenarios.
    \end{block}
\end{frame}

\begin{frame}[fragile]
    \frametitle{Greedy Search Algorithm - Overview}
    \begin{block}{Understanding the Greedy Approach in Heuristics}
        The greedy search algorithm is a problem-solving approach based on making locally optimal choices at each step, hoping to find a global optimum. Unlike other algorithms that consider the entire problem space, greedy algorithms focus on immediate benefits.
    \end{block}
    
    \begin{itemize}
        \item \textbf{Key Characteristics:}
            \begin{enumerate}
                \item Local Optimization
                \item Fast Execution
                \item Immediate Choices
            \end{enumerate}
    \end{itemize}
\end{frame}

\begin{frame}[fragile]
    \frametitle{Greedy Search Algorithm - Strengths and Weaknesses}
    \begin{block}{Strengths of Greedy Search}
        \begin{itemize}
            \item \textbf{Efficiency:} Can solve problems quickly.
            \item \textbf{Simplicity:} Easy to implement and understand.
            \item \textbf{Useful in Certain Problems:} Effective for problems with the 'greedy choice property'.
                \begin{itemize}
                    \item Minimum Spanning Trees (Kruskal’s and Prim’s algorithms)
                    \item Huffman Coding for data compression
                \end{itemize}
        \end{itemize}
    \end{block}

    \begin{block}{Weaknesses of Greedy Search}
        \begin{itemize}
            \item \textbf{Not Always Optimal:} Can lead to suboptimal solutions.
            \item \textbf{No Backtracking:} Choices are not reconsidered.
            \item \textbf{Limited Applicability:} Best suited for specific types of problems.
        \end{itemize}
    \end{block}
\end{frame}

\begin{frame}[fragile]
    \frametitle{Greedy Search Algorithm - Example}
    \begin{block}{Example: Coin Change Problem}
        Suppose you need to make change for a specific amount using the least number of coins. Available denominations are 1 cent, 5 cents, and 10 cents.
        \begin{itemize}
            \item Target Amount: 27 cents
            \item Steps:
            \begin{enumerate}
                \item Take 2 x 10 cent coins (20 cents)
                \item Remaining amount: 7 cents
                \item Take 1 x 5 cent coin (5 cents)
                \item Remaining amount: 2 cents
                \item Take 2 x 1 cent coins (2 cents)
            \end{enumerate}
            \item Total coins used: 5 coins (2 + 1 + 2).
        \end{itemize}
    \end{block}

    \begin{block}{Algorithm Steps}
        \begin{enumerate}
            \item Start with the largest denomination.
            \item While the amount is not zero:
            \begin{itemize}
                \item Choose the maximum denomination that is less than or equal to the remaining amount.
                \item Subtract that denomination from the amount.
                \item Increment the coin count.
            \end{itemize}
        \end{enumerate}
    \end{block}
\end{frame}

\begin{frame}[fragile]
    \frametitle{Key Concepts in Greedy Search}
    \begin{block}{Key Points to Emphasize}
        \begin{itemize}
            \item Greedy algorithms are best for optimization problems with certain properties.
            \item Understanding when to use a greedy approach versus other searching techniques (like backtracking or dynamic programming) is crucial.
            \item Analyze the problem to determine if the greedy approach will yield a global optimum.
        \end{itemize}
    \end{block}
\end{frame}

\begin{frame}[fragile]
    \frametitle{Hill Climbing Method - Overview}
    \begin{block}{Definition}
        Hill climbing is a local search algorithm that continuously moves towards solutions with increasing value, aiming to find the best solution to an optimization problem.
    \end{block}
    \begin{itemize}
        \item Used for optimization problems
        \item Goal: maximize or minimize a function
    \end{itemize}
\end{frame}

\begin{frame}[fragile]
    \frametitle{Hill Climbing Method - Key Concepts}
    \begin{itemize}
        \item \textbf{Objective Function}: The function to optimize (e.g., minimize distance in a traveling salesman problem).
        \item \textbf{Current State}: The current position in the search space, represented by a possible solution.
        \item \textbf{Neighbors}: States accessible from the current state with small changes.
        \item \textbf{Local Maximum}: A solution that is better than its neighbors but not necessarily the best overall.
    \end{itemize}
\end{frame}

\begin{frame}[fragile]
    \frametitle{Hill Climbing Method - Process and Example}
    \begin{enumerate}
        \item \textbf{Initialization}: Start at an arbitrary solution.
        \item \textbf{Generate Neighbors}: Evaluate neighboring solutions.
        \item \textbf{Evaluate}: Compare neighbors using the objective function.
        \item \textbf{Move}: Move to a better neighbor and repeat.
        \item \textbf{Termination}: Stop when no neighbors are better.
    \end{enumerate}
    
    \begin{block}{Example}
        \textbf{Problem}: Maximize \( f(x) = -x^2 + 4x \) on [0, 4]. \\
        Steps: Start at \( x=0 \), find \( f(0.1), f(0.2)... \), and move until reaching maximum at \( x=2 \) with \( f(2)=4 \).
    \end{block}
\end{frame}

\begin{frame}[fragile]
    \frametitle{Hill Climbing Method - Types and Limitations}
    \begin{itemize}
        \item \textbf{Types of Hill Climbing}:
        \begin{enumerate}
            \item Simple Hill Climbing: Moves to the first better neighbor.
            \item Stochastic Hill Climbing: Randomly picks a better neighbor.
            \item Random Restart Hill Climbing: Multiple sessions from different starting points.
        \end{enumerate}
        
        \item \textbf{Limitations}:
        \begin{itemize}
            \item Local Optima: Can get stuck in local maxima.
            \item Plateaus: Areas with no improvement can slow progress.
            \item No Backtracking: Once moved, cannot reverse the decision.
        \end{itemize}
    \end{itemize}
\end{frame}

\begin{frame}[fragile]
    \frametitle{Hill Climbing Method - Key Points and Summary}
    \begin{itemize}
        \item Powerful optimization technique with limitations related to local maxima.
        \item A greedy algorithm making decisions based on local information.
    \end{itemize}
    \begin{block}{Summary}
        Hill Climbing is a heuristic search method optimizing a problem by iteratively moving towards better solutions; awareness of its limitations is crucial for practical applications.
    \end{block}
\end{frame}

\begin{frame}[fragile]{Comparative Analysis of Heuristic Search Methods}
    \frametitle{Introduction to Heuristic Search Methods}
    Heuristic search methods are strategies used to solve complex problems more quickly when classic methods are too slow. They use "rules of thumb" or educated guesses to find optimal solutions efficiently, particularly in vast search spaces.
\end{frame}

\begin{frame}[fragile]{Common Heuristic Search Methods}
    \begin{enumerate}
        \item \textbf{Hill Climbing} 
            \begin{itemize}
                \item An iterative algorithm that continually moves in the direction of increasing value (uphill) to find the peak or optimal solution.
            \end{itemize}
        \item \textbf{A* Search}
            \begin{itemize}
                \item A pathfinding and graph traversal algorithm that finds the least-cost path from a start to a goal node, leveraging both known costs and heuristics.
            \end{itemize}
        \item \textbf{Simulated Annealing}
            \begin{itemize}
                \item Inspired by metallurgy, it explores the search space by allowing occasional downhill steps to avoid local maxima and eventually converges towards the global optimum.
            \end{itemize}
        \item \textbf{Genetic Algorithms}
            \begin{itemize}
                \item Utilizes the principles of natural selection to evolve solutions over generations, combining and mutating them to explore the search space.
            \end{itemize}
    \end{enumerate}
\end{frame}

\begin{frame}[fragile]{Comparative Analysis: Efficiency vs. Applicability}
    \begin{table}[h!]
        \centering
        \begin{tabular}{|l|l|l|l|}
            \hline
            \textbf{Method} & \textbf{Efficiency (Time Complexity)} & \textbf{Applicability} & \textbf{Notes} \\ 
            \hline
            \textbf{Hill Climbing} & O(n) (may become O(infinity) if stuck) & Best for problems with a single peak. & Simple to implement, but can get stuck. \\ 
            \hline
            \textbf{A* Search} & O(b^d) where b is branching factor & Suitable for pathfinding with known heuristics. & Guarantees shortest path if heuristic is admissible. \\ 
            \hline
            \textbf{Simulated Annealing} & O(k) (k = number of temperature declines) & Problems too complex for hill climbing. & Avoids getting stuck due to its probabilistic approach. \\ 
            \hline
            \textbf{Genetic Algorithms} & Varies widely; can be high & Global optimization in complex spaces. & Robust but requires substantial processing resources. \\ 
            \hline
        \end{tabular}
    \end{table}
\end{frame}

\begin{frame}[fragile]{Key Points and Conclusion}
    \begin{itemize}
        \item \textbf{Trade-Offs:} Efficiency often comes at the cost of applicability depending on the problem's nature.
        \item \textbf{Heuristics Importance:} The choice of heuristic can significantly affect performance and outcome.
        \item \textbf{Real-World Scenarios:} Each method has unique strengths making it suitable for specific problems like navigation and optimization.
    \end{itemize}

    \textbf{Example Illustration: A* Search in Action}:
    Consider navigating from Point A to Point B with obstacles. A* uses:
    \begin{equation}
        f(n) = g(n) + h(n)
    \end{equation}
    where:
    \begin{align*}
        g(n) & = \text{Cost from Start to Current Node}, \\
        h(n) & = \text{Estimated Cost from Current Node to Goal}.
    \end{align*}
    
    By understanding the comparative efficiencies and applicable scenarios, students can select the most suitable algorithms for problems in computer science and AI.
\end{frame}

\begin{frame}[fragile]
    \frametitle{Limitations of Heuristic Search - Introduction}
    \begin{block}{Overview}
        Heuristic search methods are widely used in artificial intelligence to solve complex problems efficiently. However, they come with limitations that can affect their effectiveness. Understanding these challenges is essential for effective application and development of heuristic techniques.
    \end{block}
\end{frame}

\begin{frame}[fragile]
    \frametitle{Limitations of Heuristic Search - Common Limitations}
    \begin{enumerate}
        \item \textbf{Suboptimal Solutions}:
        \begin{itemize}
            \item Heuristic methods do not guarantee optimal solutions; they aim for "good enough" solutions quickly.
            \item \textbf{Example}: In the Traveling Salesman Problem, methods like the nearest neighbor may yield longer routes than the optimal.
        \end{itemize}

        \item \textbf{Domain Dependence}:
        \begin{itemize}
            \item The effectiveness of a heuristic is often context-dependent; heuristics suitable for one domain may fail in another.
            \item \textbf{Example}: A chess heuristic may not work for vehicle routing problems.
        \end{itemize}
    \end{enumerate}
\end{frame}

\begin{frame}[fragile]
    \frametitle{Limitations of Heuristic Search - Further Challenges}
    \begin{enumerate}
        \setcounter{enumi}{2} % Continue numbering from the previous frame
        \item \textbf{Local Optima}:
        \begin{itemize}
            \item Heuristic searches can get stuck in local optima, missing better overall solutions.
            \item \textbf{Example}: In a mountainous landscape, a search may find a peak but overlook a higher peak nearby.
        \end{itemize}

        \item \textbf{Lack of Completeness}:
        \begin{itemize}
            \item Heuristic methods may not explore every possibility, potentially overlooking solutions.
            \item \textbf{Example}: A maze-solving heuristic favoring one direction may miss other viable paths to the exit.
        \end{itemize}
    \end{enumerate}
\end{frame}

\begin{frame}[fragile]
    \frametitle{Limitations of Heuristic Search - Still More Challenges}
    \begin{enumerate}
        \setcounter{enumi}{4} % Continue numbering from the previous frame
        \item \textbf{Computational Complexity}:
        \begin{itemize}
            \item Heuristic search may demand significant computational resources for complex problems.
            \item \textbf{Example}: While A* search performs well with a good heuristic, its efficiency can drop in complex environments.
        \end{itemize}
        
        \item \textbf{Need for Design Expertise}:
        \begin{itemize}
            \item Effective heuristic design often requires deep domain knowledge; poorly designed heuristics can hinder performance.
            \item \textbf{Example}: Experienced game theorists can craft strong heuristics for board games, while novices might overlook critical strategies.
        \end{itemize}
    \end{enumerate}
\end{frame}

\begin{frame}[fragile]
    \frametitle{Limitations of Heuristic Search - Conclusion and Key Points}
    \begin{block}{Key Points}
        \begin{itemize}
            \item Heuristic searches are practical, but awareness of limitations is crucial for selecting appropriate methods.
            \item Continuously refining and adapting strategies can help mitigate these challenges.
            \item Recognizing potential pitfalls enhances problem-solving capabilities and improves the effectiveness of heuristic approaches.
        \end{itemize}
    \end{block}
    
    \begin{block}{References}
        \begin{itemize}
            \item Russell, S., \& Norvig, P. (2016). \textit{Artificial Intelligence: A Modern Approach}.
            \item Kiss, G., \& Steinert, M. (2019). "Comparative Analysis of Heuristic Algorithms."
        \end{itemize}
    \end{block}
\end{frame}

\begin{frame}[fragile]{Applications of Heuristic Search in AI}
    \frametitle{Overview of Heuristic Search}
    \begin{itemize}
        \item Heuristic search methods leverage practical approaches to solve problems more efficiently than classic search algorithms.
        \item Particularly beneficial for complex problems where traditional techniques struggle due to time or resource constraints.
    \end{itemize}
\end{frame}

\begin{frame}[fragile]{Key Real-World Applications}
    \frametitle{Key Real-World Applications of Heuristic Search}
    \begin{enumerate}
        \item \textbf{Robotics}
            \begin{itemize}
                \item \textbf{Path Planning:} Robots utilize heuristic methods to navigate, avoiding obstacles while reaching their destinations.
                \item \textit{Example:} Autonomous vacuum cleaners map their cleaning paths using heuristic search.
            \end{itemize}

        \item \textbf{Logistics and Supply Chain Optimization}
            \begin{itemize}
                \item \textbf{Route Optimization:} Heuristic searches optimize complex delivery and transportation routing problems.
                \item \textit{Example:} Companies like UPS and Amazon use heuristic algorithms for efficient delivery routes considering real-time traffic.
            \end{itemize}
    
        \item \textbf{Game AI}
            \begin{itemize}
                \item Heuristic searches help in determining the best moves in games like chess or Go, where potential moves are vast.
                \item \textit{Example:} In chess, heuristics evaluate board positions based on piece value and control.
            \end{itemize}
    
        \item \textbf{Machine Learning}
            \begin{itemize}
                \item Heuristic methods like genetic algorithms optimize hyperparameters in model training.
                \item \textit{Example:} Efficiently choosing optimal learning rates in neural networks.
            \end{itemize}
    \end{enumerate}
\end{frame}

\begin{frame}[fragile]{Key Points to Remember and Code Snippet}
    \frametitle{Key Points and Implementation}
    \begin{block}{Key Points to Remember}
        \begin{itemize}
            \item \textbf{Efficiency:} Heuristic search methods reduce time and computational resources by guiding the search towards promising areas.
            \item \textbf{Flexibility:} Adaptable to various applications, making them versatile tools in AI.
            \item \textbf{Computational Trade-offs:} They may not always guarantee optimal solutions.
        \end{itemize}
    \end{block}

    \begin{equation}
        f(n) = g(n) + h(n)
    \end{equation}
    \textit{Where:}
    \begin{itemize}
        \item $f(n)$ = total estimated cost of the cheapest solution through node $n$.
        \item $g(n)$ = cost from the start node to $n$.
        \item $h(n)$ = estimated cost from $n$ to the goal (heuristic).
    \end{itemize}

    \begin{lstlisting}[language=Python]
class Node:
    def __init__(self, position, parent=None):
        self.position = position
        self.parent = parent
        self.g = 0  # Cost from start node
        self.h = 0  # Heuristic cost to goal
        self.f = 0  # Total cost

def a_star(start, goal):
    open_list = []
    closed_list = []
    # Initialize the algorithm based on start and goal nodes
    \end{lstlisting}
\end{frame}

\begin{frame}[fragile]
    \frametitle{Case Study on Optimizing Delivery Routes}
    An example of heuristic search applied to solving delivery route optimization problems.
\end{frame}

\begin{frame}[fragile]
    \frametitle{Introduction to Heuristic Search}
    \begin{block}{Definition}
        Heuristic search methods are problem-solving techniques that guide the search process to find satisfactory solutions more efficiently than traditional methods. 
    \end{block}
    \begin{itemize}
        \item Useful in large search spaces.
        \item Effective when finding optimal solutions is computationally expensive.
    \end{itemize}
\end{frame}

\begin{frame}[fragile]
    \frametitle{The Problem: Delivery Route Optimization}
    In logistics and transportation:
    \begin{itemize}
        \item Optimizing delivery routes minimizes costs and improves efficiency.
        \item Aim: Find the shortest route for a vehicle to visit multiple locations and return to the start.
    \end{itemize}
\end{frame}

\begin{frame}[fragile]
    \frametitle{Why Use Heuristic Search?}
    \begin{itemize}
        \item \textbf{Scalability}:
            \begin{itemize}
                \item Traditional algorithms can become impractical as delivery points increase.
            \end{itemize}
        \item \textbf{Speed}:
            \begin{itemize}
                \item Heuristic methods find "good enough" solutions quickly.
            \end{itemize}
        \item \textbf{Flexibility}:
            \begin{itemize}
                \item Can adapt to changes like traffic conditions or priority shifts.
            \end{itemize}
    \end{itemize}
\end{frame}

\begin{frame}[fragile]
    \frametitle{Common Heuristic Algorithms}
    \begin{enumerate}
        \item \textbf{Nearest Neighbor Heuristic}:
            \begin{itemize}
                \item Start at origin and visit the nearest unvisited point.
            \end{itemize}
        \item \textbf{Genetic Algorithms}:
            \begin{itemize}
                \item Evolve a population of routes over generations.
            \end{itemize}
        \item \textbf{A* Search Algorithm}:
            \begin{itemize}
                \item Best-first search using a cost function: 
                \begin{equation}
                    f(n) = g(n) + h(n)
                \end{equation}
                where 
                \begin{itemize}
                    \item $g(n)$: Actual cost from start to node $n$.
                    \item $h(n)$: Estimated cost from node $n$ to the goal.
                \end{itemize}
            \end{itemize}
    \end{enumerate}
\end{frame}

\begin{frame}[fragile]
    \frametitle{Case Study Example: Fast Deliveries Inc.}
    \begin{itemize}
        \item \textbf{Challenge}: Optimize routes for 10 vans servicing 100 delivery points.
        \item \textbf{Method}: Nearest Neighbor Heuristic
            \begin{enumerate}
                \item Start at the warehouse.
                \item Select closest unvisited delivery point.
                \item Mark point as visited and repeat until all points are covered.
            \end{enumerate}
        \item \textbf{Outcome}: Reduced average travel distance by 25\%, improving efficiency and lowering costs.
    \end{itemize}
\end{frame}

\begin{frame}[fragile]
    \frametitle{Key Points to Emphasize}
    \begin{itemize}
        \item Heuristic search methods provide practical solutions for complex real-world problems.
        \item Different heuristics can be applied based on problem requirements.
        \item There is always a trade-off between solution quality and computation time.
    \end{itemize}
\end{frame}

\begin{frame}[fragile]
    \frametitle{Conclusion}
    By applying heuristic search techniques, logistics companies can achieve significant improvements in delivery efficiency. 
    This case study illustrates the effective use of heuristics in solving common optimization problems, highlighting the practicality of artificial intelligence methods in real-world applications.
\end{frame}

\begin{frame}[fragile]{Evaluating Heuristic Search Performance - Overview}
    \begin{block}{Description}
        Metrics and methods to evaluate the performance of heuristic search algorithms.
    \end{block}
\end{frame}

\begin{frame}[fragile]{Understanding Heuristic Search Performance}
    Heuristic search methods are designed to find solutions efficiently, especially in complex problem spaces. 
    To determine how effectively these algorithms perform, we utilize various metrics and methods for evaluation.
\end{frame}

\begin{frame}[fragile]{Key Metrics for Evaluation}
    \begin{enumerate}
        \item \textbf{Time Complexity}  
        \begin{itemize}
            \item \textit{Definition}: Measures the computational time required by an algorithm to reach a solution as a function of the input size.
            \item \textit{Example}: An algorithm with a time complexity of \( O(n) \) will have its execution time increase linearly with the number of elements processed.
        \end{itemize}
        
        \item \textbf{Space Complexity}  
        \begin{itemize}
            \item \textit{Definition}: Assesses the amount of memory space required by the algorithm at any given point during its execution.
            \item \textit{Example}: An algorithm that occupies \( O(n^2) \) space will use memory that grows quadratically with the input size.
        \end{itemize}
        
        \item \textbf{Solution Quality}  
        \begin{itemize}
            \item \textit{Definition}: Refers to how close the solution generated by the heuristic is to the optimal solution.
            \item \textit{Example}: If a heuristic finds a cost of 20 and the optimal cost is 15, quality can be calculated as:  
            \[
            \text{Quality} = \frac{\text{Heuristic Cost} - \text{Optimal Cost}}{\text{Optimal Cost}} = \frac{20 - 15}{15} = \frac{5}{15} \approx 0.33 \text{ (or 33\% worse)}
            \]
        \end{itemize}

        \item \textbf{Search Space Exploration}  
        \begin{itemize}
            \item \textit{Definition}: Indicates how efficiently an algorithm navigates through the problem space.
            \item \textit{Example}: If one algorithm expands 100 nodes and another expands 500 to reach a solution, the former is more efficient.
        \end{itemize}
    \end{enumerate}
\end{frame}

\begin{frame}[fragile]{Methods for Evaluation}
    \begin{itemize}
        \item \textbf{Benchmarking against Standard Problems}:
        \begin{itemize}
            \item Use established problem sets (e.g., Traveling Salesman Problem, 8-puzzle) for consistent comparison.
        \end{itemize}
        
        \item \textbf{A/B Testing}:
        \begin{itemize}
            \item Run multiple versions of heuristic algorithms to see which performs better under identical conditions.
        \end{itemize}

        \item \textbf{Statistical Analysis}:
        \begin{itemize}
            \item Analyze performance data (e.g., average solution cost and time taken) using statistical tools for reliable conclusions.
        \end{itemize}
    \end{itemize}
\end{frame}

\begin{frame}[fragile]{Key Examples of Heuristic Search Evaluation}
    \begin{itemize}
        \item \textbf{A* Algorithm Performance}:
        \begin{itemize}
            \item Evaluate by measuring path cost optimality, time, and space complexity in indexed graphs.
        \end{itemize}
        
        \item \textbf{Greedy Search}:
        \begin{itemize}
            \item Analyze the impact of different heuristics on exploration efficiency by comparing solution quality.
        \end{itemize}
    \end{itemize}
\end{frame}

\begin{frame}[fragile]{Conclusion}
    Evaluating heuristic search performance is critical for enhancing algorithm efficiency and effectiveness. 
    By understanding these metrics and methods, we can identify the strengths and weaknesses of various heuristics, improving problem-solving strategies in complex environments.
\end{frame}

\begin{frame}[fragile]
    \frametitle{Ethical Implications in Heuristic Search}
    Heuristic search methods are widely used in AI applications, such as:
    \begin{itemize}
        \item Game playing
        \item Route planning
        \item Optimization problems
    \end{itemize}
    However, their deployment raises important ethical considerations that need to be addressed.
\end{frame}

\begin{frame}[fragile]
    \frametitle{Key Ethical Considerations}
    \begin{enumerate}
        \item \textbf{Bias and Fairness:}
            \begin{itemize}
                \item Heuristics may incorporate biases from training data.
                \item Example: Hiring algorithms favoring specific demographics.
            \end{itemize}
        
        \item \textbf{Transparency and Explainability:}
            \begin{itemize}
                \item Heuristic methods can function as "black boxes."
                \item Example: Navigation apps that don’t explain route choices.
            \end{itemize}
        
        \item \textbf{Accountability:}
            \begin{itemize}
                \item Difficulty in determining responsibility for decisions made by heuristic-based systems.
                \item Example: Accidents involving autonomous vehicles and heuristic choices.
            \end{itemize}
    \end{enumerate}
\end{frame}

\begin{frame}[fragile]
    \frametitle{Continued Key Ethical Considerations}
    \begin{enumerate}[resume]
        \item \textbf{Privacy Concerns:}
            \begin{itemize}
                \item Large datasets required may include sensitive personal information.
                \item Example: AI product recommendations collecting user data.
            \end{itemize}

        \item \textbf{Impact on Employment:}
            \begin{itemize}
                \item Efficiency gains can lead to job automation.
                \item Example: Tools optimizing delivery routes displacing human roles.
            \end{itemize}
    \end{enumerate}
\end{frame}

\begin{frame}[fragile]
    \frametitle{Challenges in Mitigating Ethical Issues}
    \begin{itemize}
        \item Lack of standardized guidelines in the AI field.
        \item Dynamic environments complicate oversight of heuristic actions.
        \item Balancing efficiency against ethical considerations is delicate.
    \end{itemize}
\end{frame}

\begin{frame}[fragile]
    \frametitle{Conclusion}
    As heuristic search methods advance:
    \begin{itemize}
        \item Continuous dialogue among developers, policymakers, and users is essential.
        \item Establishing frameworks for accountability and fairness minimises negative outcomes.
        \item Important to engage with evolving AI ethics frameworks to govern development.
    \end{itemize}
\end{frame}

\begin{frame}{Future Trends in Heuristic Search Methods}
    \frametitle{Overview}
    \begin{itemize}
        \item Heuristic search methods are essential in AI for efficient problem-solving.
        \item This presentation explores emerging trends and advancements in heuristic search techniques.
    \end{itemize}
\end{frame}

\begin{frame}{Emerging Trends and Advancements}
    \frametitle{Integration with Machine Learning}
    \begin{itemize}
        \item **Description**: Combining heuristic methods with machine learning for enhanced efficiency and adaptability.
        \item **Example**: In robotics, reinforcement learning optimizes search paths based on environmental feedback.
    \end{itemize}
\end{frame}

\begin{frame}{Emerging Trends and Advancements (cont'd)}
    \frametitle{Hybrid Approaches}
    \begin{itemize}
        \item **Description**: Combining multiple search techniques (e.g., genetic algorithms with local search).
        \item **Example**: Hybrid genetic algorithms can improve convergence rates in complex landscapes.
    \end{itemize}
\end{frame}

\begin{frame}{Emerging Trends and Advancements (cont'd)}
    \frametitle{Parallel and Distributed Computing}
    \begin{itemize}
        \item **Description**: Distributing search processes across multiple processors for faster computation.
        \item **Example**: Heuristic searches significantly cut time in large-scale optimization problems in logistics.
    \end{itemize}
\end{frame}

\begin{frame}{Emerging Trends and Advancements (cont'd)}
    \frametitle{Dynamic and Adaptive Heuristics}
    \begin{itemize}
        \item **Description**: Heuristics that adapt in real-time based on changing problem landscapes.
        \item **Example**: Heuristics adjust parameters on-the-fly for traffic conditions in navigation systems.
    \end{itemize}
\end{frame}

\begin{frame}{Emerging Trends and Advancements (cont'd)}
    \frametitle{Incorporation of Big Data Analytics}
    \begin{itemize}
        \item **Description**: Using big data techniques to improve heuristic methods and search strategies.
        \item **Example**: Personalized content delivery systems optimize heuristics with user interaction data.
    \end{itemize}
\end{frame}

\begin{frame}{Emerging Trends and Advancements (cont'd)}
    \frametitle{Quantum Computing Applications}
    \begin{itemize}
        \item **Description**: Research on quantum computing influencing heuristic search methods with new, faster algorithms.
        \item **Example**: Grover's Search provides quicker database searches, revolutionizing heuristic search capabilities.
    \end{itemize}
\end{frame}

\begin{frame}{Conclusion and Key Points}
    \frametitle{Conclusion}
    \begin{itemize}
        \item Heuristic methods combined with machine learning lead to smarter searches.
        \item Hybrid approaches can enhance traditional heuristics significantly.
        \item Quantum computing may drastically change heuristic search capabilities.
    \end{itemize}
\end{frame}

\begin{frame}[fragile]{Pseudocode Example}
    \frametitle{Hybrid Heuristic Search Algorithm}
    \begin{lstlisting}[language=Python]
def hybrid_search(problem):
    population = initialize_population(problem)
    while not termination_condition_met():
        evaluate_fitness(population)
        selected_parents = selection(population)
        offspring = crossover(selected_parents)
        mutate(offspring)
        population = replace(population, offspring)
    return best_solution(population)
    \end{lstlisting}
\end{frame}

\begin{frame}[fragile]{Q\&A Session}
    \begin{block}{Overview}
        This slide provides an opportunity for students to engage actively in discussing heuristic search methods. The aim is to clarify any doubts, elaborate on concepts, and link theoretical knowledge to practical applications. Use this time effectively to enhance your understanding.
    \end{block}
\end{frame}

\begin{frame}[fragile]{Key Concepts: Heuristic Search Methods}
    \begin{enumerate}
        \item \textbf{Heuristic Search Methods}:
            \begin{itemize}
                \item Heuristics solve problems faster when classic methods are slow or infeasible.
                \item Common techniques include A*, greedy best-first search, and genetic algorithms.
            \end{itemize}
        
        \item \textbf{Evaluation Function}:
            \begin{equation}
                f(n) = g(n) + h(n)
            \end{equation}
            where \( g(n) \) is the cost from the start node to node \( n \), and \( h(n) \) is the heuristic estimate to the goal.
        
        \item \textbf{Trade-offs}:
            \begin{itemize}
                \item Discuss the trade-offs between optimality and efficiency.
                \item Some heuristics ensure optimal solutions (e.g., A* with an admissible heuristic).
                \item Others may sacrifice optimality for faster results.
            \end{itemize}
    \end{enumerate}
\end{frame}

\begin{frame}[fragile]{Encouraging Discussion}
    \begin{block}{Common Questions}
        \begin{itemize}
            \item What are real-world applications of heuristic search methods?
            \item How do we choose or design an appropriate heuristic for a specific problem?
            \item Can you explain the difference between informed and uninformed search strategies?
            \item What challenges do heuristic methods face in complex search spaces?
        \end{itemize}
    \end{block}

    \begin{block}{Engaging Students}
        \begin{itemize}
            \item Think of scenarios requiring heuristic solutions.
            \item Use examples like Google Maps routing (A* algorithm) or games like chess.
        \end{itemize}
    \end{block}
\end{frame}

\begin{frame}[fragile]{Summary Points}
    \begin{itemize}
        \item Heuristic methods are vital for solving complex problems efficiently.
        \item Understanding evaluation functions and their components is crucial in applying these techniques.
        \item Be prepared to discuss examples, challenges, and the importance of heuristics in modern computational problems.
    \end{itemize}
\end{frame}

\begin{frame}[fragile]{Summary and Key Takeaways - Part 1}
    \frametitle{Recap of Heuristic Search Methods}
    \begin{block}{Definition of Heuristic}
        Heuristics are strategies or techniques that simplify decision-making and problem-solving by providing useful shortcuts or estimations.
    \end{block}
    \begin{itemize}
        \item **Speed:** Quicker solutions compared to exhaustive search methods.
        \item **Approximation:** No guarantee of an optimal solution but often yield satisfactory results.
        \item **Domain-Specific:** Many heuristics are tailored for specific types of problems, utilizing domain knowledge.
    \end{itemize}
\end{frame}

\begin{frame}[fragile]{Summary and Key Takeaways - Part 2}
    \frametitle{Common Heuristic Algorithms}
    \begin{enumerate}
        \item **Greedy Search:** Chooses the best immediate option without regard for future consequences.
        \item **A* Search:** Combines Dijkstra's Algorithm with heuristics to estimate cost from the current node to goal.
        \item **Genetic Algorithms:** Mimic natural selection by evolving solutions through mutation and crossover over iterations.
    \end{enumerate}
    \begin{block}{Evaluation of Heuristic Methods}
        \begin{itemize}
            \item Performance is measured based on time complexity, optimality, and accuracy.
            \item Quality can be analyzed by comparing results against optimal solutions using metrics like average-case performance.
        \end{itemize}
    \end{block}
\end{frame}

\begin{frame}[fragile]{Summary and Key Takeaways - Part 3}
    \frametitle{Key Takeaways and Final Thoughts}
    \begin{itemize}
        \item Heuristic search methods are invaluable for solving complex problems efficiently.
        \item They are essential in AI applications such as pathfinding, scheduling, and optimization.
        \item Understanding the strengths and limitations of each algorithm is crucial for selecting the right heuristic.
    \end{itemize}
    \begin{block}{Final Thoughts}
        Heuristic search methods provide a practical approach to problem-solving by balancing speed and accuracy. Mastering these techniques will enhance your skills in artificial intelligence and computer science.
    \end{block}
\end{frame}


\end{document}