\documentclass[aspectratio=169]{beamer}

% Theme and Color Setup
\usetheme{Madrid}
\usecolortheme{whale}
\useinnertheme{rectangles}
\useoutertheme{miniframes}

% Additional Packages
\usepackage[utf8]{inputenc}
\usepackage[T1]{fontenc}
\usepackage{graphicx}
\usepackage{booktabs}
\usepackage{listings}
\usepackage{amsmath}
\usepackage{amssymb}
\usepackage{xcolor}
\usepackage{tikz}
\usepackage{pgfplots}
\pgfplotsset{compat=1.18}
\usetikzlibrary{positioning}
\usepackage{hyperref}

% Custom Colors
\definecolor{myblue}{RGB}{31, 73, 125}
\definecolor{mygray}{RGB}{100, 100, 100}
\definecolor{mygreen}{RGB}{0, 128, 0}
\definecolor{myorange}{RGB}{230, 126, 34}
\definecolor{mycodebackground}{RGB}{245, 245, 245}

% Set Theme Colors
\setbeamercolor{structure}{fg=myblue}
\setbeamercolor{frametitle}{fg=white, bg=myblue}
\setbeamercolor{title}{fg=myblue}
\setbeamercolor{section in toc}{fg=myblue}
\setbeamercolor{item projected}{fg=white, bg=myblue}
\setbeamercolor{block title}{bg=myblue!20, fg=myblue}
\setbeamercolor{block body}{bg=myblue!10}
\setbeamercolor{alerted text}{fg=myorange}

% Set Fonts
\setbeamerfont{title}{size=\Large, series=\bfseries}
\setbeamerfont{frametitle}{size=\large, series=\bfseries}
\setbeamerfont{caption}{size=\small}
\setbeamerfont{footnote}{size=\tiny}

% Code Listing Style
\lstdefinestyle{customcode}{
  backgroundcolor=\color{mycodebackground},
  basicstyle=\footnotesize\ttfamily,
  breakatwhitespace=false,
  breaklines=true,
  commentstyle=\color{mygreen}\itshape,
  keywordstyle=\color{blue}\bfseries,
  stringstyle=\color{myorange},
  numbers=left,
  numbersep=8pt,
  numberstyle=\tiny\color{mygray},
  frame=single,
  framesep=5pt,
  rulecolor=\color{mygray},
  showspaces=false,
  showstringspaces=false,
  showtabs=false,
  tabsize=2,
  captionpos=b
}
\lstset{style=customcode}

% Custom Commands
\newcommand{\hilight}[1]{\colorbox{myorange!30}{#1}}
\newcommand{\source}[1]{\vspace{0.2cm}\hfill{\tiny\textcolor{mygray}{Source: #1}}}
\newcommand{\concept}[1]{\textcolor{myblue}{\textbf{#1}}}
\newcommand{\separator}{\begin{center}\rule{0.5\linewidth}{0.5pt}\end{center}}

% Footer and Navigation Setup
\setbeamertemplate{footline}{
  \leavevmode%
  \hbox{%
  \begin{beamercolorbox}[wd=.3\paperwidth,ht=2.25ex,dp=1ex,center]{author in head/foot}%
    \usebeamerfont{author in head/foot}\insertshortauthor
  \end{beamercolorbox}%
  \begin{beamercolorbox}[wd=.5\paperwidth,ht=2.25ex,dp=1ex,center]{title in head/foot}%
    \usebeamerfont{title in head/foot}\insertshorttitle
  \end{beamercolorbox}%
  \begin{beamercolorbox}[wd=.2\paperwidth,ht=2.25ex,dp=1ex,center]{date in head/foot}%
    \usebeamerfont{date in head/foot}
    \insertframenumber{} / \inserttotalframenumber
  \end{beamercolorbox}}%
  \vskip0pt%
}

% Turn off navigation symbols
\setbeamertemplate{navigation symbols}{}

% Title Page Information
\title[Introduction to AI and Agent Architectures]{Week 1: Introduction to AI and Agent Architectures}
\author[J. Smith]{John Smith, Ph.D.}
\institute[University Name]{
  Department of Computer Science\\
  University Name\\
  \vspace{0.3cm}
  Email: email@university.edu\\
  Website: www.university.edu
}
\date{\today}

% Document Start
\begin{document}

\frame{\titlepage}

\begin{frame}[fragile]
    \frametitle{Introduction to Artificial Intelligence - Overview}
    \begin{block}{Definition}
        Artificial Intelligence (AI) refers to the simulation of human intelligence processes by computer systems. 
        These processes include learning, reasoning, problem-solving, perception, and language understanding.
    \end{block}
\end{frame}

\begin{frame}[fragile]
    \frametitle{Introduction to Artificial Intelligence - Relevance}
    \begin{itemize}
        \item \textbf{Transformative Technology:}
            \begin{itemize}
                \item \textit{Healthcare:} AI algorithms analyze medical data to assist in diagnosis and treatment recommendations.
                \item \textit{Finance:} Automated trading systems use AI to analyze financial markets and execute trades at optimal times.
            \end{itemize}
        \item \textbf{Daily Applications:}
            \begin{itemize}
                \item Interactions through virtual assistants (Siri, Alexa), recommendation engines (Netflix, Amazon), and customer service chatbots.
            \end{itemize}
        \item \textbf{Business Optimization:}
            \begin{itemize}
                \item AI enhances decision-making processes and optimizes operations, forecasting trends, and reducing costs via predictive maintenance.
            \end{itemize}
    \end{itemize}
\end{frame}

\begin{frame}[fragile]
    \frametitle{Introduction to Artificial Intelligence - Key Points}
    \begin{itemize}
        \item \textbf{Machine Learning (ML):} A subset focused on algorithms that allow computers to learn from and make predictions based on data.
        \item \textbf{Natural Language Processing (NLP):} Enables machines to understand and respond to human language for better communication.
        \item \textbf{Robotics:} AI enables autonomous task performance in robots, applicable in factories and delivery services.
    \end{itemize}
\end{frame}

\begin{frame}[fragile]
    \frametitle{Introduction to Artificial Intelligence - Examples}
    \begin{itemize}
        \item \textbf{AI in Action:}
            \begin{itemize}
                \item An AI healthcare application analyzes CT scans for early tumor detection, improving accuracy through learning from numerous cases.
            \end{itemize}
        \item \textbf{Conceptual Flow of AI Development:}
            \begin{enumerate}
                \item Data Collection
                \item Pre-processing
                \item Model Training (Machine Learning)
                \item Testing \& Evaluation
                \item Deployment
            \end{enumerate}
    \end{itemize}
\end{frame}

\begin{frame}[fragile]
    \frametitle{Introduction to Artificial Intelligence - Engagement}
    \begin{block}{Engagement Question}
        "Can you think of an AI technology you've interacted with today? Share your experiences!"
    \end{block}
    \begin{block}{Conclusion}
        AI is a powerful catalyst for innovation, enhancing how we live and work. Understanding its foundational concepts prepares us for deeper exploration in this field.
    \end{block}
\end{frame}

\begin{frame}[fragile]
    \frametitle{AI Concepts and Terminology - Introduction}
    \begin{block}{Overview}
        This presentation introduces foundational AI concepts including:
        \begin{itemize}
            \item Intelligent agents
            \item Search algorithms
            \item Logic reasoning
            \item Probabilistic reasoning
        \end{itemize}
    \end{block}
\end{frame}

\begin{frame}[fragile]
    \frametitle{AI Concepts and Terminology - Intelligent Agents}
    \begin{block}{1. Intelligent Agents}
        \begin{itemize}
            \item \textbf{Definition}: An intelligent agent is an entity that perceives its environment through sensors and acts upon it through actuators to achieve specific goals.
            \item \textbf{Example}: A self-driving car, which perceives its surroundings using cameras and sensors, and acts by steering and accelerating.
        \end{itemize}
    \end{block}
\end{frame}

\begin{frame}[fragile]
    \frametitle{AI Concepts and Terminology - Search Algorithms}
    \begin{block}{2. Search Algorithms}
        \begin{itemize}
            \item \textbf{Definition}: Algorithms used to find a solution from a set of possible solutions, especially in problem-solving contexts.
            \item \textbf{Types}:
            \begin{itemize}
                \item \textbf{Uninformed Search}: Searches without any additional information (e.g., Breadth-First Search).
                \item \textbf{Informed Search}: Utilizes heuristics to find solutions more efficiently (e.g., A* algorithm).
            \end{itemize}
            \item \textbf{Example}: In a maze, a search algorithm may systematically explore paths until it finds the exit.
        \end{itemize}
    \end{block}
\end{frame}

\begin{frame}[fragile]
    \frametitle{AI Concepts and Terminology - Logic Reasoning}
    \begin{block}{3. Logic Reasoning}
        \begin{itemize}
            \item \textbf{Definition}: The process of using formal logical systems to deduce new information or validate conclusions.
            \item \textbf{Two Main Types}:
            \begin{itemize}
                \item \textbf{Propositional Logic}: Deals with propositions that can be either true or false.
                \item \textbf{Predicate Logic}: Extends propositional logic by dealing with predicates and quantifiers (e.g., $\forall$ for "all"; $\exists$ for "exists").
            \end{itemize}
            \item \textbf{Example}: If "All humans are mortal" and "Socrates is a human", we can conclude that "Socrates is mortal".
        \end{itemize}
    \end{block}
\end{frame}

\begin{frame}[fragile]
    \frametitle{AI Concepts and Terminology - Probabilistic Reasoning}
    \begin{block}{4. Probabilistic Reasoning}
        \begin{itemize}
            \item \textbf{Definition}: The ability to reason and make decisions based on uncertain information, using probabilities.
            \item \textbf{Key Concepts}:
            \begin{itemize}
                \item \textbf{Bayes' Theorem}:
                \begin{equation}
                    P(A|B) = \frac{P(B|A) \cdot P(A)}{P(B)}
                \end{equation}
                Where:
                \begin{itemize}
                    \item $P(A|B)$ is the probability of event A occurring given that B is true.
                    \item $P(B|A)$ is the probability of event B occurring given that A is true.
                \end{itemize}
            \end{itemize}
            \item \textbf{Example}: Weather prediction where the likelihood of rain is updated with new information from weather models.
        \end{itemize}
    \end{block}
\end{frame}

\begin{frame}[fragile]
    \frametitle{AI Concepts and Terminology - Key Points}
    \begin{block}{Key Points to Emphasize}
        \begin{itemize}
            \item Intelligent agents form the backbone of AI, enabling systems to perform tasks autonomously.
            \item Understanding various search algorithms is crucial for problem-solving and optimizing decision-making in AI.
            \item Logic reasoning underpins the ability of AI systems to make conclusions based on defined rules and relationships.
            \item Probabilistic reasoning equips AI with the capability to operate under uncertainty, framing decisions in real-world scenarios.
        \end{itemize}
    \end{block}
\end{frame}

\begin{frame}[fragile]
    \frametitle{Types of AI Agents}
    % Overview of AI agents and their classification
    AI agents are systems that perceive their environment and act based on perceptions. The main types are:
    \begin{itemize}
        \item Simple Reflex Agents
        \item Model-Based Agents
        \item Goal-Based Agents
        \item Utility-Based Agents
    \end{itemize}
\end{frame}

\begin{frame}[fragile]
    \frametitle{1. Simple Reflex Agents}
    % Definition and characteristics of simple reflex agents
    \begin{block}{Definition}
        These agents respond directly to the current situation using condition-action rules.
    \end{block}
    \begin{itemize}
        \item \textbf{No Memory}: Do not retain information about past states.
        \item \textbf{Reactive}: Decisions based only on current input.
    \end{itemize}
    \textbf{Example}: A thermostat that activates heating when the temperature drops below a set threshold.
    
    \textbf{Key Point}: Simple reflex agents are best for straightforward and predictable environments.
\end{frame}

\begin{frame}[fragile]
    \frametitle{2. Model-Based Agents}
    % Definition and characteristics of model-based agents
    \begin{block}{Definition}
        Agents that maintain a model of the world to understand the current state from observations and actions.
    \end{block}
    \begin{itemize}
        \item \textbf{State Representation}: Update internal state based on new sensory information.
        \item \textbf{Memory Utilization}: Respond based on past experiences.
    \end{itemize}
    \textbf{Example}: A robotic vacuum that remembers cleaned areas and avoids obstacles.
    
    \textbf{Key Point}: Model-based agents handle partially observable environments with more flexibility.
\end{frame}

\begin{frame}[fragile]
    \frametitle{3. Goal-Based Agents}
    % Definition and characteristics of goal-based agents
    \begin{block}{Definition}
        Agents that act with specific goals and evaluate actions based on their effectiveness in reaching those goals.
    \end{block}
    \begin{itemize}
        \item \textbf{Goal Evaluation}: Assess various possible actions to choose the best one.
        \item \textbf{Planning Capability}: Create plans to achieve desired outcomes.
    \end{itemize}
    \textbf{Example}: A chess program aiming for checkmate while evaluating various moves.
    
    \textbf{Key Point}: Goal-based agents adapt to dynamic environments by re-evaluating strategies.
\end{frame}

\begin{frame}[fragile]
    \frametitle{4. Utility-Based Agents}
    % Definition and characteristics of utility-based agents
    \begin{block}{Definition}
        Agents that seek to maximize overall utility based on preferences across outcomes.
    \end{block}
    \begin{itemize}
        \item \textbf{Utility Function}: Evaluate the desirability of states and actions considering risks and rewards.
        \item \textbf{Decision Making Under Uncertainty}: Make probabilistic decisions based on potential outcomes.
    \end{itemize}
    \textbf{Example}: A self-driving car evaluating routes for time efficiency, fuel, and safety.
    
    \textbf{Key Point}: Utility-based agents excel in complex environments influenced by multiple factors.
\end{frame}

\begin{frame}[fragile]
    \frametitle{Conclusion}
    % Summary and next topics
    Understanding different types of AI agents is crucial for developing sophisticated AI systems. Each type has strengths and limitations for specific tasks. 

    \textbf{Next Up: Knowledge Acquisition}
    % Introduction to the next topic
    In the following slide, we will delve into how these agents acquire knowledge and its importance for effective AI performance.
\end{frame}

\begin{frame}[fragile]
    \frametitle{Knowledge Acquisition - Introduction}
    \begin{block}{What is Knowledge Acquisition?}
        Knowledge Acquisition is the process of extracting, structuring, and organizing information and knowledge from various sources.
    \end{block}
    It enables effective decision-making and problem-solving in AI systems, facilitating the development of intelligent agents that can learn, reason, and act in dynamic environments.
\end{frame}

\begin{frame}[fragile]
    \frametitle{Knowledge Acquisition - Importance}
    \begin{itemize}
        \item \textbf{Foundation for Intelligent Behavior:} Essential for informed decision-making and adaptability to changing environments.
        \item \textbf{Adaptability:} Continuous learning allows AI systems to adjust to new data and evolving requirements.
        \item \textbf{Enhancement of Problem-Solving Capabilities:} High-quality knowledge aids agents in solving complex problems systematically.
    \end{itemize}
\end{frame}

\begin{frame}[fragile]
    \frametitle{Knowledge Acquisition - Techniques}
    \begin{enumerate}
        \item \textbf{Manual Knowledge Acquisition:}
            \begin{itemize}
                \item Human experts provide knowledge through interviews or workshops.
                \item Example: A doctor’s expertise in diagnosing diseases.
            \end{itemize}
        \item \textbf{Automated Knowledge Acquisition:}
            \begin{itemize}
                \item Algorithms gather knowledge from large datasets.
                \item Example: Machine learning models predicting stock market trends.
            \end{itemize}
        \item \textbf{Reinforcement Learning:}
            \begin{itemize}
                \item Agents learn from environmental interactions using rewards or penalties.
                \item Example: A robot navigating a maze.
            \end{itemize}
        \item \textbf{Ontologies:}
            \begin{itemize}
                \item Structures that organize knowledge hierarchically for clearer inference.
                \item Example: Medical ontologies organizing diseases and treatments.
            \end{itemize}
    \end{enumerate}
\end{frame}

\begin{frame}[fragile]
    \frametitle{Knowledge Representation}
    \begin{block}{Importance of Knowledge Representation}
        Effective representation is crucial for AI agents to utilize acquired knowledge.
    \end{block}
    Common methods include:
    \begin{itemize}
        \item \textbf{Rules:} If-then statements guiding decision-making.
        \item \textbf{Frames:} Structures that hold knowledge similarly to objects in programming.
        \item \textbf{Semantic Networks:} Graph structures representing relationships between concepts.
    \end{itemize}
    
    \begin{block}{Key Points}
        \begin{itemize}
            \item Knowledge underpins intelligent systems, enabling independence and meaningful interactions.
            \item Continuous learning and updating knowledge bases are essential for relevance and accuracy.
            \item Combining different acquisition methods enhances robustness and coverage.
        \end{itemize}
    \end{block}
\end{frame}

\begin{frame}[fragile]
    \frametitle{Search Algorithms - Overview}
    Search algorithms are fundamental techniques used in AI to navigate through problem spaces to find solutions or optimize results. They can be broadly categorized into two types:
    
    \begin{itemize}
        \item \textbf{Uninformed Search}
        \item \textbf{Informed Search}
    \end{itemize}
\end{frame}

\begin{frame}[fragile]
    \frametitle{Search Algorithms - Uninformed Search}
    \textbf{Uninformed Search Algorithms} do not have additional information about the problem domain. They explore possible solutions without guidance.

    \textbf{Key Examples:}
    
    \begin{itemize}
        \item \textbf{Breadth-First Search (BFS)}
        \begin{itemize}
            \item Explores all nodes at the present depth before moving to the next level.
            \item \textbf{Characteristics:}
            \begin{itemize}
                \item Complete: Yes
                \item Time Complexity: $O(b^d)$, where $b$ = branching factor, $d$ = depth of the goal
                \item Space Complexity: $O(b^d)$
            \end{itemize}
        \end{itemize}
        
        \item \textbf{Depth-First Search (DFS)}
        \begin{itemize}
            \item Explores as far as possible along each branch before backtracking.
            \item \textbf{Characteristics:}
            \begin{itemize}
                \item Complete: No (if there are infinite paths)
                \item Time Complexity: $O(b^m)$, where $m$ = maximum depth
                \item Space Complexity: $O(b \cdot m)$
            \end{itemize}
        \end{itemize}
    \end{itemize}
\end{frame}

\begin{frame}[fragile]
    \frametitle{Search Algorithms - Informed Search}
    \textbf{Informed Search Algorithms} utilize additional information (heuristics) to guide their search, making them generally more efficient than uninformed methods.

    \textbf{Key Example:}
    
    \begin{itemize}
        \item \textbf{A* Search}
        \begin{itemize}
            \item Combines the benefits of BFS and heuristics by evaluating paths using a cost function 
            \[
            f(n) = g(n) + h(n)
            \]
            where:
            \begin{itemize}
                \item $g(n)$ = cost to reach node \(n\)
                \item $h(n)$ = estimated cost from \(n\) to the goal (heuristic)
            \end{itemize}
            \item \textbf{Characteristics:}
            \begin{itemize}
                \item Complete: Yes
                \item Time Complexity: $O(b^d)$ in the worst case
                \item Space Complexity: $O(b^d)$
            \end{itemize}
        \end{itemize}
    \end{itemize}
\end{frame}

\begin{frame}[fragile]
    \frametitle{Search Algorithms - Conclusion}
    \begin{block}{Key Points to Emphasize}
        \begin{itemize}
            \item \textbf{Choice of Algorithm:} Depends on problem characteristics.
            \item \textbf{Trade-offs:} 
            \begin{itemize}
                \item Uninformed searches are easier but may take longer and require more memory.
                \item Informed searches are typically more efficient due to heuristics.
            \end{itemize}
        \end{itemize}
    \end{block}

    Search algorithms are critical tools in AI for problem-solving and optimization. Understanding the differences between uninformed and informed search methods equips you to address various challenges in AI applications effectively. 
\end{frame}

\begin{frame}[fragile]
    \frametitle{Heuristic Search Methods - Introduction}
    Heuristic search methods are strategies that help identify the most promising path to a solution in search algorithms by utilizing domain-specific knowledge. They aim to enhance the efficiency of finding solutions by estimating costs or distances to the goal, thereby guiding the search process more effectively compared to uninformed search methods.
\end{frame}

\begin{frame}[fragile]
    \frametitle{Heuristic Search Methods - Key Concepts}
    \begin{enumerate}
        \item \textbf{Heuristic Function (h(n))}:
            \begin{itemize}
                \item Estimates the cost to reach the goal from a node \( n \).
                \item A well-designed heuristic significantly improves search performance.
                \item \begin{equation}
                    f(n) = g(n) + h(n)
                \end{equation}
                where:
                \begin{itemize}
                    \item \( f(n) \) = total estimated cost of the cheapest solution through node \( n \)
                    \item \( g(n) \) = cost from the start node to node \( n \)
                    \item \( h(n) \) = estimated cost from node \( n \) to the goal
                \end{itemize}
            \end{itemize}

        \item \textbf{Types of Heuristics}:
            \begin{itemize}
                \item \textbf{Admissible Heuristics}: Never overestimate the cost to the goal. Example: Straight-line distance in pathfinding.
                \item \textbf{Consistent (Monotonic) Heuristics}: Satisfy \( h(n) \leq c(n, n') + h(n') \) indicating that the heuristic never decreases along the path.
            \end{itemize}
    \end{enumerate}
\end{frame}

\begin{frame}[fragile]
    \frametitle{Heuristic Search Methods - Common Algorithms}
    \begin{itemize}
        \item \textbf{A* Algorithm}:
            \begin{itemize}
                \item Combines Dijkstra's algorithm benefits and heuristic (h(n)).
                \item Uses \( f(n) = g(n) + h(n) \).
                \item Suitable for pathfinding in maps, game AI, and robotics.
            \end{itemize}

        \item \textbf{Greedy Best-First Search}:
            \begin{itemize}
                \item Utilizes only the heuristic function (h(n)) for exploration.
                \item Prioritizes exploring promising paths based on heuristic but may miss optimal paths.
            \end{itemize}
    \end{itemize}
\end{frame}

\begin{frame}[fragile]
    \frametitle{Heuristic Search Methods - Applications and Key Points}
    \begin{enumerate}
        \item \textbf{Applications}:
            \begin{itemize}
                \item Pathfinding in GPS, robotics, and video games.
                \item Puzzle solvers for problems like the 8-puzzle or Rubik’s cube.
                \item AI planning and decision-making tasks for optimal solutions.
            \end{itemize}

        \item \textbf{Key Points to Emphasize}:
            \begin{itemize}
                \item Significantly optimize search processes by using domain knowledge.
                \item Proper heuristic selection can differentiate feasible and infeasible search times.
                \item Understanding different heuristics aids in designing efficient algorithms.
            \end{itemize}
    \end{enumerate}
\end{frame}

\begin{frame}[fragile]
    \frametitle{Heuristic Search Methods - Conclusion}
    Heuristic search methods are essential for advancing intelligent search strategies in AI. By leveraging cost estimates, these methods not only enhance efficiency but also expand the range of solvable problems. Gaining a thorough understanding of these techniques is critical for anyone aspiring to excel in AI and computational problem-solving.
\end{frame}

\begin{frame}[fragile]
    \frametitle{Logic Reasoning}
    \begin{block}{Overview}
        Logic reasoning is a crucial component of artificial intelligence (AI) and is foundational for building intelligent agents. It allows agents to make deductions, infer new knowledge, and solve problems based on given information.
    \end{block}
\end{frame}

\begin{frame}[fragile]
    \frametitle{Forms of Logic Reasoning}
    \begin{enumerate}
        \item \textbf{Propositional Logic (Boolean Logic)}
        \item \textbf{First-Order Logic (Predicate Logic)}
    \end{enumerate}
\end{frame}

\begin{frame}[fragile]
    \frametitle{Propositional Logic}
    \begin{block}{Definition}
        A type of logic that deals with propositions which can either be true or false.
    \end{block}
    
    \begin{block}{Symbols}
        Typically uses symbols such as:
        \begin{itemize}
            \item $P, Q, R$: Propositions
            \item $\land$: AND
            \item $\lor$: OR
            \item $\neg$: NOT
            \item $\rightarrow$: IMPLIES
        \end{itemize}
    \end{block}

    \begin{block}{Example}
        - Propositions:
          - $P$: "It is raining."
          - $Q$: "I will carry an umbrella."
        - Logical Expression:
          - If it rains, then I will carry an umbrella: $P \rightarrow Q$.
    \end{block}
\end{frame}

\begin{frame}[fragile]
    \frametitle{Truth Table for Propositional Logic}
    \begin{center}
        \begin{tabular}{|c|c|c|}
            \hline
            $P$ & $Q$ & $P \rightarrow Q$ \\
            \hline
            T & T & T \\
            T & F & F \\
            F & T & T \\
            F & F & T \\
            \hline
        \end{tabular}
    \end{center}
\end{frame}

\begin{frame}[fragile]
    \frametitle{First-Order Logic}
    \begin{block}{Definition}
        Extends propositional logic by including quantifiers and predicates that express more complex statements.
    \end{block}
    
    \begin{block}{Components}
        \begin{itemize}
            \item \textbf{Predicates:} Used to express properties or relations (e.g., $Loves(John, Mary)$).
            \item \textbf{Quantifiers:}
                \begin{itemize}
                    \item Universal Quantifier ($\forall$): For all elements in a domain (e.g., $\forall x (Human(x) \rightarrow Mortal(x))$).
                    \item Existential Quantifier ($\exists$): There exists at least one element for which the statement is true (e.g., $\exists y (Cat(y) \land Loves(y, John))$).
                \end{itemize}
        \end{itemize}
    \end{block}
\end{frame}

\begin{frame}[fragile]
    \frametitle{Examples of First-Order Logic}
    \begin{itemize}
        \item All humans are mortal: $\forall x (Human(x) \rightarrow Mortal(x))$.
        \item There exists a cat that loves John: $\exists y (Cat(y) \land Loves(y, John))$.
    \end{itemize}
\end{frame}

\begin{frame}[fragile]
    \frametitle{Key Points to Emphasize}
    \begin{itemize}
        \item \textbf{Importance of Logic:} Helps AI agents in decision-making, knowledge representation, and inference.
        \item \textbf{Propositional vs. First-Order Logic:} 
            \begin{itemize}
                \item Propositional logic is limited to simple true/false assertions.
                \item First-order logic can represent more complex relationships and quantifications.
            \end{itemize}
        \item \textbf{Applications in AI:} Used in automated theorem proving, knowledge-based systems, and natural language processing.
    \end{itemize}
\end{frame}

\begin{frame}[fragile]
    \frametitle{Conclusion}
    Understanding different forms of logic reasoning is essential for the development of intelligent agents, as it allows them to process information and make logical deductions.
\end{frame}

\begin{frame}[fragile]
    \frametitle{Probabilistic Reasoning - Overview}
    \begin{block}{Overview of Probabilistic Reasoning}
        Probabilistic reasoning is a method used in artificial intelligence (AI) to handle uncertainty in environments. Unlike deterministic reasoning that relies on fixed values, probabilistic reasoning accounts for the inherent unpredictability in real-world situations.
    \end{block}
    \begin{itemize}
        \item Allows agents to make informed decisions despite incomplete or uncertain information.
        \item Vital for applications requiring decision-making under uncertainty.
    \end{itemize}
\end{frame}

\begin{frame}[fragile]
    \frametitle{Probabilistic Reasoning - Key Concepts}
    \begin{enumerate}
        \item \textbf{Probability Basics}:
            \begin{itemize}
                \item **Probability** measures how likely an event is to occur, ranging from 0 (impossible) to 1 (certain).
                \item **Random Variables**: Variables that can take on different values with associated probabilities.
            \end{itemize}
        \item \textbf{Bayesian Networks}:
            \begin{itemize}
                \item A graphical model representing a set of variables and their conditional dependencies using directed acyclic graphs (DAGs).
                \item Nodes represent random variables, and edges depict probabilistic dependencies.
                \item Enables efficient computation of joint probabilities and inference in uncertain domains.
            \end{itemize}
    \end{enumerate}
\end{frame}

\begin{frame}[fragile]
    \frametitle{Probabilistic Reasoning - Example}
    \begin{block}{Example: Medical Diagnosis}
        Consider the following scenario:
        \begin{itemize}
            \item Variables: 
                \begin{itemize}
                    \item A = Disease Present (True/False)
                    \item B = Tested Positive (True/False)
                \end{itemize}
            \item Conditional Probabilities:
                \begin{itemize}
                    \item $P(B \mid A=True) = 0.9$ (High probability of testing positive if the disease is present)
                    \item $P(B \mid A=False) = 0.1$ (Lower probability of testing positive if the disease is not present)
                \end{itemize}
        \end{itemize}
        Using Bayes’ Theorem:
        \begin{equation}
            P(A \mid B) = \frac{P(B \mid A) \cdot P(A)}{P(B)}
        \end{equation}
        This allows calculating the probability that a patient has the disease given a positive test result.
    \end{block}
\end{frame}

\begin{frame}[fragile]
    \frametitle{Probabilistic Reasoning - Applications}
    \begin{block}{Applications}
        \begin{itemize}
            \item \textbf{Medical Diagnosis}: Inferring diseases from symptoms.
            \item \textbf{Spam Detection}: Classifying emails as spam or not based on keyword probabilities.
            \item \textbf{Robotics}: Navigating uncertain environments through estimating the likelihood of various states.
        \end{itemize}
    \end{block}
    \begin{block}{Key Points to Emphasize}
        \begin{itemize}
            \item Probabilistic models are fundamental for understanding and reasoning in uncertain environments.
            \item Bayesian networks provide a structured approach to represent and infer probabilistic relationships.
            \item Updating beliefs as new evidence is crucial for adaptive AI.
        \end{itemize}
    \end{block}
\end{frame}

\begin{frame}[fragile]
    \frametitle{Probabilistic Reasoning - Conclusion}
    \begin{block}{Conclusion}
        Probabilistic reasoning equips AI agents with tools to make decisions when certainty is not guaranteed. Understanding these concepts is essential for developing intelligent systems capable of operating in the real world.
    \end{block}
    \begin{itemize}
        \item Effectiveness lies not only in theoretical foundations but also in practical applications that enhance decision-making processes.
    \end{itemize}
\end{frame}

\begin{frame}[fragile]
    \frametitle{Decision Making and Planning - Overview}
    \begin{block}{Overview of Decision-Making Techniques in AI}
        Decision-making in AI involves selecting the best course of action from a set of alternatives based on certain criteria. It is essential for autonomous agents that operate in uncertain environments.
    \end{block}
    
    \begin{enumerate}
        \item \textbf{Deterministic Decision-Making:}
            \begin{itemize}
                \item Assumes outcomes are predictable; same action yields same result.
                \item \textit{Example}: A robot programmed to pick up a ball in a clutter-free environment.
            \end{itemize}
        
        \item \textbf{Probabilistic Decision-Making:}
            \begin{itemize}
                \item Deals with uncertainty by incorporating randomness into decision processes.
                \item \textit{Example}: A self-driving car estimating the likelihood of a pedestrian crossing the road.
            \end{itemize}
        
        \item \textbf{Static vs. Dynamic Decision-Making:}
            \begin{itemize}
                \item \textit{Static}: Decisions are made once and remain constant.
                \item \textit{Dynamic}: Decisions can be updated as new information becomes available.
            \end{itemize}
    \end{enumerate}
\end{frame}

\begin{frame}[fragile]
    \frametitle{Decision Making and Planning - Markov Decision Processes}
    \begin{block}{Markov Decision Processes (MDPs)}
        MDPs provide a mathematical framework for modeling decision-making in environments where outcomes are partly random and partly under the control of a decision-maker.
    \end{block}
    
    \begin{itemize}
        \item \textbf{States (S)}: All possible situations in which an agent can find itself.
        \item \textbf{Actions (A)}: Choices available to the agent in each state.
        \item \textbf{Transition Function (T)}: 
            \begin{equation}
                T(s, a, s') = P(s' \mid s, a)
            \end{equation}
        \item \textbf{Rewards (R)}: Immediate return after transitioning from one state to another via an action.
        \item \textbf{Policy (π)}: A strategy that defines the action to take in each state, which can be deterministic or stochastic.
    \end{itemize}
\end{frame}

\begin{frame}[fragile]
    \frametitle{Decision Making and Planning - Expected Reward}
    \begin{block}{Formula for Expected Reward}
        The expected value of the total reward from a state \( s \) can be defined as:
        \begin{equation}
            V(s) = R(s) + \gamma \sum_{s' \in S} T(s, a, s')V(s')
        \end{equation}
        Where \( \gamma \) (discount factor) determines the importance of future rewards.
    \end{block}
    
    \begin{block}{Examples of MDPs in Practice}
        \begin{itemize}
            \item \textit{Robotic Navigation}: A robot deciding which path to take to reach its goal while avoiding obstacles.
            \item \textit{Resource Allocation}: Businesses allocating resources to maximize profit considering market uncertainties.
        \end{itemize}
    \end{block}

    \begin{block}{Key Points}
        \begin{itemize}
            \item Decision-making is vital for autonomous systems in unpredictable environments.
            \item MDPs offer a structured approach to model and solve complex decision-making problems.
            \item Understanding MDP components helps in developing effective policies for various applications.
        \end{itemize}
    \end{block}
\end{frame}

\begin{frame}[fragile]
    \frametitle{Reinforcement Learning - Introduction}
    \begin{block}{Overview}
        Reinforcement Learning (RL) is a machine learning area focused on how agents take actions in an environment to maximize cumulative rewards.
    \end{block}
    
    \begin{block}{Key Concepts}
        \begin{itemize}
            \item \textbf{Agent}: The decision-maker, e.g., a robot or program.
            \item \textbf{Environment}: The context where the agent operates.
            \item \textbf{Actions}: Choices made by the agent affecting the environment.
            \item \textbf{States}: Different situations the agent encounters.
            \item \textbf{Rewards}: Feedback from the environment, positive or negative.
        \end{itemize}
    \end{block}
\end{frame}

\begin{frame}[fragile]
    \frametitle{Reinforcement Learning - How It Works}
    \begin{itemize}
        \item \textbf{Trial and Error}: Agents explore environments through various actions.
        \item \textbf{Reward Signals}: Feedback received after taking actions indicates success or failure.
        \item \textbf{Learning Policy}: Agents update strategies based on rewards to maximize long-term benefits.
    \end{itemize}
\end{frame}

\begin{frame}[fragile]
    \frametitle{Reinforcement Learning - Example and Key Points}
    \begin{block}{Example: Training a Robot to Navigate a Maze}
        \begin{itemize}
            \item \textbf{Objective}: Navigate from start to finish.
            \item \textbf{States}: Each position in the maze.
            \item \textbf{Actions}: Move up, down, left, or right.
            \item \textbf{Rewards}: Positive for reaching the finish; negative for hitting walls.
        \end{itemize}
    \end{block}
    
    \begin{block}{Key Points}
        \begin{itemize}
            \item \textbf{Exploration vs. Exploitation}: Balance between discovering new strategies and utilizing known ones.
            \item \textbf{Discount Factor ($\gamma$)}: Importance of future rewards, where $0 \leq \gamma < 1$.
        \end{itemize}
    \end{block}
\end{frame}

\begin{frame}[fragile]
    \frametitle{Reinforcement Learning - Formula and Conclusion}
    \begin{block}{Basic Reinforcement Learning Formula}
        The value function is given by:
        \begin{equation}
            V(s) = \sum_a \pi(a|s) \sum_{s'} P(s'|s, a) [R(s, a, s') + \gamma V(s')]
        \end{equation}
        Where:
        \begin{itemize}
            \item $V(s)$: Expected value of state $s$
            \item $\pi(a|s)$: Policy, probability of action $a$ in state $s$
            \item $P(s'|s, a)$: Probability of reaching state $s'$ from state $s$ after action $a$
            \item $R(s, a, s')$: Reward for moving from $s$ to $s'$ after action $a$
        \end{itemize}
    \end{block}
    
    \begin{block}{Conclusion}
        RL provides a framework for agents to learn and optimize behaviors through interactions, essential for AI decision-making tasks.
    \end{block}
\end{frame}

\begin{frame}[fragile]
    \frametitle{Reinforcement Learning - Engagement}
    \begin{block}{Engagement and Further Learning}
        Students should explore basic RL environments such as OpenAI Gym to reinforce understanding through hands-on experience.
    \end{block}
\end{frame}

\begin{frame}[fragile]
    \frametitle{Ethical Considerations in AI - Introduction}
    \begin{block}{Definition of Ethics}
        Ethics involves the principles that govern a person's behavior or conducting of an activity. In AI, it focuses on understanding right from wrong concerning the design, development, and application of AI technologies.
    \end{block}
\end{frame}

\begin{frame}[fragile]
    \frametitle{Ethical Considerations in AI - Key Implications}
    \begin{enumerate}
        \item \textbf{Bias and Fairness:}
            \begin{itemize}
                \item AI systems can inherit biases from their training data.
                \item Example: A hiring algorithm that favors male candidates over female candidates due to biased historical hiring data.
            \end{itemize}
        
        \item \textbf{Transparency:}
            \begin{itemize}
                \item Many AI systems operate as "black boxes."
                \item Example: Algorithms in judicial sentencing without explaining the basis for recommendations.
            \end{itemize}

        \item \textbf{Accountability:}
            \begin{itemize}
                \item Determining responsibility when an AI makes a mistake can be complex.
                \item Example: Accidents involving autonomous vehicles raise liability questions.
            \end{itemize}

        \item \textbf{Privacy:}
            \begin{itemize}
                \item AI analyzing personal data can infringe on privacy rights.
                \item Example: Facial recognition technology used in surveillance without consent.
            \end{itemize}

        \item \textbf{Job Displacement:}
            \begin{itemize}
                \item Automation can lead to job losses and economic implications.
                \item Example: Self-checkout systems in retail reducing the need for cashiers.
            \end{itemize}
    \end{enumerate}
\end{frame}

\begin{frame}[fragile]
    \frametitle{Ethical Considerations in AI - Dilemmas and Impact}
    \begin{block}{Potential Dilemmas}
        \begin{itemize}
            \item \textbf{Moral Decision-Making:} AI must make tough ethical decisions in critical areas like healthcare and military.
            \item \textbf{Surveillance vs. Security:} Balancing AI's use for security with implications for personal freedoms.
        \end{itemize}
    \end{block}

    \begin{block}{Societal Impact}
        \begin{itemize}
            \item \textbf{Benefits:} Enhanced efficiency and improved healthcare.
            \item \textbf{Challenges:} Increased social inequality and reliance on technology.
        \end{itemize}
    \end{block}
\end{frame}

\begin{frame}[fragile]
    \frametitle{Ethical Considerations in AI - Key Points and Conclusion}
    \begin{itemize}
        \item Ethics in AI is about human values, not just technology.
        \item Continuous dialogue among technologists, ethicists, and the public is essential.
        \item Organizations must implement ethical guidelines to minimize biases.
    \end{itemize}

    \begin{block}{Concluding Thoughts}
        As AI systems increasingly influence our lives, recognizing and addressing ethical considerations is crucial for trust and ensuring technology serves humanity positively.
    \end{block}
\end{frame}

\begin{frame}[fragile]
    \frametitle{Collaborative Problem Solving - Introduction}
    \begin{block}{What is Collaborative Problem Solving?}
        Collaborative problem solving refers to the process by which individuals or agents work together to solve a problem or achieve a common goal. In the realm of Artificial Intelligence (AI), this often involves multiple agents or teams leveraging their unique skills and perspectives to find effective solutions to complex challenges.
    \end{block}
\end{frame}

\begin{frame}[fragile]
    \frametitle{Collaborative Problem Solving - Importance of Teamwork}
    \begin{itemize}
        \item \textbf{Interdisciplinary Approach:} AI encompasses fields like computer science, psychology, ethics, and design. Collaboration combines diverse perspectives, enhancing innovation.
        \item \textbf{Diversity of Thought:} Group work encourages the sharing of different viewpoints, leading to better issue identification and a wider range of solutions.
        \item \textbf{Enhanced Skill Development:} Team-based projects help develop essential skills such as communication, negotiation, and conflict resolution.
    \end{itemize}
\end{frame}

\begin{frame}[fragile]
    \frametitle{Collaborative Problem Solving - Examples}
    \begin{enumerate}
        \item \textbf{Autonomous Vehicle Development:} 
        Multiple stakeholders, including engineers, ethicists, and regulators, collaborate to create safety algorithms while ensuring compliance with laws.
        
        \item \textbf{Crowdsourced Data Annotation:} 
        Platforms allow many contributors to annotate images for AI applications, improving the dataset quality needed for training models.
    \end{enumerate}
\end{frame}

\begin{frame}[fragile]
    \frametitle{Collaborative Problem Solving - Key Points}
    \begin{itemize}
        \item \textbf{Collaborative Tools and Techniques:} Use tools like Slack, Trello, or GitHub for project management and communication.
        \item \textbf{Agile Methodology:} Adopt agile practices such as sprints and retrospectives to improve teamwork dynamics.
        \item \textbf{Shared Responsibility:} Encourage accountability by ensuring each member feels invested in the project's success.
    \end{itemize}
\end{frame}

\begin{frame}[fragile]
    \frametitle{Collaborative Problem Solving - Tips for Success}
    \begin{itemize}
        \item \textbf{Define Clear Roles:} Establish roles based on each member’s strengths (e.g., project manager, AI specialist).
        \item \textbf{Regular Check-ins:} Schedule meetings to discuss progress and challenges, maintaining momentum.
        \item \textbf{Cultivate a Supportive Environment:} Encourage open communication where ideas can be shared freely without judgment.
    \end{itemize}
\end{frame}

\begin{frame}[fragile]
    \frametitle{Collaborative Problem Solving - Conclusion}
    \begin{block}{Conclusion}
        Collaboration in AI enhances problem-solving efficiency and prepares individuals for workplace situations where teamwork is essential. An environment of cooperation, inclusion, and innovative thinking can lead to advancements in AI technologies.
    \end{block}
\end{frame}

\begin{frame}[fragile]
    \frametitle{Next Topic}
    \begin{block}{Assessment and Evaluation}
        Next, we will cover the assessment and evaluation of collaborative projects and learning outcomes in AI.
    \end{block}
\end{frame}

\begin{frame}[fragile]
    \frametitle{Assessment and Evaluation - Overview}
    In this course on AI and Agent Architectures, we will employ a variety of assessment methods designed to evaluate your understanding and mastery of the material.
    
    \begin{itemize}
        \item Programming Assignments
        \item Projects
        \item Exams
    \end{itemize}
\end{frame}

\begin{frame}[fragile]
    \frametitle{Assessment and Evaluation - Programming Assignments}
    \textbf{1. Programming Assignments}
    
    \begin{itemize}
        \item \textbf{Purpose}: Hands-on experience with coding AI algorithms and building agent architectures using languages like Python.
        \item \textbf{Examples}:
        \begin{itemize}
            \item Implementing basic AI algorithms (e.g., search algorithms, decision trees).
            \item Creating a chatbot using natural language processing.
        \end{itemize}
        \item \textbf{Key Points}:
        \begin{itemize}
            \item Assignments are typically due bi-weekly.
            \item Collaboration with peers is encouraged, but each submission must be original work.
        \end{itemize}
    \end{itemize}
\end{frame}

\begin{frame}[fragile]
    \frametitle{Assessment and Evaluation - Projects}
    \textbf{2. Projects}
    
    \begin{itemize}
        \item \textbf{Purpose}: Foster deeper learning through practical application of complex problems within AI.
        \item \textbf{Structure}:
        \begin{itemize}
            \item Group projects for collaborative problem-solving.
            \item Individual projects for personal exploration of AI topics.
        \end{itemize}
        \item \textbf{Examples}:
        \begin{itemize}
            \item Developing a multi-agent system for a simulated environment.
            \item Analyzing the performance of algorithms in a competitive setting.
        \end{itemize}
        \item \textbf{Key Points}:
        \begin{itemize}
            \item Project milestones to ensure steady progress (proposal, mid-project review, final submission).
        \end{itemize}
    \end{itemize}
\end{frame}

\begin{frame}[fragile]
    \frametitle{Assessment and Evaluation - Exams and Criteria}
    \textbf{3. Exams}
    
    \begin{itemize}
        \item \textbf{Purpose}: Assess theoretical understanding and ability to apply concepts learned.
        \item \textbf{Format}:
        \begin{itemize}
            \item Midterm Exam: Covers foundational concepts and initial assignments.
            \item Final Exam: Comprehensive assessment including advanced topics.
        \end{itemize}
        \item \textbf{Key Points}:
        \begin{itemize}
            \item Exams include multiple-choice, short answers, and coding questions.
            \item Regular review sessions scheduled before exams.
        \end{itemize}
    \end{itemize}

    \textbf{Assessment Criteria}
    
    \begin{itemize}
        \item Grading Rubric: Clear criteria for each assessment focusing on problem-solving, efficiency, and understanding.
        \item Feedback: Constructive feedback to help improve skills and knowledge.
    \end{itemize}
\end{frame}

\begin{frame}[fragile]
    \frametitle{Assessment and Evaluation - Conclusion}
    Engagement with these assessment methods is crucial for your success in understanding AI principles and developing effective agent architectures. 

    \begin{itemize}
        \item Integration of various assessments ensures both theoretical knowledge and practical experience.
        \item Prepares you for real-world applications of AI.
    \end{itemize}

    \textbf{Reminder}: Reach out for clarification or support as you navigate through these assessments. Embrace this learning journey!
\end{frame}

\begin{frame}[fragile]
    \frametitle{Course Resources and Tools - Overview}
    \begin{block}{Introduction}
        In this course, we will explore Artificial Intelligence (AI) and Agent Architectures. Familiarizing yourself with the necessary computing resources and software tools will be vital for your success. 
    \end{block}
    \begin{itemize}
        \item Essential tools for programming in Python
        \item Software libraries and platforms for AI development
    \end{itemize}
\end{frame}

\begin{frame}[fragile]
    \frametitle{Key Computing Resources}
    \begin{block}{Hardware Requirements}
        \begin{enumerate}
            \item \textbf{Processor:} Minimum Intel i5 or equivalent AMD 
            \item \textbf{RAM:} At least 8GB
            \item \textbf{Disk Space:} Minimum 20GB free space 
        \end{enumerate}
    \end{block}
    \begin{block}{Software Requirements}
        \begin{itemize}
            \item \textbf{Operating System:} Windows, macOS, or Linux (up to date)
            \item \textbf{Python:} Download latest version (3.7 or later)
        \end{itemize}
    \end{block}
\end{frame}

\begin{frame}[fragile]
    \frametitle{Essential Python Libraries}
    \begin{itemize}
        \item \textbf{NumPy:} Numerical operations and multi-dimensional arrays
        \begin{lstlisting}
        pip install numpy
        import numpy as np
        array = np.array([[1, 2], [3, 4]])
        print(array)
        \end{lstlisting}

        \item \textbf{Pandas:} Data manipulation for tabular data
        \begin{lstlisting}
        pip install pandas
        import pandas as pd
        data = pd.DataFrame({'A': [1, 2], 'B': [3, 4]})
        print(data)
        \end{lstlisting}

        \item \textbf{Matplotlib \& Seaborn:} Data visualization tools
        \begin{lstlisting}
        pip install matplotlib seaborn
        import matplotlib.pyplot as plt
        plt.plot([1, 2, 3], [4, 5, 1])
        plt.show()
        \end{lstlisting}

        \item \textbf{Scikit-learn:} Machine learning library
        \begin{lstlisting}
        pip install scikit-learn
        from sklearn.linear_model import LinearRegression
        model = LinearRegression()
        \end{lstlisting}

        \item \textbf{TensorFlow \& PyTorch:} Deep learning framework
        \begin{lstlisting}
        pip install tensorflow     # or
        pip install torch torchvision
        \end{lstlisting}
    \end{itemize}
\end{frame}

\begin{frame}[fragile]
    \frametitle{Development Environments and Version Control}
    \begin{block}{Development Environment}
        \begin{itemize}
            \item \textbf{Jupyter Notebook:} Ideal for interactive coding
            \item \textbf{Visual Studio Code:} Versatile IDE with extensions
            \item \textbf{PyCharm:} Python-focused IDE with code assistance
        \end{itemize}
    \end{block}

    \begin{block}{Version Control}
        \begin{itemize}
            \item \textbf{Git:} Essential for tracking code changes
            \begin{itemize}
                \item Installation: Follow the instructions at \url{https://git-scm.com/downloads}
            \end{itemize}
        \end{itemize}
    \end{block}

\end{frame}

\begin{frame}
    \frametitle{Course Objectives - Overview}
    \begin{block}{Course Overview}
    This course aims to provide a foundational understanding of Artificial Intelligence (AI) and Agent Architectures, equipping students with the necessary skills to conceptualize, design, and implement AI systems.
    \end{block}
\end{frame}

\begin{frame}
    \frametitle{Course Objectives - Learning Objectives}
    By the end of this course, you will be able to:
    \begin{enumerate}
        \item \textbf{Define Key AI Concepts}
        \begin{itemize}
            \item Understand fundamental concepts such as machine learning, neural networks, and natural language processing.
            \item \textbf{Example:} Differentiate between supervised and unsupervised learning, recognizing how each impacts model training.
        \end{itemize}
        
        \item \textbf{Explore Agent Architectures}
        \begin{itemize}
            \item Comprehend various architectures for intelligent agents including reactive, deliberative, and hybrid systems.
            \item \textbf{Key Point:} Different architectures serve different purposes; for instance, reactive agents respond immediately to stimuli, while deliberative agents plan based on foresight.
        \end{itemize}
        
        \item \textbf{Utilize AI Tools and Libraries}
        \begin{itemize}
            \item Gain proficiency in using programming tools essential for AI development, particularly Python-based libraries (such as TensorFlow, Keras, and Scikit-learn).
        \end{itemize}
    \end{enumerate}
\end{frame}

\begin{frame}[fragile]
    \frametitle{Course Objectives - Example Code}
    \begin{block}{Code Snippet}
    \begin{lstlisting}[language=Python]
import numpy as np
from sklearn.model_selection import train_test_split
from sklearn.linear_model import LinearRegression

# Sample dataset
X, y = np.arange(10).reshape(-1, 1), np.arange(10)
X_train, X_test, y_train, y_test = train_test_split(X, y, test_size=0.2)

# Model training
model = LinearRegression().fit(X_train, y_train)
    \end{lstlisting}
    \end{block}
\end{frame}

\begin{frame}
    \frametitle{Course Objectives - Performance Indicators}
    To measure your understanding and progression throughout the course, we will focus on:
    \begin{itemize}
        \item \textbf{Participation in discussions and practical exercises:} Engaging actively with the material and peers to enhance learning.
        \item \textbf{Quizzes and Assignments:} Regular assessments will test comprehension of key concepts and coding skills related to AI and agent architectures.
        \item \textbf{Final Project:} A capstone project will demonstrate your ability to integrate learned concepts into a coherent AI application, showcasing both technical skills and creativity.
    \end{itemize}
\end{frame}

\begin{frame}[fragile]
    \frametitle{Conclusion and Next Steps - Summary of Key Concepts}
    \begin{block}{Introduction to AI}
        \begin{itemize}
            \item \textbf{Definition:} Artificial Intelligence (AI) refers to the simulation of human intelligence processes by machines, particularly computer systems.
            \item \textbf{Core Components:} Learning, reasoning, problem-solving, perception, and language understanding.
        \end{itemize}
    \end{block}
    
    \begin{block}{Agent Architectures}
        \begin{itemize}
            \item \textbf{Definition:} An agent architecture is a framework for creating intelligent agents that interact with their environment and make decisions.
            \item \textbf{Types of Architectures:}
            \begin{itemize}
                \item Reactive Agents: Respond to stimuli without internal representation (e.g., a basic thermostat).
                \item Deliberative Agents: Use internal models to make decisions (e.g., a chess-playing computer).
                \item Hybrid Agents: Combine both reactive and deliberative approaches (e.g., autonomous vehicles).
            \end{itemize}
        \end{itemize}
    \end{block}
\end{frame}

\begin{frame}[fragile]
    \frametitle{Conclusion and Next Steps - Importance of AI}
    \begin{block}{Importance of AI and Agent Architectures}
        \begin{itemize}
            \item \textbf{Real-World Applications:} AI is transforming industries including healthcare (diagnostic algorithms), finance (fraud detection systems), and transportation (self-driving cars).
            \item \textbf{Enhancing Human Capabilities:} AI systems augment human decision-making and can handle complex data processing at high speeds.
        \end{itemize}
    \end{block}
    
    \begin{block}{Key Points to Emphasize}
        \begin{itemize}
            \item Understanding AI and agent architectures is crucial for designing effective solutions to real-world problems.
            \item Familiarity with different agent types and architectures allows for better implementation of AI solutions.
        \end{itemize}
    \end{block}
\end{frame}

\begin{frame}[fragile]
    \frametitle{Conclusion and Next Steps - Future Directions}
    \begin{block}{Next Steps in the Course}
        \begin{enumerate}
            \item \textbf{Exploration of AI Techniques:} We will dive deeper into various AI methods, such as machine learning, neural networks, and natural language processing.
            \item \textbf{Hands-On Projects:} Practical assignments will reinforce your understanding of AI applications through development and analysis of intelligent agents.
            \item \textbf{Discussion and Collaboration:} Engage in collaborative projects and discussions focusing on current advancements and ethical implications of AI.
        \end{enumerate}
    \end{block}
    
    \begin{block}{Example to Illustrate}
        Consider a \textbf{smart home assistant} (e.g., Amazon Alexa):
        \begin{itemize}
            \item \textbf{Architecture:} Operates on a hybrid agent framework.
            \item \textbf{Functionality:} Reacts to voice commands (reactive) while leveraging machine learning to adapt to user preferences (deliberative).
        \end{itemize}
    \end{block}
\end{frame}

\begin{frame}[fragile]
    \frametitle{Conclusion and Next Steps - Code Snippet}
    \begin{block}{Code Snippet: K-Nearest Neighbors (KNN) Example}
        \begin{lstlisting}[language=Python]
from sklearn.neighbors import KNeighborsClassifier

# Sample data: features (e.g., height, weight) and labels (e.g., 'A', 'B')
X = [[1.80, 80], [1.60, 50], [1.75, 70]]
y = ['A', 'B', 'A']

# Creating the KNN model
knn = KNeighborsClassifier(n_neighbors=3)
knn.fit(X, y)

# Predicting the class for a new sample
prediction = knn.predict([[1.65, 55]])
        \end{lstlisting}
    \end{block}
    
    \begin{block}{Conclusion}
        This foundational knowledge sets the stage for exploring advanced AI concepts, ensuring you’re prepared for practical applications and theoretical discussions.
    \end{block}
\end{frame}


\end{document}