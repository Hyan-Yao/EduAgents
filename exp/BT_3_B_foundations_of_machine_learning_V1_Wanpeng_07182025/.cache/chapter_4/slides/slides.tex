\documentclass[aspectratio=169]{beamer}

% Theme and Color Setup
\usetheme{Madrid}
\usecolortheme{whale}
\useinnertheme{rectangles}
\useoutertheme{miniframes}

% Additional Packages
\usepackage[utf8]{inputenc}
\usepackage[T1]{fontenc}
\usepackage{graphicx}
\usepackage{booktabs}
\usepackage{listings}
\usepackage{amsmath}
\usepackage{amssymb}
\usepackage{xcolor}
\usepackage{tikz}
\usepackage{pgfplots}
\pgfplotsset{compat=1.18}
\usetikzlibrary{positioning}
\usepackage{hyperref}

% Custom Colors
\definecolor{myblue}{RGB}{31, 73, 125}
\definecolor{mygray}{RGB}{100, 100, 100}
\definecolor{mygreen}{RGB}{0, 128, 0}
\definecolor{myorange}{RGB}{230, 126, 34}
\definecolor{mycodebackground}{RGB}{245, 245, 245}

% Set Theme Colors
\setbeamercolor{structure}{fg=myblue}
\setbeamercolor{frametitle}{fg=white, bg=myblue}
\setbeamercolor{title}{fg=myblue}
\setbeamercolor{section in toc}{fg=myblue}
\setbeamercolor{item projected}{fg=white, bg=myblue}
\setbeamercolor{block title}{bg=myblue!20, fg=myblue}
\setbeamercolor{block body}{bg=myblue!10}
\setbeamercolor{alerted text}{fg=myorange}

% Set Fonts
\setbeamerfont{title}{size=\Large, series=\bfseries}
\setbeamerfont{frametitle}{size=\large, series=\bfseries}
\setbeamerfont{caption}{size=\small}
\setbeamerfont{footnote}{size=\tiny}

% Document Start
\begin{document}

\frame{\titlepage}

\begin{frame}[fragile]
    \frametitle{Introduction to Regression Techniques}
    \begin{block}{Overview of Regression Models in Machine Learning}
        Regression refers to a set of statistical and machine learning techniques used to estimate relationships among variables. It helps in predicting a dependent variable (target) based on one or more independent variables (features).
    \end{block}
\end{frame}

\begin{frame}[fragile]
    \frametitle{Importance of Regression Techniques}
    \begin{enumerate}
        \item \textbf{Predictive Modeling:}  
        Regression is primarily used for making predictions by modeling relationships in data.
        
        \item \textbf{Understanding Relationships:}  
        Analyzing changes in predictor variables clarifies how they affect the target variable.
        
        \item \textbf{Interpretable Results:}  
        Regression models yield coefficients showing the strength and direction of relationships, making them easier to interpret.
    \end{enumerate}
\end{frame}

\begin{frame}[fragile]
    \frametitle{Key Concepts in Regression}
    \begin{itemize}
        \item \textbf{Dependent Variable (Y):} The output variable we want to predict.
        \item \textbf{Independent Variables (X):} The input features influencing the target variable.
        \item \textbf{Regression Equation:} Represents the relationship between dependent and independent variables.
    \end{itemize}
    
    \begin{equation}
        Y = \beta_0 + \beta_1 X_1 + \epsilon
    \end{equation}
    Where:
    \begin{itemize}
        \item \( Y \) is the predicted value.
        \item \( \beta_0 \) is the y-intercept.
        \item \( \beta_1 \) is the coefficient for \( X_1 \).
        \item \( \epsilon \) is the error term.
    \end{itemize}
\end{frame}

\begin{frame}[fragile]
    \frametitle{Types of Regression Techniques}
    \begin{itemize}
        \item \textbf{Linear Regression:} Models the relationship with a linear equation.
        \item \textbf{Polynomial Regression:} Extends linear regression using polynomial terms.
        \item \textbf{Logistic Regression:} Used for binary classification, predicting probabilities for categories.
    \end{itemize}
\end{frame}

\begin{frame}[fragile]
    \frametitle{Example of Regression in Practice}
    Consider predicting house prices based on different factors:
    
    \begin{equation}
        \text{Price} = \beta_0 + \beta_1(\text{Size}) + \beta_2(\text{Location}) + \beta_3(\text{Bedrooms}) + \epsilon
    \end{equation}

    Regression techniques are foundational for predictive analytics across various fields, including finance, healthcare, and marketing.
\end{frame}

\begin{frame}[fragile]
    \frametitle{Types of Regression}
    Regression techniques are essential statistical and machine learning methods used to model the relationship between a dependent variable (the outcome) and one or more independent variables (predictors). 
\end{frame}

\begin{frame}[fragile]
    \frametitle{Overview of Regression Techniques}
    \begin{itemize}
        \item **Linear Regression**
        \item **Polynomial Regression**
        \item **Logistic Regression**
    \end{itemize}
\end{frame}

\begin{frame}[fragile]
    \frametitle{1. Linear Regression}
    \begin{block}{Definition}
        Linear regression is a method that models the relationship between a dependent variable and one or more independent variables by fitting a linear equation to observed data.
    \end{block}
    
    \begin{block}{Formula}
        \begin{equation}
            y = mx + b 
        \end{equation}
        Where:
        \begin{itemize}
            \item \( y \) = Dependent variable
            \item \( m \) = Slope of the line
            \item \( x \) = Independent variable
            \item \( b \) = Y-intercept
        \end{itemize}
    \end{block}
    
    \begin{block}{Example}
        Predicting the price of a house (\(y\)) based on its size in square feet (\(x\)).
    \end{block}
\end{frame}

\begin{frame}[fragile]
    \frametitle{2. Polynomial Regression}
    \begin{block}{Definition}
        Polynomial regression extends linear regression by fitting a non-linear relationship using polynomial equations.
    \end{block}
    
    \begin{block}{Formula}
        \begin{equation}
            y = a_0 + a_1x + a_2x^2 + ... + a_nx^n 
        \end{equation}
        Where:
        \begin{itemize}
            \item \( a_0, a_1, \dots, a_n \) = Coefficients to be determined
        \end{itemize}
    \end{block}

    \begin{block}{Example}
        Modeling the trajectory of a projectile, such as a thrown ball.
    \end{block}
    
    \begin{block}{Key Points}
        \begin{itemize}
            \item Can fit curves, but higher-degree polynomials may lead to overfitting.
        \end{itemize}
    \end{block}
\end{frame}

\begin{frame}[fragile]
    \frametitle{3. Logistic Regression}
    \begin{block}{Definition}
        Logistic regression is used when the dependent variable is categorical (e.g., binary). It predicts the probability that a given input point belongs to a certain class.
    \end{block}
    
    \begin{block}{Formula}
        \begin{equation}
            P(y=1|X) = \frac{1}{1 + e^{-(\beta_0 + \beta_1X)}}
        \end{equation}
        Where:
        \begin{itemize}
            \item \( P(y=1|X) \) = Probability that \( y \) is 1 given \( X \)
            \item \( \beta_0, \beta_1 \) = Coefficients 
        \end{itemize}
    \end{block}

    \begin{block}{Example}
        Classifying whether an email is spam (1) or not spam (0) based on features like word frequency.
    \end{block}

    \begin{block}{Key Points}
        \begin{itemize}
            \item Outputs probabilities that can be mapped to binary classes.
            \item Not for continuous outcomes but for classification tasks.
        \end{itemize}
    \end{block}
\end{frame}

\begin{frame}[fragile]
    \frametitle{Summary and Conclusion}
    \begin{itemize}
        \item **Linear Regression:** Simple, linear relationship between variables.
        \item **Polynomial Regression:** For complex, non-linear relationships using polynomial equations.
        \item **Logistic Regression:** For binary classification, predicting probabilities rather than a continuous outcome.
    \end{itemize}
    
    Understanding these regression types equips you with powerful tools for modeling various types of data and relationships, simplifying prediction tasks across multiple disciplines!
\end{frame}

\begin{frame}[fragile]
    \frametitle{Linear Regression - What is it?}
    \begin{itemize}
        \item A statistical technique for modeling relationships between variables.
        \item Dependent variable (Y) and independent variable(s) (X).
        \item Aims to establish a linear equation for prediction.
    \end{itemize}
\end{frame}

\begin{frame}[fragile]
    \frametitle{Linear Regression - Key Concept}
    \begin{block}{Goal of Linear Regression}
        The primary goal is to find the best-fitting straight line that minimizes the errors between predicted and actual values.
    \end{block}
\end{frame}

\begin{frame}[fragile]
    \frametitle{Linear Regression - Formula}
    \begin{block}{Simple Linear Regression Formula}
        \begin{equation}
            Y = \beta_0 + \beta_1 X + \epsilon 
        \end{equation}
        \begin{itemize}
            \item \textbf{Y}: Dependent variable 
            \item \textbf{X}: Independent variable 
            \item \textbf{$\beta_0$}: Intercept 
            \item \textbf{$\beta_1$}: Slope 
            \item \textbf{$\epsilon$}: Error term 
        \end{itemize}
    \end{block}
    
    \begin{block}{Multiple Linear Regression}
        \begin{equation}
            Y = \beta_0 + \beta_1 X_1 + \beta_2 X_2 + ... + \beta_n X_n + \epsilon 
        \end{equation}
    \end{block}
\end{frame}

\begin{frame}[fragile]
    \frametitle{Usage of Linear Regression}
    \begin{enumerate}
        \item \textbf{Predicting Continuous Outcomes:}
            \begin{itemize}
                \item House prices, sales forecasts, academic performance.
            \end{itemize}
        \item \textbf{Quantifying Relationships:}
            \begin{itemize}
                \item Strength of relationship between variables via the slope ($\beta_1$).
            \end{itemize}
    \end{enumerate}
\end{frame}

\begin{frame}[fragile]
    \frametitle{Linear Regression - Example}
    \begin{block}{Predicting Final Exam Score}
        Using the formula:
        \begin{equation}
            Y = 50 + 10X 
        \end{equation}
        \begin{itemize}
            \item If a student studies for 0 hours, predicted score is 50.
            \item For each additional hour studied, score increases by 10 points.
        \end{itemize}
    \end{block}
\end{frame}

\begin{frame}[fragile]
    \frametitle{Key Points and Conclusion}
    \begin{itemize}
        \item \textbf{Simplicity}: Easiest form of regression analysis.
        \item \textbf{Interpretability}: Results are easy to communicate.
        \item \textbf{Assumptions}: Important to understand 
              \begin{itemize}
                  \item Linearity
                  \item Independence
                  \item Homoscedasticity
                  \item Normality of residuals
              \end{itemize}
    \end{itemize}
    \begin{block}{Conclusion}
        A foundational technique pivotal for predictive analytics, leading to further exploration of regression techniques.
    \end{block}
\end{frame}

\begin{frame}[fragile]
    \frametitle{Code Snippet - Linear Regression in Python}
    \begin{lstlisting}[language=Python]
from sklearn.linear_model import LinearRegression
import numpy as np

# Sample data: hours studied vs scores
X = np.array([[1], [2], [3], [4], [5]])  # Independent variable
y = np.array([55, 60, 65, 70, 75])        # Dependent variable

# Create a linear regression model
model = LinearRegression()
model.fit(X, y)

# Predict for 6 study hours
predicted_score = model.predict([[6]])
print(predicted_score)  # Output: Predicted score for 6 hours
    \end{lstlisting}
\end{frame}

\begin{frame}[fragile]
    \frametitle{Assumptions of Linear Regression - Introduction}
    % Slide Introduction
    Linear regression is a powerful statistical tool for predicting the value of a dependent variable based on one or more independent variables. 
    However, its validity hinges on several important assumptions:
    \begin{itemize}
        \item Linearity
        \item Homoscedasticity
        \item Normality of residuals
        \item Independence of residuals
    \end{itemize}
    Understanding these assumptions is crucial for ensuring accurate and reliable results.
\end{frame}

\begin{frame}[fragile]
    \frametitle{Assumptions of Linear Regression - Key Assumptions}
    % Key Assumptions
    \begin{enumerate}
        \item \textbf{Linearity}:
        \begin{itemize}
            \item The relationship between independent and dependent variables should be linear.
            \item Example: Doubling advertising spending should approximately double sales.
            \item \textit{Check}: Use scatter plots of residuals against fitted values.
        \end{itemize}
        
        \item \textbf{Homoscedasticity}:
        \begin{itemize}
            \item Residuals should exhibit constant variance.
            \item Example: Increased spread of residuals indicates heteroscedasticity.
            \item \textit{Check}: Residual plots should show random scatter.
        \end{itemize}
    \end{enumerate}
\end{frame}

\begin{frame}[fragile]
    \frametitle{Assumptions of Linear Regression - Continued}
    % Continued Key Assumptions
    \begin{enumerate}
        \setcounter{enumi}{2} % To continue numbering
        \item \textbf{Normality of Residuals}:
        \begin{itemize}
            \item Residuals should be approximately normally distributed.
            \item \textit{Check}: Use histograms or Q-Q plots of the residuals.
        \end{itemize}

        \item \textbf{Independence of Residuals}:
        \begin{itemize}
            \item Residuals should be independent of each other.
            \item Example: Stock prices from one day affecting prices the next day.
            \item \textit{Check}: Use the Durbin-Watson statistic.
        \end{itemize}
    \end{enumerate}
    
    \textbf{Note}: These assumptions must be checked before interpreting the results of a linear regression model.
\end{frame}

\begin{frame}[fragile]
    \frametitle{Polynomial Regression - Introduction}
    \begin{block}{Introduction to Polynomial Regression}
        Polynomial Regression is used to model relationships where the interaction between independent and dependent variables is non-linear. 
        Unlike linear regression, which fits a straight line, polynomial regression fits a polynomial equation to the data.
    \end{block}
    
    \begin{itemize}
        \item Greater flexibility to capture data curvature
        \item Useful in cases of clear non-linearity
    \end{itemize}
\end{frame}

\begin{frame}[fragile]
    \frametitle{Polynomial Regression - Formula and Key Points}
    \begin{block}{Formula}
        The general form of a polynomial regression equation is:
        \begin{equation}
            y = \beta_0 + \beta_1 x + \beta_2 x^2 + \beta_3 x^3 + \ldots + \beta_n x^n + \epsilon
        \end{equation}
        \begin{itemize}
            \item $y$: Dependent variable
            \item $x$: Independent variable
            \item $\beta_0$: y-intercept
            \item $\beta_1, \beta_2, \ldots, \beta_n$: Polynomial coefficients
            \item $n$: Degree of the polynomial
            \item $\epsilon$: Error term
        \end{itemize}
    \end{block}

    \begin{itemize}
        \item Non-linearity in data
        \item Degree impacts model complexity
        \item Risk of overfitting with high-degree polynomials
    \end{itemize}
\end{frame}

\begin{frame}[fragile]
    \frametitle{Polynomial Regression - Example and Applications}
    \begin{block}{Example}
        Consider the dataset:
        \begin{center}
            \begin{tabular}{|c|c|}
                \hline
                $x$ & $y$ \\
                \hline
                1 & 2 \\
                2 & 3 \\
                3 & 5 \\
                4 & 8 \\
                5 & 15 \\
                \hline
            \end{tabular}
        \end{center}
        A polynomial regression model of degree 2 might look like:
        \begin{equation}
            y = 0.5 + 0.5x + 0.5x^2
        \end{equation}
        This allows for better representation of the increasing values of $y$ as $x$ rises.
    \end{block}

    \begin{block}{Applications}
        \begin{itemize}
            \item Physics: Modeling trajectories
            \item Economics: Analyzing demand-supply curves
            \item Biology: Understanding growth patterns
            \item Machine Learning: Feature preprocessing for non-linear relationships
        \end{itemize}
    \end{block}
\end{frame}

\begin{frame}[fragile]
    \frametitle{Logistic Regression - Overview}
    \begin{itemize}
        \item Logistic regression is a statistical method for binary classification problems.
        \item It predicts the probability of an outcome belonging to a specific category (e.g., success/failure).
        \item Unlike linear regression, logistic regression outputs values between 0 and 1.
    \end{itemize}
\end{frame}

\begin{frame}[fragile]
    \frametitle{Logistic Regression - The Logistic Function}
    \begin{block}{Logistic Function}
        The logistic function is defined as:
        \begin{equation}
        f(z) = \frac{1}{1 + e^{-z}}
        \end{equation}
        where:
        \begin{itemize}
            \item \( z = \beta_0 + \beta_1 x_1 + \beta_2 x_2 + ... + \beta_n x_n \)
            \item \( e \) is approximately equal to 2.71828.
        \end{itemize}
    \end{block}
\end{frame}

\begin{frame}[fragile]
    \frametitle{Logistic Regression - How It Works}
    \begin{enumerate}
        \item \textbf{Model Specification:}
            \begin{equation}
            P(Y=1 | X) = f(Z)
            \end{equation}
            where \( P(Y=1 | X) \) is the probability that the dependent variable \( Y \) equals 1.
        
        \item \textbf{Decision Boundary:}
            \begin{itemize}
                \item A threshold (common is 0.5) is used for classification.
                \item If \( P(Y=1 | X) \geq 0.5 \), predict \( Y = 1 \); otherwise, \( Y = 0 \).
            \end{itemize}
        
        \item \textbf{Loss Function:}
            \begin{itemize}
                \item Maximized using Maximum Likelihood Estimation (MLE).
                \item Binary cross-entropy is often used as the loss function.
            \end{itemize}
    \end{enumerate}
\end{frame}

\begin{frame}[fragile]
    \frametitle{Logistic Regression - Example}
    \begin{itemize}
        \item Example: Predicting disease presence based on features like age, blood pressure, and cholesterol levels.
        \item Model Output:
            \begin{equation}
            P(\text{Disease} = 1 | \text{Age: 50, BP: 120, Chol: 200}) = 0.84
            \end{equation}
        \item Interpretation: 84\% probability that the patient has the disease.
    \end{itemize}
\end{frame}

\begin{frame}[fragile]
    \frametitle{Logistic Regression - Key Points}
    \begin{itemize}
        \item Best suited for binary outcomes.
        \item Output interpretations as probabilities ([0, 1]).
        \item Simple to implement; powerful for various applications:
            \begin{itemize}
                \item Medical diagnosis
                \item Credit scoring
                \item Marketing
            \end{itemize}
    \end{itemize}
\end{frame}

\begin{frame}[fragile]
    \frametitle{Logistic Regression - Python Code Snippet}
    \begin{lstlisting}[language=Python]
from sklearn.model_selection import train_test_split
from sklearn.linear_model import LogisticRegression
from sklearn.metrics import accuracy_score

# Sample dataset
X = [[30, 120, 220], [45, 130, 210], [50, 140, 250]]  # Features
y = [0, 1, 1]  # Labels

# Split the dataset
X_train, X_test, y_train, y_test = train_test_split(X, y, test_size=0.2, random_state=42)

# Create Logistic Regression model
model = LogisticRegression()

# Train the model
model.fit(X_train, y_train)

# Make predictions
predictions = model.predict(X_test)

# Evaluate model
accuracy = accuracy_score(y_test, predictions)
print(f'Accuracy: {accuracy}')
    \end{lstlisting}
\end{frame}

\begin{frame}[fragile]
    \frametitle{Logistic Regression - Conclusion}
    \begin{itemize}
        \item Logistic regression is a foundational tool in data science for binary classification.
        \item Provides interpretable probabilities which are essential across various fields.
    \end{itemize}
\end{frame}

\begin{frame}[fragile]
  \frametitle{Model Evaluation Metrics - Overview}
  \begin{itemize}
    \item Model evaluation metrics are essential to assess regression model performance.
    \item These metrics help understand prediction outcomes based on input features.
    \item Key metrics to explore:
      \begin{itemize}
        \item R-squared
        \item Mean Absolute Error (MAE)
        \item Mean Squared Error (MSE)
      \end{itemize}
  \end{itemize}
\end{frame}

\begin{frame}[fragile]
  \frametitle{Model Evaluation Metrics - R-squared}
  \begin{block}{R-squared (Coefficient of Determination)}
    \begin{itemize}
      \item **Definition**: Measures the proportion of variance explained by independent variables.
      \item **Range**: 0 to 1
      \begin{itemize}
        \item 0: Model explains none of the variability.
        \item 1: Model explains all variability.
      \end{itemize}
      \item **Formula**: 
      \begin{equation}
        R^2 = 1 - \frac{\text{SS}_{\text{res}}}{\text{SS}_{\text{tot}}}
      \end{equation}
      Where:
      \begin{itemize}
        \item $\text{SS}_{\text{res}} = \sum{(y_i - \hat{y}_i)^2}$
        \item $\text{SS}_{\text{tot}} = \sum{(y_i - \bar{y})^2}$
      \end{itemize}
      \item **Example**: An \(R^2 = 0.75\) means 75\% of the variance is explained.
    \end{itemize}
  \end{block}
\end{frame}

\begin{frame}[fragile]
  \frametitle{Model Evaluation Metrics - MAE and MSE}
  \begin{block}{Mean Absolute Error (MAE)}
    \begin{itemize}
      \item **Definition**: Average magnitude of errors in predictions without direction consideration.
      \item **Range**: 0 to ∞
      \begin{itemize}
        \item Lower MAE indicates better model performance.
      \end{itemize}
      \item **Formula**: 
      \begin{equation}
        \text{MAE} = \frac{1}{n} \sum_{i=1}^{n} | y_i - \hat{y}_i |
      \end{equation}
      \item **Example**: MAE of \$5,000 suggests average prediction error is \$5,000.
    \end{itemize}
  \end{block}
  
  \begin{block}{Mean Squared Error (MSE)}
    \begin{itemize}
      \item **Definition**: Average of the squared differences between estimated and actual values.
      \item **Range**: 0 to ∞
      \begin{itemize}
        \item Lower value indicates better predictive performance.
      \end{itemize}
      \item **Formula**: 
      \begin{equation}
        \text{MSE} = \frac{1}{n} \sum_{i=1}^{n} (y_i - \hat{y}_i)^2
      \end{equation}
      \item **Example**: MSE of 100,000 indicates larger errors affect performance significantly.
    \end{itemize}
  \end{block}
\end{frame}

\begin{frame}[fragile]
    \frametitle{Regularization Techniques - Part 1}
    \begin{block}{Introduction to Regularization}
        Regularization techniques are essential in regression to prevent overfitting. Overfitting occurs when a model learns both the underlying relationships and noise in the training data, leading to poor performance on unseen data.
    \end{block}
\end{frame}

\begin{frame}[fragile]
    \frametitle{Regularization Techniques - Part 2}
    \begin{block}{Lasso Regression}
        \begin{itemize}
            \item \textbf{Definition:} Lasso (Least Absolute Shrinkage and Selection Operator) regression adds a penalty equal to the absolute value of the coefficients.
            \item \textbf{Formula:}
            \begin{equation}
                \text{Loss Function} = \text{MSE} + \lambda \sum_{j=1}^{n} |\beta_j|
            \end{equation}
            where \( \lambda \) is the regularization parameter.
            \item \textbf{Key Point:} Lasso is useful in datasets with many features, reducing model complexity by eliminating less relevant features.
        \end{itemize}
    \end{block}
\end{frame}

\begin{frame}[fragile]
    \frametitle{Regularization Techniques - Part 3}
    \begin{block}{Ridge Regression}
        \begin{itemize}
            \item \textbf{Definition:} Ridge regression introduces a penalty equal to the square of the coefficients, shrinking all coefficients towards zero.
            \item \textbf{Formula:}
            \begin{equation}
                \text{Loss Function} = \text{MSE} + \lambda \sum_{j=1}^{n} \beta_j^2
            \end{equation}
            \item \textbf{Key Point:} Ridge regression is beneficial in cases of multicollinearity, stabilizing coefficient estimates and preventing distortion in model predictions.
        \end{itemize}

        \begin{block}{Comparison: Lasso vs. Ridge}
            \begin{itemize}
                \item \textbf{Variable Selection:} Lasso can reduce coefficients to zero; Ridge cannot.
                \item \textbf{Use Cases:} Lasso for few relevant predictors; Ridge for retaining all predictors.
            \end{itemize}
        \end{block}
    \end{block}
\end{frame}

\begin{frame}[fragile]
    \frametitle{Implementation in Python - Overview}
    \begin{block}{Introduction}
        Regression techniques are crucial for modeling relationships between variables. In Python, the Scikit-learn library provides a user-friendly framework for various regression algorithms.
    \end{block}
    
    \begin{block}{Topics Covered}
        \begin{itemize}
            \item Installation of required libraries
            \item Importing necessary modules
            \item Dataset preparation
            \item Model creation and training
            \item Making predictions
            \item Evaluating model performance
        \end{itemize}
    \end{block}
\end{frame}

\begin{frame}[fragile]
    \frametitle{Step-by-Step Implementation - Part 1}
    \begin{block}{Step 1: Install Required Libraries}
        Ensure you have the necessary libraries. Use the following command:
        \begin{lstlisting}[language=bash]
pip install numpy pandas scikit-learn matplotlib
        \end{lstlisting}
    \end{block}
    
    \begin{block}{Step 2: Import Libraries}
        Import the modules needed for regression analysis:
        \begin{lstlisting}[language=python]
import numpy as np
import pandas as pd
import matplotlib.pyplot as plt
from sklearn.model_selection import train_test_split
from sklearn.linear_model import LinearRegression, Lasso, Ridge
from sklearn.metrics import mean_squared_error, r2_score
        \end{lstlisting}
    \end{block}
\end{frame}

\begin{frame}[fragile]
    \frametitle{Step-by-Step Implementation - Part 2}
    \begin{block}{Step 3: Prepare Your Dataset}
        Load your dataset, for example:
        \begin{lstlisting}[language=python]
data = pd.read_csv('data.csv')
X = data[['feature1', 'feature2']]  # Independent variables
y = data['target']                   # Dependent variable
        \end{lstlisting}
    \end{block}
    
    \begin{block}{Step 4: Split the Dataset}
        Split the data into training and testing sets:
        \begin{lstlisting}[language=python]
X_train, X_test, y_train, y_test = train_test_split(X, y, test_size=0.2, random_state=42)
        \end{lstlisting}
    \end{block}
    
    \begin{block}{Step 5: Create and Train the Model}
        Create and fit the models:
        \begin{enumerate}
            \item Linear Regression
            \begin{lstlisting}[language=python]
linear_model = LinearRegression()
linear_model.fit(X_train, y_train)
            \end{lstlisting}
            
            \item Lasso Regression
            \begin{lstlisting}[language=python]
lasso_model = Lasso(alpha=1.0)
lasso_model.fit(X_train, y_train)
            \end{lstlisting}
            
            \item Ridge Regression
            \begin{lstlisting}[language=python]
ridge_model = Ridge(alpha=1.0)
ridge_model.fit(X_train, y_train)
            \end{lstlisting}
        \end{enumerate}
    \end{block}
\end{frame}

\begin{frame}[fragile]
    \frametitle{Use Cases and Real-World Applications - Understanding Regression Techniques}
    \begin{block}{Overview}
        Regression techniques are powerful statistical methods used to model and analyze relationships between variables. They help in predicting a dependent variable based on one or more independent variables, making them invaluable across various domains.
    \end{block}
\end{frame}

\begin{frame}[fragile]
    \frametitle{Use Cases and Real-World Applications - Key Areas of Application}
    \begin{enumerate}
        \item \textbf{Finance}
        \begin{itemize}
            \item Stock Price Prediction: Analysts leverage regression models to forecast stock prices based on historical data, economic indicators, and market trends.
            \item \textbf{Formula:} 
            \begin{equation}
                Y = a + bX
            \end{equation}
            where \(Y\) is the predicted stock price, \(a\) is the y-intercept, \(b\) is the slope, and \(X\) represents historical prices.
        \end{itemize}
        
        \item \textbf{Healthcare}
        \begin{itemize}
            \item Predicting Patient Outcomes: Regression models help in forecasting patient outcomes, such as recovery times or likelihood of complications.
            \item Example: A hospital may use logistic regression to identify high-risk patients and customize follow-up care.
        \end{itemize}

        \item \textbf{Marketing}
        \begin{itemize}
            \item Customer Behavior Analysis: Marketers use regression analysis to understand customer behavior and how various marketing strategies influence sales.
            \item \textbf{Illustration:} 
            \begin{equation}
                Sales = \beta_0 + \beta_1(Ad\ Spend) + \beta_2(Discount) + \beta_3(Promotions) + \epsilon
            \end{equation}
            where \(\beta\) values represent the contribution of each factor to sales.
        \end{itemize}
    \end{enumerate}
\end{frame}

\begin{frame}[fragile]
    \frametitle{Use Cases and Real-World Applications - Summary and Engagement}
    \begin{block}{Summary of Key Points}
        \begin{itemize}
            \item Regression techniques are essential for predictive analytics across various fields.
            \item They aid in decision-making by quantifying relationships between variables.
            \item Common applications include financial forecasting, health outcome prediction, and analyzing marketing strategies.
        \end{itemize}
    \end{block}
    
    \begin{block}{Engagement}
        Consider how regression could address challenges in other sectors such as agriculture (predicting crop yields) or real estate (estimating property values). Think about a problem in your field of interest that could be solved with regression analysis!
    \end{block}
\end{frame}

\begin{frame}
    \frametitle{Hands-On Project}
    \begin{block}{Overview of the Project}
        In this hands-on project, you will apply a regression technique to analyze a real-world dataset. 
        Regression analysis is a powerful tool to model relationships between a dependent variable and one or more independent variables. 
        This project will enable you to solidify your understanding of regression concepts and enhance your coding skills in a practical scenario.
    \end{block}
\end{frame}

\begin{frame}
    \frametitle{Project Steps}
    \begin{enumerate}
        \item \textbf{Select a Dataset}
            \begin{itemize}
                \item \textbf{Sources:} Kaggle, UCI Machine Learning Repository, Government databases
                \item \textbf{Criteria:} Choose datasets with clear dependent and independent variables
            \end{itemize}

        \item \textbf{Understanding the Data}
            \begin{itemize}
                \item Data Exploration with descriptive statistics
                \item Visualization using scatter plots
            \end{itemize}
    \end{enumerate}
\end{frame}

\begin{frame}[fragile]
    \frametitle{Data Exploration Examples}
    \begin{block}{Data Exploration with Python}
        Use the following code to load and describe your dataset:
        \begin{lstlisting}[language=Python]
import pandas as pd

data = pd.read_csv('your_dataset.csv')
print(data.describe())
        \end{lstlisting}
    \end{block}
    
    \begin{block}{Visualize Data}
        Utilize the following code to create scatter plots:
        \begin{lstlisting}[language=Python]
import matplotlib.pyplot as plt

plt.scatter(data['SquareFootage'], data['Price'])
plt.xlabel('Square Footage')
plt.ylabel('Price')
plt.title('Relationship between Square Footage and Price')
plt.show()
        \end{lstlisting}
    \end{block}
\end{frame}

\begin{frame}
    \frametitle{Continuing the Project Steps}
    \begin{enumerate}[resume]
        \item \textbf{Preprocessing the Data}
            \begin{itemize}
                \item Cleaning: Handle missing values or outliers
                \item Normalization: Scale features if necessary
            \end{itemize}

        \item \textbf{Implementing Regression}
            \begin{itemize}
                \item Choose a Regression Model: Linear, Polynomial, Ridge, or Lasso Regression
            \end{itemize}
    \end{enumerate}
\end{frame}

\begin{frame}[fragile]
    \frametitle{Regression Implementation Example}
    \begin{block}{Linear Regression Example}
        Use the following code to implement Linear Regression:
        \begin{lstlisting}[language=Python]
from sklearn.model_selection import train_test_split
from sklearn.linear_model import LinearRegression

# Features and target
X = data[['SquareFootage', 'Bedrooms']]
y = data['Price']

# Splitting data
X_train, X_test, y_train, y_test = train_test_split(X, y, test_size=0.2, random_state=42)

# Model
model = LinearRegression()
model.fit(X_train, y_train)
predictions = model.predict(X_test)
        \end{lstlisting}
    \end{block}
\end{frame}

\begin{frame}
    \frametitle{Final Project Steps}
    \begin{enumerate}[resume]
        \item \textbf{Evaluating Model Performance}
            \begin{itemize}
                \item Metrics: Mean Absolute Error (MAE), Mean Squared Error (MSE), R-squared
            \end{itemize}

        \item \textbf{Interpret Results}
            \begin{itemize}
                \item Analyze coefficients for variable influence
                \item Visualize predictions versus actual values
            \end{itemize}
    \end{enumerate}
\end{frame}

\begin{frame}
    \frametitle{Conclusion and Key Points}
    \begin{block}{Key Points to Emphasize}
        \begin{itemize}
            \item Regression techniques provide insights into data relationships.
            \item Importance of data preprocessing and model evaluation.
            \item Validate models with different data splits for generalizability.
        \end{itemize}
    \end{block}

    \begin{block}{Conclusion}
        This project applies theoretical knowledge to real-world scenarios, reinforcing your learning and providing practical coding experience. Document your process and findings for future reference. Good luck!
    \end{block}
\end{frame}

\begin{frame}[fragile]
    \frametitle{Summary and Conclusion - Key Points Recap}
    \begin{enumerate}
        \item \textbf{Definition of Regression}:
        \begin{itemize}
            \item Regression is a statistical method to model and analyze numerical relationships.
            \item It predicts a continuous outcome based on one or more predictor variables.
        \end{itemize}

        \item \textbf{Types of Regression Techniques}:
        \begin{itemize}
            \item \textbf{Linear Regression}: Models the relationship using a straight line.
            \begin{itemize}
                \item Formula: \( y = mx + b \)
                \item Example: Predicting house prices based on size.
            \end{itemize}
            \item \textbf{Multiple Regression}: Involves multiple predictor variables.
            \item \textbf{Polynomial Regression}: Models nonlinear relationships.
            \item \textbf{Ridge and Lasso Regression}: Techniques to prevent overfitting.
            \item \textbf{Logistic Regression}: Used for binary outcomes.
        \end{itemize}
    \end{enumerate}
\end{frame}

\begin{frame}[fragile]
    \frametitle{Summary and Conclusion - Importance of Regression}
    \begin{enumerate}
        \setcounter{enumi}{3}
        \item \textbf{Importance of Regression in Machine Learning}:
        \begin{itemize}
            \item \textbf{Predictive Power}: Enables accurate predictions in various fields.
            \item \textbf{Interpretability}: Provides insights into how predictors influence outcomes.
            \item \textbf{Foundation for Advanced Techniques}: Fundamental concepts support complex models.
        \end{itemize}

        \item \textbf{Evaluation of Regression Models}:
        \begin{itemize}
            \item \textbf{Performance Metrics}:
            \begin{itemize}
                \item Mean Squared Error (MSE): Measures average squared difference.
                \item R-squared: Proportion of variance explained.
                \end{itemize}
            \item Example MSE Calculation:
            \begin{equation}
                MSE = \frac{1}{n} \sum (y_i - \hat{y}_i)^2
            \end{equation}
        \end{itemize}
    \end{enumerate}
\end{frame}

\begin{frame}[fragile]
    \frametitle{Summary and Conclusion - Final Thoughts}
    \begin{block}{Conclusion}
        \begin{itemize}
            \item \textbf{Integration with Hands-On Projects}: Applying regression techniques in real-world datasets reinforces understanding.
            \item \textbf{Ongoing Relevance}: Mastery of these techniques is crucial across various machine learning applications.
        \end{itemize}
    \end{block}

    \begin{block}{Encouragement}
        Utilize this summary to reinforce your understanding and prepare for hands-on application of these concepts!
    \end{block}
\end{frame}


\end{document}