\documentclass[aspectratio=169]{beamer}

% Theme and Color Setup
\usetheme{Madrid}
\usecolortheme{whale}
\useinnertheme{rectangles}
\useoutertheme{miniframes}

% Additional Packages
\usepackage[utf8]{inputenc}
\usepackage[T1]{fontenc}
\usepackage{graphicx}
\usepackage{booktabs}
\usepackage{listings}
\usepackage{amsmath}
\usepackage{amssymb}
\usepackage{xcolor}
\usepackage{tikz}
\usepackage{pgfplots}
\pgfplotsset{compat=1.18}
\usetikzlibrary{positioning}
\usepackage{hyperref}

% Custom Colors
\definecolor{myblue}{RGB}{31, 73, 125}
\definecolor{mygray}{RGB}{100, 100, 100}
\definecolor{mygreen}{RGB}{0, 128, 0}
\definecolor{myorange}{RGB}{230, 126, 34}
\definecolor{mycodebackground}{RGB}{245, 245, 245}

% Set Theme Colors
\setbeamercolor{structure}{fg=myblue}
\setbeamercolor{frametitle}{fg=white, bg=myblue}
\setbeamercolor{title}{fg=myblue}
\setbeamercolor{section in toc}{fg=myblue}
\setbeamercolor{item projected}{fg=white, bg=myblue}
\setbeamercolor{block title}{bg=myblue!20, fg=myblue}
\setbeamercolor{block body}{bg=myblue!10}
\setbeamercolor{alerted text}{fg=myorange}

% Set Fonts
\setbeamerfont{title}{size=\Large, series=\bfseries}
\setbeamerfont{frametitle}{size=\large, series=\bfseries}
\setbeamerfont{caption}{size=\small}
\setbeamerfont{footnote}{size=\tiny}

% Document Start
\begin{document}

\frame{\titlepage}

\begin{frame}[fragile]
    \frametitle{Introduction to Storage Solutions for Big Data}
    \begin{block}{Overview}
        This slide provides an overview of the importance of data storage solutions in managing big data, highlighting the challenges posed by large datasets.
    \end{block}
\end{frame}

\begin{frame}[fragile]
    \frametitle{Importance of Data Storage Solutions}
    \begin{itemize}
        \item \textbf{Big Data Definition}: Refers to extremely large datasets that can be structured, semi-structured, or unstructured.
        \item \textbf{Data-Driven Decisions}: As organizations rely on data-driven decision-making, effective storage solutions become essential.
        \item \textbf{Key Characteristics}:
            \begin{itemize}
                \item \textbf{Scale}: Traditional systems struggle with vast data volumes.
                \item \textbf{Variety}: Must handle both structured (e.g., SQL) and unstructured data (e.g., documents, multimedia).
                \item \textbf{Velocity}: Real-time storage solutions are necessary for fast data generation and processing.
            \end{itemize}
    \end{itemize}
\end{frame}

\begin{frame}[fragile]
    \frametitle{Challenges of Big Data Storage}
    \begin{enumerate}
        \item \textbf{Data Volume}: Must achieve horizontal scalability to handle increasing data sizes.
            \begin{block}{Example}
                Distributed storage systems like Hadoop Distributed File System (HDFS) allow data to span multiple machines.
            \end{block}
        \item \textbf{Data Variety}: Flexible mechanisms are essential to accommodate various data types beyond traditional databases.
        \item \textbf{Data Integrity and Consistency}: Implementing robust techniques for accuracy across distributed systems is crucial.
            \begin{block}{Key Technique}
                Utilize strong consistency models to ensure data synchronization across nodes.
            \end{block}
        \item \textbf{Cost}: Consider cost-effective options for high storage needs, balancing performance and budget.
            \begin{block}{Strategy}
                Cloud storage options like Amazon S3 or Google Cloud Storage offer scalable solutions with lower upfront costs.
            \end{block}
    \end{enumerate}
\end{frame}

\begin{frame}[fragile]
    \frametitle{Conclusion and Key Points}
    \begin{itemize}
        \item \textbf{Evolving Storage Needs}: Understanding the relationship between data types and storage solutions is essential.
        \item \textbf{Flexibility is Vital}: Storage solutions should adapt to evolving data management requirements.
        \item \textbf{Future Trends}: The rise of edge computing and AI-driven storage will both complicate and enhance big data opportunities.
    \end{itemize}
    
    \begin{block}{Conclusion}
        Effective management of big data storage is crucial for actionable insights. Adaptable storage solutions are necessary to address the challenges presented by large datasets.
    \end{block}
\end{frame}

\begin{frame}[fragile]
    \frametitle{Next Steps}
    \begin{block}{Upcoming Discussion}
        In the next slide, we will delve deeper into SQL databases, exploring their structure and typical use cases in the big data landscape.
    \end{block}
\end{frame}

\begin{frame}[fragile]
    \frametitle{Understanding SQL Databases - Introduction}
    \begin{block}{What is an SQL Database?}
    SQL (Structured Query Language) databases are relational databases designed to manage structured data. They utilize a tabular format for storing data and provide a systematic way to create, retrieve, update, and delete data.
    \end{block}
\end{frame}

\begin{frame}[fragile]
    \frametitle{Understanding SQL Databases - Structure}
    \begin{block}{Structure of SQL Databases}
    \begin{enumerate}
        \item \textbf{Tables}: The primary units of storage, organized into rows and columns. Each table represents an entity (e.g., Users, Products).
            \begin{itemize}
                \item \textbf{Rows}: Individual records (also called tuples) representing a single item.
                \item \textbf{Columns}: Attributes of the entity (e.g., UserID, UserName, Email).
            \end{itemize}
        \item \textbf{Schema}: The blueprint that defines the structure of the database, including tables, fields, data types, and relationships.
        \item \textbf{Relationships}: SQL databases support relationships such as:
            \begin{itemize}
                \item One-to-One
                \item One-to-Many
                \item Many-to-Many
            \end{itemize}
    \end{enumerate}
    \end{block}
\end{frame}

\begin{frame}[fragile]
    \frametitle{Understanding SQL Databases - Key Features}
    \begin{block}{Key Features of SQL Databases}
        \begin{itemize}
            \item \textbf{ACID Compliance}: Ensures reliable transactions.
            \item \textbf{Data Integrity}: Accuracy and reliability through constraints (e.g., primary keys).
            \item \textbf{Complex Queries}: Supports joins, aggregations, and subqueries.
            \item \textbf{Standardization}: SQL is a standardized language across different implementations.
        \end{itemize}
    \end{block}
\end{frame}

\begin{frame}[fragile]
    \frametitle{Understanding SQL Databases - Use Cases}
    \begin{block}{Typical Use Cases in Big Data Scenarios}
        \begin{enumerate}
            \item \textbf{Transaction Processing}: Handling large volumes of transactions.
            \item \textbf{Enterprise Data Warehousing}: Aggregating data for insights.
            \item \textbf{Reporting and Analytics}: Generating reports for decision-making.
        \end{enumerate}
    \end{block}
\end{frame}

\begin{frame}[fragile]
    \frametitle{Understanding SQL Databases - Example Query}
    \begin{block}{Example SQL Query}
    \begin{lstlisting}[language=SQL]
SELECT UserName, Email 
FROM Users 
WHERE SignUpDate > '2022-01-01';
    \end{lstlisting}
    This query retrieves names and email addresses of users who signed up after January 1, 2022.
    \end{block}
\end{frame}

\begin{frame}[fragile]
    \frametitle{Understanding SQL Databases - Key Takeaways}
    \begin{block}{Key Points to Emphasize}
        \begin{itemize}
            \item SQL databases are ideal for structured data analytics.
            \item Understanding schema and relationships is crucial for data integrity.
            \item SQL's robust querying capability supports insights into big data applications.
        \end{itemize}
    \end{block}
\end{frame}

\begin{frame}[fragile]
    \frametitle{Understanding NoSQL Databases - Overview}
    \begin{itemize}
        \item NoSQL (Not Only SQL) databases manage large volumes of diverse and unstructured data.
        \item Offer more flexibility than traditional SQL databases which are table-structured.
        \item Designed for horizontal scalability to handle big data applications.
    \end{itemize}
\end{frame}

\begin{frame}[fragile]
    \frametitle{Types of NoSQL Databases}
    \begin{enumerate}
        \item \textbf{Document Stores}
            \begin{itemize}
                \item \textbf{Description}: Flexible schemas using documents (JSON, XML).
                \item \textbf{Example}: MongoDB.
                \item \textbf{When to Use}: Content management systems, evolving data structures.
            \end{itemize}
        
        \item \textbf{Key-Value Stores}
            \begin{itemize}
                \item \textbf{Description}: Data stored in key-value pairs.
                \item \textbf{Example}: Redis, Amazon DynamoDB.
                \item \textbf{When to Use}: Rapid lookups, caching, user session management.
            \end{itemize}

        \item \textbf{Column-Family Stores}
            \begin{itemize}
                \item \textbf{Description}: Stores data in columns, suited for large databases.
                \item \textbf{Example}: Apache Cassandra, HBase.
                \item \textbf{When to Use}: Write-heavy applications, real-time data processing.
            \end{itemize}

        \item \textbf{Graph Databases}
            \begin{itemize}
                \item \textbf{Description}: Handle data with complex relationships using graph structures.
                \item \textbf{Example}: Neo4j, Amazon Neptune.
                \item \textbf{When to Use}: Analyzing interconnected data, social networks.
            \end{itemize}
    \end{enumerate}
\end{frame}

\begin{frame}[fragile]
    \frametitle{Key Points and Conclusion}
    \begin{block}{Key Points to Emphasize}
        \begin{itemize}
            \item \textbf{Scalability}: Designed for distributed data across multiple servers.
            \item \textbf{Flexibility}: Supports various data structures without predefined schemas.
            \item \textbf{Performance}: Optimized for high-throughput and low-latency operations.
        \end{itemize}
    \end{block}

    \begin{block}{Conclusion}
        NoSQL databases offer solutions for managing diverse and quickly changing data. Choosing the right type based on application needs is essential for effective big data storage.
    \end{block}
\end{frame}

\begin{frame}[fragile]
    \frametitle{Comparing SQL vs. NoSQL Databases}
    \begin{block}{Overview}
        In the realm of data storage and management, SQL (Structured Query Language) and NoSQL (Not Only SQL) databases serve different purposes and possess unique characteristics. This presentation will provide a comparison emphasizing scalability and flexibility.
    \end{block}
\end{frame}

\begin{frame}[fragile]
    \frametitle{SQL Databases}
    \begin{itemize}
        \item \textbf{Definition}: Relational databases using structured query language with a fixed schema.
        \item \textbf{Examples}: MySQL, PostgreSQL, Oracle, Microsoft SQL Server.
    \end{itemize}

    \begin{block}{Advantages}
        \begin{enumerate}
            \item Structured data with relationships - Ideal for complex queries and transactions.
            \item ACID compliance - Ensures reliable transaction processing.
            \item Mature technologies - Established ecosystem with extensive support.
        \end{enumerate}
    \end{block}

    \begin{block}{Disadvantages}
        \begin{enumerate}
            \item Scalability constraints - Vertical scaling can be expensive and limited.
            \item Rigid schema - Changes can lead to downtime and complexity.
            \item Performance impact - Degradation with large write volumes.
        \end{enumerate}
    \end{block}
\end{frame}

\begin{frame}[fragile]
    \frametitle{NoSQL Databases}
    \begin{itemize}
        \item \textbf{Definition}: Non-relational databases that accommodate various data structures with a dynamic schema.
        \item \textbf{Examples}: MongoDB, Cassandra, Redis, Neo4j.
    \end{itemize}

    \begin{block}{Advantages}
        \begin{enumerate}
            \item High scalability - Designed for horizontal scaling across servers.
            \item Flexible data models - Supports unstructured and semi-structured data.
            \item High write performance - Optimized for big data applications.
        \end{enumerate}
    \end{block}

    \begin{block}{Disadvantages}
        \begin{enumerate}
            \item Eventual consistency - May not guarantee immediate consistency.
            \item Limited query capabilities - Some lack complex querying.
            \item Learning curve - Variety may confuse traditional SQL users.
        \end{enumerate}
    \end{block}
\end{frame}

\begin{frame}[fragile]
    \frametitle{Key Comparisons}
    \begin{table}[h]
        \centering
        \begin{tabular}{|l|l|l|}
            \hline
            Feature                   & SQL Databases                & NoSQL Databases            \\ \hline
            Data Model                & Structured                   & Unstructured/Semi-structured \\ \hline
            Schema                    & Fixed                        & Dynamic                   \\ \hline
            Query Language            & SQL                          & Varies (e.g., JSON, CQL) \\ \hline
            Scalability               & Vertical                     & Horizontal                \\ \hline
            Transactions              & ACID compliant              & BASE compliant (mostly)   \\ \hline
            Performance               & Moderate with high volume    & High with large datasets   \\ \hline
        \end{tabular}
    \end{table}
\end{frame}

\begin{frame}[fragile]
    \frametitle{Conclusion and Example Scenarios}
    Understanding the differences between SQL and NoSQL allows for better database solution choices based on project specifics. 

    \begin{block}{Example Scenarios}
        \begin{itemize}
            \item \textbf{SQL Use Case}: E-commerce platforms with complex transactions (e.g., inventory management).
            \item \textbf{NoSQL Use Case}: Social media apps managing diverse user-generated content quickly.
        \end{itemize}
    \end{block}

    \begin{block}{Summary}
        When choosing between SQL and NoSQL, consider the nature of the data, scalability needs, and required consistency levels. Each serves distinct purposes in modern applications.
    \end{block}
\end{frame}

\begin{frame}
    \frametitle{Case Studies of Storage Solutions}
    \begin{block}{Introduction}
        Organizations today face an ever-increasing amount of data, requiring effective storage solutions. SQL and NoSQL databases offer distinct approaches to handling large datasets, each with unique challenges and successes.
    \end{block}
\end{frame}

\begin{frame}[fragile]
    \frametitle{Case Study 1: SQL Database - Netflix}
    \begin{itemize}
        \item \textbf{Overview}
            \begin{itemize}
                \item \textbf{Challenge}: Performance issues managing large library and user data during peak times.
                \item \textbf{Database Used}: PostgreSQL.
            \end{itemize}
        \item \textbf{Solution}
            \begin{itemize}
                \item Transitioned to a distributed SQL architecture for improved performance.
            \end{itemize}
        \item \textbf{Successes}
            \begin{itemize}
                \item Improved scalability for millions of concurrent users.
                \item Enhanced performance during peak loads.
                \item Improved data replication for high availability.
            \end{itemize}
        \item \textbf{Key Takeaway}: SQL databases can effectively manage relational data at scale when optimized for specific use cases.
    \end{itemize}
\end{frame}

\begin{frame}[fragile]
    \frametitle{Case Study 2: NoSQL Database - Amazon}
    \begin{itemize}
        \item \textbf{Overview}
            \begin{itemize}
                \item \textbf{Challenge}: Amazon's platform faced issues with large volumes of transactional data.
                \item \textbf{Database Used}: Amazon DynamoDB.
            \end{itemize}
        \item \textbf{Solution}
            \begin{itemize}
                \item Adopted a NoSQL solution for flexible data types and structures.
            \end{itemize}
        \item \textbf{Successes}
            \begin{itemize}
                \item Flexibility in integrating various data types.
                \item Achieved high throughput and low latency.
                \item Automatic scaling for unpredictable workloads.
            \end{itemize}
        \item \textbf{Key Takeaway}: NoSQL databases, like DynamoDB, excel in environments requiring high flexibility and scalability for unstructured data.
    \end{itemize}
\end{frame}

\begin{frame}
    \frametitle{Comparison Summary and Conclusion}
    \begin{itemize}
        \item \textbf{Comparison Summary}
            \begin{itemize}
                \item \textbf{SQL}: Best for structured transactional data and complex queries.
                \item \textbf{NoSQL}: Ideal for unstructured or semi-structured data with quick scalability.
            \end{itemize}
        \item \textbf{Conclusion}
            The choice between SQL and NoSQL depends on specific organizational needs, data types, and performance requirements. Real-world case studies can guide organizations in selecting the right database technology.
    \end{itemize}
\end{frame}

\begin{frame}[fragile]
    \frametitle{Example Code Snippets}
    \begin{block}{SQL Example (PostgreSQL)}
        \begin{lstlisting}[language=sql]
CREATE TABLE users (
    id SERIAL PRIMARY KEY,
    name VARCHAR(100),
    email VARCHAR(100) UNIQUE,
    created_at TIMESTAMP DEFAULT CURRENT_TIMESTAMP
);
        \end{lstlisting}
    \end{block}

    \begin{block}{NoSQL Example (DynamoDB)}
        \begin{lstlisting}[language=JavaScript]
const params = {
    TableName: "Products",
    Item: {
        "ProductID": "12345",
        "ProductName": "Example Item",
        "Price": 29.99,
        "Reviews": []
    }
};
dynamoDB.put(params, function(err, data) {
    if (err) console.log(err);
    else console.log("Success:", data);
});
        \end{lstlisting}
    \end{block}
\end{frame}

\begin{frame}[fragile]
    \frametitle{Choosing the Right Storage Solution}
    \begin{block}{Introduction to Storage Solutions}
        When selecting a storage solution for Big Data, determining whether to use SQL (relational) or NoSQL (non-relational) databases is crucial. This choice impacts the performance, scalability, and flexibility of data management based on the unique requirements of your project.
    \end{block}
\end{frame}

\begin{frame}[fragile]
    \frametitle{Key Criteria for Selection}
    \begin{enumerate}
        \item \textbf{Data Volume}
            \begin{itemize}
                \item \textbf{Definition:} The amount of data processed (TB to PB).
                \item \textbf{SQL Suitability:} Best for structured data, maintaining ACID properties.
                    \begin{itemize}
                        \item \textit{Example:} Traditional retail databases (e.g., Oracle, MySQL).
                    \end{itemize}
                \item \textbf{NoSQL Suitability:} Excels with large volumes, focusing on horizontal scalability.
                    \begin{itemize}
                        \item \textit{Example:} Social media platforms (e.g., Cassandra, MongoDB).
                    \end{itemize}
            \end{itemize}
            
        \item \textbf{Data Velocity}
            \begin{itemize}
                \item \textbf{Definition:} The speed of data generation and processing.
                \item \textbf{SQL Suitability:} Suitable for lower frequency updates.
                    \begin{itemize}
                        \item \textit{Example:} Financial systems updating daily transactions.
                    \end{itemize}
                \item \textbf{NoSQL Suitability:} Ideal for high-speed data ingestion and real-time processing.
                    \begin{itemize}
                        \item \textit{Example:} IoT applications collecting real-time sensor data.
                    \end{itemize}
            \end{itemize}
          
        \item \textbf{Data Variety}
            \begin{itemize}
                \item \textbf{Definition:} Different formats/types of data (structured, semi-structured, unstructured).
                \item \textbf{SQL Suitability:} Requires predefined schema, less flexible.
                    \begin{itemize}
                        \item \textit{Example:} Traditional ERP systems.
                    \end{itemize}
                \item \textbf{NoSQL Suitability:} Accommodates diverse formats (JSON, XML), perfect for unpredictable types.
                    \begin{itemize}
                        \item \textit{Example:} E-commerce websites storing mixed data types.
                    \end{itemize}
            \end{itemize}
    \end{enumerate}
\end{frame}

\begin{frame}[fragile]
    \frametitle{Summary and Conclusion}
    \begin{block}{Summary of Key Points}
        \begin{itemize}
            \item \textbf{SQL Databases:} Best for structured data, stringent consistency, lower data volume, and velocity.
            \item \textbf{NoSQL Databases:} Ideal for high volume and velocity, various data formats, and scalability.
        \end{itemize}
    \end{block}
    
    \begin{block}{Conclusion}
        Choosing the right storage solution is a balance of understanding project requirements against the strengths and weaknesses of SQL and NoSQL databases. A thorough analysis of data volume, velocity, and variety will lead you to an informed decision that aligns with your organizational needs.
    \end{block}
    
    \begin{block}{Illustration for Understanding}
        \begin{itemize}
            \item Data Requirements:
                \begin{itemize}
                    \item Volume (TB/PB)
                    \item Velocity
                    \item Variety
                \end{itemize}
        \end{itemize}
    \end{block}
\end{frame}

\begin{frame}[fragile]
    \frametitle{Integrating Storage Solutions in Data Architecture}
    \begin{block}{Understanding Storage Solutions}
        In modern data architectures, various storage solutions cater to the diverse needs of big data applications. The main categories of storage solutions include:
    \end{block}
    \begin{itemize}
        \item \textbf{Data Lakes}: Store raw data, structured or unstructured, without strict schema definitions. Cost-effective for vast amounts of data.
        \item \textbf{Data Warehouses}: Store processed data, optimized for query and analysis. Utilize predefined schemas suitable for analytical processing.
    \end{itemize}
\end{frame}

\begin{frame}[fragile]
    \frametitle{Integration of Storage Solutions}
    \begin{block}{How Different Storage Solutions Integrate into Data Architecture}
        \begin{enumerate}
            \item \textbf{Complementary Roles}:
                \begin{itemize}
                    \item Data lakes and data warehouses serve different but complementary roles. 
                    \item \textit{Example}: Raw customer data in a data lake for exploration versus structured reports in a data warehouse for business intelligence.
                \end{itemize}
            \item \textbf{ETL vs. ELT}:
                \begin{itemize}
                    \item \textbf{ETL}: Transform data before loading into warehouses.
                    \item \textbf{ELT}: Load raw data first and transform later, primarily with data lakes.
                    \item \textit{Key Point}: Choice depends on use case and data nature.
                \end{itemize}
            \item \textbf{Hybrid Storage Architecture}:
                \begin{itemize}
                    \item Organizations often combine both solutions. 
                    \item \textit{Example}: Retail company uses a data lake for IoT sensor data and a warehouse for structured sales reporting.
                \end{itemize}
        \end{enumerate}
    \end{block}
\end{frame}

\begin{frame}[fragile]
    \frametitle{Key Points and Conclusion}
    \begin{block}{Key Points to Emphasize}
        \begin{itemize}
            \item \textbf{Scalability}: Data lakes offer scalability for diverse data types.
            \item \textbf{Cost Efficiency}: Data lakes are generally cheaper for storing raw data.
            \item \textbf{Real-Time Analysis}: Suitable for real-time data analytics use cases.
            \item \textbf{Data Governance}: Requires robust governance frameworks to manage data quality.
        \end{itemize}
    \end{block}
    \begin{block}{Conclusion}
        Integrating various storage solutions is crucial for developing effective data architectures, aligning with organizational goals and ensuring optimal data accessibility and usability.
    \end{block}
\end{frame}

\begin{frame}
    \frametitle{Performance Considerations}
    \begin{block}{Overview}
        Performance metrics are crucial when selecting and configuring storage solutions for big data.
    \end{block}
    \begin{itemize}
        \item Read/Write Speed
        \item Latency
        \item Data Retrieval Times
    \end{itemize}
\end{frame}

\begin{frame}
    \frametitle{Key Performance Metrics}
    \begin{enumerate}
        \item \textbf{Read/Write Speed}
            \begin{itemize}
                \item \textit{Definition:} Rate of data read from or written to storage (MB/s or IOPS).
                \item \textit{Importance:} Enhances data processing times, essential for real-time analytics.
                \item \textit{Example:} 500 MB/s write speed fully utilizes SSD bandwidth.
            \end{itemize}

        \item \textbf{Latency}
            \begin{itemize}
                \item \textit{Definition:} Delay before data transfer begins (measured in ms).
                \item \textit{Importance:} Low latency is crucial for immediate data access.
                \item \textit{Example:} 5 ms latency suitable for OLTP systems.
            \end{itemize}
    \end{enumerate}
\end{frame}

\begin{frame}
    \frametitle{More Key Performance Metrics}
    \begin{enumerate}[resume]
        \item \textbf{Data Retrieval Times}
            \begin{itemize}
                \item \textit{Definition:} Time taken to locate and fetch requested data.
                \item \textit{Importance:} Faster data retrieval improves user experience and systemic efficiency.
                \item \textit{Example:} 2 seconds retrieval vs. 200 milliseconds in big data applications.
            \end{itemize}
    \end{enumerate}

    \begin{block}{Summary of Key Points}
        \begin{itemize}
            \item Read/Write Speed: Efficiency of data access.
            \item Latency: Responsiveness of storage systems.
            \item Data Retrieval Times: Speed of data fetching.
        \end{itemize}
    \end{block}
\end{frame}

\begin{frame}[fragile]
    \frametitle{Code Snippet - Measuring Latency}
    The following Python code measures latency for a simple read process:
    \begin{lstlisting}[language=Python]
import time
import random

def read_data_from_storage():
    # Simulate reading data with a random delay (latency)
    latency = random.uniform(0.01, 0.1)  # Latency between 10ms to 100ms
    time.sleep(latency)
    return "Data Retrieved"

start_time = time.time()
data = read_data_from_storage()
end_time = time.time()

print(f"Data: {data}, Latency: {end_time - start_time:.3f} seconds")
    \end{lstlisting}

    This example simulates the read process and measures the time taken, illustrating practical latency in data operations.
\end{frame}

\begin{frame}
    \frametitle{Conclusion}
    \begin{block}{Conclusion}
        Consider how performance metrics align with application needs. Optimizing storage systems can improve speed, efficiency, and overall data architecture effectiveness.
    \end{block}
\end{frame}

\begin{frame}[fragile]
    \frametitle{Ethical Implications of Data Storage - Introduction}
    \begin{itemize}
        \item Data storage is largely influenced by ethical considerations.
        \item Key themes include:
        \begin{itemize}
            \item Compliance with laws and regulations
            \item Data privacy
            \item Moral obligations of organizations
        \end{itemize}
    \end{itemize}
\end{frame}

\begin{frame}[fragile]
    \frametitle{Key Ethical Considerations in Data Storage}
    \begin{block}{Data Privacy}
        \begin{itemize}
            \item \textbf{Definition}: The right of individuals to control their personal data.
            \item \textbf{Importance}: Fosters trust between organizations and customers.
            \item \textbf{Example}: Implementing encryption techniques to protect personal identifiers in databases.
        \end{itemize}
    \end{block}
    
    \begin{block}{Compliance with Regulations}
        \begin{itemize}
            \item \textbf{GDPR}: Strict guidelines in the EU for personal data processing.
                \begin{itemize}
                    \item Key Principles: Lawfulness, fairness, transparency, purpose limitation, and data minimization.
                \end{itemize}
            \item \textbf{HIPAA}: Protects sensitive patient information in healthcare.
            \item \textbf{Key Point}: Non-compliance can result in fines and legal repercussions.
        \end{itemize}
    \end{block}
\end{frame}

\begin{frame}[fragile]
    \frametitle{Security Concerns and Best Practices}
    \begin{block}{Security Concerns}
        \begin{itemize}
            \item \textbf{Data Breaches}
                \begin{itemize}
                    \item Unauthorized access to sensitive data.
                    \item Risks: Identity theft and loss of consumer trust.
                    \item Example: Equifax breach (2017) impacting 147 million individuals.
                \end{itemize}
            \item \textbf{Data Loss}
                \begin{itemize}
                    \item Accidental loss due to hardware failure or deletion.
                    \item Prevention: Data redundancy strategies (e.g., RAID) and regular backups.
                \end{itemize}
        \end{itemize}
    \end{block}
    
    \begin{block}{Best Practices for Ethical Data Storage}
        \begin{itemize}
            \item Implement strong access controls.
            \item Conduct regular audits and compliance checks.
            \item Educate employees on data ethics through training sessions.
        \end{itemize}
    \end{block}
\end{frame}

\begin{frame}[fragile]
    \frametitle{Conclusion and Key Takeaways}
    \begin{itemize}
        \item Ethical implications in data storage are complex, focusing on:
            \begin{itemize}
                \item Data privacy
                \item Compliance
                \item Security concerns
            \end{itemize}
        \item Organizations should develop policies that respect individual rights and adhere to legal frameworks.
    \end{itemize}
    
    \begin{block}{Key Takeaways}
        \begin{itemize}
            \item Data privacy and compliance are crucial for trust in data storage.
            \item Ethical practices must extend beyond compliance, fostering a responsible data environment.
        \end{itemize}
    \end{block}
\end{frame}

\begin{frame}[fragile]
    \frametitle{Conclusion and Future Trends - Key Points}
    \begin{block}{Conclusion of Key Points Discussed}
        \begin{enumerate}
            \item \textbf{Definition of Big Data Storage Solutions}:
                \begin{itemize}
                    \item Big Data encompasses vast volumes of structured and unstructured data that traditional applications cannot manage effectively.
                \end{itemize}
            \item \textbf{Types of Storage Solutions}:
                \begin{itemize}
                    \item \textbf{On-Premises Storage}: Ideal for stringent security and compliance.
                    \item \textbf{Cloud-Based Storage}: Offers flexibility, scalability, and on-demand resources.
                \end{itemize}
            \item \textbf{Hybrid Storage Models}: Combine on-premises and cloud solutions for optimal local control and flexibility.
            \item \textbf{Important Considerations}:
                \begin{itemize}
                    \item Scalability, Cost-Efficiency, and Security.
                \end{itemize}
            \item \textbf{Ethical and Compliance Factors}:
                \begin{itemize}
                    \item Importance of regulations such as GDPR and HIPAA.
                \end{itemize}
        \end{enumerate}
    \end{block}
\end{frame}

\begin{frame}[fragile]
    \frametitle{Conclusion and Future Trends - Future Trends}
    \begin{block}{Future Trends in Storage Solutions for Big Data}
        \begin{enumerate}
            \item \textbf{Increased Adoption of Artificial Intelligence (AI)}:
                \begin{itemize}
                    \item AI will enhance data management and analytics capabilities.
                \end{itemize}
            \item \textbf{Serverless Architecture}:
                \begin{itemize}
                    \item Facilitates data storage management without server infrastructure.
                \end{itemize}
            \item \textbf{Enhanced Security Measures}:
                \begin{itemize}
                    \item Focus on encryption and AI-driven threat detection.
                \end{itemize}
            \item \textbf{Multi-Cloud Strategies}:
                \begin{itemize}
                    \item Helps avoid vendor lock-in and ensures redundancy.
                \end{itemize}
            \item \textbf{Edge Computing}:
                \begin{itemize}
                    \item Processing data closer to the source for reduced latency.
                \end{itemize}
        \end{enumerate}
    \end{block}
\end{frame}

\begin{frame}[fragile]
    \frametitle{Conclusion and Future Trends - Summary}
    \begin{block}{Key Points to Emphasize}
        \begin{itemize}
            \item The shift towards cloud-based storage is transforming big data management.
            \item Balancing on-premises and cloud storage is vital for future strategies.
            \item Ethical considerations must be integrated into all data storage approaches.
        \end{itemize}
    \end{block}
    
    \begin{center}
        \textbf{Example Diagram: Future Trends in Big Data Storage Solutions}
    \end{center}
    \begin{verbatim}
    +------------------------------------+
    |         Future Trends in           |
    |      Big Data Storage Solutions     |
    +------------------------------------+
    |   1. AI Integration                 |
    |   2. Serverless Architecture        |
    |   3. Enhanced Security              |
    |   4. Multi-Cloud Strategies         |
    |   5. Edge Computing                 |
    +------------------------------------+
    \end{verbatim}
\end{frame}


\end{document}