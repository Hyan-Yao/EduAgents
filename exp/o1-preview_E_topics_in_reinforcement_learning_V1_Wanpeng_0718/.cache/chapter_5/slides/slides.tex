\documentclass{beamer}

% Theme choice
\usetheme{Madrid} % You can change to e.g., Warsaw, Berlin, CambridgeUS, etc.

% Encoding and font
\usepackage[utf8]{inputenc}
\usepackage[T1]{fontenc}

% Graphics and tables
\usepackage{graphicx}
\usepackage{booktabs}

% Code listings
\usepackage{listings}
\lstset{
basicstyle=\ttfamily\small,
keywordstyle=\color{blue},
commentstyle=\color{gray},
stringstyle=\color{red},
breaklines=true,
frame=single
}

% Math packages
\usepackage{amsmath}
\usepackage{amssymb}

% Colors
\usepackage{xcolor}

% TikZ and PGFPlots
\usepackage{tikz}
\usepackage{pgfplots}
\pgfplotsset{compat=1.18}
\usetikzlibrary{positioning}

% Hyperlinks
\usepackage{hyperref}

% Title information
\title{Week 5: Policy Gradient Methods}
\author{Your Name}
\institute{Your Institution}
\date{\today}

\begin{document}

\frame{\titlepage}

\begin{frame}[fragile]
    \frametitle{Introduction to Policy Gradient Methods}
    
    \begin{block}{Overview}
        Policy Gradient Methods are a class of algorithms in Reinforcement Learning (RL) that directly optimize the policy function, focusing on the agent's strategy to choose actions based on current states.
    \end{block}
    
    \begin{itemize}
        \item Directly parameterizes the policy.
        \item Uses gradient ascent for optimization.
        \item Distinct from value-based methods.
    \end{itemize}
\end{frame}

\begin{frame}[fragile]
    \frametitle{Key Concepts}
    
    \begin{enumerate}
        \item \textbf{Policy (π)}: Mapping from states to actions.
            \begin{itemize}
                \item Deterministic: \( a = \pi(s) \)
                \item Stochastic: \( a \sim \pi(a | s) \)
            \end{itemize}
        
        \item \textbf{Objective}: Maximize the expected cumulative reward:
            \begin{equation}
                J(\theta) = \mathbb{E}[R_t]
            \end{equation}
        
        \item \textbf{Gradient Ascent}: Update policy parameters as follows:
            \begin{equation}
                \theta_{new} = \theta_{old} + \alpha \nabla J(\theta)
            \end{equation}
            where \( \alpha \) is the learning rate.
    \end{enumerate}
\end{frame}

\begin{frame}[fragile]
    \frametitle{Significance and Example}
    
    \begin{block}{Significance}
        \begin{itemize}
            \item Effectively handles high-dimensional and continuous action spaces.
            \item Allows for stochastic policies, representing uncertainty.
            \item Suitable for complex environments (e.g., robotics).
        \end{itemize}
    \end{block}
    
    \begin{block}{Example}
        Consider an agent navigating a grid:
        \begin{itemize}
            \item State Representation: Current position (e.g., (2,3)).
            \item Actions: Move up, down, left, right.
            \item Reward: Positive for reaching the target, negative for hitting walls.
        \end{itemize}
        The policy can be parameterized and updated based on rewards received to maximize overall returns.
    \end{block}
\end{frame}

\begin{frame}[fragile]
    \frametitle{Reinforcement Learning Basics}
    \begin{block}{Key Concepts in Reinforcement Learning (RL)}
        The key concepts include agents, environments, states, actions, rewards, and policies, which are fundamental to understanding RL.
    \end{block}
\end{frame}

\begin{frame}[fragile]
    \frametitle{Reinforcement Learning: Key Concepts}
    \begin{enumerate}
        \item \textbf{Agent:}  
            The entity that makes decisions and takes actions in the environment. \\
            \textit{Example:} A robot navigating an obstacle course.
            
        \item \textbf{Environment:}  
            The external system the agent interacts with. \\
            \textit{Example:} The obstacle course.
            
        \item \textbf{States:}  
            Representations of the current situation within the environment. \\
            \textit{Example:} Robot's location and position of obstacles.

        \item \textbf{Actions:}  
            Choices available to the agent that affect the environment. \\
            \textit{Example:} Move forward, turn left, turn right.
    \end{enumerate}
\end{frame}

\begin{frame}[fragile]
    \frametitle{Reinforcement Learning: Rewards and Policies}
    \begin{enumerate}
        \setcounter{enumi}{4}  % Start numbering from 5
        \item \textbf{Rewards:}  
            Feedback to the agent after it takes an action. Can be positive or negative. \\
            \textit{Example:} Positive reward for navigating an obstacle; negative for bumping into one.

        \item \textbf{Policies:}  
            A strategy for determining actions based on states. Can be deterministic or stochastic. \\
            \textit{Example:} "If in state X, move forward 70\% of the time, turn left 30\%."
    \end{enumerate}
\end{frame}

\begin{frame}[fragile]
    \frametitle{Key Points and Interaction in RL}
    \begin{block}{Key Points to Emphasize}
        \begin{itemize}
            \item Reinforcement learning revolves around the interaction between the agent and the environment.
            \item Goal: Maximize cumulative rewards over time by learning an optimal policy.
            \item Unlike supervised learning, no explicit labels are provided; the agent learns from outcomes.
        \end{itemize}
    \end{block}

    \begin{block}{Basic Interaction in RL}
        \begin{center}
            \texttt{Agent $\rightarrow$ [Takes Action] $\rightarrow$ Environment} \\
            \texttt{|} \\
            \texttt{| $\leftarrow$ [Receives Reward (R)], [New State (S')] $\leftarrow$ } 
        \end{center}
    \end{block}
\end{frame}

\begin{frame}[fragile]
    \frametitle{Illustration of the RL Process}
    \begin{center}
        Transition from a state (S) $\rightarrow$ Agent takes an action (A) $\rightarrow$ Environment updates to a new state (S') $\rightarrow$ Agent receives a reward (R).
    \end{center}
    \begin{block}{Conclusion}
        This foundational understanding sets the stage for deeper dives into specific methods like Policy Gradient methods and their applications in reinforcement learning.
    \end{block}
\end{frame}

\begin{frame}[fragile]
    \frametitle{What Are Policy Gradient Methods? - Overview}
    \begin{block}{Definition}
        Policy Gradient Methods are a class of reinforcement learning algorithms that optimize the policy directly.
    \end{block}
    
    \begin{itemize}
        \item Focus on directly computing gradients of expected rewards.
        \item Parameterize the policy instead of estimating values.
    \end{itemize}
    
    \begin{block}{Key Concepts}
        \begin{itemize}
            \item \textbf{Policy} (\(\pi\)): A function mapping states to actions.
            \item \textbf{Direct Optimization}: Computes gradients of expected return concerning policy parameters.
        \end{itemize}
    \end{block}
\end{frame}

\begin{frame}[fragile]
    \frametitle{How Policy Gradient Methods Differ from Value-Based Methods}
    \begin{enumerate}
        \item \textbf{Approach}:
            \begin{itemize}
                \item \textbf{Policy-Based Methods}: Update the policy directly.
                \item \textbf{Value-Based Methods}: Learn a value function and derive the optimal policy.
            \end{itemize}
        
        \item \textbf{Exploration vs. Exploitation}:
            \begin{itemize}
                \item \textbf{Policy Gradient}: Incorporates exploration with stochastic policies.
                \item \textbf{Value-Based}: Selects actions based on estimated values, often struggles with exploration.
            \end{itemize}

        \item \textbf{Convergence Properties}:
            \begin{itemize}
                \item \textbf{Policy Gradient}: Can converge to local optima, requires careful tuning.
                \item \textbf{Value-Based}: Generally converges faster but risks local optima.
            \end{itemize}
    \end{enumerate}
\end{frame}

\begin{frame}[fragile]
    \frametitle{Examples and Applications of Policy Gradient Methods}
    \begin{block}{Example of a Policy Gradient Method}
        The REINFORCE algorithm computes the gradient as follows:
        \begin{equation}
            \nabla J(\theta) = \mathbb{E}_{\tau \sim \pi_{\theta}} \left[ \nabla \log \pi_{\theta}(a_t | s_t) \cdot R_t \right]
        \end{equation}
        where \( R_t \) is the return following action \( a_t \).
    \end{block}

    \begin{block}{Applications}
        \begin{itemize}
            \item Effective in complex tasks such as video games (e.g., Atari) and robotics.
            \item Directly learn policies in high-dimensional action spaces.
        \end{itemize}
    \end{block}
\end{frame}

\begin{frame}[fragile]
    \frametitle{Mathematics Behind Policy Gradient Methods}
    
    \begin{block}{Overview}
        Policy Gradient Methods optimize policies directly in reinforcement learning.
        This slide covers the mathematical foundations including the policy optimization objective and gradient estimation.
    \end{block}
\end{frame}

\begin{frame}[fragile]
    \frametitle{Policy Optimization Objective}
    
    In reinforcement learning, we define the policy \(\pi_{\theta}(a|s)\), the probability of taking action \(a\) in state \(s\) given parameters \(\theta\). The goal is to maximize the expected return (cumulative reward) by optimizing the policy.

    \begin{block}{Objective Function}
        \[
        J(\theta) = \mathbb{E}_{\tau \sim \pi_{\theta}} \left[ R(\tau) \right]
        \]
        - \(R(\tau)\) is the return following trajectory \(\tau\), which represents a sequence of states, actions, and rewards.
    \end{block}
\end{frame}

\begin{frame}[fragile]
    \frametitle{Gradient Estimation}
    
    To optimize the policy, we compute the gradient of the objective function \(J(\theta)\). Using the **policy gradient theorem**, we derive the policy gradient:

    \begin{block}{Policy Gradient}
        \[
        \nabla J(\theta) = \mathbb{E}_{\tau \sim \pi_{\theta}} \left[ \nabla \log \pi_{\theta}(a|s) R(\tau) \right]
        \]
        - The gradient is expressed as the expected value of the log-probability of actions taken, multiplied by the return.
    \end{block}

    \begin{block}{Importance Sampling for Efficient Estimates}
        \[
        \nabla J(\theta) \approx \frac{1}{N} \sum_{i=1}^{N} \left( \frac{\pi_{\theta}(a_i|s_i)}{\pi_{\beta}(a_i|s_i)} \nabla \log \pi_{\theta}(a_i|s_i) R_i \right)
        \]
        - \(\beta\) is a behavior policy used for sampling, and \(N\) is the number of sampled trajectories.
    \end{block}
\end{frame}

\begin{frame}[fragile]
    \frametitle{Key Points and Closing Remarks}

    \begin{itemize}
        \item \textbf{Direct Optimization:} Policy gradients directly optimize the policy rather than estimating action values.
        \item \textbf{Stochastic Policies:} Effective for environments with high-dimensional and continuous action spaces.
        \item \textbf{Exploration vs. Exploitation:} Stochastic policies help balance exploration during training.
    \end{itemize}

    \begin{block}{Closing Remarks}
        Understanding this mathematical framework enables effective implementation of policy gradient methods in reinforcement learning tasks.
    \end{block}
\end{frame}

\begin{frame}[fragile]
    \frametitle{Advantages of Policy Gradient Methods}
    
    \begin{block}{Key Advantages Overview}
        Policy Gradient methods directly optimize the policy, making them particularly useful for:
        \begin{itemize}
            \item Handling high-dimensional action spaces
            \item Accommodating continuous actions
        \end{itemize}
    \end{block}
\end{frame}

\begin{frame}[fragile]
    \frametitle{Handling High-Dimensional Action Spaces}
    
    \begin{block}{Explanation}
        High-dimensional action spaces involve selecting from a vast range of possible actions. 
        Policy Gradient methods allow agents to learn a probability distribution over actions, 
        enabling effective handling of extensive decision-making scenarios.
    \end{block}
    
    \begin{block}{Example}
        In robotic manipulation, a robot arm can control multiple joints, each with several degrees of freedom. 
        Policy Gradient methods help the agent learn the best action distribution without needing to enumerate all combinations.
    \end{block}
    
    \begin{itemize}
        \item \textbf{Key Point:} Direct optimization of policies over complex high-dimensional spaces.
    \end{itemize}
\end{frame}

\begin{frame}[fragile]
    \frametitle{Accommodating Continuous Actions}
    
    \begin{block}{Explanation}
        Traditional reinforcement learning struggles with continuous action spaces. 
        Policy Gradient methods solve this by representing the policy as a parametrized function 
        that outputs action probabilities or continuous action values.
    \end{block}
    
    \begin{block}{Example}
        In autonomous driving, a vehicle controls its speed and steering angle continuously. 
        The policy can produce outputs like "accelerate to 30 mph" or "steer 15 degrees left".
    \end{block}
    
    \begin{itemize}
        \item \textbf{Key Point:} Flexible modeling of policies as continuous functions allows for smoother control.
    \end{itemize}
\end{frame}

\begin{frame}[fragile]
    \frametitle{Summarizing Advantages}
    
    \begin{itemize}
        \item \textbf{Scalability:} Manages high-dimensional action spaces efficiently.
        \item \textbf{Versatility:} Directly models continuous outputs, enhancing adaptability to real-world scenarios.
    \end{itemize}

    \begin{block}{Mathematical Insight}
        The policy \( \pi_\theta(a|s) \) represents the probability distribution over actions \( a \) given state \( s \) and parameters \( \theta \). 
        The policy can be updated using the policy gradient theorem:
        \begin{equation}
            \nabla J(\theta) = \mathbb{E}_{\pi_\theta} \left[ \nabla \log \pi_\theta(a|s) Q(s, a) \right]
        \end{equation}
        Where:
        \begin{itemize}
            \item \( J(\theta) \) is the expected return.
            \item \( Q(s, a) \) is the action-value function estimating the expected return of action \( a \) in state \( s \).
        \end{itemize}
    \end{block}
\end{frame}

\begin{frame}[fragile]
    \frametitle{Challenges of Policy Gradient Methods - Overview}
    \begin{block}{Introduction}
        Policy Gradient Methods have significant advantages but also face key challenges that impact their effectiveness in reinforcement learning.
    \end{block}
    \begin{itemize}
        \item High Variance
        \item Sample Inefficiency
        \item Convergence Issues
    \end{itemize}
\end{frame}

\begin{frame}[fragile]
    \frametitle{Challenges of Policy Gradient Methods - High Variance}
    \begin{block}{Concept Explanation}
        High variance refers to the large fluctuations in the estimated policy gradient from one update to another primarily due to the stochastic nature of environments and sampled actions.
    \end{block}
    
    \begin{block}{Example}
        An agent playing a video game may have erratic performance due to lucky or unlucky circumstances in various episodes.
    \end{block}
    
    \begin{itemize}
        \item **Key Point:** High variance can slow convergence by causing oscillations in behavior.
        \item **Possible Solutions:** 
        \begin{itemize}
            \item Use variance reduction techniques, such as baselines or advantage functions.
        \end{itemize}
    \end{itemize}
\end{frame}

\begin{frame}[fragile]
    \frametitle{Challenges of Policy Gradient Methods - Sample Inefficiency}
    \begin{block}{Concept Explanation}
        Policy Gradient Methods often require many samples to effectively learn the policy due to noisy gradients.
    \end{block}
    
    \begin{block}{Example}
        A robotic arm learning to grasp objects may take thousands of trials because of the high dimensionality of its action space.
    \end{block}
    
    \begin{itemize}
        \item **Key Point:** Inefficiency leads to long training times and high computational costs.
        \item **Possible Solutions:**
        \begin{itemize}
            \item Use experience replay or combine with value-based methods, such as Actor-Critic.
        \end{itemize}
    \end{itemize}
\end{frame}

\begin{frame}[fragile]
    \frametitle{Challenges of Policy Gradient Methods - Convergence Issues}
    \begin{block}{Concept Explanation}
        Convergence issues occur when the algorithm fails to find an optimal policy or becomes stuck in local minima.
    \end{block}
    
    \begin{block}{Example}
        An agent in a maze might repeatedly find a satisfactory but non-optimal path, missing better solutions.
    \end{block}
    
    \begin{itemize}
        \item **Key Point:** Mis-tuned hyperparameters can worsen convergence issues, leading to divergence or poor solutions.
        \item **Possible Solutions:**
        \begin{itemize}
            \item Properly tune learning rates and use entropy regularization to enhance exploration.
        \end{itemize}
    \end{itemize}
\end{frame}

\begin{frame}[fragile]
    \frametitle{Conclusion and Additional Considerations}
    \begin{block}{Conclusion}
        Addressing high variance, sample inefficiency, and convergence issues is critical for the effective application of Policy Gradient Methods.
    \end{block}
    
    \begin{itemize}
        \item \textbf{Formula:} The policy gradient is computed as:
        \begin{equation}
            \nabla J(\theta) \approx \mathbb{E}_{\tau \sim \pi_{\theta}}\left[\nabla \log \pi_{\theta}(a|s) R\right]
        \end{equation}
        where \(R\) is the return following action \(a\) in state \(s\).
    \end{itemize}
    \begin{block}{Diagrams}
        Consider illustrating these concepts with flowcharts showing learning processes and the impacts of variations.
    \end{block}
\end{frame}

\begin{frame}[fragile]
    \frametitle{Types of Policy Gradient Methods - Overview}
    \begin{block}{Overview}
        Policy gradient methods optimize the policy directly instead of estimating the value function like value-based methods. They are effective in high-dimensional action spaces and continuous actions.
    \end{block}
    \begin{itemize}
        \item Two primary types of policy gradient methods:
        \begin{itemize}
            \item REINFORCE
            \item Actor-Critic approaches
        \end{itemize}
    \end{itemize}
\end{frame}

\begin{frame}[fragile]
    \frametitle{Types of Policy Gradient Methods - REINFORCE}
    \begin{block}{REINFORCE Algorithm}
        \begin{itemize}
            \item \textbf{Concept}:
            \begin{itemize}
                \item A Monte Carlo policy gradient method.
                \item Updates the policy based on returns from each episode.
                \item Utilizes a reward signal to learn a parameterized policy \( \pi_{\theta}(a|s) \).
            \end{itemize}
            \item \textbf{How it Works}:
            \begin{itemize}
                \item Calculates total return \( G_t \) after each episode.
                \item Updates policy using:
                \begin{equation}
                \theta \leftarrow \theta + \alpha \cdot G_t \nabla_\theta \log \pi_{\theta}(a_t|s_t)
                \end{equation}
            \end{itemize}
            \item \textbf{Example}:
            \begin{itemize}
                \item In a game environment, assesses total score to update action probabilities based on outcomes.
            \end{itemize}
        \end{itemize}
    \end{block}
\end{frame}

\begin{frame}[fragile]
    \frametitle{Types of Policy Gradient Methods - Actor-Critic}
    \begin{block}{Actor-Critic Methods}
        \begin{itemize}
            \item \textbf{Concept}:
            \begin{itemize}
                \item Combines value-based and policy gradient methods.
                \item “Actor” updates the policy; “Critic” evaluates actions by estimating the value function \( V(s) \).
            \end{itemize}
            \item \textbf{How it Works}:
            \begin{itemize}
                \item Actor proposes actions; Critic provides feedback through estimated state values.
                \item Update rule for the actor:
                \begin{equation}
                \theta \leftarrow \theta + \alpha \cdot \delta_t \nabla_\theta \log \pi_{\theta}(a_t|s_t)
                \end{equation}
                \item Where \( \delta_t \) is the Temporal Difference (TD) error:
                \begin{equation}
                \delta_t = r_t + \gamma V(s_{t+1}) - V(s_t)
                \end{equation}
            \end{itemize}
            \item \textbf{Example}:
            \begin{itemize}
                \item In a robot navigation task, the actor selects movements while the critic evaluates the reward received.
            \end{itemize}
        \end{itemize}
    \end{block}
\end{frame}

\begin{frame}[fragile]
    \frametitle{Types of Policy Gradient Methods - Key Points and Conclusion}
    \begin{block}{Key Points}
        \begin{itemize}
            \item \textbf{Variance}:
            \begin{itemize}
                \item REINFORCE has high variance; Actor-Critic mitigates this with value estimates.
            \end{itemize}
            \item \textbf{Sample Efficiency}:
            \begin{itemize}
                \item Actor-Critic methods are more sample-efficient than REINFORCE.
            \end{itemize}
            \item \textbf{Applicability}:
            \begin{itemize}
                \item Both methods are suitable for complex environments with high-dimensional action spaces.
            \end{itemize}
        \end{itemize}
    \end{block}
    \begin{block}{Conclusion}
        Understanding the different types of policy gradient methods is crucial for developing effective reinforcement learning algorithms. REINFORCE offers a straightforward approach, while Actor-Critic methods enhance stability and efficiency.
    \end{block}
\end{frame}

\begin{frame}
    \titlepage
\end{frame}

\begin{frame}[fragile]
    \frametitle{Understanding Policy Gradient Methods}
    \begin{block}{Definition}
        Policy Gradient methods are a class of algorithms in reinforcement learning that optimize the policy directly. 
    \end{block}
    \begin{itemize}
        \item Enable agents to learn and make decisions based on experiences.
        \item Applicable in various fields like gaming, robotics, and finance.
    \end{itemize}
\end{frame}

\begin{frame}[fragile]
    \frametitle{Practical Applications}
    \begin{enumerate}
        \item \textbf{Gaming}  
        \begin{itemize}
            \item \textbf{Example:} \textit{AlphaGo} defeated human champions in Go using policy gradients.
            \item \textbf{Key Point:} Effective for shaping strategies in complex environments.
        \end{itemize}
        
        \item \textbf{Robotics}  
        \begin{itemize}
            \item \textbf{Example:} Robot navigation using policy gradients for path planning.
            \item \textbf{Key Point:} Helps robots learn through trial and error for real-time adaptation.
        \end{itemize}
        
        \item \textbf{Finance}  
        \begin{itemize}
            \item \textbf{Example:} Algorithmic trading systems that adapt strategies based on market data.
            \item \textbf{Key Point:} Optimizes decisions in uncertain environments for better profitability.
        \end{itemize}
    \end{enumerate}
\end{frame}

\begin{frame}[fragile]
    \frametitle{Summary of Key Concepts}
    \begin{itemize}
        \item \textbf{Direct Optimization:} Focus on optimizing the policy directly for flexible decision-making.
        \item \textbf{Adaptability:} Excels in dynamic environments requiring rapid information adaptation.
        \item \textbf{Exploration vs. Exploitation:} Balances exploring new strategies with exploiting known actions.
    \end{itemize}
\end{frame}

\begin{frame}[fragile]
    \frametitle{Formulas and Code Snippet}
    \begin{block}{Basic Policy Gradient Update}
        \begin{equation}
            \theta \leftarrow \theta + \alpha \cdot \nabla J(\theta)
        \end{equation}
        Where \( \theta \) represents the policy parameters, \( \alpha \) is the learning rate, and \( J(\theta) \) is the performance measure.
    \end{block}
    
    \begin{block}{Code Snippet}
        \begin{lstlisting}[language=Python]
def update_policy(theta, alpha, gradient):
    theta += alpha * gradient
    return theta
        \end{lstlisting}
    \end{block}
\end{frame}

\begin{frame}[fragile]
    \frametitle{Conclusion}
    By leveraging Policy Gradient methods, diverse industries can achieve sophisticated decision-making capabilities and improved performance, solidifying their role in modern AI applications.
\end{frame}

\begin{frame}[fragile]
    \frametitle{Future Directions in Policy Gradient Research}
    \begin{block}{Introduction}
        Policy Gradient Methods have emerged as a popular choice in Reinforcement Learning (RL) due to their ability to optimize policies directly. Ongoing research aims to enhance efficiency and reduce variance, paving the way for broader applications and improved performance.
    \end{block}
\end{frame}

\begin{frame}[fragile]
    \frametitle{Current Trends in Research - Variance Reduction}
    \begin{itemize}
        \item \textbf{Improved Variance Reduction Techniques}
        \begin{itemize}
            \item Variance in policy gradient estimates can lead to unstable learning and slower convergence.
            \item Recent methods focus on:
                \begin{itemize}
                    \item \textbf{Baseline Methods}: Use advantage functions to subtract a baseline from the reward.
                    \item \textbf{Importance Sampling}: Modify sample drawing and utilization to correct data distribution discrepancies.
                \end{itemize}
        \end{itemize}
        \item \textbf{Example: Generalized Advantage Estimation (GAE)}
        \begin{equation}
            A_t = \sum_{l=0}^{\infty} (\gamma \lambda)^l \delta_{t+l}
        \end{equation}
        where \( \delta_t = R_t + \gamma V(s_{t+1}) - V(s_t) \).
    \end{itemize}
\end{frame}

\begin{frame}[fragile]
    \frametitle{Current Trends in Research - Adaptation and Scalability}
    \begin{itemize}
        \item \textbf{Meta-learning and Online Adaptation}
        \begin{itemize}
            \item Make policy gradient methods more adaptive to changing environments.
            \item \textbf{Meta-learning}: Training agents to learn how to learn, enabling faster adaptation to new tasks.
            \item \textbf{Online Learning}: Real-time policy updates, valuable for dynamic settings.
        \end{itemize}
        \item \textbf{Scalable Architectures}
        \begin{itemize}
            \item \textbf{Distributed and Parallel Training}: Utilize multiple agents or cores for simultaneous evaluation and learning.
            \item \textbf{Neural Architecture Search (NAS)}: Automate the discovery of optimal architectures for policy learning.
        \end{itemize}
    \end{itemize}
\end{frame}

\begin{frame}[fragile]
    \frametitle{Current Trends in Research - Hybrid Approaches}
    \begin{itemize}
        \item \textbf{Hybrid Approaches}
        \begin{itemize}
            \item Combining policy gradient methods with other techniques leverages the strengths of both.
            \item \textbf{Actor-Critic Methods}: Stabilize training and improve sample efficiency.
            \item \textbf{Model-Based RL}: Use learned models to reduce interactions with the actual environment.
        \end{itemize}
        \item \textbf{Example: Actor-Critic Method}
        \begin{itemize}
            \item The critic provides feedback on policy performance, guiding the actor towards better exploration strategies.
        \end{itemize}
    \end{itemize}
\end{frame}

\begin{frame}[fragile]
    \frametitle{Key Points and Conclusion}
    \begin{block}{Key Points to Emphasize}
        \begin{itemize}
            \item Variance reduction methods are crucial for stabilizing learning.
            \item Adaptation and online learning lead to more robust policies.
            \item Scalability is pivotal as environments become more complex.
            \item Hybrid methods present a promising future approach to enhance policy gradient effectiveness.
        \end{itemize}
    \end{block}
    \begin{block}{Conclusion}
        The advancement of policy gradient methods is a vital area of research. Understanding and implementing these emerging trends will be key for leveraging Reinforcement Learning in practical applications.
    \end{block}
\end{frame}

\begin{frame}[fragile]
    \frametitle{Summary and Key Takeaways - Overview}
    \begin{block}{Policy Gradient Methods Overview}
        Policy Gradient methods are a class of reinforcement learning algorithms that optimize the policy directly. 
        They enable agents to learn the optimal actions based on learned probability distributions.
    \end{block}
    
    \begin{block}{Key Advantages}
        \begin{itemize}
            \item \textbf{Direct Policy Learning:} Effectively model complex, high-dimensional action spaces, especially continuous ones.
            \item \textbf{Stochastic Policies:} Supports exploration during training, crucial in dynamic environments.
            \item \textbf{Robustness to Large Action Spaces:} Handles large action spaces better than traditional methods.
        \end{itemize}
    \end{block}
    
    Example: In robotic control tasks, Policy Gradient methods enable learning through exploration rather than value function calculations.
\end{frame}

\begin{frame}[fragile]
    \frametitle{Summary and Key Takeaways - Challenges}
    \begin{block}{Challenges}
        \begin{enumerate}
            \item \textbf{High Variance:} Estimates can be high variance, making learning unstable. Techniques like baselines can help.
            \item \textbf{Sample Inefficiency:} Requires a significant number of episodes to converge, often leading to inefficient learning.
            \item \textbf{Local Optima:} The policy optimization can get trapped in local optima due to non-convex parameter spaces.
        \end{enumerate}
    \end{block}
    
    Example: In gaming, a policy might converge too quickly on suboptimal strategies risking performance stagnation.
\end{frame}

\begin{frame}[fragile]
    \frametitle{Summary and Key Takeaways - Key Points and Formulas}
    \begin{block}{Key Points to Emphasize}
        \begin{itemize}
            \item Policy Gradient methods excel in environments with large or continuous action spaces.
            \item Balancing variance and bias is crucial for effective implementation.
            \item Ongoing research focuses on efficiency and stability improvements.
        \end{itemize}
    \end{block}
    
    \begin{block}{Useful Formulas}
        \begin{equation}
            \theta_{new} = \theta_{old} + \alpha \nabla J(\theta)
        \end{equation}
        where \( \alpha \) is the learning rate and \( J(\theta) \) is the expected reward under the policy \( \pi_\theta \).
        
        \begin{equation}
            \nabla J(\theta) = \mathbb{E}_t \left[\nabla \log \pi_\theta(a_t | s_t) R_t\right]
        \end{equation}
        where \( R_t \) is the return following time \( t \).
    \end{block}
\end{frame}


\end{document}