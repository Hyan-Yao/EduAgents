\documentclass{beamer}

% Theme choice
\usetheme{Madrid} % You can change to e.g., Warsaw, Berlin, CambridgeUS, etc.

% Encoding and font
\usepackage[utf8]{inputenc}
\usepackage[T1]{fontenc}

% Graphics and tables
\usepackage{graphicx}
\usepackage{booktabs}

% Code listings
\usepackage{listings}
\lstset{
  basicstyle=\ttfamily\small,
  keywordstyle=\color{blue},
  commentstyle=\color{gray},
  stringstyle=\color{red},
  breaklines=true,
  frame=single
}

% Math packages
\usepackage{amsmath}
\usepackage{amssymb}

% Colors
\usepackage{xcolor}

% TikZ and PGFPlots
\usepackage{tikz}
\usepackage{pgfplots}
\pgfplotsset{compat=1.18}
\usetikzlibrary{positioning}

% Hyperlinks
\usepackage{hyperref}

% Title information
\title{Week 4: Deep Reinforcement Learning}
\author{Your Name}
\institute{Your Institution}
\date{\today}

\begin{document}

\frame{\titlepage}

\begin{frame}[fragile]
    \titlepage
\end{frame}

\begin{frame}[fragile]
    \frametitle{Overview of Deep Reinforcement Learning}
    Deep Reinforcement Learning (DRL) is an advanced form of machine learning that combines \textbf{Reinforcement Learning (RL)} and \textbf{Deep Learning} techniques. It enables algorithms to learn optimal behaviors through interactions with their environment by maximizing cumulative rewards.
\end{frame}

\begin{frame}[fragile]
    \frametitle{Key Concepts in DRL}
    \begin{block}{Reinforcement Learning}
        \begin{itemize}
            \item \textbf{Agent}: Learner or decision-maker.
            \item \textbf{Environment}: The context within which the agent operates.
            \item \textbf{Actions}: Possible moves the agent can make.
            \item \textbf{States}: Different situations in which the agent can find itself.
            \item \textbf{Rewards}: Feedback received after taking actions; determines the desirability of states.
        \end{itemize}
        The agent learns to take actions in a way that maximizes the total reward over time, known as \textbf{reward maximization}.
    \end{block}
    
    \begin{block}{Deep Learning}
        \begin{itemize}
            \item Involves using neural networks with multiple layers (deep networks) to extract features from raw input data.
            \item Used within DRL to approximate complex functions or policies.
        \end{itemize}
    \end{block}
\end{frame}

\begin{frame}[fragile]
    \frametitle{Significance and Applications of DRL}
    \begin{block}{Significance of DRL}
        \begin{itemize}
            \item capacity to handle high-dimensional state spaces.
            \item Significant advancements in AI, performing tasks that were once thought to be exclusively human.
        \end{itemize}
    \end{block}
    
    \begin{block}{Applications}
        \begin{enumerate}
            \item \textbf{Game Playing}: Achieved superhuman performance in games like Atari and Go.
            \item \textbf{Robotics}: Training autonomous systems for complex tasks.
            \item \textbf{Healthcare}: Aiding in treatment planning and personalized medicine.
            \item \textbf{Finance}: Learning trading strategies from financial market interactions.
        \end{enumerate}
    \end{block}
\end{frame}

\begin{frame}[fragile]
    \frametitle{Key Points and Formula}
    \begin{itemize}
        \item \textbf{Combines RL and Deep Learning}: DRL leverages deep neural networks to process and learn from huge amounts of data.
        \item \textbf{Versatile Applications}: Applications extend across various fields, showcasing extensive utility.
        \item \textbf{Autonomous Learning}: Systems learn directly from environment interactions, continuously improving strategies.
    \end{itemize}
    
    \begin{equation}
        R_t = r_t + \gamma r_{t+1} + \gamma^2 r_{t+2} + \ldots
    \end{equation}
    Where \( R_t \) is the total expected reward, \( r_t \) is the immediate reward at time t, and \( \gamma \) (gamma) is the discount factor for future rewards.
\end{frame}

\begin{frame}[fragile]
    \frametitle{Conclusion}
    Deep Reinforcement Learning represents a cutting-edge area of AI that redefines how machines learn and operate independently, leading to innovative applications that enhance skills across diverse industries.
\end{frame}

\begin{frame}[fragile]
    \frametitle{What is a Deep Q-Network (DQN)?}
    \begin{block}{Definition}
        A Deep Q-Network (DQN) combines Q-Learning—a form of reinforcement learning—with deep learning techniques, effectively handling large or continuous state and action spaces.
    \end{block}
\end{frame}

\begin{frame}[fragile]
    \frametitle{Key Concepts of DQN}
    \begin{enumerate}
        \item \textbf{Reinforcement Learning}:
        DQNs operate within the reinforcement learning framework where agents learn to maximize cumulative rewards through actions.
        
        \item \textbf{Q-Learning}:
        The core mechanism involves learning a Q-value function, \( Q(s, a) \), estimating future rewards for a given action \( a \) in state \( s \).
        
        \item \textbf{Deep Learning}:
        DQNs leverage deep neural networks to approximate Q-values, making it feasible to manage high-dimensional state spaces.
    \end{enumerate}
\end{frame}

\begin{frame}[fragile]
    \frametitle{Architecture of DQN}
    \begin{block}{Network Structure}
        The architecture typically consists of:
        \begin{itemize}
            \item \textbf{Input Layer}: Receives state representation (e.g., pixel values from a game).
            \item \textbf{Hidden Layers}: One or more computational layers with non-linear activations, such as ReLU.
            \item \textbf{Output Layer}: Outputs Q-values corresponding to potential actions.
        \end{itemize}
    \end{block}
    \begin{center}
        \textbf{DQN Architecture Diagram}
    \end{center}
    \begin{lstlisting}
            +---------------------+
            |    Input Layer      |
            | (State representation)|
            +---------------------+
                      |
                     \|/
            +---------------------+
            |    Hidden Layer(s)  |
            |  (Neural network)   |
            +---------------------+
                      |
                     \|/
            +---------------------+
            |    Output Layer     |
            |  Q-values for actions|
            +---------------------+
    \end{lstlisting}
\end{frame}

\begin{frame}[fragile]
    \frametitle{How DQN Works}
    \begin{enumerate}
        \item \textbf{Experience Replay}:
        Experiences (state, action, reward, next state) are stored in a replay buffer to sample during training, reducing correlation and stabilizing learning.
        
        \item \textbf{Target Network}:
        DQNs use two networks; the main and target networks, where the target's weights are updated less frequently to enhance stability.
        
        \item \textbf{Loss Function}:
        The DQN is trained to minimize the loss:
        \begin{equation}
            L = \mathbb{E}\left[(r + \gamma \max_{a'} Q'(s', a') - Q(s, a))^2\right]
        \end{equation}
        where \( r \) is the reward, \( \gamma \) is the discount factor, and \( Q' \) refers to the target network.
    \end{enumerate}
\end{frame}

\begin{frame}[fragile]
    \frametitle{Example: Gaming}
    \begin{block}{Atari Games}
        In playing Atari games, the DQN inputs screen pixels, produces Q-values for actions like "move left", "move right", or "jump", and learns over time through accumulated gameplay experience, improving performance progressively.
    \end{block}
    \begin{block}{Key Points}
        \begin{itemize}
            \item DQNs enable RL agents to learn from high-dimensional inputs.
            \item They integrate Q-Learning with deep learning techniques.
            \item Mechanisms like experience replay and target networks are essential for stable training.
        \end{itemize}
    \end{block}
\end{frame}

\begin{frame}[fragile]
    \frametitle{Key Components of DQN - Overview}
    % Introduction to DQNs, combining deep learning and reinforcement learning.
    \begin{itemize}
        \item DQNs enable agents to make optimal decisions in complex environments.
        \item Key components include:
        \begin{itemize}
            \item Q-values
            \item Neural networks
            \item Experience replay
        \end{itemize}
        \item Understanding these components is essential for grasping DQNs.
    \end{itemize}
\end{frame}

\begin{frame}[fragile]
    \frametitle{Key Components of DQN - Q-values}
    \begin{itemize}
        \item \textbf{Definition:} Q-values (or action values) represent the expected future rewards of actions in specific states.
        \item \textbf{Mathematical Representation:}
        \begin{equation}
            Q(s, a) = \mathbb{E}[R_t | s_t = s, a_t = a]
        \end{equation}
        \item \textbf{Purpose:} DQNs approximate Q-values using neural networks to maximize expected rewards.
    \end{itemize}

    \begin{block}{Example}
        In a game of chess:
        \begin{itemize}
            \item \textbf{State:} Current configuration of the chessboard.
            \item \textbf{Action:} Moving a piece to another square.
            \item \textbf{Q-value:} Reflects expected reward from that move.
        \end{itemize}
    \end{block}
\end{frame}

\begin{frame}[fragile]
    \frametitle{Key Components of DQN - Neural Networks & Experience Replay}
    \begin{itemize}
        \item \textbf{Neural Networks:}
        \begin{itemize}
            \item Used to approximate the Q-value function: \( Q(s, a; \theta) \)
            \item Neural network architecture:
            \begin{itemize}
                \item \textbf{Input Layer:} Represents state.
                \item \textbf{Hidden Layers:} Captures patterns between states and Q-values.
                \item \textbf{Output Layer:} Produces Q-values for all actions.
            \end{itemize}
        \end{itemize}
        
        \item \textbf{Experience Replay:}
        \begin{itemize}
            \item Technique for storing past experiences in a buffer to enhance learning.
            \item \textbf{Buffer:} Stores transitions \( (s, a, r, s', d) \).
            \item \textbf{Sampling:} Random batches sampled during training to reduce correlation between experiences.
        \end{itemize}
        \begin{block}{Example}
            In a video game:
            \begin{itemize}
                \item Moves and results are stored in replay buffer.
                \item Agent reviews past moves to learn effective strategies.
            \end{itemize}
        \end{block}
    \end{itemize}
\end{frame}

\begin{frame}[fragile]
    \frametitle{Implementation Steps for DQN - Introduction}
    \begin{block}{Introduction to DQN}
        DQN (Deep Q-Network) combines Q-Learning with deep neural networks to approximate the optimal Q-value function. It allows agents to learn optimal policies directly from high-dimensional sensory inputs, such as images.
    \end{block}
    \begin{block}{Overview of Implementation Steps}
        \begin{itemize}
            \item Step 1: Setup the Environment
            \item Step 2: Initialize Components
            \item Step 3: Experience Replay Memory
            \item Step 4: Policy and Target Networks
            \item Step 5: Training Loop
            \item Step 6: Sample Replay and Optimize
        \end{itemize}
    \end{block}
\end{frame}

\begin{frame}[fragile]
    \frametitle{Implementation Steps for DQN - Setup and Initialization}
    \begin{block}{Step 1: Setup the Environment}
        \begin{itemize}
            \item **Select an Environment**: Start with a simple task (e.g., CartPole) from OpenAI Gym.
            \item **Initialize Environment**:
        \end{itemize}
        \begin{lstlisting}[language=Python]
        import gym
        env = gym.make('CartPole-v1')
        \end{lstlisting}
    \end{block}
    
    \begin{block}{Step 2: Initialize Components}
        \begin{itemize}
            \item **Neural Network**: Define a neural network to approximate the Q-values.
        \end{itemize}
        \begin{lstlisting}[language=Python]
        import torch
        import torch.nn as nn
        import torch.optim as optim
        
        class DQN(nn.Module):
            def __init__(self, input_dim, output_dim):
                super(DQN, self).__init__()
                self.fc1 = nn.Linear(input_dim, 128)
                self.fc2 = nn.Linear(128, output_dim)
            
            def forward(self, x):
                x = torch.relu(self.fc1(x))
                return self.fc2(x)
        \end{lstlisting}
    \end{block}
\end{frame}

\begin{frame}[fragile]
    \frametitle{Implementation Steps for DQN - Experience Replay and Training}
    \begin{block}{Step 3: Experience Replay Memory}
        \begin{itemize}
            \item **Store Experiences**: Implement a replay memory to store (state, action, reward, next state).
        \end{itemize}
        \begin{lstlisting}[language=Python]
        from collections import deque
        import random

        replay_memory = deque(maxlen=replay_memory_size)

        def store_experience(state, action, reward, next_state):
            replay_memory.append((state, action, reward, next_state))
        \end{lstlisting}
    \end{block}

    \begin{block}{Key Steps Continuing...}
        \begin{itemize}
            \item Characterizing Steps 4, 5, and 6 involves managing networks, training loops, and optimizing through experience replay.
            \item Focus on the importance of the ε-greedy policy and its role in exploration vs exploitation.
        \end{itemize}
    \end{block}
    
    \begin{block}{Conclusion}
        By carefully following these implementation steps, you can create a DQN agent for simple tasks and build the foundation for more complex reinforcement learning challenges.
    \end{block}
\end{frame}

\begin{frame}[fragile]
    \frametitle{Working with the Environment - Introduction}
    % Introduction to OpenAI Gym
    OpenAI Gym is a toolkit designed for developing and comparing reinforcement learning algorithms. It provides a simple and consistent interface for various environments where agents can learn through interactions.
\end{frame}

\begin{frame}[fragile]
    \frametitle{Setting Up OpenAI Gym}
    % Step-by-step guide on setting up OpenAI Gym
    \begin{block}{Step 1: Installation}
        Install OpenAI Gym via pip:
        \begin{lstlisting}[language=bash]
pip install gym
        \end{lstlisting}
    \end{block}

    \begin{block}{Step 2: Importing Libraries}
        Import the necessary Python libraries:
        \begin{lstlisting}[language=python]
import gym
        \end{lstlisting}
    \end{block}

    \begin{block}{Step 3: Creating an Environment}
        Create an environment using:
        \begin{lstlisting}[language=python]
env = gym.make('CartPole-v1')  # Create CartPole environment
        \end{lstlisting}
        \textbf{Key Points:}
        \begin{itemize}
            \item Gym supports various environments, see the Gym documentation.
            \item Choose environments based on task complexity and requirements.
        \end{itemize}
\end{frame}

\begin{frame}[fragile]
    \frametitle{Interacting with the Environment}
    % Steps to interact with OpenAI Gym environment
    \begin{block}{Step 4: Resetting the Environment}
        Initialize the environment for a new episode:
        \begin{lstlisting}[language=python]
state = env.reset()  # Reset and get initial observation
        \end{lstlisting}
    \end{block}

    \begin{block}{Step 5: Taking Actions}
        Make the agent take actions:
        \begin{lstlisting}[language=python]
action = env.action_space.sample()  # Sample a random action
next_state, reward, done, info = env.step(action)  # Take action
        \end{lstlisting}
        \textbf{Detailed Explanation:}
        \begin{itemize}
            \item \texttt{next\_state}: Resulting state from action.
            \item \texttt{reward}: Reward received after action.
            \item \texttt{done}: Boolean indicating episode conclusion.
            \item \texttt{info}: Additional episode information (optional).
        \end{itemize}
\end{frame}

\begin{frame}[fragile]
    \frametitle{Example of a Single Episode Loop}
    % Code illustrating a single episode loop
    Here’s how to run a single episode loop:
    \begin{lstlisting}[language=python]
for episode in range(100):  # Run for 100 episodes
    state = env.reset()
    done = False
    while not done:
        action = env.action_space.sample()  # Random action
        next_state, reward, done, info = env.step(action)
        # [Optional: Add logic to update agent here]
    \end{lstlisting}
\end{frame}

\begin{frame}[fragile]
    \frametitle{Closing the Environment}
    % Final steps when done with OpenAI Gym
    Always close the environment to free up resources:
    \begin{lstlisting}[language=python]
env.close()  # Close the environment window
    \end{lstlisting}

    \textbf{Summary:}
    \begin{itemize}
        \item OpenAI Gym is vital for reinforcement learning.
        \item Setting up involves installation, environment creation, and resetting.
        \item Interacting includes taking actions, receiving rewards, and checking if the episode is done.
        \item Always close the environment after use.
    \end{itemize}
    \textbf{Next Steps:} Explore techniques for training the DQN agent and implementing various training mechanisms.
\end{frame}

\begin{frame}
    \frametitle{Training the DQN Agent - Overview}
    Deep Q-Learning Networks (DQN) are a cornerstone of deep reinforcement learning. Their training process consists of several techniques designed to optimize how the agent learns from its environment.
    
    \begin{block}{Key Steps in Training a DQN Agent}
        \begin{itemize}
            \item Define the environment and the action space.
            \item Use experience replay to store and sample experiences.
            \item Implement a target network to stabilize training.
        \end{itemize}
    \end{block}
\end{frame}

\begin{frame}
    \frametitle{Training the DQN Agent - Loss Functions}
    The loss function quantifies how well the network's predicted Q-values match the expected Q-values. The primary loss function is derived from the Q-learning update rule:
    
    \begin{equation}
        Q(s, a) \leftarrow r + \gamma \max_a Q'(s', a)
    \end{equation}
    
    Where:
    \begin{itemize}
        \item $Q(s, a)$ is the estimated Q-value.
        \item $r$ is the reward received after taking action $a$ in state $s$.
        \item $\gamma$ is the discount factor.
        \item $Q'(s', a)$ is the Q-value of the next state $s'$ for the best action.
    \end{itemize}
    
    The loss function $L$ is given by:
    \begin{equation}
        L = \frac{1}{N} \sum_{i=1}^{N} \left(r_i + \gamma \max_a Q'(s_i', a) - Q(s_i, a_i)\right)^2
    \end{equation}
    
    Where $N$ is the batch size, and $(s_i, a_i, r_i, s_i')$ are tuples in the replay memory.
\end{frame}

\begin{frame}[fragile]
    \frametitle{Training the DQN Agent - Optimization Methods}
    \begin{itemize}
        \item \textbf{Stochastic Gradient Descent (SGD)}: 
        Adjusts weights by calculating the gradient of the loss function with respect to the weights for a random mini-batch of samples.

        \item \textbf{Adam Optimizer}: 
        An adaptive learning rate method that tends to converge faster than traditional SGD. Key features include:
        \begin{itemize}
            \item Adaptive learning rates based on first and second moments of the gradients.
            \item Use of momentum to navigate ravines and noise in the loss landscape.
        \end{itemize}
    \end{itemize}

    \begin{block}{Example Code Snippet for Optimizer}
    \begin{lstlisting}[language=Python]
import torch
import torch.optim as optim

# Assuming model is your DQN model
optimizer = optim.Adam(model.parameters(), lr=0.001)

# During the training loop
optimizer.zero_grad()
loss.backward()
optimizer.step()
    \end{lstlisting}
    \end{block}
\end{frame}

\begin{frame}
    \frametitle{Training the DQN Agent - Key Techniques}
    \begin{itemize}
        \item \textbf{Experience Replay}: 
        Sample random batches from the replay memory to break correlation between consecutive experiences and improve learning stability.
        
        \item \textbf{Target Network}: 
        Use a separate network to compute the target Q-values, updated less frequently to reduce oscillations in learning.
    \end{itemize}
\end{frame}

\begin{frame}
    \frametitle{Training the DQN Agent - Summary}
    \begin{block}{Key Points to Emphasize}
        \begin{itemize}
            \item The combination of a well-defined loss function, effective optimization techniques, and strategies like experience replay contribute to the success of DQN training.
            \item Monitoring and adjusting hyperparameters, such as learning rates and discount factors, can significantly affect the agent's learning efficiency and performance.
        \end{itemize}
    \end{block}
    
    \textbf{Diagram Suggestion:} Flowchart illustrating the DQN training loop, including interactions between experience replay, loss calculation, and optimization.
\end{frame}

\begin{frame}[fragile]
    \frametitle{Evaluating DQN Performance - Introduction}
    \begin{itemize}
        \item Evaluating DQN performance is essential for assessing its effectiveness in reinforcement learning tasks.
        \item Primary metrics for evaluation:
        \begin{itemize}
            \item Cumulative Rewards
            \item Convergence
        \end{itemize}
    \end{itemize}
\end{frame}

\begin{frame}[fragile]
    \frametitle{Evaluating DQN Performance - Cumulative Rewards}
    \begin{block}{Definition}
        Cumulative rewards refer to the total rewards an agent accumulates over time while interacting with the environment.
    \end{block}
    \begin{block}{Calculation}
        If \( r_t \) is the reward received at time \( t \), the cumulative reward after \( T \) time steps is calculated as:
        \begin{equation}
        R_T = \sum_{t=0}^{T} r_t
        \end{equation}
    \end{block}
    \begin{block}{Importance}
        Higher cumulative rewards indicate better performance, reflecting the agent's decision-making ability.
    \end{block}
\end{frame}

\begin{frame}[fragile]
    \frametitle{Evaluating DQN Performance - Examples and Key Points}
    \begin{itemize}
        \item Example: 
        \begin{itemize}
            \item In a game, if an agent wins 10 points in the first round and 15 in the second, the cumulative reward after two rounds is \( R_2 = 10 + 15 = 25 \).
        \end{itemize}
    
        \item Regular evaluation during training is crucial:
        \begin{itemize}
            \item Use periodic assessments.
            \item Consider using a validation set to prevent overfitting.
        \end{itemize}
    \end{itemize}
\end{frame}

\begin{frame}[fragile]
    \frametitle{Evaluating DQN Performance - Convergence}
    \begin{block}{Definition}
        Convergence in DQNs refers to when the agent's policy becomes stable, with Q-values changing insignificantly with further training.
    \end{block}
    \begin{block}{Indicators of Convergence}
        \begin{itemize}
            \item Graphical Analysis: Plot average cumulative rewards over episodes.
            \item Threshold Values: Set a threshold for Q-value changes (e.g., less than \( \epsilon = 0.01 \)).
        \end{itemize}
    \end{block}
    \begin{block}{Example}
        If average cumulative rewards progress to 20, 22, 23, and finally settle around 24, this indicates that DQN's learning has stabilized.
    \end{block}
\end{frame}

\begin{frame}[fragile]
    \frametitle{Evaluating DQN Performance - Practical Insights}
    \begin{itemize}
        \item Combine cumulative rewards and convergence metrics for a comprehensive performance view.
        \item Implement logging of rewards and Q-values for systematic evaluation.
    \end{itemize}
\end{frame}

\begin{frame}[fragile]
    \frametitle{Evaluating DQN Performance - Code Snippet}
    Below is an example of a code snippet for tracking cumulative rewards during training:
    \begin{lstlisting}[language=Python]
cumulative_reward = 0
for episode in range(num_episodes):
    state = env.reset()
    done = False
    while not done:
        action = dqn_agent.get_action(state)
        next_state, reward, done, _ = env.step(action)
        cumulative_reward += reward
        state = next_state
    print(f'Episode {episode} - Cumulative Reward: {cumulative_reward}')
    \end{lstlisting}
\end{frame}

\begin{frame}
    \titlepage
\end{frame}

\begin{frame}
    \frametitle{Overview}
    This hands-on task focuses on implementing the Deep Q-Network (DQN) algorithm to solve a designated simple task, reinforcing concepts learned in previous sessions about DQN's structure and performance evaluation.
\end{frame}

\begin{frame}
    \frametitle{Learning Objectives}
    \begin{enumerate}
        \item \textbf{Understand DQN Structure}: Familiarize participants with the architecture of DQN and its components.
        \item \textbf{Exercise Implementation}: Gain practical experience in coding the DQN algorithm.
        \item \textbf{Apply Reinforcement Learning}: Use a simple environment to observe how an agent learns optimal actions.
    \end{enumerate}
\end{frame}

\begin{frame}
    \frametitle{Components of DQN}
    \begin{itemize}
        \item \textbf{Q-Learning}: A reinforcement learning algorithm that uses a value function to estimate the quality of actions taken in a given state.
        \item \textbf{Neural Network}: DQN employs a neural network to approximate the Q-value function, allowing it to generalize to unseen states.
        \item \textbf{Experience Replay}: Stores past experiences in a replay memory to break correlation and improve learning stability.
        \item \textbf{Target Network}: Helps stabilize training by providing consistent Q-value targets.
    \end{itemize}
\end{frame}

\begin{frame}
    \frametitle{Simple Task Setup}
    \textbf{Environment}: The task could be based on a grid-world scenario, such as navigating a simple maze or a classic game like CartPole.
    
    \begin{itemize}
        \item \textbf{State Representation}: Map the environment's states (e.g., player position, goal position) as numerical values.
        \item \textbf{Action Space}: Define possible actions (e.g., move left, right, up, down).
        \item \textbf{Reward Structure}: Establish how rewards are given (e.g., moving closer to the goal = +1, hitting a wall = -1).
    \end{itemize}
\end{frame}

\begin{frame}[fragile]
    \frametitle{Implementation Steps}
    \begin{enumerate}
        \item \textbf{Set Up Libraries}: Use Python libraries such as TensorFlow or PyTorch for the neural network, along with OpenAI Gym for environment simulation.
        
        \begin{lstlisting}[language=Python]
import numpy as np
import random
import gym
from collections import deque
from keras.models import Sequential
from keras.layers import Dense
        \end{lstlisting}

        \item \textbf{Define Parameters}: Initialize hyperparameters important for DQN.
        \begin{itemize}
            \item learning\_rate = 0.001
            \item discount\_factor = 0.99
            \item initial\_exploration = 1.0
            \item exploration\_decay = 0.995
            \item min\_exploration = 0.01
        \end{itemize}
    \end{enumerate}
\end{frame}

\begin{frame}[fragile]
    \frametitle{Implementation Steps (cont'd)}
    \begin{enumerate}[resume]
        \item \textbf{Build the DQN Model}: Create a neural network with an input layer that matches the state size and output layer equal to the action space.
        
        \begin{lstlisting}[language=Python]
def build_dqn(input_shape, action_space):
    model = Sequential()
    model.add(Dense(24, input_dim=input_shape, activation='relu'))
    model.add(Dense(24, activation='relu'))
    model.add(Dense(action_space, activation='linear'))
    model.compile(loss='mse', optimizer='adam')
    return model
        \end{lstlisting}

        \item \textbf{Training Loop}: Implement the main training loop with actions selection, experience storage, and periodic training of the model.
        
        \item \textbf{Evaluation}: Compare DQN's performance with random actions initially and observe how it learns optimal actions over time.
    \end{enumerate}
\end{frame}

\begin{frame}
    \frametitle{Key Points to Emphasize}
    \begin{itemize}
        \item \textbf{Exploration vs. Exploitation}: Highlight the balance between trying out new actions (exploration) versus selecting known actions that yield high rewards (exploitation).
        \item \textbf{Convergence}: Emphasize that DQN should stabilize towards optimal Q-values, which can be evaluated using the criteria from the last slide on performance.
    \end{itemize}
\end{frame}

\begin{frame}
    \frametitle{Closing Remarks}
    After implementing the proposed task, be prepared to discuss the implementation experience, challenges faced, and what learning took place during the execution. This will prepare us for the next session, where we tackle common challenges in DQN implementation.
\end{frame}

\begin{frame}
    \frametitle{Ready to Code!}
    This task aims not only to enhance your coding skills but also to deepen your understanding of how DQNs function in practice. Happy coding!
\end{frame}

\begin{frame}[fragile]
    \frametitle{Common Challenges in DQN Implementation}
    \begin{block}{Overview}
        Deep Q-Networks (DQN) represent a significant advancement in reinforcement learning (RL) techniques. However, various challenges may hinder performance.
    \end{block}
    This slide will explore common pitfalls, particularly focusing on:
    \begin{itemize}
        \item Overfitting
        \item Exploratory Behavior
    \end{itemize}
\end{frame}

\begin{frame}[fragile]
    \frametitle{Key Challenges: Overfitting}
    \begin{block}{Definition}
        Overfitting occurs when a model learns noise and details in the training data to the extent that it negatively impacts its performance on unseen data.
    \end{block}
    \begin{itemize}
        \item \textbf{Symptoms}:
        \begin{itemize}
            \item High training accuracy but low testing accuracy.
            \item Poor generalization to different scenarios.
        \end{itemize}
        \item \textbf{Example}: A DQN agent trained in a static environment may struggle with real-world variations.
    \end{itemize}
    
    \begin{block}{Mitigation Strategies}
        \begin{itemize}
            \item Experience Replay
            \item Target Network
            \item Regularization Techniques
        \end{itemize}
    \end{block}
\end{frame}

\begin{frame}[fragile]
    \frametitle{Key Challenges: Exploratory Behavior}
    \begin{block}{Definition}
        Exploration refers to the agent's ability to try new actions that may lead to higher rewards, contrasting with exploiting known best actions.
    \end{block}
    \begin{itemize}
        \item \textbf{Symptoms}:
        \begin{itemize}
            \item Excessive exploration leads to suboptimal policies.
            \item Insufficient exploration causes stagnation in learning.
        \end{itemize}
        \item \textbf{Example}: An agent relying solely on safe actions may miss out on profitable strategies.
    \end{itemize}
    
    \begin{block}{Mitigation Strategies}
        \begin{itemize}
            \item Epsilon-Greedy Strategy
            \item Boltzmann Exploration
            \item Noisy Networks
        \end{itemize}
    \end{block}
\end{frame}

\begin{frame}[fragile]
    \frametitle{Key Takeaways}
    \begin{itemize}
        \item Recognizing and addressing overfitting is essential for generalization in DQN implementations.
        \item Effective exploration strategies are crucial for discovering optimal policies and ensuring robust learning.
        \item Balancing exploration and exploitation is a dynamic process that significantly impacts an agent's performance.
    \end{itemize}
    By understanding and addressing these challenges, practitioners can enhance the implementation of DQNs for various reinforcement learning tasks.
\end{frame}

\begin{frame}[fragile]
    \frametitle{Overview}
    Deep Reinforcement Learning (DRL) continues to evolve rapidly, presenting exciting new research opportunities and emerging trends. Exploration in these areas can lead to advancements in various applications, from robotics to game playing. 
    \begin{itemize}
        \item Key future directions in DRL:
        \item Efficiency, interpretability, and specialized algorithm development
    \end{itemize}
\end{frame}

\begin{frame}[fragile]
    \frametitle{1. Improved Sample Efficiency}
    \begin{itemize}
        \item \textbf{Concept}:
        \begin{itemize}
            \item Current DRL algorithms require vast amounts of data and interaction with the environment.
        \end{itemize}
        \item \textbf{Emerging Techniques}:
        \begin{itemize}
            \item \textit{Model-Based Reinforcement Learning}: Learn a model of the environment for better decision-making.
            \item \textit{Meta-Learning}: Enables models to adapt quickly to new tasks with limited data.
        \end{itemize}
    \end{itemize}
\end{frame}

\begin{frame}[fragile]
    \frametitle{2. Enhanced Exploration Strategies}
    \begin{itemize}
        \item \textbf{Concept}:
        \begin{itemize}
            \item Effective exploration of the state-action space is essential.
        \end{itemize}
        \item \textbf{Research Opportunities}:
        \begin{itemize}
            \item \textit{Curiosity-driven Exploration}: Facilitate exploration of novel states.
            \item \textit{Hierarchical Reinforcement Learning (HRL)}: Decomposes complex tasks into simpler sub-tasks.
        \end{itemize}
    \end{itemize}
\end{frame}

\begin{frame}[fragile]
    \frametitle{3. Interpretability in DRL}
    \begin{itemize}
        \item \textbf{Concept}:
        \begin{itemize}
            \item Understanding model decisions is critical in sensitive areas like healthcare.
        \end{itemize}
        \item \textbf{Techniques}:
        \begin{itemize}
            \item \textit{Visualization of Policies}: Tools to visualize learned policies and agent behavior.
            \item \textit{Explainable AI (XAI)}: Integrating XAI with DRL for decision-making insights.
        \end{itemize}
    \end{itemize}
\end{frame}

\begin{frame}[fragile]
    \frametitle{4. Multi-Agent Reinforcement Learning (MARL)}
    \begin{itemize}
        \item \textbf{Concept}:
        \begin{itemize}
            \item Environments with multiple agents present unique challenges.
        \end{itemize}
        \item \textbf{Trends}:
        \begin{itemize}
            \item \textit{Cooperative vs. Competitive Learning}: Collaboration or competition among agents.
            \item \textit{Scaling to Real-World Environments}: Adapting to complex real-world problems.
        \end{itemize}
    \end{itemize}
\end{frame}

\begin{frame}[fragile]
    \frametitle{5. Integration with Other ML Paradigms}
    \begin{itemize}
        \item \textbf{Concept}:
        \begin{itemize}
            \item Combining DRL with other fields can yield breakthroughs.
        \end{itemize}
        \item \textbf{Examples}:
        \begin{itemize}
            \item \textit{Supervised Pre-training}: Pre-training networks to improve DRL performance.
            \item \textit{Neuroevolution}: Using evolutionary algorithms to optimize network architectures.
        \end{itemize}
    \end{itemize}
\end{frame}

\begin{frame}[fragile]
    \frametitle{Key Points to Remember}
    \begin{itemize}
        \item DRL is rapidly evolving with numerous avenues for research.
        \item Focus areas include sample efficiency, interpretability, and multi-agent systems.
        \item Interdisciplinary approaches can significantly enhance DRL applications.
    \end{itemize}
\end{frame}

\begin{frame}[fragile]
    \frametitle{Summary}
    The future of Deep Reinforcement Learning holds promise, driven by methodology advancements and interdisciplinary integration. By addressing challenges such as sample efficiency and agent interpretability, researchers can greatly enhance DRL's capabilities and applicability.
\end{frame}


\end{document}