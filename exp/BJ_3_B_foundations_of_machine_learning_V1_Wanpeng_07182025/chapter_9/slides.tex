\documentclass[aspectratio=169]{beamer}

% Theme and Color Setup
\usetheme{Madrid}
\usecolortheme{whale}
\useinnertheme{rectangles}
\useoutertheme{miniframes}

% Additional Packages
\usepackage[utf8]{inputenc}
\usepackage[T1]{fontenc}
\usepackage{graphicx}
\usepackage{booktabs}
\usepackage{listings}
\usepackage{amsmath}
\usepackage{amssymb}
\usepackage{xcolor}
\usepackage{tikz}
\usepackage{pgfplots}
\pgfplotsset{compat=1.18}
\usetikzlibrary{positioning}
\usepackage{hyperref}

% Custom Colors
\definecolor{myblue}{RGB}{31, 73, 125}
\definecolor{mygray}{RGB}{100, 100, 100}
\definecolor{mygreen}{RGB}{0, 128, 0}
\definecolor{myorange}{RGB}{230, 126, 34}
\definecolor{mycodebackground}{RGB}{245, 245, 245}

% Set Theme Colors
\setbeamercolor{structure}{fg=myblue}
\setbeamercolor{frametitle}{fg=white, bg=myblue}
\setbeamercolor{title}{fg=myblue}
\setbeamercolor{section in toc}{fg=myblue}
\setbeamercolor{item projected}{fg=white, bg=myblue}
\setbeamercolor{block title}{bg=myblue!20, fg=myblue}
\setbeamercolor{block body}{bg=myblue!10}
\setbeamercolor{alerted text}{fg=myorange}

% Set Fonts
\setbeamerfont{title}{size=\Large, series=\bfseries}
\setbeamerfont{frametitle}{size=\large, series=\bfseries}
\setbeamerfont{caption}{size=\small}
\setbeamerfont{footnote}{size=\tiny}

% Code Listing Style
\lstdefinestyle{customcode}{
  backgroundcolor=\color{mycodebackground},
  basicstyle=\footnotesize\ttfamily,
  breakatwhitespace=false,
  breaklines=true,
  commentstyle=\color{mygreen}\itshape,
  keywordstyle=\color{blue}\bfseries,
  stringstyle=\color{myorange},
  numbers=left,
  numbersep=8pt,
  numberstyle=\tiny\color{mygray},
  frame=single,
  framesep=5pt,
  rulecolor=\color{mygray},
  showspaces=false,
  showstringspaces=false,
  showtabs=false,
  tabsize=2,
  captionpos=b
}
\lstset{style=customcode}

% Custom Commands
\newcommand{\hilight}[1]{\colorbox{myorange!30}{#1}}
\newcommand{\source}[1]{\vspace{0.2cm}\hfill{\tiny\textcolor{mygray}{Source: #1}}}
\newcommand{\concept}[1]{\textcolor{myblue}{\textbf{#1}}}
\newcommand{\separator}{\begin{center}\rule{0.5\linewidth}{0.5pt}\end{center}}

% Document Start
\begin{document}

\frame{\titlepage}

\begin{frame}[fragile]
    \title{Introduction to Advanced Topics in Machine Learning}
    \author{Lecture Notes}
    \date{\today}
    \maketitle
\end{frame}

\begin{frame}[fragile]
    \frametitle{Overview of Machine Learning}
    \begin{itemize}
        \item \textbf{Machine Learning Overview:}
        \begin{itemize}
            \item A field of artificial intelligence focused on algorithms that learn from data.
            \item Approaches include supervised learning, unsupervised learning, and reinforcement learning.
        \end{itemize}
        
        \item \textbf{Neural Networks:}
        \begin{itemize}
            \item Inspired by biological neural networks.
            \item Composed of interconnected layers of nodes (neurons) that process data.
        \end{itemize}
    \end{itemize}
\end{frame}

\begin{frame}[fragile]
    \frametitle{Significance of Neural Networks}
    \begin{itemize}
        \item \textbf{Handling Non-linear Relationships:}
        \begin{itemize}
            \item Models complex non-linear relationships in tasks like image recognition and NLP.
        \end{itemize}
        
        \item \textbf{Feature Extraction:}
        \begin{itemize}
            \item Automatically extracts relevant features from raw data.
        \end{itemize}
        
        \item \textbf{Scalability:}
        \begin{itemize}
            \item Performs well with vast amounts of data; improves as more data is available.
        \end{itemize}
    \end{itemize}
\end{frame}

\begin{frame}[fragile]
    \frametitle{Practical Applications and Architecture}
    \begin{itemize}
        \item \textbf{Applications:}
        \begin{itemize}
            \item \textbf{Image and Speech Recognition:} 
            \begin{itemize}
                \item CNNs for image classification, RNNs for sequential data like speech.
            \end{itemize}
            \item \textbf{Health Care:} 
            \begin{itemize}
                \item Used for predictive analytics and disease diagnosis from medical images.
            \end{itemize}
        \end{itemize}
        
        \item \textbf{Basic Neural Network Structure:}
        \begin{itemize}
            \item \textbf{Input Layer:} Receives input features.
            \item \textbf{Hidden Layer(s):} Detects complex patterns.
            \item \textbf{Output Layer:} Produces final output.
        \end{itemize}
        
        \item \textbf{Neuron Activation Formula:}
        \begin{equation}
            a_j = f\left(\sum_{i} w_{ij} a_i + b_j\right)
        \end{equation}
    \end{itemize}
\end{frame}

\begin{frame}[fragile]
    \frametitle{Key Points to Emphasize}
    \begin{itemize}
        \item Neural networks represent a significant advancement in modeling due to flexibility and generalization ability.
        \item Play a crucial role in modern ML applications and are foundational to deep learning.
        \item Understanding neural network architecture is essential for grasping advanced models.
    \end{itemize}
\end{frame}

\begin{frame}{What are Neural Networks?}
    \begin{block}{Definition}
        Neural networks are a subset of machine learning models designed to recognize patterns, inspired by the human brain with interconnected groups of artificial neurons. They process information using dynamic state responses to external inputs.
    \end{block}
\end{frame}

\begin{frame}{Key Features of Neural Networks}
    \begin{itemize}
        \item \textbf{Structure}
        \begin{itemize}
            \item \textbf{Input Layer}: Receives various forms of input data.
            \item \textbf{Hidden Layer(s)}: Processes inputs and generates outputs. The configuration of layers impacts performance.
            \item \textbf{Output Layer}: Produces final output from hidden layers' processing.
        \end{itemize}
    \end{itemize}
\end{frame}

\begin{frame}{Distinguishing Neural Networks from Traditional Models}
    \begin{itemize}
        \item \textbf{Representation}:
        \begin{itemize}
            \item Traditional models require manual feature engineering.
            \item Neural networks automatically learn representations from raw data.
        \end{itemize}
        
        \item \textbf{Complexity and Non-linearity}:
        \begin{itemize}
            \item Neural networks model complex non-linear relationships.
            \item Traditional models struggle with non-linear boundaries.
        \end{itemize}
        
        \item \textbf{Training Method}:
        \begin{itemize}
            \item Neural networks use backpropagation for training.
            \item Traditional models rely on simpler optimization methods.
        \end{itemize}
    \end{itemize}
\end{frame}

\begin{frame}{Examples}
    \begin{enumerate}
        \item \textbf{Image Recognition}: 
        Neural networks automatically learn features to identify objects (e.g., cats vs. dogs). Traditional models require explicit feature engineering.
        
        \item \textbf{Natural Language Processing}:
        Recurrent Neural Networks (RNNs) and Transformers excel in understanding and generating language, capturing context better than traditional models.
    \end{enumerate}
\end{frame}

\begin{frame}{Neural Network Architecture}
    \begin{block}{Diagram Description}
        \textit{Illustrate a basic neural network structure showcasing the Input Layer, Hidden Layer(s), and Output Layer. Use arrows to indicate connections and highlight how data flows from Input to Output.}
    \end{block}
\end{frame}

\begin{frame}[fragile]{Code Snippet}
    \begin{lstlisting}[language=Python]
import tensorflow as tf

# Simple neural network model using Keras
model = tf.keras.Sequential([
    tf.keras.layers.Dense(128, activation='relu', input_shape=(input_dim,)),
    tf.keras.layers.Dense(64, activation='relu'),
    tf.keras.layers.Dense(1, activation='sigmoid')
])

model.compile(optimizer='adam', loss='binary_crossentropy', metrics=['accuracy'])
    \end{lstlisting}
\end{frame}

\begin{frame}{Conclusion}
    \begin{block}{Key Points}
        \begin{itemize}
            \item Neural networks are powerful tools that learn from data, suitable for complex tasks.
            \item Their ability to capture non-linear relationships distinguishes them from traditional models.
            \item Understanding their architecture is crucial for advancing in machine learning.
        \end{itemize}
    \end{block}
    
    Neural networks represent a revolutionary approach to machine learning, enabling innovations across various fields.
\end{frame}

\begin{frame}[fragile]
    \frametitle{Basic Structure of Neural Networks}
    \begin{block}{Introduction to Neural Networks}
        Neural networks are computational models inspired by the human brain, designed to recognize patterns and learn from data.
        They consist of interconnected nodes, called neurons, organized in layers.
    \end{block}
\end{frame}

\begin{frame}[fragile]
    \frametitle{Key Components of Neural Networks}
    \begin{enumerate}
        \item \textbf{Neurons}
        \begin{itemize}
            \item Fundamental building blocks of neural networks, analogous to biological neurons.
            \item Each neuron processes inputs with weights and produces an output via an activation function.
        \end{itemize}
        \begin{block}{Mathematical Representation}
            \begin{equation}
              z = \sum (w_i \cdot x_i) + b
            \end{equation}
            Where:
            \begin{itemize}
                \item \( z \): weighted sum
                \item \( w_i \): weights of the inputs
                \item \( x_i \): inputs
                \item \( b \): bias term
            \end{itemize}
        \end{block}
    \end{enumerate}
\end{frame}

\begin{frame}[fragile]
    \frametitle{Layers in Neural Networks}
    \begin{itemize}
        \item \textbf{Input Layer}
        \begin{itemize}
            \item Receives input data directly.
            \item Each input feature corresponds to a neuron.
        \end{itemize}
        \item \textbf{Hidden Layers}
        \begin{itemize}
            \item Intermediate layers where processing occurs.
            \item Can have multiple layers, transforming inputs through learned weights and activation functions.
        \end{itemize}
        \item \textbf{Output Layer}
        \begin{itemize}
            \item Final layer producing the output.
            \item Number of neurons corresponds to output classes or target values.
        \end{itemize}
    \end{itemize}
\end{frame}

\begin{frame}[fragile]
    \frametitle{Connections and Learning Process}
    \begin{itemize}
        \item \textbf{Weights}
        \begin{itemize}
            \item Each connection has an associated weight indicating its importance.
        \end{itemize}
        \item Data travels from the input layer through hidden layers to the output layer.
        \item During training, weights and biases adjust through a process called backpropagation based on prediction errors.
    \end{itemize}
\end{frame}

\begin{frame}[fragile]
    \frametitle{Example: Digit Recognition}
    \begin{itemize}
        \item \textbf{Input Layer:} Each pixel of a 28x28 image as inputs (784 neurons).
        \item \textbf{Hidden Layers:} One or more layers, often 128 or 256 neurons depending on architecture.
        \item \textbf{Output Layer:} 10 neurons representing each digit (0-9).
    \end{itemize}
    \begin{block}{Key Points}
        \begin{itemize}
            \item Neural networks mimic brain connections with inputs, hidden transformations, and outputs.
            \item They learn by adjusting weights and biases based on prediction errors.
            \item Depth (number of layers) and breadth (number of neurons per layer) significantly impact performance.
        \end{itemize}
    \end{block}
\end{frame}

\begin{frame}
    \frametitle{Activation Functions - Definition}
    \begin{block}{Definition}
        Activation functions are mathematical equations that determine the output of a neural network node (or "neuron") based on its input. 
        They introduce non-linearities, which allow the network to learn complex patterns in data.
    \end{block}
\end{frame}

\begin{frame}
    \frametitle{Activation Functions - Common Types}
    \begin{itemize}
        \item Sigmoid Function
        \item Hyperbolic Tangent (Tanh)
        \item Rectified Linear Unit (ReLU)
    \end{itemize}
\end{frame}

\begin{frame}[fragile]
    \frametitle{Sigmoid Function}
    \begin{itemize}
        \item \textbf{Formula}: 
        \begin{equation}
        f(x) = \frac{1}{1 + e^{-x}}
        \end{equation}
        \item \textbf{Range}: (0, 1)
        \item \textbf{Characteristics}:
        \begin{itemize}
            \item Smooth gradient, making optimization easier.
            \item Output values are bounded, aiding in interpretations of probabilities.
        \end{itemize}
        \item \textbf{Use Case}: Often used in binary classification problems.
        \item \textbf{Example}: Output of a node in a logistic regression scenario.
    \end{itemize}
\end{frame}

\begin{frame}[fragile]
    \frametitle{Hyperbolic Tangent (Tanh)}
    \begin{itemize}
        \item \textbf{Formula}: 
        \begin{equation}
        f(x) = \frac{e^x - e^{-x}}{e^x + e^{-x}}
        \end{equation}
        \item \textbf{Range}: (-1, 1)
        \item \textbf{Characteristics}:
        \begin{itemize}
            \item Centered around zero for faster convergence.
            \item Stronger gradients near zero compared to the sigmoid.
        \end{itemize}
        \item \textbf{Use Case}: Typically used in hidden layers of neural networks.
        \item \textbf{Example}: Commonly used in recurrent neural networks (RNNs).
    \end{itemize}
\end{frame}

\begin{frame}[fragile]
    \frametitle{Rectified Linear Unit (ReLU)}
    \begin{itemize}
        \item \textbf{Formula}: 
        \begin{equation}
        f(x) = \max(0, x)
        \end{equation}
        \item \textbf{Range}: [0, ∞)
        \item \textbf{Characteristics}:
        \begin{itemize}
            \item Computationally efficient; requires simple thresholding.
            \item Introduces sparsity; during training, some neurons may not activate.
            \item Can suffer from the "dying ReLU" problem where neurons output zeros.
        \end{itemize}
        \item \textbf{Use Case}: Popular in convolutional neural networks and deep learning architectures.
        \item \textbf{Example}: Activation in deep layers of a CNN for image classification tasks.
    \end{itemize}
\end{frame}

\begin{frame}
    \frametitle{Key Points and Illustrative Example}
    \begin{itemize}
        \item Activation functions are crucial for neural networks to model complex relationships.
        \item Choice of activation function impacts performance, convergence rates, and learning capabilities.
    \end{itemize}
    \begin{block}{Illustrative Example}
        (Insert a diagram illustrating the shapes of the Sigmoid, Tanh, and ReLU functions, showing their outputs over a range of inputs.)
    \end{block}
\end{frame}

\begin{frame}[fragile]
    \frametitle{Example Code Snippet for ReLU}
    \begin{lstlisting}[language=Python]
import numpy as np

def relu(x):
    return np.maximum(0, x)

# Example usage
input_array = np.array([-5, 0, 5])
output_array = relu(input_array)
print(output_array)  # Outputs: [0, 0, 5]
    \end{lstlisting}
\end{frame}

\begin{frame}[fragile]
    \frametitle{Forward Propagation - Overview}
    \begin{block}{Definition}
        Forward Propagation is a fundamental concept in neural networks that explains how input data is transformed as it passes through the layers of the network to produce an output.
    \end{block}
    
    \begin{block}{Process}
        The process involves calculating the weighted sum of inputs, applying an activation function, and propagating the results forward.
    \end{block}
\end{frame}

\begin{frame}[fragile]
    \frametitle{Forward Propagation - Steps}
    \begin{enumerate}
        \item \textbf{Input Layer:} Features of the dataset, e.g., pixels in image recognition.
        \item \textbf{Weighted Sum Calculation:}
            \[
            z = w \cdot x + b
            \]
            Where:
            \begin{itemize}
                \item \( z \) is the weighted input
                \item \( w \) is the weight vector
                \item \( x \) is the input vector
                \item \( b \) is the bias term
            \end{itemize}
        \item \textbf{Activation Function:}
            \[
            a = f(z)
            \]
            Common functions include:
            \begin{itemize}
                \item \textbf{Sigmoid:} \( f(z) = \frac{1}{1 + e^{-z}} \)
                \item \textbf{ReLU:} \( f(z) = \max(0, z) \)
                \item \textbf{Tanh:} \( f(z) = \frac{e^{z} - e^{-z}}{e^{z} + e^{-z}} \)
            \end{itemize}
    \end{enumerate}
\end{frame}

\begin{frame}[fragile]
    \frametitle{Forward Propagation - Example}
    \begin{block}{Example Neural Network Setup}
        \begin{itemize}
            \item Input Layer: 2 features (e.g., \( x_1, x_2 \))
            \item Hidden Layer: 2 neurons
            \item Output Layer: 1 neuron for binary classification
        \end{itemize}
    \end{block}
    
    \begin{block}{Steps in Forward Propagation}
        \begin{itemize}
            \item \textbf{Step 1:} Input \( x = [0.5, 0.8] \)
            \item \textbf{Step 2:} Calculate weighted sums:
            \begin{itemize}
                \item Neuron 1: \( z_1 = w_{11}*0.5 + w_{12}*0.8 + b_1 \)
                \item Neuron 2: \( z_2 = w_{21}*0.5 + w_{22}*0.8 + b_2 \)
            \end{itemize}
            \item \textbf{Step 3:} Apply activation functions:
            \begin{itemize}
                \item \( a_1 = f(z_1) \)
                \item \( a_2 = f(z_2) \)
            \end{itemize}
            \item \textbf{Step 4:} Pass outputs to the output layer and receive final prediction.
        \end{itemize}
    \end{block}
\end{frame}

\begin{frame}[fragile]
    \frametitle{Forward Propagation - Key Points}
    \begin{itemize}
        \item Forward propagation is essential for moving data through the neural network.
        \item Each layer builds upon the previous layer, allowing the network to capture complex relationships in the data.
        \item Activation functions are critical for enabling networks to learn and model non-linear patterns.
    \end{itemize}
    
    \begin{block}{Visualization Suggestion}
        Consider including a diagram showing a simple neural network with labeled layers, inputs, and outputs to illustrate the forward propagation process.
    \end{block}
\end{frame}

\begin{frame}[fragile]
    \frametitle{Loss Functions - Overview}
    \begin{itemize}
        \item In machine learning, loss functions quantify how well the model's predictions match the actual data.
        \item A lower loss indicates better performance of the model.
        \item Essential during training, as the main goal is to minimize the loss.
        \item Adjust model parameters using optimization algorithms like gradient descent.
    \end{itemize}
\end{frame}

\begin{frame}[fragile]
    \frametitle{Loss Functions - Common Types}
    \begin{enumerate}
        \item \textbf{Mean Squared Error (MSE)}
            \begin{itemize}
                \item Measures the average of the squares of errors.
                \item \textbf{Formula:} 
                \begin{equation}
                    \text{MSE} = \frac{1}{n} \sum_{i=1}^{n} (y_i - \hat{y}_i)^2
                \end{equation}
                \item Example calculation provided.
            \end{itemize}
        
        \item \textbf{Binary Cross-Entropy (BCE)}
            \begin{itemize}
                \item Assesses performance in binary classification tasks.
                \item \textbf{Formula:} 
                \begin{equation}
                    \text{BCE} = -\frac{1}{N} \sum_{i=1}^{N} [y_i \log(\hat{y}_i) + (1 - y_i) \log(1 - \hat{y}_i)]
                \end{equation}
                \item Example calculation provided.
            \end{itemize}
    \end{enumerate}
\end{frame}

\begin{frame}[fragile]
    \frametitle{Loss Functions - Additional Types}
    \begin{enumerate}\setcounter{enumi}{2}
        \item \textbf{Categorical Cross-Entropy (CCE)}
            \begin{itemize}
                \item Used in multi-class classification tasks.
                \item \textbf{Formula:} 
                \begin{equation}
                    \text{CCE} = -\sum_{i=1}^{C} y_i \log(\hat{y}_i)
                \end{equation}
                \item Example calculation provided.
            \end{itemize}

        \item \textbf{Hinge Loss}
            \begin{itemize}
                \item Primarily used in "maximum-margin" classification.
                \item \textbf{Formula:} 
                \begin{equation}
                    \text{Hinge Loss} = \max(0, 1 - y_i \cdot \hat{y}_i)
                \end{equation}
                \item Example calculation provided.
            \end{itemize}
    \end{enumerate}
\end{frame}

\begin{frame}[fragile]
    \frametitle{Key Points to Emphasize}
    \begin{itemize}
        \item Choice of loss function significantly impacts model training and performance.
        \item Understanding the nature of the prediction task (regression vs. classification) is critical.
        \item Regularly monitor the loss during training to ensure effective learning and model tuning.
    \end{itemize}
\end{frame}

\begin{frame}[fragile]
    \frametitle{Backward Propagation - Definition}
    \begin{block}{Definition}
        Backward propagation, often referred to as "backpropagation," is a supervised learning algorithm primarily used for training artificial neural networks. 
        The main goal of this process is to minimize the error of the network's predictions by updating the weights in the network.
    \end{block}
\end{frame}

\begin{frame}[fragile]
    \frametitle{Backward Propagation - Overview of the Process}
    \begin{enumerate}
        \item \textbf{Forward Pass}:
        \begin{itemize}
            \item Input data is fed into the neural network, and output predictions are made.
            \item The loss function is calculated to evaluate the prediction accuracy by comparing the predicted outputs to the actual target values.
        \end{itemize}
    
        \item \textbf{Compute Loss}:
        \begin{itemize}
            \item Measure performance using a loss function (e.g., Mean Squared Error, Cross-Entropy).
            \item Indicates how far off the predictions are from the actual results.
        \end{itemize}
    
        \item \textbf{Backward Pass}:
        \begin{itemize}
            \item Calculate the gradient of the loss with respect to each weight using the chain rule of calculus.
            \item Compute gradients layer by layer by propagating the loss backward.
        \end{itemize}
    
        \item \textbf{Weight Update}:
        \begin{itemize}
            \item Update weights using Stochastic Gradient Descent (SGD):
            \begin{equation}
                w = w - \eta \cdot \frac{\partial L}{\partial w}
            \end{equation}
            \item Where \( w \) is the weight, \( \eta \) is the learning rate, and \( \frac{\partial L}{\partial w} \) is the gradient of the loss function.
        \end{itemize}
    \end{enumerate}
\end{frame}

\begin{frame}[fragile]
    \frametitle{Backward Propagation - Key Points and Example}
    \begin{block}{Key Points to Emphasize}
        \begin{itemize}
            \item \textbf{Learning Rate (\( \eta \))}: Affects convergence; too high can cause divergence, too low can slow down convergence.
            \item \textbf{Gradient Descent}: Common optimization algorithm; variants include SGD and Adam optimizer.
            \item \textbf{Chain Rule}: Fundamental for calculating gradients layer by layer during backpropagation.
        \end{itemize}
    \end{block}

    \begin{block}{Example: Simple Neural Network}
        \begin{itemize}
            \item \textbf{Input Layer}: \( x_1, x_2 \)
            \item \textbf{Hidden Layer}: Neurons \( h_1, h_2 \)
            \item \textbf{Output Layer}: Neuron \( y \)
        \end{itemize}
        \begin{itemize}
            \item \textbf{Forward Pass}: Compute outputs: 
            \begin{equation}
                h_1 = f(W_1 \cdot x + b_1), \quad 
                h_2 = f(W_2 \cdot x + b_2)
            \end{equation}
            \begin{equation}
                y_{pred} = f(W_o \cdot h + b_o)
            \end{equation}
            \item \textbf{Loss Calculation}: 
            \begin{equation}
                L = \text{Loss Function}(y_{pred}, y_{true})
            \end{equation}
            \item Compute gradients for weights and update using the learning rate.
        \end{itemize}
    \end{block}
\end{frame}

\begin{frame}[fragile]
    \frametitle{Training Neural Networks: Key Concepts}

    \begin{itemize}
        \item **Batch Size** 
        \item **Epoch** 
        \item **Learning Rate**
        \item **Overfitting**
    \end{itemize}
\end{frame}

\begin{frame}[fragile]
    \frametitle{Batch Size}

    \begin{block}{Definition}
        The batch size is the number of training examples utilized in one iteration of the training process.
    \end{block}

    \begin{block}{Importance}
        \begin{itemize}
            \item Smaller batch sizes lead to a noisier gradient estimation, enhancing generalization but requiring more updates.
            \item Larger batch sizes reduce training time but can lead to overfitting and poor generalization.
        \end{itemize}
    \end{block}

    \begin{block}{Example}
        \begin{itemize}
            \item Batch Size = 32: 32 images processed in one update of model weights.
            \item Batch Size = 256: All 256 images processed together, resulting in faster computation but less gradient noise.
        \end{itemize}
    \end{block}
\end{frame}

\begin{frame}[fragile]
    \frametitle{Epoch and Learning Rate}

    \begin{block}{Epoch}
        An epoch is one complete pass through the entire training dataset.
    \end{block}

    \begin{itemize}
        \item Training typically requires multiple epochs for convergence.
        \item Too few epochs might lead to underfitting, while too many can lead to overfitting.
    \end{itemize}

    \begin{block}{Learning Rate}
        The learning rate determines the size of the steps taken towards the minimum of the loss function during optimization.
    \end{block}

    \begin{itemize}
        \item A small learning rate may lead to slow convergence.
        \item A large learning rate can cause the model to converge too quickly to a suboptimal solution or diverge completely.
    \end{itemize}

    \begin{equation}
    w_{new} = w_{old} - \eta \cdot \nabla L(w_{old})
    \end{equation}
\end{frame}

\begin{frame}[fragile]
    \frametitle{Overfitting}

    \begin{block}{Definition}
        Overfitting occurs when the model learns not only the underlying patterns in the training data but also the noise and outliers, resulting in poor performance on unseen data.
    \end{block}

    \begin{itemize}
        \item Signs of Overfitting:
        \begin{itemize}
            \item High training accuracy but low validation/test accuracy.
        \end{itemize}
        \item Solutions:
        \begin{itemize}
            \item Techniques like dropout, early stopping, or regularization can combat overfitting.
        \end{itemize}
    \end{itemize}
\end{frame}

\begin{frame}[fragile]
    \frametitle{Conclusion and Key Points}

    \begin{itemize}
        \item Experimentation with batch size and learning rate is essential, as optimal settings vary across datasets.
        \item Regular evaluation on validation data helps detect and mitigate overfitting early in training.
        \item More epochs are not always better; assess convergence through validation accuracy or loss.
    \end{itemize}

    Understanding and tuning these parameters is crucial for effective neural network training and directly impacts model performance and efficiency.
\end{frame}

\begin{frame}[fragile]
    \frametitle{Applications of Neural Networks - Overview}
    \begin{block}{Overview}
        Neural networks are a cornerstone of modern artificial intelligence, enabling machines to perform tasks that traditionally require human intelligence. 
    \end{block}
    \begin{block}{Key Applications}
        This slide explores key applications of neural networks in various fields, showcasing their versatility and impact on real-world problems.
    \end{block}
\end{frame}

\begin{frame}[fragile]
    \frametitle{Applications of Neural Networks - Image Recognition}
    \begin{itemize}
        \item \textbf{Image Recognition}
            \begin{itemize}
                \item \textbf{Description}: Neural networks, specifically Convolutional Neural Networks (CNNs), are widely used for image classification and object detection.
                \item \textbf{Examples}:
                    \begin{itemize}
                        \item Facial recognition systems in security and social media platforms.
                        \item Automatic tagging of photos based on recognized objects or people.
                    \end{itemize}
                \item \textbf{Illustration Idea}: A diagram illustrating CNN architecture, highlighting layers like convolutional layers, pooling layers, and fully connected layers.
            \end{itemize}
    \end{itemize}
\end{frame}

\begin{frame}[fragile]
    \frametitle{Applications of Neural Networks - Natural Language Processing and Healthcare}
    \begin{itemize}
        \item \textbf{Natural Language Processing (NLP)}
            \begin{itemize}
                \item \textbf{Description}: Recurrent Neural Networks (RNNs) and Transformer models excel in understanding and generating human language.
                \item \textbf{Examples}:
                    \begin{itemize}
                        \item Chatbots and virtual assistants (like Siri and Alexa) that understand user queries.
                        \item Machine translation services (like Google Translate) that convert text from one language to another.
                    \end{itemize}
                \item \textbf{Key Technologies}: Attention mechanisms in Transformer architectures enhance context understanding, increasing translation accuracy.
            \end{itemize}

        \item \textbf{Healthcare}
            \begin{itemize}
                \item \textbf{Description}: Neural networks analyze medical data, aiding in diagnosis and treatment recommendations.
                \item \textbf{Examples}:
                    \begin{itemize}
                        \item Medical imaging analysis (detecting tumors in X-rays and MRIs).
                        \item Predictive analytics for patient health outcomes based on historical data.
                    \end{itemize}
                \item \textbf{Impact}: Improved diagnosis accuracy leads to timely intervention, potentially saving lives.
            \end{itemize}
    \end{itemize}
\end{frame}

\begin{frame}[fragile]
    \frametitle{Applications of Neural Networks - Autonomous Systems and Recommendation Systems}
    \begin{itemize}
        \item \textbf{Autonomous Systems}
            \begin{itemize}
                \item \textbf{Description}: Neural networks power self-driving cars and drones, enabling them to navigate complex environments.
                \item \textbf{Examples}:
                    \begin{itemize}
                        \item Object detection and path planning in autonomous vehicles.
                        \item Real-time decision making based on environmental cues.
                    \end{itemize}
                \item \textbf{Key Insight}: Safety and efficiency improvements as vehicles learn from vast amounts of data in simulated and real-world environments.
            \end{itemize}

        \item \textbf{Recommendation Systems}
            \begin{itemize}
                \item \textbf{Description}: Neural networks enhance user experiences by personalizing content delivery.
                \item \textbf{Examples}:
                    \begin{itemize}
                        \item Movie and music recommendation algorithms used by platforms like Netflix and Spotify.
                        \item E-commerce product recommendations based on user behavior and preferences.
                    \end{itemize}
                \item \textbf{Mechanism}: Collaborative filtering and deep learning techniques help in predicting user choices.
            \end{itemize}
    \end{itemize}
\end{frame}

\begin{frame}[fragile]
    \frametitle{Applications of Neural Networks - Key Points and Conclusion}
    \begin{itemize}
        \item \textbf{Key Points to Emphasize}
            \begin{itemize}
                \item Neural networks are adaptable and widely applicable across various fields.
                \item They outperform traditional methods in complex pattern recognition tasks.
                \item Ethical considerations and data privacy are essential in deploying these technologies.
            \end{itemize}

        \item \textbf{Conclusion}
            \begin{itemize}
                \item Neural networks are revolutionizing how we approach and solve complex problems, impacting various aspects of our daily lives.
                \item Understanding their applications is crucial for leveraging their potential responsibly.
            \end{itemize}
    \end{itemize}
\end{frame}

\begin{frame}[fragile]
    \frametitle{Challenges and Considerations - Data Requirements}
    \begin{enumerate}
        \item \textbf{Data Requirements}
        \begin{itemize}
            \item \textbf{Quality and Quantity:} 
            \begin{itemize}
                \item Neural networks need large datasets for effective learning.
                \item Poor quality data can lead to overfitting.
                \item \textit{Example:} In image recognition, a model trained on a small dataset performs poorly on new images.
            \end{itemize} 
            \item \textbf{Imbalanced Data:}
            \begin{itemize}
                \item Bias towards the majority class in unbalanced datasets.
                \item \textit{Illustration:} A pie chart showing 90\% healthy vs. 10\% diseased cases in medical datasets.
            \end{itemize}
        \end{itemize}
    \end{enumerate}
\end{frame}

\begin{frame}[fragile]
    \frametitle{Challenges and Considerations - Training Complexity}
    \begin{enumerate}
        \setcounter{enumi}{1}
        \item \textbf{Training Complexity}
        \begin{itemize}
            \item \textbf{Compute Resources:} Deep learning requires significant computational power and time.
            \item \textbf{Hyperparameter Tuning:} 
            \begin{itemize}
                \item Key to model performance but can be tedious.
                \item \textit{Tip:} Use Automated Hyperparameter Optimization techniques such as Grid Search or Random Search.
            \end{itemize}
        \end{itemize}
    \end{enumerate}
\end{frame}

\begin{frame}[fragile]
    \frametitle{Challenges and Considerations - Ethical Implications}
    \begin{enumerate}
        \setcounter{enumi}{2}
        \item \textbf{Ethical Implications}
        \begin{itemize}
            \item \textbf{Bias and Fairness:} 
            \begin{itemize}
                \item Models may learn and propagate existing biases from training data.
                \item \textit{Example:} AI recruitment tools may favor certain demographics unjustly.
            \end{itemize}
            \item \textbf{Transparency and Accountability:} 
            \begin{itemize}
                \item Neural networks are often seen as "black boxes," hindering understanding of decision-making.
                \item \textit{Illustration:} A flow chart of decision paths taken by a neural network.
            \end{itemize}
        \end{itemize}
    \end{enumerate}
    
    \vfill
    \textbf{Key Points to Emphasize:}
    \begin{itemize}
        \item Data quality and diversity are crucial.
        \item Continuous monitoring of models is essential for avoiding unintended consequences.
        \item Ethical design practices must be implemented throughout development.
    \end{itemize}
    
    \textbf{Conclusion:} 
    Addressing these challenges requires a multidisciplinary approach for responsible AI development.
\end{frame}

\begin{frame}[fragile]
    \frametitle{Conclusion - Key Takeaways}
    \begin{itemize}
        \item \textbf{Significance of Neural Networks:} Revolutionized machine learning, enabling advancements in tasks like image recognition and natural language processing.
        \item \textbf{Continuous Learning:} Systems improve over time with more data; exemplified by Transfer Learning.
        \item \textbf{Versatility:} Applicable in diverse fields such as healthcare, finance, and autonomous vehicles.
    \end{itemize}
\end{frame}

\begin{frame}[fragile]
    \frametitle{Conclusion - Importance for the Future}
    \begin{itemize}
        \item \textbf{Innovation:} Expected to drive advancements like self-driving cars and personalized medicine.
        \item \textbf{Integration with Other Technologies:} Combining neural networks with techniques like reinforcement learning can lead to robust AI solutions.
        \item \textbf{Ethical Considerations:} Deployment must address data bias, privacy, and accountability to ensure responsible AI practices.
    \end{itemize}
\end{frame}

\begin{frame}[fragile]
    \frametitle{Conclusion - Example, Formula, and Final Thoughts}
    \begin{itemize}
        \item \textbf{Example - Image Recognition:} CNNs enhance classification tasks; achieving over 95\% accuracy on datasets like ImageNet aids in various applications.
    \end{itemize}
    \begin{block}{Formula Insight}
        Neural networks rely on loss functions like the Mean Squared Error (MSE):
        \begin{equation}
            MSE = \frac{1}{n} \sum_{i=1}^{n} (y_i - \hat{y}_i)^2
        \end{equation}
        where $y_i$ is the true value and $\hat{y}_i$ is the predicted value.
    \end{block}
    \begin{itemize}
        \item \textbf{Closing Thoughts:} Embracing advanced topics will enhance technical skills and prepare for challenges ahead; continuous learning and ethical considerations are essential.
    \end{itemize}
\end{frame}


\end{document}