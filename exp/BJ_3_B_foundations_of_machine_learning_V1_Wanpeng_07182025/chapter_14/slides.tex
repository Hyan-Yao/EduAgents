\documentclass[aspectratio=169]{beamer}

% Theme and Color Setup
\usetheme{Madrid}
\usecolortheme{whale}
\useinnertheme{rectangles}
\useoutertheme{miniframes}

% Additional Packages
\usepackage[utf8]{inputenc}
\usepackage[T1]{fontenc}
\usepackage{graphicx}
\usepackage{booktabs}
\usepackage{listings}
\usepackage{amsmath}
\usepackage{amssymb}
\usepackage{xcolor}
\usepackage{tikz}
\usepackage{pgfplots}
\pgfplotsset{compat=1.18}
\usetikzlibrary{positioning}
\usepackage{hyperref}

% Custom Colors
\definecolor{myblue}{RGB}{31, 73, 125}
\definecolor{mygray}{RGB}{100, 100, 100}
\definecolor{mygreen}{RGB}{0, 128, 0}
\definecolor{myorange}{RGB}{230, 126, 34}
\definecolor{mycodebackground}{RGB}{245, 245, 245}

% Set Theme Colors
\setbeamercolor{structure}{fg=myblue}
\setbeamercolor{frametitle}{fg=white, bg=myblue}
\setbeamercolor{title}{fg=myblue}
\setbeamercolor{section in toc}{fg=myblue}
\setbeamercolor{item projected}{fg=white, bg=myblue}
\setbeamercolor{block title}{bg=myblue!20, fg=myblue}
\setbeamercolor{block body}{bg=myblue!10}
\setbeamercolor{alerted text}{fg=myorange}

% Set Fonts
\setbeamerfont{title}{size=\Large, series=\bfseries}
\setbeamerfont{frametitle}{size=\large, series=\bfseries}
\setbeamerfont{caption}{size=\small}
\setbeamerfont{footnote}{size=\tiny}

% Footer and Navigation Setup
\setbeamertemplate{footline}{
  \leavevmode%
  \hbox{%
  \begin{beamercolorbox}[wd=.3\paperwidth,ht=2.25ex,dp=1ex,center]{author in head/foot}%
    \usebeamerfont{author in head/foot}\insertshortauthor
  \end{beamercolorbox}%
  \begin{beamercolorbox}[wd=.5\paperwidth,ht=2.25ex,dp=1ex,center]{title in head/foot}%
    \usebeamerfont{title in head/foot}\insertshorttitle
  \end{beamercolorbox}%
  \begin{beamercolorbox}[wd=.2\paperwidth,ht=2.25ex,dp=1ex,center]{date in head/foot}%
    \usebeamerfont{date in head/foot}
    \insertframenumber{} / \inserttotalframenumber
  \end{beamercolorbox}}%
  \vskip0pt%
}

% Turn off navigation symbols
\setbeamertemplate{navigation symbols}{}

% Title Page Information
\title[Course Reflection]{Week 14: Project Presentations and Course Reflection}
\author[J. Smith]{John Smith, Ph.D.}
\institute[University Name]{
  Department of Computer Science\\
  University Name\\
  \vspace{0.3cm}
  Email: email@university.edu\\
  Website: www.university.edu
}
\date{\today}

% Document Start
\begin{document}

\frame{\titlepage}

\begin{frame}[fragile]
    \frametitle{Introduction to Project Presentations}
    \begin{block}{Overview of Objectives for Week 14}
        It's essential to focus on effective sharing and discussion of your project findings. This week, our key objectives include:
    \end{block}
\end{frame}

\begin{frame}[fragile]
    \frametitle{Objectives: Showcasing Your Work}
    \begin{enumerate}
        \item \textbf{Purpose:} Present the results and insights from your projects to your peers.
        \item \textbf{Expectations:} Each team or individual will deliver a concise presentation highlighting:
        \begin{itemize}
            \item Research question
            \item Methodology
            \item Findings
            \item Implications
        \end{itemize}
        \item \textbf{Example:} Cover your data gathering, models utilized, and accuracy of predictions.
    \end{enumerate}
\end{frame}

\begin{frame}[fragile]
    \frametitle{Objectives: Fostering Peer Discussion}
    \begin{enumerate}
        \item \textbf{Purpose:} Engage in discussions for constructive feedback and idea exchange.
        \item \textbf{Expectations:} Designated Q\&A session after each presentation.
        \item \textbf{Example:} Peers may ask, "What alternative models did you consider?"
    \end{enumerate}
\end{frame}

\begin{frame}[fragile]
    \frametitle{Objectives: Reflecting on Learning}
    \begin{enumerate}
        \item \textbf{Purpose:} Integrate knowledge and experiences gained throughout the course.
        \item \textbf{Expectations:} Reflect on critical lessons learned:
        \begin{itemize}
            \item Successes
            \item Areas for improvement
        \end{itemize}
        \item \textbf{Example:} Discuss challenges that evolved your skills.
    \end{enumerate}
\end{frame}

\begin{frame}[fragile]
    \frametitle{Objectives: Gathering Feedback for Future Growth}
    \begin{enumerate}
        \item \textbf{Purpose:} Use insights gained from presentations to inform future work.
        \item \textbf{Expectations:} Collect feedback to identify strengths and areas for improvement.
        \item \textbf{Example:} Identify common challenges like data preprocessing in future projects.
    \end{enumerate}
\end{frame}

\begin{frame}[fragile]
    \frametitle{Key Points for Presentations}
    \begin{itemize}
        \item Prepare a clear and engaging presentation.
        \item Practice communicating findings effectively (e.g., 10-15 minutes).
        \item Use visual aids to enhance understanding without overcrowding slides.
        \item Consider audience engagement strategies, such as polling or asking questions.
    \end{itemize}
\end{frame}

\begin{frame}[fragile]
    \frametitle{Conclusion}
    Week 14 serves as a vital capstone in our course, where you will synthesize and communicate your learning. The skills practiced will benefit you in both academic settings and future professional environments. Let's embrace this opportunity to learn from each other and grow together!
\end{frame}

\begin{frame}[fragile]
    \frametitle{Technical Considerations}
    \begin{block}{Note:}
        Be prepared for technical issues during presentations and ensure backup options for your slides.
    \end{block}
\end{frame}

\begin{frame}[fragile]
    \frametitle{Project Goals - Overview}
    \begin{block}{Goals of the Final Project}
        \begin{enumerate}
            \item Demonstrate Understanding of Machine Learning Concepts
            \item Enhance Problem-Solving Skills
            \item Foster Collaboration and Communication
            \item Assess the Ethical Implications of Machine Learning
        \end{enumerate}
    \end{block}
\end{frame}

\begin{frame}[fragile]
    \frametitle{Project Goals - Key Objectives}
    \begin{itemize}
        \item \textbf{Demonstrate Understanding of Machine Learning Concepts:}
        \begin{itemize}
            \item Apply theoretical knowledge in a practical setting.
            \item Selection of appropriate algorithms and implementation of workflows.
        \end{itemize}

        \item \textbf{Enhance Problem-Solving Skills:}
        \begin{itemize}
            \item Address real-world problems using machine learning solutions (e.g., predicting housing prices).
            \item Define the problem clearly and communicate the solution effectively.
        \end{itemize}

        \item \textbf{Foster Collaboration and Communication:}
        \begin{itemize}
            \item Work in teams to improve collaboration.
            \item Regular progress meetings to share insights and boost communication skills.
        \end{itemize}

        \item \textbf{Assess the Ethical Implications of Machine Learning:}
        \begin{itemize}
            \item Understand ethical concerns, such as bias in algorithms.
            \item Discuss case studies on biased decision-making systems.
        \end{itemize}
    \end{itemize}
\end{frame}

\begin{frame}[fragile]
    \frametitle{Significance of the Project}
    \begin{block}{Practical Applications}
        \begin{itemize}
            \item \textbf{Application of Concepts:} Synthesize and apply machine learning knowledge.
            \item \textbf{Portfolio Building:} Provide tangible evidence of skills for future opportunities.
        \end{itemize}
    \end{block}

    \begin{block}{Conclusion}
        Engaging with technical skills and ethical implications is essential for aspiring machine learning professionals.
    \end{block}

    \begin{block}{Engagement Prompt}
        Think about a real-world problem you are passionate about. Consider how to apply machine learning for solutions and prepare to share in upcoming presentations!
    \end{block}
\end{frame}

\begin{frame}[fragile]
    \frametitle{Presentation Guidelines - Introduction}
    \begin{block}{Key Components to Include}
        This presentation outlines essential elements for effective presentations, focusing on:
        \begin{itemize}
            \item Clarity
            \item Depth
            \item Ethical Considerations
        \end{itemize}
    \end{block}
\end{frame}

\begin{frame}[fragile]
    \frametitle{Presentation Guidelines - Clarity}
    \begin{block}{Clarity}
        \textbf{Definition:} Ensuring that the message is easily understood.
    \end{block}
    \begin{itemize}
        \item \textbf{Tips:}
        \begin{itemize}
            \item Use Simple Language: Avoid jargon and define terms when necessary.
            \item Organize Content Logically: Start with an introduction, then main points, and conclude succinctly.
            \item Visual Aids: Utilize slides, graphs, and diagrams; aim for clean layouts.
        \end{itemize}
    \end{itemize}
\end{frame}

\begin{frame}[fragile]
    \frametitle{Presentation Guidelines - Clarity Example}
    \begin{block}{Example}
        Instead of saying "The model's hyperparameters were optimized using grid search," you might say, 
        "We improved our model's performance by carefully tuning its settings, like adjusting the learning rate, through a method called grid search."
    \end{block}
\end{frame}

\begin{frame}[fragile]
    \frametitle{Presentation Guidelines - Depth}
    \begin{block}{Depth}
        \textbf{Definition:} Providing comprehensive information that shows a deep understanding of the topic.
    \end{block}
    \begin{itemize}
        \item \textbf{Tips:}
        \begin{itemize}
            \item Cover Key Concepts: Explain fundamental principles thoroughly.
            \item Use Data: Support claims with relevant statistics and examples.
            \item Address Possible Questions: Prepare to answer potential audience questions.
        \end{itemize}
    \end{itemize}
\end{frame}

\begin{frame}[fragile]
    \frametitle{Presentation Guidelines - Depth Example}
    \begin{block}{Example}
        While discussing a specific machine learning technique, briefly explain its mechanics: 
        “For instance, in decision trees, we create branches based on feature values to aid in classification.”
    \end{block}
\end{frame}

\begin{frame}[fragile]
    \frametitle{Presentation Guidelines - Ethical Considerations}
    \begin{block}{Ethical Considerations}
        \textbf{Definition:} Recognizing and addressing the ethical implications surrounding your project.
    \end{block}
    \begin{itemize}
        \item \textbf{Tips:}
        \begin{itemize}
            \item Data Privacy: Ensure compliance with relevant privacy regulations (e.g., GDPR).
            \item Bias Mitigation: Discuss measures taken to mitigate bias in datasets or models.
            \item Transparency: Be open about limitations and potential impacts.
        \end{itemize}
    \end{itemize}
\end{frame}

\begin{frame}[fragile]
    \frametitle{Presentation Guidelines - Ethical Example}
    \begin{block}{Example}
        You might state, "We anonymized all participant data to protect privacy and conducted bias assessments to ensure our model treats all demographic groups fairly."
    \end{block}
\end{frame}

\begin{frame}[fragile]
    \frametitle{Presentation Guidelines - Key Points}
    \begin{itemize}
        \item Engagement is Key: Encourage audience participation—ask questions, invite thoughts.
        \item Practice Makes Perfect: Rehearse multiple times for clarity.
        \item Feedback is Valuable: Seek peer feedback to identify areas for improvement.
    \end{itemize}
\end{frame}

\begin{frame}[fragile]
    \frametitle{Presentation Guidelines - Optional Checklist}
    \begin{itemize}
        \item[$\bullet$] Is the content organized logically?
        \item[$\bullet$] Are complex terms explained clearly?
        \item[$\bullet$] Are visual aids used effectively?
        \item[$\bullet$] Are ethical considerations thoroughly covered?
        \item[$\bullet$] Is the presentation engaging and encouraging audience interaction?
    \end{itemize}
\end{frame}

\begin{frame}[fragile]
    \frametitle{Examples of Effective Presentations - Key Strategies}
    \begin{enumerate}
        \item \textbf{Clear Structure}
            \begin{itemize}
                \item \textbf{Introduction, Body, Conclusion}
                \begin{itemize}
                    \item \textbf{Introduction}: State purpose and outline main points.
                    \item \textbf{Body}: Present key information in segments using headings or bullet points.
                    \item \textbf{Conclusion}: Summarize key takeaways and call to action.
                \end{itemize}
            \end{itemize}
        \item \textbf{Engaging Visuals}
            \begin{itemize}
                \item Use diagrams/charts to enhance understanding (e.g., flowcharts).
                \item Maintain consistent design with uniform fonts, colors, and styles.
            \end{itemize}
    \end{enumerate}
\end{frame}

\begin{frame}[fragile]
    \frametitle{Examples of Effective Presentations - Continued}
    \begin{enumerate}
        \setcounter{enumi}{2} % Set the counter to continue from previous frame
        \item \textbf{Storytelling Element}
            \begin{itemize}
                \item Use narrative techniques to make content relatable with personal experiences or case studies.
            \end{itemize}
        \item \textbf{Interactive Elements}
            \begin{itemize}
                \item Engage the audience with questions, polls, or activities.
                \item Example: Ask the audience to identify problems your project addresses.
            \end{itemize}
    \end{enumerate}
\end{frame}

\begin{frame}[fragile]
    \frametitle{Examples of Effective Presentations - Key Points to Remember}
    \begin{itemize}
        \item \textbf{Simplicity is Key}:
            \begin{itemize}
                \item Avoid overcrowding slides with text; aim for clarity.
            \end{itemize}
        \item \textbf{Speak Clearly}:
            \begin{itemize}
                \item Focus on verbal delivery; articulate and modulate your voice.
            \end{itemize}
        \item \textbf{Prepare for Questions}:
            \begin{itemize}
                \item Anticipate potential questions and prepare concise answers.
            \end{itemize}
    \end{itemize}
\end{frame}

\begin{frame}[fragile]
    \frametitle{Examples of Effective Presentations - Example Breakdown}
    \begin{itemize}
        \item \textbf{Successful Past Presentation Example}:
            \begin{itemize}
                \item \textbf{Title}: "Innovative Approaches to Sustainability"
                \item \textbf{Structure}:
                    \begin{itemize}
                        \item Clear introduction outlining sustainable practices.
                        \item Engaging visuals depicting data.
                        \item Conclusion calling for community action.
                    \end{itemize}
                \item \textbf{Techniques Used}:
                    \begin{itemize}
                        \item Informative visuals (graphs showing carbon footprint reduction).
                        \item Storytelling (personal narrative of project impact).
                        \item Audience Q\&A segment.
                    \end{itemize}
            \end{itemize}
    \end{itemize}
\end{frame}

\begin{frame}[fragile]{Peer Review Process - Overview}
    \begin{block}{Overview of Peer Review}
        The peer review process is a crucial component in academic and professional settings, where feedback from peers helps refine work. In our presentations, peer reviews enable students to assess each other's projects constructively and provide actionable insights for improvement.
    \end{block}
\end{frame}

\begin{frame}[fragile]{Peer Review Process - Steps}
    \frametitle{Peer Review Process - Steps}
    \begin{enumerate}
        \item \textbf{Presentation Delivery}: Groups present their projects, showcasing research, design, and implementation.
        
        \item \textbf{Feedback Form Distribution}: Peers receive structured feedback forms including questions on:
        \begin{itemize}
            \item Clarity of presentation
            \item Understandability of concepts
            \item Relevance and depth of content
            \item Engagement and delivery style
        \end{itemize}

        \item \textbf{Peer Review Submission}: Students fill out feedback forms and submit them anonymously.
        
        \item \textbf{Reflection on Feedback}: Presenters review and analyze the feedback to identify strengths and areas for improvement.
    \end{enumerate}
\end{frame}

\begin{frame}[fragile]{Peer Review Process - Key Points}
    \frametitle{Peer Review Process - Key Points}
    \begin{itemize}
        \item \textbf{Constructive Criticism}: Aim to offer helpful insights rather than just pointing out flaws.
        
        \item \textbf{Anonymity Encouragement}: Feedback is anonymous, reducing apprehension and promoting honesty.
        
        \item \textbf{Actionable Suggestions}: Provide specific recommendations that enhance feedback quality.
    \end{itemize}
\end{frame}

\begin{frame}[fragile]{Peer Review Process - Examples and Reflection}
    \frametitle{Peer Review Process - Examples}
    \begin{block}{Example Feedback Phrases}
        \begin{itemize}
            \item \textit{Positive Comment}: "Your visuals were engaging and supported your arguments well."
            \item \textit{Constructive Suggestion}: "Including a case study could provide a real-world example that enhances your claims."
        \end{itemize}
    \end{block}

    \begin{block}{Reflection Opportunities}
        Consider the following questions after reviewing feedback:
        \begin{itemize}
            \item What did I learn from the feedback?
            \item How can I apply this to my current project or future presentations?
            \item Did any comments resonate with previous feedback I've received?
        \end{itemize}
    \end{block}
\end{frame}

\begin{frame}[fragile]{Peer Review Process - Closing Note}
    \begin{block}{Closing Note}
        The peer review process is not just about evaluation—it’s an opportunity for collaborative growth and learning! 
        Ensure that your feedback is thoughtful and thorough to create a richer learning environment for everyone.
    \end{block}
\end{frame}

\begin{frame}[fragile]
    \frametitle{Course Reflection - Part 1}

    \begin{block}{Understanding Your Learning Journey}
        Reflect on your learning throughout this course:
        \begin{itemize}
            \item \textbf{Initial Knowledge vs. Current Understanding}:
            Assess the changes in your perceptions of machine learning.
            \item \textbf{Conceptual Growth}:
            Identify challenging concepts and strategies to overcome them.
        \end{itemize}
    \end{block}
\end{frame}

\begin{frame}[fragile]
    \frametitle{Course Reflection - Part 2}

    \begin{block}{Application of Machine Learning Concepts}
        Consider real-world applications of your knowledge:
        \begin{itemize}
            \item \textbf{Practical Applications}:
            Reflect on projects involving machine learning algorithms and insights gained.
            \item \textbf{Interdisciplinary Connections}:
            Explore how machine learning can impact fields like healthcare, finance, or environmental science.
        \end{itemize}
    \end{block}
\end{frame}

\begin{frame}[fragile]
    \frametitle{Course Reflection - Part 3}

    \begin{block}{Critical Thinking and Reflection}
        Enhance your learning with these considerations:
        \begin{itemize}
            \item \textbf{Problem-Solving}:
            Describe a challenge faced in a project and how you adapted your approach.
            \item \textbf{Feedback Utilization}:
            Discuss how peer reviews impacted your understanding of machine learning.
        \end{itemize}
    \end{block}

    \begin{block}{Key Concepts and Skills Acquired}
        Summarize foundational skills developed:
        \begin{itemize}
            \item Familiarity with algorithms (decision trees, SVMs, neural networks).
            \item Data handling skills: preprocessing, feature selection, evaluation techniques.
            \item Understanding of ethics in machine learning: fairness, accountability, transparency.
        \end{itemize}
    \end{block}
\end{frame}

\begin{frame}[fragile]
    \frametitle{Course Reflection - Part 4}

    \begin{block}{Engaging with the Material}
        Deepen your reflection by considering:
        \begin{itemize}
            \item Most surprising learning about machine learning.
            \item Envisioning the use of machine learning in your career or studies.
            \item Skills to further develop post-course.
        \end{itemize}
    \end{block}

    \begin{block}{Conclusion}
        Reflective practices solidify knowledge and clarify future goals. Document your thoughts to empower your journey in machine learning.
    \end{block}

    \begin{quote}
        "Learning is a treasure that will follow its owner everywhere." - Chinese Proverb
    \end{quote}
\end{frame}

\begin{frame}[fragile]
    \frametitle{Key Takeaways from Course - Foundational Concepts}
    \begin{block}{1. Machine Learning Fundamentals}
        \begin{itemize}
            \item \textbf{Definition}: Machine Learning (ML) is a subset of artificial intelligence enabling systems to learn from data.
            \item \textbf{Types of ML}: 
            \begin{itemize}
                \item \textbf{Supervised Learning}: Trained on labeled data (e.g., classification, regression).
                \item \textbf{Unsupervised Learning}: Identifies patterns in unlabeled data (e.g., clustering).
                \item \textbf{Reinforcement Learning}: Learns through feedback in a trial-and-error manner.
            \end{itemize}
        \end{itemize}
    \end{block}
    
    \begin{block}{2. Data Preprocessing}
        \begin{itemize}
            \item \textbf{Importance}: Crucial for model performance. Key steps include:
            \begin{itemize}
                \item \textbf{Cleaning}: Remove noise and handle missing values.
                \item \textbf{Normalization}: Scale features to a common range.
                \item \textbf{Feature Engineering}: Create new features from existing data.
            \end{itemize}
        \end{itemize}
    \end{block}
    
    \begin{block}{3. Model Evaluation and Validation}
        \begin{itemize}
            \item \textbf{Metrics}: Accuracy, precision, recall, F1 score, ROC-AUC assess model effectiveness rigorously.
            \item \textbf{Cross-Validation}: Assesses model performance on independent datasets to avoid overfitting.
        \end{itemize}
    \end{block}
\end{frame}

\begin{frame}[fragile]
    \frametitle{Key Takeaways from Course - Skills Acquired}
    \begin{block}{1. Programming Skills}
        \begin{itemize}
            \item Proficient in Python and libraries such as \texttt{sklearn} for implementing ML algorithms.
        \end{itemize}
    \end{block}

    \begin{block}{Example Code: Simple ML Model Implementation}
        \begin{lstlisting}[language=Python]
from sklearn.model_selection import train_test_split
from sklearn.ensemble import RandomForestClassifier
from sklearn.metrics import classification_report

# Example Code
X_train, X_test, y_train, y_test = train_test_split(X, y, test_size=0.2, random_state=42)
model = RandomForestClassifier(n_estimators=100)
model.fit(X_train, y_train)
predictions = model.predict(X_test)
print(classification_report(y_test, predictions))
        \end{lstlisting}
    \end{block}

    \begin{block}{2. Critical Thinking and Problem Solving}
        \begin{itemize}
            \item Ability to analyze datasets, formulate hypotheses, and interpret results systematically.
        \end{itemize}
    \end{block}
\end{frame}

\begin{frame}[fragile]
    \frametitle{Key Takeaways from Course - Ethical Considerations}
    \begin{block}{1. Bias in Data}
        \begin{itemize}
            \item Understand biases in training data leading to biased predictions affecting fairness.
        \end{itemize}
    \end{block}
    
    \begin{block}{2. Data Privacy}
        \begin{itemize}
            \item Importance of protecting sensitive information and adhering to regulations like GDPR.
        \end{itemize}
    \end{block}

    \begin{block}{3. Transparency and Accountability}
        \begin{itemize}
            \item Advocating for transparency in ML models to build trust with stakeholders and users.
        \end{itemize}
    \end{block}

    \begin{block}{Key Points to Emphasize}
        \begin{itemize}
            \item ML can transform industries but must be used responsibly.
            \item Continuous learning and adaptation are essential in the evolving field of ML.
            \item Understanding real-world applications is crucial for success.
        \end{itemize}
    \end{block}
\end{frame}

\begin{frame}[fragile]{Closing Remarks - Part 1}
  \begin{block}{Importance of Continuous Learning and Adaptation in Machine Learning}
    \begin{itemize}
      \item \textbf{The Evolving Nature of Machine Learning:} 
        Machine Learning (ML) is continuously evolving due to advancements in technology and research.
      \item \textbf{Lifelong Learning:} 
        Continuous education is essential for success, including formal and informal opportunities.
    \end{itemize}
  \end{block}
\end{frame}

\begin{frame}[fragile]{Closing Remarks - Part 2}
  \begin{block}{Key Concepts}
    \begin{itemize}
      \item \textbf{Adaptation to Change:} 
        Adapting to new methodologies is crucial for innovative problem-solving.
      \item \textbf{Embracing Failure and Feedback:} 
        Analyzing and learning from failures is vital for refining experiments.
      \item \textbf{Building a Growth Mindset:} 
        A growth mindset enhances resilience and encourages creativity in problem-solving.
    \end{itemize}
  \end{block}
\end{frame}

\begin{frame}[fragile]{Closing Remarks - Part 3}
  \begin{block}{Collaborative Learning and Conclusion}
    \begin{itemize}
      \item \textbf{Collaborative Learning:} 
        Projects often require teamwork to leverage diverse skills.
      \item \textbf{Conclusion:} 
        Continuous learning and agility are essential characteristics for thriving in the machine learning landscape.
    \end{itemize}
  \end{block}
  \begin{block}{Additional Resources}
    \begin{itemize}
      \item Online platforms: Coursera, edX, Kaggle
      \item Recommended books: "Deep Learning" by Ian Goodfellow, "Pattern Recognition and Machine Learning" by Christopher Bishop
      \item Conferences: NeurIPS, ICML, CVPR
    \end{itemize}
  \end{block}
\end{frame}


\end{document}