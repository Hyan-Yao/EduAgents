\documentclass[aspectratio=169]{beamer}

% Theme and Color Setup
\usetheme{Madrid}
\usecolortheme{whale}
\useinnertheme{rectangles}
\useoutertheme{miniframes}

% Additional Packages
\usepackage[utf8]{inputenc}
\usepackage[T1]{fontenc}
\usepackage{graphicx}
\usepackage{booktabs}
\usepackage{listings}
\usepackage{amsmath}
\usepackage{amssymb}
\usepackage{xcolor}
\usepackage{tikz}
\usepackage{pgfplots}
\pgfplotsset{compat=1.18}
\usetikzlibrary{positioning}
\usepackage{hyperref}

% Custom Colors
\definecolor{myblue}{RGB}{31, 73, 125}
\definecolor{mygray}{RGB}{100, 100, 100}
\definecolor{mygreen}{RGB}{0, 128, 0}
\definecolor{myorange}{RGB}{230, 126, 34}
\definecolor{mycodebackground}{RGB}{245, 245, 245}

% Set Theme Colors
\setbeamercolor{structure}{fg=myblue}
\setbeamercolor{frametitle}{fg=white, bg=myblue}
\setbeamercolor{title}{fg=myblue}
\setbeamercolor{section in toc}{fg=myblue}
\setbeamercolor{item projected}{fg=white, bg=myblue}
\setbeamercolor{block title}{bg=myblue!20, fg=myblue}
\setbeamercolor{block body}{bg=myblue!10}
\setbeamercolor{alerted text}{fg=myorange}

% Set Fonts
\setbeamerfont{title}{size=\Large, series=\bfseries}
\setbeamerfont{frametitle}{size=\large, series=\bfseries}
\setbeamerfont{caption}{size=\small}
\setbeamerfont{footnote}{size=\tiny}

% Document Start
\begin{document}

\frame{\titlepage}

\begin{frame}[fragile]
    \frametitle{Introduction to Reinforcement Learning - Overview}
    \begin{block}{What is Reinforcement Learning?}
        Reinforcement Learning (RL) is a subset of Machine Learning where an agent learns to make decisions by interacting with an environment.
        Through trial and error, it gains knowledge and optimizes its actions to achieve a specific goal.
    \end{block}
    
    \begin{itemize}
        \item **Agent**: The learner or decision-maker interacting with the environment.
        \item **Environment**: Everything the agent interacts with, providing feedback based on its actions.
        \item **State (s)**: A specific situation or configuration of the environment at a given time.
        \item **Action (a)**: Any operation the agent can execute that impacts the state of the environment.
        \item **Reward (r)**: Feedback indicating the effectiveness of an action (can be positive or negative).
    \end{itemize}
\end{frame}

\begin{frame}[fragile]
    \frametitle{Introduction to Reinforcement Learning - Learning Process}
    \begin{block}{Process of Reinforcement Learning}
        The learning process involves several key steps:
    \end{block}
    \begin{enumerate}
        \item **Initialization**: The agent starts with little or no prior knowledge of the environment.
        \item **Exploration vs. Exploitation**: Balancing exploring new actions to gather information with exploiting known high-reward actions.
        \item **Learning**: The agent updates its knowledge through interaction and feedback:
        \[
        Q(s, a) \leftarrow Q(s, a) + \alpha [r + \gamma \max_{a'} Q(s', a') - Q(s, a)]
        \]
        where:
        \begin{itemize}
            \item \(Q(s, a)\) is the action-value function,
            \item \(\alpha\) is the learning rate,
            \item \(\gamma\) is the discount factor,
            \item \(s'\) is the resultant state after taking action \(a\), and
            \item \(a'\) represents possible future actions.
        \end{itemize}
    \end{enumerate}
\end{frame}

\begin{frame}[fragile]
    \frametitle{Introduction to Reinforcement Learning - Significance}
    \begin{block}{Significance of RL in Machine Learning}
        Reinforcement Learning is pivotal in various applications:
    \end{block}
    \begin{itemize}
        \item **Autonomous Systems**: Trains systems that operate independently (e.g., self-driving cars).
        \item **Game Playing**: Achieves superhuman performance in games (e.g., AlphaGo defeating world champions).
        \item **Complex Decision Making**: Used in finance (trading strategies), healthcare (optimizing treatment plans), and more.
    \end{itemize}
    
    \begin{block}{Key Differences}
        \begin{itemize}
            \item RL differs from supervised learning as it does not rely on labeled data but learns from consequences.
            \item The decision-making process in RL involves a balance of exploring new strategies and leveraging known ones to maximize long-term rewards.
        \end{itemize}
    \end{block}
    
    \begin{block}{Closing Thought}
        Reinforcement Learning represents a dynamic and interactive approach for machines to learn and adapt, making it a cornerstone of advancements in AI.
    \end{block}
\end{frame}

\begin{frame}[fragile]
    \frametitle{Difference Between Supervised and Reinforcement Learning - Overview}
    \begin{itemize}
        \item **Learning Paradigms**:
            \begin{itemize}
                \item Supervised Learning: Mapping inputs to outputs with labeled data.
                \item Reinforcement Learning: Optimal policy learning via actions and rewards.
            \end{itemize}
        \item **Feedback Mechanisms**:
            \begin{itemize}
                \item Supervised: Direct feedback with known labels.
                \item Reinforcement: Delayed feedback from action results.
            \end{itemize}
        \item **Data Requirements**:
            \begin{itemize}
                \item Supervised: Requires large labeled datasets.
                \item Reinforcement: Generates data through interactions in the environment.
            \end{itemize}
    \end{itemize}
\end{frame}

\begin{frame}[fragile]
    \frametitle{Difference Between Supervised and Reinforcement Learning - Learning Paradigms}
    \begin{block}{Supervised Learning}
        \begin{itemize}
            \item Involves learning a mapping from inputs to outputs using labeled data.
            \item Example: Classifying emails as "spam" or "not spam" based on labeled datasets.
        \end{itemize}
    \end{block}
    
    \begin{block}{Reinforcement Learning}
        \begin{itemize}
            \item Involves learning an optimal policy for taking actions in an environment.
            \item Example: A robot learns to navigate a maze via trial and error, maximizing rewards.
        \end{itemize}
    \end{block}
\end{frame}

\begin{frame}[fragile]
    \frametitle{Difference Between Supervised and Reinforcement Learning - Key Points}
    \begin{itemize}
        \item **Feedback Nature**:
            \begin{itemize}
                \item Supervised: Immediate feedback from known labels.
                \item Reinforcement: Delayed rewards based on actions taken.
            \end{itemize}
        \item **Data Dependency**:
            \begin{itemize}
                \item Supervised: Heavily reliant on labeled datasets.
                \item Reinforcement: Generates data through exploration and learning from experiences.
            \end{itemize}
        \item **Goal Orientation**:
            \begin{itemize}
                \item Supervised: Focus on prediction accuracy.
                \item Reinforcement: Focus on maximizing long-term rewards in decision-making tasks.
            \end{itemize}
    \end{itemize}
    
    \begin{block}{Conclusion}
        Understanding these differences allows for better application of each learning technique in relevant contexts.
    \end{block}
\end{frame}

\begin{frame}[fragile]
    \frametitle{Foundational Concepts in Reinforcement Learning - Introduction}
    \begin{block}{Introduction to Core Concepts}
        Reinforcement Learning (RL) is a subfield of machine learning where an agent learns to make decisions by interacting with an environment. This interaction involves several foundational components:
    \end{block}
\end{frame}

\begin{frame}[fragile]
    \frametitle{Foundational Concepts in Reinforcement Learning - Key Components}
    \begin{enumerate}
        \item \textbf{Agent}
        \begin{itemize}
            \item \textbf{Definition}: The learner or decision-maker that takes actions to maximize cumulative rewards.
            \item \textbf{Example}: A self-driving car is an agent that learns to navigate roads.
        \end{itemize}
        
        \item \textbf{Environment}
        \begin{itemize}
            \item \textbf{Definition}: The external system the agent interacts with, providing feedback based on actions.
            \item \textbf{Example}: The environment for the self-driving car, including other vehicles and traffic signals.
        \end{itemize}
        
        \item \textbf{State}
        \begin{itemize}
            \item \textbf{Definition}: Representation of the current situation of the agent within the environment.
            \item \textbf{Example}: The current speed and location of the self-driving car.
        \end{itemize}
    \end{enumerate}
\end{frame}

\begin{frame}[fragile]
    \frametitle{Foundational Concepts in Reinforcement Learning - Actions and Rewards}
    \begin{enumerate}
        \setcounter{enumi}{3} % Resume enumeration
        \item \textbf{Action}
        \begin{itemize}
            \item \textbf{Definition}: Choices available to the agent that influence its state and the environment.
            \item \textbf{Example}: Accelerating, braking, or turning for a self-driving car.
        \end{itemize}
        
        \item \textbf{Reward}
        \begin{itemize}
            \item \textbf{Definition}: Feedback signal received after taking an action in a given state.
            \item \textbf{Example}: Positive reward for reaching a destination quickly, negative reward for running a red light.
        \end{itemize}
    \end{enumerate}

    \begin{block}{Key Interaction Flow}
        The agent observes the current state, selects an action, receives a reward, and transitions to a new state. This loop continues to guide the agent toward optimal behavior.
    \end{block}
\end{frame}

\begin{frame}[fragile]
    \frametitle{Foundational Concepts in Reinforcement Learning - Example and Summary}
    \begin{block}{Illustrative Example}
        Consider a simplified RL scenario: A robot (agent) learns to navigate a maze (environment).
        \begin{itemize}
            \item \textbf{States}: Each position in the maze.
            \item \textbf{Actions}: Move up, down, left, right.
            \item \textbf{Rewards}: Positive for reaching the exit, negative for hitting walls.
        \end{itemize}
        
        \textbf{Conceptual Diagram}:
        \begin{equation*}
            \text{State (Position in Maze)} \xrightarrow{\text{Action (Move)}} \text{State (New Position)}
        \end{equation*}
        \begin{equation*}
            \text{Feedback (Reward)}
        \end{equation*}
    \end{block}

    \begin{block}{Summary}
        Understanding these foundational concepts is crucial for delving deeper into RL. They set the stage for developing algorithms that can learn optimal policies through trial and error.
    \end{block}
\end{frame}

\begin{frame}[fragile]
    \frametitle{Types of Learning in Reinforcement Learning - Introduction}
    \begin{itemize}
        \item Reinforcement Learning (RL) agents learn through interactions with their environment.
        \item Three main learning paradigms:
        \begin{itemize}
            \item \textbf{Value-Based}
            \item \textbf{Policy-Based}
            \item \textbf{Model-Based}
        \end{itemize}
        \item Each approach has unique methodologies and applications suitable for different problems.
    \end{itemize}
\end{frame}

\begin{frame}[fragile]
    \frametitle{Types of Learning in Reinforcement Learning - Value-Based Learning}
    \begin{block}{Definition}
        Value-based methods estimate the value of state-action pairs to maximize long-term rewards.
    \end{block}
    
    \begin{block}{Key Concept}
        The \textbf{Q-Value} function \( Q(s, a) \):
        \begin{equation}
        Q(s, a) = \mathbb{E} \left[ R_t + \gamma R_{t+1} + \gamma^2 R_{t+2} + \cdots \mid S_t = s, A_t = a \right]
        \end{equation}
    \end{block}
    
    \begin{block}{Example: Q-Learning}
        The Q-values are updated through:
        \begin{equation}
        Q(s, a) \leftarrow Q(s, a) + \alpha \left( R + \gamma \max_{a'} Q(s', a') - Q(s, a) \right)
        \end{equation}
        where \( \alpha \) is the learning rate and \( \gamma \) is the discount factor.
    \end{block}
\end{frame}

\begin{frame}[fragile]
    \frametitle{Types of Learning in Reinforcement Learning - Policy-Based and Model-Based Learning}
    \begin{block}{Policy-Based Learning}
        \begin{itemize}
            \item \textbf{Definition}: Optimizes the policy directly without using value functions.
            \item \textbf{Key Concept}: The Policy \( \pi(a | s) \) gives the probability of action \( a \) in state \( s \).
            \item \textbf{Example}: Policy Gradient Methods maximize expected reward:
                \begin{equation}
                J(\theta) = \mathbb{E}_{\pi_\theta}[R]
                \end{equation}
                Update policy using:
                \begin{equation}
                \theta \leftarrow \theta + \alpha \nabla J(\theta)
                \end{equation}
        \end{itemize}
    \end{block}

    \begin{block}{Model-Based Learning}
        \begin{itemize}
            \item \textbf{Definition}: Learns a model of the environment to simulate future states and rewards.
            \item \textbf{Key Concept}: Learning transition probabilities \( P(s' | s, a) \) and reward function \( R(s, a) \).
            \item \textbf{Example}: Using historical data for transition probabilities and planning with algorithms like Monte Carlo Tree Search (MCTS).
        \end{itemize}
    \end{block}
\end{frame}

\begin{frame}[fragile]
    \frametitle{Key Points to Emphasize}
    \begin{itemize}
        \item \textbf{Value-Based}:
            \begin{itemize}
                \item Focus on action values.
                \item High sample efficiency but struggles in high-dimensional problems.
            \end{itemize}
        \item \textbf{Policy-Based}:
            \begin{itemize}
                \item Direct action selection.
                \item Flexibility in complex action spaces but can be sample inefficient.
            \end{itemize}
        \item \textbf{Model-Based}:
            \begin{itemize}
                \item Allows for planning and foresight.
                \item Requires accurate model learning, complex but powerful.
            \end{itemize}
    \end{itemize}
\end{frame}

\begin{frame}
    \titlepage
\end{frame}

\begin{frame}[fragile]
    \frametitle{Key RL Algorithms - Overview}
    \begin{itemize}
        \item Prominent algorithms in Reinforcement Learning (RL):
        \begin{enumerate}
            \item Q-Learning
            \item Deep Q-Networks (DQN)
            \item Policy Gradients
        \end{enumerate}
    \end{itemize}
\end{frame}

\begin{frame}[fragile]
    \frametitle{Key RL Algorithms - Q-Learning}
    \begin{block}{1. Q-Learning}
        \begin{itemize}
            \item \textbf{Concept:} A value-based algorithm that learns the optimal action-value function \( Q^*(s, a) \).
            \item \textbf{Update Rule:} 
            \begin{equation}
                Q(s, a) \leftarrow Q(s, a) + \alpha \left( r + \gamma \max_a Q(s', a) - Q(s, a) \right)
            \end{equation}
            where:
            \begin{itemize}
                \item \( \alpha \) = learning rate
                \item \( r \) = reward
                \item \( \gamma \) = discount factor
                \item \( s' \) = next state
            \end{itemize}
            \item \textbf{Example:} In a grid world, an agent updates its Q-values based on rewards.
        \end{itemize}
    \end{block}
\end{frame}

\begin{frame}[fragile]
    \frametitle{Key RL Algorithms - Deep Q-Networks (DQN)}
    \begin{block}{2. Deep Q-Networks (DQN)}
        \begin{itemize}
            \item \textbf{Concept:} Combines Q-learning with deep learning to approximate the Q-function.
            \item \textbf{Architecture:} A neural network outputs Q-values for all actions.
            \item \textbf{Experience Replay:} Stores past experiences to improve learning stability.
            \item \textbf{Example:} Applied in Atari games, learning from raw screen pixels.
        \end{itemize}
    \end{block}
\end{frame}

\begin{frame}[fragile]
    \frametitle{Key RL Algorithms - Policy Gradients}
    \begin{block}{3. Policy Gradients}
        \begin{itemize}
            \item \textbf{Concept:} Learns the policy \( \pi(a|s) \) directly, maximizing expected rewards.
            \item \textbf{Main Idea:} 
            \begin{equation}
                J(\theta) = \mathbb{E}_{\tau} \left[ \sum_{t=0}^{T} r_t \right]
            \end{equation}
            \item \textbf{Example:} STAGE-based methods like REINFORCE improve action selection over time.
        \end{itemize}
    \end{block}
\end{frame}

\begin{frame}[fragile]
    \frametitle{Key RL Algorithms - Key Points to Emphasize}
    \begin{itemize}
        \item \textbf{Flexibility:} Different algorithms for different environments (discrete vs. continuous).
        \item \textbf{Trade-offs:} 
        \begin{itemize}
            \item Q-learning is simpler for small environments.
            \item DQNs excel in large spaces but require tuning.
            \item Policy-gradient algorithms suit continuous actions but can be sample inefficient.
        \end{itemize}
    \end{itemize}
\end{frame}

\begin{frame}[fragile]
    \frametitle{Key RL Algorithms - Code Example}
    \begin{block}{Code Snippet Example (Q-Learning)}
    \begin{lstlisting}[style=mystyle]
import numpy as np

# Initialize Q-table
Q = np.zeros((state_space_size, action_space_size))

# Q-learning parameters
alpha = 0.1   # Learning rate
gamma = 0.99  # Discount factor

# Q-learning update
Q[state, action] += alpha * (reward + gamma * np.max(Q[next_state]) - Q[state, action])
    \end{lstlisting}
    \end{block}
\end{frame}

\begin{frame}[fragile]
    \frametitle{Key RL Algorithms - Summary}
    \begin{itemize}
        \item Understanding key RL algorithms is crucial for effective solutions.
        \item Each algorithm has strengths and weaknesses for different scenarios.
    \end{itemize}
\end{frame}

\begin{frame}[fragile]
    \frametitle{Performance Metrics in Reinforcement Learning}
    
    \textbf{Introduction:} \\
    Evaluating the performance of an RL agent is crucial for understanding its learning efficacy. This presentation focuses on three key metrics:
    \begin{itemize}
        \item Cumulative Reward
        \item Convergence Rates
        \item Overfitting
    \end{itemize}
\end{frame}

\begin{frame}[fragile]
    \frametitle{Cumulative Reward}
    
    \textbf{Definition:} \\
    The cumulative reward \( G_t \) is the total reward an agent receives over a specific time period or episode:
    \begin{equation}
        G_t = R_t + R_{t+1} + R_{t+2} + \ldots + R_T
    \end{equation}
    
    \textbf{Example:} \\
    If an agent earns 10, 5, and -2 points over three steps:
    \begin{equation}
        G_1 = 10 + 5 - 2 = 13
    \end{equation}
    
    \textbf{Key Point:} \\
    Higher cumulative rewards indicate better performance; however, reward distribution should also be considered.
\end{frame}

\begin{frame}[fragile]
    \frametitle{Convergence Rates and Overfitting}
    
    \textbf{Convergence Rates:} \\
    Refers to how quickly an RL algorithm approaches a stable policy:
    \begin{itemize}
        \item \textbf{Factors Affecting Convergence:}
        \begin{itemize}
            \item Learning Rate
            \item Exploration Strategy
        \end{itemize}
        \item \textbf{Key Point:} Faster convergence increases training efficiency and reduces costs.
    \end{itemize}
    
    \textbf{Overfitting:} \\
    Occurs when an agent performs well in training but fails to generalize:
    \begin{itemize}
        \item \textbf{Symptoms:} 
        \begin{itemize}
            \item High training reward, low testing reward
            \item Poor performance on slightly altered environments
        \end{itemize}
        \item \textbf{Prevention Strategies:}
        \begin{itemize}
            \item Regularization
            \item Diverse training environments
        \end{itemize}
    \end{itemize}
\end{frame}

\begin{frame}[fragile]
    \frametitle{Challenges in Reinforcement Learning - Introduction}
    \begin{itemize}
        \item Reinforcement Learning (RL) is a paradigm for training decision-making agents through interaction with an environment.
        \item Key challenges in developing effective RL algorithms include:
        \begin{itemize}
            \item Exploration vs. Exploitation
            \item Reward Structure Design
        \end{itemize}
    \end{itemize}
\end{frame}

\begin{frame}[fragile]
    \frametitle{Challenges in Reinforcement Learning - Exploration vs. Exploitation}
    \begin{block}{Concept Explanation}
        \begin{itemize}
            \item \textbf{Exploration:} Trying new actions to discover their effects.
            \item \textbf{Exploitation:} Leveraging known information to maximize immediate rewards.
        \end{itemize}
    \end{block}
    
    \begin{block}{The Dilemma}
        \begin{itemize}
            \item Balancing exploration and exploitation is crucial:
            \begin{itemize}
                \item Too much exploration wastes resources.
                \item Too much exploitation may prevent discovering better strategies.
            \end{itemize}
        \end{itemize}
    \end{block}
    
    \begin{block}{Key Strategies}
        \begin{itemize}
            \item \textbf{Epsilon-Greedy Strategy:} With a probability of $\epsilon$, explore; otherwise, exploit the best-known action.
            \item \textbf{Upper Confidence Bound (UCB):} Select actions based on their potential for high reward, factoring uncertainty.
        \end{itemize}
    \end{block}
\end{frame}

\begin{frame}[fragile]
    \frametitle{Challenges in Reinforcement Learning - Reward Structure Design}
    \begin{block}{Concept Explanation}
        \begin{itemize}
            \item The reward structure defines what outcomes are 'good' or 'bad' for the agent.
            \item Critical for effective learning.
        \end{itemize}
    \end{block}
    
    \begin{block}{Key Considerations}
        \begin{itemize}
            \item \textbf{Sparse Rewards:} Infrequent rewards lead to slow learning.
            \item \textbf{Shaping Rewards:} Introduce auxiliary rewards for intermediate actions to guide learning.
            \item \textbf{Negative Rewards:} Use caution with punishment to avoid discouraging necessary actions.
        \end{itemize}
    \end{block}

    \begin{block}{Conclusion}
        \begin{itemize}
            \item Understanding and addressing these challenges is crucial for developing effective RL agents.
            \item The balance between exploration, exploitation, and careful reward design impacts learning and success.
        \end{itemize}
    \end{block}
\end{frame}

\begin{frame}[fragile]
    \frametitle{Ethical Considerations in RL - Overview}
    \begin{itemize}
        \item Reinforcement Learning (RL) has diverse applications:
        \begin{itemize}
            \item Game playing
            \item Robotics
            \item Finance
            \item Healthcare
        \end{itemize}
        \item Ethical considerations are paramount:
        \begin{itemize}
            \item Focus on biases
            \item Emphasize algorithmic transparency
        \end{itemize}
    \end{itemize}
\end{frame}

\begin{frame}[fragile]
    \frametitle{Ethical Considerations in RL - Biases}
    \begin{block}{Definition}
        Bias in RL arises when learning algorithms favor certain outcomes systematically, potentially leading to unfair treatment.
    \end{block}
    \begin{block}{Sources of Bias}
        \begin{itemize}
            \item \textbf{Training Data Bias}:
                If data reflects societal biases, the model replicates these biases.
            \item \textbf{Reward Structure}:
                Rewards defined to maximize efficiency can promote biased behaviors.
        \end{itemize}
    \end{block}
    \begin{example}
        A hiring algorithm trained on biased historical data may favor male candidates, perpetuating gender inequality.
    \end{example}
\end{frame}

\begin{frame}[fragile]
    \frametitle{Ethical Considerations in RL - Algorithmic Transparency}
    \begin{block}{Definition}
        Algorithmic transparency is critical for understanding how decisions in an RL model are made.
    \end{block}
    \begin{block}{Importance}
        \begin{itemize}
            \item \textbf{Accountability}: Understanding decision processes is essential for addressing poor or biased decisions.
            \item \textbf{Trust}: Users trust RL systems when they comprehend the rationale behind decisions, particularly in sensitive domains.
        \end{itemize}
    \end{block}
    \begin{block}{Strategies for Transparency}
        \begin{itemize}
            \item Use interpretable models or explain complex models.
            \item Implement auditing mechanisms to evaluate and identify biases.
        \end{itemize}
    \end{block}
\end{frame}

\begin{frame}[fragile]
    \frametitle{Importance of Continual Learning - Introduction}
    \begin{itemize}
        \item Reinforcement Learning (RL) agents operate in dynamic environments.
        \item Continual learning is essential for maintaining performance as environments change.
        \item This slide discusses the significance of continual learning for RL agents.
    \end{itemize}
\end{frame}

\begin{frame}[fragile]
    \frametitle{Importance of Continual Learning - Key Concepts}
    \begin{block}{What is Continual Learning?}
        \begin{itemize}
            \item \textbf{Definition:} The ability to learn and adapt continuously over time using new data.
            \item \textbf{Goal:} Update knowledge as agents encounter new information or environmental changes.
        \end{itemize}
    \end{block}

    \begin{block}{Significance in Dynamic Environments}
        \begin{itemize}
            \item \textbf{Adaptability:} Adjusting policies to new conditions (e.g., self-driving cars).
            \item \textbf{Efficiency:} Building on past knowledge for faster convergence.
            \item \textbf{Avoiding Catastrophic Forgetting:} Techniques like Elastic Weight Consolidation (EWC) help retain important knowledge.
        \end{itemize}
    \end{block}
\end{frame}

\begin{frame}[fragile]
    \frametitle{Importance of Continual Learning - Example & Conclusion}
    \begin{block}{Example of Continual Learning in RL}
        \begin{lstlisting}[language=Python]
class ContinuousLearningAgent:
    def __init__(self):
        self.model = initialize_model()
        self.experience_replay = []

    def learn(self, new_experience):
        # Store new experience in memory
        self.experience_replay.append(new_experience)
        # Sample experiences to update the model
        sample_experiences = sample_from_memory(self.experience_replay)
        self.model.update(sample_experiences)

# Example of new experience in a dynamic environment
new_experience = (state, action, reward, next_state)
agent.learn(new_experience)
        \end{lstlisting}
    \end{block}

    \begin{block}{Conclusion}
        Incorporating continual learning strategies is crucial for RL agents to adapt to complex, dynamic environments.
    \end{block}
\end{frame}

\begin{frame}[fragile]
    \frametitle{Conclusion and Future Directions - Key Takeaways}
    
    \begin{enumerate}
        \item \textbf{Understanding Reinforcement Learning (RL)}:
        \begin{itemize}
            \item \textbf{Definition}: A type of machine learning where agents learn to maximize cumulative rewards by interacting with their environment.
            \item \textbf{Components}:
            \begin{itemize}
                \item \textbf{Agent}: The learner that makes decisions.
                \item \textbf{Environment}: The context where the agent operates.
                \item \textbf{Actions}: Choices made by the agent that influence the environment.
                \item \textbf{Rewards}: Feedback that helps evaluate the effectiveness of actions taken.
            \end{itemize}
        \end{itemize}

        \item \textbf{Exploration vs Exploitation}:
        \begin{itemize}
            \item Balancing exploration (trying new actions) with exploitation (leveraging known rewards) is essential.
            \item \textbf{Example}: In a grid world, the agent must choose between exploring unvisited cells or exploiting a known rewarding path.
        \end{itemize}
    \end{enumerate}
\end{frame}

\begin{frame}[fragile]
    \frametitle{Conclusion and Future Directions - Key Concepts}
    
    \begin{enumerate}[resume]
        \item \textbf{Policy and Value Functions}:
        \begin{itemize}
            \item \textbf{Policy}: A strategy that determines the next action based on the current state.
            \item \textbf{Value Function}: Estimates expected future rewards.
            \begin{itemize}
                \item State Value Function: \( V(s) = \mathbb{E}[R_t | S_t = s] \)
                \item Action Value Function: \( Q(s, a) = \mathbb{E}[R_t | S_t = s, A_t = a] \)
            \end{itemize}
        \end{itemize}
    \end{enumerate}
\end{frame}

\begin{frame}[fragile]
    \frametitle{Conclusion and Future Directions - Future Trends}
    
    \begin{enumerate}
        \item \textbf{Integrating Continual Learning}: Focus on agents that adapt to changing environments while retaining previous knowledge.
        \item \textbf{Improvement in Sample Efficiency}: Research on methods that require fewer interactions to learn effectively, such as model-based RL.
        \item \textbf{Human-Robot Collaboration}: Development of agents that work alongside humans, enhancing cooperative behavior in complex settings.
        \item \textbf{Real-World Applications}: Increasing use in finance, gaming, and autonomous vehicles for optimization and decision-making.
        \item \textbf{Ethical Considerations}: Ensure fairness and transparency in RL systems, especially in sensitive areas like healthcare.
    \end{enumerate}
    
    \textbf{Conclusion:} RL is evolving rapidly; addressing efficiency and ethical challenges will be vital for its effective application.
\end{frame}


\end{document}