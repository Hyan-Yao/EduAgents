\documentclass[aspectratio=169]{beamer}

% Theme and Color Setup
\usetheme{Madrid}
\usecolortheme{whale}
\useinnertheme{rectangles}
\useoutertheme{miniframes}

% Additional Packages
\usepackage[utf8]{inputenc}
\usepackage[T1]{fontenc}
\usepackage{graphicx}
\usepackage{booktabs}
\usepackage{listings}
\usepackage{amsmath}
\usepackage{amssymb}
\usepackage{xcolor}
\usepackage{tikz}
\usepackage{pgfplots}
\pgfplotsset{compat=1.18}
\usetikzlibrary{positioning}
\usepackage{hyperref}

% Custom Colors
\definecolor{myblue}{RGB}{31, 73, 125}
\definecolor{mygray}{RGB}{100, 100, 100}
\definecolor{mygreen}{RGB}{0, 128, 0}
\definecolor{myorange}{RGB}{230, 126, 34}
\definecolor{mycodebackground}{RGB}{245, 245, 245}

% Set Theme Colors
\setbeamercolor{structure}{fg=myblue}
\setbeamercolor{frametitle}{fg=white, bg=myblue}
\setbeamercolor{title}{fg=myblue}
\setbeamercolor{section in toc}{fg=myblue}
\setbeamercolor{item projected}{fg=white, bg=myblue}
\setbeamercolor{block title}{bg=myblue!20, fg=myblue}
\setbeamercolor{block body}{bg=myblue!10}
\setbeamercolor{alerted text}{fg=myorange}

% Set Fonts
\setbeamerfont{title}{size=\Large, series=\bfseries}
\setbeamerfont{frametitle}{size=\large, series=\bfseries}
\setbeamerfont{caption}{size=\small}
\setbeamerfont{footnote}{size=\tiny}

% Code Listing Style
\lstdefinestyle{customcode}{
  backgroundcolor=\color{mycodebackground},
  basicstyle=\footnotesize\ttfamily,
  breakatwhitespace=false,
  breaklines=true,
  commentstyle=\color{mygreen}\itshape,
  keywordstyle=\color{blue}\bfseries,
  stringstyle=\color{myorange},
  numbers=left,
  numbersep=8pt,
  numberstyle=\tiny\color{mygray},
  frame=single,
  framesep=5pt,
  rulecolor=\color{mygray},
  showspaces=false,
  showstringspaces=false,
  showtabs=false,
  tabsize=2,
  captionpos=b
}
\lstset{style=customcode}

% Custom Commands
\newcommand{\hilight}[1]{\colorbox{myorange!30}{#1}}
\newcommand{\source}[1]{\vspace{0.2cm}\hfill{\tiny\textcolor{mygray}{Source: #1}}}
\newcommand{\concept}[1]{\textcolor{myblue}{\textbf{#1}}}
\newcommand{\separator}{\begin{center}\rule{0.5\linewidth}{0.5pt}\end{center}}

% Footer and Navigation Setup
\setbeamertemplate{footline}{
  \leavevmode%
  \hbox{%
  \begin{beamercolorbox}[wd=.3\paperwidth,ht=2.25ex,dp=1ex,center]{author in head/foot}%
    \usebeamerfont{author in head/foot}\insertshortauthor
  \end{beamercolorbox}%
  \begin{beamercolorbox}[wd=.5\paperwidth,ht=2.25ex,dp=1ex,center]{title in head/foot}%
    \usebeamerfont{title in head/foot}\insertshorttitle
  \end{beamercolorbox}%
  \begin{beamercolorbox}[wd=.2\paperwidth,ht=2.25ex,dp=1ex,center]{date in head/foot}%
    \usebeamerfont{date in head/foot}
    \insertframenumber{} / \inserttotalframenumber
  \end{beamercolorbox}}%
  \vskip0pt%
}

% Turn off navigation symbols
\setbeamertemplate{navigation symbols}{}

% Title Page Information
\title[Capstone Project Presentations]{Week 14: Capstone Project Presentations}
\author[J. Smith]{John Smith, Ph.D.}
\institute[University Name]{
  Department of Computer Science\\
  University Name\\
  \vspace{0.3cm}
  Email: email@university.edu\\
  Website: www.university.edu
}
\date{\today}

% Document Start
\begin{document}

\frame{\titlepage}

\begin{frame}[fragile]
    \frametitle{Capstone Project Presentations}
    \begin{block}{Overview}
        The Capstone Project is a culmination of your learning journey, focusing on the application of reinforcement learning (RL) technologies. 
        In this session, students will present projects leveraging RL to solve real-world problems or simulate complex systems.
    \end{block}
    \begin{itemize}
        \item Demonstrate understanding of key RL concepts
        \item Showcase implementation skills
        \item Engage the audience in discussions about your work
    \end{itemize}
\end{frame}

\begin{frame}[fragile]
    \frametitle{Key Concepts of Reinforcement Learning}
    \begin{enumerate}
        \item \textbf{Reinforcement Learning Basics}
            \begin{itemize}
                \item \textbf{Agent}: The learner or decision maker (e.g., a robot or algorithm)
                \item \textbf{Environment}: The setting in which the agent operates
                \item \textbf{State (s)}: Current situation description in the environment
                \item \textbf{Action (a)}: Choices available to the agent
                \item \textbf{Reward (r)}: Feedback from the environment
            \end{itemize}
        \item \textbf{Goal of RL}
            \begin{itemize}
                \item Learn a policy ($\pi$) that maximizes the expected cumulative reward over time
            \end{itemize}
    \end{enumerate}
\end{frame}

\begin{frame}[fragile]
    \frametitle{Examples of Potential Projects}
    \begin{enumerate}
        \item \textbf{Game Playing}
            \begin{itemize}
                \item Project Idea: Develop an RL agent for games like Tic-Tac-Toe or Chess
                \item Approach: Implement Q-learning or Deep Q-Networks (DQN)
            \end{itemize}
        \item \textbf{Robotics}
            \begin{itemize}
                \item Project Idea: Use RL for a robotic arm to perform assembly tasks
                \item Approach: Utilize Proximal Policy Optimization (PPO)
            \end{itemize}
        \item \textbf{Finance}
            \begin{itemize}
                \item Project Idea: Create a trading algorithm using RL
                \item Approach: Implement a Multi-Agent RL system considering market fluctuations
            \end{itemize}
    \end{enumerate}
\end{frame}

\begin{frame}[fragile]
    \frametitle{Presentation Structure}
    \begin{enumerate}
        \item \textbf{Introduction}
            \begin{itemize}
                \item Briefly explain the problem your project addresses
                \item Highlight the relevance of RL in your project context
            \end{itemize}
        \item \textbf{Methodology}
            \begin{itemize}
                \item Discuss RL algorithms used (e.g., Q-learning, Policy Gradient)
                \item Outline the framework and tools applied (e.g., OpenAI Gym, TensorFlow, PyTorch)
            \end{itemize}
        \item \textbf{Results}
            \begin{itemize}
                \item Present key outcomes with visual aids
                \item Compare performance against baseline methods
            \end{itemize}
        \item \textbf{Conclusion}
            \begin{itemize}
                \item Share insights gained and potential future work
                \item Invite questions to engage the audience
            \end{itemize}
    \end{enumerate}
\end{frame}

\begin{frame}[fragile]
    \frametitle{Key Points to Emphasize}
    \begin{itemize}
        \item \textbf{Understanding and Application}: Demonstrate understanding of RL dynamics
        \item \textbf{Engagement}: Be prepared to answer questions about your methodology and findings
        \item \textbf{Future Directions}: Discuss scalability and evolution of RL in practical scenarios
    \end{itemize}
\end{frame}

\begin{frame}[fragile]
    \frametitle{Formulas \& Code Snippet}
    \begin{block}{Q-Learning Update Rule}
        \begin{equation}
            Q(s, a) \leftarrow Q(s, a) + \alpha \left[ r + \gamma \max_a Q(s', a) - Q(s, a) \right]
        \end{equation}
        \begin{itemize}
            \item \(\alpha\): Learning rate
            \item \(\gamma\): Discount factor
            \item \(s'\): Next state after action \(a\) in state \(s\)
        \end{itemize}
    \end{block}
    \begin{block}{Sample Python Code for Q-Learning}
    \begin{lstlisting}[language=Python]
    import numpy as np

    def q_learning(env, num_episodes, learning_rate, discount_factor):
        Q = np.zeros([env.state_space_dim, env.action_space_dim])
        for episode in range(num_episodes):
            state = env.reset()
            done = False
            while not done:
                action = np.argmax(Q[state, :] + np.random.randn(1, env.action_space_dim))
                next_state, reward, done = env.step(action)
                Q[state, action] += learning_rate * (reward + discount_factor * np.max(Q[next_state, :]) - Q[state, action])
                state = next_state
        return Q
    \end{lstlisting}
    \end{block}
\end{frame}

\begin{frame}[fragile]
    \frametitle{Learning Objectives Overview}
    As we reach the conclusion of our course and prepare for the capstone project presentations, it's essential to reflect on the key learning objectives we've achieved. Our focus throughout has been on foundational understanding and practical implementation of algorithms, particularly in reinforcement learning (RL). Below are the primary objectives we've met.
\end{frame}

\begin{frame}[fragile]
    \frametitle{1. Foundational Understanding of Reinforcement Learning}
    \begin{itemize}
        \item \textbf{Definition and Components}:
        \begin{itemize}
            \item \textbf{Reinforcement Learning}: A type of machine learning where an agent learns to make decisions by taking actions in an environment to maximize some notion of cumulative reward.
            \item \textbf{Key Components}:
            \begin{itemize}
                \item \textbf{Agent}: The learner or decision-maker.
                \item \textbf{Environment}: The context within which the agent operates.
                \item \textbf{Actions}: Choices made by the agent, influencing the environment.
                \item \textbf{Rewards}: Feedback signal from the environment, guiding the agent's learning.
            \end{itemize}
        \end{itemize}
        \item \textbf{Importance}: Understanding these fundamental components prepares students to conceptualize how RL algorithms function in practical applications.
    \end{itemize}
\end{frame}

\begin{frame}[fragile]
    \frametitle{2. Algorithm Implementation}
    \begin{itemize}
        \item \textbf{Core Algorithms}:
        \begin{itemize}
            \item \textbf{Q-Learning}: A value-based method where the agent learns the value of actions in different states.
            \item \textbf{Deep Q-Networks (DQN)}: An enhancement of Q-Learning using deep neural networks, allowing the agent to handle larger state spaces efficiently.
            \item \textbf{Policy Gradient Methods}: Approaches that optimize the policy directly rather than focusing on value functions.
        \end{itemize}
        \item \textbf{Example of Q-Learning}:
        \begin{equation}
            Q(s, a) \gets Q(s, a) + \alpha \left( r + \gamma \max_{a'} Q(s', a') - Q(s, a) \right)
        \end{equation}
        Where:
        \begin{itemize}
            \item \( s \): current state
            \item \( a \): action taken
            \item \( \alpha \): learning rate
            \item \( r \): immediate reward
            \item \( \gamma \): discount factor
            \item \( s' \): new state
        \end{itemize}
        \item \textbf{Implementation Insights}: Emphasis on coding environments such as Python with libraries like NumPy or TensorFlow to implement these algorithms effectively.
    \end{itemize}
\end{frame}

\begin{frame}[fragile]
    \frametitle{3. Problem-Solving and Critical Thinking}
    \begin{itemize}
        \item \textbf{Application of Concepts}: Students have engaged in solving complex problems using RL techniques, fostering critical thinking and innovative solutions.
        \item \textbf{Capstone Project}: They have the opportunity to apply their learning in a real-world context, showcasing creativity and problem-solving skills through the capstone project.
    \end{itemize}
    \textbf{Key Points to Emphasize}:
    \begin{itemize}
        \item Mastery of reinforcement learning fundamentals is crucial for understanding complex ML frameworks.
        \item Practical implementation of algorithms enhances coding skills and theoretical knowledge.
        \item The capstone project serves as a culmination of learning, combining theory and practice, and preparing students for future endeavors in machine learning and artificial intelligence.
    \end{itemize}
\end{frame}

\begin{frame}[fragile]
    \frametitle{Project Format - Overview}
    \begin{block}{Overview of Project Submission Formats}
        As part of the Capstone Project, you are required to submit essential deliverables that reflect your project's progress and final outcomes. This slide details the three main formats: 
        \begin{itemize}
            \item Project Proposal
            \item Progress Report
            \item Final Presentation
        \end{itemize}
    \end{block}
\end{frame}

\begin{frame}[fragile]
    \frametitle{Project Format - 1. Project Proposal}
    \begin{block}{Purpose}
        The project proposal outlines your project's objectives, methodology, and anticipated outcomes. It serves as a contract between you and your instructor, ensuring a clear plan moving forward.
    \end{block}
    
    \begin{block}{Key Components}
        \begin{itemize}
            \item \textbf{Problem Statement}: A clear and concise statement outlining the issue you aim to address.
            \item \textbf{Proposed Methods}: Description of the methodologies or technologies you plan to use.
            \item \textbf{Timeline}: Estimated schedule with key milestones.
            \item \textbf{Evaluation Metrics}: Criteria for measuring project success.
        \end{itemize}
    \end{block}
    
    \begin{block}{Example}
        Example: “To develop a model that predicts house prices based on various features such as location, size, and amenities.”
    \end{block}
\end{frame}

\begin{frame}[fragile]
    \frametitle{Project Format - 2. Progress Report}
    \begin{block}{Purpose}
        This report provides updates on your project's status and any challenges faced. It encourages reflection and adaptation to ensure alignment with project goals.
    \end{block}
    
    \begin{block}{Key Components}
        \begin{itemize}
            \item \textbf{Work Completed}: Summary of tasks achieved since the last submission.
            \item \textbf{Challenges Encountered}: Discussion of obstacles and how you plan to address them.
            \item \textbf{Next Steps}: Outline of upcoming tasks and expectations for future progress.
        \end{itemize}
    \end{block}
    
    \begin{block}{Example}
        Example: Reporting on progress with data collection and difficulties faced in acquiring accurate datasets.
    \end{block}
\end{frame}

\begin{frame}[fragile]
    \frametitle{Project Format - 3. Final Presentation}
    \begin{block}{Purpose}
        The final presentation encapsulates your entire project, showcasing your work, communicating findings, and demonstrating understanding.
    \end{block}
    
    \begin{block}{Key Components}
        \begin{itemize}
            \item \textbf{Introduction}: Briefly introduce the problem and its significance.
            \item \textbf{Methodology}: Explain your approach and design decisions.
            \item \textbf{Results}: Present findings clearly using visuals like charts and graphs.
            \item \textbf{Conclusion}: Summarize impact and suggest future work or improvements.
        \end{itemize}
    \end{block}
    
    \begin{block}{Example}
        Example: Displaying graphs comparing predicted vs. actual house prices to visually show the effectiveness of your model.
    \end{block}
\end{frame}

\begin{frame}[fragile]
    \frametitle{Key Points and Diagram Suggestion}
    \begin{block}{Key Points to Emphasize}
        \begin{itemize}
            \item \textbf{Clarity and Structure}: Aim for a logical flow in all submissions.
            \item \textbf{Engagement}: Make your presentation interesting; use visuals and engaging language.
            \item \textbf{Peer Review}: Seek feedback on your proposal and reports before final submission.
        \end{itemize}
    \end{block}

    \begin{block}{Diagram Suggestion}
        A simple timeline indicating deadlines for the proposal, progress report, and final presentation can enhance understanding.
    \end{block}
\end{frame}

\begin{frame}[fragile]
    \frametitle{Project Proposal Overview - Introduction}
    \begin{block}{Overview}
        A project proposal is a crucial document in the capstone project, serving as a roadmap that outlines the objectives, methods, and expected outcomes. It effectively communicates your project's significance and feasibility to your audience, including faculty, peers, and stakeholders.
    \end{block}
\end{frame}

\begin{frame}[fragile]
    \frametitle{Project Proposal Overview - Key Elements}
    \begin{enumerate}
        \item \textbf{Title Page}
        \item \textbf{Problem Statement}
        \item \textbf{Objectives}
        \item \textbf{Literature Review}
        \item \textbf{Proposed Methods}
        \item \textbf{Timeline}
        \item \textbf{Expected Outcomes}
        \item \textbf{Budget (if applicable)}
    \end{enumerate}
\end{frame}

\begin{frame}[fragile]
    \frametitle{Project Proposal Overview - Problem Statement and Objectives}
    \begin{block}{Problem Statement}
        \begin{itemize}
            \item \textbf{Definition}: Clearly articulate the issue or challenge your project addresses.
            \item \textbf{Example}: "Despite the prevalence of remote work, companies struggle with maintaining team cohesion and productivity."
            \item \textbf{Importance}: Justify why this problem needs addressing and whom it affects.
        \end{itemize}
    \end{block}

    \begin{block}{Objectives}
        \begin{itemize}
            \item \textbf{Definition}: Define what you aim to achieve with your project.
            \item \textbf{Example}: "To develop a digital platform that fosters better communication and collaboration among remote teams."
            \item \textbf{SMART Criteria}: Ensure objectives are Specific, Measurable, Achievable, Relevant, and Time-bound.
        \end{itemize}
    \end{block}
\end{frame}

\begin{frame}[fragile]
    \frametitle{Project Proposal Overview - Methods and Timeline}
    \begin{block}{Proposed Methods}
        \begin{itemize}
            \item \textbf{Definition}: Describe how you will achieve your objectives.
            \item \textbf{Components}:
            \begin{itemize}
                \item \textbf{Research Design}: Qualitative, quantitative, or mixed-methods.
                \item \textbf{Data Collection}: Surveys, interviews, experiments.
                \item \textbf{Data Analysis}: Statistical software, thematic analysis.
            \end{itemize}
        \end{itemize}
    \end{block}

    \begin{block}{Timeline}
        \begin{itemize}
            \item Include a schedule that maps out key milestones.
            \item \textbf{Example}: "Weeks 1-3: Literature Review; Weeks 4-6: Data Collection; Weeks 7-8: Analysis and Reporting."
        \end{itemize}
    \end{block}
\end{frame}

\begin{frame}[fragile]
    \frametitle{Project Proposal Overview - Expected Outcomes and Conclusion}
    \begin{block}{Expected Outcomes}
        \begin{itemize}
            \item Detail what you anticipate discovering or creating as a result of your project.
            \item \textbf{Example}: "This research is expected to yield insights into effective remote team dynamics and propose functional strategies for enhancement."
        \end{itemize}
    \end{block}

    \begin{block}{Conclusion}
        Creating a well-structured project proposal is foundational in guiding your capstone project. By thoughtfully addressing each key element, you lay the groundwork for a successful outcome. Remember, your proposal is not just a requirement; it’s a chance to showcase your ideas and passion.
    \end{block}
\end{frame}

\begin{frame}[fragile]
    \frametitle{Reminder for Students}
    \begin{itemize}
        \item \textbf{Review your proposal} thoroughly for coherence and logical flow.
        \item \textbf{Engage with peers} for feedback before final submission.
        \item \textbf{Prepare to defend} your proposal during presentations—be ready to answer questions and elaborate on your methods.
    \end{itemize}
\end{frame}

\begin{frame}[fragile]
    \frametitle{Progress Report Components - Overview}
    A progress report is a crucial document in the life cycle of a capstone project. It serves to communicate the current status of the project, outlining:
    \begin{itemize}
        \item Achievements
        \item Challenges
        \item Next Steps
    \end{itemize}
    This document keeps stakeholders informed and aids in assessing the project's trajectory toward completion.
\end{frame}

\begin{frame}[fragile]
    \frametitle{Progress Report Components - Key Elements}
    \begin{enumerate}
        \item \textbf{Introduction}
        \item \textbf{Methodology}
        \item \textbf{Initial Results}
        \item \textbf{Analysis of Methodology}
        \item \textbf{Challenges Faced}
        \item \textbf{Next Steps}
    \end{enumerate}
\end{frame}

\begin{frame}[fragile]
    \frametitle{Progress Report Components - Methodology}
    \textbf{Methodology}
    \begin{itemize}
        \item \textbf{Definition}: Describe the methods for data gathering or research.
        \item \textbf{Example}: For a survey, detail aspects like sample size and analysis tools (e.g., Python, R).
    \end{itemize}
    
    \textbf{Initial Results}
    \begin{itemize}
        \item Summarize collected data and preliminary findings.
        \item Include visual aids like tables or graphs (e.g., survey results).
    \end{itemize}
\end{frame}

\begin{frame}[fragile]
    \frametitle{Progress Report Components - Analysis and Challenges}
    \textbf{Analysis of Methodology}
    \begin{itemize}
        \item Discuss effectiveness and feasibility; note any adjustments.
        \item Example: Adjusted from online surveys to mixed methods due to low response rates.
    \end{itemize}
    
    \textbf{Challenges Faced}
    \begin{itemize}
        \item Identify obstacles, including:
        \begin{itemize}
            \item Technical difficulties
            \item Limitations in data collection
            \item Time management issues
        \end{itemize}
    \end{itemize}
\end{frame}

\begin{frame}[fragile]
    \frametitle{Progress Report Components - Next Steps & Conclusion}
    \textbf{Next Steps}
    \begin{itemize}
        \item Outline upcoming phases:
        \begin{itemize}
            \item Further data collection
            \item Analysis strategies
            \item Timelines for future milestones
        \end{itemize}
    \end{itemize}
    
    \textbf{Conclusion}
    A well-structured progress report reflects understanding and ownership of the capstone project. Make sure to regularly review these components as your project evolves toward its final presentation.
\end{frame}

\begin{frame}[fragile]
    \frametitle{Final Project Presentation - Overview}
    The final project presentation is an opportunity to:
    \begin{itemize}
        \item Showcase your research and technical execution.
        \item Communicate complex information effectively.
        \item Evaluate your methodologies and demonstrate the impact of your findings.
    \end{itemize}
\end{frame}

\begin{frame}[fragile]
    \frametitle{Final Project Presentation - Structure}
    \begin{enumerate}
        \item Introduction: Explain project purpose and research question.
        \item Methodology: Describe techniques and tools used, with rationale.
        \item Technical Execution: Present implementation details.
        \item Results: Showcase findings quantitatively.
        \item Evaluation of Results: Assess results against hypotheses.
    \end{enumerate}
\end{frame}

\begin{frame}[fragile]
    \frametitle{Final Project Presentation - Methodology Example}
    \begin{block}{Methodology}
        \begin{itemize}
            \item Describe techniques and tools (e.g., algorithms, software).
            \item Rationale for chosen methods: suitability for objectives.
        \end{itemize}
    \end{block}
    \begin{block}{Code Snippet Example}
        \begin{lstlisting}[language=Python]
import tensorflow as tf
model = tf.keras.Sequential([
    tf.keras.layers.Dense(128, activation='relu', input_shape=(input_dim,)),
    tf.keras.layers.Dense(1, activation='sigmoid')
])
model.compile(optimizer='adam', loss='binary_crossentropy', metrics=['accuracy'])
        \end{lstlisting}
    \end{block}
\end{frame}

\begin{frame}[fragile]
    \frametitle{Final Project Presentation - Results and Evaluation}
    \begin{itemize}
        \item Results should be showcased through:
        \begin{itemize}
            \item Tables, graphs, or charts.
            \item Key metrics: accuracy, precision, recall.
            \item Trends and anomalies discussion.
        \end{itemize}
        \item Evaluation of Results:
        \begin{itemize}
            \item Assess results against benchmarks.
            \item Discuss implications of findings in a wider context.
        \end{itemize}
    \end{itemize}
\end{frame}

\begin{frame}[fragile]
    \frametitle{Final Project Presentation - Evaluation Criteria}
    \begin{itemize}
        \item Clarity: Logical structure and clear explanations.
        \item Technical Depth: Thorough understanding of methodologies.
        \item Engagement: Effective use of visual aids.
    \end{itemize}
\end{frame}

\begin{frame}[fragile]
    \frametitle{Final Project Presentation - Tips}
    \begin{itemize}
        \item Practice delivery for smooth transitions.
        \item Anticipate questions and prepare your responses.
        \item Be concise: Include necessary detail while avoiding jargon.
    \end{itemize}
\end{frame}

\begin{frame}[fragile]
    \frametitle{Performance Evaluation Metrics - Overview}
    \begin{itemize}
        \item Performance evaluation metrics are critical in assessing the effectiveness of machine learning models, especially in reinforcement learning (RL).
        \item Accurate measurement of performance is essential for justifying approaches and results in your project.
        \item Key metrics include:
            \begin{itemize}
                \item \textbf{Cumulative Rewards}
                \item \textbf{Convergence Rates}
            \end{itemize}
    \end{itemize}
\end{frame}

\begin{frame}[fragile]
    \frametitle{Performance Evaluation Metrics - Cumulative Rewards}
    \begin{block}{Cumulative Rewards}
        \begin{itemize}
            \item \textbf{Definition}: Represents the total rewards received by an agent during the execution of a task.
            \item \textbf{Formula}: 
            \[
            R_t = r_t + r_{t+1} + r_{t+2} + \ldots + r_T
            \]
            where \( R_t \) is the cumulative reward from time \( t \) to \( T \), and \( r_t \) denotes the reward received at time \( t \).
            \item \textbf{Example}: If an agent receives rewards of 1, 2, and 5 at different time steps, the cumulative reward starting from the first step would be \( R = 1 + 2 + 5 = 8 \).
            \item \textbf{Importance}: Quantifies the overall payoff of the agent’s strategy, facilitating comparisons between different strategies or iterations.
        \end{itemize}
    \end{block}
\end{frame}

\begin{frame}[fragile]
    \frametitle{Performance Evaluation Metrics - Convergence Rates}
    \begin{block}{Convergence Rates}
        \begin{itemize}
            \item \textbf{Definition}: Indicates how quickly an algorithm approaches optimal performance as it learns over time. 
            \item \textbf{Characteristics}: 
            \begin{itemize}
                \item A high convergence rate signifies quicker learning and stabilization of the policy or value function.
                \item Convergence can be graphed with the number of iterations on the x-axis and the performance measure (e.g., average reward) on the y-axis.
            \end{itemize}
            \item \textbf{Example}: If an algorithm achieves an average reward of 100 after 10 iterations and 200 after 15 iterations, this suggests the algorithm's convergence speed.
            \item \textbf{Importance}: Understanding convergence rates assists in evaluating algorithm efficiency, guiding choices for project strategies.
        \end{itemize}
    \end{block}
\end{frame}

\begin{frame}[fragile]
    \frametitle{Performance Evaluation Metrics - Summary Points}
    \begin{itemize}
        \item Accurately measuring cumulative rewards helps validate the effectiveness of your project outcomes.
        \item Convergence rates provide insight into the learning efficiency of your methods, assisting in optimization.
        \item Both metrics deepen understanding of your RL approach and enhance your project's credibility in presentations.
    \end{itemize}
    \begin{block}{Visual Representation}
        \begin{itemize}
            \item Consider adding a graph showing cumulative rewards over time.
            \item Include a second graph demonstrating convergence rates for different algorithms.
        \end{itemize}
    \end{block}
\end{frame}

\begin{frame}[fragile]
    \frametitle{Ethical Considerations - Introduction}
    \begin{block}{Understanding Ethical Implications in Reinforcement Learning}
        Reinforcement Learning (RL) is a powerful machine learning paradigm where agents learn to make decisions by interacting with their environment to maximize cumulative rewards. However, deploying RL in real-world applications brings forth several ethical considerations that must be addressed.
    \end{block}
\end{frame}

\begin{frame}[fragile]
    \frametitle{Ethical Considerations - Key Ethical Concerns}
    \begin{enumerate}
        \item \textbf{Bias and Fairness}  
            \begin{itemize}
                \item RL algorithms can learn biases from data.
                \item \textit{Example:} Hiring algorithms may prefer certain demographics.
            \end{itemize}
        
        \item \textbf{Transparency and Accountability}  
            \begin{itemize}
                \item RL systems are often "black boxes," obscuring decision-making.
                \item \textit{Example:} Difficulty in understanding driving decisions in autonomous vehicles.
            \end{itemize}
        
        \item \textbf{Safety Concerns}  
            \begin{itemize}
                \item Proper management of exploration is essential to avoid harm.
                \item \textit{Example:} An RL agent might exploit game mechanics to cause harmful outcomes.
            \end{itemize}
        
        \item \textbf{Environmental Impact}  
            \begin{itemize}
                \item Training RL models can be resource-intensive.
                \item \textit{Example:} Large-scale simulations require significant electricity, leading to a higher carbon footprint.
            \end{itemize}
    \end{enumerate}
\end{frame}

\begin{frame}[fragile]
    \frametitle{Ethical Considerations - Key Points to Emphasize}
    \begin{itemize}
        \item \textbf{Mitigation of Bias:} Use diverse training datasets and fairness-aware algorithms.
        \item \textbf{Promoting Transparency:} Develop interpretable models and guidelines for accountability.
        \item \textbf{Ensuring Safety:} Incorporate safety constraints and conduct thorough robustness testing.
        \item \textbf{Sustainability Practices:} Optimize energy consumption during training and consider renewable energy options.
    \end{itemize}

    \begin{block}{Final Thoughts}
        When working on projects, it is crucial to consider these ethical implications to ensure positive contributions to society. Documenting mitigation strategies in project proposals is recommended.
    \end{block}
\end{frame}

\begin{frame}[fragile]
    \frametitle{Real-World Applications of Reinforcement Learning}
    
    \begin{block}{Introduction to Reinforcement Learning (RL)}
        Reinforcement Learning is a computational approach where agents learn to make decisions by interacting with their environment to achieve a goal.
        The agent receives feedback in the form of rewards or penalties based on its actions, facilitating learning through trial and error.
    \end{block}
\end{frame}

\begin{frame}[fragile]
    \frametitle{Key Real-World Applications - Part 1}
    
    \begin{enumerate}
        \item \textbf{Autonomous Vehicles:}
        \begin{itemize}
            \item RL is utilized in self-driving cars to optimize real-time decision making based on sensor data.
            \item \textit{Example:} Google’s Waymo employs RL to discern when to accelerate, brake, or turn, ensuring efficient navigation through traffic.
        \end{itemize}
        
        \item \textbf{Robotics:}
        \begin{itemize}
            \item RL enables robots to learn complex tasks through experiential learning.
            \item \textit{Example:} Boston Dynamics uses RL to help its robots walk, run, and navigate obstacles autonomously.
        \end{itemize}
        
        \item \textbf{Healthcare:}
        \begin{itemize}
            \item RL can optimize treatment plans or medication dosages based on patient responses.
            \item \textit{Example:} Researchers are using RL for personalized diabetes management by adjusting insulin delivery based on glucose levels.
        \end{itemize}
    \end{enumerate}
\end{frame}

\begin{frame}[fragile]
    \frametitle{Key Real-World Applications - Part 2}

    \begin{enumerate}
        \setcounter{enumi}{3} % To continue numbering
        \item \textbf{Finance:}
        \begin{itemize}
            \item RL algorithms optimize trading strategies and portfolio management by learning from market conditions.
            \item \textit{Example:} Investment firms model market movements and adjust asset allocations dynamically to maximize returns.
        \end{itemize}

        \item \textbf{Game Playing:}
        \begin{itemize}
            \item RL was famously used by OpenAI’s Dota 2 bot and DeepMind’s AlphaGo to excel in complex games.
            \item \textit{Example:} AlphaGo defeated human champions by learning thousands of strategies through self-play.
        \end{itemize}

        \item \textbf{Energy Management:}
        \begin{itemize}
            \item In smart grids, RL optimizes resource allocation for energy consumption and distribution.
            \item \textit{Example:} Utilities use RL to predict energy demands and adjust supply in real-time, reducing waste and costs.
        \end{itemize}
    \end{enumerate}
\end{frame}

\begin{frame}[fragile]
    \frametitle{Student Presentations - Format and Expectations}
    \begin{block}{Overview}
        The capstone project presentations are the culmination of students' hard work and academic growth. 
        This is an opportunity to showcase projects, demonstrate understanding, and receive constructive feedback.
    \end{block}
\end{frame}

\begin{frame}[fragile]
    \frametitle{Presentation Structure}
    \begin{enumerate}
        \item \textbf{Duration:}
            \begin{itemize}
                \item Each student/group: \textbf{10-15 minutes} presentation
                \item Followed by a \textbf{5-minute Q\&A} session
            \end{itemize}
        \item \textbf{Content Requirement:}
            \begin{itemize}
                \item Introduction: Project topic, objectives, significance
                \item Background Research: Relevant literature and theories
                \item Methodology: Approach, techniques, and frameworks
                \item Results: Key findings with visuals
                \item Conclusion: Implications and future work
            \end{itemize}
    \end{enumerate}
\end{frame}

\begin{frame}[fragile]
    \frametitle{Expectations for Students}
    \begin{itemize}
        \item \textbf{Clarity and Engagement:} Communicate clearly and engage the audience.
        \item \textbf{Practice:} Rehearse multiple times for effective time management and delivery.
        \item \textbf{Know Your Audience:} Tailor explanations to the audience's technical knowledge.
    \end{itemize}
\end{frame}

\begin{frame}[fragile]
    \frametitle{Feedback Mechanisms}
    \begin{enumerate}
        \item \textbf{Peer Feedback:} Provides structured feedback on content clarity, organization, and delivery.
        \item \textbf{Instructor Evaluation:} Based on:
            \begin{itemize}
                \item Relevance of content
                \item Depth of analysis
                \item Quality of delivery
                \item Use of visuals
            \end{itemize}
        \item \textbf{Reflection:} Post-presentation reflections to identify improvement areas.
    \end{enumerate}
\end{frame}

\begin{frame}[fragile]
    \frametitle{Key Points to Emphasize}
    \begin{itemize}
        \item \textbf{Preparation is Key:} Time invested in rehearsing pays off.
        \item \textbf{Engagement Matters:} Interactive dialogues and visuals keep audiences engaged.
        \item \textbf{Constructive Feedback is Valuable:} Use feedback to grow and enhance skills.
    \end{itemize}
\end{frame}

\begin{frame}[fragile]
    \frametitle{Summary}
    The capstone project presentations showcase learning journeys and provide an opportunity to explore new ideas. Effective communication of complex concepts is vital for impactful presentations.
\end{frame}

\begin{frame}[fragile]
    \frametitle{Conclusion and Reflections - Overview}
    As we conclude the presentations, let's reflect on the key takeaways from the capstone projects. 
    \begin{itemize}
        \item Culmination of course learning experiences.
        \item Unique opportunity for students to apply theoretical knowledge.
        \item Addressing real-world challenges.
    \end{itemize}
\end{frame}

\begin{frame}[fragile]
    \frametitle{Key Takeaways - Application of Knowledge}
    \begin{enumerate}
        \item \textbf{Application of Knowledge:}
        \begin{itemize}
            \item Students leveraged concepts learned throughout the course.
            \item Examples:
            \begin{itemize}
                \item Statistical models used for trend prediction.
                \item Algorithms implemented for process optimization.
            \end{itemize}
        \end{itemize}
    \end{enumerate}
\end{frame}

\begin{frame}[fragile]
    \frametitle{Key Takeaways - Collaboration and Problem Solving}
    \begin{enumerate}
        \setcounter{enumi}{1}
        \item \textbf{Collaboration and Teamwork:}
        \begin{itemize}
            \item Importance highlighted in many projects.
            \item Diverse team contributions improve project success.
        \end{itemize}
        
        \item \textbf{Critical Thinking and Problem-Solving:}
        \begin{itemize}
            \item Innovative solutions were required to tackle challenges.
            \item Students demonstrated critical thinking and adaptability.
        \end{itemize}
    \end{enumerate}
\end{frame}

\begin{frame}[fragile]
    \frametitle{Key Takeaways - Presentation Skills and Iteration}
    \begin{enumerate}
        \setcounter{enumi}{3}
        \item \textbf{Presentation Skills:}
        \begin{itemize}
            \item Clarity in presenting complex ideas was evident.
            \item Effective communication crucial for stakeholder engagement.
            \item Example: Summary of complex data analysis using visuals.
        \end{itemize}
        
        \item \textbf{Feedback and Iteration:}
        \begin{itemize}
            \item Importance of incorporating feedback was evident.
            \item Iterative improvement resulted in stronger outcomes.
            \item Students who adapted projects following feedback saw enhancements.
        \end{itemize}
    \end{enumerate}
\end{frame}

\begin{frame}[fragile]
    \frametitle{Final Thoughts and Call to Action}
    \textbf{Final Thoughts:}
    \begin{itemize}
        \item Reflect on learning about teamwork and subject understanding.
        \item Consider future applications of the capstone experiences.
    \end{itemize}
    
    \textbf{Call to Action:}
    \begin{itemize}
        \item Each student should write a brief reflection considering:
        \begin{itemize}
            \item Challenges faced and strategies applied.
            \item Areas of success and potential improvements.
        \end{itemize}
    \end{itemize}
\end{frame}

\begin{frame}[fragile]
    \frametitle{Conclusion}
    \centering
    Thank you for your participation and hard work this semester! 
    \\
    Let's carry these lessons forward into our future experiences.
\end{frame}


\end{document}