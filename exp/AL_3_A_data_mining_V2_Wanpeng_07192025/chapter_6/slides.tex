\documentclass[aspectratio=169]{beamer}

% Theme and Color Setup
\usetheme{Madrid}
\usecolortheme{whale}
\useinnertheme{rectangles}
\useoutertheme{miniframes}

% Additional Packages
\usepackage[utf8]{inputenc}
\usepackage[T1]{fontenc}
\usepackage{graphicx}
\usepackage{booktabs}
\usepackage{listings}
\usepackage{amsmath}
\usepackage{amssymb}
\usepackage{xcolor}
\usepackage{tikz}
\usepackage{pgfplots}
\pgfplotsset{compat=1.18}
\usetikzlibrary{positioning}
\usepackage{hyperref}

% Custom Colors
\definecolor{myblue}{RGB}{31, 73, 125}
\definecolor{mygray}{RGB}{100, 100, 100}
\definecolor{mygreen}{RGB}{0, 128, 0}
\definecolor{myorange}{RGB}{230, 126, 34}
\definecolor{mycodebackground}{RGB}{245, 245, 245}

% Set Theme Colors
\setbeamercolor{structure}{fg=myblue}
\setbeamercolor{frametitle}{fg=white, bg=myblue}
\setbeamercolor{title}{fg=myblue}
\setbeamercolor{section in toc}{fg=myblue}
\setbeamercolor{item projected}{fg=white, bg=myblue}
\setbeamercolor{block title}{bg=myblue!20, fg=myblue}
\setbeamercolor{block body}{bg=myblue!10}
\setbeamercolor{alerted text}{fg=myorange}

% Set Fonts
\setbeamerfont{title}{size=\Large, series=\bfseries}
\setbeamerfont{frametitle}{size=\large, series=\bfseries}
\setbeamerfont{caption}{size=\small}
\setbeamerfont{footnote}{size=\tiny}

% Code Listing Style
\lstdefinestyle{customcode}{
  backgroundcolor=\color{mycodebackground},
  basicstyle=\footnotesize\ttfamily,
  breakatwhitespace=false,
  breaklines=true,
  commentstyle=\color{mygreen}\itshape,
  keywordstyle=\color{blue}\bfseries,
  stringstyle=\color{myorange},
  numbers=left,
  numbersep=8pt,
  numberstyle=\tiny\color{mygray},
  frame=single,
  framesep=5pt,
  rulecolor=\color{mygray},
  showspaces=false,
  showstringspaces=false,
  showtabs=false,
  tabsize=2,
  captionpos=b
}
\lstset{style=customcode}

% Custom Commands
\newcommand{\hilight}[1]{\colorbox{myorange!30}{#1}}
\newcommand{\source}[1]{\vspace{0.2cm}\hfill{\tiny\textcolor{mygray}{Source: #1}}}
\newcommand{\concept}[1]{\textcolor{myblue}{\textbf{#1}}}
\newcommand{\separator}{\begin{center}\rule{0.5\linewidth}{0.5pt}\end{center}}

% Footer and Navigation Setup
\setbeamertemplate{footline}{
  \leavevmode%
  \hbox{%
  \begin{beamercolorbox}[wd=.3\paperwidth,ht=2.25ex,dp=1ex,center]{author in head/foot}%
    \usebeamerfont{author in head/foot}\insertshortauthor
  \end{beamercolorbox}%
  \begin{beamercolorbox}[wd=.5\paperwidth,ht=2.25ex,dp=1ex,center]{title in head/foot}%
    \usebeamerfont{title in head/foot}\insertshorttitle
  \end{beamercolorbox}%
  \begin{beamercolorbox}[wd=.2\paperwidth,ht=2.25ex,dp=1ex,center]{date in head/foot}%
    \usebeamerfont{date in head/foot}
    \insertframenumber{} / \inserttotalframenumber
  \end{beamercolorbox}}%
  \vskip0pt%
}

% Turn off navigation symbols
\setbeamertemplate{navigation symbols}{}

% Title Page Information
\title[Week 6: Random Forests]{Week 6: Random Forests}
\author[J. Smith]{John Smith, Ph.D.}
\institute[University Name]{Department of Computer Science\\University Name\\Email: email@university.edu\\Website: www.university.edu}
\date{\today}

% Document Start
\begin{document}

\frame{\titlepage}

\begin{frame}[fragile]
    \titlepage
\end{frame}

\begin{frame}[fragile]
    \frametitle{Overview of Ensemble Methods}
    \begin{block}{Definition}
        Ensemble methods are powerful techniques in machine learning that combine multiple models to improve performance and robustness. The idea is that aggregating the predictions of several models leads to more accurate outcomes.
    \end{block}
    
    \begin{itemize}
        \item \textbf{Key Types of Ensemble Methods:}
        \begin{enumerate}
            \item \textbf{Bagging (Bootstrap Aggregating):}
                \begin{itemize}
                    \item Creates multiple training dataset versions through random sampling (with replacement).
                    \item Trains each model (e.g., decision trees) on different subsets and averages their predictions.
                    \item \textit{Example}: Random Forests use bagging to combine predictions.
                \end{itemize}
            \item \textbf{Boosting:}
                \begin{itemize}
                    \item Models are trained sequentially, with each focusing on previous errors.
                    \item Predictions are combined by weighting to improve accuracy.
                    \item \textit{Example}: AdaBoost and Gradient Boosting.
                \end{itemize}
        \end{enumerate}
    \end{itemize}
\end{frame}

\begin{frame}[fragile]
    \frametitle{Significance of Random Forests in Data Mining}
    \begin{block}{Key Advantages}
        Random Forests offer pivotal advantages in classification and regression tasks:
    \end{block}
    \begin{itemize}
        \item \textbf{High Accuracy:} Outperform single models due to aggregated predictions.
        \item \textbf{Robustness:} Less sensitive to outliers and noise in datasets.
        \item \textbf{Feature Importance:} Identifies influential features, aiding in model interpretability.
        \item \textbf{Versatility:} Applicable to both classification and regression problems.
    \end{itemize}
\end{frame}

\begin{frame}[fragile]
    \frametitle{Example of Random Forests in Action}
    \begin{block}{Case Study: Loan Default Prediction}
        Imagine a bank predicting loan default using a Random Forest:
    \end{block}
    \begin{enumerate}
        \item \textbf{Step 1:} Create multiple decision trees from random samples of applicant data.
        \item \textbf{Step 2:} Each tree predicts default status based on factors like income, credit score, and employment history.
        \item \textbf{Step 3:} Final prediction is made by a majority vote among all trees.
    \end{enumerate}
    This approach ensures even with some misclassifications, the model achieves better accuracy overall.
\end{frame}

\begin{frame}[fragile]
    \frametitle{Key Points and Conclusion}
    \begin{itemize}
        \item Ensemble methods like Random Forests leverage collective learning.
        \item Random Forests are effective in high-dimensional spaces and with missing values.
        \item Feature importance insights support informed decisions in feature selection.
    \end{itemize}
    
    \begin{block}{Conclusion}
        Random Forests are a cornerstone of machine learning, offering robustness, accuracy, and interpretability. Understanding these concepts helps appreciate the impact of ensemble methods in real-world scenarios.
    \end{block}
\end{frame}

\begin{frame}[fragile]
    \frametitle{Understanding Ensemble Methods - Part 1}
    \begin{block}{Definition of Ensemble Methods}
        Ensemble methods are machine learning techniques where multiple models, often referred to as “learners,” are combined to solve a particular problem. The idea is to create a robust predictive model by leveraging the strengths of various algorithms.
    \end{block}
    
    \begin{block}{Purpose of Ensemble Methods}
        \begin{itemize}
            \item \textbf{Improved Accuracy:} Reduce overfitting and increase prediction accuracy by using multiple models.
            \item \textbf{White Noise Reduction:} Mitigate individual model errors by averaging predictions, leading to a more stable output.
            \item \textbf{Robustness:} Perform well even when individual models struggle on certain data subsets.
        \end{itemize}
    \end{block}
\end{frame}

\begin{frame}[fragile]
    \frametitle{Understanding Ensemble Methods - Part 2}
    \begin{block}{Focus Techniques: Bagging and Boosting}
        \begin{enumerate}
            \item \textbf{Bagging (Bootstrap Aggregating)}
                \begin{itemize}
                    \item \textbf{Definition:} Train multiple instances of the same algorithm on different subsets of training data obtained through bootstrapping.
                    \item \textbf{How it Works:}
                        \begin{itemize}
                            \item Create multiple bootstrapped datasets.
                            \item Train an individual model on each dataset.
                            \item Aggregate predictions (e.g., voting for classification, averaging for regression).
                        \end{itemize}
                    \item \textbf{Example:} Random Forests, where many decision trees are trained on bootstrapped samples.
                \end{itemize}

            \item \textbf{Boosting}
                \begin{itemize}
                    \item \textbf{Definition:} Combines weak learners to create a strong learner by adjusting instance weights based on prior model errors.
                    \item \textbf{How it Works:}
                        \begin{itemize}
                            \item Train a model and evaluate its performance.
                            \item Adjust weights of misclassified instances for the next model.
                            \item Sequentially add models to improve overall performance.
                        \end{itemize}
                    \item \textbf{Example:} AdaBoost, where each subsequent model corrects mistakes by highlighting misclassified observations.
                \end{itemize}
        \end{enumerate}
    \end{block}
\end{frame}

\begin{frame}[fragile]
    \frametitle{Understanding Ensemble Methods - Part 3}
    \begin{block}{Key Points to Emphasize}
        \begin{itemize}
            \item Ensemble methods tackle complex data and improve model performance.
            \item Bagging and boosting serve distinct purposes and can be applied based on the problem.
            \item Random Forests utilize bagging; boosting focuses on learning from errors iteratively.
        \end{itemize}
    \end{block}

    \begin{block}{Illustrative Summary}
        \begin{itemize}
            \item \textbf{Bagging:} Train on random subsets → Average predictions → Robustness
            \item \textbf{Boosting:} Focus on errors → Add models sequentially → High accuracy
        \end{itemize}
    \end{block}

    \begin{block}{Conclusion}
        Ensemble methods, through strategies like bagging and boosting, leverage multiple algorithms for improved accuracy, stability, and robustness in machine learning models.
    \end{block}
\end{frame}

\begin{frame}[fragile]
    \frametitle{What are Random Forests?}
    \begin{block}{Overview}
        Random Forests is a powerful ensemble learning algorithm primarily used for classification and regression tasks. 
        It operates by constructing multiple decision trees during training and aggregating their outputs to enhance prediction accuracy and reduce overfitting.
    \end{block}
\end{frame}

\begin{frame}[fragile]
    \frametitle{Key Concepts of Random Forests}
    \begin{itemize}
        \item \textbf{Ensemble Learning}:
            \begin{itemize}
                \item Combines predictions from multiple decision trees to improve performance and reduce biases.
            \end{itemize}
        \item \textbf{Decision Trees}:
            \begin{itemize}
                \item A flowchart-like structure that makes decisions based on input attributes.
                \item Each internal node represents a feature, each branch a decision rule, and each leaf an outcome.
            \end{itemize}
        \item \textbf{Bootstrap Aggregating (Bagging)}:
            \begin{itemize}
                \item Involves training each decision tree on different data subsets using random sampling with replacement.
                \item This introduces diversity, helping reduce variance.
            \end{itemize}
    \end{itemize}
\end{frame}

\begin{frame}[fragile]
    \frametitle{Construction of Random Forests}
    \begin{enumerate}
        \item \textbf{Data Sampling}:
            \begin{itemize}
                \item Randomly select 'n' samples from the training dataset using bootstrapping.
            \end{itemize}
        \item \textbf{Building Trees}:
            \begin{itemize}
                \item Construct a decision tree for each sampled dataset.
                \item Use a random subset of features at each node to prevent overfitting.
            \end{itemize}
        \item \textbf{Aggregating Predictions}:
            \begin{itemize}
                \item For classification, select the class with the majority votes.
                \item For regression, compute the average of predictions from all trees.
            \end{itemize}
    \end{enumerate}
\end{frame}

\begin{frame}[fragile]
    \frametitle{Example: Email Classification}
    \begin{block}{Scenario}
        Suppose we are building a model to classify emails as "spam" or "not spam":
    \end{block}
    \begin{itemize}
        \item Collect a labeled dataset of emails categorized as spam or not spam.
        \item Create multiple subsets using bootstrapping.
        \item Construct decision trees using different subsets, focusing on random feature sets (e.g., word count, presence of links).
        \item Aggregate predictions: if 70\% of the trees classify an email as spam, classify it as spam.
    \end{itemize}
\end{frame}

\begin{frame}[fragile]
    \frametitle{Key Points to Emphasize}
    \begin{itemize}
        \item \textbf{Robustness}:
            \begin{itemize}
                \item Less susceptible to noise and overfitting compared to individual decision trees.
            \end{itemize}
        \item \textbf{Flexibility}:
            \begin{itemize}
                \item Can handle both numerical and categorical data.
                \item Models complex relationships effectively.
            \end{itemize}
        \item \textbf{Feature Importance}:
            \begin{itemize}
                \item Provides insights into important features for predictions, aiding feature selection.
            \end{itemize}
    \end{itemize}
\end{frame}

\begin{frame}[fragile]
    \frametitle{Summary Formula}
    \begin{block}{Final Prediction}
        The final predictions can be summarized as follows:
    \end{block}
    \begin{equation}
        P = \text{majority\_vote}(T_1, T_2, ..., T_n) \quad \text{(for classifications)}
    \end{equation}
    \begin{equation}
        P = \frac{1}{n} \sum_{i=1}^{n} T_i \quad \text{(for regressions)}
    \end{equation}
    Where \(T_i\) is the prediction from the \(i^{th}\) decision tree and \(P\) is the final prediction.
\end{frame}

\begin{frame}[fragile]
    \frametitle{How Random Forests Work - Overview}
    \begin{block}{What is Random Forest?}
        Random Forest is an ensemble learning technique that constructs multiple decision trees during training. The output is based on:
        \begin{itemize}
            \item Mode of classes for classification
            \item Mean prediction for regression
        \end{itemize}
    \end{block}
\end{frame}

\begin{frame}[fragile]
    \frametitle{How Random Forests Function}
    \begin{enumerate}
        \item \textbf{Data Bootstrapping:} 
        \begin{itemize}
            \item Multiple subsets of the original dataset are created by sampling with replacement.
            \item Example: From 100 samples, several subsets with duplicates may be formed.
        \end{itemize}
        
        \item \textbf{Building Decision Trees:}
        \begin{itemize}
            \item Each bootstrapped dataset produces a decision tree.
            \item Randomly select features for each split to ensure tree diversity.
        \end{itemize}
    \end{enumerate}
\end{frame}

\begin{frame}[fragile]
    \frametitle{Key Processes in Random Forests}
    \begin{enumerate}[resume]
        \item \textbf{Tree Predictions:}
        \begin{itemize}
            \item Each tree predicts a class (classification) or value (regression).
            \item Example: 5 trees predict email as Spam or Not Spam.
            \begin{itemize}
                \item Tree 1: Spam
                \item Tree 2: Not Spam
                \item Tree 3: Spam
                \item Tree 4: Spam
                \item Tree 5: Not Spam
                \item \textbf{Final Prediction:} Spam (3 votes for Spam).
            \end{itemize}
        \end{itemize}

        \item \textbf{Aggregation:}
        \begin{itemize}
            \item Combine predictions from all trees to enhance accuracy and reduce variance.
        \end{itemize}
    \end{enumerate}
\end{frame}

\begin{frame}[fragile]
    \frametitle{Formulas and Key Benefits}
    \begin{block}{Formulas}
        \begin{equation}
        \text{Predicted Class} = \text{mode}(\text{Predictions from all Trees}) 
        \end{equation}
        
        \begin{equation}
        \text{Predicted Value} = \frac{1}{N} \sum_{i=1}^{N} \text{Prediction}_i
        \end{equation}
    \end{block}

    \begin{block}{Key Benefits}
        \begin{itemize}
            \item Improved accuracy through averaging.
            \item Reduced overfitting using random selections.
        \end{itemize}
    \end{block}
\end{frame}

\begin{frame}[fragile]
    \frametitle{Real-World Application Example}
    \begin{block}{Example in Agriculture}
        Random Forests can predict crop yields based on environmental factors. 
        \begin{itemize}
            \item Multiple trees trained on different data aspects (e.g., soil type, rainfall).
            \item Provides a robust yield forecast by aggregating diverse predictions.
        \end{itemize}
    \end{block}
\end{frame}

\begin{frame}[fragile]
    \frametitle{Advantages of Random Forests - Overview}
    \begin{itemize}
        \item Random Forests are an ensemble learning method that combines predictions from multiple decision trees.
        \item They enhance accuracy and reduce the risk of overfitting.
        \item Their unique structure leads to several compelling advantages that contribute to their popularity in machine learning.
    \end{itemize}
\end{frame}

\begin{frame}[fragile]
    \frametitle{Advantages of Random Forests - Key Advantages}
    \begin{enumerate}
        \item \textbf{Improved Accuracy}
        \begin{itemize}
            \item Aggregates predictions from many decision trees.
            \item Example: In a customer purchase prediction, accuracy can increase from 75\% to 85\% through averaging.
        \end{itemize}
        
        \item \textbf{Reduced Risk of Overfitting}
        \begin{itemize}
            \item Introduces randomness in data and features to build trees.
            \item Collective decision reduces risk of overfitting.
        \end{itemize}
        
        \item \textbf{Ability to Handle Large Datasets}
        \begin{itemize}
            \item Effective with high dimensionality and large datasets.
            \item Example: Handles thousands of gene expression datasets without extensive preprocessing.
        \end{itemize}
    \end{enumerate}
\end{frame}

\begin{frame}[fragile]
    \frametitle{Advantages of Random Forests - Continued}
    \begin{enumerate}[resume]
        \item \textbf{Variable Importance Estimation}
        \begin{itemize}
            \item Provides insights into influential features for predictions.
            \item Example: Can identify critical variables like size and location in house price predictions.
        \end{itemize}
        
        \item \textbf{Robustness to Noise}
        \begin{itemize}
            \item Less sensitive to noise compared to other models.
            \item Example: In datasets with outliers, individual trees are less affected due to the ensemble nature.
        \end{itemize}
    \end{enumerate}
\end{frame}

\begin{frame}[fragile]
    \frametitle{Advantages of Random Forests - Conclusion}
    \begin{itemize}
        \item Random Forests offer improved accuracy, reduced overfitting, and robustness.
        \item They manage large datasets effectively and provide insights on variable importance.
        \item A powerful tool in machine learning leading to better performance in complex applications.
    \end{itemize}
\end{frame}

\begin{frame}[fragile]
    \frametitle{Code Snippet for Implementation}
    \begin{lstlisting}[language=Python]
from sklearn.ensemble import RandomForestClassifier

# Create the model
model = RandomForestClassifier(n_estimators=100)

# Fit the model on training data
model.fit(X_train, y_train)

# Get feature importance
importance = model.feature_importances_
    \end{lstlisting}
\end{frame}

\begin{frame}[fragile]
    \frametitle{Key Takeaway}
    \begin{itemize}
        \item Random Forests provide a robust, accurate, and flexible approach to classification and regression.
        \item Suitable for a wide range of data science and machine learning applications.
        \item Consider the strengths of Random Forests for solving complex problems!
    \end{itemize}
\end{frame}

\begin{frame}[fragile]
    \frametitle{Limitations of Random Forests - Overview}
    Random Forests are powerful ensemble learning methods combining multiple decision trees for improved predictive accuracy. 
    While they offer significant advantages, recognizing their limitations is essential for informed data analysis.
\end{frame}

\begin{frame}[fragile]
    \frametitle{Limitations of Random Forests - Key Limitations}
    \begin{enumerate}
        \item \textbf{Complexity and Interpretability}
        \begin{itemize}
            \item Difficult to interpret due to multiple trees.
            \item Less transparent than simpler models (e.g., logistic regression).
        \end{itemize}

        \item \textbf{Sensitivity to Noisy Data}
        \begin{itemize}
            \item Prone to overfitting in datasets with irrelevant features.
            \item Noise or outliers can adversely affect tree performance.
        \end{itemize}

        \item \textbf{High Memory Usage}
        \begin{itemize}
            \item Requires significant computational resources.
            \item Inefficient for large datasets (e.g., image or genomic data).
        \end{itemize}
    \end{enumerate}
\end{frame}

\begin{frame}[fragile]
    \frametitle{Limitations of Random Forests - More Key Limitations}
    \begin{enumerate}
        \setcounter{enumi}{3}
        \item \textbf{Long Training Time}
        \begin{itemize}
            \item Creating numerous trees increases training time.
            \item Extensive model tuning can prolong preprocessing phases.
        \end{itemize}

        \item \textbf{Overfitting in Small Datasets}
        \begin{itemize}
            \item Robust against overfitting but can still fit noise in small datasets.
            \item May lead to poor validation performance.
        \end{itemize}
    \end{enumerate}
\end{frame}

\begin{frame}[fragile]
    \frametitle{Limitations of Random Forests - Conclusion}
    While Random Forests are versatile and effective, it is important to be cautious of their limitations. Understanding these drawbacks leads to better decision-making in model selection and application.
    
    \textbf{Key Points to Remember}:
    \begin{itemize}
        \item Challenge in interpretability due to complexity.
        \item Sensitivity to noise can lead to overfitting.
        \item Significant computational demands in resource-intensive contexts.
        \item Generalization decreases with small datasets.
    \end{itemize}

    \textbf{References}:
    \begin{itemize}
        \item Breiman, L. (2001). "Random Forests." Machine Learning.
    \end{itemize}
\end{frame}

\begin{frame}[fragile]
    \frametitle{Overview}
    \begin{itemize}
        \item Random Forests (RF) are an ensemble learning method that utilizes multiple decision trees.
        \item Key advantages:
        \begin{itemize}
            \item Improved predictive accuracy
            \item Control overfitting
        \end{itemize}
        \item RF is versatile for both classification and regression tasks.
        \item This presentation explores several real-world applications across various industries.
    \end{itemize}
\end{frame}

\begin{frame}[fragile]
    \frametitle{Healthcare Applications}
    \begin{itemize}
        \item **Predictive Analytics:**
        \begin{itemize}
            \item Used to predict disease outcomes based on patient data.
            \item Example: Identifying patients at risk for cancers using genetic markers and lifestyle factors.
        \end{itemize}
    \end{itemize}
    \begin{block}{Example}
        A Random Forest model analyzes features (age, genetic markers, family history) to classify patients' breast cancer risk, aiding in preventative screenings.
    \end{block}
\end{frame}

\begin{frame}[fragile]
    \frametitle{Applications in Other Industries}
    \begin{enumerate}
        \item **Finance:**
        \begin{itemize}
            \item RF assesses creditworthiness using historical data (income, debt, repayment history).
            \item Example: Classifying loan applicants into risk categories.
        \end{itemize}

        \item **Marketing:**
        \begin{itemize}
            \item RF helps in customer segmentation for targeted marketing based on behavior analysis.
            \item Example: Predicting customer lifetime value (CLV) for personalized campaign strategies.
        \end{itemize}

        \item **Environmental Science:**
        \begin{itemize}
            \item RF classifies species and predicts habitats for conservation.
            \item Example: Predicting suitable areas for endangered species based on environmental data.
        \end{itemize}

        \item **Agriculture:**
        \begin{itemize}
            \item RF predicts crop yields using historical weather and soil data.
            \item Example: Analyzing past crop performance under varying climatic conditions.
        \end{itemize}
    \end{enumerate}
\end{frame}

\begin{frame}[fragile]
    \frametitle{Key Points and Conclusion}
    \begin{itemize}
        \item **Versatility:** Applicable in numerous sectors, handling both structured and unstructured data.
        \item **Accuracy:** High accuracy with reduced overfitting through aggregation of predictions.
        \item **Interpretability:** Techniques like feature importance scores enhance understanding of influential factors.
    \end{itemize}
    \begin{block}{Conclusion}
        Random Forests enhance decision-making across various domains, showcasing their adaptability and inviting further exploration for real-world implementations.
    \end{block}
\end{frame}

\begin{frame}
    \titlepage
\end{frame}

\begin{frame}
    \frametitle{Overview of Random Forests}
    \begin{itemize}
        \item \textbf{Random Forest} is an ensemble learning method that combines multiple decision trees.
        \item It improves predictive accuracy and controls overfitting.
        \item Widely used due to its robustness and effectiveness in classification and regression tasks.
    \end{itemize}
\end{frame}

\begin{frame}
    \frametitle{Required Libraries}
    \begin{block}{Python}
        \begin{itemize}
            \item \texttt{scikit-learn}: Machine learning library that includes a Random Forest implementation.
            \item \texttt{pandas}: For data manipulation and analysis.
            \item \texttt{numpy}: Essential for numerical operations.
            \item \texttt{matplotlib} and/or \texttt{seaborn}: For visualization.
        \end{itemize}
    \end{block}
    \begin{block}{R}
        \begin{itemize}
            \item \texttt{randomForest}: Provides functions to implement Random Forest.
            \item \texttt{dplyr}: For data manipulation.
            \item \texttt{ggplot2}: For data visualization.
        \end{itemize}
    \end{block}
\end{frame}

\begin{frame}[fragile]
    \frametitle{Step-by-Step Implementation - Python}
    \begin{enumerate}
        \item \textbf{Install Libraries}:
            \begin{lstlisting}[language=bash]
pip install numpy pandas scikit-learn matplotlib
            \end{lstlisting}

        \item \textbf{Import Libraries}:
            \begin{lstlisting}[language=Python]
import pandas as pd
import numpy as np
from sklearn.model_selection import train_test_split
from sklearn.ensemble import RandomForestClassifier
from sklearn.metrics import classification_report, confusion_matrix
import matplotlib.pyplot as plt
            \end{lstlisting}

        \item \textbf{Load Dataset}:
            \begin{lstlisting}[language=Python]
data = pd.read_csv('your_dataset.csv')  # Load your dataset
            \end{lstlisting}

        \item \textbf{Preprocess Data} \\
        Handle missing values, encode categorical variables, and feature scaling if necessary.
    \end{enumerate}
\end{frame}

\begin{frame}[fragile]
    \frametitle{Step-by-Step Implementation - Python (cont.)}
    \begin{enumerate}
        \setcounter{enumi}{4}
        \item \textbf{Split Data}:
            \begin{lstlisting}[language=Python]
X = data.drop('target', axis=1)  # Features
y = data['target']                # Target variable
X_train, X_test, y_train, y_test = train_test_split(X, y, test_size=0.2, random_state=42)
            \end{lstlisting}

        \item \textbf{Create Random Forest Model}:
            \begin{lstlisting}[language=Python]
rf_model = RandomForestClassifier(n_estimators=100, random_state=42)
rf_model.fit(X_train, y_train)     # Training the model
            \end{lstlisting}

        \item \textbf{Make Predictions}:
            \begin{lstlisting}[language=Python]
y_pred = rf_model.predict(X_test)
            \end{lstlisting}

        \item \textbf{Evaluate Model}:
            \begin{lstlisting}[language=Python]
print(confusion_matrix(y_test, y_pred))
print(classification_report(y_test, y_pred))
            \end{lstlisting}
    \end{enumerate}
\end{frame}

\begin{frame}[fragile]
    \frametitle{Step-by-Step Implementation - R}
    \begin{enumerate}
        \item \textbf{Install Libraries}:
            \begin{lstlisting}[language=R]
install.packages("randomForest")
install.packages("dplyr")
install.packages("ggplot2")
            \end{lstlisting}

        \item \textbf{Load Libraries}:
            \begin{lstlisting}[language=R]
library(randomForest)
library(dplyr)
library(ggplot2)
            \end{lstlisting}

        \item \textbf{Load Dataset}:
            \begin{lstlisting}[language=R]
data <- read.csv('your_dataset.csv')  # Load your dataset
            \end{lstlisting}

        \item \textbf{Preprocess Data} \\
        Handle missing values and encode categorical variables as needed.
    \end{enumerate}
\end{frame}

\begin{frame}[fragile]
    \frametitle{Step-by-Step Implementation - R (cont.)}
    \begin{enumerate}
        \setcounter{enumi}{4}
        \item \textbf{Split Data}:
            \begin{lstlisting}[language=R]
set.seed(42)
training_indices <- sample(1:nrow(data), 0.8*nrow(data))
training_set <- data[training_indices, ]
test_set <- data[-training_indices, ]
            \end{lstlisting}

        \item \textbf{Create Random Forest Model}:
            \begin{lstlisting}[language=R]
rf_model <- randomForest(target ~ ., data = training_set, ntree=100)  # Training the model
            \end{lstlisting}

        \item \textbf{Make Predictions}:
            \begin{lstlisting}[language=R]
predictions <- predict(rf_model, test_set)
            \end{lstlisting}

        \item \textbf{Evaluate Model}:
            \begin{lstlisting}[language=R]
conf_matrix <- table(test_set$target, predictions)
print(conf_matrix)
            \end{lstlisting}
    \end{enumerate}
\end{frame}

\begin{frame}
    \frametitle{Key Points to Emphasize}
    \begin{itemize}
        \item \textbf{Ensemble Learning}: Random Forests reduce overfitting and increase accuracy by averaging predictions from multiple trees.
        \item \textbf{Hyperparameters}: Key parameters (e.g., number of trees, maximum depth) significantly impact model performance.
        \item \textbf{Feature Importance}: The model provides insights into which features contribute most to predictions, aiding in feature selection.
    \end{itemize}
\end{frame}

\begin{frame}
    \frametitle{Conclusion}
    By following this guide, you will have implemented a Random Forest model in Python or R and will be prepared to assess its performance in real-world applications.
\end{frame}

\begin{frame}[fragile]
    \frametitle{Evaluating Model Performance - Introduction}
    \begin{block}{Introduction}
        To ensure that our Random Forest model performs effectively, it is crucial to evaluate its performance using different metrics. Understanding these metrics helps us gain insights into the model's strengths and weaknesses, leading to better decision-making.
    \end{block}
\end{frame}

\begin{frame}[fragile]
    \frametitle{Evaluating Model Performance - Key Metrics}
    \begin{block}{Key Performance Metrics}
        \begin{enumerate}
            \item \textbf{Accuracy}
                \begin{itemize}
                    \item \textbf{Definition}: The ratio of correctly predicted instances to the total instances.
                    \item \textbf{Formula}:
                    \begin{equation}
                        \text{Accuracy} = \frac{\text{True Positives + True Negatives}}{\text{Total Instances}}
                    \end{equation}
                    \item \textbf{Example}: If a model correctly predicts 90 out of 100 cases, the accuracy is:
                    \begin{equation}
                        \text{Accuracy} = \frac{90}{100} = 0.90 \text{ (or 90\%)}
                    \end{equation}
                    \item \textbf{Key Point}: Works well when classes are balanced; may be misleading with class imbalance.
                \end{itemize}

            \item \textbf{Precision}
                \begin{itemize}
                    \item \textbf{Definition}: The ratio of true positive predictions to the total predicted positives.
                    \item \textbf{Formula}:
                    \begin{equation}
                        \text{Precision} = \frac{\text{True Positives}}{\text{True Positives + False Positives}}
                    \end{equation}
                    \item \textbf{Example}: If out of 40 predicted positive cases, 30 are true positives:
                    \begin{equation}
                        \text{Precision} = \frac{30}{40} = 0.75 \text{ (or 75\%)}
                    \end{equation}
                    \item \textbf{Key Point}: High precision is important in applications like spam detection.
                \end{itemize}
        \end{enumerate}
    \end{block}
\end{frame}

\begin{frame}[fragile]
    \frametitle{Evaluating Model Performance - Continued}
    \begin{block}{Key Performance Metrics (Continued)}
        \begin{enumerate}
            \setcounter{enumii}{2} % Continue numbering
            \item \textbf{Recall (Sensitivity)}
                \begin{itemize}
                    \item \textbf{Definition}: The ratio of true positives to the total actual positives.
                    \item \textbf{Formula}:
                    \begin{equation}
                        \text{Recall} = \frac{\text{True Positives}}{\text{True Positives + False Negatives}}
                    \end{equation}
                    \item \textbf{Example}: If there are 50 actual positive cases and 30 are correctly predicted:
                    \begin{equation}
                        \text{Recall} = \frac{30}{50} = 0.60 \text{ (or 60\%)}
                    \end{equation}
                    \item \textbf{Key Point}: Crucial when the cost of missing a positive is high, as in disease detection.
                \end{itemize}
        \end{enumerate}
    \end{block}
    \begin{block}{Additional Metrics to Consider}
        \begin{itemize}
            \item \textbf{F1 Score}: The harmonic mean of precision and recall, useful for balancing both metrics.
            \item \textbf{ROC-AUC}: Represents the trade-off between the true positive rate and the false positive rate at various thresholds.
        \end{itemize}
    \end{block}
\end{frame}

\begin{frame}[fragile]
    \frametitle{Evaluating Model Performance - Summary}
    \begin{block}{Summary}
        \begin{itemize}
            \item \textbf{Importance}: Evaluating model performance is essential for understanding its reliability.
            \item \textbf{Application}: Choose appropriate metrics based on the project's context. For example, prioritize recall in medical diagnoses or precision in fraud detection.
        \end{itemize}
    \end{block}
\end{frame}

\begin{frame}[fragile]
    \frametitle{Evaluating Model Performance - Conclusion}
    \begin{block}{Conclusion}
        Evaluating the performance of Random Forest models using metrics like accuracy, precision, and recall ensures that we make informed decisions based on the model's effectiveness in the real world. By leveraging these metrics, we can optimize our models to better suit specific application needs.
    \end{block}
\end{frame}

\begin{frame}[fragile]
    \frametitle{Conclusion and Key Takeaways - Importance of Random Forests}
    \begin{block}{What are Random Forests?}
        Random Forests are an ensemble learning technique that combines multiple decision trees to improve predictive accuracy and control overfitting. 
        Each tree in a Random Forest predicts output (classification or regression) independently, and the final prediction is made based on:
        \begin{itemize}
            \item Majority voting (for classification)
            \item Averaging (for regression)
        \end{itemize}
    \end{block}

    \begin{block}{Why Use Random Forests?}
        \begin{enumerate}
            \item \textbf{Robustness to Overfitting:} Prevents overfitting by averaging outputs, useful in noisy datasets.
            \item \textbf{Handling Large Datasets:} Manages high-dimensional datasets, suitable for real-world applications.
            \item \textbf{Feature Importance:} Reveals the influence of variables on predictions.
        \end{enumerate}
    \end{block}
\end{frame}

\begin{frame}[fragile]
    \frametitle{Conclusion and Key Takeaways - Real-World Applications}
    \begin{itemize}
        \item \textbf{Healthcare:} Predicting patient outcomes based on medical signals and demographics.
        \item \textbf{Finance:} Credit scoring and default prediction models.
        \item \textbf{Marketing:} Customer segmentation and factors leading to purchases.
    \end{itemize}
    
    \begin{block}{Key Points to Emphasize}
        \begin{itemize}
            \item Ensemble learning is foundational for machine learning techniques.
            \item Critical performance metrics include accuracy, precision, recall, and the F1-score.
            \item Limitations: computationally intensive and less interpretable than individual trees.
        \end{itemize}
    \end{block}
\end{frame}

\begin{frame}[fragile]
    \frametitle{Final Thoughts on Implementing Ensemble Methods}
    \begin{itemize}
        \item \textbf{Start Simple:} Begin with a single decision tree to understand data patterns.
        \item \textbf{Parameter Tuning:} Use techniques like grid search or random search to optimize hyperparameters (e.g., number of trees).
    \end{itemize}

    \begin{block}{Practical Implementation Example}
    \begin{lstlisting}[language=Python]
from sklearn.datasets import load_iris
from sklearn.model_selection import train_test_split
from sklearn.ensemble import RandomForestClassifier
from sklearn.metrics import classification_report

# Load dataset
iris = load_iris()
X = iris.data
y = iris.target

# Split data into training and test sets
X_train, X_test, y_train, y_test = train_test_split(X, y, test_size=0.2, random_state=42)

# Create a Random Forest model
model = RandomForestClassifier(n_estimators=100, random_state=42)
model.fit(X_train, y_train)

# Make predictions
predictions = model.predict(X_test)

# Evaluate the model
print(classification_report(y_test, predictions))
    \end{lstlisting}
    \end{block}
\end{frame}


\end{document}