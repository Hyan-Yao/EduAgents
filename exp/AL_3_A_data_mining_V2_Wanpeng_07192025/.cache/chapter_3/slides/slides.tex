\documentclass[aspectratio=169]{beamer}

% Theme and Color Setup
\usetheme{Madrid}
\usecolortheme{whale}
\useinnertheme{rectangles}
\useoutertheme{miniframes}

% Additional Packages
\usepackage[utf8]{inputenc}
\usepackage[T1]{fontenc}
\usepackage{graphicx}
\usepackage{booktabs}
\usepackage{listings}
\usepackage{amsmath}
\usepackage{amssymb}
\usepackage{xcolor}
\usepackage{tikz}
\usepackage{pgfplots}
\pgfplotsset{compat=1.18}
\usetikzlibrary{positioning}
\usepackage{hyperref}

% Custom Colors and Other Configurations...

%%%% Document Start %%%%
\begin{document}

\frame{\titlepage}

\begin{frame}[fragile]
    \frametitle{Introduction to Classification in Data Mining}
    \begin{block}{What is Classification?}
        Classification is a fundamental task in data mining that involves predicting the categorical label of new observations based on past observations with known labels. 
    \end{block}
\end{frame}

\begin{frame}[fragile]
    \frametitle{Significance of Classification in Data Mining}
    \begin{itemize}
        \item \textbf{Data Organization:} Helps organize large datasets into manageable categories.
        \item \textbf{Decision Making:} Enables businesses to make informed decisions by classifying data, e.g., categorizing customer feedback based on sentiment.
        \item \textbf{Predictive Analysis:} Facilitates prediction of future outcomes, such as spam detection in emails.
    \end{itemize}
\end{frame}

\begin{frame}[fragile]
    \frametitle{Real-world Implications of Classification}
    \begin{itemize}
        \item \textbf{Healthcare:} Classification algorithms used for disease diagnosis.
        \item \textbf{Finance:} Credit scoring models classify loan applicants for approvals.
        \item \textbf{Marketing:} Customer segmentation on purchasing behavior to enable targeted strategies.
    \end{itemize}
\end{frame}

\begin{frame}[fragile]
    \frametitle{Why is Classification Key in Data Analysis?}
    \begin{enumerate}
        \item \textbf{Efficiency:} Automates categorization, reducing time and errors.
        \item \textbf{Scalability:} Handles large datasets effectively, suited for big data.
        \item \textbf{Improvement over Time:} Models can be retrained to enhance accuracy.
        \item \textbf{Informed Insights:} Identifies trends for better planning and resource allocation.
    \end{enumerate}
\end{frame}

\begin{frame}[fragile]
    \frametitle{Example of Classification Problem}
    \begin{block}{Problem Statement}
        Classify bank customers as "Churn" or "Not Churn" based on account behavior.
    \end{block}
    \begin{itemize}
        \item \textbf{Input Features:}
            \begin{itemize}
                \item Age
                \item Account Balance
                \item Transaction History
                \item Customer Service Calls
            \end{itemize}
        \item \textbf{Output:} Class label ("Churn" or "Not Churn").
    \end{itemize}
\end{frame}

\begin{frame}[fragile]
    \frametitle{Summary}
    \begin{itemize}
        \item Understanding classification is essential for advanced data mining methods.
        \item Real-world applications enhance decision-making across various sectors.
        \item Continuous learning and adaptation lead to improved outcomes in data analysis.
    \end{itemize}
\end{frame}

\begin{frame}[fragile]
    \frametitle{Learning Objectives}
    This slide focuses on three primary learning objectives that you should accomplish by the end of this week:
    \begin{enumerate}
        \item \textbf{Understanding Classification Basics}
        \item \textbf{Exploring Classification Algorithms}
        \item \textbf{Recognizing Ethical Practices in Classification}
    \end{enumerate}
\end{frame}

\begin{frame}[fragile]
    \frametitle{1. Understanding Classification Basics}
    \begin{block}{Definition}
        Classification is a supervised learning approach in data mining and machine learning where the goal is to predict the category to which new observations belong, based on a training dataset.
    \end{block}
    
    \begin{itemize}
        \item \textbf{Importance}: Grasping the fundamental concepts of classification is crucial because it underpins many applications, including customer segmentation, fraud detection, and diagnostic systems.
        \item \textbf{Key Terms}:
        \begin{itemize}
            \item \textbf{Training Data}: The dataset used to train the model, containing both input features and known outcomes (labels).
            \item \textbf{Test Data}: A separate dataset used to evaluate the model's performance after training.
        \end{itemize}
    \end{itemize}

    \begin{block}{Example}
        Consider a hospital that wishes to classify patients as "high risk" or "low risk" for heart disease based on features such as age, blood pressure, and cholesterol levels.
    \end{block}
\end{frame}

\begin{frame}[fragile]
    \frametitle{2. Exploring Classification Algorithms}
    \begin{block}{Overview}
        Algorithms are the mathematical procedures that enable machines to classify data. Familiarizing yourself with various types of algorithms will bolster your capability to choose the appropriate method for different problems.
    \end{block}
    
    \begin{itemize}
        \item \textbf{Examples of Classification Algorithms}:
        \begin{itemize}
            \item \textbf{Decision Trees}: Visual representation of decisions and their possible consequences. Simple but can overfit.
            \item \textbf{Support Vector Machines (SVM)}: Effective in high-dimensional spaces for classifying data using hyperplanes.
            \item \textbf{k-Nearest Neighbors (k-NN)}: Classifies based on the closest training examples in the feature space.
            \item \textbf{Random Forest}: Ensemble method using multiple decision trees to improve accuracy.
        \end{itemize}
    \end{itemize}

    \begin{block}{Key Point}
        Understand the pros and cons of each algorithm to apply them effectively in real-world scenarios.
    \end{block}
\end{frame}

\begin{frame}[fragile]
    \frametitle{3. Recognizing Ethical Practices in Classification}
    \begin{block}{Ethics in Data Science}
        As data classification influences significant outcomes, ethical considerations are paramount to ensure fairness, accountability, and transparency.
    \end{block}
    
    \begin{itemize}
        \item \textbf{Considerations}:
        \begin{itemize}
            \item \textbf{Bias}: Analyze the training data for potential biases that could lead to unfair classification.
            \item \textbf{Privacy}: Always consider the privacy of individuals in your dataset. Ensure compliance with laws such as GDPR or HIPAA.
            \item \textbf{Explainability}: Aim to make models interpretable, especially in critical sectors like healthcare and finance where decisions significantly impact lives.
        \end{itemize}
    \end{itemize}

    \begin{block}{Example}
        When developing a classification model for loan approvals, it’s crucial to ensure that it does not disproportionately affect applicants based on race or gender.
    \end{block}
\end{frame}

\begin{frame}[fragile]
    \frametitle{Classification Problems Defined - Part 1}
    \begin{block}{What are Classification Problems?}
        \textbf{Definition:} Classification problems are a type of supervised machine learning task where the goal is to assign an input data point to one of several predefined categories or classes. Unlike regression tasks, which predict continuous values, classification outputs discrete labels.
    \end{block}

    \begin{itemize}
        \item \textbf{Supervised Learning:} Relies on labeled datasets for training models.
        \item \textbf{Discrete Outputs:} Outputs belong to specific categories (e.g., "spam" or "not spam").
        \item \textbf{Training and Testing:} Models are trained on a training set, then evaluated on a separate testing set.
    \end{itemize}
\end{frame}

\begin{frame}[fragile]
    \frametitle{Classification Problems Defined - Part 2}
    \begin{block}{Examples of Classification Problems}
        \begin{enumerate}
            \item \textbf{Email Spam Detection}
                \begin{itemize}
                    \item \textbf{Scenario:} Classifying emails as "spam" or "not spam."
                    \item \textbf{Input:} Features from an email (sender, keywords, etc.)
                    \item \textbf{Output:} Class label indicating the status.
                \end{itemize}
            \item \textbf{Medical Diagnosis}
                \begin{itemize}
                    \item \textbf{Scenario:} Determining whether a patient has a disease.
                    \item \textbf{Input:} Patient data (age, symptoms, tests, etc.)
                    \item \textbf{Output:} Class labels ("Positive" or "Negative").
                \end{itemize}
            \item \textbf{Image Recognition}
                \begin{itemize}
                    \item \textbf{Scenario:} Identifying objects in images (cat or dog).
                    \item \textbf{Input:} Pixel data from images.
                    \item \textbf{Output:} Class labels ("Cat" or "Dog").
                \end{itemize}
            \item \textbf{Sentiment Analysis}
                \begin{itemize}
                    \item \textbf{Scenario:} Classifying customer reviews.
                    \item \textbf{Input:} Text from reviews.
                    \item \textbf{Output:} Sentiment categories.
                \end{itemize}
        \end{enumerate}
    \end{block}
\end{frame}

\begin{frame}[fragile]
    \frametitle{Classification Problems Defined - Part 3}
    \begin{block}{Key Points to Emphasize}
        \begin{itemize}
            \item \textbf{Importance of Labeled Data:} Essential for effective training of models.
            \item \textbf{Evaluation Metrics:} Accuracy, precision, recall, and F1 score for performance assessment.
            \item \textbf{Real-world Applications:} Used extensively in finance, healthcare, and social media.
        \end{itemize}
    \end{block}

    \begin{block}{Additional Insights}
        \begin{itemize}
            \item \textbf{Common Challenges:}
                \begin{itemize}
                    \item Class imbalance leading to biased models.
                    \item Overfitting resulting in poor performance on unseen data.
                \end{itemize}
            \item \textbf{Technical Considerations:}
                \begin{itemize}
                    \item Feature selection to improve accuracy.
                    \item Algorithm selection based on the problem’s nature.
                \end{itemize}
        \end{itemize}
    \end{block}
\end{frame}

\begin{frame}[fragile]
    \frametitle{Classification Algorithms Overview}
    
    \begin{block}{Introduction to Classification Algorithms}
        Classification algorithms are a fundamental part of machine learning that enables us to categorize data into predefined classes or labels. 
        These algorithms are essential in various real-world applications, from spam detection in emails to predicting customer churn.
    \end{block}
\end{frame}

\begin{frame}[fragile]
    \frametitle{Main Types of Classification Algorithms - Part 1}
    
    \begin{enumerate}
        \item \textbf{Logistic Regression}
            \begin{itemize}
                \item \textbf{Concept}: A statistical method used to model the probability of a binary outcome based on one or more predictor variables. The output is a value between 0 and 1, which can be interpreted as probabilities.
                \item \textbf{Key Formula}:
                \begin{equation}
                    P(Y=1|X) = \frac{1}{1 + e^{-(\beta_0 + \beta_1 X_1 + \beta_2 X_2 + ... + \beta_n X_n)}}
                \end{equation}
                \item \textbf{Application}: Widely used in binary classification tasks such as credit scoring (good risk vs. bad risk).
                \item \textbf{Example}: Predicting whether an email is spam (1) or not spam (0).
            \end{itemize}
        \item \textbf{Decision Trees}
            \begin{itemize}
                \item \textbf{Concept}: A tree-like model used for both classification and regression tasks. It recursively splits the data based on the feature that provides the maximum information gain.
                \item \textbf{Structure}: Consists of nodes (features), branches (decision rules), and leaves (outcomes).
                \item \textbf{Application}: Commonly used for customer segmentation and risk assessment in finance.
                \item \textbf{Example}: Classifying whether a customer will buy a product based on age, income, and past purchase history.
            \end{itemize}
    \end{enumerate}
\end{frame}

\begin{frame}[fragile]
    \frametitle{Main Types of Classification Algorithms - Part 2}
    
    \begin{enumerate}
        \setcounter{enumi}{2}
        \item \textbf{Random Forests}
            \begin{itemize}
                \item \textbf{Concept}: An ensemble learning method that creates a multitude of decision trees to improve classification accuracy. Each tree is trained on a random subset of the data, and the final prediction is made by averaging the predictions (in regression) or majority voting (in classification).
                \item \textbf{Benefits}: Reduces overfitting compared to individual decision trees and handles both classification and regression tasks.
                \item \textbf{Application}: Used in healthcare for disease prediction and in finance for credit scoring.
                \item \textbf{Example}: Predicting the loan default risk based on various borrower characteristics using numerous decision trees.
            \end{itemize}
    \end{enumerate}
    
    \begin{block}{Key Points to Emphasize}
        \begin{itemize}
            \item Real-World Relevance: Highlighting practical applications makes the concept relatable.
            \item Choice of Algorithm: Depends on the dataset and the problem at hand.
            \item Performance Evaluation: Understand strengths and weaknesses for effective model selection.
        \end{itemize}
    \end{block}
\end{frame}

\begin{frame}[fragile]
    \frametitle{Engagement and Interactivity Suggestions}
    
    \begin{itemize}
        \item \textbf{In-Class Practice}: Provide datasets for hands-on practice with logistic regression and decision trees.
        \item \textbf{Discussion Questions}: Encourage students to discuss the advantages and disadvantages of each algorithm in small groups.
        \item \textbf{Interactive Polls}: Use live polls to ask students which algorithm they would use for specific case studies.
    \end{itemize}
    
    By covering these algorithms' fundamental concepts, applications, and examples, we set the stage for deeper dives into each classification technique in the following slides.
\end{frame}

\begin{frame}[fragile]
    \frametitle{Logistic Regression - Principles}
    \begin{block}{Classification Overview}
        Logistic Regression is a statistical method used for binary classification tasks, where the outcome variable is categorical and typically consists of two classes, such as success/failure or yes/no.
    \end{block}
    
    \begin{block}{Key Concept}
        Logistic Regression estimates the probability that a given input point belongs to a particular category using the logistic (sigmoid) function.
    \end{block}

    The formula for the logistic function is:
    \begin{equation}
    P(Y=1|X) = \frac{1}{1 + e^{-(\beta_0 + \beta_1X_1 + \beta_2X_2 + ... + \beta_nX_n)}}
    \end{equation}
    Where:
    \begin{itemize}
        \item $P(Y=1|X)$ = Probability that the output is 1 given input features $X$
        \item $\beta_0, \beta_1, \ldots, \beta_n$ = Coefficients to be estimated from data
        \item $e$ = Euler's number, approximately 2.71828
    \end{itemize}
\end{frame}

\begin{frame}[fragile]
    \frametitle{Logistic Regression - Applications, Benefits, and Limitations}
    \begin{block}{Application Areas}
        Logistic Regression is widely used in:
        \begin{itemize}
            \item Healthcare: Predicting the likelihood of a patient having a disease based on diagnostic features.
            \item Finance: Estimating the probability of loan default or creditworthiness.
            \item Marketing: Classifying whether a customer will respond positively to a marketing campaign (purchase/no purchase).
        \end{itemize}
    \end{block}

    \begin{block}{Benefits}
        \begin{itemize}
            \item Simplicity: Easy to understand and implement with interpretable coefficients.
            \item Efficiency: Requires less computational power compared to more complex models.
            \item Probabilistic Interpretation: Provides probabilities for class predictions suitable for decision thresholds.
        \end{itemize}
    \end{block}

    \begin{block}{Limitations}
        \begin{itemize}
            \item Linearity: Assumes a linear relationship between independent variables and the log-odds of the dependent variable.
            \item Binary Classification: Most effective for binary outcomes; can be extended but becomes complex.
            \item Sensitive to Outliers: Outliers can disproportionately influence model parameters.
        \end{itemize}
    \end{block}
\end{frame}

\begin{frame}[fragile]
    \frametitle{Logistic Regression - Example}
    \textbf{Problem Statement:} Predict whether an email is spam (1) or not (0) based on the frequency of certain keywords.

    \textbf{Data Example:}
    \begin{itemize}
        \item Features: Keyword Frequency (X1: "Free", X2: "Win")
        \item Coefficients: $\beta_0 = -2$, $\beta_1 = 1.5$, $\beta_2 = 2$
    \end{itemize}

    \textbf{Logit Model:}
    \begin{equation}
    \text{Logit}(P) = -2 + 1.5 \times X_1 + 2 \times X_2
    \end{equation}

    \textbf{Prediction:}
    For an email with "Free" keyword count of 3 and "Win" count of 1:
    \begin{equation}
    \text{Logit}(P) = -2 + 1.5 \times 3 + 2 \times 1 = 2.5
    \end{equation}
    
    This logit value can be converted into a probability:
    \begin{equation}
    P = \frac{1}{1 + e^{-2.5}} \approx 0.924
    \end{equation}
    
    \textbf{Interpretation:} There’s a 92.4\% probability that this email is spam.
\end{frame}

\begin{frame}[fragile]
    \frametitle{Decision Trees - Overview}
    \begin{block}{What are Decision Trees?}
        A Decision Tree is a versatile and powerful machine learning algorithm used for both classification and regression tasks. 
    \end{block}
    The structure resembles a flowchart:
    \begin{itemize}
        \item Each internal node represents a decision based on a feature.
        \item Each branch represents the outcome of that decision.
        \item Each leaf node signifies a final class label or value.
    \end{itemize}
\end{frame}

\begin{frame}[fragile]
    \frametitle{Decision Trees - Structure}
    \begin{itemize}
        \item \textbf{Root Node:} The topmost node representing the entire dataset. Typically the feature that best splits the data.
        \item \textbf{Internal Nodes:} Represent features for decision making, splitting the data based on feature values.
        \item \textbf{Branches:} The links that represent the outcome of a decision.
        \item \textbf{Leaf Nodes:} Terminal nodes representing the final output or prediction (class labels for classification tasks).
    \end{itemize}
    \begin{block}{Simple Decision Tree Example}
        \begin{itemize}
            \item Root Node: Is it a warm-blooded animal?
            \item Yes: Has fur?
            \item No: Leaves: [Bird, Fish, etc.]
        \end{itemize}
    \end{block}
\end{frame}

\begin{frame}[fragile]
    \frametitle{Decision Trees - Usage and Benefits}
    \begin{itemize}
        \item \textbf{Classification Tasks:} Used for problems where the output is a discrete label.
        \item \textbf{Interpretability:} Clear visual representation allows for easy interpretation.
        \item \textbf{Flexibility:} Effectively handles both numerical and categorical data.
    \end{itemize}
    \begin{block}{Advantages and Limitations}
        \textbf{Advantages:}
        \begin{itemize}
            \item Easy to understand and visualize.
            \item Requires little data preprocessing.
            \item Handles large datasets well.
        \end{itemize}
        \textbf{Limitations:}
        \begin{itemize}
            \item Prone to overfitting.
            \item Sensitive to noisy data.
            \item May struggle with many classes or features.
        \end{itemize}
    \end{block}
\end{frame}

\begin{frame}[fragile]
    \frametitle{Visualization of a Simple Decision Tree}
    \begin{center}
        \texttt{
            Is it warm-blooded? \\
            / \ \ \ \ \ \ \ \ \ \ \  \backslash \\
            Yes \ \ \ \ \ \ \ \ \ \  No \\
            / \ \ \ \ \ \ \ \ \ \ \  \backslash \\
            Has fur? \ \ \ \ \  Fish \\
            / \ \ \ \ \ \ \ \ \  \backslash \\
            Yes \ \ \ \ \ \  No \\
            Dog \ \ \ \  Lizard \ \  Output: Fish
        }
    \end{center}
\end{frame}

\begin{frame}[fragile]
    \frametitle{Random Forests - Introduction}
    \begin{block}{Definition}
        Random Forests is an ensemble learning method that combines multiple decision trees to improve classification accuracy and control overfitting.
    \end{block}
\end{frame}

\begin{frame}[fragile]
    \frametitle{Random Forests - How It Works}
    \begin{itemize}
        \item \textbf{Ensemble Method}: Constructs multiple trees during training and outputs the mode of classes (classification) or mean prediction (regression).
        \item \textbf{Process}:
        \begin{itemize}
            \item \textbf{Bootstrapping}: Random samples of the dataset are taken (with replacement) to create diverse trees.
            \item \textbf{Feature Randomness}: A random subset of features is considered at each node for splitting.
        \end{itemize}
    \end{itemize}
\end{frame}

\begin{frame}[fragile]
    \frametitle{Random Forests - Advantages}
    \begin{itemize}
        \item \textbf{Reduced Overfitting}: Randomization helps prevent overly complex models.
        \item \textbf{Improved Accuracy}: Combines multiple weak models to form a robust classifier.
        \item \textbf{Handles Large Datasets}: Efficiently works with higher dimensionality data.
        \item \textbf{Robust to Noise}: Less sensitive to outliers compared to individual decision trees.
    \end{itemize}
\end{frame}

\begin{frame}[fragile]
    \frametitle{Random Forests - Practical Examples}
    \begin{itemize}
        \item \textbf{Healthcare}: Predicting patient diseases based on indicators (e.g., age, blood pressure).
        \item \textbf{Finance}: Risk assessment in loan approvals by analyzing financial histories.
        \item \textbf{E-commerce}: Product recommendations based on customer behavior.
    \end{itemize}
\end{frame}

\begin{frame}[fragile]
    \frametitle{Random Forests - Key Points}
    \begin{itemize}
        \item \textbf{Interpretability}: Individual trees are easy to understand, but the ensemble's complexity can obscure decision-making.
        \item \textbf{Feature Importance}: Assesses feature importance through metrics like Gini importance or mean decrease in impurity.
    \end{itemize}
\end{frame}

\begin{frame}[fragile]
    \frametitle{Random Forests - Example Code (Python)}
    \begin{lstlisting}[language=Python]
from sklearn.ensemble import RandomForestClassifier
from sklearn.model_selection import train_test_split
from sklearn.datasets import load_iris

# Load dataset
data = load_iris()
X, y = data.data, data.target

# Split data into training and test sets
X_train, X_test, y_train, y_test = train_test_split(X, y, test_size=0.3, random_state=42)

# Create and fit the model
rf_model = RandomForestClassifier(n_estimators=100, random_state=42)
rf_model.fit(X_train, y_train)

# Evaluate the model
accuracy = rf_model.score(X_test, y_test)
print(f'Accuracy: {accuracy:.2f}')
    \end{lstlisting}
\end{frame}

\begin{frame}[fragile]
    \frametitle{Random Forests - Conclusion}
    \begin{block}{Conclusion}
        Random Forests are a robust and versatile algorithm combining multiple decision trees' strengths. Their ability to reduce overfitting and capture complex data patterns makes them essential in various machine learning applications.
    \end{block}
\end{frame}

\begin{frame}[fragile]
    \frametitle{Comparison of Algorithms - Overview}
    In this slide, we will compare three classification algorithms:
    \begin{itemize}
        \item \textbf{Random Forests}
        \item \textbf{Support Vector Machines (SVM)}
        \item \textbf{K-Nearest Neighbors (K-NN)}
    \end{itemize}
    Each algorithm has its strengths and weaknesses based on the nature of the dataset and the specific problem requirements.
\end{frame}

\begin{frame}[fragile]
    \frametitle{Random Forests}
    \begin{itemize}
        \item \textbf{Type:} Ensemble Learning
        \item \textbf{Strengths:}
            \begin{itemize}
                \item Handles overfitting effectively by averaging multiple decision trees.
                \item Robust to noise and manages large datasets with diverse features.
                \item Provides feature importance scores, aiding in feature selection.
            \end{itemize}
        \item \textbf{Weaknesses:}
            \begin{itemize}
                \item More complex and less interpretable than individual trees.
                \item Requires greater computational resources and memory.
            \end{itemize}
        \item \textbf{Use Cases:} Ideal for large datasets with many features, e.g., customer segmentation, fraud detection.
    \end{itemize}
\end{frame}

\begin{frame}[fragile]
    \frametitle{Support Vector Machines (SVM) and K-Nearest Neighbors (K-NN)}
    \begin{itemize}
        \item \textbf{Support Vector Machines (SVM)}
        \begin{itemize}
            \item \textbf{Type:} Discriminative Classifier
            \item \textbf{Strengths:}
                \begin{itemize}
                    \item Effective in high-dimensional spaces.
                    \item Uses kernel trick for non-linear relationships.
                    \item Maximizes the margin between classes.
                \end{itemize}
            \item \textbf{Weaknesses:}
                \begin{itemize}
                    \item Memory-intensive and slower on large datasets.
                    \item Requires careful tuning of hyperparameters.
                \end{itemize}
            \item \textbf{Use Cases:} Suitable for text classification and image recognition tasks.
        \end{itemize}
        
        \item \textbf{K-Nearest Neighbors (K-NN)}
        \begin{itemize}
            \item \textbf{Type:} Instance-Based Learning
            \item \textbf{Strengths:}
                \begin{itemize}
                    \item Simple and easy to implement without a training phase.
                    \item Naturally handles multi-class classification.
                    \item Adapts easily to new data.
                \end{itemize}
            \item \textbf{Weaknesses:}
                \begin{itemize}
                    \item Computationally intensive for large datasets.
                    \item Sensitive to noisy data.
                    \item Difficult to interpret results in high dimensions.
                \end{itemize}
            \item \textbf{Use Cases:} Effective for recommendation systems and real-time predictions.
        \end{itemize}
    \end{itemize}
\end{frame}

\begin{frame}[fragile]
    \frametitle{Conclusion}
    When selecting a classification algorithm, consider the following:
    \begin{itemize}
        \item \textbf{Random Forests} are ideal for robustness and handling large datasets.
        \item \textbf{SVM} excels with complex and high-dimensional data.
        \item \textbf{K-NN} is best for simple, quick predictions.
    \end{itemize}
    
    \textbf{Key Points:}
    \begin{itemize}
        \item Consider dataset size, dimensionality, and interpretability.
        \item Each algorithm has unique strengths suited for specific domains.
    \end{itemize}
\end{frame}

\begin{frame}[fragile]
    \frametitle{Classroom Activity}
    \textbf{Activity:} 
    Split students into groups and assign each group a different dataset. 
    Ask them to choose the most suitable classification algorithm, justifying their choice based on the strengths and weaknesses outlined.
    
    \textbf{Note:}
    Encourage students to use Python libraries like \texttt{scikit-learn} for real-time applications in class.
\end{frame}

\begin{frame}[fragile]
    \frametitle{Ethical Considerations in Classification - Introduction}
    \begin{block}{Introduction to Ethical Implications}
        Classification algorithms are widely used in applications such as:
        \begin{itemize}
            \item Healthcare predictions
            \item Loan approvals
        \end{itemize}
        However, their deployment raises significant ethical concerns, primarily focused on:
        \begin{itemize}
            \item Data privacy
            \item Fairness
        \end{itemize}
    \end{block}
\end{frame}

\begin{frame}[fragile]
    \frametitle{Ethical Considerations in Classification - Key Concepts}
    \begin{block}{Data Privacy}
        Classification algorithms rely on personal data for predictions, leading to essential considerations:
        \begin{itemize}
            \item \textbf{Informed Consent:} Are users aware of data usage? Opt-in vs. opt-out mechanisms can affect trust.
            \item \textbf{Data Anonymization:} Can we guarantee that data cannot be traced back to individuals? 
                  Techniques like pseudonymization help protect privacy but are not foolproof.
        \end{itemize}
    \end{block}

    \begin{block}{Bias and Fairness}
        Classification models may perpetuate societal biases present in training data:
        \begin{itemize}
            \item \textbf{Discrimination:} Models trained on biased historical data may disadvantage specific groups.
            \item \textbf{Transparency:} Understanding a model's decision-making is crucial for fairness assessment. This necessitates explainable AI (XAI).
        \end{itemize}
    \end{block}
\end{frame}

\begin{frame}[fragile]
    \frametitle{Ethical Considerations in Classification - Real-World Examples}
    \begin{block}{Real-World Examples}
        \begin{itemize}
            \item \textbf{Healthcare:} Algorithms may favor certain demographics, leading to unequal treatment. For instance, a model trained primarily on one ethnicity may not serve well for others.
            \item \textbf{Hiring Algorithms:} Automated tools could discard applications from specific demographics if trained on biased historical data.
        \end{itemize}
    \end{block}

    \begin{block}{Ethical Frameworks}
        \begin{itemize}
            \item \textbf{Fairness Metrics:} Assess demographic parity or equal opportunity to evaluate equitable outcomes.
            \item \textbf{Data Protection Regulations:} Laws like GDPR emphasize ethical data management in classification.
        \end{itemize}
    \end{block}
\end{frame}

\begin{frame}[fragile]
    \frametitle{Ethical Considerations in Classification - Conclusion}
    \begin{block}{Key Points to Emphasize}
        \begin{itemize}
            \item Ensure informed consent in data collection and usage.
            \item Verify and strive for fairness; avoid algorithmic bias.
            \item Utilize explainable AI (XAI) for transparent decision-making processes.
        \end{itemize}
    \end{block}

    \begin{block}{Conclusion}
        Ethical considerations in classification are essential, affecting the handling and interpretation of data. Understanding these principles fosters responsible algorithmic development.
    \end{block}
\end{frame}

\begin{frame}
    \frametitle{Hands-On Exercise}
    \begin{block}{Title}
        Practical Classification Exercise Using R or Python
    \end{block}

    \begin{block}{Objective}
        To engage students in applying classification algorithms on a real dataset using R or Python, enhancing their understanding of classification concepts and ethical considerations.
    \end{block}
\end{frame}

\begin{frame}[fragile]
    \frametitle{Step-by-Step Activity - Part 1}
    \begin{enumerate}
        \item \textbf{Dataset Selection:}
        \begin{itemize}
            \item Iris dataset: 
            \begin{itemize}
                \item Features: Sepal length, Sepal width, Petal length, Petal width
                \item Target variable: Species (Setosa, Versicolor, Virginica)
            \end{itemize}
            \item Titanic dataset (alternative):
            \begin{itemize}
                \item Features: Passenger class, Age, Gender, Number of siblings/spouses aboard, etc.
                \item Target variable: Survival (0 = No, 1 = Yes)
            \end{itemize}
        \end{itemize}

        \item \textbf{Setting Up the Environment:}
        \begin{itemize}
            \item \textbf{In R:}
            \begin{lstlisting}
install.packages("caret")
library(caret)
data(iris)  # load the dataset
            \end{lstlisting}
            \item \textbf{In Python:}
            \begin{lstlisting}
import pandas as pd
from sklearn import datasets
iris = datasets.load_iris()
iris_df = pd.DataFrame(data=iris.data, columns=iris.feature_names)
iris_df['species'] = iris.target
            \end{lstlisting}
        \end{itemize}
    \end{enumerate}
\end{frame}

\begin{frame}[fragile]
    \frametitle{Step-by-Step Activity - Part 2}
    \begin{enumerate}[resume]
        \item \textbf{Data Preprocessing:}
        \begin{itemize}
            \item Handle missing values (if any).
            \item Normalize or standardize features if necessary.
            \item Split dataset into training and testing sets (80/20 split).
        \end{itemize}

        \item \textbf{R Example:}
        \begin{lstlisting}
set.seed(123)
trainIndex <- createDataPartition(iris$Species, p = .8, list = FALSE)
train_set <- iris[trainIndex, ]
test_set <- iris[-trainIndex, ]
        \end{lstlisting}

        \item \textbf{Python Example:}
        \begin{lstlisting}
from sklearn.model_selection import train_test_split
train_set, test_set = train_test_split(iris_df, test_size=0.2, random_state=42)
        \end{lstlisting}
    \end{enumerate}
\end{frame}

\begin{frame}[fragile]
    \frametitle{Step-by-Step Activity - Part 3}
    \begin{enumerate}[resume]
        \item \textbf{Model Implementation:}
        \begin{itemize}
            \item Choose a classification algorithm (Decision Trees, Logistic Regression, Random Forest).

            \item \textbf{R Example: Random Forest}
            \begin{lstlisting}
library(randomForest)
model <- randomForest(Species ~ ., data=train_set)
            \end{lstlisting}

            \item \textbf{Python Example: Decision Tree}
            \begin{lstlisting}
from sklearn.tree import DecisionTreeClassifier
model = DecisionTreeClassifier()
model.fit(train_set.drop('species', axis=1), train_set['species'])
            \end{lstlisting}
        \end{itemize}

        \item \textbf{Model Evaluation:}
        \begin{itemize}
            \item Use confusion matrix and accuracy to evaluate the model on test data.
            \item R Example: 
            \begin{lstlisting}
predictions <- predict(model, test_set)
confusionMatrix(data=predictions, reference=test_set$Species)
            \end{lstlisting}
            \item Python Example:
            \begin{lstlisting}
from sklearn.metrics import confusion_matrix, accuracy_score
predictions = model.predict(test_set.drop('species', axis=1))
print(confusion_matrix(test_set['species'], predictions))
print("Accuracy:", accuracy_score(test_set['species'], predictions))
            \end{lstlisting}
        \end{itemize}
    \end{enumerate}
\end{frame}

\begin{frame}
    \frametitle{Key Points & Closing Thoughts}
    \begin{itemize}
        \item Understand how to interpret the results of classification models and the implications of their predictions.
        \item Discuss ethical considerations, such as data privacy and bias when using real-world datasets.
        \item Encourage collaboration and peer feedback during the exercise.
    \end{itemize}

    \begin{block}{Closing Thought}
        This hands-on experience not only solidifies your understanding of classification algorithms but also prepares you for real-world applications and challenges.
    \end{block}
\end{frame}

\begin{frame}[fragile]
    \frametitle{Project Overview}
    \begin{block}{Overview of Upcoming Projects}
    This week, students will engage in hands-on projects exploring the fundamentals of classification in data science. 
    The projects will provide practical experience, reinforce class concepts, and enhance problem-solving skills.
    \end{block}
\end{frame}

\begin{frame}[fragile]
    \frametitle{Project Focus: Classification Tasks}
    \begin{itemize}
        \item \textbf{Objective}: Apply classification algorithms to real-world datasets to solve problems such as predicting outcomes or categorizing data.
        \item \textbf{Examples}:
        \begin{itemize}
            \item \textbf{Email Spam Detection}: Classifying emails as 'spam' or 'not spam' based on features like sender, subject line, and keywords.
            \item \textbf{Iris Flower Classification}: Using features like petal and sepal dimensions to classify iris flowers into three species: Setosa, Versicolor, and Virginica.
        \end{itemize}
    \end{itemize}
\end{frame}

\begin{frame}[fragile]
    \frametitle{Expectations for Students}
    \begin{enumerate}
        \item \textbf{Dataset Selection}: Choose a dataset that aligns with project goals (e.g., UCI Machine Learning Repository, Kaggle).
        \item \textbf{Analysis Plan}: Develop a structured plan including:
        \begin{itemize}
            \item Data preprocessing steps
            \item Choosing classification algorithms (e.g., Logistic Regression, Decision Trees)
            \item Evaluation metrics (e.g., accuracy, precision, recall)
        \end{itemize}
        \item \textbf{Implementation}: Write code in R or Python documenting your process and findings.
        \item \textbf{Presentation}: Summarize your project, including methodology, results, and insights.
    \end{enumerate}
\end{frame}

\begin{frame}[fragile]
    \frametitle{Resources Available}
    \begin{itemize}
        \item \textbf{Tutorials and Guides}:
        \begin{itemize}
            \item Online tutorials for R and Python classification libraries (e.g., Scikit-learn, caret).
            \item Example projects on platforms like GitHub.
        \end{itemize}
        \item \textbf{In-Class Support}:
        \begin{itemize}
            \item Weekly office hours for project challenges.
            \item Peer collaboration sessions for insights and support.
        \end{itemize}
        \item \textbf{Reference Materials}:
        \begin{itemize}
            \item Textbooks and research articles on classification techniques.
            \item Recordings of previous lectures.
        \end{itemize}
    \end{itemize}
\end{frame}

\begin{frame}[fragile]
    \frametitle{Key Points to Emphasize}
    \begin{itemize}
        \item Real-world applications of classification models.
        \item Importance of data preparation and choosing correct algorithms.
        \item Benefits of peer review and collaborative learning for deeper understanding.
    \end{itemize}
\end{frame}

\begin{frame}[fragile]
    \frametitle{Example Code Snippet (Python)}
    \begin{lstlisting}[language=Python]
from sklearn.datasets import load_iris
from sklearn.model_selection import train_test_split
from sklearn.tree import DecisionTreeClassifier
from sklearn.metrics import classification_report

# Load dataset
iris = load_iris()
X = iris.data
y = iris.target

# Split the data
X_train, X_test, y_train, y_test = train_test_split(X, y, test_size=0.2, random_state=42)

# Initialize and fit classifier
classifier = DecisionTreeClassifier()
classifier.fit(X_train, y_train)

# Predict and evaluate
y_pred = classifier.predict(X_test)
print(classification_report(y_test, y_pred))
    \end{lstlisting}
\end{frame}

\begin{frame}[fragile]
    \frametitle{Summary and Q\&A - Key Takeaways}
    \begin{enumerate}
        \item \textbf{Definition of Classification}:
        \begin{itemize}
            \item Organizing data into categories based on shared characteristics.
        \end{itemize}
        
        \item \textbf{Types of Classification}:
        \begin{itemize}
            \item \textbf{Binary Classification}: Two distinct groups (e.g., spam vs. not spam).
            \item \textbf{Multiclass Classification}: More than two categories (e.g., species of flowers).
        \end{itemize}
        
        \item \textbf{Importance of Features}:
        \begin{itemize}
            \item Measurable properties for predictions; selecting the right features is crucial.
            \item Example: Features for classifying emails may include keywords, sender information, and links.
        \end{itemize}
    \end{enumerate}
\end{frame}

\begin{frame}[fragile]
    \frametitle{Summary and Q\&A - Algorithms and Metrics}
    \begin{enumerate}[resume]
        \item \textbf{Common Algorithms}:
        \begin{itemize}
            \item \textbf{Decision Trees}: Splits data based on feature values.
            \item \textbf{Support Vector Machines (SVM)}: Optimally separates classes in high-dimensional spaces.
            \item \textbf{K-Nearest Neighbors (KNN)}: Classifies based on the majority among nearest neighbors.
        \end{itemize}
        
        \item \textbf{Evaluation Metrics}:
        \begin{itemize}
            \item \textbf{Accuracy}:
            \begin{equation}
                \text{Accuracy} = \frac{\text{True Positives} + \text{True Negatives}}{\text{Total Population}}
            \end{equation}
            \item \textbf{Precision and Recall}: Vital for assessing model performance in imbalanced datasets.
        \end{itemize}

        \item \textbf{Real-World Applications}:
        \begin{itemize}
            \item \textbf{Healthcare}: Classifying tumors as benign or malignant.
            \item \textbf{Finance}: Detecting fraudulent transactions by analyzing account behaviors.
        \end{itemize}
    \end{enumerate}
\end{frame}

\begin{frame}[fragile]
    \frametitle{Summary and Q\&A - Engagement and Next Steps}
    \textbf{Engagement and Discussion}:
    \begin{itemize}
        \item What characteristics are important for classification in your domain?
        \item Identify scenarios where binary classification may not suffice.
        \item How do imbalanced class distributions affect classification performance?
    \end{itemize}

    \textbf{Next Steps}:
    \begin{itemize}
        \item Think of a project idea to apply what you've learned about classification.
        \item We will discuss project expectations in the next slide!
    \end{itemize}
\end{frame}


\end{document}