\documentclass[aspectratio=169]{beamer}

% Theme and Color Setup
\usetheme{Madrid}
\usecolortheme{whale}
\useinnertheme{rectangles}
\useoutertheme{miniframes}

% Additional Packages
\usepackage[utf8]{inputenc}
\usepackage[T1]{fontenc}
\usepackage{graphicx}
\usepackage{booktabs}
\usepackage{listings}
\usepackage{amsmath}
\usepackage{amssymb}
\usepackage{xcolor}
\usepackage{tikz}
\usepackage{pgfplots}
\pgfplotsset{compat=1.18}
\usetikzlibrary{positioning}
\usepackage{hyperref}

% Custom Colors
\definecolor{myblue}{RGB}{31, 73, 125}
\definecolor{mygray}{RGB}{100, 100, 100}
\definecolor{mygreen}{RGB}{0, 128, 0}
\definecolor{myorange}{RGB}{230, 126, 34}
\definecolor{mycodebackground}{RGB}{245, 245, 245}

% Set Theme Colors
\setbeamercolor{structure}{fg=myblue}
\setbeamercolor{frametitle}{fg=white, bg=myblue}
\setbeamercolor{title}{fg=myblue}
\setbeamercolor{section in toc}{fg=myblue}
\setbeamercolor{item projected}{fg=white, bg=myblue}
\setbeamercolor{block title}{bg=myblue!20, fg=myblue}
\setbeamercolor{block body}{bg=myblue!10}
\setbeamercolor{alerted text}{fg=myorange}

\title[Week 3: Hadoop Ecosystem]{Week 3: Hadoop Ecosystem}
\author[J. Smith]{John Smith, Ph.D.}
\institute[University Name]{
  Department of Computer Science\\
  University Name\\
  \vspace{0.3cm}
  Email: email@university.edu\\
  Website: www.university.edu
}
\date{\today}

% Document Start
\begin{document}

\frame{\titlepage}

\begin{frame}[fragile]
    \frametitle{Introduction to Hadoop Ecosystem}
    \begin{block}{What is the Hadoop Ecosystem?}
        The Hadoop ecosystem is a suite of tools and technologies designed to store, process, and analyze large datasets, commonly referred to as “big data.” Hadoop itself is an open-source framework for distributed storage and processing of data across clusters.
    \end{block}
\end{frame}

\begin{frame}[fragile]
    \frametitle{Significance in Big Data Processing}
    \begin{enumerate}
        \item \textbf{Scalability}:
            \begin{itemize}
                \item Efficiently stores and processes petabytes of data.
                \item Scales out easily by adding more nodes to the cluster.
                \item \textit{Example:} A business can start with a small cluster and grow as data needs increase.
            \end{itemize}

        \item \textbf{Fault Tolerance}:
            \begin{itemize}
                \item Automatically replicates data across multiple nodes.
                \item Ensures system recovery without data loss if one node fails.
                \item \textit{Example:} In retail, if a transaction log node fails, replicas can seamlessly take over.
            \end{itemize}

        \item \textbf{Cost-Effective}:
            \begin{itemize}
                \item Runs on commodity hardware, making it economically accessible.
                \item \textit{Example:} Startups can utilize inexpensive servers rather than costly proprietary systems.
            \end{itemize}

        \item \textbf{Flexibility}:
            \begin{itemize}
                \item Compatible with various data formats (structured, semi-structured, unstructured).
                \item \textit{Example:} Social media data and traditional records can be processed together.
            \end{itemize}
    \end{enumerate}
\end{frame}

\begin{frame}[fragile]
    \frametitle{Key Components of the Hadoop Ecosystem}
    \begin{itemize}
        \item \textbf{Hadoop Distributed File System (HDFS)}: A distributed file system for data storage across machines.
        \item \textbf{MapReduce}: A programming model for processing large data sets with distributed algorithms.
        \item \textbf{YARN (Yet Another Resource Negotiator)}: Resource management that handles scheduling and allocation.
        \item \textbf{Other Tools}:
            \begin{itemize}
                \item \textit{Apache Hive}: Data warehousing and SQL-like queries.
                \item \textit{Apache Pig}: Platform for analyzing large datasets using scripts.
                \item \textit{Apache HBase}: NoSQL database running on HDFS.
                \item \textit{Apache Spark}: Fast computation engine complementing Hadoop storage.
            \end{itemize}
    \end{itemize}
\end{frame}

\begin{frame}[fragile]
    \frametitle{What is Hadoop? - Definition}
    Hadoop is an open-source framework designed to facilitate the processing, storage, and analysis of large datasets across clusters of computers. It can handle structured and unstructured data, making it a pivotal tool in the realm of big data processing.
\end{frame}

\begin{frame}[fragile]
    \frametitle{What is Hadoop? - Purpose}
    \begin{itemize}
        \item \textbf{Scalability:} Hadoop scales from a single server to thousands of machines, allowing for easy expansion as data grows.
        \item \textbf{Cost-Effectiveness:} It is designed to work with commodity hardware, making it affordable for organizations of all sizes.
        \item \textbf{Fault Tolerance:} Data is automatically replicated across different nodes, ensuring system operation even if machines fail.
    \end{itemize}
\end{frame}

\begin{frame}[fragile]
    \frametitle{What is Hadoop? - Key Motivations}
    \begin{enumerate}
        \item \textbf{The Need for Big Data Processing:} The exponential growth of data generated from sources like social media and IoT devices requires efficient management.
        \item \textbf{Complexity of Data Management:} Traditional systems struggle with vast datasets. Hadoop effectively handles various data formats, including text, images, and video.
        \item \textbf{Distributed Computing:} The use of multiple servers to distribute workloads allows parallel processing, resulting in significant time savings.
    \end{enumerate}
\end{frame}

\begin{frame}[fragile]
    \frametitle{What is Hadoop? - Key Points}
    \begin{itemize}
        \item \textbf{Open-source:} Available for free, facilitating global collaboration and innovation.
        \item \textbf{Community Support:} Strong backing by a community and an ecosystem of tools (e.g., Hive, Pig, Spark) enhancing Hadoop's functionality.
        \item \textbf{Real-world Applications:} Used across various industries like:
            \begin{itemize}
                \item \textbf{Finance:} Fraud detection and risk management.
                \item \textbf{Retail:} Customer behavior analytics.
                \item \textbf{Healthcare:} Genomic analysis and patient data processing.
            \end{itemize}
    \end{itemize}
\end{frame}

\begin{frame}[fragile]
    \frametitle{Hadoop Architecture}
    \begin{center}
        \texttt{
        ┌───────────────────────────────────────────┐ \\
        │             Hadoop Ecosystem              │ \\
        │                                           │ \\
        │ ┌────────────┐         ┌────────────┐    │ \\
        │ │    HDFS    │         │   YARN     │    │ \\
        │ │ (Storage)  │         │ (Resource  │    │ \\
        │ │            │         │ Management)│    │ \\
        │ └────────────┘         └────────────┘    │ \\
        │         │                  │               │ \\
        ┌────────┼────────┐     ┌────┼─────┐         │ \\
        │      Hadoop         ├───┤   Node    ├──────┤ \\
        │    MapReduce    │   │   │             │   │ \\
        │ (Processing)     │   │   │             │   │ \\
        │                  │    └───────────────┘    │ \\
        └─────────────────────────────────────────────┘
        }
    \end{center}
\end{frame}

\begin{frame}[fragile]{Core Components of Hadoop - Overview}
    \begin{itemize}
        \item The Hadoop ecosystem is a framework for distributed storage and processing of large data sets.
        \item Focus on two core components: 
        \begin{itemize}
            \item Hadoop Distributed File System (HDFS)
            \item Yet Another Resource Negotiator (YARN)
        \end{itemize}
    \end{itemize}
\end{frame}

\begin{frame}[fragile]{Core Components of Hadoop - HDFS}
    \frametitle{Core Components of Hadoop - HDFS}
    \begin{block}{Hadoop Distributed File System (HDFS)}
        \begin{itemize}
            \item **Definition:** Primary storage system of Hadoop, enabling large file storage across multiple machines.
            \item **Architecture:**
                \begin{itemize}
                    \item Master-Slave Architecture: Consists of a single NameNode (master) and multiple DataNodes (slaves).
                    \item File Blocks: Splits large files into smaller blocks (default 128 MB or 256 MB).
                    \item Replication: Maintains data integrity by replicating each block across multiple DataNodes (default replication factor is 3).
                \end{itemize}
            \item **Example:** 1 TB of data can be stored across 8 blocks of 128 MB, ensuring that data remains accessible even if one server fails.
        \end{itemize}
    \end{block}
    \begin{figure}
        \centering
        \includegraphics[width=0.7\textwidth]{hdfs_architecture_diagram} % Placeholder for diagram
        \caption{Diagram of HDFS Architecture}
    \end{figure}
\end{frame}

\begin{frame}[fragile]{Core Components of Hadoop - YARN}
    \frametitle{Core Components of Hadoop - YARN}
    \begin{block}{Yet Another Resource Negotiator (YARN)}
        \begin{itemize}
            \item **Definition:** Resource management layer of Hadoop, managing and scheduling resources across the cluster.
            \item **Components:**
                \begin{itemize}
                    \item ResourceManager: Manages cluster resources and serves client requests.
                    \item NodeManager: Monitors resource usage of individual nodes.
                    \item ApplicationMaster: Manages lifecycle of applications running on the cluster.
                \end{itemize}
            \item **Example:** Dynamically allocates resources for data processing jobs based on availability and demand, allowing for parallel application execution.
        \end{itemize}
    \end{block}
    \begin{figure}
        \centering
        \includegraphics[width=0.7\textwidth]{yarn_architecture_diagram} % Placeholder for diagram
        \caption{Diagram of YARN Architecture}
    \end{figure}
\end{frame}

\begin{frame}[fragile]{Core Components of Hadoop - Key Insights}
    \frametitle{Core Components of Hadoop - Key Insights}
    \begin{itemize}
        \item **Scalability:** HDFS and YARN allow for seamless horizontal scaling.
        \item **Fault Tolerance:** HDFS ensures data recovery with replication; YARN improves application stability.
        \item **Efficiency in Data Handling:** Collaboratively manage vast data efficiently, making Hadoop ideal for big data analysis.
    \end{itemize}
    \begin{block}{Conclusion}
        Understanding HDFS and YARN is crucial to leveraging the Hadoop ecosystem for efficient data processing and analysis.
    \end{block}
\end{frame}

\begin{frame}[fragile]
    \frametitle{Hadoop Distributed File System (HDFS) - Overview}
    \begin{block}{What is HDFS?}
        The \textbf{Hadoop Distributed File System (HDFS)} is a core component of the Hadoop ecosystem that facilitates the storage of large data sets across a distributed cluster of commodity hardware.
    \end{block}
    \begin{itemize}
        \item Designed for high throughput access to application data.
        \item Suitable for large-scale data processing.
        \item Operates on a master-slave architecture.
    \end{itemize}
\end{frame}

\begin{frame}[fragile]
    \frametitle{HDFS Structure}
    \begin{enumerate}
        \item \textbf{Nodes:}
        \begin{itemize}
            \item \textbf{NameNode} (Master):
                \begin{itemize}
                    \item Manages the file system namespace.
                    \item Regulates access to files by clients.
                \end{itemize}
            \item \textbf{DataNodes} (Slaves):
                \begin{itemize}
                    \item Store actual data blocks.
                    \item Serve read/write requests from clients.
                \end{itemize}
        \end{itemize}
        \item \textbf{Blocks:}
        \begin{itemize}
            \item Files split into fixed-size blocks (default is 128 MB).
            \item Each block replicated (default is 3 copies) across nodes for fault tolerance.
        \end{itemize}
        \item \textbf{Cluster:} Collection of NameNode and DataNodes for parallel storage and processing.
    \end{enumerate}
\end{frame}

\begin{frame}[fragile]
    \frametitle{Data Management in HDFS}
    \begin{itemize}
        \item \textbf{Data Redundancy:}
        \begin{itemize}
            \item Replication ensures data remains available even with DataNode failures.
            \item Crucial for preventing data loss in large-scale applications.
        \end{itemize}
        \item \textbf{Data Locality:}
        \begin{itemize}
            \item Computations are performed on nodes where data resides.
            \item Minimizes bandwidth usage and improves efficiency.
        \end{itemize}
        \item \textbf{High Throughput:}
        \begin{itemize}
            \item Optimized for high throughput for large datasets.
            \item Ideal for batch processing workloads rather than low latency.
        \end{itemize}
    \end{itemize}
\end{frame}

\begin{frame}[fragile]
    \frametitle{Key Points and Example Scenario}
    \begin{block}{Key Points}
        \begin{itemize}
            \item \textbf{Scalability:} Easily scaled by adding nodes to the cluster.
            \item \textbf{Fault Tolerance:} Replication ensures high availability and durability.
            \item \textbf{Cost-effectiveness:} Runs on commodity hardware, lowering storage costs.
        \end{itemize}
    \end{block}
    \begin{block}{Example Scenario}
        A company analyzing user behavior can:
        \begin{itemize}
            \item Store large amounts of clickstream data in HDFS.
            \item Ensure data replication across DataNodes for reliability.
            \item Run analysis jobs in parallel to speed up processing.
        \end{itemize}
    \end{block}
\end{frame}

\begin{frame}[fragile]
    \frametitle{Yet Another Resource Negotiator (YARN)}
    \begin{block}{Overview of YARN}
        YARN, which stands for Yet Another Resource Negotiator, is a core component of the Hadoop ecosystem, acting as the resource management layer.
        It optimizes scheduling and execution of applications, managing cluster resources effectively.
    \end{block}
\end{frame}

\begin{frame}[fragile]
    \frametitle{Key Functions of YARN}
    \begin{itemize}
        \item \textbf{Resource Management:} Efficiently allocates system resources (CPU, memory) to applications.
        \item \textbf{Job Scheduling:} Determines job priorities and execution based on resource availability and policies.
        \item \textbf{Monitoring:} Tracks resource usage and job progress for optimal performance and issue identification.
    \end{itemize}
\end{frame}

\begin{frame}[fragile]
    \frametitle{Architecture of YARN}
    YARN consists of the following key components:
    \begin{enumerate}
        \item \textbf{ResourceManager (RM)}: The master daemon managing resource allocation across all applications.
        \item \textbf{NodeManager (NM)}: Each node has a daemon managing the lifecycle of containers, monitoring resource usage.
        \item \textbf{ApplicationMaster (AM)}: Each application has an instance negotiating resources from the ResourceManager.
    \end{enumerate}
    \begin{block}{Diagram: YARN Architecture}
        % Include your diagram here depicting the ResourceManager, NodeManagers, and ApplicationMasters interactions
    \end{block}
\end{frame}

\begin{frame}[fragile]
    \frametitle{How YARN Works}
    \begin{enumerate}
        \item \textbf{Submitting an Application:} Requests resources to execute tasks upon submission.
        \item \textbf{Resource Allocation:} ResourceManager allocates resources to ApplicationMaster.
        \item \textbf{Task Execution:} ApplicationMaster communicates with NodeManagers to launch applications in containers.
        \item \textbf{Monitoring:} YARN monitors application and node resources continuously.
    \end{enumerate}
\end{frame}

\begin{frame}[fragile]
    \frametitle{Examples of YARN in Action}
    \begin{itemize}
        \item \textbf{Batch Processing:} Utilizes YARN for applications like Apache Spark and Apache Flink.
        \item \textbf{Real-Time Processing:} Manages real-time workloads for technologies like Apache Storm.
    \end{itemize}
\end{frame}

\begin{frame}[fragile]
    \frametitle{Key Points to Emphasize}
    \begin{itemize}
        \item YARN decouples resource management from data processing, allowing various processing models (e.g., MapReduce, Spark).
        \item Its scalability and versatility make it essential for big data applications.
        \item Understanding YARN is crucial for optimizing big data workloads in a Hadoop environment.
    \end{itemize}
\end{frame}

\begin{frame}[fragile]
    \frametitle{Summary}
    YARN is vital in the Hadoop ecosystem, providing a robust framework for resource management and job scheduling. Its architecture:
    \begin{itemize}
        \item Facilitates effective scaling of applications.
        \item Enhances overall performance through better resource utilization.
    \end{itemize}
\end{frame}

\begin{frame}
    \titlepage
\end{frame}

\begin{frame}
    \frametitle{Introduction to MapReduce}
    \begin{block}{Definition}
        \textbf{MapReduce} is a programming model utilized for processing and generating large datasets with a parallel and distributed algorithm on a cluster. It simplifies the complexity of data processing and allows developers to handle vast amounts of data efficiently.
    \end{block}
\end{frame}

\begin{frame}
    \frametitle{Core Concepts}
    \begin{enumerate}
        \item \textbf{Map Phase}:
        \begin{itemize}
            \item Divides input dataset into smaller pieces.
            \item Processes each piece independently using the \textbf{Mapper} to produce intermediate key-value pairs.
            \item \textit{Example:}
            \begin{itemize}
                \item Input: "Hadoop is great and Hadoop is open source"
                \item Output: (Hadoop, 1), (is, 1), (great, 1), (and, 1), (open, 1), (source, 1)
            \end{itemize}
        \end{itemize}

        \item \textbf{Shuffle Phase}:
        \begin{itemize}
            \item Aggregates and transfers intermediate results across the cluster based on keys.
            \item Ensures all values associated with the same key are grouped together.
        \end{itemize}

        \item \textbf{Reduce Phase}:
        \begin{itemize}
            \item The \textbf{Reducer} processes aggregated key-value pairs to produce the final result.
            \item \textit{Example:} Input: (Hadoop, 1), (Hadoop, 1) → Output: (Hadoop, 2)
        \end{itemize}
    \end{enumerate}
\end{frame}

\begin{frame}
    \frametitle{Example Workflow}
    \begin{itemize}
        \item \textbf{Input}: A dataset consists of text files containing articles.
        \item \textbf{Map}: Split each article into words; emit each word with a count of 1.
        \item \textbf{Shuffle}: Group all words with the same name together.
        \item \textbf{Reduce}: Each group emits the final count of occurrences for each word.
    \end{itemize}

    \begin{block}{Key Points to Emphasize}
        \begin{itemize}
            \item \textbf{Scalability}: Scales from a single server to thousands of machines.
            \item \textbf{Fault Tolerance}: Re-executes failed tasks ensuring reliability.
            \item \textbf{Data Locality}: Processes data where it is stored, minimizing data movement.
        \end{itemize}
    \end{block}
\end{frame}

\begin{frame}[fragile]
    \frametitle{Sample Code Snippet}
    Below is a simple implementation using the Hadoop MapReduce API:
    \begin{lstlisting}[language=Java]
public static class Mapper extends MapReduceBase implements Mapper<LongWritable, Text, Text, IntWritable> {
    private final static IntWritable one = new IntWritable(1);
    private Text word = new Text();

    public void map(LongWritable key, Text value, OutputCollector<Text, IntWritable> output, Reporter reporter) throws IOException {
        StringTokenizer tokenizer = new StringTokenizer(value.toString());
        while (tokenizer.hasMoreTokens()) {
            word.set(tokenizer.nextToken());
            output.collect(word, one);
        }
    }
}
    \end{lstlisting}
\end{frame}

\begin{frame}
    \frametitle{MapReduce Process Diagram}
    Visualize the MapReduce process as a pipeline:
    \begin{center}
        \texttt{Input Data} $\rightarrow$ \texttt{[Map Function]} $\rightarrow$ \texttt{Intermediate Key-Value Pairs} \\
        $\quad \quad \quad \quad \quad \quad \quad \quad \downarrow$ \\
        $\quad \quad \quad \quad \texttt{[Shuffle Phase]}$ \\
        $\quad \quad \quad \quad \downarrow$ \\
        \texttt{[Reduce Function]} $\rightarrow$ \texttt{Final Output}
    \end{center}
\end{frame}

\begin{frame}
    \frametitle{Conclusion}
    By understanding the MapReduce framework, you can appreciate its pivotal role in the Hadoop ecosystem for processing and analyzing large datasets efficiently. This knowledge sets the stage for exploring tools like Hive and Pig in the next slide, which build on these fundamental concepts.
\end{frame}

\begin{frame}
    \frametitle{Next Steps}
    \begin{itemize}
        \item Review the roles of Hive and Pig in simplifying data manipulation within the Hadoop framework.
    \end{itemize}
\end{frame}

\begin{frame}[fragile]
    \frametitle{Common Tools in the Hadoop Ecosystem - Introduction}
    \begin{block}{Overview}
        The Hadoop Ecosystem is a suite of tools designed to work with the Hadoop framework to facilitate Big Data storage, processing, and analysis. Understanding these tools is essential for leveraging Hadoop's full potential when tackling large datasets.
    \end{block}
\end{frame}

\begin{frame}[fragile]
    \frametitle{Common Tools in the Hadoop Ecosystem - Key Tools}
    \begin{itemize}
        \item \textbf{Apache Hive}
            \begin{itemize}
                \item \textbf{What is It?} A data warehousing tool with HiveQL, similar to SQL.
                \item \textbf{Use Cases:}
                    \begin{itemize}
                        \item Data analysis and reporting
                        \item Running SQL-like queries on data stored in Hadoop
                    \end{itemize}
                \item \textbf{Integration with Hadoop:} Translates SQL queries into MapReduce jobs.
            \end{itemize}
        
        \item \textbf{Apache Pig}
            \begin{itemize}
                \item \textbf{What is It?} A platform for processing large datasets via Pig Latin.
                \item \textbf{Use Cases:}
                    \begin{itemize}
                        \item Data transformation and loading
                        \item Complex data processing (joins, aggregations)
                    \end{itemize}
                \item \textbf{Integration with Hadoop:} Compiles scripts into MapReduce tasks.
            \end{itemize}
        
        \item \textbf{Apache HBase}
            \begin{itemize}
                \item \textbf{What is It?} A distributed NoSQL database for real-time access.
                \item \textbf{Use Cases:}
                    \begin{itemize}
                        \item Low-latency data access applications
                        \item Storing massive data across servers
                    \end{itemize}
                \item \textbf{Integration with Hadoop:} Utilizes HDFS for storage with real-time access.
            \end{itemize}
    \end{itemize}
\end{frame}

\begin{frame}[fragile]
    \frametitle{Common Tools in the Hadoop Ecosystem - Key Points}
    \begin{itemize}
        \item \textbf{Ease of Use:} Hive and Pig provide user-friendly abstractions, making Hadoop more accessible.
        \item \textbf{Real-Time Access:} HBase complements batch processing with real-time capabilities for applications like online data analytics.
        \item \textbf{Data Storage:} All tools collaborate with HDFS, ensuring robust storage while allowing varied access/processing patterns.
    \end{itemize}
\end{frame}

\begin{frame}[fragile]
    \frametitle{Common Tools in the Hadoop Ecosystem - Tool Integration}
    \begin{block}{Diagram of Tool Integration}
        \begin{center}
            \includegraphics[width=0.8\linewidth]{hadoop_tools_integration_diagram.png}
        \end{center}
        % Ensure to replace 'hadoop_tools_integration_diagram.png' with the actual path of your diagram image.
    \end{block}
\end{frame}

\begin{frame}[fragile]
    \frametitle{Data Ingestion and ETL in Hadoop - Introduction}
    \begin{block}{Introduction to Data Ingestion}
        Data ingestion is the process of obtaining and importing data for immediate use or storage in a database. 
        In the Hadoop ecosystem, it plays a crucial role in managing vast quantities of data efficiently and is integrated into the ETL (Extract, Transform, Load) process.
    \end{block}
\end{frame}

\begin{frame}[fragile]
    \frametitle{Data Ingestion and ETL in Hadoop - Overview}
    \begin{block}{ETL Overview}
        \begin{enumerate}
            \item \textbf{Extract}: Data is extracted from various sources (e.g., databases, applications, audio, video, social media).
            \item \textbf{Transform}: The extracted data is cleaned and transformed into a suitable format (filtering, mapping, aggregating).
            \item \textbf{Load}: The transformed data is loaded into a storage system (data warehouse or HDFS) for analysis.
        \end{enumerate}
    \end{block}
\end{frame}

\begin{frame}[fragile]
    \frametitle{Data Ingestion Tools and Workflow in Hadoop}
    \begin{block}{How Data Ingestion Works in Hadoop}
        \begin{itemize}
            \item \textbf{Data Sources}: 
                \begin{itemize}
                    \item Relational databases (e.g., MySQL)
                    \item NoSQL databases (e.g., MongoDB)
                    \item Streaming data (e.g., Apache Kafka)
                    \item File systems (e.g., CSV, JSON formats)
                \end{itemize}
            \item \textbf{Ingestion Tools}:
                \begin{itemize}
                    \item \textbf{Apache Flume}: Designed for streaming logs into Hadoop.
                    \item \textbf{Apache Sqoop}: Transfers bulk data between Hadoop and relational databases.
                    \item \textbf{Kafka}: A distributed streaming platform for real-time data pipelines.
                \end{itemize}
        \end{itemize}
    \end{block}

    \begin{block}{Workflow Example}
        \begin{itemize}
            \item \textbf{Extract}: Data from a MySQL database is extracted using Apache Sqoop. 
            \item \textbf{Transform}: Data cleansed using Apache Pig, removing unnecessary fields and aggregating.
            \item \textbf{Load}: Transformed data is loaded into HDFS for storage and processing.
        \end{itemize}
    \end{block}
\end{frame}

\begin{frame}[fragile]
    \frametitle{Key Points and Use Cases}
    \begin{block}{Key Points to Remember}
        \begin{itemize}
            \item Hadoop handles large volumes of data efficiently.
            \item ETL processes are fundamental for data analytics preparation.
            \item Tool choice (e.g., Flume, Sqoop) depends on source type and functionality.
        \end{itemize}
    \end{block}

    \begin{block}{Practical Use Cases}
        \begin{enumerate}
            \item \textbf{Log Analysis}: Using Flume to ingest server logs into Hadoop for network usage analysis.
            \item \textbf{Data Migration}: Using Sqoop to transfer customer data from an RDBMS to HDFS.
        \end{enumerate}
    \end{block}
\end{frame}

\begin{frame}[fragile]
    \frametitle{Conclusion and Explore Further}
    Understanding data ingestion and the ETL process is essential in leveraging Hadoop for big data analytics. These processes ensure that data is accurate, relevant, and ready for insightful analysis, supporting data-driven decisions in organizations.

    \begin{block}{Note}
        For further exploration, consider diving into each tool's documentation and experimenting with sample data ingestion workflows.
    \end{block}
\end{frame}

\begin{frame}[fragile]
    \frametitle{Case Studies in the Hadoop Ecosystem}
    \begin{block}{Introduction}
        Hadoop is an open-source framework allowing distributed processing of large data sets across clusters of computers. It is a powerful tool for industries, enabling efficient big data management.
    \end{block}
\end{frame}

\begin{frame}[fragile]
    \frametitle{Key Highlights of Hadoop Case Studies}
    \begin{itemize}
        \item \textbf{Scalability}: Handles increasing data amounts without major infrastructure changes.
        \item \textbf{Cost-Effectiveness}: Utilizes commodity hardware, making storage and processing cheaper.
        \item \textbf{Flexibility}: Manages structured and unstructured data for diverse applications.
    \end{itemize}
\end{frame}

\begin{frame}[fragile]
    \frametitle{Case Study Examples}
    \begin{enumerate}
        \item \textbf{Yahoo}
            \begin{itemize}
                \item \textbf{Challenge}: Processing enormous user interaction data.
                \item \textbf{Solution}: Implemented Hadoop for web indexing, processing over 24 petabytes.
                \item \textbf{Outcome}: Faster search results and improved user experience.
            \end{itemize}
        \item \textbf{Facebook}
            \begin{itemize}
                \item \textbf{Challenge}: Analyzing user data for ad targeting and content personalization.
                \item \textbf{Solution}: Processed log data using Hadoop.
                \item \textbf{Outcome}: Improved ad targeting, increasing click-through rates.
            \end{itemize}
    \end{enumerate}
\end{frame}

\begin{frame}[fragile]
    \frametitle{Case Study Examples (Contd.)}
    \begin{enumerate}
        \setcounter{enumi}{2}
        \item \textbf{eBay}
            \begin{itemize}
                \item \textbf{Challenge}: Understanding and improving customer behavior.
                \item \textbf{Solution}: Analyzed vast transaction and user data using Hadoop.
                \item \textbf{Outcome}: Better inventory management and optimized pricing strategies.
            \end{itemize}
        \item \textbf{Netflix}
            \begin{itemize}
                \item \textbf{Challenge}: Streamlining recommendations based on watching habits.
                \item \textbf{Solution}: Implemented a Hadoop-based architecture for real-time data processing.
                \item \textbf{Outcome}: Enhanced personalized recommendations and increased subscriber retention.
            \end{itemize}
    \end{enumerate}
\end{frame}

\begin{frame}[fragile]
    \frametitle{Key Takeaways}
    \begin{itemize}
        \item \textbf{Real-World Application}: Hadoop's effectiveness spans various industries.
        \item \textbf{Impact on Decision Making}: Insights from analyses drive business strategy and improve performance.
        \item \textbf{Continuous Evolution}: As data grows, the need for frameworks like Hadoop for strategic planning becomes clear.
    \end{itemize}
\end{frame}

\begin{frame}[fragile]
    \frametitle{Hadoop Architecture Overview}
    \begin{lstlisting}
    [Hadoop Architecture Overview]
      +---------------+
      |   HDFS        |  --> Stores large data sets in a distributed manner
      +---------------+
             |
             v
      +---------------+
      |   MapReduce   |  --> Processes data in parallel using various algorithms
      +---------------+
             |
             v
      +---------------+
      |   YARN        |  --> Resource management for scheduling tasks
      +---------------+
    \end{lstlisting}
\end{frame}

\begin{frame}[fragile]{Challenges and Limitations in the Hadoop Ecosystem}
    \begin{block}{Introduction to Challenges}
        Hadoop has revolutionized big data processing, but it is not without its challenges. Understanding these limitations is key to effectively leveraging Hadoop for data analytics.
    \end{block}
\end{frame}

\begin{frame}[fragile]{Key Challenges}
    \begin{enumerate}
        \item Scalability
        \item Data Consistency
        \item Complex Ecosystem
        \item Performance Issues
        \item Security Concerns
    \end{enumerate}
\end{frame}

\begin{frame}[fragile]{Scalability}
    \begin{itemize}
        \item \textbf{Explanation}: While Hadoop is designed to be highly scalable, there are still challenges as datasets grow exponentially.
        \item \textbf{Factors Affecting Scalability}:
        \begin{itemize}
            \item \textbf{Cluster Management}: Adding nodes requires careful balancing of data across the cluster, which can become complex.
            \item \textbf{Network Bottlenecks}: As more nodes are added, network congestion can occur, leading to slower data processing speeds.
        \end{itemize}
    \end{itemize}
\end{frame}

\begin{frame}[fragile]{Data Consistency}
    \begin{itemize}
        \item \textbf{Explanation}: Hadoop follows an 'eventual consistency' model, which can lead to issues when data integrity is paramount.
        \item \textbf{Challenges}:
        \begin{itemize}
            \item \textbf{Concurrent Writes}: Multiple processes trying to write to the same dataset can result in conflicts.
            \item \textbf{Data Updates}: Hadoop is optimized for batch processing; frequent updates can hinder performance and lead to stale data.
        \end{itemize}
    \end{itemize}
\end{frame}

\begin{frame}[fragile]{Complex Ecosystem and Performance Issues}
    \begin{itemize}
        \item \textbf{Complex Ecosystem}:
        \begin{itemize}
            \item \textbf{Integration}: Ensuring all components (Hadoop HDFS, MapReduce, Hive, etc.) work seamlessly can be a daunting task.
            \item \textbf{Skill Requirements}: Sufficient expertise is needed to manage the ecosystem effectively.
        \end{itemize}
        
        \item \textbf{Performance Issues}:
        \begin{itemize}
            \item \textbf{High Latency}: Hadoop's batch processing architecture can introduce delays that are not acceptable in real-time applications.
            \item \textbf{Resource Consumption}: Performing heavy computational tasks can require significant memory and CPU resources.
        \end{itemize}
    \end{itemize}
\end{frame}

\begin{frame}[fragile]{Security Concerns}
    \begin{itemize}
        \item \textbf{Explanation}: Data security is a critical concern when working with sensitive information in the Hadoop ecosystem.
        \item \textbf{Risks}:
        \begin{itemize}
            \item \textbf{Data Breaches}: Without adequate security measures, sensitive data can be exposed.
            \item \textbf{Lack of Built-in Security}: Older Hadoop versions lack robust security features, necessitating third-party solutions.
        \end{itemize}
    \end{itemize}
\end{frame}

\begin{frame}[fragile]{Summary of Key Points}
    \begin{itemize}
        \item \textbf{Scalability is a double-edged sword}: While Hadoop scales well, inefficiencies can arise as workloads increase.
        \item \textbf{Data consistency is challenging}: Understanding eventual consistency is crucial for effective Hadoop usage.
        \item \textbf{Managing the ecosystem} is complex and requires specialized knowledge.
        \item \textbf{Performance} is not suitable for all applications; latency may affect real-time processes.
        \item \textbf{Security measures} must be actively implemented to protect data.
    \end{itemize}
\end{frame}

\begin{frame}[fragile]{Example of Cluster Configuration}
    \begin{lstlisting}[language=bash]
# Sample command to add a new node to the Hadoop cluster
hdfs dfsadmin -addNodes <Node_IP>
    \end{lstlisting}
\end{frame}

\begin{frame}[fragile]{Conclusion}
    Recognizing these challenges is the first step toward developing effective strategies to mitigate them, ensuring successful Hadoop implementation in big data projects. Being proactive in addressing these limitations can lead to more reliable and efficient data processing solutions.
\end{frame}

\begin{frame}[fragile]
    \frametitle{Future Trends in Hadoop and Big Data}
    Insights into emerging trends and technologies that may impact the future of Hadoop and big data processing.
\end{frame}

\begin{frame}[fragile]
    \frametitle{Overview}
    \begin{itemize}
        \item The landscape of data is continuously evolving.
        \item Emerging trends and technologies are shaping the future of Hadoop and big data processing.
        \item Advancements will enhance efficiency, scalability, and applicability.
    \end{itemize}
\end{frame}

\begin{frame}[fragile]
    \frametitle{Key Trends in Hadoop and Big Data - Part 1}
    \begin{enumerate}
        \item \textbf{Integration with AI and Machine Learning (ML)}
        \begin{itemize}
            \item \textit{Description}: Integration brings predictive analytics capabilities.
            \item \textit{Example}: Use of Apache Spark MLlib alongside Hadoop for real-time data processing.
        \end{itemize}
        
        \item \textbf{Serverless Architecture}
        \begin{itemize}
            \item \textit{Description}: Run applications without managing servers for seamless scaling.
            \item \textit{Example}: AWS Lambda processes data in Hadoop without server provisioning.
        \end{itemize}

        \item \textbf{Data Governance and Security}
        \begin{itemize}
            \item \textit{Description}: Enhanced measures are critical due to stricter privacy regulations.
            \item \textit{Example}: Integration with Apache Ranger and Apache Knox for security management.
        \end{itemize}
    \end{enumerate}
\end{frame}

\begin{frame}[fragile]
    \frametitle{Key Trends in Hadoop and Big Data - Part 2}
    \begin{enumerate}
        \setcounter{enumi}{3} % Start from 4
        \item \textbf{Real-Time Data Processing}
        \begin{itemize}
            \item \textit{Description}: Rising demand for real-time analytics drives enhancements in streaming platforms.
            \item \textit{Example}: Real-time analytics help monitor user behavior in e-commerce platforms.
        \end{itemize}

        \item \textbf{Edge Computing}
        \begin{itemize}
            \item \textit{Description}: Processing data closer to IoT sources enhances speed and reduces latency.
            \item \textit{Example}: IoT sensors process data at the edge before sending aggregated insights to Hadoop.
        \end{itemize}
    \end{enumerate}
\end{frame}

\begin{frame}[fragile]
    \frametitle{Implications for Hadoop}
    \begin{itemize}
        \item Hadoop must evolve to leverage emerging technologies.
        \item Meeting demands for flexibility, speed, and data governance is essential.
        \item Organizations need to adopt these technologies to remain competitive.
    \end{itemize}
\end{frame}

\begin{frame}[fragile]
    \frametitle{Conclusion}
    \begin{itemize}
        \item Trend comprehension aids in preparing for future data strategies.
        \item Key: Embrace AI, real-time processing, serverless architectures, and security measures.
        \item Positioning organizations to harness the full potential of big data is crucial.
    \end{itemize}
\end{frame}

\begin{frame}[fragile]
    \frametitle{Key Points to Remember}
    \begin{itemize}
        \item Integrating AI/ML enhances data insights.
        \item Serverless architectures simplify deployment.
        \item Data governance is essential for compliance.
        \item Real-time processing is becoming a necessity.
        \item Edge computing significantly reduces latency.
    \end{itemize}
\end{frame}

\begin{frame}[fragile]{Summary and Wrap-Up - Part 1}
    \frametitle{Recap of Key Points Discussed in the Chapter}
    
    \begin{enumerate}
        \item \textbf{Overview of Hadoop Ecosystem}
            \begin{itemize}
                \item \textbf{Definition:} Hadoop is an open-source framework for distributed processing of large data sets.
                \item \textbf{Components:} Includes HDFS, MapReduce, and tools like Pig, Hive, and HBase.
            \end{itemize}
        
        \item \textbf{HDFS - Data Storage}
            \begin{itemize}
                \item \textbf{Purpose:} Efficiently stores large files, splitting them into blocks across nodes.
                \item \textbf{Key Feature:} Data replication ensures fault tolerance (default factor = 3).
            \end{itemize}
        
        \item \textbf{MapReduce - Data Processing}
            \begin{itemize}
                \item \textbf{Definition:} A programming model for processing large datasets in parallel.
                \item \textbf{Workflow:} 
                \begin{itemize}
                    \item Map: Creates key-value pairs from input data.
                    \item Reduce: Aggregates Map results based on keys.
                \end{itemize}
            \end{itemize}
    \end{enumerate}
\end{frame}

\begin{frame}[fragile]{Summary and Wrap-Up - Part 2}
    \frametitle{Additional Tools in the Ecosystem}
    
    \begin{enumerate}
        \setcounter{enumi}{3}  % Continue enumeration from the previous frame
        
        \item \textbf{Additional Tools in the Ecosystem}
            \begin{itemize}
                \item \textbf{Hive:} 
                \begin{itemize}
                    \item An SQL-like interface for querying data in HDFS.
                    \item \textbf{Example:} Running queries on web server logs for user logins.
                \end{itemize}
                
                \item \textbf{Pig:} 
                \begin{itemize}
                    \item High-level scripting language (Pig Latin) for data flow.
                    \item \textbf{Example:} Managing complex ETL jobs more intuitively. 
                \end{itemize}
                
                \item \textbf{HBase:} 
                \begin{itemize}
                    \item A NoSQL database for random real-time read/write access on HDFS.
                    \item \textbf{Example:} Mobile apps for user messaging utilizing real-time data.
                \end{itemize}
            \end{itemize}
    \end{enumerate}
\end{frame}

\begin{frame}[fragile]{Summary and Wrap-Up - Part 3}
    \frametitle{Real-World Applications and Key Takeaways}
    
    \begin{itemize}
        \item \textbf{Real-World Applications:}
            \begin{itemize}
                \item \textbf{E-commerce:} Utilizing Hadoop for recommendation engines powered by customer data.
                \item \textbf{Healthcare:} Analyzing patient records for insights and predictive analytics.
                \item \textbf{Social Media:} Processing user interaction data to improve engagement strategies.
            \end{itemize}
        
        \item \textbf{Key Points to Emphasize:}
            \begin{itemize}
                \item The Hadoop ecosystem's scalability and fault tolerance are essential for handling big data.
                \item Understanding the integration of components promotes efficient data processing.
            \end{itemize}
        
        \item \textbf{Concluding Note:}
            \begin{itemize}
                \item Hadoop serves as a cornerstone for Big Data analytics, enabling effective data management.
            \end{itemize}
    \end{itemize}
\end{frame}


\end{document}