\documentclass[aspectratio=169]{beamer}

% Theme and Color Setup
\usetheme{Madrid}
\usecolortheme{whale}
\useinnertheme{rectangles}
\useoutertheme{miniframes}

% Additional Packages
\usepackage[utf8]{inputenc}
\usepackage[T1]{fontenc}
\usepackage{graphicx}
\usepackage{booktabs}
\usepackage{listings}
\usepackage{amsmath}
\usepackage{amssymb}
\usepackage{xcolor}
\usepackage{tikz}
\usepackage{pgfplots}
\pgfplotsset{compat=1.18}
\usetikzlibrary{positioning}
\usepackage{hyperref}

% Custom Colors
\definecolor{myblue}{RGB}{31, 73, 125}
\definecolor{mygray}{RGB}{100, 100, 100}
\definecolor{mygreen}{RGB}{0, 128, 0}
\definecolor{myorange}{RGB}{230, 126, 34}
\definecolor{mycodebackground}{RGB}{245, 245, 245}

% Set Theme Colors
\setbeamercolor{structure}{fg=myblue}
\setbeamercolor{frametitle}{fg=white, bg=myblue}
\setbeamercolor{title}{fg=myblue}
\setbeamercolor{section in toc}{fg=myblue}
\setbeamercolor{item projected}{fg=white, bg=myblue}
\setbeamercolor{block title}{bg=myblue!20, fg=myblue}
\setbeamercolor{block body}{bg=myblue!10}
\setbeamercolor{alerted text}{fg=myorange}

% Set Fonts
\setbeamerfont{title}{size=\Large, series=\bfseries}
\setbeamerfont{frametitle}{size=\large, series=\bfseries}
\setbeamerfont{caption}{size=\small}
\setbeamerfont{footnote}{size=\tiny}

% Document Start
\begin{document}

\frame{\titlepage}

\begin{frame}[fragile]
    \maketitle
\end{frame}

\begin{frame}[fragile]
    \frametitle{What is Data Quality?}
    Data quality refers to the condition of a dataset, specifically its suitability for its intended purpose. High-quality data ensures that the information can be trusted and is relevant for decision-making.
\end{frame}

\begin{frame}[fragile]
    \frametitle{Importance of Data Quality}
    \begin{itemize}
        \item \textbf{Accuracy}: Data must accurately reflect the real-world construct it is intended to represent.
        \item \textbf{Integrity}: Ensures that relationships within the data remain intact.
        \item \textbf{Consistency}: Data should be uniform and consistent across all datasets.
        \item \textbf{Completeness}: All necessary information must be present.
        \item \textbf{Timeliness}: Data must be up-to-date.
    \end{itemize}
\end{frame}

\begin{frame}[fragile]
    \frametitle{The Role of Data Validation}
    Data validation is the process of ensuring that data is of high quality before it enters the data processing systems. It acts as a checkpoint to maintain data integrity and reduce errors. Validation can occur through:
    \begin{itemize}
        \item \textbf{Automated checks}: Using algorithms to check data against defined rules.
        \item \textbf{User verification}: Having individuals review data to confirm its accuracy.
    \end{itemize}
\end{frame}

\begin{frame}[fragile]
    \frametitle{Why is Data Quality Essential?}
    \begin{enumerate}
        \item \textbf{Preventing Decision-Making Errors}:
            \begin{itemize}
                \item *Example*: A retailer relying on inaccurate sales projections may overstock or understock products.
            \end{itemize}
        \item \textbf{Enhancing Operational Efficiency}:
            \begin{itemize}
                \item *Example*: Accurate contact lists ensure marketing campaigns reach the right audience.
            \end{itemize}
        \item \textbf{Regulatory Compliance}:
            \begin{itemize}
                \item *Example*: Financial institutions must maintain accurate records to comply with regulations.
            \end{itemize}
    \end{enumerate}
\end{frame}

\begin{frame}[fragile]
    \frametitle{Key Points to Emphasize}
    \begin{itemize}
        \item Data quality is a critical business concern.
        \item The cost of poor data quality can exceed the investment in quality assurance.
        \item Ongoing monitoring and validation are essential for maintaining data quality.
    \end{itemize}
\end{frame}

\begin{frame}[fragile]
    \frametitle{Illustrative Example}
    Imagine a company launching a new product. If their customer data were of low quality:
    \begin{itemize}
        \item Improper targeting could result in wasted marketing dollars.
        \item Misleading sales forecasts could lead to production issues.
    \end{itemize}
    By implementing strong data quality measures, the company ensures effective audience reach and optimized operations.
\end{frame}

\begin{frame}[fragile]
    \frametitle{Conclusion}
    By focusing on data quality and validation, organizations can safeguard their data's integrity, leading to better insights and informed decision-making. Effective data management is essential for navigating the complexities of modern data landscapes.
\end{frame}

\begin{frame}[fragile]
    \frametitle{Definitions of Data Quality - Overview}
    \begin{block}{What is Data Quality?}
        Data quality refers to the condition of a dataset based on several characteristics that determine its suitability for a particular purpose. 
        High-quality data is:
        \begin{itemize}
            \item Accurate
            \item Complete
            \item Consistent
            \item Timely
            \item Unique
        \end{itemize}
    \end{block}
    \begin{block}{Importance}
        Ensuring high data quality is crucial for making informed decisions and driving business insights.
    \end{block}
\end{frame}

\begin{frame}[fragile]
    \frametitle{Dimensions of Data Quality - Accuracy and Completeness}
    \begin{enumerate}
        \item \textbf{Accuracy}
        \begin{itemize}
            \item Measures how closely data values reflect true or real-world values.
            \item \textit{Example:} A customer database lists John Doe's age as 35 when he is actually 30.
            \item \textit{Key Point:} Ensuring data accuracy is vital for analyses relying on precise information.
        \end{itemize}
        
        \item \textbf{Completeness}
        \begin{itemize}
            \item Indicates whether all required data is present; missing values can limit analysis effectiveness.
            \item \textit{Example:} A sales dataset is incomplete if some transaction records are missing.
            \item \textit{Key Point:} Always verify that all necessary fields are filled to enhance analytical outcomes.
        \end{itemize}
    \end{enumerate}
\end{frame}

\begin{frame}[fragile]
    \frametitle{Dimensions of Data Quality - Consistency, Timeliness, and Uniqueness}
    \begin{enumerate}
        \setcounter{enumi}{2}
        \item \textbf{Consistency}
        \begin{itemize}
            \item Refers to the uniformity of data across datasets; data that are consistent do not contradict each other.
            \item \textit{Example:} A customer's address is "123 Main St" in one database and "123 Main Street" in another.
            \item \textit{Key Point:} Regular audits can help identify and resolve inconsistencies across data sources.
        \end{itemize}
        
        \item \textbf{Timeliness}
        \begin{itemize}
            \item Assesses whether data is up-to-date and available when needed; outdated data can lead to incorrect conclusions.
            \item \textit{Example:} Using sales data from 2019 may not accurately represent current market conditions.
            \item \textit{Key Point:} Data should be regularly updated to ensure analyses reflect the current state of affairs.
        \end{itemize}

        \item \textbf{Uniqueness}
        \begin{itemize}
            \item Ensures each record in a dataset is distinct without unnecessary duplicates.
            \item \textit{Example:} Two identical entries for the same customer can mislead analytics dependent on frequency.
            \item \textit{Key Point:} Implement deduplication processes during data entry to maintain uniqueness of records.
        \end{itemize}
    \end{enumerate}
\end{frame}

\begin{frame}[fragile]
    \frametitle{Summary and Diagram Suggestion}
    \begin{block}{Summary}
        Understanding the dimensions of data quality is critical for effective data management and analysis. Always strive for high-quality datasets to ensure valuable and actionable insights.
    \end{block}
    \begin{block}{Diagram/Illustration Suggestion}
        A flowchart illustrating how data enters the pipeline, highlighting checks for:
        \begin{itemize}
            \item Accuracy
            \item Completeness
            \item Consistency
            \item Timeliness
            \item Uniqueness
        \end{itemize}
    \end{block}
\end{frame}

\begin{frame}[fragile]
    \frametitle{Significance of Data Quality - Introduction}
    \begin{block}{What is Data Quality?}
        Data quality refers to the condition of a dataset based on various dimensions such as:
        \begin{itemize}
            \item Accuracy
            \item Completeness
            \item Consistency
            \item Timeliness
            \item Uniqueness
        \end{itemize}
        High data quality is essential for effective decision-making, smooth business operations, and successful analytics outcomes.
    \end{block}
    Poor data quality can lead to significant setbacks impacting organizations across various sectors.
\end{frame}

\begin{frame}[fragile]
    \frametitle{Impact of Poor Data Quality - Decision Making}
    \begin{block}{Consequences in Decision Making}
        \begin{itemize}
            \item \textbf{Misleading Information:} Inaccurate data misleads managers, leading to faulty conclusions.
            \item \textbf{Loss of Trust:} Persistent quality issues degrade stakeholder confidence in data-driven decisions.
        \end{itemize}
        \begin{example}
            A marketing department using flawed demographic data may launch ineffective campaigns.
        \end{example}
    \end{block}
\end{frame}

\begin{frame}[fragile]
    \frametitle{Impact of Poor Data Quality - Operations and Analytics}
    \begin{block}{Consequences in Business Operations}
        \begin{itemize}
            \item \textbf{Increased Costs:} Resources spent fixing errors lead to higher operational costs.
            \item \textbf{Inefficiency:} Bad data leads to repeated challenges, such as incorrect inventory counts.
        \end{itemize}
        \begin{example}
            A retail company with outdated inventory data may face canceled orders and customer dissatisfaction.
        \end{example}
    \end{block}

    \begin{block}{Consequences in Analytics Outcomes}
        \begin{itemize}
            \item \textbf{Inconsistent Insights:} Conflicting insights complicate analytical processes.
            \item \textbf{Inaccurate Predictions:} Low-quality data yields unreliable predictions.
        \end{itemize}
        \begin{example}
            A predictive analytics model for customer churn using biased data can result in misguided retention strategies.
        \end{example}
    \end{block}
\end{frame}

\begin{frame}[fragile]
    \frametitle{Key Points and Diagram}
    \begin{block}{Key Points to Emphasize}
        \begin{itemize}
            \item \textbf{Investing in Data Quality:} Essential for informed decisions, optimized operations, and enhanced analytics.
            \item \textbf{Continuous Monitoring:} Ongoing checks to prevent errors in datasets are necessary.
        \end{itemize}
    \end{block}

    \begin{block}{Diagram (Mental Model)}
        \begin{center}
            Poor Data Quality 
            \begin{itemize}
                \item Misleading Decisions
                \item Inefficient Operations
                \item Flawed Analytics Outcomes
            \end{itemize}
        \end{center}
    \end{block}

    By ensuring high-quality data, organizations can improve decision-making, streamline operations, and produce accurate analytics, essential for sustaining a competitive edge.
\end{frame}

\begin{frame}
    \frametitle{Data Validation Techniques}
    \begin{block}{Introduction to Data Validation}
        Data validation is a critical step in ensuring data quality, which directly impacts decision-making, business operations, and analysis outcomes. By validating data, we can catch errors before analysis and improve the reliability of results. Below are the primary techniques used for validating data:
    \end{block}
\end{frame}

\begin{frame}[fragile]
    \frametitle{Data Validation Techniques - Range Checks}
    \begin{itemize}
        \item \textbf{Definition:} A range check validates that data falls within a specified range.
        \item \textbf{Example:} If a dataset includes ages, a range check might ensure that all ages are between 0 and 120.
        \item \textbf{Implementation:} In Python, a simple range check could be implemented as follows:
    \end{itemize}
    
    \begin{lstlisting}[language=Python]
def is_valid_age(age):
    return 0 <= age <= 120
    \end{lstlisting}
\end{frame}

\begin{frame}[fragile]
    \frametitle{Data Validation Techniques - Format and Consistency Checks}
    \begin{itemize}
        \item \textbf{Format Checks:}
        \begin{itemize}
            \item \textbf{Definition:} Ensure that data adheres to a specified format or structure.
            \item \textbf{Example:} A format check for email addresses might verify that they contain an '@' symbol and a domain (e.g., “user@example.com”).
            \item \textbf{Implementation:} The following regular expression can be used for email validation in Python:
        \end{itemize}
        \begin{lstlisting}[language=Python]
import re

def is_valid_email(email):
    pattern = r'^[\w\.-]+@[\w\.-]+\.\w+$'
    return re.match(pattern, email) is not None
        \end{lstlisting}
        
        \item \textbf{Consistency Checks:}
        \begin{itemize}
            \item \textbf{Definition:} Ensure that data across different datasets or within a dataset does not conflict.
            \item \textbf{Example:} If a dataset contains a field for "start date" and another for "end date," ensure that "end date" is later than "start date."
            \item \textbf{Implementation:} A simple check could be as follows in Python:
        \end{itemize}
        \begin{lstlisting}[language=Python]
def is_start_before_end(start_date, end_date):
    return start_date < end_date
        \end{lstlisting}
    \end{itemize}
\end{frame}

\begin{frame}
    \frametitle{Key Points and Visual Aid}
    \begin{itemize}
        \item \textbf{Importance of Data Validation:} Employing these techniques reduces errors and improves data integrity.
        \item \textbf{Automation:} Many validation techniques can be automated using data validation libraries or features in programming languages.
        \item \textbf{Documentation:} Always document the validation rules applied to datasets for reference and audits.
    \end{itemize}

    \begin{block}{Visual Aid Suggestion}
        A flowchart illustrating the data validation process could enhance understanding. It would show how an input undergoes various validation checks (range, format, consistency) before reaching the analysis stage, highlighting decision points for each type of check.
    \end{block}
\end{frame}

\begin{frame}[fragile]
    \frametitle{Data Cleaning Process - Overview}
    \begin{block}{Overview}
        The data cleaning process is essential for ensuring the integrity, accuracy, and consistency of datasets. 
        We will explore the key steps involved in data cleaning and common techniques to remove inaccuracies and inconsistencies.
    \end{block}
\end{frame}

\begin{frame}[fragile]
    \frametitle{Data Cleaning Process - Key Steps}
    \begin{enumerate}
        \item \textbf{Data Profiling}
            \begin{itemize}
                \item \textbf{Definition}: Assessing the completeness, uniqueness, and consistency of the dataset.
                \item \textbf{Technique}: Summary statistics or data visualizations to identify potential issues.
                \item \textbf{Example}: Analyzing sales data for uniform dates within a defined range.
            \end{itemize}
        
        \item \textbf{Identifying Inaccuracies}
            \begin{itemize}
                \item \textbf{Definition}: Detecting incorrect, incomplete, or misleading data points.
                \item \textbf{Technique}: Consistency checks by cross-referencing values against an authoritative source.
                \item \textbf{Example}: Verifying customer addresses against postal databases.
            \end{itemize}
    \end{enumerate}
\end{frame}

\begin{frame}[fragile]
    \frametitle{Data Cleaning Process - More Key Steps}
    \begin{enumerate}[resume]
        \item \textbf{Handling Missing Values}
            \begin{itemize}
                \item \textbf{Definition}: Addressing gaps where data points are absent.
                \item \textbf{Techniques}:
                    \begin{itemize}
                        \item Imputation: Filling in missing values with mean, median, or mode.
                        \item Deletion: Removing records with too many missing fields.
                    \end{itemize}
                \item \textbf{Example}: Imputing mean age in a survey dataset.
            \end{itemize}

        \item \textbf{Removing Duplicates}
            \begin{itemize}
                \item \textbf{Definition}: Identifying and eliminating duplicate records.
                \item \textbf{Technique}: Using unique identifiers or algorithms to detect duplicates.
                \item \textbf{Example}: Combining entries for the same customer by unique identifiers.
            \end{itemize}

        \item \textbf{Standardizing Data Formats}
            \begin{itemize}
                \item \textbf{Definition}: Ensuring consistency in data representation.
                \item \textbf{Techniques}:
                    \begin{itemize}
                        \item Normalization: Converting data to a common format.
                        \item Categorization: Transforming continuous variables into categorical ones.
                    \end{itemize}
                \item \textbf{Example}: Using the YYYY-MM-DD format for date entries.
            \end{itemize}
    \end{enumerate}
\end{frame}

\begin{frame}[fragile]
    \frametitle{Data Cleaning Process - Final Steps and Summary}
    \begin{enumerate}[resume]
        \item \textbf{Validation Checks}
            \begin{itemize}
                \item \textbf{Definition}: Verifying that data meets predefined rules or conditions.
                \item \textbf{Technique}: Implementing range checks for values.
                \item \textbf{Example}: Ensuring transaction amounts are not negative.
            \end{itemize}

    \end{enumerate}

    \begin{block}{Summary}
        \begin{itemize}
            \item Data cleaning is crucial in data analytics.
            \item Techniques enhance the quality of datasets for reliable insights.
            \item Regular audits ensure ongoing data quality throughout the data lifecycle.
        \end{itemize}
    \end{block}
\end{frame}

\begin{frame}[fragile]
    \frametitle{Common Data Quality Issues - Overview}
    \textbf{Objective:} Understand and identify prevalent data quality issues that can compromise the integrity of datasets.
    
    \begin{itemize}
        \item Duplicate Records
        \item Missing Values
        \item Incorrect Data Entries
    \end{itemize}
\end{frame}

\begin{frame}[fragile]
    \frametitle{Common Data Quality Issues - Duplicate Records}
    \begin{block}{1. Duplicate Records}
        \textbf{Explanation:} 
        Duplicate records occur when the same data entry appears multiple times within a dataset. This can happen due to data entry errors or merging datasets from disparate sources.
        
        \textbf{Example:} 
        Imagine a customer database with two entries for the same individual:
        \begin{itemize}
            \item John Doe, 123 Main St, johndoe@email.com
            \item John Doe, 123 Main St, johndoe@email.com
        \end{itemize}

        \textbf{Key Points:}
        \begin{itemize}
            \item Duplicates skew analysis results, leading to overestimation in metrics (e.g., sales numbers).
            \item Use data deduplication techniques, such as filtering by unique identifiers (e.g., email address).
        \end{itemize}
    \end{block}
\end{frame}

\begin{frame}[fragile]
    \frametitle{Common Data Quality Issues - Missing Values and Incorrect Entries}
    \begin{block}{2. Missing Values}
        \textbf{Explanation:} 
        Missing values occur when data points are absent from a dataset due to collection errors or incomplete entries.
        
        \textbf{Example:} 
        In a survey dataset, one entry may look like:
        \begin{itemize}
            \item Customer ID: 001, Age: 30, Gender: Female, Rating: 
        \end{itemize}

        \textbf{Key Points:}
        \begin{itemize}
            \item Missing values can lead to biased analysis.
            \item Techniques for handling:
            \begin{itemize}
                \item \textbf{Deletion}: Remove records with missing values.
                \item \textbf{Imputation}: Replace missing values with estimates (mean, median, mode).
            \end{itemize}
        \end{itemize}
    \end{block}
    
    \begin{block}{3. Incorrect Data Entries}
        \textbf{Explanation:} 
        Incorrect data entries occur when data is input incorrectly, leading to inaccuracies.
        
        \textbf{Example:} 
        An employee dataset may include:
        \begin{itemize}
            \item Employee ID: 12345, Name: Jane Smith, Salary: $50,000
            \item Incorrectly recorded as: Employee ID: 12345, Name: Jane Smith, Salary: fifty thousand dollars
        \end{itemize}

        \textbf{Key Points:}
        \begin{itemize}
            \item Incorrect entries can distort analysis and reporting.
            \item Regular validation and verification (data type checks, range validations) help prevent issues.
        \end{itemize}
    \end{block}
\end{frame}

\begin{frame}[fragile]
    \frametitle{Tools and Frameworks for Data Validation - Overview}
    \begin{block}{Importance of Data Validation}
        Data validation is essential in ensuring that datasets are accurate, complete, and reliable before analysis and decision-making. Various tools and frameworks facilitate this process, helping data professionals manage and enhance data quality effectively.
    \end{block}
    \begin{block}{Focus of Presentation}
        In this overview, we will discuss:
        \begin{itemize}
            \item Apache Spark
            \item Pandas
        \end{itemize}
    \end{block}
\end{frame}

\begin{frame}[fragile]
    \frametitle{Tools and Frameworks for Data Validation - Apache Spark}
    \begin{block}{Description}
        Apache Spark is an open-source distributed computing system designed for processing large volumes of data quickly. It supports various programming languages and provides a robust framework for data validation tasks.
    \end{block}
    \begin{block}{Capabilities}
        \begin{itemize}
            \item \textbf{Data Processing}: Handles big data through distributed computing, enabling quick validations across massive datasets.
            \item \textbf{DataFrame API}: Offers easy-to-use APIs for data manipulation, allowing users to filter and validate data efficiently.
            \item \textbf{Integration}: Works seamlessly with various data sources (e.g., HDFS, Spark SQL) making it easier to fetch and validate data from diverse environments.
        \end{itemize}
    \end{block}
    \begin{block}{Example}
        \begin{lstlisting}[language=Python, basicstyle=\footnotesize]
from pyspark.sql import SparkSession

spark = SparkSession.builder.appName("DataValidation").getOrCreate()
df = spark.read.csv("data.csv", header=True)

# Check for missing values
missing_count = df.filter(df['column_name'].isNull()).count()
print(f"Missing values in column_name: {missing_count}")
        \end{lstlisting}
    \end{block}
\end{frame}

\begin{frame}[fragile]
    \frametitle{Tools and Frameworks for Data Validation - Pandas}
    \begin{block}{Description}
        Pandas is a powerful Python library for data analysis and manipulation. It's particularly suited for small to medium datasets and provides functionalities for data cleaning and validation.
    \end{block}
    \begin{block}{Capabilities}
        \begin{itemize}
            \item \textbf{Data Structures}: Offers Series and DataFrame objects, making it easy to handle structured data.
            \item \textbf{Validation Functions}: Contains built-in methods to check for duplicates, missing data, and apply custom validation logic.
            \item \textbf{Visualization}: Built-in capabilities to visualize data trends, which can hint at potential validation issues.
        \end{itemize}
    \end{block}
    \begin{block}{Example}
        \begin{lstlisting}[language=Python, basicstyle=\footnotesize]
import pandas as pd

# Load the dataset
df = pd.read_csv("data.csv")

# Validate for missing values
missing_values = df.isnull().sum()
print(f"Missing values:\n{missing_values}")

# Check for duplicates
duplicates = df.duplicated().sum()
print(f"Duplicate records: {duplicates}")
        \end{lstlisting}
    \end{block}
\end{frame}

\begin{frame}[fragile]
    \frametitle{Key Points and Conclusion}
    \begin{block}{Key Points to Emphasize}
        \begin{itemize}
            \item Data validation is critical for ensuring data quality, impacting analytics and decision-making.
            \item \textbf{Apache Spark} is ideal for big data environments due to its distributed nature and scalability.
            \item \textbf{Pandas} is best suited for smaller datasets, offering simpler and more intuitive data manipulation tools.
            \item Both tools provide functions to identify and rectify common data quality issues.
        \end{itemize}
    \end{block}
    \begin{block}{Conclusion}
        Efficient data validation frameworks like Apache Spark and Pandas empower data professionals to maintain high data quality standards. Leveraging these tools will enhance analytics efforts and improve overall decision-making processes.
    \end{block}
\end{frame}

\begin{frame}[fragile]
    \frametitle{Case Studies on Data Quality - Part 1}
    
    \begin{block}{Understanding Data Quality}
        Data quality refers to the condition of data based on several factors:
        \begin{itemize}
            \item Accuracy
            \item Consistency
            \item Completeness
            \item Reliability
            \item Relevance
        \end{itemize}
        Poor data quality can lead to:
        \begin{itemize}
            \item Erroneous conclusions
            \item Inefficient processes
            \item Substantial financial losses
        \end{itemize}
        It is crucial for organizations to prioritize data validation strategies for high-quality data.
    \end{block}
\end{frame}

\begin{frame}[fragile]
    \frametitle{Case Studies on Data Quality - Part 2}
    
    \begin{block}{Consequences of Poor Data Quality}
        \textbf{Case Study 1: Target's Customer Data Breach}
        \begin{itemize}
            \item \textbf{Background}: In 2013, Target Corporation suffered a data breach involving sensitive information of 40 million credit and debit card accounts.
            \item \textbf{Failure}: Lack of proper data validation allowed hackers to infiltrate systems through unverified customer data.
            \item \textbf{Consequence}: The breach cost Target over \$200 million in legal fees and public relations costs, severely damaging their reputation.
        \end{itemize}
        \textbf{Key Point}: Poor data quality leads to financial loss and damages customer trust, which can take years to rebuild.
    \end{block}
\end{frame}

\begin{frame}[fragile]
    \frametitle{Case Studies on Data Quality - Part 3}

    \begin{block}{Successful Validation Strategies}
        \textbf{Case Study 2: Air France's Data Initiative}
        \begin{itemize}
            \item \textbf{Background}: Air France developed an advanced data validation strategy utilizing Apache Spark and Pandas.
            \item \textbf{Strategy}:
            \begin{itemize}
                \item Automated Testing: Implemented automated data quality tests to identify anomalies in real-time.
                \item Data Profiling: Conducted extensive profiling to ensure conformity with established quality metrics.
            \end{itemize}
            \item \textbf{Outcome}: Resulted in a 30\% decrease in data inconsistencies and improved operational efficiency, enhancing customer satisfaction.
        \end{itemize}
        \textbf{Key Point}: Investing in technology for systematic data validation can drastically improve data quality and trust in data-driven decisions.
    \end{block}
\end{frame}

\begin{frame}[fragile]
    \frametitle{Performance Metrics for Data Quality}
    \begin{block}{Understanding Data Quality Metrics}
        Data quality metrics are the standards and measures used to evaluate quality across various datasets. They provide insights into:
        \begin{itemize}
            \item Accuracy
            \item Completeness
            \item Consistency
            \item Reliability
        \end{itemize}
        These metrics are crucial for making informed decisions in data processing.
    \end{block}
\end{frame}

\begin{frame}[fragile]
    \frametitle{Key Performance Metrics}

    \begin{enumerate}
        \item \textbf{Accuracy}
            \begin{itemize}
                \item Measures data values' correspondence to true values.
                \item High accuracy minimizes analysis errors.
                \item Example: Correct ages in a dataset.
            \end{itemize}
        
        \item \textbf{Completeness}
            \begin{itemize}
                \item Assesses presence of all required data.
                \item Missing values can lead to biased conclusions.
                \item Example: Customer records should have all fields filled.
            \end{itemize}

        \item \textbf{Consistency}
            \begin{itemize}
                \item Determines if data is consistent across datasets.
                \item Inconsistent data can create confusion.
                \item Example: Gender recorded as 'Male' vs. 'M'.
            \end{itemize}

        \item \textbf{Timeliness}
            \begin{itemize}
                \item Evaluates if data is up-to-date.
                \item Outdated data leads to poor decision-making.
                \item Example: Frequent updates on stock prices.
            \end{itemize}

        \item \textbf{Validity}
            \begin{itemize}
                \item Measures if data values fall within defined ranges.
                \item Ensures adherence to rules.
                \item Example: Age should range from 0 to 120.
            \end{itemize}
    \end{enumerate}
\end{frame}

\begin{frame}[fragile]
    \frametitle{Importance of Metrics in Data Processing Strategies}
    \begin{itemize}
        \item \textbf{Informed Decision-Making:} Ensures data reliability and validity, reducing risks.
        \item \textbf{Improvement of Data Quality:} Regular evaluation helps identify weaknesses for targeted improvements.
        \item \textbf{Operational Efficiency:} Metrics streamline processes and reduce costs associated with poor data.
        \item \textbf{Regulatory Compliance:} Assists in meeting industry standards for data quality.
    \end{itemize}

    \begin{block}{Key Points to Remember}
        \begin{itemize}
            \item Evaluate multiple metrics for a comprehensive view.
            \item Use automated tools for efficiency.
            \item Proactively monitor metrics for continuous improvement.
        \end{itemize}
    \end{block}
\end{frame}

\begin{frame}
    \frametitle{Practical Applications and Lab Session}
    \begin{block}{Overview}
        In this section, we will engage in hands-on exercises focused on data quality validation. We will employ practical examples that apply the concepts discussed in our previous sessions, particularly regarding performance metrics for data quality.
    \end{block}
    \begin{itemize}
        \item Strong emphasis on real-world datasets
        \item Application of theoretical knowledge
    \end{itemize}
\end{frame}

\begin{frame}
    \frametitle{Learning Objectives}
    \begin{itemize}
        \item Understand the importance of data quality validation in data processing.
        \item Gain experience with real datasets to practice data quality assessments.
        \item Learn how to implement validation techniques using popular data analysis tools.
    \end{itemize}
\end{frame}

\begin{frame}[fragile]
    \frametitle{Hands-On Exercises: Exercise 1}
    \textbf{Exercise 1: Data Profiling}
    \begin{itemize}
        \item \textbf{Objective:} Conduct a data profiling exercise on a sample dataset.
        \item \textbf{Description:} Load a dataset (e.g., a CSV file containing sales data) and generate summary statistics.
        \item \textbf{Key Metrics to Calculate:}
            \begin{itemize}
                \item Count of Missing Values
                \item Unique Value Counts
                \item Descriptive Statistics (mean, median, mode)
            \end{itemize}
        \item \textbf{Tools:} Use Python with Pandas.
    \end{itemize}
    \begin{lstlisting}[language=Python]
import pandas as pd

# Load dataset
df = pd.read_csv('sales_data.csv')

# Generate summary statistics
profile = df.describe()
missing_values = df.isnull().sum()
unique_values = df.nunique()

print(profile)
print(missing_values)
print(unique_values)
    \end{lstlisting}
\end{frame}

\begin{frame}[fragile]
    \frametitle{Hands-On Exercises: Exercise 2 and 3}
    \textbf{Exercise 2: Data Cleansing}
    \begin{itemize}
        \item \textbf{Objective:} Clean a dataset by addressing identified data quality issues.
        \item \textbf{Description:} Perform data cleansing by:
            \begin{itemize}
                \item Removing duplicates
                \item Filling or dropping missing values
                \item Standardizing inconsistent entries (e.g., date formats)
            \end{itemize}
        \item \textbf{Tools:} Python with Pandas.
    \end{itemize}
    \begin{lstlisting}[language=Python]
# Remove duplicates
df = df.drop_duplicates()

# Fill missing values with mean for numerical columns
df['sales_amount'].fillna(df['sales_amount'].mean(), inplace=True)

# Standardize date format
df['order_date'] = pd.to_datetime(df['order_date']).dt.strftime('%Y-%m-%d')
    \end{lstlisting}
    
    \textbf{Exercise 3: Data Validation Techniques}
    \begin{itemize}
        \item \textbf{Objective:} Implement validation rules to ensure data accuracy and reliability.
        \item \textbf{Tools:} Use Python with NumPy.
    \end{itemize}
    \begin{lstlisting}[language=Python]
from scipy import stats

# Outlier detection using Z-score
z_scores = stats.zscore(df['sales_amount'])
abs_z_scores = abs(z_scores)
filtered_entries = (abs_z_scores < 3)
df_cleaned = df[filtered_entries]
    \end{lstlisting}
\end{frame}

\begin{frame}
    \frametitle{Key Points to Emphasize}
    \begin{itemize}
        \item Data quality validation is crucial for accurate analyses and decision-making.
        \item Practical exercises help solidify knowledge through application.
        \item Using real datasets allows students to experience real-world challenges in data handling.
    \end{itemize}
\end{frame}

\begin{frame}[fragile]
    \frametitle{Conclusion and Future Trends - Key Takeaways}
    
    \begin{block}{Importance of Data Quality}
        Data quality directly impacts decision-making processes, analytics outcomes, and the effectiveness of machine learning models. 
        Quality data leads to accurate insights and reliable predictions. 
        \textit{Example}: In healthcare, patient records must be accurate and complete for effective treatment decisions.
    \end{block}

    \begin{block}{Dimensions of Data Quality}
        \begin{itemize}
            \item \textbf{Accuracy}: Data correctly reflects the real-world scenario.
            \item \textbf{Completeness}: All required data is present.
            \item \textbf{Consistency}: Data is reliable across different datasets.
            \item \textbf{Timeliness}: Data is up-to-date and available when needed.
            \item \textbf{Uniqueness}: No data duplication occurs.
        \end{itemize}
    \end{block}
\end{frame}

\begin{frame}[fragile]
    \frametitle{Conclusion and Future Trends - Data Validation Techniques}

    \begin{block}{Data Validation Techniques}
        \begin{itemize}
            \item Implement validation rules (e.g., mandatory fields, data type checks).
            \item Use automated tools for anomaly detection.
            \item Employ manual review processes for complex data sets, especially in critical applications such as finance and healthcare.
        \end{itemize}
    \end{block}

    \begin{block}{Real-World Application}
        \textit{Example}: In e-commerce, validating customer data (like email addresses and shipping information) reduces fraud and improves user experience.
    \end{block}
\end{frame}

\begin{frame}[fragile]
    \frametitle{Conclusion and Future Trends - Future Trends in Data Quality Management}
    
    \begin{block}{AI and Machine Learning}
        \begin{itemize}
            \item AI-driven algorithms are being increasingly used for automated data cleaning and validation.
            \item Systems learn from past errors, improving their accuracy over time.
            \item \textit{Example}: Using machine learning to detect fraud in financial transactions by flagging anomalies.
        \end{itemize}
    \end{block}

    \begin{block}{Continuous Monitoring}
        \begin{itemize}
            \item Transitioning to continuous monitoring frameworks to assess data quality in real-time.
            \item \textit{Example}: Implementing dashboards that monitor data quality metrics continuously.
        \end{itemize}
    \end{block}
    
    \begin{block}{Integration of Data Quality Tools}
        A rise in platforms that combine data management, quality, and governance tools will ensure all stakeholders access quality data seamlessly.
    \end{block}
\end{frame}


\end{document}