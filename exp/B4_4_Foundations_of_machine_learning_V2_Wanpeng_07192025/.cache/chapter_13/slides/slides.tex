\documentclass[aspectratio=169]{beamer}

% Theme and Color Setup
\usetheme{Madrid}
\usecolortheme{whale}
\useinnertheme{rectangles}
\useoutertheme{miniframes}

% Additional Packages
\usepackage[utf8]{inputenc}
\usepackage[T1]{fontenc}
\usepackage{graphicx}
\usepackage{booktabs}
\usepackage{listings}
\usepackage{amsmath}
\usepackage{amssymb}
\usepackage{xcolor}
\usepackage{tikz}
\usepackage{pgfplots}
\pgfplotsset{compat=1.18}
\usetikzlibrary{positioning}
\usepackage{hyperref}

% Custom Colors
\definecolor{myblue}{RGB}{31, 73, 125}
\definecolor{mygray}{RGB}{100, 100, 100}
\definecolor{mygreen}{RGB}{0, 128, 0}
\definecolor{myorange}{RGB}{230, 126, 34}
\definecolor{mycodebackground}{RGB}{245, 245, 245}

% Set Theme Colors
\setbeamercolor{structure}{fg=myblue}
\setbeamercolor{frametitle}{fg=white, bg=myblue}
\setbeamercolor{title}{fg=myblue}
\setbeamercolor{section in toc}{fg=myblue}
\setbeamercolor{item projected}{fg=white, bg=myblue}
\setbeamercolor{block title}{bg=myblue!20, fg=myblue}
\setbeamercolor{block body}{bg=myblue!10}
\setbeamercolor{alerted text}{fg=myorange}

% Set Fonts
\setbeamerfont{title}{size=\Large, series=\bfseries}
\setbeamerfont{frametitle}{size=\large, series=\bfseries}
\setbeamerfont{caption}{size=\small}
\setbeamerfont{footnote}{size=\tiny}

% Document Start
\begin{document}

\frame{\titlepage}

\begin{frame}[fragile]
    \frametitle{Introduction to Reinforcement Learning}
    \begin{block}{Overview of Reinforcement Learning (RL)}
        Reinforcement Learning is a type of machine learning where an agent learns to make decisions by taking actions within an environment to maximize cumulative rewards. Unlike supervised learning, RL involves learning through interaction, trial, and error.
    \end{block}
\end{frame}

\begin{frame}[fragile]
    \frametitle{Key Components of Reinforcement Learning}
    \begin{enumerate}
        \item \textbf{Agent}: The learner or decision-maker (e.g., a robot, software program).
        \item \textbf{Environment}: The external system with which the agent interacts (e.g., a game, real-world situation).
        \item \textbf{Actions}: Choices made by the agent that affect the state of the environment.
        \item \textbf{States}: The current situation of the agent within the environment.
        \item \textbf{Rewards}: Feedback from the environment based on the actions taken by the agent, guiding learning toward beneficial behaviors.
    \end{enumerate}
\end{frame}

\begin{frame}[fragile]
    \frametitle{Significance of Reinforcement Learning}
    \begin{itemize}
        \item \textbf{Autonomous Learning}: Allows machines to learn from experiences without explicit programming.
        \item \textbf{Adaptability}: Agents adapt to changing environments, making RL suitable for dynamic contexts.
        \item \textbf{Real-World Applications}: 
        \begin{itemize}
            \item \textbf{Gaming}: Superhuman performance in games like Chess or Go.
            \item \textbf{Robotics}: Learning to perform tasks such as navigation and manipulation.
            \item \textbf{Finance}: Developing trading strategies by learning from market behaviors.
        \end{itemize}
    \end{itemize}
\end{frame}

\begin{frame}[fragile]
    \frametitle{Example: The Game of Chess}
    In a game of chess:
    \begin{itemize}
        \item The agent (AI) chooses moves based on its assessment of the state (current board configuration).
        \item It receives a reward (win, loss, draw) that shapes its future decisions to improve gameplay.
    \end{itemize}
\end{frame}

\begin{frame}[fragile]
    \frametitle{Formulas and Concepts}
    \begin{block}{Return}
        The total discounted reward received over time, calculated as:
        \begin{equation}
            G_t = R_{t+1} + \gamma R_{t+2} + \gamma^2 R_{t+3} + ... 
        \end{equation}
        where \( \gamma \) (0 ≤ \( \gamma \) < 1) is the discount factor prioritizing immediate rewards.
    \end{block}
\end{frame}

\begin{frame}[fragile]
    \frametitle{Conclusion}
    Reinforcement Learning is at the forefront of AI advancements, enabling systems to learn autonomously and make informed decisions based on experience, thereby shaping the future of smart technologies.
\end{frame}

\begin{frame}{Presentation Overview}
  \tableofcontents[hideallsubsections]
\end{frame}

\begin{frame}{Introduction to Core Concepts}
    Reinforcement Learning (RL) is a type of machine learning where an agent learns to make decisions by taking actions in an environment to maximize cumulative rewards over time. Understanding the following core concepts is crucial to grasp the mechanics of RL.
\end{frame}

\begin{frame}{Core Concepts Explained}
    \begin{itemize}
        \item \textbf{Agent}:
          \begin{itemize}
              \item The learner or decision-maker in the RL scenario.
              \item Example: An algorithm controlling a character in a video game.
          \end{itemize}
          
        \item \textbf{Environment}:
          \begin{itemize}
              \item Everything the agent interacts with; it defines the context.
              \item Example: The game, including terrain and characters.
          \end{itemize}

        \item \textbf{Actions}:
          \begin{itemize}
              \item The set of all possible moves the agent can take.
              \item Example: Moving a pawn or castling in chess. 
          \end{itemize}

        \item \textbf{States}:
          \begin{itemize}
              \item A snapshot of the environment that provides information for decision-making.
              \item Example: The arrangement of chess pieces on the board.
          \end{itemize}

        \item \textbf{Rewards}:
          \begin{itemize}
              \item A feedback signal received after taking an action in a specific state, 
                usually a numerical value.
              \item Example: Scoring points in a video game after completing a level.
          \end{itemize}
    \end{itemize}
\end{frame}

\begin{frame}{Key Points to Emphasize}
    \begin{itemize}
        \item \textbf{Temporal Aspect}: The agent learns over time from the consequences of past actions.
        \item \textbf{Exploration vs. Exploitation}: Balancing exploration of new actions with exploiting known rewarding actions.
        \item \textbf{Objective}: The goal is to develop a policy that maximizes the expected cumulative reward.
    \end{itemize}
\end{frame}

\begin{frame}{Related Formulation}
    In formal terms, Reinforcement Learning can be structured as a Markov Decision Process (MDP), where:
    
    \begin{itemize}
        \item The agent interacts with the environment at discrete time steps, \( t \).
        \item States, actions, and rewards can be represented by:
    \end{itemize}
    
    \begin{equation}
      R(s, a) = \textit{reward for taking action } a \textit{ in state } s
    \end{equation}
\end{frame}

\begin{frame}[fragile]{Code Snippet Example}
    Consider a simplified representation of an RL agent's decision process:
    
    \begin{lstlisting}[language=Python]
class RLAgent:
    def __init__(self):
        self.state = None
        self.policy = {}  # Mapping from states to actions

    def choose_action(self, state):
        # Implementing a simple policy based on current state
        return self.policy.get(state, 'default_action')

    def receive_reward(self, reward):
        # Update agent's understanding based on received reward
        pass
    \end{lstlisting}
\end{frame}

\begin{frame}{Summary}
    Mastering these core concepts of agent, environment, actions, states, and rewards lays the foundation for understanding more advanced topics in Reinforcement Learning, including learning algorithms and policy optimization techniques.
\end{frame}

\begin{frame}[fragile]
  \frametitle{Types of Reinforcement Learning - Overview}
  Reinforcement Learning (RL) is a paradigm in machine learning where an agent learns to make decisions by interacting with an environment.
  
  This slide discusses two primary types of RL:
  \begin{itemize}
      \item \textbf{Model-Based Learning}
      \item \textbf{Model-Free Learning}
  \end{itemize}
\end{frame}

\begin{frame}[fragile]
  \frametitle{Types of Reinforcement Learning - Model-Based Learning}
  
  \textbf{Definition:}
  In model-based reinforcement learning, the agent builds an internal model of the environment to predict future states and expected rewards.
  
  \textbf{Key Characteristics:}
  \begin{itemize}
      \item \textbf{Planning:} Allows simulation of future actions.
      \item \textbf{Learning from Fewer Samples:} Reduces the number of interactions needed with the environment.
  \end{itemize}
  
  \textbf{Example:}
  Consider a chess-playing AI that simulates moves and outcomes to select the best one.
  
  \textbf{Illustration:}
  Imagine a driver using a roadmap to predict traffic conditions based on past experiences.
\end{frame}

\begin{frame}[fragile]
  \frametitle{Types of Reinforcement Learning - Model-Free Learning}

  \textbf{Definition:}
  In model-free reinforcement learning, the agent learns directly from interactions without creating a model of the environment.
  
  \textbf{Key Characteristics:}
  \begin{itemize}
      \item \textbf{Simplicity:} Easier to implement; focuses on direct experience.
      \item \textbf{Exploration vs. Exploitation:} Balances trying new actions and using known high-reward actions.
  \end{itemize}

  \textbf{Examples:}
  \begin{itemize}
      \item \textbf{Q-Learning:} Learns values of action-state pairs directly.
      \item \textbf{SARSA:} Updates Q-values based on actual actions taken.
  \end{itemize}

  \textbf{Illustration:}
  Imagine a robot navigating a maze by trial and error and refining its strategy over time.
\end{frame}

\begin{frame}[fragile]
  \frametitle{Markov Decision Processes (MDPs)}
  
  \begin{block}{Definition}
    A Markov Decision Process (MDP) is a mathematical framework used for modeling decision-making situations where outcomes are partly random and partly under the control of a decision-maker.
  \end{block}
  
  \begin{block}{Key Components}
    An MDP is defined by five key components:
  \end{block}
\end{frame}

\begin{frame}[fragile]
  \frametitle{Key Components of an MDP}
  
  \begin{enumerate}
    \item \textbf{States (S)}: A finite set of states that represent all possible configurations of the environment.
    \item \textbf{Actions (A)}: A finite set of possible actions available to the agent in each state.
    \item \textbf{Transition Function (P)}: Defines the probability of moving from one state to another given a specific action, represented as \( P(s'|s, a) \).
    \item \textbf{Reward Function (R)}: Assigns a numerical reward received after transitioning from one state to another due to an action, denoted as \( R(s, a) \).
    \item \textbf{Discount Factor ($\gamma$)}: A value between 0 and 1 that discounts future rewards.
  \end{enumerate}
\end{frame}

\begin{frame}[fragile]
  \frametitle{How MDPs Work}
  
  MDPs utilize a policy \( \pi \) that maps states to actions, guiding the agent in action selection.
  
  \begin{block}{Bellman Equation}
    A key feature is the Bellman equation, relating the value of a state to its successor states:
    \begin{equation}
      V(s) = R(s, \pi(s)) + \gamma \sum_{s'} P(s'|s, \pi(s)) V(s')
    \end{equation}
  \end{block}
  
  \begin{block}{Example of MDP in Practice}
    Consider a robot navigating a grid:
    \begin{itemize}
      \item States: Each cell in the grid.
      \item Actions: Moving up, down, left, or right.
      \item Transition Function: Probability of slipping while moving.
      \item Reward Function: +10 for reaching the goal, -10 for falling off a cliff.
      \item Discount Factor: Closer to 1 for immediate rewards.
    \end{itemize}
  \end{block}
\end{frame}

\begin{frame}[fragile]
  \frametitle{Conclusion}
  
  \begin{itemize}
    \item MDPs are foundational to reinforcement learning (RL).
    \item They model uncertainty through probabilistic transitions and rewards.
    \item Understanding MDPs is crucial for developing algorithms like Q-learning and Policy Gradient methods.
  \end{itemize}
  
  MDPs enable effective decision-making across a variety of applications from robotics to game playing.
\end{frame}

\begin{frame}[fragile]
    \frametitle{Exploration vs. Exploitation}
    \begin{block}{Understanding Exploration vs. Exploitation}
        In Reinforcement Learning (RL), agents face a fundamental dilemma: 
        \textbf{exploration} vs. \textbf{exploitation}.
    \end{block}
    \begin{itemize}
        \item \textbf{Exploration}: Actions to discover new strategies or gather information about the environment.
        \item \textbf{Exploitation}: Using known information to maximize the reward with the current knowledge.
    \end{itemize}
    \begin{block}{Key Concept}
        The goal of an RL agent is to maximize cumulative rewards over time, requiring a balance between exploring unknown actions and exploiting known, rewarding actions.
    \end{block}
\end{frame}

\begin{frame}[fragile]
    \frametitle{The Dilemma Illustrated}
    \begin{block}{Scenario}
        Imagine an agent navigating a maze:
        \begin{itemize}
            \item If it tries different paths (exploration), it might find a shortcut or the exit.
            \item If it always chooses the same path (exploitation), it may miss better routes or solutions.
        \end{itemize}
    \end{block}
    \begin{center}
        \texttt{[Exploration]} \\
        \texttt{    /   \\} 
        \texttt{   A     B} \\
        \texttt{  /       \\} 
        \texttt{ X         Y  [Exploitation]} \\
        \texttt{(A – known route, B – new route)}
    \end{center}
\end{frame}

\begin{frame}[fragile]
    \frametitle{Strategies to Balance Exploration and Exploitation}
    \begin{enumerate}
        \item \textbf{Epsilon-Greedy Strategy}:
        \begin{itemize}
            \item Choose best action with probability \(1 - \epsilon\) and explore a random action with probability \(\epsilon\).
            \item Adjust \(\epsilon\) over time.
        \end{itemize}
        \begin{equation}
            a_t = 
            \begin{cases} 
            \text{random action} & \text{with probability } \epsilon \\ 
            \text{argmax}_a Q(s, a) & \text{with probability } 1 - \epsilon 
            \end{cases}
        \end{equation}

        \item \textbf{UCB (Upper Confidence Bound)}:
        \begin{itemize}
            \item Balance between trying less explored actions and exploiting high-reward actions.
        \end{itemize}
        \begin{equation}
            a_t = \text{argmax}_a \left( 
            \overline{X}_a + c \sqrt{\frac{\ln T}{n_a}} 
            \right)
        \end{equation}
        
        \item \textbf{Softmax Action Selection}:
        \begin{itemize}
            \item Actions are chosen proportionate to their estimated value.
        \end{itemize}
        \begin{equation}
            P(a) = \frac{e^{Q(s,a)/\tau}}{\sum_{b} e^{Q(s,b)/\tau}}
        \end{equation}
    \end{enumerate}
\end{frame}

\begin{frame}[fragile]
    \frametitle{Key Takeaways}
    \begin{itemize}
        \item Balancing exploration and exploitation is crucial for effective learning in RL.
        \item Strategies like Epsilon-Greedy, UCB, and Softmax selection are widely used to strike this balance.
        \item Continuous assessment and adjustment of strategies based on the problem context can optimize agent performance.
    \end{itemize}
    \begin{block}{Conclusion}
        By understanding and applying these concepts, students can enhance their grasp of how RL agents learn and make decisions in complex environments.
    \end{block}
\end{frame}

\begin{frame}[fragile]
    \frametitle{Rewards in Reinforcement Learning - Understanding Rewards}
    In reinforcement learning (RL), \textbf{rewards} are critical feedback signals that inform an agent about the effectiveness of its actions in achieving a defined goal.
    
    \begin{block}{Key Concepts}
        \begin{enumerate}
            \item \textbf{Definition of Reward:} A scalar value given to an agent after it performs an action in a given state. It measures success or failure.
            \item \textbf{Types of Rewards:}
            \begin{itemize}
                \item \textbf{Immediate Rewards:} Provided right after an action (e.g., scoring a point).
                \item \textbf{Delayed Rewards:} Occur after several actions (e.g., points awarded at the end of a puzzle).
            \end{itemize}
            \item \textbf{Reward Structure:} Affects the learning process; sparse versus dense rewards lead to different trajectories.
        \end{enumerate}
    \end{block}
\end{frame}

\begin{frame}[fragile]
    \frametitle{Rewards in Reinforcement Learning - Impacts on Learning}
    The design of the reward structure can influence both the learning speed and the final performance of the RL agent.
    
    \begin{block}{Example: Maze Navigation}
        \begin{itemize}
            \item \textbf{Dense Reward:} Small reward for each step may lead to many small steps without considering the overall path.
            \item \textbf{Sparse Reward:} Large reward upon exiting the maze encourages focus on the significant goal, potentially delaying learning.
        \end{itemize}
    \end{block}
\end{frame}

\begin{frame}[fragile]
    \frametitle{Rewards in Reinforcement Learning - Formulas}
    The reward function \( R(s, a) \) defines the reward received after taking action \( a \) in state \( s \).
    
    \begin{block}{Cumulative Reward}
        We often are interested in the cumulative reward over time, defined as:
        \begin{equation}
            R_t = \sum_{k=0}^{\infty} \gamma^k r_{t+k}
        \end{equation}
        where \( R_t \) is the cumulative reward starting from time \( t \), \( r_t \) is the reward at time \( t \), and \( \gamma \) (0 ≤ \( \gamma \) < 1) is the discount factor prioritizing immediate rewards.
    \end{block}
\end{frame}

\begin{frame}[fragile]
    \frametitle{Rewards in Reinforcement Learning - Conclusion}
    In summary, understanding reward structures is paramount for designing effective reinforcement learning algorithms.
    
    \begin{block}{Key Takeaway}
        Balancing immediate and delayed rewards while ensuring that the structure aligns with overall learning objectives will facilitate better learning and adaptability in RL agents.
    \end{block}
\end{frame}

\begin{frame}[fragile]
    \frametitle{Value Functions - Overview}
    \begin{block}{Understanding Value Functions}
        Value functions are crucial elements in reinforcement learning that estimate the expected long-term rewards from any given state or action. They guide decision-making over time.
    \end{block}
    
    \begin{itemize}
        \item Evaluate the desirability of states and actions
        \item Critical for many reinforcement learning algorithms
    \end{itemize}
\end{frame}

\begin{frame}[fragile]
    \frametitle{Value Functions - Types}
    \begin{enumerate}
        \item \textbf{State-Value Function (V)}
            \begin{itemize}
                \item \textbf{Definition:} Measures the expected return from a given state $s$ assuming policy $\pi$.
                \item \textbf{Formula:}
                \begin{equation}
                    V(s) = \mathbb{E}_{\pi} \left[ R_t | S_t = s \right]
                \end{equation}
                \item \textbf{Example:} In a grid world, $V(s_1)$ reflects expected rewards from state $s_1$.
            \end{itemize}
        \item \textbf{Action-Value Function (Q)}
            \begin{itemize}
                \item \textbf{Definition:} Represents expected return from taking action $a$ in state $s$ under policy $\pi$.
                \item \textbf{Formula:}
                \begin{equation}
                    Q(s, a) = \mathbb{E}_{\pi} \left[ R_t | S_t = s, A_t = a \right]
                \end{equation}
                \item \textbf{Example:} In a grid world, $Q(s_1, a)$ quantifies anticipated rewards after taking action $a$ from $s_1$.
            \end{itemize}
    \end{enumerate}
\end{frame}

\begin{frame}[fragile]
    \frametitle{Policy in Reinforcement Learning - Definition}
    \begin{block}{Definition of Policy}
        In Reinforcement Learning (RL), a **policy** is a strategy employed by an agent to determine its actions based on the current state. 
        Formally, a policy can be defined as:
    \end{block}
    \begin{itemize}
        \item \textbf{Deterministic Policy}: A function \( \pi: S \rightarrow A \) that maps each state \( s \) directly to an action \( a \).
        \item \textbf{Stochastic Policy}: A function \( \pi: S \times A \rightarrow [0, 1] \) that represents the probability of taking action \( a \) in state \( s \).
    \end{itemize}
\end{frame}

\begin{frame}[fragile]
    \frametitle{Policy in Reinforcement Learning - Role}
    \begin{block}{Role of Policy in Guiding Behavior}
        The policy is crucial because it directly influences the agent's behavior within the environment.
    \end{block}
    \begin{enumerate}
        \item \textbf{Action Selection}: Determines the next action based on the current state, either deterministically or stochastically.
        \item \textbf{Learning \& Adaptation}: Agents adjust their policies to maximize cumulative rewards, employing exploration and exploitation techniques.
    \end{enumerate}
\end{frame}

\begin{frame}[fragile]
    \frametitle{Policy in Reinforcement Learning - Example}
    \begin{block}{Example: Robot Navigation}
        Consider a robot navigating a maze:
    \end{block}
    \begin{itemize}
        \item \textbf{Environment (State)}: Robot's position in the maze (coordinates).
        \item \textbf{Actions}: Move up, down, left, or right.
        \item \textbf{Policy}: A stochastic policy might specify a 70\% chance to move forward and a 30\% chance to turn left.
    \end{itemize}
    \begin{block}{Key Points}
        \begin{itemize}
            \item Types of policies: deterministic vs. stochastic.
            \item Policies evolve through reinforcement signals shaping future actions.
            \item Balance between exploration and exploitation is vital for effective learning.
        \end{itemize}
    \end{block}
\end{frame}

\begin{frame}[fragile]
    \frametitle{Algorithms in Reinforcement Learning - Overview}
    \begin{itemize}
        \item Reinforcement Learning (RL) involves training agents to make decisions through interactions with environments.
        \item Key algorithms:
        \begin{itemize}
            \item Q-learning
            \item SARSA (State-Action-Reward-State-Action)
        \end{itemize}
        \item Both algorithms aim to find the optimal policy for agents.
    \end{itemize}
\end{frame}

\begin{frame}[fragile]
    \frametitle{Algorithms in Reinforcement Learning - Q-learning}
    \begin{block}{Definition}
        Q-learning is an off-policy algorithm that learns the value of an action in a particular state.
    \end{block}
    
    \begin{block}{Key Concept}
        Uses Q-value to estimate expected future rewards from taking action \( a \) in state \( s \) and following the optimal policy thereafter.
    \end{block}
    
    \begin{block}{Q-learning Update Rule}
        \begin{equation}
            Q(s, a) \leftarrow Q(s, a) + \alpha \left[ r + \gamma \max_a Q(s', a) - Q(s, a) \right]
        \end{equation}
        \begin{itemize}
            \item \( Q(s, a) \): Current estimate of the action-value function
            \item \( \alpha \): Learning rate (0 < \( \alpha \) ≤ 1)
            \item \( r \): Reward after action \( a \)
            \item \( \gamma \): Discount factor (0 ≤ \( \gamma \) < 1)
            \item \( s' \): New state after action \( a \)
        \end{itemize}
    \end{block}
    
    \begin{block}{Example}
        In a grid world, an agent learns that moving right leads to positive rewards and updates its Q-values accordingly.
    \end{block}
\end{frame}

\begin{frame}[fragile]
    \frametitle{Algorithms in Reinforcement Learning - SARSA}
    \begin{block}{Definition}
        SARSA is an on-policy algorithm where learning is based on the actions actually taken by the agent.
    \end{block}

    \begin{block}{Key Concept}
        The value of an action is updated based on the next action chosen by the policy.
    \end{block}
    
    \begin{block}{SARSA Update Rule}
        \begin{equation}
            Q(s, a) \leftarrow Q(s, a) + \alpha \left[ r + \gamma Q(s', a') - Q(s, a) \right]
        \end{equation}
        \begin{itemize}
            \item \( s' \): New state after action \( a \)
            \item \( a' \): Action taken in new state \( s' \)
        \end{itemize}
    \end{block}
    
    \begin{block}{Example}
        In the grid world, SARSA adjusts Q-values based on actions following its current policy.
    \end{block}
\end{frame}

\begin{frame}[fragile]
    \frametitle{Key Points to Emphasize}
    \begin{itemize}
        \item **Off-policy vs On-policy:** 
        \begin{itemize}
            \item Q-learning is off-policy and learns the best policy regardless of actions taken.
            \item SARSA is on-policy and learns based on actions it actually takes.
        \end{itemize}
        \item **Exploration vs Exploitation:**
        \begin{itemize}
            \item Both algorithms balance exploring new actions and exploiting known rewarding actions, e.g., through \( \epsilon \)-greedy methods.
        \end{itemize}
        \item **Applications:** 
        \begin{itemize}
            \item Widely used in robotics, video games, finance, etc.
        \end{itemize}
    \end{itemize}
\end{frame}

\begin{frame}[fragile]
    \frametitle{Deep Reinforcement Learning - Introduction}
    \begin{block}{Overview}
        Deep Reinforcement Learning (DRL) combines Reinforcement Learning (RL) and Deep Learning (DL) principles to enable agents to make decisions in complex environments using neural networks. 
    \end{block}
    
    \begin{itemize}
        \item Handles high-dimensional input spaces (e.g., images).
        \item Overcomes limitations of traditional RL methods.
    \end{itemize}
\end{frame}

\begin{frame}[fragile]
    \frametitle{Key Concepts in DRL}
    \begin{block}{Reinforcement Learning Basics}
        \begin{itemize}
            \item \textbf{Agent}: Decision-maker in the environment.
            \item \textbf{Environment}: Context in which the agent operates.
            \item \textbf{Action (A)}: Choices available to the agent.
            \item \textbf{State (S)}: Current situation of the environment.
            \item \textbf{Reward (R)}: Feedback from the environment based on actions.
        \end{itemize}
    \end{block}

    \begin{block}{Deep Learning Techniques}
        \begin{itemize}
            \item \textbf{Neural Networks}: Used to predict value functions or policy distributions.
            \item \textbf{Deep Q-Networks (DQN)}: Combines Q-learning with deep neural networks, enabling learning from raw data.
        \end{itemize}
    \end{block}
\end{frame}

\begin{frame}[fragile]
    \frametitle{How Deep Q-Networks Work}
    \begin{block}{Operational Steps of DQN}
        \begin{enumerate}
            \item **Input Representation**: Converts current state into a neural network-friendly format.
            \item **Policy Learning**: The agent selects actions based on a policy defined by neural networks (deterministic or stochastic).
            \item **Value Function Approximation**: Estimates how good a state is for the agent; minimizes loss between predicted and target Q-values.
        \end{enumerate}
    \end{block}
    
    \begin{block}{Example: Video Game Agent}
        - **State**: Current game frame
        - **Action**: Move left, right, shoot, etc.
        - **Reward**: Points scored
    \end{block}
\end{frame}

\begin{frame}[fragile]
    \frametitle{Applications of Reinforcement Learning}
    
    \begin{block}{Overview}
        Reinforcement Learning (RL) focuses on how agents take actions in an environment to maximize cumulative rewards. This slide highlights real-world applications across different fields.
    \end{block}
\end{frame}

\begin{frame}[fragile]
    \frametitle{Applications of Reinforcement Learning - Robotics}
    
    \begin{itemize}
        \item \textbf{Concept}: RL is used to train robots for complex tasks through trial and error, learning optimal actions based on environmental feedback.
        
        \item \textbf{Examples}:
        \begin{itemize}
            \item \textbf{Robot Manipulation}: A robotic arm learns to stack blocks by receiving rewards for successful placements.
            \item \textbf{Autonomous Driving}: RL helps self-driving cars navigate by rewarding safe driving behaviors.
        \end{itemize}
    \end{itemize}
\end{frame}

\begin{frame}[fragile]
    \frametitle{Applications of Reinforcement Learning - Game Playing and Decision Making}
    
    \begin{itemize}
        \item \textbf{Game Playing}:
        \begin{itemize}
            \item \textbf{AlphaGo}: Used RL to master Go, defeating a world champion by playing against itself millions of times.
            \item \textbf{Atari Games}: DQN taught an AI to play various Atari games from pixels, achieving human-level performance.
        \end{itemize}
        
        \item \textbf{Artificial Intelligence in Decision Making}:
        \begin{itemize}
            \item \textbf{Healthcare}: RL optimizes personalized treatment plans based on patients' reactions and health metrics.
            \item \textbf{Finance}: Algorithms learn to execute trades to maximize returns based on historical data.
        \end{itemize}
    \end{itemize}
\end{frame}

\begin{frame}[fragile]
    \frametitle{Key Points and Summary}
    
    \begin{itemize}
        \item \textbf{Trial and Error Learning}: Agents learn from the consequences of their actions rather than explicit instructions.
        \item \textbf{Reward Signals}: Crucial for guiding the learning process, rewarding successful actions and discouraging failures.
        \item \textbf{Versatility}: RL applications are adaptable and impactful across various domains.
    \end{itemize}
    
    \begin{block}{Summary}
        Reinforcement Learning is transforming industries by enabling machines to learn autonomously. Its principles provide powerful tools for tackling complex problems in robotics, gaming, and decision-making.
    \end{block}
\end{frame}

\begin{frame}[fragile]
    \frametitle{Challenges in Reinforcement Learning - Introduction}
    \begin{block}{Overview}
        Reinforcement Learning (RL) is a powerful tool in artificial intelligence, enabling systems to learn optimal behaviors through interactions with their environment. However, several challenges hinder performance and efficiency.
    \end{block}
    \begin{block}{Key Topics}
        \begin{itemize}
            \item Sample Efficiency
            \item Exploration vs. Exploitation Dilemma
            \item Credit Assignment Problem
            \item Function Approximation
            \item High-Dimensional State Spaces
        \end{itemize}
    \end{block}
\end{frame}

\begin{frame}[fragile]
    \frametitle{Challenges in Reinforcement Learning - Sample Efficiency}
    \begin{block}{Definition}
        Sample efficiency refers to the number of training examples (or interactions with the environment) needed to learn a successful policy.
    \end{block}
    \begin{itemize}
        \item RL systems often require significant amounts of experience to achieve high performance, which can be costly or impractical.
        \item \textbf{Example:} A robot learning to walk may need countless movements over hours of real time.
    \end{itemize}
\end{frame}

\begin{frame}[fragile]
    \frametitle{Challenges in Reinforcement Learning - Exploration vs. Exploitation}
    \begin{block}{Concept}
        The trade-off between exploring new actions to find better rewards and exploiting known actions that yield high rewards.
    \end{block}
    \begin{itemize}
        \item Too much exploration slows down training.
        \item Too much exploitation leads to local optima and suboptimal policies.
        \item \textbf{Example:} In a maze-solving task, an agent may spend excessive time exploring unbeneficial paths.
    \end{itemize}
\end{frame}

\begin{frame}[fragile]
    \frametitle{Challenges in Reinforcement Learning - Credit Assignment Problem}
    \begin{block}{Definition}
        The challenge of determining which actions are responsible for outcomes that occur over a long sequence of decisions.
    \end{block}
    \begin{itemize}
        \item It can be difficult to ascertain which specific actions led to an outcome when rewards or punishments are received for a series of actions.
        \item \textbf{Example:} In a game, identifying which moves contributed to a win after a complex series of decisions is challenging.
    \end{itemize}
\end{frame}

\begin{frame}[fragile]
    \frametitle{Challenges in Reinforcement Learning - Function Approximation}
    \begin{block}{Challenges}
        RL agents use function approximation to estimate value functions or policies, leading to stability and convergence issues.
    \end{block}
    \begin{itemize}
        \item Using neural networks as function approximators can introduce complexities such as overfitting or oscillation in learning.
    \end{itemize}
\end{frame}

\begin{frame}[fragile]
    \frametitle{Challenges in Reinforcement Learning - High-Dimensional State Spaces}
    \begin{block}{Problem}
        Complex environments often have vast state spaces, making it difficult for RL algorithms to generalize effectively.
    \end{block}
    \begin{itemize}
        \item \textbf{Example:} In video games, the massive number of possible states (combinations of pixels and game positions) challenges the RL agent to learn an effective strategy.
    \end{itemize}
\end{frame}

\begin{frame}[fragile]
    \frametitle{Challenges in Reinforcement Learning - Conclusion}
    \begin{block}{Key Takeaways}
        Addressing issues like sample efficiency, exploration-exploitation balance, credit assignment, and high-dimensional spaces can enhance RL performance.
    \end{block}
    \begin{itemize}
        \item Sample efficiency is pivotal for practical RL applications.
        \item Balancing exploration and exploitation is crucial for optimal learning.
        \item Solving the credit assignment problem leads to more effective learning.
        \item Function approximation requires careful handling to ensure stability.
        \item High-dimensional state spaces necessitate innovative strategies for effective learning.
    \end{itemize}
\end{frame}

\begin{frame}[fragile]
    \frametitle{Ethical Considerations in Reinforcement Learning - Introduction}
    \begin{block}{Overview}
        Reinforcement Learning (RL) involves training models through interactions with their environment, 
        aiming to maximize cumulative rewards. While RL has numerous applications, it raises significant 
        ethical considerations due to its influence on decision-making processes.
    \end{block}
\end{frame}

\begin{frame}[fragile]
    \frametitle{Ethical Considerations in Reinforcement Learning - Key Implications}
    \begin{enumerate}
        \item \textbf{Autonomy and Control}
            \begin{itemize}
                \item RL systems operate with varying degrees of autonomy.
                \item Example: Autonomous vehicles making split-second safety decisions.
            \end{itemize}
        
        \item \textbf{Bias and Fairness}
            \begin{itemize}
                \item Training on biased data leads to unfair treatment.
                \item Example: Hiring agents favoring specific demographics based on biased historical data.
            \end{itemize}

        \item \textbf{Safety and Robustness}
            \begin{itemize}
                \item RL agents may exhibit unsafe behaviors without proper training.
                \item Example: Industrial robots bypassing safety zones for faster performance.
            \end{itemize}
    \end{enumerate}
\end{frame}

\begin{frame}[fragile]
    \frametitle{Ethical Considerations in Reinforcement Learning - Conclusion and Code Snippet}
    \begin{block}{Importance of Ethical Design}
        Ethical considerations ensure public trust, legal accountability, and promote fairness in RL systems. 
        Organizations must ensure inclusivity to prevent perpetuation of biases.
    \end{block}

    \begin{block}{Example Code Snippet: Framework for Ethical Evaluation}
    \begin{lstlisting}[language=Python]
class EthicalEvaluation:
    def __init__(self, model):
        self.model = model

    def evaluate_bias(self, training_data):
        # Implement bias detection algorithms
        pass

    def check_transparency(self):
        # Confirm explainability mechanisms in place
        pass

    def assess_safety(self):
        # Rigorous safety tests under unexpected conditions
        pass

# Usage
rl_model = YourRLModel()
evaluation = EthicalEvaluation(rl_model)
evaluation.evaluate_bias(training_data)
evaluation.check_transparency()
evaluation.assess_safety()
    \end{lstlisting}
    \end{block}
\end{frame}

\begin{frame}[fragile]
    \frametitle{Future Directions in Reinforcement Learning}
    \begin{itemize}
        \item Reinforcement Learning (RL) is evolving rapidly to tackle complex challenges.
        \item Understanding future research directions is imperative for leveraging RL effectively.
    \end{itemize}
\end{frame}

\begin{frame}[fragile]
    \frametitle{Future Directions - Interdisciplinary Integration & Inverse RL}
    \begin{block}{Interdisciplinary Integration}
        \begin{itemize}
            \item \textbf{Description}: Integrating insights from neuroscience, cognitive science, and behavioral economics.
            \item \textbf{Example}: Agents mimicking human decision-making can enhance algorithm robustness.
        \end{itemize}
    \end{block}
    
    \begin{block}{Inverse Reinforcement Learning (IRL)}
        \begin{itemize}
            \item \textbf{Description}: Focuses on extracting reward functions from observed behavior.
            \item \textbf{Example}: Studying human actions to teach robots tasks without manual programming.
        \end{itemize}
    \end{block}
\end{frame}

\begin{frame}[fragile]
    \frametitle{Future Directions - Transfer Learning & Hierarchical Learning}
    \begin{block}{Transfer Learning in RL}
        \begin{itemize}
            \item \textbf{Description}: Enables agents to apply knowledge from one context to different environments.
            \item \textbf{Example}: Training robots in simulations and applying knowledge to real-world tasks reduces training time and costs.
        \end{itemize}
    \end{block}
    
    \begin{block}{Hierarchical Reinforcement Learning}
        \begin{itemize}
            \item \textbf{Description}: Models complex tasks as a hierarchy of simpler sub-tasks for better efficiency.
            \item \textbf{Example}: A cooking robot learns to gather ingredients before cooking.
        \end{itemize}
    \end{block}
\end{frame}

\begin{frame}[fragile]
    \frametitle{Future Directions - Safety, Explainability, & Applications}
    \begin{block}{Safety and Robustness}
        \begin{itemize}
            \item \textbf{Description}: Focusing on robustness against adversarial inputs and ensuring safe exploration.
            \item \textbf{Example}: Constraining unsafe actions in RL agents, such as in self-driving cars.
        \end{itemize}
    \end{block}
    
    \begin{block}{Explainability in RL}
        \begin{itemize}
            \item \textbf{Description}: Understanding decision-making processes in critical applications.
            \item \textbf{Example}: Creating interpretable models to enhance user trust and accountability.
        \end{itemize}
    \end{block}
    
    \begin{block}{Real-World Applications}
        \begin{itemize}
            \item \textbf{Description}: Increased deployment in healthcare, finance, and robotics.
            \item \textbf{Example}: Personalizing healthcare treatments based on patient reactions.
        \end{itemize}
    \end{block}
\end{frame}

\begin{frame}[fragile]
    \frametitle{Conclusion & Key Points}
    \begin{block}{Key Points to Emphasize}
        \begin{itemize}
            \item \textbf{Interdisciplinary Collaboration}: Unlocking new potentials in RL.
            \item \textbf{Adaptability}: Future RL systems need to adapt to various environments.
            \item \textbf{Ethical Implications}: Addressing ethical concerns is crucial for responsible RL deployment.
        \end{itemize}
    \end{block}

    \begin{block}{Conclusion}
        Future directions in RL promise enhanced applicability across fields, requiring innovative research with a strong focus on safety, adaptability, and interdisciplinary approaches.
    \end{block}
\end{frame}

\begin{frame}[fragile]
    \frametitle{Conclusion - Summary of Key Learnings in Reinforcement Learning (RL)}
    
    \begin{block}{1. Definition and Core Concepts}
        \begin{itemize}
            \item Reinforcement Learning (RL) is a type of machine learning focused on agents learning to make decisions through actions in an environment to maximize cumulative rewards.
            \item Key components of RL include:
            \begin{itemize}
                \item \textbf{Agent}: The learner or decision maker.
                \item \textbf{Environment}: Everything the agent interacts with.
                \item \textbf{Action (A)}: Choices made by the agent.
                \item \textbf{State (S)}: The current situation of the agent.
                \item \textbf{Reward (R)}: Feedback received after taking an action.
            \end{itemize}
        \end{itemize}
    \end{block}
    
    \begin{block}{2. Learning Strategies}
        \begin{itemize}
            \item \textbf{Exploration vs. Exploitation}: Balancing new actions and known high-reward actions.
            \item \textbf{Value Functions}: Estimate future rewards for guiding actions.
            \item \textbf{Q-Learning}: A model-free RL algorithm using Q-values to assess action quality.
        \end{itemize}
    \end{block}
    
\end{frame}

\begin{frame}[fragile]
    \frametitle{Conclusion - Applications of RL and Relevance to Machine Learning}
    
    \begin{block}{3. Applications of RL}
        \begin{itemize}
            \item Widely applied in:
            \begin{itemize}
                \item \textbf{Robotics}: Automating tasks through trial and error.
                \item \textbf{Game Playing}: Achievements like DeepMind's AlphaGo in complex games.
                \item \textbf{Autonomous Vehicles}: Navigating environments and making real-time decisions.
            \end{itemize}
        \end{itemize}
    \end{block}
    
    \begin{block}{Relevance to the Broader ML Field}
        \begin{itemize}
            \item Integrates with other ML paradigms, enhancing traditional methods.
            \item \textbf{Deep Reinforcement Learning}: Combines neural networks with RL for high-dimensional state/action handling.
        \end{itemize}
    \end{block}
    
\end{frame}

\begin{frame}[fragile]
    \frametitle{Conclusion - Future Directions and Key Points}
    
    \begin{block}{Future Directions}
        \begin{itemize}
            \item Emerging trends in RL:
            \begin{itemize}
                \item Meta-learning for task adaptation.
                \item Multi-agent systems where agents learn and interact.
                \item Transfer learning for applying prior knowledge to new tasks.
            \end{itemize}
        \end{itemize}
    \end{block}
    
    \begin{block}{Key Points to Emphasize}
        \begin{itemize}
            \item RL serves as a robust framework for decision-making across diverse domains.
            \item Integration with other ML methods indicates AI's evolving nature.
            \item Continuous advancement in RL methodologies enhances efficiency and applicability.
        \end{itemize}
    \end{block}
    
    \begin{block}{Illustrative Example: The Cart-Pole Problem}
        \begin{itemize}
            \item An agent learns to balance a pole on a cart, receiving rewards for pole stability.
            \item Actions include pushing the cart left or right; learning is through trial and error.
        \end{itemize}
    \end{block}
    
\end{frame}

\begin{frame}[fragile]
    \frametitle{Q \& A Session on Reinforcement Learning}
    % Introduction
    \begin{block}{Introduction}
        After our detailed exploration of Reinforcement Learning (RL) concepts in this chapter, it's time to engage in an interactive Q\&A session.
        This is a valuable opportunity to clarify doubts, delve deeper into specific topics, and discuss real-world applications and implications of RL.
    \end{block}
\end{frame}

\begin{frame}[fragile]
    \frametitle{Key Topics for Discussion}
    \begin{enumerate}
        \item \textbf{Fundamentals of Reinforcement Learning:}
            \begin{itemize}
                \item What are the core components of RL: Agent, Environment, Actions, States, and Rewards?
                \item How do these components interact to form a learning loop?
            \end{itemize}
        
        \item \textbf{Exploration vs. Exploitation:}
            \begin{itemize}
                \item Can you elaborate on the balance between these two strategies in RL?
                \item What are some real-world implications of this trade-off in applications like robotics or game playing?
            \end{itemize}

        \item \textbf{Methods and Algorithms:}
            \begin{itemize}
                \item Are there particular algorithms that have shown superior performance in specific environments (e.g., DQN, PPO, A3C)?
                \item How do you choose the right algorithm for a given problem?
            \end{itemize}

        \item \textbf{Applications of Reinforcement Learning:}
            \begin{itemize}
                \item In which domains are you most interested in applying RL? Examples include autonomous vehicles, recommendation systems, or finance.
                \item What opportunities do you see for implementation in emerging fields?
            \end{itemize}
    \end{enumerate}
\end{frame}

\begin{frame}[fragile]
    \frametitle{Example Questions and Key Points}
    \begin{block}{Example Question Starters}
        \begin{itemize}
            \item "Can you explain how \textbf{deep Q-learning} differs from traditional Q-learning?"
            \item "How does \textbf{policy gradient} method work in contrast to value-based methods?"
            \item "What role does \textbf{reward shaping} play in speeding up the learning process?"
        \end{itemize}
    \end{block}

    \begin{block}{Key Points to Keep in Mind}
        \begin{itemize}
            \item Reinforcement Learning is complex. Don’t hesitate to ask even basic questions.
            \item ENGAGEMENT is crucial—participating enriches everyone's understanding.
            \item \textbf{No question is too small or trivial}—every inquiry can lead to deeper insights.
        \end{itemize}
    \end{block}
    
    \begin{block}{Conclusion of Q\&A}
        We encourage open dialogue during this session. Your questions foster understanding and benefit your peers.
    \end{block}
\end{frame}


\end{document}