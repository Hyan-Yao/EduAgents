\documentclass[aspectratio=169]{beamer}

% Theme and Color Setup
\usetheme{Madrid}
\usecolortheme{whale}
\useinnertheme{rectangles}
\useoutertheme{miniframes}

% Additional Packages
\usepackage[utf8]{inputenc}
\usepackage[T1]{fontenc}
\usepackage{graphicx}
\usepackage{booktabs}
\usepackage{listings}
\usepackage{amsmath}
\usepackage{amssymb}
\usepackage{xcolor}
\usepackage{tikz}
\usepackage{pgfplots}
\pgfplotsset{compat=1.18}
\usetikzlibrary{positioning}
\usepackage{hyperref}

% Custom Colors
\definecolor{myblue}{RGB}{31, 73, 125}
\definecolor{mygray}{RGB}{100, 100, 100}
\definecolor{mygreen}{RGB}{0, 128, 0}
\definecolor{myorange}{RGB}{230, 126, 34}
\definecolor{mycodebackground}{RGB}{245, 245, 245}

% Set Theme Colors
\setbeamercolor{structure}{fg=myblue}
\setbeamercolor{frametitle}{fg=white, bg=myblue}
\setbeamercolor{title}{fg=myblue}
\setbeamercolor{section in toc}{fg=myblue}
\setbeamercolor{item projected}{fg=white, bg=myblue}
\setbeamercolor{block title}{bg=myblue!20, fg=myblue}
\setbeamercolor{block body}{bg=myblue!10}
\setbeamercolor{alerted text}{fg=myorange}

% Set Fonts
\setbeamerfont{title}{size=\Large, series=\bfseries}
\setbeamerfont{frametitle}{size=\large, series=\bfseries}
\setbeamerfont{caption}{size=\small}
\setbeamerfont{footnote}{size=\tiny}

% Title Page Information
\title[Logic Reasoning: FOL]{Chapter 8: Logic Reasoning: First-Order Logic}
\author[J. Smith]{John Smith, Ph.D.}
\institute[University Name]{
  Department of Computer Science\\
  University Name\\
  Email: email@university.edu\\
  Website: www.university.edu
}
\date{\today}

% Document Start
\begin{document}

\frame{\titlepage}

\begin{frame}[fragile]
  \titlepage
\end{frame}

\begin{frame}[fragile]
  \frametitle{Overview of First-Order Logic (FOL)}
  
  \begin{block}{Definition}
    First-Order Logic (FOL), also known as predicate logic, is a formal system used in mathematics, philosophy, linguistics, and computer science. 
    It extends propositional logic by dealing with predicates and quantifiers, allowing statements about objects and their relationships.
  \end{block}
  
  \begin{block}{Significance in Artificial Intelligence}
    \begin{itemize}
      \item \textbf{Richness of Expression:} 
        FOL enables expressions of complex statements involving relationships. 
        For example: $\forall x (\text{Human}(x) \to \text{Mortal}(x))$ means "All humans are mortal."
  
      \item \textbf{Reasoning Capabilities:} 
        FOL supports inferencing, crucial for automated reasoning in AI applications.
  
      \item \textbf{Knowledge Representation:} 
        FOL facilitates a structured way to represent knowledge, enhancing understanding and processing of information.
    \end{itemize}
  \end{block}
\end{frame}

\begin{frame}[fragile]
  \frametitle{Key Components of First-Order Logic}
  
  \begin{enumerate}
    \item \textbf{Predicates:} Symbols representing properties or relations between objects.
      \begin{itemize}
        \item Example: $\text{Loves(John, Mary)}$ indicates that John loves Mary.
      \end{itemize}
  
    \item \textbf{Terms:} 
      These include constants (specific objects), variables (placeholders for objects), or functions returning objects.
      \begin{itemize}
        \item Constants: John, Mary
        \item Variables: $x$, $y$
        \item Functions: $\text{MotherOf}(x)$
      \end{itemize}
  
    \item \textbf{Quantifiers:}
      \begin{itemize}
        \item Universal Quantifier ($\forall$): Applies to all instances.
        \item Existential Quantifier ($\exists$): Applies to at least one instance.
        \item Examples:
          \begin{itemize}
            \item $\forall x (\text{Bird}(x) \to \text{CanFly}(x))$ means "All birds can fly."
            \item $\exists y (\text{Cat}(y) \land \text{Loves}(y, \text{John}))$ means "There exists a cat that loves John."
          \end{itemize}
      \end{itemize}
  \end{enumerate}
\end{frame}

\begin{frame}[fragile]
  \frametitle{Example of FOL in Action}
  
  Consider the following sentences:
  
  \begin{itemize}
    \item \textbf{Statement:} "Every dog is an animal."
      \begin{itemize}
        \item \textbf{FOL Representation:} $\forall x (\text{Dog}(x) \to \text{Animal}(x))$
      \end{itemize}
  
    \item \textbf{Statement:} "There exists a pet that is a cat."
      \begin{itemize}
        \item \textbf{FOL Representation:} $\exists y (\text{Pet}(y) \land \text{Cat}(y))$
      \end{itemize}
  \end{itemize}
  
  From these representations, conclusions can be drawn about the existence and properties of animals or pets in a knowledge base.
\end{frame}

\begin{frame}[fragile]
  \frametitle{Key Points to Emphasize}
  \begin{itemize}
    \item \textbf{Flexibility:} FOL is more expressive than propositional logic, enabling detailed relationships and properties.
    
    \item \textbf{Foundation for AI:} Many AI systems, including expert systems, rely on FOL for knowledge representation and reasoning.
    
    \item \textbf{Applications:} FOL is crucial in areas such as natural language processing, theorem proving, and knowledge-based systems.
  \end{itemize}
  
  \begin{block}{Conclusion}
    First-Order Logic serves as a powerful tool in Artificial Intelligence, enhancing our ability to represent, infer, and utilize knowledge effectively. Understanding FOL is pivotal for those delving into advanced AI topics.
  \end{block}
\end{frame}

\begin{frame}[fragile]
    \frametitle{What is Logic?}
    \begin{block}{Definition of Logic}
        Logic is the systematic study of valid reasoning and argument. 
        It provides a framework for understanding how conclusions can be drawn 
        from premises, helping us discern truth from falsehood.
    \end{block}
\end{frame}

\begin{frame}[fragile]
    \frametitle{Importance of Logic in Reasoning}
    \begin{itemize}
        \item \textbf{Critical Thinking}: Develops the ability to analyze arguments, 
        identify fallacies, and construct sound reasoning.
        
        \item \textbf{Decision Making}: Enables informed and rational decisions, 
        reducing biases and subjective influences.
        
        \item \textbf{Problem Solving}: Assists in breaking down complex problems 
        into simpler components for systematic analysis and resolution.
    \end{itemize}
\end{frame}

\begin{frame}[fragile]
    \frametitle{Importance of Logic in Artificial Intelligence (AI)}
    \begin{itemize}
        \item \textbf{Knowledge Representation}: Forms the foundation for representing 
        knowledge in AI, allowing reasoning about information and inference of new knowledge.
        
        \item \textbf{Automated Reasoning}: Utilizes logical frameworks, such as 
        First-Order Logic (FOL), for theorem proving and deriving conclusions from 
        facts and rules.
        
        \item \textbf{Natural Language Processing}: Facilitates parsing and 
        interpretation of language, improving communication between humans and machines.
    \end{itemize}
\end{frame}

\begin{frame}[fragile]
    \frametitle{Key Points and Example}
    \begin{block}{Key Points}
        \begin{itemize}
            \item Logic is essential for structured reasoning and effective communication.
            \item It serves as a core component in various applications of AI.
            \item Understanding logical principles is foundational for exploring areas like First-Order Logic.
        \end{itemize}
    \end{block}

    \begin{block}{Example}
        \textbf{Logical Statement:} "All humans are mortal. Socrates is a human. 
        Therefore, Socrates is mortal." 
        This illustrates the application of logical reasoning where premises lead to a valid conclusion.
    \end{block}
\end{frame}

\begin{frame}[fragile]
    \frametitle{Conclusion}
    Logic enhances cognitive skills and serves as a pivotal tool in artificial intelligence. 
    It improves both human and machine reasoning capabilities. As we delve into 
    First-Order Logic, we will explore how these logical foundations can be effectively 
    applied in AI systems and reasoning processes.
\end{frame}

\begin{frame}[fragile]
    \frametitle{Propositional Logic Recap}
    \begin{block}{Introduction to Propositional Logic}
        Propositional logic, also known as propositional calculus, is a fundamental branch of logic that deals with propositions—statements that can be either true or false. It serves as the building block for more complex systems of logic, including First-Order Logic (FOL).
    \end{block}
\end{frame}

\begin{frame}[fragile]
    \frametitle{Key Concepts of Propositional Logic}
    \begin{enumerate}
        \item \textbf{Propositions:}
        \begin{itemize}
            \item A proposition is a declarative sentence that expresses a fact or opinion.
            \item \textit{Example:} 
            \begin{itemize}
                \item "The sky is blue." (True)
                \item "2 + 2 = 5." (False)
            \end{itemize}
        \end{itemize}

        \item \textbf{Logical Connectives:}
        \begin{itemize}
            \item \textbf{AND ($\land$)}: True if both propositions are true.
            \item \textbf{OR ($\lor$)}: True if at least one proposition is true.
            \item \textbf{NOT ($\neg$)}: Inverts the truth value of a proposition.
            \item \textbf{IMPLICATION ($\rightarrow$)}: True unless a true proposition implies a false one.
            \item \textbf{BICONDITIONAL ($\leftrightarrow$)}: True if both propositions are either true or false.
        \end{itemize}
    \end{enumerate}
\end{frame}

\begin{frame}[fragile]
    \frametitle{Truth Tables}
    Truth tables are used to represent the possible truth values of propositions and their combinations using connectives.

    \begin{center}
    \begin{tabular}{|c|c|c|c|c|c|c|}
        \hline
        P     & Q     & P $\land$ Q & P $\lor$ Q & $\neg$P & P $\rightarrow$ Q & P $\leftrightarrow$ Q \\
        \hline
        True  & True  & True  & True  & False & True  & True  \\
        \hline
        True  & False & False & True  & False & False & False \\
        \hline
        False & True  & False & True  & True  & True  & False \\
        \hline
        False & False & False & False & True  & True  & True  \\
        \hline
    \end{tabular}
    \end{center}
\end{frame}

\begin{frame}[fragile]
    \frametitle{Importance of Propositional Logic}
    \begin{itemize}
        \item \textbf{Foundation for More Complex Logic:} Understanding propositional logic is crucial, as it forms the basis for more advanced logical systems, such as FOL.
        \item \textbf{Applications:} Widely used in computer science, particularly in algorithms, programming, and artificial intelligence for decision-making processes.
    \end{itemize}
\end{frame}

\begin{frame}[fragile]
    \frametitle{Summary}
    Propositional logic provides essential groundwork for reasoning about simple statements and their relationships. Mastery of these concepts will enhance your understanding as we transition to First-Order Logic (FOL), where we will explore richer representations of knowledge.

    \begin{block}{Key Takeaway}
        Remember: Logic is about clarity and precision—developing skills in propositional logic will empower you to tackle more complex logical reasoning in FOL and beyond!
    \end{block}
\end{frame}

\begin{frame}[fragile]
    \frametitle{Understanding First-Order Logic}
    \begin{block}{Introduction to First-Order Logic (FOL)}
        First-Order Logic (FOL) is a powerful extension of propositional logic that allows us to express more complex statements and engage in deeper reasoning about the world. To fully grasp FOL, we need to explore its structure, syntax, and semantics.
    \end{block}
\end{frame}

\begin{frame}[fragile]
    \frametitle{Structure of FOL}
    \begin{itemize}
        \item \textbf{Predicates}: Represent properties or relations among objects. 
        \begin{itemize}
            \item Example: \( P(x) \) might mean "x is a person".
        \end{itemize}
        
        \item \textbf{Constants}: Specific objects in our domain.
        \begin{itemize}
            \item Example: \( a \) could represent "Alice".
        \end{itemize}
        
        \item \textbf{Variables}: Symbols that can take on values from the domain.
        \begin{itemize}
            \item Example: \( x, y, z \) are variables that can represent any object.
        \end{itemize}
        
        \item \textbf{Functions}: Map inputs (objects) to outputs (objects).
        \begin{itemize}
            \item Example: \( f(x) \) might represent "the mother of x".
        \end{itemize}
    \end{itemize}
\end{frame}

\begin{frame}[fragile]
    \frametitle{Syntax of FOL}
    FOL utilizes a formal structure for writing statements, which includes:
    \begin{enumerate}
        \item \textbf{Atomic Formulas}: Simple statements denoted by predicates and their arguments.
        \begin{itemize}
            \item Example: \( P(a) \) states that "Alice is a person".
        \end{itemize}
        
        \item \textbf{Quantifiers}:
        \begin{itemize}
            \item \textbf{Universal Quantifier} (\( \forall \)): Indicates that a statement applies to all members of a domain.
            \begin{itemize}
                \item Example: \( \forall x P(x) \): "For every x, x is a person."
            \end{itemize}
            \item \textbf{Existential Quantifier} (\( \exists \)): There exists at least one member of the domain for which the statement is true.
            \begin{itemize}
                \item Example: \( \exists y Q(y) \): "There exists a y such that y is happy."
            \end{itemize}
        \end{itemize}
        
        \item \textbf{Logical Connectives}: 
        \begin{itemize}
            \item AND (\( \land \)), OR (\( \lor \)), NOT (\( \neg \)), IMPLIES (\( \rightarrow \)).
            \item Example: \( \forall x (P(x) \rightarrow Q(x)) \): "For every x, if x is a person, then x is happy."
        \end{itemize}
    \end{enumerate}
\end{frame}

\begin{frame}[fragile]
    \frametitle{Semantics of FOL}
    The semantics of FOL define the meanings of the symbols within the logic:
    \begin{itemize}
        \item \textbf{Interpretation}: Assigns meaning to constants, functions, and predicates. 
        \begin{itemize}
            \item For instance, in a domain of people, \( P(a) \) where \( a \) refers to "Alice" checks if "Alice is a person" is true under that interpretation.
        \end{itemize}
        
        \item \textbf{Truth Values}: Statements in FOL can be evaluated as true or false based on the chosen interpretation.
    \end{itemize}
\end{frame}

\begin{frame}[fragile]
    \frametitle{Key Points to Emphasize}
    \begin{itemize}
        \item FOL provides a richer language for expressing logic compared to propositional logic.
        \item It allows the use of quantifiers to make general statements about groups of objects.
        \item Understanding the structure (predicates, constants, etc.), syntax (formulas, quantifiers), and semantics (meaning of symbols) is essential for effective reasoning in FOL.
    \end{itemize}
\end{frame}

\begin{frame}[fragile]
    \frametitle{Example in Context}
    Consider a domain where we examine relationships between people:
    \begin{itemize}
        \item \textbf{Predicates}: \( Friend(x, y) \) means "x is a friend of y".
        \item \textbf{Statement}: \( \forall x \exists y (Friend(x, y)) \) means "Everyone has at least one friend".
    \end{itemize}
    
    With these concepts, we can formulate and reason about a wide array of scenarios effectively within FOL.
\end{frame}

\begin{frame}[fragile]
    \frametitle{Conclusion}
    This understanding of First-Order Logic lays the groundwork for exploring its components in more detail in the next slide.
\end{frame}

\begin{frame}[fragile]
  \frametitle{Components of First-Order Logic (FOL) - Key Definitions}
  \begin{enumerate}
    \item \textbf{Terms}:
      \begin{itemize}
        \item Basic elements that refer to objects in the domain of discourse.
        \item \textbf{Types of Terms}:
          \begin{itemize}
            \item \textbf{Constants}: Specific, identifiable objects (e.g., `Alice`, `3`).
            \item \textbf{Variables}: Represent arbitrary objects (e.g., `x`, `y`, `z`).
            \item \textbf{Functions}: Map terms to other terms, constructing new terms (e.g., `fatherOf(x)`).
          \end{itemize}
      \end{itemize}
    
    \item \textbf{Predicates}:
      \begin{itemize}
        \item Statements expressing properties or relationships.
        \item Represented by uppercase letters followed by terms (e.g., `Human(x)`, `Loves(x, y)`).
        \item \textbf{Example}: `Human(Alice)` indicates that Alice is a human.
      \end{itemize}
  \end{enumerate}
\end{frame}

\begin{frame}[fragile]
  \frametitle{Components of First-Order Logic (FOL) - Examples}
  \begin{block}{Constants, Variables, Functions, and Predicates}
    \begin{itemize}
      \item \textbf{Constants}: `3`, `John`, `City`
        \begin{itemize}
          \item Fixed values denoting specific entities.
        \end{itemize}

      \item \textbf{Variables}: `x`, `y`, `z`
        \begin{itemize}
          \item Can take any value from the domain. E.g., in `Human(x)`, `x` could refer to any human.
        \end{itemize}

      \item \textbf{Functions}: `motherOf(x)`, `+`
        \begin{itemize}
          \item `motherOf(Alice)` returns the mother of Alice.
          \item `+(2, 3)` yields `5`.
        \end{itemize}

      \item \textbf{Predicates}: `Likes(x, y)`, `IsTall(x)`
        \begin{itemize}
          \item `Likes(John, IceCream)` asserts John likes ice cream.
          \item `IsTall(Emily)` asserts Emily is tall.
        \end{itemize}
    \end{itemize}
  \end{block}
\end{frame}

\begin{frame}[fragile]
  \frametitle{Components of First-Order Logic (FOL) - Formal Structure}
  \begin{block}{Typical Expression Structure}
    A typical expression in FOL can be structured as:
    \[
    P(t_1, t_2, \ldots, t_n)
    \]
    Where $P$ is a predicate and $t_1, t_2, \ldots, t_n$ are terms.
  \end{block}

  \begin{block}{Example Expression}
    \[
    Loves(John, Alice)
    \]
    This can be interpreted as "John loves Alice".
  \end{block}

  \begin{block}{Conclusion}
    Understanding the components of FOL allows for effective representation of logic statements and reasoning. Future slides will explore interactions of these components, particularly using quantifiers.
  \end{block}
\end{frame}

\begin{frame}[fragile]
    \frametitle{Quantifiers in First-Order Logic (FOL)}
    \begin{block}{Understanding Universal and Existential Quantifiers}
        Quantifiers are essential tools in FOL that help express statements about individuals in a domain.
    \end{block}
\end{frame}

\begin{frame}[fragile]
    \frametitle{Universal Quantifier (∀)}
    \begin{itemize}
        \item \textbf{Definition:} The statement $\forall x \, P(x)$ asserts that the predicate \( P \) holds true for every element \( x \) in the domain.
        
        \item \textbf{Example:} 
        \begin{equation}
            \forall x (Human(x) \rightarrow Mortal(x))
        \end{equation}
        
        \item \textbf{Key Point:} Must hold for every instance; if one instance is false, the entire expression is false.
    \end{itemize}
\end{frame}

\begin{frame}[fragile]
    \frametitle{Existential Quantifier (∃)}
    \begin{itemize}
        \item \textbf{Definition:} The statement $\exists x \, P(x)$ asserts that there exists at least one element \( x \) in the domain for which the predicate \( P \) holds true.
        
        \item \textbf{Example:}
        \begin{equation}
            \exists x (Cat(x) \land Black(x))
        \end{equation}
        
        \item \textbf{Key Point:} Only requires one instance to be true; if at least one element satisfies the predicate, the entire expression is true.
    \end{itemize}
\end{frame}

\begin{frame}[fragile]
    \frametitle{Combining Quantifiers}
    \begin{itemize}
        \item Be mindful of the scope when combining quantifiers.
        
        \item \textbf{Example:} 
        \begin{itemize}
            \item $\forall x \, \exists y \, P(x, y)$: For every \( x \), there exists at least one \( y \) such that \( P \) holds.
            \item $\exists y \, \forall x \, P(x, y)$: There is a single \( y \) that works for all \( x \).
        \end{itemize}

        \item \textbf{Example with Both Quantifiers:}
        \begin{equation}
            \forall x (Student(x) \rightarrow \exists y (Book(y) \land Has(x, y)))
        \end{equation}
    \end{itemize}
\end{frame}

\begin{frame}[fragile]
    \frametitle{Summary of Key Points}
    \begin{itemize}
        \item \textbf{Universal Quantifier (∀):} States that a property holds for all members of a domain.
        \item \textbf{Existential Quantifier (∃):} States that there is at least one member in the domain for which the property holds.
    \end{itemize}
    \begin{block}{Conclusion}
        Understanding the difference between these quantifiers is crucial for constructing logical statements and performing valid inferences in FOL.
    \end{block}
\end{frame}

\begin{frame}[fragile]
    \frametitle{Inference in First-Order Logic}
    \textbf{Introduction to Inference Rules}
    \begin{itemize}
        \item Inference rules are essential for reasoning in First-Order Logic (FOL).
        \item They enable the derivation of new statements from existing ones.
    \end{itemize}
\end{frame}

\begin{frame}[fragile]
    \frametitle{Key Inference Rules - Part 1}
    \begin{enumerate}
        \item \textbf{Modus Ponens}:
        \begin{itemize}
            \item If \( P \implies Q \) and \( P \) is true, then \( Q \) is true.
            \item \textbf{Example}:
            \begin{itemize}
                \item Premise 1: "If it is raining, then the ground is wet" (\( R \implies W \))
                \item Premise 2: "It is raining" (\( R \))
                \item Conclusion: "The ground is wet" (\( W \))
            \end{itemize}
        \end{itemize}
        \item \textbf{Modus Tollens}:
        \begin{itemize}
            \item If \( P \implies Q \) is true and \( Q \) is false, then \( P \) is false.
            \item \textbf{Example}:
            \begin{itemize}
                \item Premise 1: "If it is raining, then the ground is wet" (\( R \implies W \))
                \item Premise 2: "The ground is not wet" (\( \neg W \))
                \item Conclusion: "It is not raining" (\( \neg R \))
            \end{itemize}
        \end{itemize}
    \end{enumerate}
\end{frame}

\begin{frame}[fragile]
    \frametitle{Key Inference Rules - Part 2}
    \begin{enumerate}
        \setcounter{enumi}{2} % Continue numbering from previous frame
        \item \textbf{Universal Instantiation}:
        \begin{itemize}
            \item If a property holds for all members of a set, it holds for any specific member.
            \item \textbf{Example}:
            \begin{itemize}
                \item Premise 1: "All humans are mortal" (\( \forall x \, Human(x) \implies Mortal(x) \))
                \item Premise 2: "Socrates is a human"
                \item Conclusion: "Socrates is mortal"
            \end{itemize}
        \end{itemize}
        \item \textbf{Existential Instantiation}:
        \begin{itemize}
            \item If there exists at least one element in the domain, we can introduce a constant for that element.
            \item \textbf{Example}:
            \begin{itemize}
                \item Premise: "There exists a person who is a philosopher" (\( \exists x \, Philosopher(x) \))
                \item Conclusion: Let \( Socrates \) be that philosopher. \( Philosopher(Socrates) \)
            \end{itemize}
        \end{itemize}
    \end{enumerate}
\end{frame}

\begin{frame}[fragile]
    \frametitle{Summary and Conclusion}
    \begin{itemize}
        \item Inference rules are essential for drawing logical conclusions in FOL.
        \item Key rules:
        \begin{itemize}
            \item Modus Ponens
            \item Modus Tollens
            \item Universal Instantiation
            \item Existential Instantiation
        \end{itemize}
        \item Understanding these rules strengthens critical thinking and analytical skills.
    \end{itemize}
\end{frame}

\begin{frame}[fragile]
  \frametitle{FOL vs. Propositional Logic - Overview}
  % Brief summary of the comparison
  \begin{block}{Understanding the Difference}
    \begin{itemize}
      \item Propositional Logic (PL) represents simple true/false statements.
      \item First-Order Logic (FOL) extends PL with quantifiers and predicates for richer expressions.
    \end{itemize}
  \end{block}
\end{frame}

\begin{frame}[fragile]
  \frametitle{FOL vs. Propositional Logic - Definitions}
  % Definitions of PL and FOL
  \begin{block}{Definitions}
    \begin{itemize}
      \item \textbf{Propositional Logic (PL)}:
        \begin{itemize}
          \item Statements are either true or false.
          \item Examples: 
            \begin{itemize}
              \item ``It is raining."
              \item ``The sky is blue." 
            \end{itemize}
        \end{itemize}
      \item \textbf{First-Order Logic (FOL)}:
        \begin{itemize}
          \item Includes quantifiers and predicates to express statements about objects.
          \item Examples: 
            \begin{itemize}
              \item ``All humans are mortal." ( $\forall x$ (Human(x) $\rightarrow$ Mortal(x)) )
              \item ``Some cats are black." ( $\exists y$ (Cat(y) $\land$ Black(y)) )
            \end{itemize}
        \end{itemize}
    \end{itemize}
  \end{block}
\end{frame}

\begin{frame}[fragile]
  \frametitle{FOL vs. Propositional Logic - Expressiveness and Limitations}
  % Expressiveness and limitations of PL and FOL
  \begin{block}{Expressiveness}
    \begin{itemize}
      \item \textbf{PL}:
        \begin{itemize}
          \item Limited expressiveness, only represents simple true/false statements.
          \item Example: ``If it rains, then the ground is wet." (R $\rightarrow$ W)
        \end{itemize}
      \item \textbf{FOL}:
        \begin{itemize}
          \item Much more expressive; represents complex relationships and quantifies over objects.
          \item Example: ``For every person, if they are a teacher, they have students." 
          ( $\forall x$ (Teacher(x) $\rightarrow$ $\exists y$ (Student(y, x))) )
        \end{itemize}
    \end{itemize}
  \end{block}
  
  \begin{block}{Limitations}
    \begin{itemize}
      \item \textbf{PL}:
        \begin{itemize}
          \item Cannot express properties of objects or relationships between them.
        \end{itemize}
      \item \textbf{FOL}:
        \begin{itemize}
          \item More complex, requires a deeper understanding; computationally demanding.
        \end{itemize}
    \end{itemize}
  \end{block}
\end{frame}

\begin{frame}[fragile]
  \frametitle{FOL vs. Propositional Logic - Key Takeaways}
  % Summary of key takeaways
  \begin{block}{Key Takeaways}
    \begin{itemize}
      \item \textbf{FOL is More Powerful}: It allows for richer, more meaningful knowledge representation.
      \item \textbf{Layered Complexity}: While FOL handles complex scenarios, it demands thoughtful structuring.
    \end{itemize}
  \end{block}
  
  \begin{block}{Conclusion}
    Understanding the distinction between PL and FOL is essential for effective logic reasoning, especially in AI, enhancing decision-making and inference capabilities.
  \end{block}
\end{frame}

\begin{frame}[fragile]
    \frametitle{Applications of First-Order Logic}
    \begin{block}{Introduction to First-Order Logic (FOL) in AI}
        FOL extends propositional logic by incorporating quantifiers and predicates, enabling complex relationships and properties about objects. Its power makes FOL essential in AI and automated reasoning.
    \end{block}
\end{frame}

\begin{frame}[fragile]
    \frametitle{Key Applications of First-Order Logic - Knowledge Representation}
    \begin{itemize}
        \item \textbf{Definition:} Structured way to represent knowledge in a format understandable by machines.
        \item \textbf{Application:} Used in expert systems for specific domains (e.g., medical diagnosis).
        \item \textbf{Example:}
            \begin{itemize}
                \item Facts: $\forall x (\text{Cat}(x) \rightarrow \text{Mammal}(x))$ (Every cat is a mammal)
                \item Rule: $\forall y (\text{Mammal}(y) \land \text{HasFur}(y) \rightarrow \text{WarmBlooded}(y))$ (All mammals with fur are warm-blooded)
            \end{itemize}
    \end{itemize}
\end{frame}

\begin{frame}[fragile]
    \frametitle{Key Applications of First-Order Logic - Automated Reasoning}
    \begin{itemize}
        \item \textbf{Definition:} Used for proving theorems or deriving conclusions via logical deduction.
        \item \textbf{Application:} Helps AI systems infer new information or check consistency of knowledge bases.
        \item \textbf{Example:}
            \begin{itemize}
                \item Given: $\forall x (\text{Student}(x) \rightarrow \text{Enrolled}(x, \text{Math}))$ (Every student is enrolled in Math)
                \item Query: Is John enrolled in Math? (Based on inference rules)
            \end{itemize}
    \end{itemize}
\end{frame}

\begin{frame}[fragile]
    \frametitle{Key Applications of First-Order Logic - Natural Language Processing (NLP)}
    \begin{itemize}
        \item \textbf{Definition:} Models semantics of natural language for better understanding by machines.
        \item \textbf{Application:} Involved in semantic analysis and understanding context of sentences.
        \item \textbf{Example:}
            \begin{itemize}
                \item Sentence: "All dogs bark."
                \item FOL Representation: $\forall x (\text{Dog}(x) \rightarrow \text{Bark}(x))$
            \end{itemize}
    \end{itemize}
\end{frame}

\begin{frame}[fragile]
    \frametitle{Key Applications of First-Order Logic - Database Querying}
    \begin{itemize}
        \item \textbf{Definition:} Provides logical structure for database queries, aiding precise information retrieval.
        \item \textbf{Application:} Used in relational databases to retrieve information based on logical conditions.
        \item \textbf{Example:}
            \begin{itemize}
                \item Query: Find all employees with salaries greater than $50,000.
                \item FOL Representation: $\exists e (\text{Employee}(e) \land \text{Salary}(e) > 50000)$
            \end{itemize}
    \end{itemize}
\end{frame}

\begin{frame}[fragile]
    \frametitle{Key Applications of First-Order Logic - Robotics and Automated Planning}
    \begin{itemize}
        \item \textbf{Definition:} Assists robots in reasoning about their environment and planning actions.
        \item \textbf{Application:} Utilized in AI planning systems for deriving sequences of actions.
        \item \textbf{Example:}
            \begin{itemize}
                \item If it is raining, then the robot must carry an umbrella.
                \item FOL Representation: $\text{Rain} \rightarrow \text{CarryUmbrella}$
            \end{itemize}
    \end{itemize}
\end{frame}

\begin{frame}[fragile]
    \frametitle{Key Points to Emphasize}
    \begin{itemize}
        \item \textbf{Expressiveness:} FOL expresses complex relationships via predicates and quantifiers.
        \item \textbf{Versatility:} Applicable in multiple domains, driving advancements in AI.
        \item \textbf{Inference Power:} Capable of deriving conclusions to enhance decision-making.
    \end{itemize}
\end{frame}

\begin{frame}[fragile]
    \frametitle{Conclusion}
    First-Order Logic is fundamental in AI, supporting knowledge representation, reasoning, and decision-making across various applications. Understanding FOL empowers students to apply these concepts in real-world scenarios, bridging the gap between theoretical logic and practical AI implementations.
\end{frame}

\begin{frame}[fragile]
    \frametitle{Limitations of First-Order Logic (FOL) - Introduction}
    \begin{block}{Introduction}
        First-Order Logic (FOL) is a powerful formalism used in various fields such as computer science, linguistics, and artificial intelligence. However, while it provides a robust framework for reasoning about the world, it has several notable limitations that practitioners must understand.
    \end{block}
\end{frame}

\begin{frame}[fragile]
    \frametitle{Limitations of FOL - Expressiveness and Decidability}
    \begin{enumerate}
        \item \textbf{Expressiveness Limitations}
        \begin{itemize}
            \item Inability to express certain concepts (e.g., all possible properties).
            \item \textit{Example:} FOL struggles with statements like "for every property P, there exists an object that has property P".
            \item \textit{Illustration:} Expressing "there exists a beautiful object that is not a tree" presents challenges.
        \end{itemize}
        
        \item \textbf{Decidability Issues}
        \begin{itemize}
            \item FOL is undecidable; no algorithm can determine the truth of every statement.
            \item \textit{Example:} The validity of statements in FOL, as seen in Peano Arithmetic, can lead to contradictions.
        \end{itemize}
    \end{enumerate}
\end{frame}

\begin{frame}[fragile]
    \frametitle{Limitations of FOL - Complexity and Non-Monotonicity}
    \begin{enumerate}
        \setcounter{enumi}{2} % Continue enumeration from previous frame
        \item \textbf{Complexity of Inference}
        \begin{itemize}
            \item Reasoning in FOL is computationally intensive; the resources grow with the number of variables and predicates.
            \item \textit{Key Point:} This complexity can make FOL impractical for large datasets or real-time applications.
        \end{itemize}
        
        \item \textbf{Lack of Non-Monotonicity}
        \begin{itemize}
            \item FOL adheres to monotonic reasoning; adding new premises cannot invalidate previous conclusions.
            \item \textit{Example:} Inferring "birds can fly" becomes problematic when encountering a flightless bird.
        \end{itemize}
    \end{enumerate}
\end{frame}

\begin{frame}[fragile]
    \frametitle{Limitations of FOL - Quantifier Issues and Conclusion}
    \begin{enumerate}
        \setcounter{enumi}{4} % Continue enumeration from previous frame
        \item \textbf{Quantifier Limitations}
        \begin{itemize}
            \item FOL is limited to quantifying over individual objects, not over sets or relations.
            \item \textit{Example:} Richer logical frameworks are needed to express statements like "For every set of cats, there exists a member that is also a mammal".
        \end{itemize}
    \end{enumerate}
    
    \begin{block}{Conclusion}
        While First-Order Logic remains a fundamental tool in logical reasoning, understanding its limitations is crucial for selecting the appropriate framework for a given problem. Recognizing when FOL falls short can guide us towards alternative approaches and improvements.
    \end{block}
    
    \begin{block}{Key Takeaways}
        - FOL is powerful but has expressiveness and decidability limitations.
        - Complex reasoning challenges restrict practical applications.
        - Recognizing these constraints is vital for effective AI knowledge representation.
    \end{block}
\end{frame}

\begin{frame}[fragile]
    \frametitle{Further Study}
    \begin{block}{Further Study}
        Reflect on potential alternative logics (e.g., Higher-Order Logic or Modal Logic) and explore their capabilities in overcoming FOL limitations.
    \end{block}
\end{frame}

\begin{frame}[fragile]
    \frametitle{Knowledge Representation - Introduction}
    Knowledge representation is a critical component in artificial intelligence (AI) that involves encoding information for computer systems. 
    First-Order Logic (FOL) is a powerful tool for knowledge representation due to its expressiveness, allowing nuanced descriptions of the relationships and properties of objects.
\end{frame}

\begin{frame}[fragile]
    \frametitle{Knowledge Representation - Key Concepts}
    \begin{block}{Predicates and Predication}
        A predicate expresses a property of or a relationship between objects. 
        For example: 
        \begin{itemize}
            \item $Loves(John, Mary)$ indicates that John loves Mary.
        \end{itemize}
    \end{block}
    
    \begin{block}{Quantifiers}
        \begin{itemize}
            \item Universal Quantifier ($\forall$): Indicates that a property holds for all instances. 
                \begin{itemize}
                    \item Example: $\forall x (Cat(x) \rightarrow Animal(x))$.
                \end{itemize}
            \item Existential Quantifier ($\exists$): Suggests that there exists at least one instance for which the property holds true.
                \begin{itemize}
                    \item Example: $\exists y (Likes(John, y))$.
                \end{itemize}
        \end{itemize}
    \end{block}
\end{frame}

\begin{frame}[fragile]
    \frametitle{Knowledge Representation - Structuring Knowledge}
    FOL enables structured representation of knowledge using the following components:
    \begin{itemize}
        \item Atomic Sentences: Basic statements that can be either true or false.
        \item Complex Sentences: Constructed using logical connectives such as AND ($\land$), OR ($\lor$), NOT ($\neg$), and IMPLIES ($\to$).
    \end{itemize}

    \begin{block}{Example of Knowledge Representation}
        Consider the following statements:
        \begin{enumerate}
            \item All humans are mortal: $\forall x (Human(x) \rightarrow Mortal(x))$.
            \item Socrates is a human: $Human(Socrates)$.
        \end{enumerate}
        From these, we can conclude: $Mortal(Socrates)$.
    \end{block}
\end{frame}

\begin{frame}[fragile]
    \frametitle{Knowledge Representation - Benefits of FOL}
    \begin{itemize}
        \item \textbf{Expressiveness}: FOL can articulate a wide variety of assertions about the world.
        \item \textbf{Inference Capabilities}: Supports automated reasoning and decision-making based on facts.
        \item \textbf{Modularity}: Knowledge can be represented modularly, simplifying updates and management.
    \end{itemize}

    \begin{block}{Conclusion}
        First-Order Logic serves as a foundational framework for knowledge representation in AI systems, facilitating reasoning and interaction with the world.
    \end{block}
    
    \begin{block}{Key Points to Remember}
        \begin{itemize}
            \item FOL allows representation of complex relationships with predicates and quantifiers.
            \item Knowledge is structured into atomic and complex sentences for effective reasoning.
            \item Automated inference capabilities make FOL a powerful tool in AI applications.
        \end{itemize}
    \end{block}
\end{frame}

\begin{frame}[fragile]
  \frametitle{FOL in Modern AI}
  \begin{block}{Introduction to First-Order Logic (FOL)}
    First-Order Logic (FOL) is a powerful framework for knowledge representation and reasoning in AI. 
    It extends propositional logic by allowing the use of predicates, quantifiers, and variables, thus enabling more complex statements and relationships.
  \end{block}
\end{frame}

\begin{frame}[fragile]
  \frametitle{Integration of FOL with Other AI Techniques}
  \begin{enumerate}
    \item \textbf{Combining FOL with Machine Learning (ML)}
      \begin{itemize}
        \item FOL enhances natural language processing (NLP) by formalizing logical rules from patterns recognized by ML models.
        \item Improves interpretability and enables reasoning about relationships in text data.
      \end{itemize}
    
    \item \textbf{FOL and Ontologies}
      \begin{itemize}
        \item FOL aids in defining ontologies, standardizing information across domains.
        \item In semantic web technologies, FOL-based ontologies facilitate intelligent information retrieval.
      \end{itemize}
    
    \item \textbf{FOL in Robotic Systems}
      \begin{itemize}
        \item Robots use FOL to understand commands and reason about environments.
        \item Enables robots to infer facts, like optimal paths based on existing knowledge.
      \end{itemize}
  \end{enumerate}
\end{frame}

\begin{frame}[fragile]
  \frametitle{Relevance of FOL Today}
  \begin{itemize}
    \item \textbf{Enhanced Decision-Making:} 
      FOL supports AI systems in making logical deductions, leading to improved decision capabilities.
    \item \textbf{Complex Problem Solving:} 
      It allows reasoning in uncertain environments, aiding in logical conclusions.
    \item \textbf{Interdisciplinary Applications:} 
      FOL's versatility is evident in fields like healthcare (e.g., diagnosis) and finance (e.g., fraud detection).
  \end{itemize}
\end{frame}

\begin{frame}[fragile]
  \frametitle{Key Points and Conclusion}
  \begin{block}{Key Points to Emphasize}
    \begin{itemize}
      \item \textbf{Native Expressiveness:} 
        FOL enables inferences and deduction of new facts from existing knowledge.
      \item \textbf{Bridging Paradigms:} 
        Integration with AI paradigms (e.g., ML, ontologies) leverages strengths to tackle complex issues.
      \item \textbf{Ongoing Research:} 
        Continuous evolution ensures FOL remains relevant in intelligent system development.
    \end{itemize}
  \end{block}
  
  \begin{block}{Conclusion}
    The integration of FOL with various AI techniques underscores its critical role in modern AI applications. 
  \end{block}
\end{frame}

\begin{frame}[fragile]
    \frametitle{Case Study: FOL in Action}
    \begin{block}{What is First-Order Logic (FOL)?}
        First-Order Logic is a formal system used in mathematical logic, computer science, and artificial intelligence. 
        It extends propositional logic by allowing the use of quantified variables, representing objects in a defined domain. This enables complex statements about relationships and properties.
    \end{block}
\end{frame}

\begin{frame}[fragile]
    \frametitle{Real-World Scenario: A Smart Home Automation System}
    \begin{block}{Context}
        Consider a smart home system that automates tasks based on user preferences and environmental conditions. 
        FOL can effectively represent and reason about the behaviors and rules governing this system.
    \end{block}
\end{frame}

\begin{frame}[fragile]
    \frametitle{Key Concepts Illustrated}
    \begin{enumerate}
        \item \textbf{Predicates and Quantifiers}
        \begin{itemize}
            \item \textbf{Predicates}: Functions returning true or false based on input. E.g., $SmartDevice(Device)$ indicates if a device is a smart device.
            \item \textbf{Quantifiers}: ‘$\forall$’ (for all) and ‘$\exists$’ (there exists). E.g., $\forall x (SmartDevice(x) \to CanControl(x))$ means every smart device can be controlled.
        \end{itemize}
        
        \item \textbf{Rules and Inference}
        \begin{itemize}
            \item FOL allows formulating rules governing the system. 
            \item Example: If a user is home and the outdoor temperature is below 60°F, the system should turn on the thermostat.
            \item FOL Representation:
            \begin{lstlisting}
            ∀y (UserHome(y) ∧ TempOutside < 60°F → ThermostatOn)
            \end{lstlisting}
        \end{itemize}

        \item \textbf{Example Statements}
        \begin{itemize}
            \item Base predicates:
                \begin{itemize}
                    \item $SmartDevice(thermostat)$
                    \item $UserHome(Alice)$
                    \item $TempOutside < 60$
                \end{itemize}
            \item From these, we can derive:
            \begin{lstlisting}
            SmartDevice(thermostat) ∧ UserHome(Alice) ∧ (TempOutside < 60) → ThermostatOn
            \end{lstlisting}
        \end{itemize}
    \end{enumerate}
\end{frame}

\begin{frame}[fragile]
    \frametitle{Key Points to Emphasize}
    \begin{itemize}
        \item \textbf{Enhanced Reasoning Capability}: FOL allows the smart home system to infer actions from relationships and rules.
        \item \textbf{Flexibility and Scalability}: As devices and rules are added, FOL accommodates changes without a complete overhaul.
        \item \textbf{Automated Decision Making}: FOL enables automation of responses based on logical deductions for intelligent adaptation.
    \end{itemize}
\end{frame}

\begin{frame}[fragile]
    \frametitle{Conclusion}
    The application of First-Order Logic in scenarios like smart home automation illustrates its power in reasoning and inference. 
    By structuring rules and relationships logically, we ensure that systems respond accurately to user needs and environmental changes.
    
    \begin{block}{Note}
        Practice framing your own FOL statements based on different scenarios to get comfortable with syntax and logic structure!
    \end{block}
\end{frame}

\begin{frame}[fragile]
    \frametitle{Summary of Key Points}
    \begin{block}{Key Concepts in First-Order Logic (FOL)}
        Recap of the main topics discussed regarding First-Order Logic.
    \end{block}
\end{frame}

\begin{frame}[fragile]
    \frametitle{Definition of First-Order Logic}
    \begin{itemize}
        \item \textbf{First-Order Logic (FOL)} extends propositional logic by incorporating quantifiers and predicates, enabling more expressive reasoning about objects and their relationships.
        \item \textbf{Key Terms:}
        \begin{itemize}
            \item \textbf{Predicates:} Functions that describe properties of objects (e.g., $IsTall(x)$).
            \item \textbf{Quantifiers:}
            \begin{itemize}
                \item \textbf{Universal Quantifier ($\forall$):} Indicates that a property holds for all elements in a domain.
                \item \textbf{Existential Quantifier ($\exists$):} Indicates that at least one element in the domain satisfies the property.
            \end{itemize}
        \end{itemize}
    \end{itemize}
\end{frame}

\begin{frame}[fragile]
    \frametitle{Structure of FOL Statements and Inference Rules}
    \begin{itemize}
        \item FOL allows us to construct statements in the form of:
        \begin{itemize}
            \item \textbf{General Statements:} $\forall x (IsTall(x) \rightarrow HasHeight(x, y))$
            \item \textbf{Existential Statements:} $\exists y (HasHeight(x, y) \land IsTall(x))$
        \end{itemize}

        \item FOL employs inference rules to derive logical conclusions, such as:
        \begin{itemize}
            \item \textbf{Modus Ponens:} If $P \rightarrow Q$ and $P$ is true, then $Q$ is true.
            \item \textbf{Universal Instantiation:} If $\forall x P(x)$ is true, then $P(c)$ is true for any specific $c$.
        \end{itemize}
    \end{itemize}
\end{frame}

\begin{frame}[fragile]
    \frametitle{Applications and Limitations of FOL}
    \begin{itemize}
        \item \textbf{Applications of FOL}
        \begin{itemize}
            \item \textbf{Artificial Intelligence:} FOL is used for knowledge representation, enabling machines to reason about complex properties and operations.
            \item \textbf{Computer Science:} It underpins database query languages, automated theorem proving, and more.
        \end{itemize}
        
        \item \textbf{Limitations of FOL}
        \begin{itemize}
            \item \textbf{Complexity:} The decision problem for FOL is undecidable; not all statements can be proven true or false.
            \item \textbf{Expressiveness vs. Computability:} FOL can express many logical statements, but reasoning may become computationally intensive.
        \end{itemize}
    \end{itemize}
\end{frame}

\begin{frame}[fragile]
    \frametitle{Key Takeaways}
    \begin{itemize}
        \item FOL provides a robust framework for formal reasoning and is vital in various fields such as AI and computer science.
        \item Understanding quantifiers, predicates, and inference rules is essential for mastering FOL.
        \item While powerful, FOL has inherent limitations in terms of decidability and computational complexity.
    \end{itemize}
\end{frame}

\begin{frame}[fragile]
    \frametitle{Examples}
    \begin{itemize}
        \item \textbf{Statement:} "All humans are mortal."
        \begin{equation}
            \forall x (Human(x) \rightarrow Mortal(x))
        \end{equation}

        \item \textbf{Existential example:} "There exists a human who is a philosopher."
        \begin{equation}
            \exists x (Human(x) \land Philosopher(x))
        \end{equation}
    \end{itemize}
\end{frame}

\begin{frame}[fragile]
    \frametitle{Further Reading - Introduction to FOL}
    \begin{block}{Introduction to First-Order Logic (FOL)}
        First-Order Logic is a powerful framework used in mathematics, philosophy, artificial intelligence, and computer science for reasoning about the properties of objects and their relations. 
        To deepen your understanding of FOL, the following resources and readings are recommended.
    \end{block}
\end{frame}

\begin{frame}[fragile]
    \frametitle{Further Reading - Suggested Readings}
    \begin{block}{Suggested Readings}
        \begin{enumerate}
            \item \textbf{Books:}
                \begin{itemize}
                    \item \textit{"First-Order Logic" by Raymond Smullyan} - A solid foundation emphasizing logical puzzles.
                    \item \textit{"Logic: A Very Short Introduction" by Graham Priest} - Concise overview and applications.
                    \item \textit{"Mathematical Logic" by Joseph Rosen} - Comprehensive coverage with focus on FOL.
                \end{itemize}
            \item \textbf{Academic Papers:}
                \begin{itemize}
                    \item \textit{"A Survey of First-Order Logic"} - Reviews essential topics including syntax and semantics.
                    \item \textit{"Automated Theorem Proving: A Historical Perspective" by John McCarthy} - Discusses theorem proving techniques.
                \end{itemize}
            \item \textbf{Online Resources:}
                \begin{itemize}
                    \item \textit{Stanford Encyclopedia of Philosophy} - In-depth articles on foundational concepts.
                    \item \textit{Coursera and edX} - Courses on logic and AI with modules on FOL.
                \end{itemize}
        \end{enumerate}
    \end{block}
\end{frame}

\begin{frame}[fragile]
    \frametitle{Further Reading - Key Points and Example}
    \begin{block}{Key Points to Emphasize}
        \begin{itemize}
            \item \textbf{Importance of FOL:} Foundation for advanced reasoning tasks in computer science and formal proofs.
            \item \textbf{Connections to AI and Computer Science:} Crucial for areas like Natural Language Processing and Knowledge Representation.
            \item \textbf{Practical Application:} Engage with materials that cover both theory and real-world use cases.
        \end{itemize}
    \end{block}
    
    \begin{block}{Example Formula}
        To illustrate FOL, consider the statement:
        \begin{equation}
            \forall x (Human(x) \rightarrow Mortal(x))
        \end{equation}
        This means "For all x, if x is a human, then x is mortal."
    \end{block}
\end{frame}

\begin{frame}[fragile]
  \frametitle{Q\&A Session - Introduction}
  \begin{itemize}
    \item Welcome to the Q\&A session focused on \textbf{First-Order Logic (FOL)}.
    \item Opportunity for clarification of concepts and sharing thoughts.
    \item Discussion on applications of FOL in various fields.
  \end{itemize}
\end{frame}

\begin{frame}[fragile]
  \frametitle{Q\&A Session - Key Concepts}
  \begin{block}{Key Concepts to Review}
    \begin{enumerate}
      \item \textbf{First-Order Logic (FOL)}:
        \begin{itemize}
          \item An extension of propositional logic that allows reasoning about objects and their relationships.
          \item Comprises predicates, quantifiers, functions, and constants.
        \end{itemize}
      \item \textbf{Predicates}:
        \begin{itemize}
          \item Functions that return true or false based on the input. 
          \item Example: P(x): "x is a student."
        \end{itemize}
      \item \textbf{Quantifiers}:
        \begin{itemize}
          \item \textbf{Universal Quantifier (∀)}: Applies to all elements. E.g., $\forall x \, P(x)$ means "For all x, P(x) is true."
          \item \textbf{Existential Quantifier (∃)}: There is at least one element for which the statement is true. E.g., $\exists x \, P(x)$ means "There exists an x such that P(x) is true."
        \end{itemize}
    \end{enumerate}
  \end{block}
\end{frame}

\begin{frame}[fragile]
  \frametitle{Q\&A Session - Applications and Discussion}
  \begin{block}{Practical Applications of FOL}
    \begin{itemize}
      \item \textbf{Artificial Intelligence}: Knowledge representation and reasoning for understanding and manipulating information logically.
      \item \textbf{Database Query Languages}: Expression of schema and constraints in FOL to enhance database capabilities.
      \item \textbf{Natural Language Processing}: Helps in modeling relationships and meanings logically.
    \end{itemize}
  \end{block}
  
  \begin{block}{Example for Discussion}
    \begin{itemize}
      \item Example Statement: "All humans are mortal." 
      \item FOL Representation: $\forall x \, (Human(x) \rightarrow Mortal(x))$
      \item Discussion: What does this say about our ability to infer other statements from this premise?
    \end{itemize}
  \end{block}
\end{frame}

\begin{frame}[fragile]
  \frametitle{Q\&A Session - Open Floor}
  \begin{block}{Key Points to Emphasize}
    \begin{itemize}
      \item Importance of clarity in predicates and the proper use of quantifiers.
      \item The vast and growing relationship between FOL and computational applications.
      \item FOL serves as a foundational element for more complex logic systems.
    \end{itemize}
  \end{block}
  
  \begin{block}{Open Floor for Questions}
    \begin{itemize}
      \item What specific aspects of FOL would you like to explore further?
      \item Are there practical scenarios or applications of FOL you find intriguing or confusing?
      \item How do you see FOL influencing advancements in technology and computational logic?
    \end{itemize}
  \end{block}
  
  \begin{block}{Conclusion}
    \begin{itemize}
      \item Your questions drive deeper understanding; no question is too small or complex.
      \item Let’s engage in a thoughtful discussion to expand our collective knowledge of FOL and its applications!
    \end{itemize}
  \end{block}
\end{frame}


\end{document}