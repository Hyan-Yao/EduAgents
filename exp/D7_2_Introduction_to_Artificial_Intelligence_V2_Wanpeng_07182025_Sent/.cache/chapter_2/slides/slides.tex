\documentclass[aspectratio=169]{beamer}

% Theme and Color Setup
\usetheme{Madrid}
\usecolortheme{whale}
\useinnertheme{rectangles}
\useoutertheme{miniframes}

% Additional Packages
\usepackage[utf8]{inputenc}
\usepackage[T1]{fontenc}
\usepackage{graphicx}
\usepackage{booktabs}
\usepackage{listings}
\usepackage{amsmath}
\usepackage{amssymb}
\usepackage{xcolor}
\usepackage{tikz}
\usepackage{pgfplots}
\pgfplotsset{compat=1.18}
\usetikzlibrary{positioning}
\usepackage{hyperref}

% Custom Colors
\definecolor{myblue}{RGB}{31, 73, 125}
\definecolor{mygray}{RGB}{100, 100, 100}
\definecolor{mygreen}{RGB}{0, 128, 0}
\definecolor{myorange}{RGB}{230, 126, 34}
\definecolor{mycodebackground}{RGB}{245, 245, 245}

% Set Theme Colors
\setbeamercolor{structure}{fg=myblue}
\setbeamercolor{frametitle}{fg=white, bg=myblue}
\setbeamercolor{title}{fg=myblue}
\setbeamercolor{section in toc}{fg=myblue}
\setbeamercolor{item projected}{fg=white, bg=myblue}
\setbeamercolor{block title}{bg=myblue!20, fg=myblue}
\setbeamercolor{block body}{bg=myblue!10}
\setbeamercolor{alerted text}{fg=myorange}

% Set Fonts
\setbeamerfont{title}{size=\Large, series=\bfseries}
\setbeamerfont{frametitle}{size=\large, series=\bfseries}
\setbeamerfont{caption}{size=\small}
\setbeamerfont{footnote}{size=\tiny}

% Footer and Navigation Setup
\setbeamertemplate{footline}{
  \leavevmode%
  \hbox{%
  \begin{beamercolorbox}[wd=.3\paperwidth,ht=2.25ex,dp=1ex,center]{author in head/foot}%
    \usebeamerfont{author in head/foot}\insertshortauthor
  \end{beamercolorbox}%
  \begin{beamercolorbox}[wd=.5\paperwidth,ht=2.25ex,dp=1ex,center]{title in head/foot}%
    \usebeamerfont{title in head/foot}\insertshorttitle
  \end{beamercolorbox}%
  \begin{beamercolorbox}[wd=.2\paperwidth,ht=2.25ex,dp=1ex,center]{date in head/foot}%
    \usebeamerfont{date in head/foot}
    \insertframenumber{} / \inserttotalframenumber
  \end{beamercolorbox}}%
  \vskip0pt%
}

% Turn off navigation symbols
\setbeamertemplate{navigation symbols}{}

% Title Page Information
\title[AI Foundations]{Chapter 2: AI Foundations and Terminologies}
\author[J. Smith]{John Smith, Ph.D.}
\institute[University Name]{
  Department of Computer Science\\
  University Name\\
  \vspace{0.3cm}
  Email: email@university.edu\\
  Website: www.university.edu
}
\date{\today}

% Document Start
\begin{document}

\frame{\titlepage}

\begin{frame}[fragile]
    \frametitle{Introduction to AI Foundations}
    \begin{block}{Objectives of the Chapter}
        \begin{enumerate}
            \item \textbf{Understand Key Terminologies}: Familiarize yourself with fundamental AI concepts such as machine learning, neural networks, and natural language processing.
            \item \textbf{Explore the Scope of AI}: Recognize the different domains and applications of AI technology, including healthcare, finance, and autonomous systems.
            \item \textbf{Identify AI Techniques}: Gain insights into various AI methodologies and their practical implications, such as supervised learning, unsupervised learning, and reinforcement learning.
        \end{enumerate}
    \end{block}
\end{frame}

\begin{frame}[fragile]
    \frametitle{Importance of AI Concepts}
    \begin{itemize}
        \item \textbf{Impact on Industries}: AI is transforming how businesses operate. For instance, in healthcare, AI algorithms support diagnosis by analyzing medical imaging, while in finance, they detect fraudulent activities by recognizing patterns in transaction data.
        \item \textbf{Foundation for Innovation}: A solid understanding of AI principles is essential for future innovations. Startups and tech companies increasingly rely on AI to create competitive advantages.
        \item \textbf{Integration with Everyday Life}: From virtual assistants like Siri and Alexa to recommendation systems used by Netflix and Amazon, understanding AI helps us navigate and utilize these technologies effectively.
    \end{itemize}
\end{frame}

\begin{frame}[fragile]
    \frametitle{Key Points to Emphasize}
    \begin{itemize}
        \item \textbf{Terminologies Matter}: Mastering the language of AI is critical for clear communication in both academic and professional settings.
        \item \textbf{Cross-Disciplinary Relevance}: AI concepts cut across various fields—engineering, medicine, economics—indicating the breadth of their application.
        \item \textbf{Ethical Considerations}: As AI capabilities grow, understanding its ethical implications becomes essential. Knowledge of AI terminologies also facilitates discussions about responsible AI use.
    \end{itemize}
\end{frame}

\begin{frame}[fragile]
    \frametitle{Illustrative Example}
    \begin{block}{Smart Home AI System}
        Consider how an AI system is set up in a smart home. Through voice commands, users interact with AI, which processes natural language and executes tasks (turning on lights, setting reminders). The AI system learns from user preferences, demonstrating concepts like machine learning in a relatable context.
    \end{block}
\end{frame}

\begin{frame}[fragile]
    \frametitle{Conclusion}
    Grasping the foundations of AI will not only empower you within this course but also enhance your ability to engage with AI technologies in real-world scenarios. The journey begins here—let's explore the fascinating world of Artificial Intelligence!
\end{frame}

\begin{frame}[fragile]
    \frametitle{What is Artificial Intelligence? - Definition}
    \begin{block}{Definition of AI}
        Artificial Intelligence (AI) is a branch of computer science focused on creating systems capable of performing tasks that typically require human intelligence. These tasks can include:
    \end{block}
    \begin{itemize}
        \item Problem-solving
        \item Learning
        \item Reasoning
        \item Perception
        \item Understanding natural language
    \end{itemize}
    By leveraging algorithms and large datasets, AI systems can analyze information, recognize patterns, and make decisions with varying degrees of autonomy.
\end{frame}

\begin{frame}[fragile]
    \frametitle{What is Artificial Intelligence? - Significance in Applications}
    \begin{block}{Significance of AI in Various Applications}
        AI is transforming various sectors through innovative applications:
    \end{block}
    \begin{enumerate}
        \item \textbf{Healthcare}
            \begin{itemize}
                \item AI Applications: Diagnostic systems, personalized medicine, robotic surgeries
                \item Example: AI algorithms can analyze medical images (like X-rays or MRIs) to detect anomalies such as tumors.
            \end{itemize}
        \item \textbf{Finance}
            \begin{itemize}
                \item AI Applications: Fraud detection, algorithmic trading, credit risk assessment
                \item Example: AI monitors transactions in real-time to identify potential fraud.
            \end{itemize}
        \item \textbf{Automotive}
            \begin{itemize}
                \item AI Applications: Autonomous vehicles, driver assistance systems
                \item Example: Self-driving cars use AI to navigate safely using sensor data.
            \end{itemize}
        \item \textbf{Retail and E-commerce}
            \begin{itemize}
                \item AI Applications: Customer recommendation systems, inventory management
                \item Example: Platforms like Amazon use AI to suggest products based on previous behavior.
            \end{itemize}
        \item \textbf{Education}
            \begin{itemize}
                \item AI Applications: Personalized learning platforms, automated grading systems
                \item Example: AI tutoring systems adapt to individual student needs.
            \end{itemize}
    \end{enumerate}
\end{frame}

\begin{frame}[fragile]
    \frametitle{What is Artificial Intelligence? - Key Points and Conclusion}
    \begin{block}{Key Points to Emphasize}
        \begin{itemize}
            \item \textbf{Dynamics of Learning:} AI systems improve over time through machine learning.
            \item \textbf{Human-AI Collaboration:} AI enhances human capabilities and augments decision-making.
            \item \textbf{Ethical Considerations:} Advances in AI raise ethical questions regarding privacy, bias, and job displacement.
        \end{itemize}
    \end{block}
    
    \begin{block}{Conclusion}
        Understanding AI's definition and its applications helps set the stage for deeper exploration into technologies, concepts, and ethical considerations shaping AI's future.
    \end{block}
    \vspace{10pt}
    \begin{block}{Next Steps}
        In the next slide, we will delve into key AI terminologies essential for comprehending this evolving field.
    \end{block}
\end{frame}

\begin{frame}[fragile]
    \frametitle{Key AI Terminologies - Introduction}
    \begin{block}{Introduction to Essential AI Terms}
        In this section, we'll define fundamental terms that are pivotal to understanding Artificial Intelligence (AI). Familiarizing yourself with these terminologies will prepare you for the subsequent topics covered in this course.
    \end{block}
\end{frame}

\begin{frame}[fragile]
    \frametitle{Key AI Terminologies - Core Concepts}
    \begin{enumerate}
        \item \textbf{Algorithm}
        \begin{itemize}
            \item \textbf{Definition}: A set of rules or instructions given to a computer to help it achieve a specific task.
            \item \textbf{Example}: Decision Tree for classification tasks.
        \end{itemize}
        
        \item \textbf{Model}
        \begin{itemize}
            \item \textbf{Definition}: A mathematical representation of a real-world process created through training.
            \item \textbf{Example}: Neural network for image recognition.
        \end{itemize}
    \end{enumerate}
\end{frame}

\begin{frame}[fragile]
    \frametitle{Key AI Terminologies - Continued}
    \begin{enumerate}[resume]
        \item \textbf{Training Data}
        \begin{itemize}
            \item \textbf{Definition}: Datasets used to train Machine Learning models; must represent real-world scenarios.
            \item \textbf{Example}: Labeled emails for a spam classifier.
        \end{itemize}

        \item \textbf{Feature}
        \begin{itemize}
            \item \textbf{Definition}: Individual measurable properties or characteristics of the data used as input variables.
            \item \textbf{Example}: Square footage and location in housing price prediction.
        \end{itemize}

        \item \textbf{Overfitting}
        \begin{itemize}
            \item \textbf{Definition}: When a model learns the training data too well, leading to poor performance on unseen data.
            \item \textbf{Example}: A stock price prediction model that fails to generalize.
        \end{itemize}
        
        \item \textbf{Underfitting}
        \begin{itemize}
            \item \textbf{Definition}: Occurs when a model is too simple to capture underlying patterns, resulting in poor performance.
            \item \textbf{Example}: Linear regression on a complex nonlinear relationship.
        \end{itemize}
    \end{enumerate}
\end{frame}

\begin{frame}[fragile]
    \frametitle{Key AI Terminologies - Key Points}
    \begin{block}{Key Points to Emphasize}
        \begin{itemize}
            \item Understanding these terminologies is crucial for grasping more complex concepts in AI.
            \item Algorithms and models are the backbone of AI systems; knowing how they work helps in predicting outcomes.
            \item The quality of training data significantly impacts model effectiveness; always consider the representativeness of your training data.
        \end{itemize}
    \end{block}
\end{frame}

\begin{frame}[fragile]
    \frametitle{Key AI Terminologies - Visual Aids}
    \begin{block}{Diagrams (To Consider Adding in Slide)}
        \begin{itemize}
            \item \textbf{Decision Tree Illustration}: A simple diagram showing the structure of a decision tree algorithm.
            \item \textbf{Neural Network Schema}: An outline of a basic neural network structure illustrating input, hidden, and output layers.
        \end{itemize}
    \end{block}
\end{frame}

\begin{frame}[fragile]
    \frametitle{Search Strategies - Overview}
    \begin{block}{Understanding Search Strategies in AI}
        Search strategies are fundamental techniques used to navigate and solve problems by exploring potential solutions. They can be categorized into two main types:
        \begin{itemize}
            \item \textbf{Uninformed Search Strategies}
            \item \textbf{Informed Search Strategies}
        \end{itemize}
    \end{block}
\end{frame}

\begin{frame}[fragile]
    \frametitle{Search Strategies - Uninformed Search}
    \begin{block}{Uninformed Search Strategies}
        \textbf{Definition:} Do not have additional information about the states beyond the problem definition and operate without knowledge of the goal state.
        
        \textbf{Key Properties:}
        \begin{itemize}
            \item Systematic approaches to explore the search space.
            \item Do not utilize domain-specific knowledge.
        \end{itemize}

        \textbf{Common Types:}
        \begin{itemize}
            \item \textbf{Breadth-First Search (BFS)}
            \begin{itemize}
                \item Explores all nodes at the present depth level before moving to the next.
                \item \textbf{Example:} Finding the shortest path in an unweighted graph.
            \end{itemize}

            \item \textbf{Depth-First Search (DFS)}
            \begin{itemize}
                \item Explores as far as possible along one branch before backtracking.
                \item Efficient in memory but may get trapped in deep paths.
            \end{itemize}
        \end{itemize}
    \end{block}
\end{frame}

\begin{frame}[fragile]
    \frametitle{Search Strategies - Informed Search}
    \begin{block}{Informed Search Strategies}
        \textbf{Definition:} Use additional information (heuristics) to estimate the cost from the current state to the goal.

        \textbf{Key Properties:}
        \begin{itemize}
            \item Use domain knowledge to guide the search.
            \item More efficient in time and space compared to uninformed strategies.
        \end{itemize}

        \textbf{Common Types:}
        \begin{itemize}
            \item \textbf{A* Search}
            \begin{itemize}
                \item Combines the cost to reach a node \(g(n)\) and the estimated cost to get from that node to the goal \(h(n)\).
                \item \textbf{Formula:} \( f(n) = g(n) + h(n) \)
            \end{itemize}
            \item \textbf{Greedy Best-First Search}
            \begin{itemize}
                \item Selects the node with the lowest heuristic cost \(h(n)\).
                \item May not find the shortest path but is faster than A* search in some contexts.
            \end{itemize}
        \end{itemize}
    \end{block}
\end{frame}

\begin{frame}[fragile]
    \frametitle{Search Strategies - Key Points and Conclusion}
    \begin{block}{Key Points to Emphasize}
        \begin{itemize}
            \item \textbf{Uninformed searches} provide thorough exploration but may be inefficient.
            \item \textbf{Informed searches} facilitate a more efficient path to the goal using heuristics.
            \item Choosing which strategy to apply depends on the specific problem context and available information.
        \end{itemize}
    \end{block}

    \begin{block}{Conclusion}
        Search strategies are crucial for problem-solving in AI. Understanding the difference between uninformed and informed strategies allows practitioners to choose the most efficient methods for their applications, leading to more effective AI systems.
    \end{block}
\end{frame}

\begin{frame}[fragile]
  \frametitle{Logic Reasoning in AI - Introduction}
  \begin{block}{Introduction to Logic in AI}
    Logic is a fundamental aspect of Artificial Intelligence (AI) that underlies reasoning, decision-making, and inference processes. 
    It is used to represent knowledge in a structured way that a computer can interpret.
  \end{block}
\end{frame}

\begin{frame}[fragile]
  \frametitle{Logic Reasoning in AI - Propositional Logic}
  \begin{block}{Definition}
    Propositional Logic, also known as Boolean logic, deals with propositions that can be true or false.
  \end{block}
  
  \begin{itemize}
    \item \textbf{Basic Components:}
      \begin{itemize}
        \item \textbf{Propositions:} Statements like "It is raining" (P) or "It is sunny" (Q).
        \item \textbf{Connectives:} 
          \begin{itemize}
            \item \textbf{AND ($\land$)}: P $\land$ Q is true if both P and Q are true.
            \item \textbf{OR ($\lor$)}: P $\lor$ Q is true if at least one of P or Q is true.
            \item \textbf{NOT ($\neg$)}: $\neg P$ is true if P is false.
            \item \textbf{IMPLIES ($\rightarrow$)}: P $\rightarrow$ Q is true if whenever P is true, Q is also true.
          \end{itemize}
      \end{itemize}
    
    \item \textbf{Example:} Given P: "It is raining" and Q: "The ground is wet,"
    \[
      P \rightarrow Q
    \]
  \end{itemize}
\end{frame}

\begin{frame}[fragile]
  \frametitle{Logic Reasoning in AI - First-Order Logic}
  \begin{block}{Definition}
    First-Order Logic (FOL) extends propositional logic by allowing quantification over objects and includes predicates, enabling more complex representations.
  \end{block}
  
  \begin{itemize}
    \item \textbf{Key Components:}
      \begin{itemize}
        \item \textbf{Predicates:} Represents properties of objects (e.g., Loves(John, Mary) means "John loves Mary").
        \item \textbf{Quantifiers:} 
          \begin{itemize}
            \item \textbf{Universal ($\forall$)}: States a property holds for all members of a domain (e.g., $\forall x$ Loves(x, Mary)).
            \item \textbf{Existential ($\exists$)}: States there exists at least one member that satisfies a property (e.g., $\exists x$ Loves(x, Mary)).
          \end{itemize}
      \end{itemize}
    
    \item \textbf{Example:} The statement "All humans are mortal" can be represented as:
    \[
      \forall x (Human(x) \rightarrow Mortal(x))
    \]
  \end{itemize}
\end{frame}

\begin{frame}[fragile]
  \frametitle{Logic Reasoning in AI - Applications}
  \begin{block}{Applications of Logic in AI}
    \begin{itemize}
      \item \textbf{Knowledge Representation:} Logic encodes knowledge for inference.
      \item \textbf{Automated Theorem Proving:} Logic used to prove theorems based on axioms.
      \item \textbf{Natural Language Processing:} Aids in understanding and generating languages through logical structuring.
      \item \textbf{Expert Systems:} Uses logical inference to provide solutions in specific domains (e.g., medical diagnosis).
    \end{itemize}
  \end{block}
  
  \begin{block}{Key Points to Emphasize}
    \begin{itemize}
      \item Propositional logic is fundamental but limited.
      \item First-Order Logic allows for expressive knowledge representation.
      \item Logic reasoning underpins many AI applications.
    \end{itemize}
  \end{block}
\end{frame}

\begin{frame}[fragile]
    \frametitle{Probabilistic Models - Overview}
    \begin{itemize}
        \item \textbf{Definition}:
        Probabilistic reasoning involves making inferences and decisions under uncertainty, accounting for variability and uncertainty in data.
        
        \item \textbf{Importance in AI}:
        \begin{itemize}
            \item AI often deals with incomplete or noisy information.
            \item Probabilistic models enhance decision-making in uncertain environments.
        \end{itemize}
    \end{itemize}
\end{frame}

\begin{frame}[fragile]
    \frametitle{Probabilistic Models - Bayesian Networks}
    \begin{itemize}
        \item \textbf{Definition}:
        A Bayesian Network is a graphical model that represents variables and their conditional dependencies using directed acyclic graphs (DAGs).
        
        \item \textbf{Components}:
        \begin{itemize}
            \item \textbf{Nodes}: Represent random variables (e.g., Weather, Traffic, Accident).
            \item \textbf{Edges}: Indicate dependencies between variables.
        \end{itemize}
        
        \item \textbf{Example}:
        \begin{itemize}
            \item Variables: Rain (True/False), Traffic (Heavy/Light), Accident (Yes/No).
        \end{itemize}
        \pause
        \textit{(Refer to visual representation showing nodes and edges.)}
    \end{itemize}
\end{frame}

\begin{frame}[fragile]
    \frametitle{Probabilistic Models - Key Points and Formulas}
    \begin{itemize}
        \item \textbf{Key Points}:
        \begin{enumerate}
            \item Uncertainty Handling - Crucial for real-world applications.
            \item Flexibility - Adaptable to diverse domains (healthcare, finance).
            \item Reasoning and Learning - Can learn probabilistic relationships from data.
        \end{enumerate}
        
        \item \textbf{Key Formula - Bayes' Theorem}:
        \begin{equation}
          P(A | B) = \frac{P(B | A) \cdot P(A)}{P(B)}
        \end{equation}
        Where:
        \begin{itemize}
            \item \(P(A | B)\): Posterior probability of \(A\) given \(B\).
            \item \(P(B | A)\): Likelihood of \(B\) given \(A\).
            \item \(P(A)\): Prior probability of \(A\).
            \item \(P(B)\): Prior probability of \(B\).
        \end{itemize}
        
        \item \textbf{Conclusion}:
        Probabilistic models and Bayesian networks are foundational in AI, enabling reasoning under uncertainty and learning from complex datasets.
    \end{itemize}
\end{frame}

\begin{frame}[fragile]
    \frametitle{Algorithms for AI Problem Solving}
    \begin{block}{Overview of Algorithms in AI}
        This presentation discusses various algorithms crucial for AI, including \textbf{search algorithms} and \textbf{planning algorithms}.
    \end{block}
\end{frame}

\begin{frame}[fragile]
    \frametitle{Search Algorithms}
    \begin{block}{Definition}
        Search algorithms are fundamental to AI as they navigate vast search spaces to find solutions.
    \end{block}

    \begin{itemize}
        \item \textbf{Types of Search Algorithms:}
        \begin{itemize}
            \item \textbf{Uninformed Search}
            \begin{itemize}
                \item Searches without additional information.
                \item \textbf{Examples:}
                \begin{itemize}
                    \item \textbf{Breadth-First Search (BFS)}: Explores neighbor nodes at the same depth.
                    \item \textbf{Depth-First Search (DFS)}: Explores as far as possible along a branch.
                \end{itemize}
            \end{itemize}

            \item \textbf{Informed Search}
            \begin{itemize}
                \item Uses domain knowledge for efficient solutions.
                \item \textbf{Examples:}
                \begin{itemize}
                    \item \textbf{A* Search}: Combines BFS and heuristics.
                    \item \textbf{Greedy Best-First Search}: Expands nodes closest to the goal based on heuristic.
                \end{itemize}
            \end{itemize}
        \end{itemize}
    \end{itemize}
\end{frame}

\begin{frame}[fragile]
    \frametitle{Planning Algorithms}
    \begin{block}{Definition}
        Planning algorithms enable AI systems to create sequences of actions to achieve specific goals.
    \end{block}

    \begin{itemize}
        \item \textbf{Types of Planning Algorithms:}
        \begin{itemize}
            \item \textbf{Classical Planning}
            \begin{itemize}
                \item Uses a model of the environment to generate plans.
                \item \textbf{Example:} STRIPS (Stanford Research Institute Problem Solver).
            \end{itemize}
            \item \textbf{Probabilistic Planning}
            \begin{itemize}
                \item Handles uncertainty in the environment.
                \item \textbf{Example:} Markov Decision Processes.
            \end{itemize}
        \end{itemize}
    \end{itemize}
\end{frame}

\begin{frame}[fragile]
    \frametitle{Key Points and Conclusion}
    \begin{block}{Key Points}
        \begin{itemize}
            \item Algorithms are essential for AI problem-solving.
            \item Both search and planning algorithms are critical for designing effective AI systems.
        \end{itemize}
    \end{block}

    \begin{block}{Conclusion}
        Effective AI problem-solving relies on various algorithms tailored for specific tasks, empowering robust system design.
    \end{block}
\end{frame}

\begin{frame}[fragile]
  \frametitle{Markov Decision Processes}
  \begin{block}{Overview}
    Markov Decision Processes (MDPs) are frameworks used to model decision-making in scenarios where outcomes are partly random and partly under the control of a decision-maker.
  \end{block}
\end{frame}

\begin{frame}[fragile]
  \frametitle{Components of MDPs}
  \begin{enumerate}
    \item \textbf{States (S)}: All possible configurations of the environment.
    \item \textbf{Actions (A)}: The choices available to the decision-maker.
    \item \textbf{Transition Model (P)}: The probability of moving from one state to another given an action (P(s'|s, a)).
    \item \textbf{Reward Function (R)}: Assigns a numerical reward for each action taken in a state.
    \item \textbf{Discount Factor ($\gamma$)}: A value between 0 and 1 determining the importance of future rewards.
  \end{enumerate}
\end{frame}

\begin{frame}[fragile]
  \frametitle{Example of an MDP}
  \begin{itemize}
    \item \textbf{States (S)}: {(1,1), (1,2), (1,3), (2,1), (2,2), (2,3), (3,1), (3,2), (3,3)}
    \item \textbf{Actions (A)}: {UP, DOWN, LEFT, RIGHT}
    \item \textbf{Transition Model (P)}: e.g., From state (1,1) with action RIGHT to (1,2) with probability 1.
    \item \textbf{Reward Function (R)}: Moving to (2,3) provides +10; to (1,3) provides +1.
    \item \textbf{Discount Factor ($\gamma$)}: 0.9, indicating future rewards are fairly important.
  \end{itemize}
\end{frame}

\begin{frame}[fragile]
  \frametitle{Applications of MDPs}
  \begin{itemize}
    \item \textbf{Robotics}: Path planning and navigation.
    \item \textbf{Finance}: Optimizing investment strategies.
    \item \textbf{Game AI}: Guiding decision-making in games.
  \end{itemize}
\end{frame}

\begin{frame}[fragile]
  \frametitle{Key Points}
  \begin{itemize}
    \item MDPs model decision-making under uncertainty.
    \item They allow for the calculation of optimal policies for cumulative reward maximization.
    \item Fundamental in reinforcement learning, where agents learn optimal behaviors through experience.
  \end{itemize}
\end{frame}

\begin{frame}[fragile]
  \frametitle{Mathematical Framework}
  \begin{block}{Bellman Equation}
    The value $V(s)$ of a state is computed using:
    \begin{equation}
      V(s) = R(s) + \gamma \sum_{s'} P(s'|s, a)V(s')
    \end{equation}
    This states that the value of a state is the immediate reward plus the discounted expected value of the next states.
  \end{block}
\end{frame}

\begin{frame}[fragile]
  \frametitle{Conclusion}
  \begin{block}{Summary}
    Markov Decision Processes are vital for systematic decision-making in unpredictable environments. Understanding MDPs lays the groundwork for more complex topics in reinforcement learning.
  \end{block}
\end{frame}

\begin{frame}[fragile]
  \frametitle{Reinforcement Learning Techniques - Introduction}
  \begin{block}{What is Reinforcement Learning (RL)?}
    Reinforcement Learning (RL) is a type of machine learning where an agent learns to make decisions by taking actions in an environment to maximize cumulative rewards. Unlike supervised learning, RL relies on exploring the environment and learning from the consequences of its actions.
  \end{block}
\end{frame}

\begin{frame}[fragile]
  \frametitle{Reinforcement Learning Techniques - Key Characteristics}
  \begin{itemize}
    \item \textbf{Agent}: The learner or decision-maker (e.g., a robot).
    \item \textbf{Environment}: Everything the agent interacts with (e.g., the world).
    \item \textbf{Actions}: Choices the agent can make (e.g., move left, right, up, or down).
    \item \textbf{Rewards}: Feedback received after an action, indicating goodness or badness (e.g., +10 for a goal, -1 for hitting a wall).
    \item \textbf{State}: The current situation of the agent in the environment (e.g., current location in a maze).
  \end{itemize}
\end{frame}

\begin{frame}[fragile]
  \frametitle{Reinforcement Learning Techniques - Difference from Other Learning Techniques}
  \begin{block}{Learning Techniques Comparison}
    \begin{itemize}
      \item \textbf{Supervised Learning}: Learns from labeled data (input-output pairs).
        \begin{itemize}
          \item \textit{Example}: Predicting house prices based on features.
        \end{itemize}
      \item \textbf{Unsupervised Learning}: Learns from unlabeled data by finding patterns.
        \begin{itemize}
          \item \textit{Example}: Clustering customers based on purchasing behavior.
        \end{itemize}
      \item \textbf{Reinforcement Learning}: Learns through trial-and-error, with a focus on long-term rewards.
    \end{itemize}
  \end{block}
\end{frame}

\begin{frame}[fragile]
  \frametitle{Reinforcement Learning Techniques - Example}
  \begin{block}{Example of Reinforcement Learning}
    Consider a robot navigating through a maze:
    \begin{itemize}
      \item \textbf{States}: Different positions in the maze.
      \item \textbf{Actions}: Move left, right, or forward.
      \item \textbf{Rewards}: 
      \begin{itemize}
        \item +10 for reaching the exit
        \item -1 for hitting a wall
      \end{itemize}
    \end{itemize}
    The robot starts with no knowledge of the maze and learns the optimal path through exploration.
  \end{block}
\end{frame}

\begin{frame}[fragile]
  \frametitle{Reinforcement Learning Techniques - Key Algorithms}
  \begin{itemize}
    \item \textbf{Q-Learning}: 
      An off-policy learning algorithm that uses a value function to determine the best action in a given state.
      \begin{equation}
        Q(s, a) \leftarrow Q(s, a) + \alpha \left( r + \gamma \max_{a'} Q(s', a') - Q(s, a) \right)
      \end{equation}
      Where:
      \begin{itemize}
        \item \( Q(s, a) \): Estimated value of action \( a \) in state \( s \)
        \item \( \alpha \): Learning rate
        \item \( r \): Reward received
        \item \( \gamma \): Discount factor
        \item \( s' \): New state after taking action \( a \)
      \end{itemize}
    \item \textbf{Deep Reinforcement Learning (DRL)}: Combines RL with deep learning for complex tasks.
  \end{itemize}
\end{frame}

\begin{frame}[fragile]
  \frametitle{Reinforcement Learning Techniques - Key Takeaways}
  \begin{block}{Key Takeaways}
    \begin{itemize}
      \item RL focuses on learning from interactions with the environment and optimizing long-term rewards.
      \item It differs from supervised and unsupervised learning by emphasizing a sequence of actions and their consequences.
      \item RL is widely applicable, with advancements in both theory and practical implementations.
    \end{itemize}
  \end{block}
\end{frame}

\begin{frame}[fragile]
    \frametitle{Critical Analysis of AI Models}
    \begin{block}{Objective}
        This slide provides a framework for assessing AI models, focusing on three essential aspects: correctness, performance, and applicability. Understanding these dimensions will enable better decision-making in selecting and deploying AI solutions.
    \end{block}
\end{frame}

\begin{frame}[fragile]
    \frametitle{Assessing Correctness}
    \begin{itemize}
        \item \textbf{Definition:} Correctness refers to how accurately an AI model performs its intended task, aligning its outputs with expected results.
        \item \textbf{Methods:}
        \begin{itemize}
            \item \textbf{Validation Datasets:} Use datasets that the model hasn’t seen to evaluate its predictions.
            \item \textbf{Metrics:} Common metrics include Accuracy, Precision, Recall, and F1 Score.
        \end{itemize}
        \item \textbf{Example:} For a model predicting whether emails are spam:
        \begin{itemize}
            \item \textbf{Precision} = True Positives / (True Positives + False Positives)  
            \item \textbf{Recall} = True Positives / (True Positives + False Negatives)
        \end{itemize}
    \end{itemize}
\end{frame}

\begin{frame}[fragile]
    \frametitle{Evaluating Performance}
    \begin{itemize}
        \item \textbf{Definition:} Performance examines how well the model operates under various conditions, including speed, scalability, and resource efficiency.
        \item \textbf{Methods:}
        \begin{itemize}
            \item \textbf{Benchmarking:} Compare the model against established standards or previous models.
            \item \textbf{Profiling:} Analyze model run-time, memory usage, and response time.
        \end{itemize}
        \item \textbf{Illustration:} A bar chart can visualize performance differences in processing time (in seconds) of different models while solving the same problem.
    \end{itemize}
\end{frame}

\begin{frame}[fragile]
    \frametitle{Assessing Applicability}
    \begin{itemize}
        \item \textbf{Definition:} Applicability determines whether an AI model is suitable for a specific domain or problem, considering factors like data availability and business needs.
        \item \textbf{Methods:}
        \begin{itemize}
            \item \textbf{Contextual Analysis:} Examine the domain-specific challenges that might affect model success.
            \item \textbf{Stakeholder Input:} Gather feedback from users and stakeholders to identify real-world requirements.
        \end{itemize}
        \item \textbf{Example:} A facial recognition AI might perform well theoretically, but its applicability is limited if ethical implications or privacy concerns aren't addressed.
    \end{itemize}
\end{frame}

\begin{frame}[fragile]
    \frametitle{Key Points and Conclusion}
    \begin{itemize}
        \item \textbf{Holistic Assessment:} Always consider Correctness, Performance, and Applicability together to make informed decisions.
        \item \textbf{Iterative Process:} Model evaluation should be an ongoing aspect of the AI lifecycle, continuously improving as more data becomes available or as requirements evolve.
        \item \textbf{Conclusion:} Evaluating AI models through correctness, performance, and applicability helps ensure that the chosen solutions fulfill their intended objectives and provide tangible value to users.
    \end{itemize}
\end{frame}

\begin{frame}[fragile]
    \frametitle{Collaborative AI Projects}
    \begin{block}{Importance of Collaboration in AI Projects}
        Collaboration is essential for leveraging diverse expertise, enhancing problem-solving, and effectively utilizing resources.
    \end{block}
\end{frame}

\begin{frame}[fragile]
    \frametitle{Importance of Collaboration}
    \begin{enumerate}
        \item \textbf{Diverse Expertise} 
        \begin{itemize}
            \item AI projects require knowledge from data science, software engineering, and domain-specific expertise.
            \item \textit{Example:} A medical AI project requires input from clinicians, data analysts, and ethicists.
        \end{itemize}

        \item \textbf{Enhanced Problem-Solving}
        \begin{itemize}
            \item Collaboration fosters innovative solutions through pooling ideas.
            \item \textit{Illustration:} A brainstorming session in a collaborative project can lead to novel algorithms.
        \end{itemize}

        \item \textbf{Effective Resource Utilization}
        \begin{itemize}
            \item Optimizes funding, time, and human capital for success.
        \end{itemize}
    \end{enumerate}
\end{frame}

\begin{frame}[fragile]
    \frametitle{Communication of Findings}
    \begin{block}{Importance of Clear Communication}
        Articulating findings in an understandable way is crucial for stakeholders, ensuring actionable strategies.
    \end{block}
    
    \begin{itemize}
        \item \textbf{Tools and Techniques:}
        \begin{itemize}
            \item \textbf{Data Visualization:} Graphs, charts, and dashboards present complex data understandably.
            \item \textbf{Documentation:} Thorough methodologies and results foster transparency.
        \end{itemize}

        \item \textbf{Example of Effective Communication:}
        \begin{itemize}
            \item Successful project reports may include visualizations summarizing accuracy with an executive summary for non-technical stakeholders.
        \end{itemize}
    \end{itemize}
\end{frame}

\begin{frame}[fragile]
    \frametitle{Key Points to Emphasize}
    \begin{itemize}
        \item \textbf{Collaborative Culture:} Encourage open communication and idea sharing.
        \item \textbf{Regular Check-ins:} Frequent meetings to discuss progress and challenges.
        \item \textbf{Feedback Loop:} Mechanisms for continuous improvement during the project.
    \end{itemize}
\end{frame}

\begin{frame}[fragile]
    \frametitle{Conclusion}
    Collaboration in AI projects is essential for success. Combining diverse skills with clear communication results in more effective models and implementations, ultimately leading to greater societal impacts.
\end{frame}

\begin{frame}[fragile]
    \frametitle{Ethical Considerations in AI}
    \begin{block}{Introduction}
        As Artificial Intelligence (AI) increasingly integrates into society, it is crucial to address the ethical implications and societal impacts these technologies generate. Key ethical considerations include fairness, accountability, and transparency.
    \end{block}
\end{frame}

\begin{frame}[fragile]
    \frametitle{Key Ethical Issues in AI - Part 1}
    \begin{enumerate}
        \item \textbf{Bias and Fairness}
        \begin{itemize}
            \item \textbf{Definition:} AI systems can inadvertently encode bias from training data, leading to unfair outcomes.
            \item \textbf{Example:} Hiring algorithms favoring candidates from specific backgrounds.
            \item \textbf{Key Point:} Ensure diverse training datasets and continuous evaluation for bias.
        \end{itemize}
        
        \item \textbf{Privacy and Surveillance}
        \begin{itemize}
            \item \textbf{Definition:} AI may infringe on privacy by collecting data without consent.
            \item \textbf{Example:} Facial recognition tracking individuals in public spaces.
            \item \textbf{Key Point:} Implement strict data protection regulations and obtain informed consent.
        \end{itemize}
    \end{enumerate}
\end{frame}

\begin{frame}[fragile]
    \frametitle{Key Ethical Issues in AI - Part 2}
    \begin{enumerate}
        \setcounter{enumi}{2} % Continue numbering from previous frame
        \item \textbf{Accountability}
        \begin{itemize}
            \item \textbf{Definition:} Unclear responsibility for AI's autonomous decisions.
            \item \textbf{Example:} Liability in accidents involving autonomous vehicles.
            \item \textbf{Key Point:} Establish clear guidelines for accountability and liability.
        \end{itemize}

        \item \textbf{Transparency}
        \begin{itemize}
            \item \textbf{Definition:} "Black box" nature of AI algorithms complicates understanding decision-making.
            \item \textbf{Example:} Medical AI systems failing to explain reasoning can erode trust.
            \item \textbf{Key Point:} Techniques like explainable AI (XAI) aim to enhance understanding.
        \end{itemize}

        \item \textbf{Job Displacement}
        \begin{itemize}
            \item \textbf{Definition:} AI-driven automation can disrupt the job market.
            \item \textbf{Example:} Reduced human jobs in manufacturing and customer service due to AI.
            \item \textbf{Key Point:} Strategies for retraining and reskilling workers are essential.
        \end{itemize}
    \end{enumerate}
\end{frame}

\begin{frame}[fragile]
    \frametitle{Conclusion and Discussion}
    \begin{block}{Conclusion}
        AI technology offers substantial benefits but also poses significant ethical challenges. Addressing these challenges is necessary for responsible development and deployment of AI, paving the way for equitable solutions.
    \end{block}
    
    \begin{block}{Call to Discussion}
        How can we actively ensure that ethical frameworks are integrated into AI development processes across different industries?
    \end{block}
\end{frame}

\begin{frame}[fragile]
  \frametitle{AI Applications in Real-World Scenarios - Overview}
  \begin{block}{Introduction}
    Artificial Intelligence (AI) encompasses technologies that perform tasks typically requiring human intelligence. 
    These technologies transform industries by enhancing efficiency, providing insights, and automating processes.
  \end{block}
\end{frame}

\begin{frame}[fragile]
  \frametitle{AI Applications in Healthcare}
  \begin{enumerate}
    \item \textbf{Healthcare}
      \begin{itemize}
        \item \textbf{Disease Diagnosis:} AI analyzes medical data for early detection of diseases, e.g., Google's DeepMind diagnosing eye diseases.
        \item \textbf{Personalized Treatment Plans:} Machine learning predicts patient responses to treatments for tailored healthcare solutions.
      \end{itemize}
  \end{enumerate}
\end{frame}

\begin{frame}[fragile]
  \frametitle{AI Applications in Finance and Retail}
  \begin{enumerate}
    \item \textbf{Finance}
      \begin{itemize}
        \item \textbf{Fraud Detection:} AI monitors transactions in real-time to identify fraudulent activities, e.g., PayPal flagging suspicious transactions.
        \item \textbf{Algorithmic Trading:} AI analyzes market trends for optimal trading, increasing profit margins.
      \end{itemize}
    
    \item \textbf{Retail}
      \begin{itemize}
        \item \textbf{Customer Recommendations:} E-commerce platforms like Amazon utilize AI for personalized product recommendations.
        \item \textbf{Inventory Management:} Predictive analytics optimizes stock levels based on demand forecasts.
      \end{itemize}
  \end{enumerate}
\end{frame}

\begin{frame}[fragile]
  \frametitle{AI Applications in Autonomous Vehicles and NLP}
  \begin{enumerate}
    \item \textbf{Autonomous Vehicles}
      \begin{itemize}
        \item \textbf{Self-Driving Technology:} Companies like Tesla use AI for navigation in autonomous vehicles.
        \item \textbf{Traffic Management:} AI analyzes traffic to improve light timings and reduce congestion.
      \end{itemize}

    \item \textbf{Natural Language Processing (NLP)}
      \begin{itemize}
        \item \textbf{Chatbots and Virtual Assistants:} AI chatbots use NLP for customer support.
        \item \textbf{Sentiment Analysis:} Businesses analyze social media posts to gauge public sentiment.
      \end{itemize}
  \end{enumerate}
\end{frame}

\begin{frame}[fragile]
  \frametitle{Key Points and Conclusion}
  \begin{block}{Key Points to Emphasize}
    \begin{itemize}
      \item \textbf{Impact on Efficiency:} AI technologies streamline processes and increase productivity.
      \item \textbf{Data-Driven Decisions:} AI enables businesses to make informed decisions through data analysis.
      \item \textbf{Continuous Learning:} Many AI systems improve over time via learning from new data.
      \item \textbf{Ethical Considerations:} Need to address data privacy and algorithmic bias.
    \end{itemize}
  \end{block}

  \begin{block}{Conclusion}
    The integration of AI represents a significant shift in problem-solving and task execution, highlighting its relevance and setting the stage for future innovations and ethical considerations.
  \end{block}
\end{frame}

\begin{frame}[fragile]
    \frametitle{Review of Key Takeaways - Part 1}
    \begin{block}{Key Takeaways from Chapter 2: AI Foundations and Terminologies}
        \begin{enumerate}
            \item \textbf{Understanding Artificial Intelligence (AI)}
            \begin{itemize}
                \item Definition: AI refers to the simulation of human intelligence processes by machines.
                \item Key Types:
                    \begin{itemize}
                        \item Narrow AI: Specialized in one task (e.g., recommendation systems).
                        \item General AI: Hypothetical AI that can perform any intellectual task a human can.
                    \end{itemize}
            \end{itemize}
            \item \textbf{Core Concepts in AI}
            \begin{itemize}
                \item Machine Learning (ML): A subset of AI enabling systems to learn from data.
                \item Deep Learning (DL): A type of ML utilizing neural networks with many layers.
            \end{itemize}
        \end{enumerate}
    \end{block}
\end{frame}

\begin{frame}[fragile]
    \frametitle{Review of Key Takeaways - Part 2}
    \begin{block}{Key Terminology}
        \begin{itemize}
            \item \textbf{Algorithm}: A set of rules or instructions for AI systems to learn.
            \item \textbf{Dataset}: A collection of data for training and testing AI models.
            \item \textbf{Training \& Testing}: Teaching AI using training data and evaluating on testing data.
        \end{itemize}
    \end{block}

    \begin{block}{Applications of AI}
        \begin{itemize}
            \item AI is utilized in various industries:
                \begin{itemize}
                    \item Healthcare: Diagnostic systems.
                    \item Finance: Fraud detection.
                \end{itemize}
            \item Understanding real-world relevance enhances comprehension of theoretical concepts.
        \end{itemize}
    \end{block}
\end{frame}

\begin{frame}[fragile]
    \frametitle{Review of Key Takeaways - Part 3}
    \begin{block}{Challenges and Ethical Considerations}
        \begin{itemize}
            \item \textbf{Bias in AI}: AI models can inherit biases from training data.
            \item \textbf{Transparency}: Understanding how AI decisions are made (the "black box" issue).
        \end{itemize}
    \end{block}

    \begin{block}{Summary}
        This chapter lays the groundwork for understanding fundamental AI concepts and terminologies. Engaging with these ideas prepares you for analyzing AI's impact.
    \end{block}

    \begin{block}{Next Steps}
        Prepare for the upcoming section focusing on the influence of foundational concepts on AI applications and ethical discussions.
    \end{block}

    \begin{block}{Emphasized Points to Remember}
        \begin{itemize}
            \item AI is about the processes that enable machine intelligence.
            \item Ethical engagement is essential for responsible AI development.
        \end{itemize}
    \end{block}
\end{frame}

\begin{frame}[fragile]
    \frametitle{Preparatory Steps for Upcoming Topics}
    \begin{block}{Understanding the Fundamentals}
        Solidifying your understanding of the foundational concepts of AI is essential. This preparation will enhance comprehension and facilitate engagement in discussions and applications.
    \end{block}
\end{frame}

\begin{frame}[fragile]
    \frametitle{Key Preparatory Steps - Part 1}
    \begin{enumerate}
        \item \textbf{Review Terminologies}:
        \begin{itemize}
            \item \textbf{Machine Learning (ML)}: Algorithms that improve through experience.
            \item \textbf{Deep Learning (DL)}: A type of ML using neural networks with many layers.
            \item \textbf{Natural Language Processing (NLP)}: Technology enabling machines to understand and respond to human language.
        \end{itemize}
        \textit{Example:} Compare traditional programming (rules-based) with ML where models learn from data.
        
        \item \textbf{Understand AI Categories}:
        \begin{itemize}
            \item \textbf{Narrow AI vs. General AI}: Narrow AI performs specific tasks, while General AI aims to learn any intellectual task.
            \item \textbf{Supervised vs. Unsupervised Learning}: Understand labeled datasets vs. unlabeled data models.
        \end{itemize}
        \textit{Illustration:} Use a chart to differentiate Narrow AI (e.g., virtual assistants) from General AI.
    \end{enumerate}
\end{frame}

\begin{frame}[fragile]
    \frametitle{Key Preparatory Steps - Part 2}
    \begin{enumerate}
        \setcounter{enumi}{2} % Continue the enumeration
        \item \textbf{Familiarize with AI Applications}:
        \begin{itemize}
            \item Research examples in healthcare, finance, and automation (e.g., diagnostic tools, fraud detection, self-driving cars).
        \end{itemize}
        \textit{Key Point:} Case studies provide context and relevance to theoretical concepts.

        \item \textbf{Prepare Questions}:
        \begin{itemize}
            \item Keep a list of questions that arise during your study to enhance engagement in discussions.
        \end{itemize}
    \end{enumerate}
\end{frame}

\begin{frame}[fragile]
    \frametitle{Engage with the Content}
    \begin{itemize}
        \item \textbf{Utilize Online Resources}: Consider platforms like Coursera or edX for introductory AI courses and materials.
        \item \textbf{Join Study Groups}: Discuss concepts with peers to clarify doubts and solidify understanding.
    \end{itemize}
\end{frame}

\begin{frame}[fragile]
    \frametitle{Additional Recommendations}
    \begin{itemize}
        \item \textbf{Explore Programming Basics}: Learn Python if you are not familiar with coding, as it is prevalent in AI development.
        \item \textbf{Stay Updated}: Follow AI trends in reputable tech journals and news outlets.
    \end{itemize}
\end{frame}

\begin{frame}[fragile]
    \frametitle{Conclusion}
    By focusing on these preparatory steps, you will be well-equipped to delve deeper into AI in upcoming lectures, fostering a richer learning experience.
\end{frame}

\begin{frame}[fragile]
    \frametitle{Next Slide Preview}
    Join us in the next slide for an open forum on questions and discussion points related to AI foundations. Your insights will deepen our collective understanding.
\end{frame}

\begin{frame}[fragile]
    \frametitle{Questions and Discussion - Part 1}
    \begin{block}{Open Floor for Questions and Discussion}
        Engage in a discussion about AI foundations and terminologies. Feel free to ask questions or share insights.
    \end{block}
    
    \begin{itemize}
        \item Reflect on understanding of AI concepts
        \item Clarify key terminologies and principles
        \item Foster a collaborative learning environment
    \end{itemize}
\end{frame}

\begin{frame}[fragile]
    \frametitle{Suggested Discussion Questions - Part 2}
    \begin{enumerate}
        \item What are the foundational principles of AI?
            \begin{itemize}
                \item Discuss machine learning, neural networks, and natural language processing.
            \end{itemize}
        \item How do different types of AI impact our daily lives?
            \begin{itemize}
                \item Explore examples like voice assistants (narrow AI) vs. theoretical general AI.
            \end{itemize}
        \item What ethical implications do you foresee with the advancement of AI technology?
            \begin{itemize}
                \item Discuss privacy concerns, job displacement, and bias in AI algorithms.
            \end{itemize}
        \item Which AI applications have you encountered in your personal or professional life?
            \begin{itemize}
                \item Share real-world applications such as recommendation systems or image recognition.
            \end{itemize}
        \item Can anyone explain the difference between supervised and unsupervised learning?
            \begin{itemize}
                \item Use examples: spam detection (supervised) vs. customer segmentation (unsupervised).
            \end{itemize}
    \end{enumerate}
\end{frame}

\begin{frame}[fragile]
    \frametitle{Key Terms and Examples - Part 3}
    \begin{block}{Key Terms to Clarify}
        \begin{itemize}
            \item \textbf{Artificial Intelligence (AI):} Simulation of human intelligence by machines.
            \item \textbf{Machine Learning (ML):} A subset of AI that learns from data.
            \item \textbf{Deep Learning:} Specialized ML area using multi-layer neural networks.
        \end{itemize}
    \end{block}
    
    \begin{block}{Illustrative Examples}
        \begin{itemize}
            \item \textbf{Narrow AI:} Voice Assistants (e.g., Siri, Alexa) - Task-specific without general reasoning.
            \item \textbf{General AI (hypothetical):} An AI that can perform any human task, deep language understanding and contextual reasoning.
        \end{itemize}
    \end{block}
    
    \begin{block}{Key Points to Emphasize}
        - AI consists of various sub-domains with unique methodologies.
        - Ethical implications of AI are significant; careful consideration is necessary.
        - Understanding AI concepts is essential for informed discussions about technology's future.
    \end{block}
\end{frame}


\end{document}