\documentclass[aspectratio=169]{beamer}

% Theme and Color Setup
\usetheme{Madrid}
\usecolortheme{whale}
\useinnertheme{rectangles}
\useoutertheme{miniframes}

% Additional Packages
\usepackage[utf8]{inputenc}
\usepackage[T1]{fontenc}
\usepackage{graphicx}
\usepackage{booktabs}
\usepackage{listings}
\usepackage{amsmath}
\usepackage{amssymb}
\usepackage{xcolor}
\usepackage{tikz}
\usepackage{pgfplots}
\pgfplotsset{compat=1.18}
\usetikzlibrary{positioning}
\usepackage{hyperref}

% Custom Colors
\definecolor{myblue}{RGB}{31, 73, 125}
\definecolor{mygray}{RGB}{100, 100, 100}
\definecolor{mygreen}{RGB}{0, 128, 0}
\definecolor{myorange}{RGB}{230, 126, 34}
\definecolor{mycodebackground}{RGB}{245, 245, 245}

% Set Theme Colors
\setbeamercolor{structure}{fg=myblue}
\setbeamercolor{frametitle}{fg=white, bg=myblue}
\setbeamercolor{title}{fg=myblue}
\setbeamercolor{section in toc}{fg=myblue}
\setbeamercolor{item projected}{fg=white, bg=myblue}
\setbeamercolor{block title}{bg=myblue!20, fg=myblue}
\setbeamercolor{block body}{bg=myblue!10}
\setbeamercolor{alerted text}{fg=myorange}

% Set Fonts
\setbeamerfont{title}{size=\Large, series=\bfseries}
\setbeamerfont{frametitle}{size=\large, series=\bfseries}
\setbeamerfont{caption}{size=\small}
\setbeamerfont{footnote}{size=\tiny}

% Footer and Navigation Setup
\setbeamertemplate{footline}{
  \leavevmode%
  \hbox{%
  \begin{beamercolorbox}[wd=.3\paperwidth,ht=2.25ex,dp=1ex,center]{author in head/foot}%
    \usebeamerfont{author in head/foot}\insertshortauthor
  \end{beamercolorbox}%
  \begin{beamercolorbox}[wd=.5\paperwidth,ht=2.25ex,dp=1ex,center]{title in head/foot}%
    \usebeamerfont{title in head/foot}\insertshorttitle
  \end{beamercolorbox}%
  \begin{beamercolorbox}[wd=.2\paperwidth,ht=2.25ex,dp=1ex,center]{date in head/foot}%
    \usebeamerfont{date in head/foot}
    \insertframenumber{} / \inserttotalframenumber
  \end{beamercolorbox}}%
  \vskip0pt%
}

% Turn off navigation symbols
\setbeamertemplate{navigation symbols}{}

% Title Page Information
\title[Introduction to Decision Making]{Chapter 10: Introduction to Decision Making}
\author[J. Smith]{John Smith, Ph.D.}
\institute[University Name]{
  Department of Computer Science\\
  University Name\\
  \vspace{0.3cm}
  Email: email@university.edu\\
  Website: www.university.edu
}
\date{\today}

% Document Start
\begin{document}

\frame{\titlepage}

\begin{frame}[fragile]
    \frametitle{Introduction to Decision Making}
    \begin{block}{Overview of Decision Making}
        Decision making is a fundamental process involving the selection of the best course of action from a set of alternatives. In AI, it refers to machines making decisions requiring human judgment.
    \end{block}
\end{frame}

\begin{frame}[fragile]
    \frametitle{Significance in AI}
    \begin{itemize}
        \item \textbf{Automation of Complex Processes}: AI enhances efficiency by automating decision-making across fields such as healthcare, finance, and commerce.
        \item \textbf{Data-Driven Insights}: AI analyzes vast amounts of data quickly for more informed and timely decisions.
        \item \textbf{Improving Outcomes}: AI models can minimize human error and optimize decision quality.
    \end{itemize}
\end{frame}

\begin{frame}[fragile]
    \frametitle{Key Concepts and Examples}
    \begin{block}{Types of Decision Making}
        \begin{itemize}
            \item \textbf{Routine Decisions}: Automated decisions based on defined criteria (e.g., credit scoring).
            \item \textbf{Strategic Decisions}: Complex decisions involving predictions and assessments (e.g., market analysis).
        \end{itemize}
    \end{block}

    \begin{block}{Examples of Decision Making in AI}
        \begin{enumerate}
            \item \textbf{Medical Diagnosis}: AI systems analyze patient data and X-rays to assist in diagnosis.
            \item \textbf{Stock Trading}: AI algorithms evaluate market trends and execute trades to maximize profits while minimizing risks.
        \end{enumerate}
    \end{block}
\end{frame}

\begin{frame}[fragile]
    \frametitle{Key Points to Emphasize}
    \begin{itemize}
        \item The role of AI in enhancing decision making—improving efficacy and accuracy.
        \item Understanding decision-making principles is crucial for optimizing AI model development and implementation.
        \item Ethical implications of AI decision-making processes will grow increasingly important.
    \end{itemize}
\end{frame}

\begin{frame}[fragile]
    \frametitle{Conclusion}
    This chapter lays the groundwork for understanding decision-making processes within AI frameworks, exploring methodologies, applications, and challenges in developing intelligent systems.
\end{frame}

\begin{frame}[fragile]
    \titlepage
\end{frame}

\begin{frame}[fragile]
    \frametitle{Definition of Decision Making in AI}
    \begin{block}{Definition}
        Decision making in Artificial Intelligence (AI) refers to the processes by which AI systems analyze data, evaluate options, and select the best course of action to achieve specific goals.
    \end{block}
    \begin{block}{Key Aspects}
        Encompasses various methodologies and algorithms that enable machines to simulate human-like judgment in uncertain environments.
    \end{block}
\end{frame}

\begin{frame}[fragile]
    \frametitle{Relevance of Decision Making in AI}
    \begin{itemize}
        \item \textbf{Real-World Applications:}
            \begin{itemize}
                \item \textbf{Healthcare:} AI assists in diagnosing diseases and suggesting treatments.
                \item \textbf{Finance:} Algorithms assess risk and recommend investment strategies.
                \item \textbf{Autonomous Vehicles:} Self-driving cars make real-time navigation decisions.
            \end{itemize}
        \item \textbf{Enhancing Efficiency and Accuracy:}
            \begin{itemize}
                \item AI processes vast amounts of data quickly, leading to faster and more accurate decisions.
            \end{itemize}
    \end{itemize}
\end{frame}

\begin{frame}[fragile]
    \frametitle{Key Points to Emphasize}
    \begin{itemize}
        \item \textbf{Data-Driven:} AI improves as it ingests larger datasets.
        \item \textbf{Machine Learning Integration:} Algorithms learn from past decisions.
        \item \textbf{Uncertainty Management:} Techniques like probabilistic reasoning help handle incomplete information.
    \end{itemize}
\end{frame}

\begin{frame}[fragile]
    \frametitle{Illustrative Example of Decision Making}
    \begin{block}{Movie Recommendation System}
        \begin{enumerate}
            \item \textbf{Data Collection:} Gather user preferences, ratings, and viewing history.
            \item \textbf{Analysis:} Utilize collaborative filtering for user-movie similarity.
            \item \textbf{Decision:} Present top 5 movie recommendations tailored to the user's tastes.
        \end{enumerate}
    \end{block}
\end{frame}

\begin{frame}[fragile]
    \frametitle{Formulas and Algorithms}
    \begin{block}{Decision Trees}
        A flowchart structure where nodes represent decision points leading to branches (choices) that end in leaves (outcomes).
        \begin{equation}
            EV = \sum (P(outcome) \times Value(outcome))
        \end{equation}
        where \( P(outcome) \) is the probability of different outcomes.
    \end{block}
    \begin{block}{Reinforcement Learning}
        Uses reward systems to guide the decision-making process based on feedback from the environment.
    \end{block}
\end{frame}

\begin{frame}[fragile]
    \frametitle{Conclusion}
    The effectiveness of AI hinges on its ability to make informed decisions, enhancing processes in various industries and demonstrating its significance in today's technology landscape.
\end{frame}

\begin{frame}[fragile]
    \frametitle{Decision Theories Overview}
    \begin{block}{Understanding Decision Theories}
        Decision-making is an integral part of both human reasoning and artificial intelligence (AI). Various theories describe how decisions are made under differing conditions:
        \begin{itemize}
            \item \textbf{Normative}
            \item \textbf{Descriptive}
            \item \textbf{Prescriptive}
        \end{itemize}
    \end{block}
\end{frame}

\begin{frame}[fragile]
    \frametitle{Types of Decision Theories}
    
    \begin{enumerate}
        \item \textbf{Normative Decision Theory}
        \begin{itemize}
            \item \textbf{Definition}: Focuses on how decisions \textit{should} be made for favorable outcomes.
            \item \textbf{Example}: In finance, models suggest diversifying portfolios.
            \item \textbf{Key Aspect}: Assumes rationality in decision-makers.
        \end{itemize}
        
        \item \textbf{Descriptive Decision Theory}
        \begin{itemize}
            \item \textbf{Definition}: Analyzes how decisions are \textit{actually} made, accounting for biases.
            \item \textbf{Example}: The "Sunk Cost Fallacy" where people continue failing projects due to prior investments.
            \item \textbf{Key Aspect}: Incorporates psychological factors and real-world observations.
        \end{itemize}
        
        \item \textbf{Prescriptive Decision Theory}
        \begin{itemize}
            \item \textbf{Definition}: Recommends specific actions by combining insights from the other two theories.
            \item \textbf{Example}: In healthcare, suggests treatment options balancing risk and patient preferences.
            \item \textbf{Key Aspect}: Provides actionable recommendations to improve outcomes.
        \end{itemize}
    \end{enumerate}
\end{frame}

\begin{frame}[fragile]
    \frametitle{Key Takeaways and Conclusion}
    
    \begin{block}{Key Takeaways}
        \begin{itemize}
            \item \textbf{Normative Theory}: Ideal and rational approaches.
            \item \textbf{Descriptive Theory}: Real-world practices and biases.
            \item \textbf{Prescriptive Theory}: Recommendations based on analyses.
        \end{itemize}
    \end{block}
    
    \begin{block}{Conclusion}
        Understanding these decision theories provides a comprehensive approach to decision-making, emphasizing the importance of context and human behavior in both AI and everyday life. 
    \end{block}
    
    \begin{block}{Visual Aids (suggested)}
        \begin{itemize}
            \item Venn Diagram: Illustrate relationships among the three types of decision theories.
            \item Flowchart: Show a decision-making process incorporating all aspects.
        \end{itemize}
    \end{block}
    
\end{frame}

\begin{frame}[fragile]
    \frametitle{Normative Decision Theory}
    \begin{block}{Overview}
        Normative Decision Theory is a framework that assesses how decisions should be made logically and rationally. It provides guidelines to achieve the best outcomes based on established criteria and norms.
    \end{block}
\end{frame}

\begin{frame}[fragile]
    \frametitle{Key Concepts - Rational Decision-Making}
    \begin{itemize}
        \item \textbf{Rational Decision-Making}: Core of effective decision-making, relying on logical analysis and systematic processes.
        \item \textbf{Utility Maximization}: Choosing options that maximize expected utility.
    \end{itemize}
    \begin{equation}
        E(U) = \sum (P(x_i) \cdot U(x_i))
    \end{equation}
    Where:
    \begin{itemize}
        \item \( E(U) \) = Expected utility
        \item \( P(x_i) \) = Probability of outcome \( x_i \)
        \item \( U(x_i) \) = Utility of outcome \( x_i \)
    \end{itemize}
\end{frame}

\begin{frame}[fragile]
    \frametitle{Decision-Making Model & Example}
    \textbf{Decision-Making Process:}
    \begin{enumerate}
        \item Identify the Decision Problem
        \item Generate Alternatives
        \item Evaluate Alternatives
        \item Choose the Best Alternative
    \end{enumerate}

    \textbf{Example Scenario:} A company must decide whether to launch a new product.
    \begin{itemize}
        \item \textbf{Identify the Problem:} Should we launch the product?
        \item \textbf{Generate Alternatives:}
        \begin{itemize}
            \item Option A: Launch the product.
            \item Option B: Delay the launch for market research.
            \item Option C: Abandon the launch.
        \end{itemize}
        \item \textbf{Evaluate Alternatives:} Assess potential revenues and market trends.
        \item \textbf{Choose the Best Alternative:} Calculate expected utilities for each option.
    \end{itemize}
\end{frame}

\begin{frame}[fragile]
    \frametitle{Key Points to Emphasize}
    \begin{itemize}
        \item \textbf{Normative vs. Descriptive:} Normative theory prescribes how to make decisions; descriptive theory explains actual decision-making.
        \item \textbf{Decision Environment:} Assumes an idealized environment with complete information and rationality.
        \item \textbf{Limitations:} Fails to account for emotions, cognitive biases, and social influences in real-world decisions.
    \end{itemize}
    \begin{block}{Conclusion}
        This slide provides a foundational understanding of normative decision theory, setting the stage for the next discussion on Descriptive Decision Theory.
    \end{block}
\end{frame}

\begin{frame}[fragile]
    \frametitle{Descriptive Decision Theory - Overview}
    Descriptive Decision Theory explains how decisions are made in practice.
    \begin{itemize}
        \item Focuses on actual human decision-making.
        \item Contrasts with normative decision theories.
        \item Emphasizes cognitive biases and social influences.
    \end{itemize}
\end{frame}

\begin{frame}[fragile]
    \frametitle{Descriptive Decision Theory - Key Concepts}
    \begin{enumerate}
        \item \textbf{Human Behavior:} Decisions often conflict with rational models.
        \item \textbf{Heuristics:} Mental shortcuts in decision-making.
        \begin{itemize}
            \item Example: Availability heuristic — judging event likelihood based on ease of recall.
        \end{itemize}
    \end{enumerate}
\end{frame}

\begin{frame}[fragile]
    \frametitle{Descriptive Decision Theory - Principles}
    \begin{itemize}
        \item \textbf{Bounded Rationality:} 
            \begin{itemize}
                \item Limited cognitive capacity and available information affects rational decision-making (Herbert Simon).
            \end{itemize}
        \item \textbf{Framing Effects:} 
            \begin{itemize}
                \item Presentation can alter decision outcomes.
                \item Example: Describing a treatment as "90\% success" versus "10\% failure."
            \end{itemize}
    \end{itemize}
\end{frame}

\begin{frame}[fragile]
    \frametitle{Descriptive Decision Theory - Real-World Applications}
    \begin{enumerate}
        \item \textbf{Marketing:} 
            \begin{itemize}
                \item Framing in advertisements influences consumer behavior.
            \end{itemize}
        \item \textbf{Public Policy:} 
            \begin{itemize}
                \item Statistics on crime rates affect public perception based on how they are presented.
            \end{itemize}
    \end{enumerate}
\end{frame}

\begin{frame}[fragile]
    \frametitle{Descriptive Decision Theory - Summary}
    \begin{itemize}
        \item Grounded in realistic human behavior.
        \item Understanding biases improves decision quality.
        \item A key complement to normative decision principles.
        \item Insight into personal and organizational decision-making can be gained through this theory.
    \end{itemize}
\end{frame}

\begin{frame}[fragile]
    \frametitle{Prescriptive Decision Theory}
    \begin{block}{Definition}
        Prescriptive Decision Theory provides recommendations on how to make optimal decisions in uncertain situations. 
        It contrasts with descriptive decision theory, which analyzes how decisions are actually made.
    \end{block}
\end{frame}

\begin{frame}[fragile]
    \frametitle{Key Concepts}
    \begin{enumerate}
        \item \textbf{Normative vs. Prescriptive:}
            \begin{itemize}
                \item \textbf{Normative Decision Theory}: Offers criteria for selecting the best option. 
                \item \textbf{Prescriptive Decision Theory}: Provides practical tools and applications for informed decision-making.
            \end{itemize}
        \item \textbf{Decision-Making Under Uncertainty:}
            \begin{itemize}
                \item It helps to identify strategies that minimize risks and maximize rewards.
                \item Utilizes probabilities and algorithms for evaluating alternatives and predicting outcomes.
            \end{itemize}
    \end{enumerate}
\end{frame}

\begin{frame}[fragile]
    \frametitle{Common Techniques}
    \begin{enumerate}
        \item \textbf{Decision Trees:}
            \begin{itemize}
                \item Visual representations of decisions and their consequences, including chance events.
                \item Example: A company launching a new product constructs a decision tree for potential market responses.
            \end{itemize}
        \item \textbf{Utility Functions:}
            \begin{itemize}
                \item Quantifies preferences to evaluate outcomes based on risk tolerance.
                \item Example: A risk-averse individual prefers a guaranteed reward over a gamble.
            \end{itemize}
        \item \textbf{Sensitivity Analysis:}
            \begin{itemize}
                \item Examines how changes in input variables affect outcome, identifying critical assumptions.
            \end{itemize}
    \end{enumerate}
\end{frame}

\begin{frame}[fragile]
    \frametitle{Example Application: Clinical Decision-Making}
    \begin{block}{Overview}
        Physicians apply prescriptive decision theory to select treatments based on potential outcomes, side effects, patient preferences, and clinical guidelines.
    \end{block}
\end{frame}

\begin{frame}[fragile]
    \frametitle{Key Points to Emphasize}
    \begin{itemize}
        \item Guides decision-making through structured methods.
        \item Applicable in finance, healthcare, and management where uncertainty and risk are prevalent.
        \item Tools developed from this theory allow systematic evaluation for better decision outcomes.
    \end{itemize}
\end{frame}

\begin{frame}[fragile]
    \frametitle{Conclusion}
    \begin{block}{Conclusion}
        Prescriptive Decision Theory is a powerful tool for navigating uncertainty, improving decision-making processes, and leading to sound and justifiable outcomes.
    \end{block}
\end{frame}

\begin{frame}[fragile]
    \frametitle{Formula Insights}
    \begin{equation}
        EU = \sum (P(x_i) \times U(x_i))
    \end{equation}
    Where \(EU\) is expected utility, \(P(x_i)\) is the probability of outcome \(x_i\), and \(U(x_i)\) is the utility of outcome \(x_i\). This formula aids in maximizing expected outcomes based on probabilistic assessments.
\end{frame}

\begin{frame}[fragile]
    \frametitle{Markov Decision Processes (MDP)}
    \begin{block}{What is an MDP?}
        A \textbf{Markov Decision Process (MDP)} is a mathematical framework used to model decision-making situations where outcomes are partly random and partly under the control of a decision-maker.
    \end{block}
\end{frame}

\begin{frame}[fragile]
    \frametitle{Key Components of MDPs}
    An MDP is defined by the following components:
    \begin{enumerate}
        \item \textbf{States (S)}: A finite set of states representing all possible situations for the decision-maker.  
        \item \textbf{Actions (A)}: A finite set of actions available in each state.
        \item \textbf{Transition Model (P)}: The probability of moving from one state to another given an action, denoted by \( P(s' | s, a) \).
        \item \textbf{Rewards (R)}: A function \( R(s, a, s') \) providing feedback after transitioning between states.
        \item \textbf{Discount Factor ($\gamma$)}: A value between 0 and 1 representing the importance of future rewards.
    \end{enumerate}
\end{frame}

\begin{frame}[fragile]
    \frametitle{How MDPs Work and Example}
    MDPs guide decision-making through \textbf{policies}, which dictate actions in each state.
    The goal is to find an optimal policy that maximizes expected rewards, using the \textbf{value function} defined by:
    \begin{equation}
        V(s) = \max_a \sum_{s'} P(s' | s, a) [R(s, a, s') + \gamma V(s')]
    \end{equation}

    \textbf{Example}: A robot navigating a 2x2 grid:
    \begin{itemize}
        \item \textbf{States}: Positions (0,0), (0,1), (1,0), (1,1)
        \item \textbf{Actions}: Move Up, Down, Left, Right
        \item \textbf{Reward}: +10 for reaching (1,1), -1 otherwise
        \item \textbf{Transition Probabilities}: 70\% chance to move successfully, 30\% chance to slip.
    \end{itemize}
\end{frame}

\begin{frame}[fragile]
    \frametitle{Conclusion and Key Points}
    Markov Decision Processes are crucial for modeling decision-making in uncertain environments. Key points to remember:
    \begin{itemize}
        \item MDPs model decisions leading to random outcomes.
        \item Components: States, Actions, Transition Models, Rewards, Discount Factor.
        \item Aim: Identify an optimal policy that maximizes expected rewards.
    \end{itemize}
    MDPs are essential in fields like robotics, economics, and AI.
\end{frame}

\begin{frame}[fragile]
    \frametitle{Reinforcement Learning in Decision Making}
    \begin{block}{Understanding Reinforcement Learning (RL)}
        \begin{itemize}
            \item \textbf{Definition:} 
            Reinforcement Learning is a type of machine learning where an agent learns to make decisions by taking actions in an environment to maximize cumulative rewards over time.
        \end{itemize}
    \end{block}
\end{frame}

\begin{frame}[fragile]
    \frametitle{Key Concepts in Reinforcement Learning}
    \begin{itemize}
        \item \textbf{Agent:} The learner or decision maker (e.g., a robot, software program).
        \item \textbf{Environment:} Everything the agent interacts with (e.g., a game, a stock market).
        \item \textbf{Actions (A):} Choices made by the agent (e.g., move left, right, up, down).
        \item \textbf{States (S):} Possible situations in which the agent can find itself.
        \item \textbf{Reward (R):} Feedback from the environment after taking an action, which can be positive (reward) or negative (penalty).
    \end{itemize}
\end{frame}

\begin{frame}[fragile]
    \frametitle{Dynamic Decision Making with RL}
    \begin{block}{How RL Relates to Decision Making}
        \begin{itemize}
            \item In real-world problems, the environment can change, and decisions must be adaptive. 
            \item RL allows agents to learn optimal strategies by exploring various actions and observing outcomes.
        \end{itemize}
    \end{block}
    \begin{block}{Example Scenario: Autonomous Vehicle}
        \begin{itemize}
            \item \textbf{Goal:} Navigate safely to a destination, avoid obstacles.
            \item \textbf{States:} Current position, speed, distance to obstacles.
            \item \textbf{Actions:} Accelerate, brake, turn left/right.
            \item \textbf{Rewards:} Positive reward for reaching destination quickly; negative reward for collisions or unsafe maneuvers.
        \end{itemize}
    \end{block}
\end{frame}

\begin{frame}[fragile]
    \frametitle{Reinforcement Learning Framework}
    \begin{enumerate}
        \item \textbf{Exploration vs. Exploitation:} 
        \begin{itemize}
            \item \textbf{Exploration:} Trying new actions to discover their effects.
            \item \textbf{Exploitation:} Using known actions that yield high rewards.
        \end{itemize}
        
        \item \textbf{Value Function:} 
        \begin{itemize}
            \item Estimates the expected return (reward) from each state, guiding the agent’s decisions.
            \item \textbf{Formula:} 
            \begin{equation}
                V(s) = \mathbb{E}[R_t | S_t = s]
            \end{equation}
        \end{itemize}
        
        \item \textbf{Policy ($\pi$):} A strategy that specifies the action to take in each state, which can be deterministic or stochastic.
    \end{enumerate}
\end{frame}

\begin{frame}[fragile]
    \frametitle{Reinforcement Learning Algorithms}
    \begin{itemize}
        \item \textbf{Q-Learning:} 
        \begin{itemize}
            \item A model-free approach to learn the value of actions (Q-values) without requiring a model of the environment.
            \item \textbf{Update Rule:} 
            \begin{equation}
                Q(s, a) \leftarrow Q(s, a) + \alpha \left( R + \gamma \max_{a'} Q(s', a') - Q(s, a) \right)
            \end{equation}
            \begin{itemize}
                \item where:
                \begin{itemize}
                    \item $\alpha$ = learning rate,
                    \item $\gamma$ = discount factor,
                    \item $s'$ = next state.
                \end{itemize}
            \end{itemize}
        \end{itemize}
        
        \item \textbf{Deep Q-Networks (DQN):} Combines Q-learning with deep neural networks to handle large state spaces.
    \end{itemize}
\end{frame}

\begin{frame}[fragile]
    \frametitle{Conclusion}
    \begin{itemize}
        \item Reinforcement Learning is vital for developing adaptive systems capable of real-time decision making.
        \item The balance between exploration and exploitation is crucial for effective learning.
        \item The concepts of Q-values and policies form the basis of many RL algorithms.
    \end{itemize}
    \begin{block}{Final Note:}
        \textbf{Reinforcement Learning} equips agents with the ability to make informed decisions in dynamic environments, paving the way for advanced applications like robotics, finance, and autonomous systems.
    \end{block}
\end{frame}

\begin{frame}[fragile]
    \frametitle{Utility Theory - Introduction}
    \begin{block}{Definition}
        Utility theory is a framework in economics and decision-making that provides insights into how individuals rank preferences and make choices under uncertainty.
    \end{block}
    \begin{itemize}
        \item Quantifies satisfaction or value (utility) derived from consumption.
        \item Key concepts include Cardinal and Ordinal Utility.
    \end{itemize}
\end{frame}

\begin{frame}[fragile]
    \frametitle{Utility Theory - Key Concepts and Applications}
    \begin{block}{Key Concepts}
        \begin{itemize}
            \item \textbf{Utility:} Measure of preferences; higher utility = more preferred.
            \item \textbf{Cardinal Utility:} Measured in absolute terms.
            \item \textbf{Ordinal Utility:} Ranked preferences without absolute measurement.
        \end{itemize}
    \end{block}
    
    \begin{block}{Applications}
        Utility theory is applicable in:
        \begin{itemize}
            \item Economics: Modeling consumer behavior.
            \item Psychology: Understanding decision-making processes.
            \item Game Theory: Predicting strategies based on preferences.
        \end{itemize}
    \end{block}
\end{frame}

\begin{frame}[fragile]
    \frametitle{Utility Theory - Example and Key Points}
    \begin{block}{Example: Dinner Options}
        \begin{itemize}
            \item Option A: Pizza (utility = 30)
            \item Option B: Sushi (utility = 50)
            \item Option C: Salad (utility = 20)
        \end{itemize}
        \textbf{Decision:} Choose Sushi to maximize satisfaction.
    \end{block}
    
    \begin{block}{Key Points to Emphasize}
        \begin{itemize}
            \item Subjectivity of utility: personal preferences vary.
            \item Risk aversion: preference for certain outcomes.
            \item Indifference curves: represent trade-offs and preferences.
        \end{itemize}
    \end{block}
\end{frame}

\begin{frame}[fragile]
    \frametitle{Probabilistic Models in Decision Making}
    Discussion of the role of probabilistic models in handling uncertainty in decision-making.
\end{frame}

\begin{frame}[fragile]
    \frametitle{Introduction to Probabilistic Models}
    \begin{itemize}
        \item \textbf{Definition}: Probabilistic models are mathematical frameworks representing the uncertainty and variability in outcomes of uncertain events.
        \item Essential for decision-making when the future is unpredictable.
    \end{itemize}
\end{frame}

\begin{frame}[fragile]
    \frametitle{Role of Probabilistic Models in Decision-Making}
    \begin{enumerate}
        \item \textbf{Handling Uncertainty}: 
        \begin{itemize}
            \item Life is full of uncertainties (e.g., market trends, weather).
            \item Models help quantify uncertainties for more informed choices.
        \end{itemize}
        
        \item \textbf{Risk Assessment}:
        \begin{itemize}
            \item Analyze risks associated with options to align choices with risk tolerance.
            \item Example: Using historical data to estimate market acceptance for a new product launch.
        \end{itemize}

        \item \textbf{Optimization}:
        \begin{itemize}
            \item Simulating scenarios to identify favorable outcomes and maximize expected utility.
            \item Linked to utility theory concepts from previous discussions.
        \end{itemize}
    \end{enumerate}
\end{frame}

\begin{frame}[fragile]
    \frametitle{Key Components of Probabilistic Models}
    \begin{itemize}
        \item \textbf{Random Variables}: Values determined by outcomes of random phenomena (e.g., daily sales forecast).
        
        \item \textbf{Probability Distributions}: Functions providing probabilities of outcomes. Key types include:
        \begin{itemize}
            \item \textbf{Normal Distribution}: Common in nature and finance.
            \item \textbf{Binomial Distribution}: For two possible outcomes (success/failure).
            \item \textbf{Poisson Distribution}: Models counts of events over time.
        \end{itemize}
        
        \item \textbf{Expected Value}: 
        \begin{equation}
            E(X) = \sum (x_i \cdot P(x_i))
        \end{equation}
        where \(E(X)\) is the expected value, \(x_i\) are outcomes, and \(P(x_i)\) is their probability.
    \end{itemize}
\end{frame}

\begin{frame}[fragile]
    \frametitle{Example Scenario: Investment Decision}
    A company considers investing in a project with three potential outcomes:
    \begin{itemize}
        \item \textbf{Success}: Profit = \$100, Probability = 0.5
        \item \textbf{Break-even}: Profit = \$0, Probability = 0.3
        \item \textbf{Failure}: Loss = -\$50, Probability = 0.2
    \end{itemize}
    
    Using the expected value formula:
    \begin{equation}
        E(X) = (100 \cdot 0.5) + (0 \cdot 0.3) + (-50 \cdot 0.2) = 50 + 0 - 10 = 40
    \end{equation}
    
    The expected profit of \$40 informs whether to proceed with the investment.
\end{frame}

\begin{frame}[fragile]
    \frametitle{Summary of Key Points}
    \begin{itemize}
        \item Probabilistic models are critical for navigating uncertainty in decision-making.
        \item They facilitate effective risk assessment and optimization of choices.
        \item Understanding random variables and probability distributions is fundamental.
        \item Calculating expected values aids in guiding investment decisions and improving outcomes.
    \end{itemize}
\end{frame}

\begin{frame}[fragile]
    \frametitle{Decision Trees - Overview}
    Decision Trees are essential tools used across various fields for decision-making. They help visualize complex processes and forecast outcomes based on different choices.
    
    \begin{block}{Components of a Decision Tree}
        \begin{itemize}
            \item \textbf{Root Node}: The starting point representing an initial decision or question.
            \item \textbf{Decision Nodes}: Points where decisions are made, leading to further breakdowns.
            \item \textbf{Branches}: Connect nodes, showing the flow from decisions to outcomes.
            \item \textbf{Leaf Nodes}: Endpoints that denote final outcomes or consequences of decisions.
        \end{itemize}
    \end{block}
\end{frame}

\begin{frame}[fragile]
    \frametitle{Decision Trees - Building Process}
    \textbf{How to Build a Decision Tree:}
    \begin{enumerate}
        \item \textbf{Define the Decision}: Clearly identify the problem or question.
            \begin{itemize}
                \item Example: ``Should we launch Product A?''
            \end{itemize}
        \item \textbf{Identify the Options}: List possible actions.
            \begin{itemize}
                \item Options: ``Yes'', ``No''.
            \end{itemize}
        \item \textbf{Assess Outcomes}: Determine potential outcomes and their probabilities.
            \begin{itemize}
                \item Example for ``Yes'': 70\% chance of success; 30\% chance of failure.
                \item Example for ``No'': No outcomes as decision not pursued.
            \end{itemize}
        \item \textbf{Illustrate}: Create a visual representation.
    \end{enumerate}
\end{frame}

\begin{frame}[fragile]
    \frametitle{Example Decision Tree}
    \begin{block}{Example Decision Tree Structure}
    \begin{lstlisting}
            Launch Product A?
                   /        \
              Yes (70%)     No
                 /  \
           Success    Failure (30%)
    \end{lstlisting}
    \end{block}

    \begin{block}{Key Points}
        \begin{itemize}
            \item \textbf{Clarity}: Simplifies complex decisions.
            \item \textbf{Visual Aid}: Easier to analyze than tables or raw data.
            \item \textbf{Probabilistic Outcomes}: Illustrates risks and rewards.
            \item \textbf{Flexibility}: Applicable to various scenarios.
        \end{itemize}
    \end{block}
\end{frame}

\begin{frame}[fragile]
    \frametitle{Challenges in AI Decision Making - Overview}
    Artificial Intelligence (AI) decision-making systems aim to emulate human reasoning and provide insights from vast data sets. However, several challenges can undermine the effectiveness and fairness of these systems.
\end{frame}

\begin{frame}[fragile]
    \frametitle{Challenges in AI Decision Making - Key Challenges}
    \begin{enumerate}
        \item \textbf{Bias in AI Models}
        \begin{itemize}
            \item \textit{Explanation}: Bias occurs when an AI system produces prejudiced results due to flawed data or algorithmic design.
            \item \textit{Example}: A hiring algorithm trained on historical data reflecting biased hiring practices may preferentially select candidates from specific demographics.
            \item \textit{Impact}: Perpetuates stereotypes and unequal opportunities.
        \end{itemize}

        \item \textbf{Data Quality}
        \begin{itemize}
            \item \textit{Explanation}: The effectiveness of AI decision-making heavily relies on the quality of data it processes.
            \item \textit{Example}: Missing or inaccurate data points can lead to faulty predictions, such as an AI model misdiagnosing a medical condition.
            \item \textit{Impact}: Compromises the reliability of AI systems and leads to erroneous conclusions.
        \end{itemize}

        \item \textbf{Computational Limits}
        \begin{itemize}
            \item \textit{Explanation}: AI models require significant computational resources, particularly when processing large volumes of data or performing complex calculations.
            \item \textit{Example}: High-dimensional data may necessitate advanced algorithms that demand extensive processing power and time.
            \item \textit{Impact}: Limits the scalability and practicality of AI solutions, especially in time-sensitive environments.
        \end{itemize}
    \end{enumerate}
\end{frame}

\begin{frame}[fragile]
    \frametitle{Challenges in AI Decision Making - Key Points and Conclusion}
    \begin{block}{Key Points to Emphasize}
        \begin{itemize}
            \item \textbf{Understanding Bias}: Recognizing the sources of bias in training data is critical to developing fair AI systems.
            \item \textbf{Ensuring Data Integrity}: Continual evaluation and cleaning of data inputs can enhance decision-making accuracy.
            \item \textbf{Resource Management}: Organizations must balance the need for advanced computational capabilities with constraints in resources.
        \end{itemize}
    \end{block}

    \begin{block}{Conclusion}
        Addressing these challenges is essential for creating effective, ethical, and trustworthy AI systems. Ongoing research and development in AI ethics and data governance can help mitigate these issues and improve the decision-making processes employed by AI technologies.
    \end{block}
\end{frame}

\begin{frame}[fragile]
    \frametitle{Real-World Applications - Overview}
    \begin{block}{Introduction}
        Decision-making theories are crucial in AI development. They guide how AI systems analyze information, weigh alternatives, and make predictions or choices.
    \end{block}
    \begin{itemize}
        \item Understanding these theories in real-world scenarios enhances our appreciation of AI capabilities and limitations.
    \end{itemize}
\end{frame}

\begin{frame}[fragile]
    \frametitle{Real-World Applications - Decision Trees}
    \begin{block}{1. Decision Trees in Healthcare}
        \begin{itemize}
            \item \textbf{Concept:} Flowchart-like structure for decision-making.
            \item \textbf{Application:} Used for medical diagnoses by assessing patient data.
            \item \textbf{Example:} AI models classify potential diabetes patients based on various health metrics.
        \end{itemize}
    \end{block}
\end{frame}

\begin{frame}[fragile]
    \frametitle{Real-World Applications - Reinforcement Learning}
    \begin{block}{2. Reinforcement Learning in Robotics}
        \begin{itemize}
            \item \textbf{Concept:} Focuses on taking actions to maximize cumulative rewards.
            \item \textbf{Application:} Robots learn to improve their functions over time.
            \item \textbf{Example:} A robotic vacuum cleaner learns to navigate by receiving rewards and punishments for its actions.
        \end{itemize}
    \end{block}
\end{frame}

\begin{frame}[fragile]
    \frametitle{Real-World Applications - Game Theory}
    \begin{block}{3. Game Theory in Autonomous Vehicles}
        \begin{itemize}
            \item \textbf{Concept:} Studies strategic interactions where outcomes depend on others' actions.
            \item \textbf{Application:} Autonomous vehicles make decisions based on predicted behavior of other vehicles.
            \item \textbf{Example:} Self-driving cars at intersections assess each other's likely actions for safe navigation.
        \end{itemize}
    \end{block}
\end{frame}

\begin{frame}[fragile]
    \frametitle{Real-World Applications - Key Points}
    \begin{block}{Key Points to Emphasize}
        \begin{itemize}
            \item Decision-making theories streamline processes in AI, improving efficiency.
            \item Applications enhance decision accuracy across industries.
            \item Understanding these applications highlights challenges and ethical considerations in AI.
        \end{itemize}
    \end{block}
\end{frame}

\begin{frame}[fragile]
    \frametitle{Real-World Applications - Conclusion}
    \begin{block}{Conclusion}
        Incorporating decision-making theories into AI enhances functionality and user experience, creating scorecards for AI effectiveness.
        \begin{itemize}
            \item Essential for designing reliable and responsible AI systems as technologies evolve.
        \end{itemize}
    \end{block}
\end{frame}

\begin{frame}[fragile]
    \frametitle{Ethical Considerations - Introduction}
    As Artificial Intelligence (AI) increasingly influences decision-making across various sectors, it is crucial to understand the ethical considerations that accompany these processes. This discussion delves into key ethical implications, guiding principles, and real-world implications of AI decision-making.
\end{frame}

\begin{frame}[fragile]
    \frametitle{Ethical Considerations - Key Concepts}
    \begin{enumerate}
        \item \textbf{Bias and Fairness}
        \item \textbf{Transparency and Accountability}
        \item \textbf{Privacy Concerns}
        \item \textbf{Autonomy and Human Oversight}
        \item \textbf{Security and Misuse}
    \end{enumerate}
\end{frame}

\begin{frame}[fragile]
    \frametitle{Ethical Considerations - Bias and Fairness}
    \begin{block}{Definition}
        AI systems can inadvertently perpetuate biases present in the training data, leading to unfair outcomes.
    \end{block}
    \begin{itemize}
        \item \textbf{Example}: A hiring algorithm that favors male candidates over female candidates due to historical biases in the dataset.
        \item \textbf{Key Point}: Ensuring fairness in AI means continuously evaluating datasets and algorithms for bias and implementing corrective measures.
    \end{itemize}
\end{frame}

\begin{frame}[fragile]
    \frametitle{Ethical Considerations - Transparency and Accountability}
    \begin{block}{Definition}
        Transparency involves understanding how AI systems make decisions, while accountability refers to the responsibility for those decisions.
    \end{block}
    \begin{itemize}
        \item \textbf{Example}: A banking algorithm that denies loan applications needs to provide clear explanations to applicants about the decision-making process.
        \item \textbf{Key Point}: Ethical AI systems should be explainable and allow users to hold entities accountable for outcomes.
    \end{itemize}
\end{frame}

\begin{frame}[fragile]
    \frametitle{Ethical Considerations - Privacy Concerns}
    \begin{block}{Definition}
        AI often relies on vast amounts of data, raising concerns about user privacy and consent.
    \end{block}
    \begin{itemize}
        \item \textbf{Example}: Social media platforms using AI to target advertisements based on personal data without explicit user consent.
        \item \textbf{Key Point}: Organizations must balance the effectiveness of AI with respect for individual privacy rights and data protection laws.
    \end{itemize}
\end{frame}

\begin{frame}[fragile]
    \frametitle{Ethical Considerations - Autonomy and Human Oversight}
    \begin{block}{Definition}
        The extent to which AI systems can operate independently without human intervention.
    \end{block}
    \begin{itemize}
        \item \textbf{Example}: Autonomous vehicles making split-second decisions in accident scenarios, questioning human vs. AI judgment.
        \item \textbf{Key Point}: Ethical frameworks must establish bounds for autonomous decision-making to ensure human welfare is prioritized.
    \end{itemize}
\end{frame}

\begin{frame}[fragile]
    \frametitle{Ethical Considerations - Security and Misuse}
    \begin{block}{Definition}
        AI technologies can be exploited for malicious purposes, threatening security and integrity.
    \end{block}
    \begin{itemize}
        \item \textbf{Example}: Deepfake technology that can generate misleading videos or audio clips used in scams or misinformation campaigns.
        \item \textbf{Key Point}: It is vital to create safeguards to prevent the misuse of AI technology, requiring collaboration across sectors.
    \end{itemize}
\end{frame}

\begin{frame}[fragile]
    \frametitle{Ethical Considerations - Summary}
    Understanding and addressing ethical considerations in AI decision-making is paramount for responsible development and deployment. By focusing on bias, transparency, privacy, autonomy, and security, we can shape AI systems that are not only efficient but also equitable and trustworthy.
\end{frame}

\begin{frame}[fragile]
    \frametitle{Ethical Considerations - Discussion Questions}
    \begin{itemize}
        \item How can organizations ensure AI systems are fair and unbiased?
        \item What role should humans play in overseeing automated decision-making?
        \item How can data privacy be effectively protected in AI applications?
    \end{itemize}
\end{frame}

\begin{frame}[fragile]
    \frametitle{Future Trends in Decision Making}
    \begin{block}{Introduction}
        As Artificial Intelligence (AI) evolves, it influences decision-making across various fields. Understanding future trends is crucial for adaptation and leverage of benefits.
    \end{block}
\end{frame}

\begin{frame}[fragile]
    \frametitle{Emerging Trends - Part 1}
    \begin{enumerate}
        \item \textbf{Increased Integration of AI and Human Intelligence}
            \begin{itemize}
                \item AI systems will augment human decision-making.
                \item \textit{Example}: In healthcare, AI analyzes data and suggests treatments, while doctors make final decisions.
            \end{itemize}
        
        \item \textbf{Data Democratization}
            \begin{itemize}
                \item Empowering non-technical users with AI-driven tools for analysis.
                \item \textit{Illustration}: Tools like Google Data Studio allow users to create reports without extensive coding.
            \end{itemize}
    \end{enumerate}
\end{frame}

\begin{frame}[fragile]
    \frametitle{Emerging Trends - Part 2}
    \begin{enumerate}
        \setcounter{enumi}{2} % Start from 3rd point
        \item \textbf{Explainable AI (XAI)}
            \begin{itemize}
                \item Transparency in AI decision-making processes is essential.
                \item \textit{Example}: In finance, AI explains loan rejections, helping applicants understand criteria.
            \end{itemize}
        
        \item \textbf{Real-time Decision-Making}
            \begin{itemize}
                \item AI will enable decisions in real-time to respond quickly to changes.
                \item \textit{Example}: E-commerce platforms dynamically adjusting pricing based on market conditions.
            \end{itemize}

        \item \textbf{Ethical Decision-Making Frameworks}
            \begin{itemize}
                \item Development of frameworks to align AI decisions with ethical standards.
                \item \textit{Example}: Organizations ensuring AI considers fairness and accountability.
            \end{itemize}
    \end{enumerate}
\end{frame}

\begin{frame}[fragile]
    \frametitle{Emerging Trends - Part 3}
    \begin{enumerate}
        \setcounter{enumi}{5} % Start from 6th point
        \item \textbf{Collaborative AI Systems}
            \begin{itemize}
                \item Future systems will promote collaboration among AI agents for better decision-making.
                \item \textit{Example}: Autonomous vehicles sharing data to optimize routes and enhance safety.
            \end{itemize}
    \end{enumerate}
    
    \begin{block}{Key Points}
        \begin{itemize}
            \item Decision-making in AI is collaborative and transparent.
            \item Human oversight is crucial for responsible AI implementation.
            \item Future demands interdisciplinary skills, blending technical and ethical knowledge.
        \end{itemize}
    \end{block}
\end{frame}

\begin{frame}[fragile]
    \frametitle{Closing Thoughts}
    As we look towards the future of decision-making, it is crucial to be adaptable and informed. Understanding these trends will enable stakeholders to leverage AI technologies for improved outcomes in various fields.
\end{frame}

\begin{frame}[fragile]
    \frametitle{Conclusion - Key Points Summarized}
    
    \begin{enumerate}
        \item \textbf{Definition of Decision-Making}:
            \begin{itemize}
                \item Decision-making is the process of selecting the best option among several alternatives.
                \item It plays a pivotal role in both human and artificial intelligence contexts.
            \end{itemize}
        
        \item \textbf{Theoretical Foundations}:
            \begin{itemize}
                \item Decision-making theories provide frameworks that guide how choices are made.
                \item Key theories include:
                    \begin{itemize}
                        \item \textbf{Rational Decision-Making Model}: Assumes individuals make decisions by systematically analyzing all available information to maximize outcomes.
                        \item \textbf{Bounded Rationality}: Acknowledges cognitive limitations and environmental constraints that hinder rational decision-making.
                        \item \textbf{Prospect Theory}: Highlights how people value gains and losses differently, influencing their choices, marking a departure from purely rational models.
                    \end{itemize}
            \end{itemize}
    \end{enumerate}
    
\end{frame}

\begin{frame}[fragile]
    \frametitle{Conclusion - Applications and Evaluation}

    \begin{enumerate}
        \setcounter{enumi}{3} % Continue the numbering 
        
        \item \textbf{Applications in AI}:
            \begin{itemize}
                \item Decision-making theories are essential for designing AI systems that mimic human reasoning:
                \item Applications include:
                    \begin{itemize}
                        \item \textbf{AI Algorithms}: Many machine learning algorithms employ decision-making models (e.g., decision trees, neural networks) to optimize their outputs.
                        \item \textbf{Real-World Scenarios}: Applications range from healthcare (diagnostic tools) to finance (credit scoring).
                    \end{itemize}
            \end{itemize}

        \item \textbf{Evaluating Outcomes}:
            \begin{itemize}
                \item \textbf{Decision Trees \& Models}: Visual tools that aid in understanding potential outcomes of decisions.
                \item \textbf{Performance Metrics}: Techniques such as expected value and utility are used to assess the effectiveness of decisions made by AI systems.
            \end{itemize}
    \end{enumerate}
    
\end{frame}

\begin{frame}[fragile]
    \frametitle{Conclusion - Importance and Final Thoughts}

    \begin{block}{Importance of Decision-Making Theories in AI}
        \begin{itemize}
            \item \textbf{Enhances Predictive Accuracy}: Implementing robust decision-making theories allows AI systems to make more accurate predictions.
            \item \textbf{Improves User Trust}: Users are more likely to trust AI systems when the rationale behind decisions is informed by theoretical frameworks.
            \item \textbf{Facilitates Ethical Decision-Making}: Theories provide guidance for incorporating ethical considerations into AI, promoting responsible practices.
        \end{itemize}
    \end{block}

    \textbf{Final Thoughts:} Understanding decision-making theories is vital for anyone working with AI. As our exploration of future trends indicates, the role of these theories will only grow in importance, shaping effective and ethical AI applications.
    
\end{frame}


\end{document}