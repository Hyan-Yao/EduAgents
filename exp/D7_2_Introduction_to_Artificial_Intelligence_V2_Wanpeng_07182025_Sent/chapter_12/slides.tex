\documentclass[aspectratio=169]{beamer}

% Theme and Color Setup
\usetheme{Madrid}
\usecolortheme{whale}
\useinnertheme{rectangles}
\useoutertheme{miniframes}

% Additional Packages
\usepackage[utf8]{inputenc}
\usepackage[T1]{fontenc}
\usepackage{graphicx}
\usepackage{booktabs}
\usepackage{listings}
\usepackage{amsmath}
\usepackage{amssymb}
\usepackage{xcolor}
\usepackage{tikz}
\usepackage{pgfplots}
\pgfplotsset{compat=1.18}
\usetikzlibrary{positioning}
\usepackage{hyperref}

% Custom Colors
\definecolor{myblue}{RGB}{31, 73, 125}
\definecolor{mygray}{RGB}{100, 100, 100}
\definecolor{mygreen}{RGB}{0, 128, 0}
\definecolor{myorange}{RGB}{230, 126, 34}
\definecolor{mycodebackground}{RGB}{245, 245, 245}

% Set Theme Colors
\setbeamercolor{structure}{fg=myblue}
\setbeamercolor{frametitle}{fg=white, bg=myblue}
\setbeamercolor{title}{fg=myblue}
\setbeamercolor{section in toc}{fg=myblue}
\setbeamercolor{item projected}{fg=white, bg=myblue}
\setbeamercolor{block title}{bg=myblue!20, fg=myblue}
\setbeamercolor{block body}{bg=myblue!10}
\setbeamercolor{alerted text}{fg=myorange}

% Set Fonts
\setbeamerfont{title}{size=\Large, series=\bfseries}
\setbeamerfont{frametitle}{size=\large, series=\bfseries}
\setbeamerfont{caption}{size=\small}
\setbeamerfont{footnote}{size=\tiny}

% Title Page Information
\title[Reinforcement Learning Basics]{Chapter 12: Reinforcement Learning Basics}
\author[J. Smith]{John Smith, Ph.D.}
\institute[University Name]{
  Department of Computer Science\\
  University Name\\
  \vspace{0.3cm}
  Email: email@university.edu\\
  Website: www.university.edu
}
\date{\today}

% Document Start
\begin{document}

\frame{\titlepage}

\begin{frame}[fragile]
    \titlepage
\end{frame}

\begin{frame}[fragile]
    \frametitle{Overview of Reinforcement Learning (RL)}

    \begin{block}{What is Reinforcement Learning?}
        Reinforcement Learning (RL) is a type of machine learning where an agent learns to make decisions by interacting with an environment to achieve a specific goal. 
        Unlike supervised learning, RL involves trial and error to discover the best actions based on feedback received as rewards or penalties.
    \end{block}
    
    \begin{block}{Key Components of Reinforcement Learning}
        \begin{itemize}
            \item \textbf{Agent}: The learner or decision-maker that interacts with the environment (e.g., a robot or software program).
            \item \textbf{Environment}: The world through which the agent navigates (e.g., a game, a physical space).
            \item \textbf{Actions}: The choices available to the agent that can affect the state of the environment (e.g., move left, jump, pick up an item).
            \item \textbf{States}: Different situations or configurations of the environment (e.g., position on a game board).
            \item \textbf{Rewards}: Feedback signal indicating how well the agent is performing with respect to its goal (e.g., +10 for reaching a target, -1 for a wrong move).
        \end{itemize}
    \end{block}
\end{frame}

\begin{frame}[fragile]
    \frametitle{The Learning Process and Significance of RL}

    \begin{block}{Learning Process}
        \begin{itemize}
            \item \textbf{Exploration vs. Exploitation}: The agent balances exploring new actions to find more rewarding strategies and exploiting known actions to maximize rewards.
            \item \textbf{Feedback Closure}: The agent receives feedback from the environment in the form of rewards, affecting future decision-making.
        \end{itemize}
    \end{block}

    \begin{block}{Significance of Reinforcement Learning in AI}
        \begin{itemize}
            \item \textbf{Adaptability}: RL enables machines to optimize decisions based on dynamic environments, applicable in fields like robotics, game playing, finance, and healthcare.
            \item \textbf{Autonomous Learning}: RL empowers systems to learn from experience, reducing the need for human intervention.
        \end{itemize}
    \end{block}
\end{frame}

\begin{frame}[fragile]
    \frametitle{Example of Reinforcement Learning and Conclusion}

    \begin{block}{Example: Game Playing}
        Consider an RL agent designed to play chess. Each move (action) leads to different game states, with the goal of winning (receiving high rewards). The agent learns through multiple games, exploring strategies and refining its approach based on outcomes.
    \end{block}

    \begin{block}{Key Points}
        \begin{itemize}
            \item RL focuses on learning optimal actions through experience.
            \item Balances exploration of new strategies with exploitation of known ones.
            \item Adaptable and autonomous, making RL powerful in AI applications.
        \end{itemize}
    \end{block}

    \begin{block}{Conclusion}
        Reinforcement Learning represents a paradigm shift in how agents learn and adapt through interaction, paving the way for innovations in AI applications.
    \end{block}
\end{frame}

\begin{frame}[fragile]
    \frametitle{Key Terminology - Part 1}
    \begin{enumerate}
        \item \textbf{Agent}
            \begin{itemize}
                \item \textbf{Definition:} An agent is the decision-maker in a reinforcement learning system. It interacts with the environment to achieve a goal.
                \item \textbf{Example:} In a self-driving car, the car itself is the agent. It takes actions based on its observations of the environment (such as road conditions, other vehicles, etc.).
            \end{itemize}
    \end{enumerate}
\end{frame}

\begin{frame}[fragile]
    \frametitle{Key Terminology - Part 2}
    \begin{enumerate}
        \setcounter{enumi}{1} % Start enumeration from 2
        \item \textbf{Environment}
            \begin{itemize}
                \item \textbf{Definition:} The environment encompasses everything the agent interacts with. It provides the context in which the agent operates.
                \item \textbf{Example:} For the self-driving car, the environment includes the road, weather conditions, traffic signals, and pedestrians.
            \end{itemize}
        
        \item \textbf{Action}
            \begin{itemize}
                \item \textbf{Definition:} An action is a decision made by the agent that affects the state of the environment.
                \item \textbf{Example:} In a board game, an action could be moving a game piece. For the self-driving car, actions include accelerating, braking, or turning.
            \end{itemize}
    \end{enumerate}
\end{frame}

\begin{frame}[fragile]
    \frametitle{Key Terminology - Part 3}
    \begin{enumerate}
        \setcounter{enumi}{3} % Start enumeration from 4
        \item \textbf{Reward}
            \begin{itemize}
                \item \textbf{Definition:} A reward is the feedback received by the agent after it takes an action in the environment. It can be positive (incentive) or negative (punishment).
                \item \textbf{Example:} In a game, scoring a point is a positive reward, while losing a turn could be considered a negative reward. In the self-driving context, successfully avoiding an obstacle could yield a positive reward, while hitting an obstacle results in a negative reward.
            \end{itemize}
        
        \item \textbf{State}
            \begin{itemize}
                \item \textbf{Definition:} A state represents the current situation of the environment as perceived by the agent. It includes all the relevant information necessary for the agent to make decisions.
                \item \textbf{Example:} For the self-driving car, the state might include its current speed, distance from other cars, and whether traffic lights are red or green.
            \end{itemize}
    \end{enumerate}

    \begin{block}{Key Points to Emphasize}
        \begin{itemize}
            \item Interaction Cycle: The agent interacts with the environment to make decisions based on the current state, performs actions, and receives rewards.
            \item Overall Objective: The goal of the agent is to learn a policy that maximizes cumulative rewards over time.
        \end{itemize}
    \end{block}
\end{frame}

\begin{frame}[fragile]
    \frametitle{Key Terminology - Diagrammatic Representation}
    \begin{itemize}
        \item \textbf{Agent} $\leftrightarrow$ \textbf{Environment}
            \begin{itemize}
                \item The agent perceives the current \textbf{State} from the environment.
                \item Based on the state, the agent chooses an \textbf{Action}.
                \item The environment changes state and provides a \textbf{Reward} back to the agent.
            \end{itemize}
    \end{itemize}

    \textbf{Cycle Continuation:} This cycle continues, allowing the agent to learn and improve its decision-making policy.
    
    \begin{block}{Closing Note}
        Understanding these key terms is crucial as they form the foundation of reinforcement learning principles and concepts that will be built upon in subsequent slides.
    \end{block}
\end{frame}

\begin{frame}[fragile]
    \frametitle{The Reinforcement Learning Process - Overview}
    \begin{block}{Reinforcement Learning}
        Reinforcement Learning (RL) is a framework through which an agent interacts with its environment to achieve a specific goal.
    \end{block}
    
    \begin{itemize}
        \item Characterized by a cycle of actions, states, rewards, and updates.
        \item Allows the agent to learn optimal behaviors over time.
    \end{itemize}
\end{frame}

\begin{frame}[fragile]
    \frametitle{The Reinforcement Learning Process - Interaction Cycle}
    
    \begin{enumerate}
        \item \textbf{Agent:} The learner or decision-maker that interacts with the environment.
        \item \textbf{Environment:} Everything that the agent interacts with; it provides context.
        \item \textbf{State (s):} A snapshot of the environment; it represents the current situation.
        \item \textbf{Action (a):} A decision made by the agent that affects the state of the environment.
        \item \textbf{Reward (r):} A feedback signal from the environment based on the agent's action.
    \end{enumerate}
\end{frame}

\begin{frame}[fragile]
    \frametitle{The Reinforcement Learning Cycle}
    
    \begin{enumerate}
        \item \textbf{Perception of State:} The agent observes the current state of the environment (s).
        \item \textbf{Action Selection:} The agent selects an action (a) based on its policy.
        \item \textbf{Action Execution:} The action is executed, changing the state of the environment.
        \item \textbf{Reward Reception:} The environment sends a reward (r) based on the action taken.
        \item \textbf{Policy Update:} The agent updates its policy based on the new information.
    \end{enumerate}
\end{frame}

\begin{frame}[fragile]
    \frametitle{Example Scenario - Robot Navigation}
    
    \begin{itemize}
        \item **State:** Current position of the robot on the grid.
        \item **Action:** Moving up, down, left, or right.
        \item **Reward:** 
            \begin{itemize}
                \item +1 point for reaching a designated goal cell.
                \item -1 point for hitting an obstacle.
            \end{itemize}
    \end{itemize}
    
    \begin{block}{Learning Process}
        Over time, the robot learns which actions lead to positive rewards, improving its strategy.
    \end{block}
\end{frame}

\begin{frame}[fragile]
    \frametitle{Key Points and Conclusion}
    
    \begin{itemize}
        \item The interaction cycle is fundamental to RL: Iterates through states, actions, and rewards.
        \item The agent's objective is to maximize cumulative rewards over time.
        \item Exploration vs. Exploitation: The agent balances trying new actions with using known rewarding actions.
    \end{itemize}
    
    \begin{block}{Conclusion}
        Understanding the reinforcement learning process is vital as it sets the foundation for more complex concepts like Markov Decision Processes (MDPs).
    \end{block}
\end{frame}

\begin{frame}[fragile]
    \frametitle{Markov Decision Processes (MDPs)}
    \begin{block}{Introduction to MDPs}
        Markov Decision Processes (MDPs) are mathematical frameworks used to model decision-making problems in reinforcement learning. They provide a structured way to capture the dynamics between an agent and its environment, defining how the agent interacts with the environment to achieve its goals.
    \end{block}
\end{frame}

\begin{frame}[fragile]
    \frametitle{Key Components of MDPs}
    MDPs consist of four main elements:
    \begin{enumerate}
        \item **States (S)**: Different situations for the agent (e.g., configurations in chess).
        \item **Actions (A)**: Possible moves by the agent (e.g., moving a knight).
        \item **Transition Function (P)**: Probability of moving from one state to another given an action:
            \[
            P(s' \mid s, a)
            \]
        \item **Reward Function (R)**: Feedback received post-action, defined as:
            \[
            R(s, a, s')
            \]
        \item **Discount Factor ($\gamma$)**: Value between 0 and 1 determining future reward importance:
            \[
            V(s) = R(s) + \gamma \sum P(s' \mid s, a) V(s')
            \]
    \end{enumerate}
\end{frame}

\begin{frame}[fragile]
    \frametitle{Importance of MDPs and Example Scenario}
    \begin{itemize}
        \item **Structured Framework**: MDPs guide the agent's decision-making process.
        \item **Optimal Policies**: Help define the best actions to maximize rewards.
        \item **Foundation of Algorithms**: Basis for popular RL algorithms like Q-learning.
    \end{itemize}

    \begin{block}{Example Scenario}
        Consider a robotic vacuum cleaner:
        \begin{itemize}
            \item **States (S)**: Positions (e.g., corner, mid-room).
            \item **Actions (A)**: Move, turn, clean.
            \item **Transition Function (P)**: Probabilities of moving (e.g., hitting obstacles).
            \item **Reward Function (R)**: +1 for cleaning, 0 for non-action.
            \item **Discount Factor ($\gamma$)**: 0.9 emphasizes both immediate and future cleaning.
        \end{itemize}
    \end{block}
\end{frame}

\begin{frame}[fragile]
  \frametitle{Components of MDPs - Introduction}
  \begin{block}{Introduction to MDPs}
    Markov Decision Processes (MDPs) are foundational frameworks in reinforcement learning used to model decision-making tasks where outcomes are partly random and partly under the control of a decision maker.
  \end{block}
\end{frame}

\begin{frame}[fragile]
  \frametitle{Components of MDPs - Key Components}
  \begin{enumerate}
    \item \textbf{States (S)}
      \begin{itemize}
        \item \textbf{Definition}: A state represents a specific situation in which an agent can find itself, denoted as S.
        \item \textbf{Example}: In a chess game, each possible arrangement of pieces on the board is a unique state.
      \end{itemize}

    \item \textbf{Actions (A)}
      \begin{itemize}
        \item \textbf{Definition}: An action is a decision made by the agent in a given state, denoted as A(s).
        \item \textbf{Example}: In chess, valid moves such as "move pawn" or "castle" are actions taken by the player.
      \end{itemize}

    \item \textbf{Transition Function (P)}
      \begin{itemize}
        \item \textbf{Definition}: Defines the probability of moving from one state to another given an action, P(s' | s, a).
        \item \textbf{Example}: In a grid world, moving 'up' from (2,2) may lead to (1,2) or keep the agent at (2,2) if hitting a wall.
      \end{itemize}
  \end{enumerate}
\end{frame}

\begin{frame}[fragile]
  \frametitle{Components of MDPs - Rewards and Discount Factor}
  \begin{enumerate}
    \setcounter{enumi}{3} % Continue from previous frame
    \item \textbf{Rewards (R)}
      \begin{itemize}
        \item \textbf{Definition}: A scalar value received after transitioning from state s to s' due to action a, denoted as R(s, a, s').
        \item \textbf{Example}: Winning a match might yield a reward of +10, while losing incurs a penalty of -5.
      \end{itemize}

    \item \textbf{Discount Factor ($\gamma$)}
      \begin{itemize}
        \item \textbf{Definition}: A value between 0 and 1 that represents the importance of future rewards.
        \item \textbf{Example}: In financial investments, a higher $\gamma$ encourages long-term rewards rather than immediate ones.
      \end{itemize}
  \end{enumerate}
  
  \begin{block}{Key Points to Emphasize}
    \begin{itemize}
      \item MDPs encapsulate decision-making dynamics.
      \item Understanding components is crucial for designing reinforcement learning algorithms.
      \item The interplay between states, actions, and rewards drives the learning process.
    \end{itemize}
  \end{block}
\end{frame}

\begin{frame}[fragile]
    \frametitle{Value Functions - Overview}
    
    Value functions are essential concepts in Reinforcement Learning (RL) used to estimate the long-term return from states or actions in a Markov Decision Process (MDP).
    
    \begin{itemize}
        \item Value functions help agents identify beneficial actions for generating cumulative rewards.
    \end{itemize}

\end{frame}

\begin{frame}[fragile]
    \frametitle{Value Functions - Definitions}
    
    \begin{block}{1. What Are Value Functions?}
        Value functions come in two forms:
    \end{block}
    
    \begin{itemize}
        \item {\bf State Value Function (V)}: 
        \[
        V(s) = \mathbb{E}_\pi \left[ R_t + \gamma V(s_{t+1}) \mid S_t = s \right]
        \]
        This function estimates the expected return from state \( s \).
        
        \item {\bf Action Value Function (Q)}: 
        \[
        Q(s, a) = \mathbb{E}_\pi \left[ R_t + \gamma Q(s_{t+1}, a') \mid S_t = s, A_t = a \right]
        \]
        This function estimates the expected return from taking action \( a \) in state \( s \).
    \end{itemize}
\end{frame}

\begin{frame}[fragile]
    \frametitle{Value Functions - Importance}
    
    \begin{block}{2. Importance of Value Functions}
    \end{block}
    \begin{itemize}
        \item {\bf Decision Making}: Direct the agent towards actions yielding higher rewards.
        \item {\bf Policy Evaluation}: Help assess how well a policy performs, enhancing improvements over iterations.
        \item {\bf Convergence and Optimality}: Facilitate the computation of values leading to optimal policies through tools like the Bellman equation.
    \end{itemize}
\end{frame}

\begin{frame}[fragile]
    \frametitle{Value Functions - Example}
    
    \begin{block}{3. Example Scenario}
        Consider a grid world where the agent can move in four directions:
    \end{block}
    \begin{itemize}
        \item {\bf States}: Each position in the grid.
        \item {\bf Actions}: Moves in up, down, left, right directions.
        \item {\bf Rewards}: Positive for reaching goals, negative for hitting walls.
    \end{itemize}
    
    Value functions assist the agent in:
    \begin{enumerate}
        \item Using \( V \) to estimate standing on the grid cell.
        \item Using \( Q \) to estimate moving in a specific direction.
    \end{enumerate}
\end{frame}

\begin{frame}[fragile]
    \frametitle{Value Functions - Key Points}
    
    \begin{block}{4. Key Points to Remember}
    \end{block}
    \begin{itemize}
        \item Value functions encapsulate long-term effects of actions and states in RL.
        \item Essential for understanding and developing algorithms optimizing learning and decision-making.
        \item Foundation for complex RL algorithms like Q-Learning.
    \end{itemize}
\end{frame}

\begin{frame}[fragile]
    \frametitle{Value Functions - Summary}
    
    \begin{block}{5. Summary}
    \end{block}
    
    Value functions provide the framework for agents in reinforcement learning to navigate environments effectively. Mastering their computation and interpretation lays the groundwork for strategies aiming for optimal performance and long-term success.
\end{frame}

\begin{frame}[fragile]
    \frametitle{Q-Learning Algorithm}
    \begin{block}{Introduction to Q-Learning}
        Q-Learning is a model-free reinforcement learning algorithm used to find the optimal policy for an agent interacting with an environment. It enables the agent to learn how to maximize its rewards by taking actions based on observed states.
    \end{block}
\end{frame}

\begin{frame}[fragile]
    \frametitle{Key Concepts}
    \begin{itemize}
        \item \textbf{Agent}: The learner or decision-maker.
        \item \textbf{Environment}: The space in which the agent operates, making transitions based on its actions.
        \item \textbf{State (s)}: A specific situation in the environment.
        \item \textbf{Action (a)}: Choices available to the agent.
        \item \textbf{Reward (r)}: Feedback from the environment after an action is taken.
        \item \textbf{Q-Value (Q)}: The expected utility of taking action $a$ in state $s$, and then following the optimal policy thereafter.
    \end{itemize}
\end{frame}

\begin{frame}[fragile]
    \frametitle{Q-Learning Principles}
    \begin{itemize}
        \item \textbf{Value Function}: Q-Learning estimates the value of action-state pairs, aiming to learn a function $Q(s, a)$ that captures the expected return from taking action $a$ in state $s$.
        \item \textbf{Optimal Policy}: The goal of Q-Learning is to derive the optimal policy $\pi^*$, which maximizes the total reward over time.
    \end{itemize}
\end{frame}

\begin{frame}[fragile]
    \frametitle{The Q-Learning Update Rule}
    The Q-Learning algorithm relies on the fundamental update rule given by:
    \begin{equation}
    Q(s, a) \leftarrow Q(s, a) + \alpha [r + \gamma \max_{a'} Q(s', a') - Q(s, a)]
    \end{equation}
    Where:
    \begin{itemize}
        \item $Q(s, a)$: current Q-value.
        \item $s'$: next state after taking action $a$.
        \item $r$: immediate reward received.
        \item $\alpha$: learning rate (controls how new information overrides old).
        \item $\gamma$: discount factor (determines the importance of future rewards).
        \item $\max_{a'} Q(s', a')$: maximum Q-value for the next state.
    \end{itemize}
\end{frame}

\begin{frame}[fragile]
    \frametitle{Example}
    Imagine training a robot to navigate a maze:
    \begin{itemize}
        \item \textbf{States}: Each position in the maze.
        \item \textbf{Actions}: Move Up, Down, Left, Right.
        \item \textbf{Rewards}: +1 for reaching the exit, -1 for hitting a wall.
    \end{itemize}
    As the robot explores the maze, it updates its $Q$-values based on the rewards received and the expected future rewards, leading it to learn the optimal path to the exit over time.
\end{frame}

\begin{frame}[fragile]
    \frametitle{Importance of Q-Learning}
    \begin{itemize}
        \item \textbf{No Model Required}: Q-Learning does not require a model of the environment, making it versatile and widely applicable.
        \item \textbf{Flexibility}: It can handle problems with stochastic outcomes (i.e., some actions may have uncertain results).
        \item \textbf{Simplicity}: The algorithm's structure is straightforward yet powerful, making it an excellent choice for beginners in reinforcement learning.
    \end{itemize}
\end{frame}

\begin{frame}[fragile]
    \frametitle{Key Points to Remember}
    \begin{itemize}
        \item Q-Learning is crucial for learning optimal policies in uncertain environments.
        \item The Q-Learning update rule is fundamental in improving the agent's knowledge over time.
        \item Balancing exploration and exploitation is a key challenge that will be discussed in the next slide.
    \end{itemize}
    By understanding and applying Q-Learning, agents can learn from their interactions, adapt to new situations, and continually improve their decision-making processes.
\end{frame}

\begin{frame}[fragile]
  \frametitle{Exploration vs. Exploitation}
  \begin{block}{Understanding the Dilemma}
    In reinforcement learning (RL), agents must continuously make decisions involving two contrasting strategies: \textbf{exploration} and \textbf{exploitation}.
  \end{block}
  \begin{itemize}
    \item \textbf{Exploration:} Trying new strategies or actions to discover potential rewards; essential for gaining knowledge about the environment.
    \item \textbf{Exploitation:} Utilizing known information to maximize immediate reward; selecting the best-known actions based on previous experiences.
  \end{itemize}
  \begin{block}{Key Point}
    Balancing these two strategies is crucial for an RL agent to learn effectively and optimize performance over time.
  \end{block}
\end{frame}

\begin{frame}[fragile]
  \frametitle{The Dilemma Illustrated}
  To illustrate this dilemma, consider a scenario:
  \begin{itemize}
    \item Imagine a child in an ice cream shop with two flavors: chocolate and vanilla.
      \begin{itemize}
        \item \textbf{Exploration:} The child tries vanilla for the first time, learning whether they like it.
        \item \textbf{Exploitation:} If they love chocolate, they repeatedly choose it to enjoy its taste.
      \end{itemize}
  \end{itemize}
  \begin{block}{Challenge}
    The challenge for the child (or RL agent) is to decide how much time to spend trying new flavors (exploration) versus enjoying known favorites (exploitation).
  \end{block}
\end{frame}

\begin{frame}[fragile]
  \frametitle{Strategies to Balance Exploration and Exploitation}
  \begin{enumerate}
    \item \textbf{Epsilon-Greedy Strategy:}
      \begin{itemize}
        \item With probability $\epsilon$, choose a random action (exploration).
        \item With probability $1-\epsilon$, choose the action with the highest estimated reward (exploitation).
        \item \textbf{Example:} If $\epsilon = 0.1$, there's a 10\% chance of exploring.
      \end{itemize}
      \begin{lstlisting}[language=Python]
if random.random() < epsilon: 
    action = random.choice(all_actions)  # Explore
else: 
    action = best_action(Q_values)         # Exploit
      \end{lstlisting}

    \item \textbf{Upper Confidence Bound (UCB):}
      \begin{equation}
        \text{UCB} = \hat{Q}(a) + c \cdot \sqrt{\frac{\ln(n)}{n_a}}
      \end{equation}
      Where $\hat{Q}(a)$ is the average reward of action $a$, $n$ is total actions taken, and $n_a$ is times action $a$ has been taken.

    \item \textbf{Thompson Sampling:}
      \begin{itemize}
        \item A probabilistic method where the agent maintains a distribution for the estimated reward of each action and samples from it to choose actions.
      \end{itemize}
  \end{enumerate}
\end{frame}

\begin{frame}[fragile]
  \frametitle{Key Takeaways}
  \begin{itemize}
    \item Exploration and exploitation are fundamental concepts in reinforcement learning.
    \item Striking the right balance is crucial for effective learning and optimizing performance.
    \item Various strategies like Epsilon-Greedy, Upper Confidence Bound, and Thompson Sampling can be employed to manage this balance.
  \end{itemize}
  By understanding and implementing these concepts, agents can learn more effectively in complex environments.
\end{frame}

\begin{frame}[fragile]
    \frametitle{Policy and Value Iteration in Reinforcement Learning}
    % Overview of Policy and Value Iteration algorithms.
    
    \begin{block}{Introduction}
        In reinforcement learning (RL), the goal is to find the optimal policy that maximizes cumulative rewards. 
        Two primary algorithms for this task are:
        \begin{itemize}
            \item Policy Iteration
            \item Value Iteration
        \end{itemize}
        Both use dynamic programming and are applicable in Markov Decision Processes (MDPs).
    \end{block}
\end{frame}

\begin{frame}[fragile]
    \frametitle{Policy Iteration}
    % Detailed explanation of Policy Iteration.
    
    \begin{block}{Definition}
        Policy Iteration is an iterative algorithm that evaluates and improves a policy until it reaches optimality.
    \end{block}
    
    \begin{enumerate}
        \item \textbf{Initialization:} Start with an arbitrary policy.
        \item \textbf{Policy Evaluation:} Calculate value function $V^\pi$ using Bellman equation:
        \begin{equation}
            V^\pi(s) = \sum_{a} \pi(a|s) \sum_{s', r} P(s', r | s, a) [r + \gamma V^\pi(s')]
        \end{equation}
        \item \textbf{Policy Improvement:} Update the policy based on value function:
        \begin{equation}
            \pi'(s) = \underset{a}{\text{argmax}} \sum_{s', r} P(s', r | s, a) [r + \gamma V^\pi(s')]
        \end{equation}
        \item \textbf{Repeat:} Continue until policy stabilizes.
    \end{enumerate}

    \begin{block}{Example}
        In a grid world, start with a random policy. Through evaluation and improvement, the agent learns optimal paths to maximize rewards.
    \end{block}   
\end{frame}

\begin{frame}[fragile]
    \frametitle{Value Iteration}
    % Explanation of Value Iteration.
    
    \begin{block}{Definition}
        Value Iteration calculates the optimal value function directly for determining the optimal policy.
    \end{block}
    
    \begin{enumerate}
        \item \textbf{Initialization:} Set arbitrary value function $V$ for all states.
        \item \textbf{Update:} Revise the value function using the Bellman optimality equation:
        \begin{equation}
            V(s) \leftarrow \max_{a} \sum_{s', r} P(s', r | s, a) [r + \gamma V(s')]
        \end{equation}
        \item \textbf{Convergence Check:} Repeat until $V$ converges (change is smaller than threshold).
    \end{enumerate}

    \begin{block}{Example}
        Initialize values in the grid world to zero. States with high rewards will increase in value quickly, guiding the agent towards the optimal policy.
    \end{block}
\end{frame}

\begin{frame}[fragile]
    \frametitle{Key Points and Conclusion}
    % Summary of key points and conclusion.
    
    \begin{itemize}
        \item \textbf{Exploration vs. Exploitation:} Critical in both algorithms for effective learning.
        \item \textbf{Policy Iteration} requires fewer computations but often more iterations to converge.
        \item \textbf{Value Iteration} computes optimal value more quickly but may involve more calculations per update.
    \end{itemize}

    \begin{block}{Conclusion}
        Both methods are foundational in reinforcement learning, essential for applications in robotics, gaming, and autonomous systems.
    \end{block}
\end{frame}

\begin{frame}[fragile]
    \frametitle{Deep Reinforcement Learning - Overview}
    \begin{block}{Overview}
    Deep Reinforcement Learning (DRL) merges two powerful domains: \textbf{Deep Learning} and \textbf{Reinforcement Learning}. In DRL, neural networks serve as function approximators, enabling agents to tackle problems with high-dimensional state and action spaces.
    \end{block}
\end{frame}

\begin{frame}[fragile]
    \frametitle{Deep Reinforcement Learning - Key Concepts}
    \begin{enumerate}
        \item \textbf{Reinforcement Learning (RL)}:
        \begin{itemize}
            \item \textbf{Agent}: Learns to make decisions by interacting with an environment.
            \item \textbf{Environment}: The setting in which the agent operates.
            \item \textbf{States}: Different situations in which the agent can find itself.
            \item \textbf{Actions}: Choices available to the agent at each state.
            \item \textbf{Rewards}: Feedback received from the environment after an action.
            \item \textbf{Policy}: Strategy that the agent employs based on states.
        \end{itemize}
        
        \item \textbf{Deep Learning}:
        \begin{itemize}
            \item Utilizes neural networks to model complex patterns in data.
            \item Can automatically extract features from raw inputs, suitable for high-dimensional data.
        \end{itemize}
        
        \item \textbf{Combining Deep Learning with Reinforcement Learning}:
        \begin{itemize}
            \item Neural networks approximate value functions or policies, aiding scenarios with large state/action spaces.
            \item Allows agents to learn effective policies directly from high-dimensional sensory input.
        \end{itemize}
    \end{enumerate}
\end{frame}

\begin{frame}[fragile]
    \frametitle{Deep Reinforcement Learning - Training Process}
    \begin{block}{Function Approximation}
    Instead of using a table to store Q-values (as in traditional Q-learning), DRL uses a neural network to predict Q-values for all possible actions, referred to as a \textbf{Q-network}.
    \end{block}

    \begin{block}{Training Process}
    \begin{itemize}
        \item The agent interacts with the environment and collects experiences (state, action, reward, next state).
        \item These experiences are used to update the neural network via \textbf{experience replay} and \textbf{target networks} to stabilize training.
    \end{itemize}
    \end{block}
\end{frame}

\begin{frame}[fragile]
    \frametitle{Deep Q-Network (DQN) Example}
    \begin{lstlisting}[language=Python]
import tensorflow as tf

# Example structure of a simple neural network for DQN
model = tf.keras.Sequential([
    tf.keras.layers.Input(shape=(state_size,)),  # state_size: dimension of the input state
    tf.keras.layers.Dense(24, activation='relu'),
    tf.keras.layers.Dense(24, activation='relu'),
    tf.keras.layers.Dense(action_size, activation='linear')  # action_size: number of possible actions
])

model.compile(optimizer=tf.keras.optimizers.Adam(learning_rate=0.001), loss='mean_squared_error')
    \end{lstlisting}
\end{frame}

\begin{frame}[fragile]
    \frametitle{Deep Reinforcement Learning - Key Points}
    \begin{itemize}
        \item DRL enables solving complex tasks in fields such as robotics, gaming, and autonomous driving.
        \item Neural networks allow efficient learning in environments with large datasets through feature extraction.
        \item Challenges include sample efficiency, stability during training, and exploration-exploitation trade-off.
    \end{itemize}

    \begin{block}{Conclusion}
    Deep Reinforcement Learning represents a significant advancement, allowing intelligent agents to learn directly from raw data inputs, crucial for innovative applications and solutions.
    \end{block}
\end{frame}

\begin{frame}[fragile]
    \frametitle{Applications of Reinforcement Learning - Overview}
    \begin{block}{Introduction to Applications}
        Reinforcement Learning (RL) is widely utilized across various domains due to its ability to learn optimal policies through trial and error. This presentation explores three major areas where RL has significant impacts: robotics, gaming, and resource management.
    \end{block}
\end{frame}

\begin{frame}[fragile]
    \frametitle{Applications of Reinforcement Learning - Robotics}
    \begin{block}{Overview}
        RL enables robots to learn tasks by interacting with their environment, improving decision-making skills through rewards and penalties.
    \end{block}

    \begin{itemize}
        \item \textbf{Autonomous Navigation:} 
        Robots learn to navigate by receiving rewards for reaching goals and penalties for collisions (e.g., self-driving cars).
        
        \item \textbf{Manipulation Tasks:}
        Robots can learn to grasp and move objects through various strategies, improving with each attempt.
    \end{itemize}

    \begin{block}{Example}
        A robot learns to fetch an object by trial and error, developing successful retrieval strategies over time.
    \end{block}
\end{frame}

\begin{frame}[fragile]
    \frametitle{Applications of Reinforcement Learning - Gaming}
    \begin{block}{Overview}
        RL's capability to manage large state spaces makes it perfect for training AI agents in a variety of games.
    \end{block}

    \begin{itemize}
        \item \textbf{Game Playing Agents:} 
        Algorithms such as Deep Q-Network (DQN) allow agents to learn strategies in games like chess and Go.
        
        \item \textbf{Dynamic Difficulty Adjustment:}
        RL is used to adjust game difficulty in real-time, enhancing user engagement.
    \end{itemize}

    \begin{block}{Example}
        AlphaGo used RL methods to defeat a world champion in Go, refining its strategy through extensive self-play.
    \end{block}
\end{frame}

\begin{frame}[fragile]
    \frametitle{Applications of Reinforcement Learning - Resource Management}
    \begin{block}{Overview}
        RL is increasingly applied to optimize resource distribution in fields like finance, logistics, and energy systems.
    \end{block}

    \begin{itemize}
        \item \textbf{Smart Grid Management:} 
        RL algorithms optimize energy distribution based on consumption patterns.
        
        \item \textbf{Supply Chain Optimization:}
        RL enhances inventory management by learning optimal ordering policies, minimizing costs.
    \end{itemize}

    \begin{block}{Example}
        In smart energy grids, RL systems learn when to store or draw energy based on real-time consumption and generation patterns.
    \end{block}
\end{frame}

\begin{frame}[fragile]
    \frametitle{Key Points and Conclusion}
    \begin{itemize}
        \item \textbf{Learning from Interaction:} 
        RL models excel in complex, dynamic systems by learning through interactions.
        
        \item \textbf{Versatility of Applications:} 
        RL demonstrates its applicability from gaming to real-world resource management.
        
        \item \textbf{Importance of Rewards:} 
        Rewards and penalties shape the behavior of RL agents, guiding them toward successful outcomes.
    \end{itemize}

    \begin{block}{Conclusion}
        Reinforcement Learning holds the potential to revolutionize multiple industries, leading to more efficient and intelligent processes.
    \end{block}
\end{frame}

\begin{frame}[fragile]
    \frametitle{Code Snippet: Simple Q-Learning Algorithm}
    \begin{lstlisting}[language=Python]
import numpy as np

# Initialize parameters
alpha = 0.1  # learning rate
gamma = 0.9  # discount factor
epsilon = 0.1  # exploration rate
Q = np.zeros((number_of_states, number_of_actions))  # Q-table

# Update rule
def update_Q(state, action, reward, next_state):
    best_next_action = np.argmax(Q[next_state])
    Q[state, action] += alpha * (reward + gamma * Q[next_state, best_next_action] - Q[state, action])

# Exploration vs Exploitation
if np.random.rand() < epsilon:
    action = np.random.choice(number_of_actions)  # explore
else:
    action = np.argmax(Q[state])  # exploit
    \end{lstlisting}
\end{frame}

\begin{frame}[fragile]
    \frametitle{Challenges in Reinforcement Learning}
    \begin{block}{Introduction}
        Reinforcement Learning (RL) is a powerful approach for training agents to make decisions through trial and error. Despite its potential, RL faces several challenges that affect its efficiency and effectiveness. This slide highlights three major challenges: 
        \begin{itemize}
            \item Convergence Issues
            \item Scalability
            \item Need for Large Amounts of Data
        \end{itemize}
    \end{block}
\end{frame}

\begin{frame}[fragile]
    \frametitle{Convergence Issues}
    \begin{block}{Definition}
        Convergence in RL refers to the agent's ability to find and consistently apply an optimal policy based on its experiences.
    \end{block}
    \begin{itemize}
        \item \textbf{Challenges:}
        \begin{itemize}
            \item Local Optima: Algorithms can get stuck in local optima, finding suboptimal solutions.
            \item Divergence: Value functions may not converge with function approximation.
        \end{itemize}
        \item \textbf{Example:} An agent in a multi-armed bandit problem might exploit a suboptimal arm too quickly, impacting long-term performance.
    \end{itemize}
\end{frame}

\begin{frame}[fragile]
    \frametitle{Scalability and Data Requirements}
    \begin{block}{Scalability}
        \begin{itemize}
            \item \textbf{Definition:} Ability of RL algorithms to handle larger and more complex environments.
            \item \textbf{Challenges:}
            \begin{itemize}
                \item State and Action Space Growth: Increased complexity leads to exponential growth in required resources.
                \item Curse of Dimensionality: Higher dimensions complicate pattern recognition and necessitate greater exploration.
            \end{itemize}
            \item \textbf{Example:} A robot in a complex environment faces exponentially many configurations, affecting training time and computational needs.
        \end{itemize}
    \end{block}
    
    \begin{block}{Need for Large Amounts of Data}
        \begin{itemize}
            \item \textbf{Definition:} Substantial interaction data is often required for effective learning.
            \item \textbf{Challenges:}
            \begin{itemize}
                \item Sample Efficiency: Many RL algorithms are inefficient, requiring numerous interactions to converge.
                \item Environment Interaction: Collecting data is time-consuming and risky in real-world scenarios.
            \end{itemize}
            \item \textbf{Example:} Training an autonomous drone may require thousands of flight hours, with each failure being costly.
        \end{itemize}
    \end{block}
\end{frame}

\begin{frame}[fragile]
    \frametitle{Key Points and Conclusion}
    \begin{itemize}
        \item \textbf{Convergence Problems:} Addressing local optima is critical for achieving optimal policies.
        \item \textbf{Scalability Concerns:} Efficient RL algorithms for large environments are a key research focus.
        \item \textbf{Data Requirements:} Improving sample efficiency and developing less interaction-dependent methods is vital.
    \end{itemize}
    
    \begin{block}{Conclusion}
        Understanding challenges in convergence, scalability, and data requirements in RL is essential for developing effective solutions that enhance performance.
    \end{block}
    
    \begin{block}{Additional Notes}
        Consider techniques such as experience replay, sophisticated neural architectures, or transfer learning to overcome these challenges.
    \end{block}
\end{frame}

\begin{frame}[fragile]
    \frametitle{Ethical Considerations - Introduction}
    \begin{block}{Introduction to Ethical Implications of Reinforcement Learning (RL)}
        Reinforcement Learning (RL) is a powerful machine learning paradigm 
        where agents learn to make decisions by interacting with their environment 
        to maximize cumulative rewards. However, the deployment of RL systems can introduce 
        significant ethical considerations that must be addressed to ensure 
        fair and responsible use.
    \end{block}
\end{frame}

\begin{frame}[fragile]
    \frametitle{Ethical Considerations - Fairness}
    \begin{block}{1. Fairness}
        \textbf{Definition}: Fairness involves ensuring that RL systems do not 
        inadvertently perpetuate or exacerbate biases against certain groups.
    \end{block}
    \begin{itemize}
        \item \textbf{Key Points}:
        \begin{itemize}
            \item Bias in Data: If training data includes biases, the RL model may develop biased policies.
            \item Example: An RL-based hiring system may prioritize candidates based on biased historical hiring data, resulting in systemic discrimination.
        \end{itemize}
        \item \textbf{Strategies to Promote Fairness}:
        \begin{itemize}
            \item Use fairness constraints during training.
            \item Conduct regular audits to assess bias in agent behavior.
        \end{itemize}
    \end{itemize}
\end{frame}

\begin{frame}[fragile]
    \frametitle{Ethical Considerations - Accountability and Transparency}
    \begin{block}{2. Accountability}
        \textbf{Definition}: Accountability refers to the obligation to answer 
        for the decisions made by RL systems.
    \end{block}
    \begin{itemize}
        \item \textbf{Key Points}:
        \begin{itemize}
            \item Decision Transparency: Stakeholders must understand how and why RL agents make certain decisions.
            \item Example: In autonomous vehicles, if an accident occurs, it's crucial to determine whether the RL model's decision-making was at fault.
        \end{itemize}
        \item \textbf{Strategies to Enhance Accountability}:
        \begin{itemize}
            \item Incorporate explainable AI (XAI) techniques.
            \item Maintain logs of agent actions for post-analysis.
        \end{itemize}
    \end{itemize}

    \begin{block}{3. Transparency}
        \textbf{Definition}: Transparency is the degree to which the workings 
        of the RL algorithms are open to scrutiny.
    \end{block}
    \begin{itemize}
        \item \textbf{Key Points}:
        \begin{itemize}
            \item Openness: The underlying algorithms and decision-making processes should be transparent.
            \item Example: A recommender system using RL should provide insights into why certain items are recommended over others.
        \end{itemize}
        \item \textbf{Strategies to Boost Transparency}:
        \begin{itemize}
            \item Share model architectures and data used for training.
            \item Document assumptions and potential limitations of the RL model.
        \end{itemize}
    \end{itemize}
\end{frame}

\begin{frame}[fragile]
    \frametitle{Ethical Considerations - Conclusion}
    \begin{block}{Conclusion}
        As we continue to integrate reinforcement learning into various domains—ranging from healthcare to finance—it is crucial to address these ethical considerations proactively. 
        By focusing on fairness, accountability, and transparency, we can harness RL's potential while mitigating risks of harmful impacts.
    \end{block}
\end{frame}

\begin{frame}[fragile]
    \frametitle{Ethical Considerations - Formulas and Code Snippets}
    \begin{block}{Formulas / Code Snippets}
        \textbf{Reward Function Example}:
        \begin{equation}
            R(s, a) = \text{Immediate Reward based on State (s) and Action (a)}
        \end{equation}
    
        \textbf{Fairness Metric (Demographic Parity)}:
        \begin{lstlisting}[language=Python]
def demographic_parity(predictions, sensitive_attribute):
    # Calculate demographic parity by comparing positive prediction rates across groups.
    return positive_rate_group_1 == positive_rate_group_2
        \end{lstlisting}
    \end{block}
\end{frame}

\begin{frame}
    \frametitle{Future Trends in Reinforcement Learning}
    \begin{block}{Overview}
        Overview of emerging trends and future directions in reinforcement learning (RL) research and applications.
    \end{block}
\end{frame}

\begin{frame}[fragile]
    \frametitle{Integration of Deep Learning and Reinforcement Learning}
    \begin{itemize}
        \item \textbf{Concept}: The combination of deep learning (DL) techniques with RL has accelerated the performance of intelligent agents, allowing them to handle high-dimensional input spaces like images.
        \item \textbf{Example}: Deep Q-Networks (DQN) utilize convolutional neural networks to analyze visual inputs and make decisions in complex environments like video games.
    \end{itemize}
\end{frame}

\begin{frame}[fragile]
    \frametitle{Improved Sample Efficiency and Multi-Agent Reinforcement Learning}
    \begin{itemize}
        \item \textbf{Improved Sample Efficiency}
            \begin{itemize}
                \item \textbf{Concept}: Traditional RL methods often require vast amounts of data. New approaches aim to innovate on sample efficiency, minimizing data requirements for effective learning.
                \item \textbf{Example}: Techniques such as meta-learning allow models to adapt from a few examples, making it possible to transfer learned knowledge across tasks.
            \end{itemize}
        
        \item \textbf{Multi-Agent Reinforcement Learning}
            \begin{itemize}
                \item \textbf{Concept}: Multi-agent systems involve numerous agents learning and interacting in shared environments, leading to emergent behaviors.
                \item \textbf{Example}: Applications in robotics, where multiple robots collaborate to accomplish tasks (e.g., coordinated movement or resource collection).
            \end{itemize}
    \end{itemize}
\end{frame}

\begin{frame}[fragile]
    \frametitle{Model-Based Reinforcement Learning and Human-AI Collaboration}
    \begin{itemize}
        \item \textbf{Model-Based Reinforcement Learning}
            \begin{itemize}
                \item \textbf{Concept}: Instead of relying on samples from the environment, model-based RL creates a model of the environment to predict future states and optimize actions.
                \item \textbf{Example}: Algorithms like AlphaGo leverage model-based reinforcement learning to plan strategies over possible future scenarios, leading to better decision-making.
            \end{itemize}
        
        \item \textbf{Human-AI Collaboration}
            \begin{itemize}
                \item \textbf{Concept}: Systems where humans and RL agents collaborate, enhancing decision-making in various fields.
                \item \textbf{Example}: RL used to develop personalized treatment plans in medicine, allowing healthcare professionals to leverage expert AI-driven suggestions.
            \end{itemize}
    \end{itemize}
\end{frame}

\begin{frame}[fragile]
    \frametitle{Ethical and Responsible AI}
    \begin{itemize}
        \item \textbf{Concept}: The growing capacity of RL systems emphasizes the need for accountability, fairness, and transparency.
        \item \textbf{Example}: Implementing fairness constraints to ensure RL agents do not develop biased policies, particularly in sensitive applications like hiring or law enforcement.
    \end{itemize}
    \begin{block}{Key Points to Emphasize}
        \begin{itemize}
            \item Reinforcement Learning is merging with other fields, enhancing its capabilities and applicability.
            \item Focus on ethical implications and responsible AI practices is crucial for future applications.
            \item Emerging trends will shape RL's role in high-impact sectors, from autonomous vehicles to smart cities.
        \end{itemize}
    \end{block}
\end{frame}

\begin{frame}[fragile]
    \frametitle{Illustration and Sample Code}
    \begin{block}{Illustration Idea}
        A flowchart showing the integration of DL and RL, leading to improved decision-making systems.
    \end{block}
    \begin{block}{Sample Code Snippet}
        \begin{lstlisting}[language=Python]
import gym
import numpy as np
from stable_baselines3 import PPO

env = gym.make('CartPole-v1')
model = PPO('MlpPolicy', env, verbose=1)
model.learn(total_timesteps=10000)
        \end{lstlisting}
    \end{block}
\end{frame}

\begin{frame}[fragile]
    \frametitle{Summary - Overview}
    \begin{block}{Recap of Key Learning Points in Reinforcement Learning Basics}
        \begin{itemize}
            \item Definition of Reinforcement Learning (RL)
            \item Key Components of RL
            \item The RL Process
            \item Exploration vs. Exploitation
            \item Reward Signals
            \item Policies and Value Functions
            \item Common RL Algorithms
            \item Application Areas
        \end{itemize}
    \end{block}
\end{frame}

\begin{frame}[fragile]
    \frametitle{Summary - Key Concepts}
    \begin{enumerate}
        \item \textbf{Definition of Reinforcement Learning (RL)} 
            \begin{itemize}
                \item RL is a type of machine learning where an agent learns to make decisions to maximize cumulative rewards.
                \item It differs from supervised learning as RL learns from the consequences of actions rather than labeled data.
            \end{itemize}
        
        \item \textbf{Key Components of RL}
            \begin{itemize}
                \item \textbf{Agent}: The learner or decision-maker.
                \item \textbf{Environment}: The context in which the agent operates.
                \item \textbf{State (s)}: Representation of the current situation.
                \item \textbf{Action (a)}: Choices available to the agent.
                \item \textbf{Reward (r)}: Feedback from the environment.
            \end{itemize}
    \end{enumerate}
\end{frame}

\begin{frame}[fragile]
    \frametitle{Summary - RL Process and Applications}
    \begin{enumerate}
        \setcounter{enumi}{2}
        \item \textbf{The RL Process}
            \begin{itemize}
                \item Agent observes the current state (s).
                \item Selects an action (a) based on a policy.
                \item Receives a reward (r) from the environment.
                \item Environment transitions to a new state (s').
                \item Uses this information to update its policy.
            \end{itemize}

        \item \textbf{Exploration vs. Exploitation}
            \begin{itemize}
                \item \textbf{Exploration}: Trying new actions.
                \item \textbf{Exploitation}: Choosing the best-known action.
                \item Balancing both is crucial for learning.
            \end{itemize}

        \item \textbf{Application Areas}
            \begin{itemize}
                \item Game playing (e.g. AlphaGo).
                \item Robotics (autonomous navigation).
                \item Personalized recommendations.
                \item Automated trading systems.
            \end{itemize}
    \end{enumerate}
\end{frame}

\begin{frame}[fragile]
    \frametitle{Summary - Key Takeaways}
    \begin{block}{Key Takeaways}
        \begin{itemize}
            \item RL is a dynamic learning paradigm centered on trial and error.
            \item Understanding exploration, exploitation, reward signals, and value functions is essential for RL algorithm development.
            \item RL has versatile real-world applications impacting various industries.
        \end{itemize}
    \end{block}
    
    \begin{block}{Mathematical Update Formula}
        \begin{equation}
            Q(s, a) \leftarrow Q(s, a) + \alpha \left[r + \gamma \max_{a'} Q(s', a') - Q(s, a)\right]
        \end{equation}
        Where:
        \begin{itemize}
            \item $Q(s, a)$: Current value of action.
            \item $\alpha$: Learning rate.
            \item $r$: Reward received.
            \item $\gamma$: Discount factor for future rewards.
        \end{itemize}
    \end{block}
\end{frame}

\begin{frame}[fragile]
  \frametitle{Questions and Discussion - Introduction}
  This slide serves as an interactive platform where we will address any remaining questions about the concepts covered in our chapter on Reinforcement Learning Basics. 
  Engaging in discussion helps solidify understanding and clarify any uncertainties regarding the material.
\end{frame}

\begin{frame}[fragile]
  \frametitle{Key Concepts Recap}
  Before diving into questions, let's briefly revisit some key concepts from the chapter:
  \begin{itemize}
    \item \textbf{Reinforcement Learning (RL):} A type of machine learning where an agent learns to make decisions by taking actions in an environment to maximize cumulative rewards.
    \item \textbf{Agent, Environment, and Reward:}
      \begin{itemize}
        \item \textbf{Agent:} The learner or decision-maker.
        \item \textbf{Environment:} The context or space in which the agent operates.
        \item \textbf{Reward:} The feedback received by the agent after performing an action, helping it learn which actions lead to desirable outcomes.
      \end{itemize}
    \item \textbf{Exploration vs. Exploitation:}
      \begin{itemize}
        \item \textbf{Exploration:} Trying new actions to discover their effects.
        \item \textbf{Exploitation:} Utilizing known information to maximize rewards.
      \end{itemize}
    \item \textbf{Policy:} A strategy used by the agent to determine the next action based on the current state of the environment.
    \item \textbf{Value Function:} A function that estimates the expected return or value of being in a given state or taking a specific action.
  \end{itemize}
\end{frame}

\begin{frame}[fragile]
  \frametitle{Discussion Questions and Conclusion}
  To kick off our discussion, consider these questions:
  \begin{enumerate}
    \item What are some real-world applications of reinforcement learning you think could benefit from exploration strategies?
    \item How might the balance between exploration and exploitation manifest in a gambling scenario?
    \item Can you think of a situation where reinforcement learning could fail or lead to suboptimal results?
  \end{enumerate}

  \textbf{Emphasizing Key Points:}
  \begin{itemize}
    \item Encouraging open dialogue will help reinforce understanding of RL principles.
    \item Clarification of concepts such as policies, value functions, and the exploration-exploitation trade-off is crucial.
    \item Remember, all questions are valid, and there's no such thing as a silly question!
  \end{itemize}

  \textbf{Interactive Activity:}
  \begin{itemize}
    \item \textbf{Quick Poll:} What part of reinforcement learning do you find most challenging? (Options: Definitions, Algorithms, Applications, Other)
  \end{itemize}

  Utilizing this time for questions and discussions is essential for deepening your understanding of reinforcement learning. Let's dive in and clarify what's on your mind!
\end{frame}


\end{document}