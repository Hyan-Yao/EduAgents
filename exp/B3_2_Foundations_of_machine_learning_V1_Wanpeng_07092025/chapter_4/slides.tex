\documentclass[aspectratio=169]{beamer}

% Theme and Color Setup
\usetheme{Madrid}
\usecolortheme{whale}
\useinnertheme{rectangles}
\useoutertheme{miniframes}

% Additional Packages
\usepackage[utf8]{inputenc}
\usepackage[T1]{fontenc}
\usepackage{graphicx}
\usepackage{booktabs}
\usepackage{listings}
\usepackage{amsmath}
\usepackage{amssymb}
\usepackage{xcolor}
\usepackage{tikz}
\usepackage{pgfplots}
\pgfplotsset{compat=1.18}
\usetikzlibrary{positioning}
\usepackage{hyperref}

% Custom Colors
\definecolor{myblue}{RGB}{31, 73, 125}
\definecolor{mygray}{RGB}{100, 100, 100}
\definecolor{mygreen}{RGB}{0, 128, 0}
\definecolor{myorange}{RGB}{230, 126, 34}
\definecolor{mycodebackground}{RGB}{245, 245, 245}

% Set Theme Colors
\setbeamercolor{structure}{fg=myblue}
\setbeamercolor{frametitle}{fg=white, bg=myblue}
\setbeamercolor{title}{fg=myblue}
\setbeamercolor{section in toc}{fg=myblue}
\setbeamercolor{item projected}{fg=white, bg=myblue}
\setbeamercolor{block title}{bg=myblue!20, fg=myblue}
\setbeamercolor{block body}{bg=myblue!10}
\setbeamercolor{alerted text}{fg=myorange}

% Set Fonts
\setbeamerfont{title}{size=\Large, series=\bfseries}
\setbeamerfont{frametitle}{size=\large, series=\bfseries}
\setbeamerfont{caption}{size=\small}
\setbeamerfont{footnote}{size=\tiny}

% Code Listing Style
\lstdefinestyle{customcode}{
  backgroundcolor=\color{mycodebackground},
  basicstyle=\footnotesize\ttfamily,
  breakatwhitespace=false,
  breaklines=true,
  commentstyle=\color{mygreen}\itshape,
  keywordstyle=\color{blue}\bfseries,
  stringstyle=\color{myorange},
  numbers=left,
  numbersep=8pt,
  numberstyle=\tiny\color{mygray},
  frame=single,
  framesep=5pt,
  rulecolor=\color{mygray},
  showspaces=false,
  showstringspaces=false,
  showtabs=false,
  tabsize=2,
  captionpos=b
}
\lstset{style=customcode}

% Document Start
\begin{document}

\frame{\titlepage}

\begin{frame}[fragile]
    \frametitle{Introduction to Supervised Learning}
    \begin{block}{Overview}
        Supervised learning is a type of machine learning where models are trained using labeled data. The goal is to make predictions on unseen data by understanding relationships from historical data.
    \end{block}
\end{frame}

\begin{frame}[fragile]
    \frametitle{Significance in Machine Learning}
    \begin{itemize}
        \item \textbf{Predictive Power:} Enhances accuracy of predictions, helping in informed decision-making.
        \item \textbf{Wide Applications:} Utilized in finance (credit scoring), healthcare (disease prediction), and marketing (customer segmentation).
    \end{itemize}
\end{frame}

\begin{frame}[fragile]
    \frametitle{Key Components of Supervised Learning}
    \begin{enumerate}
        \item \textbf{Training Data:} Contains input features and correct output labels. 
        \item \textbf{Model:} The algorithm that learns from the training data.
        \item \textbf{Evaluation:} Assess model performance using metrics like accuracy and mean squared error.
    \end{enumerate}
\end{frame}

\begin{frame}[fragile]
    \frametitle{Process of Supervised Learning}
    \begin{enumerate}
        \item \textbf{Data Collection:} Gather a dataset with features and labels.
        \item \textbf{Data Splitting:} Divide into training and testing sets.
        \item \textbf{Model Training:} Train the model on the training set.
        \item \textbf{Model Evaluation:} Test and evaluate performance on unseen data.
        \item \textbf{Prediction:} Use the model to make predictions on new, unlabeled data.
    \end{enumerate}
\end{frame}

\begin{frame}[fragile]
    \frametitle{Illustrative Example}
    Consider predicting customer satisfaction scores based on:
    \begin{itemize}
        \item \textbf{Feature 1:} Service Time (in minutes)
        \item \textbf{Feature 2:} Product Quality Rating (1-5 scale)
        \item \textbf{Feature 3:} Staff Friendliness Rating (1-5 scale)
        \item \textbf{Output Label:} Customer Satisfaction Score (1-10 scale)
    \end{itemize}
    This data helps train a supervised model to predict scores for new customers.
\end{frame}

\begin{frame}[fragile]
    \frametitle{Key Takeaways}
    \begin{itemize}
        \item Supervised learning requires labeled datasets to learn mappings from inputs to outputs.
        \item It's foundational for diverse machine learning applications providing actionable insights.
        \item Understanding this area is crucial for implementing complex algorithms in deep learning.
    \end{itemize}
\end{frame}

\begin{frame}[fragile]
    \frametitle{Conclusion}
    By applying supervised learning techniques effectively, we can harness data for predictions and decision-making, paving the way for further exploration into algorithms like Linear Regression.
\end{frame}

\begin{frame}[fragile]{What is Linear Regression? - Definition}
    \begin{block}{Definition}
        Linear Regression is a statistical method used in supervised learning that models the relationship between one or more independent variables (features) and a dependent variable (outcome). It assumes that the relationship can be expressed as a linear equation.
    \end{block}
\end{frame}

\begin{frame}[fragile]{What is Linear Regression? - Role in Supervised Learning}
    \begin{itemize}
        \item \textbf{Foundational Algorithm}: Linear regression serves as one of the most basic and commonly used algorithms in supervised learning, providing a clear understanding of how machine learning models predict outcomes based on input features.
        \item \textbf{Predictive Modeling}: It is primarily used for predicting continuous outcomes, such as sales revenue, temperature, or test scores.
        \item \textbf{Interpretable Results}: The results of a linear regression model are easily interpretable, making it a valuable tool, particularly for beginners in data science.
    \end{itemize}
\end{frame}

\begin{frame}[fragile]{What is Linear Regression? - Key Concepts}
    \begin{itemize}
        \item \textbf{Regression Line}: Represents the best fit line through the data points in a scatter plot, which minimizes the distance (error) between the predicted values and the actual values.
        
        \item \textbf{Equation of the Line}:
        \begin{equation}
        y = mx + b
        \end{equation}
        \begin{itemize}
            \item \(y\): dependent variable (what we're trying to predict)
            \item \(m\): slope of the line (the degree of change)
            \item \(x\): independent variable (the input feature)
            \item \(b\): y-intercept (the value of \(y\) when \(x=0\))
        \end{itemize}
    \end{itemize}
\end{frame}

\begin{frame}[fragile]{What is Linear Regression? - Example}
    \textbf{Example:} Predicting house price based on size.
    
    \begin{itemize}
        \item Independent variable: House size (in square feet) \(x\)
        \item Dependent variable: House price (in dollars) \(y\)
        
        After performing linear regression analysis, we might derive an equation:
        \begin{equation}
        \text{Price} = 150 \times \text{Size} + 30,000
        \end{equation}
        
        \item \textbf{Interpretation}: For every additional square foot, house price increases by \$150; when the size is zero, the estimated house price starts at \$30,000.
    \end{itemize}
\end{frame}

\begin{frame}[fragile]{What is Linear Regression? - Emphasize}
    \begin{itemize}
        \item \textbf{Assumptions}: Linear regression assumes a linear relationship, independence of observations, and homoscedasticity (constant variance of errors).
        \item \textbf{Applications}: Commonly used in fields like economics, medicine, and social sciences for forecasting and establishing connections between variables.
    \end{itemize}
\end{frame}

\begin{frame}[fragile]{What is Linear Regression? - Summary}
    \begin{block}{Summary}
        Linear regression is a crucial starting point in the study of supervised learning, equipping learners with the tools to analyze continuous data and laying the groundwork for understanding more complex algorithms in the future. 
    \end{block}
\end{frame}

\begin{frame}[fragile]{What is Linear Regression? - Code Snippet}
    \begin{block}{Code Snippet}
        Here’s a simple Python example using \texttt{scikit-learn} to perform linear regression:
        \begin{lstlisting}[language=Python]
import numpy as np
from sklearn.linear_model import LinearRegression

# Sample data
X = np.array([[1], [2], [3], [4]])
y = np.array([1, 3, 2, 3])

# Creating the model
model = LinearRegression().fit(X, y)

# Making a prediction
predicted_price = model.predict([[5]])
print(predicted_price)
        \end{lstlisting}
        \textbf{Explanation:} This code trains a linear regression model on a small dataset and predicts the value for \(x = 5\).
    \end{block}
\end{frame}

\begin{frame}[fragile]{Basic Concepts of Linear Regression - Key Concepts}
    \begin{block}{Dependent and Independent Variables}
        \begin{itemize}
            \item \textbf{Independent Variable (Predictor):}
            \begin{itemize}
                \item This is the variable that you manipulate or control. Often referred to as "X".
                \item \textit{Example:} Number of hours studied in a study on test scores.
            \end{itemize}
            \item \textbf{Dependent Variable (Response):}
            \begin{itemize}
                \item This is the outcome that you measure. Often referred to as "Y".
                \item \textit{Example:} Test score in the study affected by hours studied.
            \end{itemize}
        \end{itemize}
    \end{block}
\end{frame}

\begin{frame}[fragile]{Basic Concepts of Linear Regression - The Regression Line}
    \begin{block}{The Regression Line}
        A regression line represents the relationship between the independent variable (X) and the dependent variable (Y).
        \begin{enumerate}
            \item It is a straight line that best fits the data points in a scatter plot.
            \item The equation of the regression line can be expressed as:
            \begin{equation}
                y = mx + b
            \end{equation}
            \begin{itemize}
                \item \( m \) is the slope (change in \( Y \) for a one-unit change in \( X \)).
                \item \( b \) is the y-intercept (value of \( Y \) when \( X = 0 \)).
            \end{itemize}
        \end{enumerate}
    \end{block}
\end{frame}

\begin{frame}[fragile]{Basic Concepts of Linear Regression - Example Illustration}
    \begin{block}{Example Illustration}
        Imagine plotting data that shows how exercise influences weight loss:
        \begin{itemize}
            \item **Point (2, 5):** 2 hours of exercise corresponds to a 5 kg weight loss.
            \item **Point (4, 10):** 4 hours of exercise corresponds to a 10 kg weight loss.
        \end{itemize}
        \begin{itemize}
            \item A line that best fits these points visually represents the relationship.
            \item An increase in exercise by 1 hour may lead to an average weight loss increase of 2.5 kg.
        \end{itemize}
    \end{block}
\end{frame}

\begin{frame}[fragile]{Mathematical Representation - Linear Regression Equation}
    \begin{block}{Linear Regression Equation}
        The linear regression equation is represented mathematically as:
        \[
        y = mx + b
        \]
    \end{block}
\end{frame}

\begin{frame}[fragile]{Mathematical Representation - Explanation of Each Term}
    \begin{enumerate}
        \item \textbf{y (Dependent Variable):} 
            \begin{itemize}
                \item The output or response variable being predicted or explained.
                \item \textit{Example:} In predicting house prices, **y** could be the price of the house.
            \end{itemize}
        
        \item \textbf{m (Slope):} 
            \begin{itemize}
                \item Represents the change in **y** for a one-unit increase in **x**.
                \item Indicates the strength and direction of the relationship.
                \item \textit{Example:} If **m = 200K**, for every unit increase in **x** (e.g., square footage), **y** increases by $200,000.
            \end{itemize}

        \item \textbf{x (Independent Variable):}
            \begin{itemize}
                \item The input or predictor variable used to predict the dependent variable.
                \item \textit{Example:} In our housing example, **x** could represent the total square footage of the house.
            \end{itemize}

        \item \textbf{b (Intercept):}
            \begin{itemize}
                \item The value of **y** when **x** is zero; where the regression line crosses the y-axis.
                \item \textit{Example:} If **b = 100K**, a house with zero square footage would hypothetically have a price of $100,000.
            \end{itemize}
    \end{enumerate}
\end{frame}

\begin{frame}[fragile]{Mathematical Representation - Visualization and Key Points}
    \begin{block}{Visualization of the Concept}
        \begin{itemize}
            \item The x-axis represents the independent variable (**x**).
            \item The y-axis represents the dependent variable (**y**).
            \item The line formed by the equation \(y = mx + b\) is the \textbf{regression line} that fits the data.
        \end{itemize}
    \end{block}

    \begin{itemize}
        \item \textbf{Key Points to Emphasize:}
            \begin{itemize}
                \item \textbf{Linear Relationship:} Assumes a direct relationship between **x** and **y**.
                \item \textbf{Interpretation of m and b:} Crucial for accurate predictions and analysis.
                \item \textbf{Applications:} Used widely in finance, real estate, and other fields to predict outcomes based on historical data.
            \end{itemize}
    \end{itemize}
\end{frame}

\begin{frame}[fragile]{Assumptions of Linear Regression - Overview}
    \begin{block}{Overview}
        Linear regression relies on key assumptions for its validity, including:
    \end{block}
    \begin{itemize}
        \item Linearity
        \item Independence
        \item Homoscedasticity
        \item Normality
    \end{itemize}
    Understanding these assumptions is crucial for effective application and interpretation.
\end{frame}

\begin{frame}[fragile]{Assumption 1: Linearity}
    \frametitle{Assumption 1: Linearity}
    \begin{block}{Explanation}
        The relationship between the independent variable(s) (X) and the dependent variable (Y) must be linear.
    \end{block}
    \begin{example}
        Predicting house prices based on size:
        \begin{itemize}
            \item A change in house size results in a proportional change in price.
        \end{itemize}
    \end{example}
    \begin{block}{Key Point}
        Scatter plots should show a linear pattern between predictor(s) and response variable.
    \end{block}
\end{frame}

\begin{frame}[fragile]{Assumption 2: Independence}
    \frametitle{Assumption 2: Independence}
    \begin{block}{Explanation}
        Observations must be independent; residuals should not be correlated.
    \end{block}
    \begin{example}
        In drug experiments, if one participant's response affects others, independence is violated.
    \end{example}
    \begin{block}{Key Point}
        Statistical tests like the Durbin-Watson test can assess independence in sequential data.
    \end{block}
\end{frame}

\begin{frame}[fragile]{Assumption 3: Homoscedasticity}
    \frametitle{Assumption 3: Homoscedasticity}
    \begin{block}{Explanation}
        Residuals must have constant variance at all levels of the independent variable(s).
    \end{block}
    \begin{example}
        In predicting sales from advertising spend, increasing variance of residuals indicates heteroscedasticity.
    \end{example}
    \begin{block}{Key Point}
        Check homoscedasticity by plotting residuals against predicted values for randomness.
    \end{block}
\end{frame}

\begin{frame}[fragile]{Assumption 4: Normality}
    \frametitle{Assumption 4: Normality}
    \begin{block}{Explanation}
        Residuals should be approximately normally distributed for valid inference.
    \end{block}
    \begin{example}
        A histogram of residuals from test scores should resemble a bell curve.
    \end{example}
    \begin{block}{Key Point}
        Evaluate normality using Q-Q plots or tests like the Shapiro-Wilk test.
    \end{block}
\end{frame}

\begin{frame}[fragile]{Summary and Formula}
    \frametitle{Summary and Linear Regression Equation}
    \begin{block}{Summary}
        Verify assumptions—linearity, independence, homoscedasticity, and normality—to ensure trustworthiness of results.
        Failing these assumptions may lead to incorrect conclusions.
    \end{block}
    \begin{equation}
        y = mx + b
    \end{equation}
    Where:
    \begin{itemize}
        \item $y$ = dependent variable
        \item $m$ = slope of the line
        \item $x$ = independent variable
        \item $b$ = y-intercept
    \end{itemize}
\end{frame}

\begin{frame}[fragile]
    \frametitle{Data Requirements for Linear Regression - Overview}
    \begin{block}{Introduction to Data Requirements}
        For linear regression analysis to be effective, certain types of data are essential. 
        Understanding these requirements will help in appropriately preparing and selecting datasets for model development.
    \end{block}
\end{frame}

\begin{frame}[fragile]
    \frametitle{Data Requirements for Linear Regression - Key Components}
    \begin{enumerate}
        \item \textbf{Continuous Dependent Variable}
        \begin{itemize}
            \item \textbf{Definition:} Dependent variable must be continuous, taking infinite values in a range.
            \item \textbf{Example:} House prices based on features like size or number of bedrooms.
        \end{itemize}
        
        \item \textbf{Independent Variables (Predictors)}
        \begin{itemize}
            \item \textbf{Types:} Continuous (e.g., age, height) and categorical (encoded as dummy variables).
            \item \textbf{Example:} Predicting scores with hours of study (continuous) and course completion (categorical).
        \end{itemize}
    \end{enumerate}
\end{frame}

\begin{frame}[fragile]
    \frametitle{Data Requirements for Linear Regression - Additional Considerations}
    \begin{enumerate}[resume]
        \item \textbf{Linearity}
        \begin{itemize}
            \item \textbf{Concept:} Relationships between independent variables and the dependent variable should be linear.
            \item \textbf{Example:} A scatter plot demonstrating a straight line relationship.
        \end{itemize}
        
        \item \textbf{No Multicollinearity}
        \begin{itemize}
            \item \textbf{Definition:} High correlation among independent variables can make estimates unreliable.
            \item \textbf{Example:} Correlated predictors like height and weight.
        \end{itemize}

        \item \textbf{Sufficient Sample Size}
        \begin{itemize}
            \item \textbf{Guideline:} At least 10-15 observations per predictor variable.
            \item \textbf{Example:} For 3 predictors, require a minimum of 30-45 observations.
        \end{itemize}

        \item \textbf{Summary}
        \begin{itemize}
            \item Ensure continuous dependent variable, check for linearity and multicollinearity, and maintain a large sample size.
        \end{itemize}
    \end{enumerate}
\end{frame}

\begin{frame}[fragile]
    \frametitle{Linear Regression Model Formula}
    The formula for fitting a linear regression model is represented as:
    \begin{equation}
        Y = \beta_0 + \beta_1 X_1 + \beta_2 X_2 + \ldots + \beta_n X_n + \epsilon
    \end{equation}
    \begin{itemize}
        \item \(Y\) = dependent variable
        \item \(\beta_0\) = intercept
        \item \(\beta_1, \beta_2, \ldots, \beta_n\) = coefficients for independent variables \(X_1, X_2, \ldots, X_n\)
        \item \(\epsilon\) = error term
    \end{itemize}
\end{frame}

\begin{frame}[fragile]{Model Training and Evaluation - Overview}
    \begin{block}{Overview}
        In this slide, we explore the crucial steps for training a linear regression model and evaluating its performance. 
        Key metrics, particularly Mean Squared Error (MSE), will be highlighted to gauge model accuracy and reliability.
    \end{block}
\end{frame}

\begin{frame}[fragile]{Model Training Steps}
    \frametitle{Steps to Train a Linear Regression Model}
    \begin{enumerate}
        \item \textbf{Data Preparation:}
            \begin{itemize}
                \item \textbf{Data Splitting:} Divide the dataset into training and testing sets (e.g., 80/20 split).
                \item \textbf{Feature Selection:} Identify independent variables to predict the target variable.
                
                \textit{Example:} Size, number of bedrooms, and location when predicting house prices.
            \end{itemize}

        \item \textbf{Model Training:}
            \begin{itemize}
                \item Use training data to fit the linear regression model.
                \item \textbf{Formula:} \begin{equation}
                    y = \beta_0 + \beta_1 x_1 + \beta_2 x_2 + ... + \beta_n x_n
                \end{equation}
            \end{itemize}

        \item \textbf{Model Fitting:}
            \begin{itemize}
                \item Fit the model using an optimization algorithm (e.g., Gradient Descent).
            \end{itemize}
    \end{enumerate}
\end{frame}

\begin{frame}[fragile]{Model Evaluation}
    \frametitle{Model Evaluation}
    \begin{enumerate}
        \item \textbf{Making Predictions:}
            \begin{itemize}
                \item Generate predictions on the testing dataset using the trained model.
            \end{itemize}

        \item \textbf{Performance Metric: Mean Squared Error (MSE):}
            \begin{itemize}
                \item \textbf{Definition:} Measures the average squared difference between predicted and actual values.
                \item \textbf{Formula:} \begin{equation}
                    MSE = \frac{1}{n} \sum_{i=1}^n (y_i - \hat{y_i})^2
                \end{equation}
                \item \textbf{Interpretation:} A lower MSE indicates a better fit; an MSE of 0 means perfect predictions.
            \end{itemize}
    \end{enumerate}

    \begin{block}{Key Points}
        \begin{itemize}
            \item Importance of Data Quality: Clean data enhances model performance.
            \item Overfitting vs. Underfitting: Balance is crucial for generalization.
            \item MSE Sensitivity: Consider using RMSE or R² alongside MSE for comprehensive evaluation.
        \end{itemize}
    \end{block}
\end{frame}

\begin{frame}[fragile]{Example Code Snippet}
    \frametitle{Example Code Snippet (in Python using Scikit-Learn)}
    \begin{lstlisting}[language=Python]
from sklearn.model_selection import train_test_split
from sklearn.linear_model import LinearRegression
from sklearn.metrics import mean_squared_error
import numpy as np

# Sample dataset
X = np.array([[1], [2], [3], [4]])
y = np.array([2, 3, 5, 7])

# Splitting the data into training and testing sets
X_train, X_test, y_train, y_test = train_test_split(X, y, test_size=0.2, random_state=42)

# Initializing and training the linear regression model
model = LinearRegression()
model.fit(X_train, y_train)

# Making predictions
y_pred = model.predict(X_test)

# Evaluating the model
mse = mean_squared_error(y_test, y_pred)
print('Mean Squared Error:', mse)
    \end{lstlisting}
\end{frame}

\begin{frame}[fragile]
    \frametitle{Implementation in Python - Introduction}
    \begin{block}{Introduction to Linear Regression with Scikit-Learn}
        Linear regression is a fundamental supervised learning algorithm used to model the relationship between a dependent variable \(y\) and one or more independent variables \(X\).
        
        In Python, **Scikit-Learn** (or sklearn) is one of the most popular libraries for implementing linear regression, providing a simple and efficient way to perform various machine learning tasks.
    \end{block}
\end{frame}

\begin{frame}[fragile]
    \frametitle{Implementation in Python - Steps}
    \begin{enumerate}
        \item \textbf{Import Libraries}: Import necessary libraries like `numpy`, `pandas`, and Scikit-Learn.
        
        \begin{lstlisting}[language=Python]
import numpy as np
import pandas as pd
from sklearn.model_selection import train_test_split
from sklearn.linear_model import LinearRegression
from sklearn.metrics import mean_squared_error
        \end{lstlisting}
        
        \item \textbf{Load the Dataset}: Load your dataset into a pandas DataFrame.
        
        \begin{lstlisting}[language=Python]
data = pd.read_csv('data.csv')
        \end{lstlisting}
        
        \item \textbf{Data Preprocessing}: Extract features and the target variable.
        
        \begin{lstlisting}[language=Python]
X = data[['area', 'bedrooms']]
y = data['price']
        \end{lstlisting}
    \end{enumerate}
\end{frame}

\begin{frame}[fragile]
    \frametitle{Implementation in Python - Continued Steps}
    \begin{enumerate}[resume]
        \item \textbf{Split the Data}: Divide data into training and testing sets.

        \begin{lstlisting}[language=Python]
X_train, X_test, y_train, y_test = train_test_split(X, y, test_size=0.2, random_state=42)
        \end{lstlisting}

        \item \textbf{Create the Model}: Initialize the Linear Regression model.

        \begin{lstlisting}[language=Python]
model = LinearRegression()
        \end{lstlisting}

        \item \textbf{Train the Model}: Fit the model on the training data.

        \begin{lstlisting}[language=Python]
model.fit(X_train, y_train)
        \end{lstlisting}

        \item \textbf{Make Predictions}: Use the model to make predictions.

        \begin{lstlisting}[language=Python]
y_pred = model.predict(X_test)
        \end{lstlisting}

        \item \textbf{Evaluate the Model}: Assess performance using Mean Squared Error (MSE).

        \begin{lstlisting}[language=Python]
mse = mean_squared_error(y_test, y_pred)
print(f'Mean Squared Error: {mse}')
        \end{lstlisting}
    \end{enumerate}
\end{frame}

\begin{frame}[fragile]
    \frametitle{Interpreting Results}
    % Content goes here
    This slide covers how to interpret coefficients, intercepts, and predictions from a linear regression model.
\end{frame}

\begin{frame}[fragile]
    \frametitle{Understanding Linear Regression Output}
    \begin{itemize}
        \item Linear regression predicts a dependent variable (Y) from independent variables (X).
        \item Key outputs include:
        \begin{itemize}
            \item Coefficients
            \item Intercept
            \item Predictions
        \end{itemize}
    \end{itemize}
\end{frame}

\begin{frame}[fragile]
    \frametitle{Coefficients, Intercept, and Predictions}
    \begin{block}{1. Coefficients}
        \begin{itemize}
            \item **Definition**: Change in Y for a one-unit change in X.
            \item **Interpretation**: E.g., if $Coefficient_{X1} = 2.5$, then a 1 unit increase in $X1$ results in a 2.5 unit increase in $Y$.
            \item **Example**: Model: $Y = 3 + 2.5*X1 - 1.5*X2$
            \begin{itemize}
                \item $X1$: +2.5 (positive relationship)
                \item $X2$: -1.5 (negative relationship)
            \end{itemize}
        \end{itemize}
    \end{block}
    
    \begin{block}{2. Intercept}
        \begin{itemize}
            \item **Definition**: Predicted value of Y when all X = 0.
            \item **Example**: If intercept = 3, then $Y=3$ when $X1=0$ and $X2=0$.
        \end{itemize}
    \end{block}
    
    \begin{block}{3. Predictions}
        To predict Y, use:
        \begin{equation}
            \text{Predicted } Y = \text{Intercept} + (Coefficient_{1} \times X_{1}) + (Coefficient_{2} \times X_{2}) 
        \end{equation}
        \begin{itemize}
            \item **Example**: If $X1 = 4$ and $X2 = 3$:
            \begin{equation}
                \text{Predicted } Y = 3 + (2.5 \times 4) + (-1.5 \times 3) = 8.5 
            \end{equation}
        \end{itemize}
    \end{block}
\end{frame}

\begin{frame}[fragile]
    \frametitle{Key Points to Emphasize}
    \begin{itemize}
        \item **Significance of Coefficients**: Check p-values to determine relevance.
        \item **Direction of Relationships**: 
        \begin{itemize}
            \item Positive coefficients indicate direct relationships.
            \item Negative coefficients indicate inverse relationships.
        \end{itemize}
        \item **Contextual Interpretation**: Always consider context; implications can vary.
    \end{itemize}
    
    \begin{block}{Notes}
        Clarify assumptions and limitations in your reports. Visual aids can enhance understanding, though not included here.
    \end{block}
\end{frame}

\begin{frame}[fragile]
    \frametitle{Limitations of Linear Regression - Overview}
    \begin{block}{Overview}
        Linear regression is a powerful modeling tool for relationships between variables, 
        but it has inherent limitations that can affect accuracy and reliability. 
        Understanding these limitations is crucial for crafting better predictive models.
    \end{block}
\end{frame}

\begin{frame}[fragile]
    \frametitle{Limitations of Linear Regression - Key Limitations}
    \begin{enumerate}
        \item \textbf{Linearity Assumption}
            \begin{itemize}
                \item Assumes linear relationships between variables.
                \item Example: Quadratic relationships yield poor predictions.
            \end{itemize}
        
        \item \textbf{Sensitivity to Outliers}
            \begin{itemize}
                \item Outliers can skew the regression line.
                \item Example: A luxury property can mislead lower-priced house predictions.
            \end{itemize}

        \item \textbf{Homoscedasticity Assumption}
            \begin{itemize}
                \item Assumes constant error variance across independent variables.
                \item Example: Increasing errors with higher values hint at violation.
            \end{itemize}
    \end{enumerate}
\end{frame}

\begin{frame}[fragile]
    \frametitle{Limitations of Linear Regression - More Key Limitations}
    \begin{enumerate}
        \setcounter{enumi}{3} % Continue numbering from previous frame
        \item \textbf{Multicollinearity}
            \begin{itemize}
                \item High correlation between predictor variables complicates effects.
                \item Example: Including 'square footage' and 'number of bedrooms' can provide redundant info.
            \end{itemize}
        
        \item \textbf{Assumption of Independence}
            \begin{itemize}
                \item Predictors should be independent for accurate results.
                \item Example: Using 'age' and 'years of experience' can mislead interpretations.
            \end{itemize}
    \end{enumerate}
\end{frame}

\begin{frame}[fragile]
    \frametitle{Addressing Limitations}
    \begin{block}{Diagnosing and Addressing Limitations}
        \begin{itemize}
            \item \textbf{Multicollinearity Detection:} 
                Use Variance Inflation Factor (VIF). A VIF \textgreater 10 indicates high multicollinearity.
            
            \item \textbf{Residual Analysis:} 
                Plot residuals to check linearity, variance, and independence. Look for patterns indicating inadequacy.
        \end{itemize}
    \end{block}
    
    \begin{block}{Practical Considerations}
        - Perform exploratory data analysis (EDA) before fitting the model. \\
        - Regularly validate model performance using metrics like R-squared and adjusted R-squared.
    \end{block}
\end{frame}

\begin{frame}[fragile]
    \frametitle{Limitations of Linear Regression - Conclusion}
    \begin{block}{Conclusion}
        Recognizing the limitations of linear regression is key to its effective implementation. 
        By understanding constraints like non-linearity, outlier influence, and multicollinearity, 
        you can refine your model and enhance predictive power.
    \end{block}
    
    \begin{block}{Key Takeaway}
        Always assess the assumptions behind your linear regression model. 
        These assumptions are often violated in real-world data, affecting prediction reliability.
    \end{block}
\end{frame}

\begin{frame}[fragile]
    \frametitle{Applications of Linear Regression - Overview}
    \begin{block}{Understanding Linear Regression}
        Linear regression is a statistical method used for predicting a dependent variable (or target) from one or more independent variables (or features) based on a linear relationship.
    \end{block}
    The equation of a simple linear regression model is:
    \begin{equation}
        Y = \beta_0 + \beta_1X_1 + \epsilon
    \end{equation}
    Where:
    \begin{itemize}
        \item $Y$ = Dependent variable
        \item $\beta_0$ = Intercept
        \item $\beta_1$ = Slope indicating change in $Y$ for a one-unit change in $X_1$
        \item $X_1$ = Independent variable
        \item $\epsilon$ = Error term
    \end{itemize}
\end{frame}

\begin{frame}[fragile]
    \frametitle{Applications of Linear Regression - Real-World Use Cases}
    \begin{enumerate}
        \item \textbf{Economics}:
            \begin{itemize}
                \item \textit{House Price Prediction}: Predict housing prices based on factors such as size, number of bedrooms, and location.
                \item \textit{Salary Analysis}: Model the relationship between salary and factors like experience, education level, and occupation type.
            \end{itemize}
            \begin{equation}
                Price = 50,000 + 200 \times (Area)
            \end{equation}

        \item \textbf{Health}:
            \begin{itemize}
                \item \textit{Cancer Research}: Explore relationships between patient characteristics and treatment outcomes.
                \item \textit{Public Health Policies}: Correlate smoking rates with lung cancer incidence to develop health policies.
            \end{itemize}
            \begin{equation}
                Blood\ Pressure = 70 + 0.5 \times (BMI)
            \end{equation}

        \item \textbf{Marketing}:
            \begin{itemize}
                \item \textit{Sales Forecasting}: Project future sales based on marketing spending.
                \item \textit{Customer Satisfaction Analysis}: Analyze feedback scores against service attributes.
            \end{itemize}
            \begin{equation}
                Sales = 15,000 + 3 \times (Marketing\ Spend)
            \end{equation}
    \end{enumerate}
\end{frame}

\begin{frame}[fragile]
    \frametitle{Key Points and Conclusion}
    \begin{block}{Key Points to Emphasize}
        \begin{itemize}
            \item Linear regression is a foundational tool in predictive analytics, providing insights across multiple fields.
            \item Understanding coefficients ($\beta$ values) allows informed decision-making.
            \item Assumptions should be evaluated (linearity, independence, homoscedasticity, and normality of residuals).
        \end{itemize}
    \end{block}
    \begin{block}{Conclusion}
        Linear regression serves as a versatile method for understanding relationships between variables, playing a crucial role in fields such as economics, health, and marketing.
    \end{block}
\end{frame}

\begin{frame}[fragile]
    \frametitle{Comparison with Other Algorithms - Introduction}
    In this section, we will compare Linear Regression with two other popular supervised learning algorithms: 
    \begin{itemize}
        \item Decision Trees
        \item Neural Networks
    \end{itemize}
    Understanding their differences, advantages, and limitations can help in selecting the appropriate model for a given dataset.
\end{frame}

\begin{frame}[fragile]
    \frametitle{Linear Regression}
    \begin{block}{Concept}
        Linear regression models the relationship between a dependent variable (Y) and one or more independent variables (X). 
        It assumes a linear relationship, expressed as:
    \end{block}
    
    \begin{equation}
        Y = \beta_0 + \beta_1X_1 + \beta_2X_2 + ... + \beta_nX_n + \epsilon
    \end{equation}
    
    \begin{itemize}
        \item \textbf{Advantages:}
        \begin{itemize}
            \item Easy to implement and interpret
            \item Efficient for low-dimensional data
            \item Works well with linearly correlated data
        \end{itemize}
        
        \item \textbf{Limitations:}
        \begin{itemize}
            \item Sensitive to outliers
            \item Assumes a linear relationship, which may not always exist
        \end{itemize}
    \end{itemize}
    
    \textbf{Example:} Predicting house prices based on square footage, number of bedrooms, etc.
\end{frame}

\begin{frame}[fragile]
    \frametitle{Decision Trees}
    \begin{block}{Concept}
        Decision Trees split data into subsets based on the value of input features. Each internal node represents a feature, 
        the branches represent decision rules, and the leaves represent outcomes.
    \end{block}

    \begin{itemize}
        \item \textbf{Advantages:}
        \begin{itemize}
            \item Intuitive and transparent; easy to visualize
            \item Can handle non-linear relationships and interactions
            \item No need for data scaling or normalization
        \end{itemize}
        
        \item \textbf{Limitations:}
        \begin{itemize}
            \item Prone to overfitting, particularly with deep trees
            \item Sensitive to small variations in data
        \end{itemize}
    \end{itemize}
    
    \textbf{Example:} Classifying whether a patient has a disease based on age, blood pressure, and sugar levels.
\end{frame}

\begin{frame}[fragile]
    \frametitle{Neural Networks}
    \begin{block}{Concept}
        Neural Networks consist of interconnected layers (input, hidden, output) that transform input data into predictions 
        through weighted connections. They are capable of capturing highly complex relationships.
    \end{block}

    \begin{itemize}
        \item \textbf{Advantages:}
        \begin{itemize}
            \item Excellent for large datasets and intricate patterns
            \item Can model non-linear relationships effectively
            \item Flexible in architecture (e.g., deep learning)
        \end{itemize}
        
        \item \textbf{Limitations:}
        \begin{itemize}
            \item Requires a large amount of data to train effectively
            \item More computationally intensive and less interpretable
        \end{itemize}
    \end{itemize}
    
    \textbf{Example:} Image recognition tasks, such as identifying objects in pictures.
\end{frame}

\begin{frame}[fragile]
    \frametitle{Key Points to Emphasize}
    \begin{itemize}
        \item \textbf{Model Selection:} 
        The choice between Linear Regression, Decision Trees, and Neural Networks should depend on the specific problem, 
        the nature of the data, and project requirements.
        
        \item \textbf{Complexity vs. Interpretability:} 
        Linear regression is straightforward, while Decision Trees offer transparency, and Neural Networks provide high accuracy 
        at the cost of interpretability.
        
        \item \textbf{Performance Trade-offs:} 
        Assess performance across different evaluation metrics (e.g., accuracy, precision, recall) tailored to the project goals.
    \end{itemize}
\end{frame}

\begin{frame}[fragile]
    \frametitle{Diagram/Code Snippet}
    % Consider including a flowchart that visually compares the strengths and weaknesses of each algorithm.
    \textbf{Consider adding a visual aid here to summarize the comparison between the three algorithms.}
\end{frame}

\begin{frame}[fragile]{Ethics in Linear Regression Use - Overview}
    \begin{block}{Overview of Ethical Implications in Linear Regression}
        Linear regression, like any predictive modeling technique, has significant ethical implications. These implications stem from data handling, interpretation, and the potential impact of model predictions on individuals or communities. Understanding these ethical dimensions is crucial for responsible data science practices.
    \end{block}
\end{frame}

\begin{frame}[fragile]{Ethics in Linear Regression Use - Key Ethical Considerations}
    \begin{block}{Key Ethical Considerations}
        \begin{enumerate}
            \item \textbf{Transparency}  
                \begin{itemize}
                    \item Explanation: Models must be interpretable for stakeholders to understand decision-making.
                    \item Example: A company should disclose how factors like income and credit history influence eligibility.
                \end{itemize}
            \item \textbf{Bias and Fairness}  
                \begin{itemize}
                    \item Explanation: Models can amplify biases present in training data.
                    \item Example: Historical data reflecting racial discrimination may disadvantage certain groups.
                \end{itemize}
            \item \textbf{Data Privacy}  
                \begin{itemize}
                    \item Explanation: Sensitive data use raises privacy concerns and must comply with data protection laws.
                    \item Example: Avoid using personally identifiable information without consent.
                \end{itemize}
            \item \textbf{Misleading Conclusions} 
                \begin{itemize}
                    \item Explanation: Regression outputs can lead to erroneous conclusions.
                    \item Example: Overstating relationships may lead to poor policy decisions.
                \end{itemize}
            \item \textbf{Use of Data}  
                \begin{itemize}
                    \item Explanation: The purpose of model usage can raise ethical questions beyond accuracy.
                    \item Example: Predicting employee attrition must not lead to unfair practices.
                \end{itemize}
        \end{enumerate}
    \end{block}
\end{frame}

\begin{frame}[fragile]{Ethics in Linear Regression Use - Responsible Practices}
    \begin{block}{Responsible Use of Linear Regression}
        \begin{itemize}
            \item Implement \textbf{Fairness Audits}: Regularly check models for bias using fairness metrics.
            \item Educate \textbf{Stakeholders}: Train users about model limitations and responsible interpretation.
            \item Utilize \textbf{Robustness Checks}: Ensure conclusions hold across different datasets and scenarios.
        \end{itemize}
    \end{block}
\end{frame}

\begin{frame}[fragile]{Ethics in Linear Regression Use - Conclusion}
    \begin{block}{Conclusion}
        While linear regression is powerful in predictive analytics, ethical implications cannot be overlooked. Awareness and proactive measures will help mitigate risks associated with bias, privacy, and misinterpretation, promoting responsible and equitable decision-making through data.
    \end{block}
\end{frame}

\begin{frame}[fragile]{Key Formula in Context}
    \begin{block}{Key Formula for Linear Regression}
        The basic formula for linear regression is:
        \begin{equation}
            Y = \beta_0 + \beta_1X_1 + \beta_2X_2 + ... + \beta_nX_n + \epsilon 
        \end{equation}
        Where \(Y\) is the predicted outcome and \(X\) represents the input features. Understanding this formula is essential, but ethical considerations in how we choose, interpret, and implement our choices are equally critical.
    \end{block}
\end{frame}

\begin{frame}[fragile]
    \frametitle{Future Trends in Linear Regression}
    \begin{block}{Overview}
        Emerging trends and advancements in linear regression techniques and their applications.
    \end{block}
\end{frame}

\begin{frame}[fragile]
    \frametitle{Emerging Trends and Advancements - Part 1}
    
    \begin{itemize}
        \item Linear regression as a foundational technique is evolving rapidly.
        \item Key trends include:
            \begin{itemize}
                \item Integration with Machine Learning Frameworks
                \item Regularization Techniques
            \end{itemize}
    \end{itemize}
\end{frame}

\begin{frame}[fragile]
    \frametitle{Integration with Machine Learning Frameworks}
    \begin{itemize}
        \item Linear regression is integrated with advanced techniques such as ensemble methods and deep learning.
        \item \textbf{Example:} Used as a base learner in Random Forest models to enhance performance for linear relationships.
    \end{itemize}
\end{frame}

\begin{frame}[fragile]
    \frametitle{Regularization Techniques}
    \begin{itemize}
        \item Regularization adds a penalty term to the loss function to improve generalization.
        \item \textbf{Types:}
            \begin{itemize}
                \item Lasso Regression (L1 regularization): Shrinks coefficients to zero for feature selection.
                \item Ridge Regression (L2 regularization): Penalizes coefficients, keeping all features in the model.
            \end{itemize}
    \end{itemize}
    \begin{block}{Formulas}
        \begin{equation}
            J_{Lasso}(\theta) = \frac{1}{m} \sum_{i=1}^{m} (y_i - h_\theta(x_i))^2 + \lambda \sum_{j=1}^{n} |\theta_j|
        \end{equation}
        \begin{equation}
            J_{Ridge}(\theta) = \frac{1}{m} \sum_{i=1}^{m} (y_i - h_\theta(x_i))^2 + \lambda \sum_{j=1}^{n} \theta_j^2
        \end{equation}
    \end{block}
\end{frame}

\begin{frame}[fragile]
    \frametitle{Emerging Trends and Advancements - Part 2}
    
    \begin{itemize}
        \item Big Data and High-Dimensional Data Handling
            \begin{itemize}
                \item Adaptation for high-dimensional datasets.
                \item \textbf{Example:} Use of PCA for dimensionality reduction before regression.
            \end{itemize}
        \item Automated Machine Learning (AutoML)
            \begin{itemize}
                \item Automates model selection including linear regression.
                \item \textbf{Implications:} Democratizes access to data science.
            \end{itemize}
    \end{itemize}
\end{frame}

\begin{frame}[fragile]
    \frametitle{Real-Time and Streaming Data Applications}
    \begin{itemize}
        \item Development of regression models for real-time applications.
        \item \textbf{Example:} Dynamic updating of linear regression models using incremental learning.
    \end{itemize}
\end{frame}

\begin{frame}[fragile]
    \frametitle{Key Points and Conclusion}
    \begin{itemize}
        \item \textbf{Accessibility:} Linear regression remains critical despite advancements in machine learning.
        \item \textbf{Versatility:} Enhanced performance through combination with other techniques.
        \item \textbf{Future Focus:} Staying updated is essential for effective decision-making.
    \end{itemize}
    \begin{block}{Conclusion}
        The evolution of linear regression reflects its significance in analytics and understanding data relationships.
    \end{block}
\end{frame}

\begin{frame}[fragile]
    \frametitle{Case Study: Application of Linear Regression}
    \begin{block}{Introduction}
        Linear regression is a powerful statistical tool used in various fields to model the relationship between a dependent variable and one or more independent variables. 
        In this case study, we will explore how a retail company utilized linear regression to predict sales based on advertising spending.
    \end{block}
\end{frame}

\begin{frame}[fragile]
    \frametitle{Problem Statement and Data Overview}
    \begin{block}{Problem Statement}
        XYZ Retail Company aimed to understand how their advertising budget impacts sales and optimize future expenditures. They collected historical data over five years, tracking:
        \begin{itemize}
            \item Monthly sales
            \item Corresponding advertising spend across different media channels (TV, Radio, and Online)
        \end{itemize}
    \end{block}
    
    \begin{block}{Data Overview}
        \textbf{Dependent Variable (Target):} Monthly Sales (in dollars)\\
        \textbf{Independent Variables (Features):}
        \begin{itemize}
            \item TV Advertising Spend (in dollars)
            \item Radio Advertising Spend (in dollars)
            \item Online Advertising Spend (in dollars)
        \end{itemize}
        
        \textbf{Sample Data:}
        \begin{tabular}{|c|c|c|c|c|}
            \hline
            Month & TV Spend & Radio Spend & Online Spend & Sales \\
            \hline
            Jan & 3000 & 1500 & 2000 & 20000 \\
            Feb & 4000 & 1800 & 2500 & 25000 \\
            Mar & 5000 & 2000 & 3000 & 30000 \\
            \hline
        \end{tabular}
    \end{block}
\end{frame}

\begin{frame}[fragile]
    \frametitle{Model Development and Key Findings}
    \begin{block}{Step-by-step Approach}
        \begin{enumerate}
            \item Data Preparation: Clean the data, handle missing values/outliers, and split into training/testing sets.
            \item Model Development: Fit a linear regression model using:
            \begin{equation}
                Sales = \beta_0 + \beta_1 \times TV + \beta_2 \times Radio + \beta_3 \times Online
            \end{equation}
            \item Model Evaluation: Analyze using metrics like R-squared and Mean Squared Error (MSE).
            \item Making Predictions: Use the model to predict future sales based on hypothetical advertising budgets.
        \end{enumerate}
    \end{block}

    \begin{block}{Key Findings}
        - A positive correlation was found between advertising spend and sales.
        - TV advertising had the highest positive coefficient, indicating it was the most effective in driving sales.
    \end{block}
\end{frame}

\begin{frame}[fragile]
    \frametitle{Conclusion - Key Points}
    \begin{itemize}
        \item \textbf{Supervised Learning}: Algorithm learns from labeled datasets (input-output pairs).
        \item \textbf{Linear Regression}:
        \begin{itemize}
            \item Models relationship between dependent and independent variables.
            \item Fits a linear equation to observed data.
        \end{itemize}
        \item \textbf{Key Terminology}:
        \begin{itemize}
            \item \textbf{Dependent Variable (y)}: The variable predicted.
            \item \textbf{Independent Variable (x)}: The predictors.
            \item \textbf{Hypothesis Function}:
            \begin{equation}
                h(x) = \beta_0 + \beta_1 x_1 + \beta_2 x_2 + ... + \beta_n x_n
            \end{equation}
            \item \textbf{Cost Function}:
            \begin{equation}
                J(\beta) = \frac{1}{m} \sum_{i=1}^{m} (h(x^{(i)}) - y^{(i)})^2
            \end{equation}
            \item \textbf{Gradient Descent}: Optimization algorithm to minimize cost function.
        \end{itemize}
    \end{itemize}
\end{frame}

\begin{frame}[fragile]
    \frametitle{Linear Regression - Assumptions and Evaluation}
    \begin{itemize}
        \item \textbf{Assumptions of Linear Regression}:
        \begin{itemize}
            \item \textbf{Linearity}: Relationship should be linear.
            \item \textbf{Homoscedasticity}: Constant variance of errors.
            \item \textbf{Independence}: Observations are independent.
            \item \textbf{Normality}: Residuals should be normally distributed.
        \end{itemize}
        \item \textbf{Evaluation Metrics}:
        \begin{itemize}
            \item \textbf{R-squared}: Proportion of variance explained by predictors.
            \item \textbf{Mean Squared Error (MSE)}: Average squared difference between predicted and actual values.
        \end{itemize}
    \end{itemize}
\end{frame}

\begin{frame}[fragile]
    \frametitle{Example - Case Study}
    \begin{itemize}
        \item \textbf{Predicting House Prices}:
        \begin{itemize}
            \item Linear model fitted as:
            \begin{equation}
                \text{Price} = 50000 + 300 \times \text{Size} + 20000 \times \text{Location\_Index}
            \end{equation}
            \item Coefficients indicate expected change in price for unit change in predictors.
        \end{itemize}
    \end{itemize}
\end{frame}

\begin{frame}[fragile]
    \frametitle{Key Takeaways and Q\&A}
    \begin{itemize}
        \item \textbf{Key Takeaways}:
        \begin{itemize}
            \item Linear regression is fundamental in predictive modeling.
            \item Understanding assumptions is critical for effective model development.
            \item Use R-squared and MSE for performance evaluation.
        \end{itemize}
        \item \textbf{Open Floor for Questions}:
        \begin{itemize}
            \item Clarifications on concepts discussed.
            \item Specific applications of linear regression.
            \item Discussion on real-world implications and challenges.
        \end{itemize}
    \end{itemize}
\end{frame}


\end{document}