\documentclass[aspectratio=169]{beamer}

% Theme and Color Setup
\usetheme{Madrid}
\usecolortheme{whale}
\useinnertheme{rectangles}
\useoutertheme{miniframes}

% Additional Packages
\usepackage[utf8]{inputenc}
\usepackage[T1]{fontenc}
\usepackage{graphicx}
\usepackage{booktabs}
\usepackage{listings}
\usepackage{amsmath}
\usepackage{amssymb}
\usepackage{xcolor}
\usepackage{tikz}
\usepackage{pgfplots}
\pgfplotsset{compat=1.18}
\usetikzlibrary{positioning}
\usepackage{hyperref}

% Custom Colors
\definecolor{myblue}{RGB}{31, 73, 125}
\definecolor{mygray}{RGB}{100, 100, 100}
\definecolor{mygreen}{RGB}{0, 128, 0}
\definecolor{myorange}{RGB}{230, 126, 34}
\definecolor{mycodebackground}{RGB}{245, 245, 245}

% Set Theme Colors
\setbeamercolor{structure}{fg=myblue}
\setbeamercolor{frametitle}{fg=white, bg=myblue}
\setbeamercolor{title}{fg=myblue}
\setbeamercolor{section in toc}{fg=myblue}
\setbeamercolor{item projected}{fg=white, bg=myblue}
\setbeamercolor{block title}{bg=myblue!20, fg=myblue}
\setbeamercolor{block body}{bg=myblue!10}
\setbeamercolor{alerted text}{fg=myorange}

% Set Fonts
\setbeamerfont{title}{size=\Large, series=\bfseries}
\setbeamerfont{frametitle}{size=\large, series=\bfseries}
\setbeamerfont{caption}{size=\small}
\setbeamerfont{footnote}{size=\tiny}

% Footer and Navigation Setup
\setbeamertemplate{footline}{
  \leavevmode%
  \hbox{%
  \begin{beamercolorbox}[wd=.3\paperwidth,ht=2.25ex,dp=1ex,center]{author in head/foot}%
    \usebeamerfont{author in head/foot}\insertshortauthor
  \end{beamercolorbox}%
  \begin{beamercolorbox}[wd=.5\paperwidth,ht=2.25ex,dp=1ex,center]{title in head/foot}%
    \usebeamerfont{title in head/foot}\insertshorttitle
  \end{beamercolorbox}%
  \begin{beamercolorbox}[wd=.2\paperwidth,ht=2.25ex,dp=1ex,center]{date in head/foot}%
    \usebeamerfont{date in head/foot}
    \insertframenumber{} / \inserttotalframenumber
  \end{beamercolorbox}}%
  \vskip0pt%
}

% Turn off navigation symbols
\setbeamertemplate{navigation symbols}{}

% Title Page Information
\title[Q-learning and SARSA]{Chapter 3: Q-learning and SARSA}
\author[J. Smith]{John Smith, Ph.D.}
\institute[University Name]{
  Department of Computer Science\\
  University Name\\
  \vspace{0.3cm}
  Email: email@university.edu\\
  Website: www.university.edu
}
\date{\today}

% Document Start
\begin{document}

\frame{\titlepage}

\begin{frame}[fragile]
    \frametitle{Introduction to Q-learning and SARSA}
    \begin{block}{Overview}
        In this chapter, we will explore two fundamental algorithms in Reinforcement Learning: 
        \textbf{Q-learning} and \textbf{SARSA}. These algorithms are critical for learning optimal policies 
        to maximize cumulative rewards in an environment.
    \end{block}
\end{frame}

\begin{frame}[fragile]
    \frametitle{Key Concepts - Part 1}
    \begin{itemize}
        \item \textbf{Reinforcement Learning Basics}:
            \begin{itemize}
                \item An agent interacts with an environment through actions.
                \item Receives feedback in forms of rewards to update its knowledge.
                \item Objective: Learn a policy that maximizes expected rewards.
            \end{itemize}
        \item \textbf{Q-learning}:
            \begin{itemize}
                \item A model-free, off-policy algorithm.
                \item Learns the value of actions in a state, known as the \textbf{Q-value}.
            \end{itemize}
        \item \textbf{SARSA}:
            \begin{itemize}
                \item An on-policy algorithm that updates Q-values based on the agent's actions. 
            \end{itemize}
    \end{itemize}
\end{frame}

\begin{frame}[fragile]
    \frametitle{Key Comparisons}
    \begin{table}[ht]
        \centering
        \begin{tabular}{|c|c|c|}
            \hline
            \textbf{Feature} & \textbf{Q-learning} & \textbf{SARSA} \\
            \hline
            Policy Type & Off-policy & On-policy \\
            \hline
            Update Rule & Uses maximum Q-value & Uses the action taken \\
            \hline
            Exploration Strategy & More flexible & More conservative \\
            \hline
            Convergence & Guarantees under certain conditions & Converges at the current policy \\
            \hline
        \end{tabular}
        \caption{Comparison of Q-learning and SARSA}
    \end{table}
\end{frame}

\begin{frame}[fragile]
    \frametitle{Mathematical Notation}
    \begin{itemize}
        \item \textbf{Q-value Update Equation (Q-learning)}:
        \begin{equation}
            Q(s, a) \leftarrow Q(s, a) + \alpha \left( r + \gamma \max_{a'} Q(s', a') - Q(s, a) \right)
        \end{equation}
        
        \item \textbf{Q-value Update Equation (SARSA)}:
        \begin{equation}
            Q(s, a) \leftarrow Q(s, a) + \alpha \left( r + \gamma Q(s', a') - Q(s, a) \right)
        \end{equation}

        \item Where:
        \begin{itemize}
            \item $s$ = current state
            \item $a$ = action taken
            \item $r$ = reward received
            \item $s'$ = next state
            \item $a'$ = action taken in next state
            \item $\alpha$ = learning rate
            \item $\gamma$ = discount factor
        \end{itemize}
    \end{itemize}
\end{frame}

\begin{frame}[fragile]
    \frametitle{Overview of Q-learning and SARSA}
    \begin{block}{Key Concepts}
        Introduction to Reinforcement Learning (RL) strategies, focusing on Q-learning and SARSA as foundational algorithms. 
    \end{block}
\end{frame}

\begin{frame}[fragile]
    \frametitle{Q-learning}
    \begin{itemize}
        \item \textbf{Definition}: 
        Q-learning is an off-policy method that learns the value of the optimal policy regardless of the actions taken.
        \item \textbf{Q-value Update Rule}:
        \begin{equation}
        Q(s, a) \leftarrow Q(s, a) + \alpha \left( R + \gamma \max_a Q(s', a') - Q(s, a) \right)
        \end{equation}
        \begin{itemize}
            \item \( s \): current state
            \item \( a \): action taken
            \item \( R \): reward received after taking action \( a \)
            \item \( s' \): next state after taking action \( a \)
            \item \( \gamma \): discount factor
            \item \( \alpha \): learning rate
        \end{itemize}
        \item \textbf{Example}:
        In a grid world, reaching a goal state might update the Q-value to reflect a reward of +10.
    \end{itemize}
\end{frame}

\begin{frame}[fragile]
    \frametitle{SARSA: State-Action-Reward-State-Action}
    \begin{itemize}
        \item \textbf{Definition}: 
        SARSA is an on-policy method that updates Q-values based on the agent's actual actions, learning from the current policy.
        \item \textbf{Q-value Update Rule}:
        \begin{equation}
        Q(s, a) \leftarrow Q(s, a) + \alpha \left( R + \gamma Q(s', a') - Q(s, a) \right)
        \end{equation}
        \begin{itemize}
            \item \( a' \): next action taken in state \( s' \)
        \end{itemize}
        \item \textbf{Example}:
        Different actions taken in the grid world will affect Q-values, potentially leading to slower convergence.
    \end{itemize}
    
    \begin{block}{Comparison of Q-learning and SARSA}
        \begin{itemize}
            \item \textbf{Learning Type}:
            \begin{itemize}
                \item Q-learning: Off-policy
                \item SARSA: On-policy
            \end{itemize}
            \item \textbf{Exploration vs. Exploitation}: Both implement epsilon-greedy strategies.
        \end{itemize}
    \end{block}
\end{frame}

\begin{frame}[fragile]
    \frametitle{Conclusion - Chapter 3: Q-learning and SARSA}
    \begin{enumerate}
        \item Summary of reinforcement learning algorithms focusing on Q-learning and SARSA.
        \item Importance of understanding off-policy and on-policy methods.
        \item Practical implications and applications in various domains.
    \end{enumerate}
\end{frame}

\begin{frame}[fragile]
    \frametitle{Summary of Key Concepts}
    \begin{block}{Q-learning}
        \begin{itemize}
            \item Off-policy reinforcement learning algorithm.
            \item Learns optimal action value without any environmental model.
            \item Q-value update: 
            \[
            Q(s, a) \gets Q(s, a) + \alpha \left[ r + \gamma \max_{a'} Q(s', a') - Q(s, a) \right]
            \]
        \end{itemize}
    \end{block}
\end{frame}

\begin{frame}[fragile]
    \frametitle{Summary of Key Concepts (continued)}
    \begin{block}{SARSA (State-Action-Reward-State-Action)}
        \begin{itemize}
            \item On-policy reinforcement learning algorithm.
            \item Q-value updates based on the action taken in the next state:
            \[
            Q(s, a) \gets Q(s, a) + \alpha \left[ r + \gamma Q(s', a') - Q(s, a) \right]
            \]
        \end{itemize}
    \end{block}

    \begin{block}{Comparison}
        \begin{itemize}
            \item Q-learning: off-policy, learns optimal policy independently of agent’s behavior.
            \item SARSA: on-policy, learns from actions taken by current policy.
        \end{itemize}
    \end{block}
\end{frame}

\begin{frame}[fragile]
    \frametitle{Practical Implications and Key Points}
    \begin{itemize}
        \item Q-learning more effective in exploratory contexts.
        \item SARSA preferred in safety-critical environments.
        \item Both methods focus on maximizing cumulative rewards through interaction.
        \item Understanding exploration vs. exploitation is critical for algorithm effectiveness.
    \end{itemize}
\end{frame}

\begin{frame}[fragile]
    \frametitle{Application Example}
    \begin{block}{Game Playing}
        \begin{itemize}
            \item Q-learning: Suitable for complex games allowing aggressive exploration (e.g., chess).
            \item SARSA: Better for scenarios where safety matters (e.g., human-robot interactions).
        \end{itemize}
    \end{block}
\end{frame}


\end{document}