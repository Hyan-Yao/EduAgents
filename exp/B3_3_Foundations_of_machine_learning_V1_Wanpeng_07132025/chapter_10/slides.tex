\documentclass[aspectratio=169]{beamer}

% Theme and Color Setup
\usetheme{Madrid}
\usecolortheme{whale}
\useinnertheme{rectangles}
\useoutertheme{miniframes}

% Additional Packages
\usepackage[utf8]{inputenc}
\usepackage[T1]{fontenc}
\usepackage{graphicx}
\usepackage{booktabs}
\usepackage{listings}
\usepackage{amsmath}
\usepackage{amssymb}
\usepackage{xcolor}
\usepackage{tikz}
\usepackage{pgfplots}
\pgfplotsset{compat=1.18}
\usetikzlibrary{positioning}
\usepackage{hyperref}

% Custom Colors
\definecolor{myblue}{RGB}{31, 73, 125}
\definecolor{mygray}{RGB}{100, 100, 100}
\definecolor{mygreen}{RGB}{0, 128, 0}
\definecolor{myorange}{RGB}{230, 126, 34}
\definecolor{mycodebackground}{RGB}{245, 245, 245}

% Set Theme Colors
\setbeamercolor{structure}{fg=myblue}
\setbeamercolor{frametitle}{fg=white, bg=myblue}
\setbeamercolor{title}{fg=myblue}
\setbeamercolor{section in toc}{fg=myblue}
\setbeamercolor{item projected}{fg=white, bg=myblue}
\setbeamercolor{block title}{bg=myblue!20, fg=myblue}
\setbeamercolor{block body}{bg=myblue!10}
\setbeamercolor{alerted text}{fg=myorange}

% Set Fonts
\setbeamerfont{title}{size=\Large, series=\bfseries}
\setbeamerfont{frametitle}{size=\large, series=\bfseries}
\setbeamerfont{caption}{size=\small}
\setbeamerfont{footnote}{size=\tiny}

% Document Start
\begin{document}

\frame{\titlepage}

\begin{frame}[fragile]
    \titlepage
\end{frame}

\begin{frame}[fragile]
    \frametitle{Introduction to the Capstone Project}
    \begin{block}{Overview}
      The Capstone Project is a culminating experience that synthesizes your knowledge and skills acquired throughout your coursework. 
      It is not just a final task but a comprehensive opportunity to engage in practical, real-world applications of your learning.
    \end{block}
\end{frame}

\begin{frame}[fragile]
    \frametitle{Significance of the Capstone Project}
    \begin{itemize}
        \item \textbf{Holistic Learning}: Integrates various disciplines—each piece a different skill or concept.
        \item \textbf{Problem-Solving}: Face real-world problems that enhance critical and creative thinking.
        \item \textbf{Leadership and Collaboration}: Teamwork mirrors professional environments, building essential skills.
        \item \textbf{Showcase Your Skills}: Acts as a portfolio piece for job interviews.
    \end{itemize}
\end{frame}

\begin{frame}[fragile]
    \frametitle{Collective Learning Objectives}
    \begin{enumerate}
        \item \textbf{Integration of Knowledge}: Solve complex problems using concepts from various courses.
        \item \textbf{Research Skills}: Enhance data collection and analysis skills for drawing insights.
        \item \textbf{Technical Proficiency}: Develop essential technical skills for your chosen field.
        \item \textbf{Critical Thinking}: Analyze situations and adapt strategies based on feedback.
        \item \textbf{Presentation Abilities}: Present findings clearly and persuasively to diverse audiences.
    \end{enumerate}
\end{frame}

\begin{frame}[fragile]
    \frametitle{Key Points to Emphasize}
    \begin{itemize}
        \item \textbf{Real-World Impact}: Create meaningful change in various sectors.
        \item \textbf{Collaboration is Key}: Engage with peers, mentors, and professionals to build a network.
        \item \textbf{Iterative Process}: Embrace feedback and refine your project as necessary.
    \end{itemize}
\end{frame}

\begin{frame}[fragile]
    \frametitle{Inspirational Questions to Consider}
    \begin{itemize}
        \item What real-world challenges are you passionate about?
        \item How can the skills you’ve honed throughout your studies make a difference?
        \item In what innovative ways can you approach a familiar problem to see it anew?
    \end{itemize}
\end{frame}

\begin{frame}[fragile]
    \frametitle{Conclusion}
    By the end of this Capstone Project, you will gain a comprehensive understanding of the subject matter and a robust set of skills that are highly sought after in any professional field. Let's embark on this journey together!
\end{frame}

\begin{frame}[fragile]{Project Objectives - Overview}
    \begin{block}{Overview of Capstone Project Goals}
        The capstone project culminates the skills and knowledge acquired during the course. 
        The main objectives include:
    \end{block}
\end{frame}

\begin{frame}[fragile]{Project Objectives - Mastering Data Collection}
    \begin{block}{1. Mastering Data Collection}
        \textbf{Concept:} Understanding how to gather relevant data is crucial for any analysis or machine learning task.
        \begin{itemize}
            \item \textbf{Key Skills:} 
                \begin{itemize}
                    \item Identifying appropriate data sources (e.g., surveys, databases, web scraping)
                    \item Using tools like Python libraries (Pandas, Beautiful Soup) to collect and preprocess data.
                \end{itemize}
            \item \textbf{Example:} For a housing price prediction project, collect data from real estate websites, including features like location, size, and price.
        \end{itemize}
    \end{block}
\end{frame}

\begin{frame}[fragile]{Project Objectives - Model Training and Evaluation}
    \begin{block}{2. Model Training}
        \textbf{Concept:} After data collection, the next step is to develop a predictive model. 
        \begin{itemize}
            \item \textbf{Key Skills:}
                \begin{itemize}
                    \item Choosing the right algorithms (e.g., regression, decision trees)
                    \item Utilizing libraries like scikit-learn or TensorFlow.
                \end{itemize}
            \item \textbf{Example:} Train a linear regression model to predict housing prices based on selected features.
        \end{itemize}
    \end{block}

    \begin{block}{3. Evaluation of Models}
        \textbf{Concept:} Evaluating model effectiveness is essential for accurate predictions.
        \begin{itemize}
            \item \textbf{Key Skills:}
                \begin{itemize}
                    \item Applying metrics such as accuracy, precision, and F1-score.
                    \item Using validation techniques (e.g., train-test split).
                \end{itemize}
            \item \textbf{Example:} Use mean squared error to assess how well your model predicts housing prices.
        \end{itemize}
    \end{block}
\end{frame}

\begin{frame}[fragile]{Project Objectives - Key Points and Conclusion}
    \begin{block}{Key Points to Emphasize}
        \begin{itemize}
            \item \textbf{Integration of Skills:} The project integrates various skills from data collection to model training, emphasizing the full data science pipeline.
            \item \textbf{Real-world Application:} Simulates real-world scenarios, enhancing problem-solving and critical thinking.
            \item \textbf{Collaboration and Communication:} Engage with peers; collaboration is vital in a professional environment.
        \end{itemize}
    \end{block}

    \begin{block}{Conclusion}
        By the end of the project, you will have a comprehensive understanding of the data science lifecycle and practical experience for your future career.
    \end{block}
\end{frame}

\begin{frame}[fragile]
    \frametitle{Expectations and Criteria - Introduction}
    \begin{block}{Objective}
        Understanding the framework for evaluating your capstone project, including expectations, scoring criteria, and evaluation rubrics.
    \end{block}
\end{frame}

\begin{frame}[fragile]
    \frametitle{Expectations and Criteria - Project Expectations}
    \begin{enumerate}
        \item \textbf{Project Scope:}
        \begin{itemize}
            \item Define a clear problem statement and objectives.
            \item Conduct thorough background research to support your project.
            \item Implement a comprehensive plan outlining data sources, methods, and tools.
        \end{itemize}
        
        \item \textbf{Documentation:}
        \begin{itemize}
            \item Maintain a project journal to track progress and challenges.
            \item Prepare detailed documentation of methodologies, findings, and technologies used.
        \end{itemize}

        \item \textbf{Final Deliverable:}
        \begin{itemize}
            \item Submit a complete project report summarizing your process, results, and conclusions.
            \item Include visual aids like charts and graphs to communicate findings effectively.
        \end{itemize}
    \end{enumerate}
\end{frame}

\begin{frame}[fragile]
    \frametitle{Expectations and Criteria - Scoring Criteria}
    \begin{block}{Scoring Criteria}
        This section outlines how your project will be graded. Each criterion will be scored on a scale of 1 to 5:
    \end{block}
    
    \begin{itemize}
        \item \textbf{Clarity of Problem Statement (1-5 pts):} Is the problem well-defined and relevant?
        \item \textbf{Research Depth (1-5 pts):} Did you conduct comprehensive background research?
        \item \textbf{Methodological Rigor (1-5 pts):} Are the methods appropriately chosen and executed?
        \item \textbf{Project Execution (1-5 pts):} How well did you follow the proposed plan?
        \item \textbf{Quality of Presentation (1-5 pts):} Is your final report well-organized and visually appealing?
    \end{itemize}
    
    \begin{block}{Total Possible Points}
        25
    \end{block}
\end{frame}

\begin{frame}[fragile]
    \frametitle{Expectations and Criteria - Evaluation Rubric Samples}
    \begin{table}[htbp]
        \centering
        \begin{tabular}{|l|l|l|l|}
            \hline
            \textbf{Category} & \textbf{1 Point} & \textbf{3 Points} & \textbf{5 Points} \\ \hline
            Problem Statement & Vague and unclear & Defined but not focused & Clear and well-articulated \\ \hline
            Research Depth & Minimal resources used & Some relevant sources & Extensive and diverse sources \\ \hline
            Methodological Rigor & Lacks appropriate method & Basic methods applied & Advanced and sound methods \\ \hline
            Project Execution & Unorganized, chaotic & Some organization & Very organized, logical flow \\ \hline
            Quality of Presentation & Poorly formatted & Basic formatting & Professionally formatted \\ \hline
        \end{tabular}
    \end{table}
\end{frame}

\begin{frame}[fragile]
    \frametitle{Expectations and Criteria - Key Points}
    \begin{enumerate}
        \item \textbf{Be Proactive:} Start early, seek feedback, and document your journey.
        \item \textbf{Focus on Learning:} This project is an opportunity to apply your knowledge critically and creatively.
        \item \textbf{Engage with Your Audience:} Your presentation should showcase your project and explain its significance and impact.
    \end{enumerate}
\end{frame}

\begin{frame}[fragile]
    \frametitle{Expectations and Criteria - Support and Conclusion}
    \begin{block}{Support}
        Resources such as project management tools and academic databases will be available. Don't hesitate to reach out for guidance or clarification regarding expectations.
    \end{block}
    
    \begin{block}{Conclusion}
        The capstone project offers an opportunity to demonstrate your capabilities. By adhering to these criteria and focusing on quality, clarity, and research depth, you will successfully navigate this rewarding challenge.
    \end{block}
\end{frame}

\begin{frame}[fragile]
    \frametitle{Project Structure - Overview}
    \begin{block}{Understanding the Structure of Your Capstone Project}
        The success of your capstone project hinges on a well-defined structure. This slide outlines the primary milestones in your project journey, crucial for organizing your work and meeting deadlines effectively.
    \end{block}
\end{frame}

\begin{frame}[fragile]
    \frametitle{Project Structure - Key Milestones}
    \begin{enumerate}
        \item Project Proposal
        \item Progress Report
        \item Final Deliverable
    \end{enumerate}
\end{frame}

\begin{frame}[fragile]
    \frametitle{Project Proposal}
    \begin{itemize}
        \item \textbf{What Is It?} The foundational document that outlines your project idea, goals, and significance.
        \item \textbf{Elements to Include:}
        \begin{itemize}
            \item Problem Statement
            \item Objectives
            \item Methodology
        \end{itemize}
        \item \textbf{Example:} If your project is improving online education, your proposal might specify gaps in current environments, objectives, and resources.
    \end{itemize}
\end{frame}

\begin{frame}[fragile]
    \frametitle{Progress Report}
    \begin{itemize}
        \item \textbf{Purpose:} Update stakeholders on your project's status and any challenges faced.
        \item \textbf{Key Points to Cover:}
        \begin{itemize}
            \item Current Status
            \item Challenges
            \item Next Steps
        \end{itemize}
        \item \textbf{Example:} Report on coding phase completion and discuss debugging issues along with a testing timeline.
    \end{itemize}
\end{frame}

\begin{frame}[fragile]
    \frametitle{Final Deliverable}
    \begin{itemize}
        \item \textbf{Definition:} The culmination of your project, where you present your findings or final product.
        \item \textbf{Components:}
        \begin{itemize}
            \item Written Report
            \item Presentation
            \item Prototype/Demo (if applicable)
        \end{itemize}
        \item \textbf{Example:} For an environmental project, the final deliverable might involve a comprehensive report, a project presentation, and a public awareness toolkit.
    \end{itemize}
\end{frame}

\begin{frame}[fragile]
    \frametitle{Key Points to Remember}
    \begin{itemize}
        \item Each milestone serves as a checkpoint to assess your progress.
        \item Engage with feedback from peers and mentors.
        \item Adhere to timelines to stay organized and stress-free towards project conclusion.
    \end{itemize}
\end{frame}

\begin{frame}[fragile]
    \frametitle{Next Step}
    \begin{block}{Action Item}
        Once you have established your project structure, prepare to propose your project, effectively communicating your ideas and setting a solid foundation.
    \end{block}
\end{frame}

\begin{frame}[fragile]
    \frametitle{Conclusion}
    \begin{block}{Final Thoughts}
        By understanding and utilizing this structured approach, you can ensure a smoother project experience that not only meets but exceeds expectations.
    \end{block}
\end{frame}

\begin{frame}[fragile]
    \frametitle{Proposing a Project - Overview}
    \begin{block}{Overview of Project Proposal}
        Creating an effective project proposal is crucial for the success of your capstone project. It serves as a roadmap that outlines what you aim to achieve and how you will accomplish it. A well-structured proposal can greatly increase your chances of approval and support.
    \end{block}
\end{frame}

\begin{frame}[fragile]
    \frametitle{Proposing a Project - Key Components}
    \begin{itemize}
        \item \textbf{Identifying the Problem}
        \begin{itemize}
            \item Clearly articulate the issue to address.
            \item Questions to Consider:
            \begin{itemize}
                \item Why is this problem important?
                \item Who is affected by it?
                \item How does it relate to existing research?
            \end{itemize}
            \item Example: Defining the extent of plastic waste in urban areas.
        \end{itemize}

        \item \textbf{Data Sources}
        \begin{itemize}
            \item Identify types of data used for investigation.
            \begin{itemize}
                \item \textbf{Primary Sources:} Data collected firsthand.
                \item \textbf{Secondary Sources:} Existing data from reports or databases.
            \end{itemize}
            \item Example: Surveys and environmental agency reports for plastic waste project.
        \end{itemize}
    \end{itemize}
\end{frame}

\begin{frame}[fragile]
    \frametitle{Proposing a Project - Methodology and Key Points}
    \begin{itemize}
        \item \textbf{Methodology}
        \begin{itemize}
            \item Outline methods for data collection and analysis.
            \begin{itemize}
                \item How will you collect data? (qualitative vs quantitative)
                \item What analysis techniques will you use? 
            \end{itemize}
            \item Example: Mixed-methods approach with surveys and interviews.
        \end{itemize}
        
        \item \textbf{Key Points to Emphasize}
        \begin{itemize}
            \item \textbf{Clarity:} Ensure your problem statement is specific.
            \item \textbf{Feasibility:} Project should be realistic within the timeframe.
            \item \textbf{Impact:} Consider contributions to knowledge and community benefits.
        \end{itemize}
    \end{itemize}
    
    \begin{block}{Engagement Prompt}
        Reflect on how your interests can drive innovative project ideas. What real-world issues resonate with you?
    \end{block}
\end{frame}

\begin{frame}[fragile]
  \frametitle{Data Collection and Preparation - Importance of Data Quality}
  
  \begin{block}{Data Quality: The Foundation of Insights}
    High-quality data is crucial for accurate analysis and decision-making. 
    Poor data quality can lead to misleading results, incorrect conclusions, and wasted resources.
    A widely used definition of data quality includes dimensions such as:
    \begin{itemize}
      \item Accuracy
      \item Completeness
      \item Consistency
      \item Timeliness
      \item Relevance
    \end{itemize}
  \end{block}

  \begin{block}{Key Questions to Reflect On}
    \begin{itemize}
      \item What might happen if we rely on bad data for our project?
      \item How can we ensure our data accurately represents the real-world scenario we are studying?
    \end{itemize}
  \end{block}

\end{frame}

\begin{frame}[fragile]
  \frametitle{Data Collection and Preparation - Techniques for Data Cleaning}

  \begin{block}{Common Data Cleaning Techniques}
    \begin{enumerate}
      \item \textbf{Removing Duplicates:}
        Identify and remove any repeated records to ensure each entry is unique.
        \begin{itemize}
          \item \textit{Example:} In a customer database, two entries for the same customer may skew analysis on purchases.
        \end{itemize}
      
      \item \textbf{Handling Missing Values:}
        \begin{itemize}
          \item \textbf{Deletion:} Remove records with missing values if they are few.
          \item \textbf{Imputation:} Replace missing values with averages, medians, or mode.
            \begin{itemize}
              \item \textit{Example:} If a dataset has missing age values, replacing them with the average age can maintain dataset integrity.
            \end{itemize}
        \end{itemize}
      
      \item \textbf{Correcting Inconsistencies:}
        Standardize formats for dates, names, and other categorical data.
        \begin{itemize}
          \item \textit{Example:} Dates formatted as "DD/MM/YYYY" versus "MM/DD/YYYY" can cause confusion; choose one consistent format.
        \end{itemize}
      
      \item \textbf{Outlier Detection:}
        Identify and assess extreme values that could skew the analysis.
        \begin{itemize}
          \item \textit{Example:} A sales figure of \$1,000,000 when typical values are around \$10,000 might indicate an error.
        \end{itemize}
    \end{enumerate}
  \end{block}

\end{frame}

\begin{frame}[fragile]
  \frametitle{Data Collection and Preparation - Resource Allocation for Data Management}

  \begin{block}{Proper Resource Management is Essential}
    Allocate sufficient time and resources for data collection and cleaning. 
    This is often underestimated but is critical for a successful project.
  \end{block}

  \begin{itemize}
    \item \textbf{Human Resources:}
      Assign team members with a keen eye for detail, data analysts, and domain experts who understand the data context.
      
    \item \textbf{Tools \& Technology:}
      Utilize data management and cleaning tools such as:
      \begin{itemize}
        \item \textbf{Python Libraries:} Pandas for data manipulation, NumPy for handling numerical data.
        \item \textbf{Visualization tools:} Tableau or Power BI to visualize data for better understanding and quality assessment.
      \end{itemize}
  \end{itemize}

  \begin{block}{Key Points to Emphasize}
    \begin{itemize}
      \item Proactive Data Approach: The better you manage your data inputs, the more reliable your project outcomes will be.
      \item Iterative Process: Data cleaning is not a one-time task; it involves multiple rounds as new insights and issues emerge.
      \item Engagement with Stakeholders: Communicate with stakeholders to understand data requirements and expectations.
    \end{itemize}
  \end{block}

\end{frame}

\begin{frame}[fragile]
    \frametitle{Model Development}
    \begin{block}{Overview}
        Model development is the process of creating algorithms that learn from data 
        to make predictions or decisions. This section outlines the steps and tools 
        for building machine learning models in an accessible way.
    \end{block}
\end{frame}

\begin{frame}[fragile]
    \frametitle{Steps in Model Development}
    \begin{enumerate}
        \item \textbf{Define the Problem}
        \begin{itemize}
            \item Example: Predicting house prices from various features.
            \item Clearly define the question to guide the project.
        \end{itemize}

        \item \textbf{Select Tools and Frameworks}
        \begin{itemize}
            \item Accessible tools: 
            \begin{itemize}
                \item Python with Scikit-learn
                \item TensorFlow and PyTorch
                \item AutoML tools like H2O.ai and Google Cloud AutoML
            \end{itemize}
            \item Consider using Jupyter Notebooks or Google Colab.
        \end{itemize}
        
        \item \textbf{Choose Model Type}
        \begin{itemize}
            \item Classification, Regression, and Deep Learning models.
        \end{itemize}
    \end{enumerate}
\end{frame}

\begin{frame}[fragile]
    \frametitle{Model Training and Evaluation}
    \begin{enumerate}[resume]
        \item \textbf{Train the Model}
        \begin{block}{Example Code}
            \begin{lstlisting}[language=Python]
from sklearn.model_selection import train_test_split
from sklearn.linear_model import LinearRegression

# Example Data
X, y = ... # Features and labels
X_train, X_test, y_train, y_test = train_test_split(X, y, test_size=0.2, random_state=42)

# Model Creation
model = LinearRegression()
model.fit(X_train, y_train)
            \end{lstlisting}
        \end{block}
        
        \item \textbf{Evaluate the Model}
        \begin{itemize}
            \item Metrics: Accuracy, Precision, Recall, and F1 Score.
            \item Always assess model performance.
        \end{itemize}

        \item \textbf{Iterate the Process}
        \begin{itemize}
            \item Use evaluation results to refine models and improve performance.
        \end{itemize}
    \end{enumerate}
\end{frame}

\begin{frame}[fragile]
    \frametitle{Evaluation and Presentation - Part 1}
    \textbf{Understanding Model Evaluation}
    \begin{itemize}
        \item Evaluating model performance is crucial for real-world applications.
        \item Evaluation helps understand how well a model predicts outcomes from new data.
    \end{itemize}
\end{frame}

\begin{frame}[fragile]
    \frametitle{Key Performance Metrics - Part 2}
    \begin{enumerate}
        \item \textbf{Accuracy:}
            \begin{itemize}
                \item Definition: Percentage of correct predictions.
                \item Example: 80 correct out of 100 predictions = 80\%.
            \end{itemize}
        \item \textbf{Precision and Recall:}
            \begin{itemize}
                \item Precision: Correct positive predictions vs. all predicted positives.
                \item Recall: Correctly predicted positives vs. all actual positives.
            \end{itemize}
        \item \textbf{F1 Score:}
            \begin{equation}
                \text{F1 Score} = 2 \times \frac{\text{Precision} \times \text{Recall}}{\text{Precision} + \text{Recall}} 
            \end{equation}
        \item \textbf{Confusion Matrix:}
            \begin{itemize}
                \item Table illustrating true positives, false positives, true negatives, false negatives.
                \item Example layout:
                \begin{center}
                    \begin{tabular}{|c|c|c|}
                        \hline
                        & \text{Predicted Positive} & \text{Predicted Negative} \\
                        \hline
                        \text{Actual Positive} & TP & FN \\
                        \hline
                        \text{Actual Negative} & FP & TN \\
                        \hline
                    \end{tabular}
                \end{center}
            \end{itemize}
    \end{enumerate}
\end{frame}

\begin{frame}[fragile]
    \frametitle{Preparing Your Presentation - Part 3}
    \textbf{Key Points for Effective Presentations}
    \begin{itemize}
        \item \textbf{Know Your Audience}: Tailor content to their knowledge level and interests.
        \item \textbf{Visual Aids}: Use graphs/charts to visualize performance metrics.
        \item \textbf{Tell a Story}: Frame your work as a narrative of discovery.
        \item \textbf{Engage Your Audience}: Encourage questions and discussions.
        \item \textbf{Practice}: Rehearse for smooth delivery and anticipate questions.
    \end{itemize}
    
    \textbf{Conclusion:}
    \begin{itemize}
        \item Effective evaluation and presentation impact audience perception.
        \item Clear metrics and engaging presentations reflect analytical skills and inspire understanding of your work.
    \end{itemize}
\end{frame}

\begin{frame}[fragile]
    \frametitle{Interdisciplinary Applications - Introduction}
    \begin{block}{Overview}
        The capstone project provides an invaluable opportunity to apply machine learning (ML) concepts across various disciplines. 
        Understanding how machine learning techniques operate within different domains can illuminate their potential impact.
    \end{block}
\end{frame}

\begin{frame}[fragile]
    \frametitle{Interdisciplinary Applications - Key Fields}
    \begin{block}{Key Fields of Application}
        \begin{enumerate}
            \item \textbf{Healthcare}
            \item \textbf{Finance}
            \item \textbf{Marketing}
        \end{enumerate}
    \end{block}
\end{frame}

\begin{frame}[fragile]
    \frametitle{Interdisciplinary Applications - Healthcare Example}
    \begin{itemize}
        \item \textbf{Predictive Analytics}:
            \begin{itemize}
                \item ML algorithms analyze patient data to forecast diseases.
                \item Example: Logistic regression predicting diabetes risk based on BMI, age, and family history.
            \end{itemize}
        \item \textbf{Medical Imaging}:
            \begin{itemize}
                \item CNNs enhance image analysis, identifying tumors in MRI scans.
                \item Illustration: A CNN trained on labeled images aids radiologists by flagging abnormalities.
            \end{itemize}
    \end{itemize}
\end{frame}

\begin{frame}[fragile]
    \frametitle{Interdisciplinary Applications - Finance and Marketing}
    \begin{itemize}
        \item \textbf{Finance}:
            \begin{itemize}
                \item \textbf{Fraud Detection}:
                    \begin{itemize}
                        \item ML models identify anomalies in transaction patterns.
                        \item Example: Decision trees classify transactions as "safe" or "suspicious."
                    \end{itemize}
                \item \textbf{Algorithmic Trading}:
                    \begin{itemize}
                        \item Predict stock price movements for swift trading decisions.
                        \item Illustration: Time series forecasting aids in projecting trends.
                    \end{itemize}
            \end{itemize}
        
        \item \textbf{Marketing}:
            \begin{itemize}
                \item \textbf{Customer Segmentation}:
                    \begin{itemize}
                        \item ML categorizes customers for targeted marketing.
                        \item Example: K-means clustering for personalized email campaigns.
                    \end{itemize}
                \item \textbf{Sentiment Analysis}:
                    \begin{itemize}
                        \item NLP evaluates customer feedback to gauge sentiment.
                        \item Illustration: A sentiment model classifies comments as "positive," "neutral," or "negative."
                    \end{itemize}
            \end{itemize}
    \end{itemize}
\end{frame}

\begin{frame}[fragile]
    \frametitle{Key Points to Emphasize}
    \begin{itemize}
        \item Integration of Knowledge: Combining ML skills with domain knowledge leads to impactful outcomes.
        \item Problem-Solving: Identify real-world problems and apply ML solutions for change.
        \item Ethical Considerations: Consider ethical implications, especially in sensitive fields like healthcare and finance.
    \end{itemize}
\end{frame}

\begin{frame}[fragile]
    \frametitle{Engaging Questions}
    \begin{itemize}
        \item How might your chosen field benefit from predictive modeling techniques?
        \item What ethical considerations must you keep in mind while applying ML methods?
        \item Can you think of a project idea that merges your passion with ML to create a positive social impact?
    \end{itemize}
\end{frame}

\begin{frame}[fragile]
    \frametitle{Resources and Support - Overview}
    As you engage with your Capstone Project, an array of resources and support mechanisms will be at your disposal. These resources will enhance your learning experience and facilitate smoother project execution. Here are the key resources available:
\end{frame}

\begin{frame}[fragile]
    \frametitle{Resources and Support - Faculty Support}
    \begin{itemize}
        \item \textbf{One-on-One Guidance}: Schedule meetings with faculty for personalized advice and feedback.
        \item \textbf{Weekly Office Hours}: Utilize office hours for quick questions or deeper discussions.
        \item \textbf{Peer Mentoring}: Collaborate with fellow students and form study groups or project teams.
    \end{itemize}
\end{frame}

\begin{frame}[fragile]
    \frametitle{Resources and Support - Online Tutorials}
    \begin{itemize}
        \item \textbf{Learning Platforms}: Use Coursera, edX, or Khan Academy for courses relevant to your project.
        \item \textbf{Webinars and Workshops}: Participate in focused skills sessions, such as data analysis or programming.
        \item \textbf{Resource Libraries}: Access digital libraries to find research papers and articles for inspiration.
    \end{itemize}
\end{frame}

\begin{frame}[fragile]
    \frametitle{Resources and Support - Feedback Mechanisms}
    \begin{itemize}
        \item \textbf{Draft Reviews}: Submit drafts for constructive feedback from faculty or peers.
        \item \textbf{Midpoint Check-ins}: Schedule presentations to showcase your progress and refine your approach.
        \item \textbf{Anonymous Surveys}: Provide feedback about resources and suggestions for future improvements.
    \end{itemize}
\end{frame}

\begin{frame}[fragile]
    \frametitle{Resources and Support - Key Points}
    \begin{block}{Key Points to Emphasize}
        \begin{itemize}
            \item Engage with available resources early to enhance project quality.
            \item Seek support proactively from faculty and peers.
            \item Embrace feedback as a tool for growth and improvement.
        \end{itemize}
    \end{block}
\end{frame}

\begin{frame}[fragile]
    \frametitle{Resources and Support - Conclusion}
    The success of your Capstone Project relies heavily on the support and resources available to you. Embrace them to enhance understanding, expand skills, and produce a noteworthy project across various fields such as healthcare, finance, and marketing.
\end{frame}


\end{document}