\documentclass[aspectratio=169]{beamer}

% Theme and Color Setup
\usetheme{Madrid}
\usecolortheme{whale}
\useinnertheme{rectangles}
\useoutertheme{miniframes}

% Additional Packages
\usepackage[utf8]{inputenc}
\usepackage[T1]{fontenc}
\usepackage{graphicx}
\usepackage{booktabs}
\usepackage{listings}
\usepackage{amsmath}
\usepackage{amssymb}
\usepackage{xcolor}
\usepackage{tikz}
\usepackage{pgfplots}
\pgfplotsset{compat=1.18}
\usetikzlibrary{positioning}
\usepackage{hyperref}

% Custom Colors
\definecolor{myblue}{RGB}{31, 73, 125}
\definecolor{mygray}{RGB}{100, 100, 100}
\definecolor{mygreen}{RGB}{0, 128, 0}
\definecolor{myorange}{RGB}{230, 126, 34}
\definecolor{mycodebackground}{RGB}{245, 245, 245}

% Set Theme Colors
\setbeamercolor{structure}{fg=myblue}
\setbeamercolor{frametitle}{fg=white, bg=myblue}
\setbeamercolor{title}{fg=myblue}
\setbeamercolor{section in toc}{fg=myblue}
\setbeamercolor{item projected}{fg=white, bg=myblue}
\setbeamercolor{block title}{bg=myblue!20, fg=myblue}
\setbeamercolor{block body}{bg=myblue!10}
\setbeamercolor{alerted text}{fg=myorange}

% Set Fonts
\setbeamerfont{title}{size=\Large, series=\bfseries}
\setbeamerfont{frametitle}{size=\large, series=\bfseries}
\setbeamerfont{caption}{size=\small}
\setbeamerfont{footnote}{size=\tiny}

% Document Start
\begin{document}

\frame{\titlepage}

\begin{frame}[fragile]
    \frametitle{Introduction to Model Evaluation}
    \begin{block}{Why Evaluate Models?}
        Evaluating models is a critical step in machine learning, essential for determining how well a model performs in real-world situations.
    \end{block}
\end{frame}

\begin{frame}[fragile]
    \frametitle{Key Reasons for Model Evaluation}
    \begin{enumerate}
        \item \textbf{Performance Understanding}
        \begin{itemize}
            \item Evaluation assesses how well a model performs on unseen data.
            \item \textbf{Example:} A spam filter detects 95\% of spam in training but falters in real deployment.
        \end{itemize}
        
        \item \textbf{Model Comparison}
        \begin{itemize}
            \item Allows comparison of various models using consistent metrics.
            \item \textbf{Example:} Evaluate two house price prediction models on the same validation set.
        \end{itemize}
        
        \item \textbf{Identifying Overfitting}
        \begin{itemize}
            \item Shows if a model has learned training data too well, affecting generalization.
            \item \textbf{Example:} A model achieves 99\% accuracy on training but only 60\% on validation.
        \end{itemize}
        
        \item \textbf{Continuous Improvement}
        \begin{itemize}
            \item Regular evaluation reveals areas that need enhancement for better model performance.
            \item \textbf{Example:} Targeted retraining after identifying poor performance in a specific demographic.
        \end{itemize}
    \end{enumerate}
\end{frame}

\begin{frame}[fragile]
    \frametitle{Impact of Good Evaluation on Model Performance}
    \begin{itemize}
        \item \textbf{Better Decision Making:} Informed choices about model deployment.
        \item \textbf{Resource Efficiency:} Focuses effort on models that provide real value.
        \item \textbf{User Trust:} Increased reliability and confidence in model performance through rigorous evaluation.
    \end{itemize}
    
    \begin{block}{Key Points to Remember}
        - Evaluation is vital for application, comparison, and enhancement.
        - Good evaluation techniques elevate model trustworthiness.
    \end{block}
\end{frame}

\begin{frame}[fragile]
    \frametitle{Closing Thoughts}
    \begin{block}{Discussion Questions}
        \begin{itemize}
            \item Why is understanding your model's performance equally important as the creative aspects of building it?
            \item How can effective evaluation practices enhance your machine learning approach?
        \end{itemize}
    \end{block}
    This slide sets the stage for discussions on performance metrics and quantifying model effectiveness.
\end{frame}

\begin{frame}[fragile]
    \frametitle{Performance Metrics Overview}
    \begin{itemize}
        \item Evaluating model performance is crucial for reliable results.
        \item Key performance metrics:
        \begin{itemize}
            \item \textbf{Accuracy}
            \item \textbf{Precision}
            \item \textbf{Recall}
        \end{itemize}
        \item Understanding these metrics helps tailor models to specific needs.
    \end{itemize}
\end{frame}

\begin{frame}[fragile]
    \frametitle{1. Accuracy}
    \begin{block}{Definition}
        Accuracy measures the proportion of correctly predicted instances.
    \end{block}
    \begin{equation}
        \text{Accuracy} = \frac{\text{True Positives} + \text{True Negatives}}{\text{Total Instances}}
    \end{equation}
    \begin{itemize}
        \item \textbf{Example}: A weather model with 90 out of 100 correct forecasts has 90\% accuracy.
        \item \textbf{Important Note}: In imbalanced datasets, accuracy can be misleading.
    \end{itemize}
\end{frame}

\begin{frame}[fragile]
    \frametitle{2. Precision and 3. Recall}
    \begin{block}{Precision}
        \begin{itemize}
            \item Proportion of true positives among all predicted positives.
            \item \[
            \text{Precision} = \frac{\text{True Positives}}{\text{True Positives} + \text{False Positives}}
            \]
            \item \textbf{Example}: In a medical test, if 80 out of 100 positives are correct, precision is 80\%.
        \end{itemize}
    \end{block}

    \begin{block}{Recall}
        \begin{itemize}
            \item Measures the proportion of actual positives correctly identified.
            \item \[
            \text{Recall} = \frac{\text{True Positives}}{\text{True Positives} + \text{False Negatives}}
            \]
            \item \textbf{Example}: If there are 100 actual cases, and the test finds 80, recall is 80\%.
        \end{itemize}
    \end{block}
\end{frame}

\begin{frame}[fragile]
    \frametitle{Key Points to Emphasize}
    \begin{itemize}
        \item Model evaluation is essential for understanding strengths and weaknesses.
        \item Trade-offs exist: improving one metric may degrade another.
        \item Context matters: The significance of these metrics varies by application.
    \end{itemize}
\end{frame}

\begin{frame}[fragile]
    \frametitle{Discussion Questions}
    \begin{itemize}
        \item How would the importance of these metrics change in life-or-death scenarios?
        \item What strategies could balance precision and recall in models?
    \end{itemize}
\end{frame}

\begin{frame}[fragile]
    \frametitle{Accuracy - Definition and Formula}
    \begin{block}{Definition of Accuracy}
        Accuracy is a fundamental metric used to evaluate the performance of classification models. 
        It is defined as the ratio of correctly predicted instances to the total instances in the dataset. 
        In simpler terms, it measures how often the model is correct.
    \end{block}
    
    \begin{block}{Formula for Accuracy}
        The formula to calculate accuracy is:
        \begin{equation}
            \text{Accuracy} = \frac{\text{Number of Correct Predictions}}{\text{Total Number of Predictions}} \times 100\%
        \end{equation}
    \end{block}
\end{frame}

\begin{frame}[fragile]
    \frametitle{Accuracy - Significance in Model Evaluation}
    \begin{itemize}
        \item \textbf{Overall Performance Indicator:} 
            Accuracy provides a quick measure of a model's performance across all classes.
        \item \textbf{Baseline for Comparison:} 
            It serves as a baseline to compare with other performance metrics such as precision or recall.
        \item \textbf{Easy to Interpret:} 
            A straightforward percentage figure makes it easy for stakeholders to understand model performance.
    \end{itemize}
\end{frame}

\begin{frame}[fragile]
    \frametitle{Accuracy - When It Might Be Misleading}
    \begin{itemize}
        \item \textbf{Imbalanced Datasets:} 
            A model could achieve high accuracy by predicting the majority class. 
            \begin{itemize}
                \item Example: In a dataset with 90\% Class A and 10\% Class B, predicting all as Class A yields 90\% accuracy.
            \end{itemize}
        
        \item \textbf{Complex Patterns:} 
            High test accuracy may not reflect real-world performance, especially with complex decision boundaries.
        
        \item \textbf{Cost of Errors:} 
            Different types of errors can have different costs, making pure accuracy an insufficient metric.
            \begin{itemize}
                \item Example: In medical diagnosis, a false negative may be more harmful than a false positive.
            \end{itemize}
    \end{itemize}
\end{frame}

\begin{frame}[fragile]
    \frametitle{Precision - Definition}
    \begin{itemize}
        \item **Precision** measures the accuracy of positive predictions in a classification model.
        \item It reflects the proportion of true positive results compared to all positive predictions made.
    \end{itemize}
\end{frame}

\begin{frame}[fragile]
    \frametitle{Precision - Formula}
    Precision is calculated using the following formula:
    \begin{equation}
        \text{Precision} = \frac{\text{True Positives (TP)}}{\text{True Positives (TP)} + \text{False Positives (FP)}}
    \end{equation}
    \begin{itemize}
        \item **True Positives (TP)**: The number of correct positive predictions.
        \item **False Positives (FP)**: The number of incorrect positive predictions (predicted positive but actually negative).
    \end{itemize}
\end{frame}

\begin{frame}[fragile]
    \frametitle{Precision - Importance}
    \begin{itemize}
        \item Precision is critical in scenarios where:
            \begin{itemize}
                \item **False Positives Have Significant Consequences**: Such as in medical screenings.
                \item **Imbalance in Class Distribution**: In datasets with rare classes (e.g., fraud detection).
            \end{itemize}
    \end{itemize}
\end{frame}

\begin{frame}[fragile]
    \frametitle{Precision - Examples of Application}
    \begin{enumerate}
        \item **Medical Diagnostics**:
            \begin{itemize}
                \item Example: In cancer detection, high precision is crucial to avoid misdiagnosis.
                \item For instance, if a test gives 80 true positives and 20 false positives, then:
                \begin{equation}
                    \text{Precision} = \frac{80}{80 + 20} = 0.80 \text{ or } 80\%
                \end{equation}
            \end{itemize}

        \item **Email Spam Detection**:
            \begin{itemize}
                \item High precision avoids falsely labeling legitimate emails as spam.
                \item Example: If 90 emails are classified as spam but 10 are regular, then:
                \begin{equation}
                    \text{Precision} = \frac{90}{90 + 10} = 0.90 \text{ or } 90\%
                \end{equation}
            \end{itemize}

        \item **Image Recognition**:
            \begin{itemize}
                \item High precision is essential in tasks like pedestrian detection in self-driving cars.
            \end{itemize}
    \end{enumerate}
\end{frame}

\begin{frame}[fragile]
    \frametitle{Key Points to Emphasize}
    \begin{itemize}
        \item Precision focuses solely on the accuracy of positive predictions.
        \item It is a crucial metric in imbalanced datasets and scenarios where false positives have serious repercussions.
        \item Precision should be considered alongside other metrics like recall for a comprehensive evaluation of model performance.
    \end{itemize}
\end{frame}

\begin{frame}[fragile]
    \frametitle{Recall - Definition and Formula}
    \begin{itemize}
        \item \textbf{Definition}:
        Recall, also known as sensitivity or true positive rate, is a metric used to assess the performance of a classification model. It measures the proportion of actual positive cases that are correctly identified by the model.
        
        \item \textbf{Formula}:
        \begin{equation}
            \text{Recall} = \frac{\text{True Positives (TP)}}{\text{True Positives (TP)} + \text{False Negatives (FN)}}
        \end{equation}
        \begin{itemize}
            \item \textbf{True Positives (TP)}: The number of positive cases correctly predicted by the model.
            \item \textbf{False Negatives (FN)}: The number of positive cases that were incorrectly predicted as negative.
        \end{itemize}
    \end{itemize}
\end{frame}

\begin{frame}[fragile]
    \frametitle{Recall - Significance and Scenarios}
    \begin{itemize}
        \item \textbf{Significance}:
        \begin{itemize}
            \item Recall is particularly crucial in scenarios where it is essential to identify all positive cases, sometimes at the cost of increasing false positives.
            \item It often involves a trade-off with precision; maximizing recall may reduce precision and vice versa.
        \end{itemize}
        
        \item \textbf{Scenarios Where Maximizing Recall is Crucial}:
        \begin{enumerate}
            \item \textbf{Medical Diagnoses}
            \item \textbf{Fraud Detection}
            \item \textbf{Search and Rescue Operations}
        \end{enumerate}
    \end{itemize}
\end{frame}

\begin{frame}[fragile]
    \frametitle{Recall - Medical Example}
    \begin{itemize}
        \item \textbf{Medical Diagnoses}:
        \begin{itemize}
            \item In diseases such as cancer, it is critical to ensure that as many true cases as possible are identified. 
            \item For example, a cancer screening model that identifies 90 out of 100 actual cancer patients would have a recall of:
            \begin{equation}
                \text{Recall} = \frac{90}{90 + 10} = 0.9 \text{ or } 90\%
            \end{equation}
            \item This means few cases are missed, crucial for timely intervention.
        \end{itemize}
        
        \item \textbf{Key Takeaways}:
        \begin{itemize}
            \item Recall is vital in high-stakes fields such as healthcare, finance, and safety-related applications.
            \item Optimizing for recall necessitates accepting a higher rate of false positives, requiring a balance between precision and recall.
        \end{itemize}
    \end{itemize}
\end{frame}

\begin{frame}[fragile]
    \frametitle{F1 Score - Introduction}
    \begin{block}{What is the F1 Score?}
        The F1 Score is a performance metric used to evaluate classification models, especially in scenarios with imbalanced classes.
        It combines **Precision** and **Recall** into a single score, providing a balanced measure of the model's effectiveness.
    \end{block}
\end{frame}

\begin{frame}[fragile]
    \frametitle{F1 Score - Precision and Recall}
    \begin{itemize}
        \item **Precision**: Proportion of true positive predictions among all positive predictions.
        \item **Recall**: Proportion of true positive predictions among all actual positive instances.
    \end{itemize}
    \begin{block}{Why Use the F1 Score?}
        The F1 Score addresses situations where both false positives and false negatives are important by providing a balanced evaluation:
        \[
        \text{F1 Score} = 2 \times \frac{\text{Precision} \times \text{Recall}}{\text{Precision} + \text{Recall}}
        \]
    \end{block}
\end{frame}

\begin{frame}[fragile]
    \frametitle{F1 Score - Key Points}
    \begin{itemize}
        \item **Balanced Metric**: Useful when the balance between precision and recall matters, such as in medical diagnoses.
        \item **Interpretation**: 
            \begin{itemize}
                \item 1 indicates perfect precision and recall.
                \item 0 indicates failure in identifying relevant instances.
            \end{itemize}
        \item **Use Cases**: 
            \begin{itemize}
                \item Ideal for imbalanced datasets.
                \item Critical in industries like healthcare, risk assessment, and fraud detection.
            \end{itemize}
    \end{itemize}
\end{frame}

\begin{frame}[fragile]
    \frametitle{F1 Score - Example Calculation}
    \begin{block}{Calculation Scenario}
        Consider a model predicting a rare disease:
        \begin{itemize}
            \item Positive Predictions: 80 patients predicted as having the disease.
            \item True Positives (TP): 60, False Positives (FP): 20, False Negatives (FN): 40.
        \end{itemize}
    \end{block}
    \begin{itemize}
        \item **Precision**: 
        \[
        \text{Precision} = \frac{\text{TP}}{\text{TP} + \text{FP}} = \frac{60}{60 + 20} = 0.75
        \]
        \item **Recall**: 
        \[
        \text{Recall} = \frac{\text{TP}}{\text{TP} + \text{FN}} = \frac{60}{60 + 40} = 0.60
        \]
    \end{itemize}
    \begin{block}{F1 Score Calculation}
        \[
        \text{F1 Score} = 2 \times \frac{0.75 \times 0.60}{0.75 + 0.60} \approx 0.67
        \]
    \end{block}
\end{frame}

\begin{frame}[fragile]
    \frametitle{Confusion Matrix - Overview}
    \begin{block}{Overview}
        A \textbf{confusion matrix} is a powerful tool used in classification model evaluation. 
        It provides a visual representation of the performance of a classification algorithm by summarizing the correct and incorrect predictions made by the model.
    \end{block}
\end{frame}

\begin{frame}[fragile]
    \frametitle{Confusion Matrix - Structure}
    \begin{block}{Structure of the Confusion Matrix}
        The confusion matrix displays the actual vs. predicted classifications in a tabular format. 
        It typically consists of four key elements:
        \begin{itemize}
            \item \textbf{True Positives (TP)}: Correct positive predictions.
            \item \textbf{True Negatives (TN)}: Correct negative predictions.
            \item \textbf{False Positives (FP)}: Incorrect positive predictions (Type I error).
            \item \textbf{False Negatives (FN)}: Incorrect negative predictions (Type II error).
        \end{itemize}
    \end{block}
    \begin{block}{Example of a Confusion Matrix}
        \begin{center}
            \begin{tabular}{|c|c|c|}
                \hline
                & \textbf{Actual Positive} & \textbf{Actual Negative} \\
                \hline
                \textbf{Predicted Positive} & TP & FP \\
                \hline
                \textbf{Predicted Negative} & FN & TN \\
                \hline
            \end{tabular}
        \end{center}
    \end{block}
\end{frame}

\begin{frame}[fragile]
    \frametitle{Confusion Matrix - Significance and Metrics}
    \begin{block}{Significance in Model Evaluation}
        The confusion matrix plays a crucial role in assessing classification model performance. It helps with:
        \begin{itemize}
            \item \textbf{Understanding Errors}: Distinguishes between FP and FN, enabling targeted improvements.
            \item \textbf{Evaluating Multi-Class Classifiers}: Extends to an \(n \times n\) matrix for multiple classes.
        \end{itemize}
    \end{block}
    
    \begin{block}{Metrics Derived from the Confusion Matrix}
        The confusion matrix enables the calculation of various performance metrics, including:
        \begin{equation}
            \text{Accuracy} = \frac{TP + TN}{TP + TN + FP + FN}
        \end{equation}
        \begin{equation}
            \text{Precision} = \frac{TP}{TP + FP}
        \end{equation}
        \begin{equation}
            \text{Recall} = \frac{TP}{TP + FN}
        \end{equation}
        \begin{equation}
            F1 \text{ Score} = 2 \cdot \frac{\text{Precision} \cdot \text{Recall}}{\text{Precision} + \text{Recall}}
        \end{equation}
    \end{block}
\end{frame}

\begin{frame}[fragile]
    \frametitle{Comparison of Metrics - Overview}
    When evaluating machine learning models, it is crucial to differentiate between various performance metrics. This slide focuses on three key metrics: 
    \begin{itemize}
        \item \textbf{Accuracy}
        \item \textbf{Precision}
        \item \textbf{Recall}
    \end{itemize}
    Understanding when to prioritize one metric over another can significantly impact decision-making in a business context.
\end{frame}

\begin{frame}[fragile]
    \frametitle{Comparison of Metrics - Definitions}
    \begin{enumerate}
        \item \textbf{Accuracy}: 
            \begin{itemize}
                \item \textbf{Definition}: The ratio of correctly predicted instances (both true positives and true negatives) to the total instances.
                \item \textbf{Formula}:
                \begin{equation}
                    \text{Accuracy} = \frac{\text{TP} + \text{TN}}{\text{TP} + \text{TN} + \text{FP} + \text{FN}}
                \end{equation}
            \end{itemize}
        \item \textbf{Precision}:
            \begin{itemize}
                \item \textbf{Definition}: The ratio of true positives to the total predicted positives (true positives + false positives). 
                \item \textbf{Formula}:
                \begin{equation}
                    \text{Precision} = \frac{\text{TP}}{\text{TP} + \text{FP}}
                \end{equation}
            \end{itemize}
        \item \textbf{Recall} (Sensitivity):
            \begin{itemize}
                \item \textbf{Definition}: The ratio of true positives to the total actual positives (true positives + false negatives). 
                \item \textbf{Formula}:
                \begin{equation}
                    \text{Recall} = \frac{\text{TP}}{\text{TP} + \text{FN}}
                \end{equation}
            \end{itemize}
    \end{enumerate}
\end{frame}

\begin{frame}[fragile]
    \frametitle{Comparison of Metrics - Examples}
    \textbf{Example 1: Email Spam Detection}
    \begin{itemize}
        \item \textbf{Scenario}: Predicting whether an email is spam or not.
        \item \textbf{Proposed Metric}:
            \begin{itemize}
                \item Focus on \textbf{Precision} to reduce false alarms.
            \end{itemize}
    \end{itemize}

    \textbf{Example 2: Disease Diagnosis}
    \begin{itemize}
        \item \textbf{Scenario}: Predicting whether a patient has a specific disease.
        \item \textbf{Proposed Metric}:
            \begin{itemize}
                \item Prioritize \textbf{Recall} to ensure fewer undetected cases.
            \end{itemize}
    \end{itemize}
\end{frame}

\begin{frame}[fragile]
    \frametitle{Comparison of Metrics - Situations to Prefer Each}
    \begin{enumerate}
        \item \textbf{Accuracy}: Best when classes are balanced and misclassification costs are similar (e.g., recognizing handwritten digits).
        \item \textbf{Precision}: Important when false positives carry high costs, such as with fraud detection.
        \item \textbf{Recall}: Preferred in scenarios where false negatives are detrimental, like medical tests.
    \end{enumerate}
\end{frame}

\begin{frame}[fragile]
    \frametitle{Comparison of Metrics - Key Takeaways}
    \begin{itemize}
        \item Choose \textbf{Accuracy} in balanced scenarios.
        \item Opt for \textbf{Precision} when false positives are costly.
        \item Favor \textbf{Recall} when missing a positive case is critical.
    \end{itemize}
    Understanding these metrics enhances the ability to tailor models to specific business needs and consequences.
\end{frame}

\begin{frame}[fragile]
    \frametitle{Practical Applications}
    \begin{block}{Understanding Model Metrics in Real-World Scenarios}
        When evaluating the performance of machine learning models, it’s crucial to understand how metrics like accuracy, precision, recall, and F1 Score impact decisions in different industries.
    \end{block}
\end{frame}

\begin{frame}[fragile]
    \frametitle{Key Concepts}
    \begin{itemize}
        \item \textbf{Accuracy}: Measures overall correctness of the model.
        \item \textbf{Precision}: Reflects the accuracy of positive predictions; essential to avoid costly false positives.
        \item \textbf{Recall}: Indicates how well the model identifies true positives; critical where missing a positive case is detrimental.
        \item \textbf{F1 Score}: The harmonic mean of precision and recall; vital for imbalanced datasets.
    \end{itemize}
\end{frame}

\begin{frame}[fragile]
    \frametitle{Real-World Examples}
    \begin{enumerate}
        \item \textbf{Healthcare Diagnosis}
        \begin{itemize}
            \item Focus on high recall to catch all cancer cases.
            \item Ensure high precision to avoid unnecessary treatments.
        \end{itemize}
        
        \item \textbf{Email Spam Detection}
        \begin{itemize}
            \item High precision to reduce legitimate emails marked as spam.
            \item Balance with overall accuracy to ensure effective filtering.
        \end{itemize}
        
        \item \textbf{Fraud Detection in Banking}
        \begin{itemize}
            \item Strive for high recall to catch fraudulent transactions.
            \item Consider F1 Score to balance precision and recall effectively.
        \end{itemize}
        
        \item \textbf{Customer Churn Prediction}
        \begin{itemize}
            \item Use high precision for targeted customer retention efforts.
            \item Maintain high recall to identify actual churners proactively.
        \end{itemize}
    \end{enumerate}
\end{frame}

\begin{frame}[fragile]
    \frametitle{Key Takeaways}
    \begin{itemize}
        \item Choice of metric is context-dependent and crucial for outcomes.
        \item Balanced metrics like F1 Score are important for imbalanced classes.
        \item Metrics have far-reaching implications on operational and financial success.
    \end{itemize}
\end{frame}

\begin{frame}[fragile]
    \frametitle{Encourage Critical Thinking}
    \begin{block}{Reflection}
        Consider scenarios in your own experiences where choosing the right metric affected outcomes.
        How might differently weighted metrics change the decisions made from your models?
    \end{block}
\end{frame}

\begin{frame}[fragile]
    \frametitle{Conclusion - Evaluating Models: The Key to Effective Decision-Making}
    In this chapter, we’ve explored the vital role of model evaluation in data science and machine learning. 
    Understanding various metrics is essential for assessing model performance and ensuring real-world applicability.
\end{frame}

\begin{frame}[fragile]
    \frametitle{Importance of Model Evaluation}
    \begin{enumerate}
        \item \textbf{Informed Decision-Making:} 
            \begin{itemize}
                \item Selecting the right model involves understanding its strengths and weaknesses.
                \item Metrics such as \textbf{accuracy}, \textbf{precision}, \textbf{recall}, and \textbf{F1 Score} are tools to quantify performance for data-driven decisions.
            \end{itemize}
        
        \item \textbf{Identifying Areas for Improvement:} 
            \begin{itemize}
                \item Evaluation reveals specific flaws needing enhancement.
                \item Example: A model with high accuracy but low precision may require adjustments to improve positive case classification.
            \end{itemize}
        
        \item \textbf{Mitigating Risks:} 
            \begin{itemize}
                \item Poor model choices can have serious financial and ethical consequences.
                \item A flawed fraud detection model could lead to wrongful accusations; continuous evaluation reduces such risks.
            \end{itemize}
        
        \item \textbf{Real-World Relevance:} 
            \begin{itemize}
                \item Metrics influence decision-making across industries—impacting everything from healthcare to marketing.
            \end{itemize}
    \end{enumerate}
\end{frame}

\begin{frame}[fragile]
    \frametitle{Encouragement for Critical Assessment}
    As future data scientists, it's important to approach model evaluation critically:
    \begin{itemize}
        \item \textbf{Ask Questions:} 
            \begin{itemize}
                \item Does this model meet performance criteria for its application?
                \item What trade-offs exist between different metrics?
                \item How can we further refine the model?
            \end{itemize}
        
        \item \textbf{Experiment and Iterate:} 
            \begin{itemize}
                \item Model evaluation is an ongoing process.
                \item Don't hesitate to revisit and revise based on new data or feedback.
            \end{itemize}
    \end{itemize}
    
    \textbf{Key Takeaways:}
    \begin{itemize}
        \item Evaluation metrics are essential for effective model selection and improvement.
        \item Regular assessments ensure alignment with desired outcomes.
        \item Embrace continuous learning and questioning for robust model deployment.
    \end{itemize}
    
    In conclusion, model evaluation is crucial for developing trustworthy data-driven solutions.
\end{frame}


\end{document}