\documentclass[aspectratio=169]{beamer}

% Theme and Color Setup
\usetheme{Madrid}
\usecolortheme{whale}
\useinnertheme{rectangles}
\useoutertheme{miniframes}

% Additional Packages
\usepackage[utf8]{inputenc}
\usepackage[T1]{fontenc}
\usepackage{graphicx}
\usepackage{booktabs}
\usepackage{listings}
\usepackage{amsmath}
\usepackage{amssymb}
\usepackage{xcolor}
\usepackage{tikz}
\usepackage{pgfplots}
\pgfplotsset{compat=1.18}
\usetikzlibrary{positioning}
\usepackage{hyperref}

% Custom Colors
\definecolor{myblue}{RGB}{31, 73, 125}
\definecolor{mygray}{RGB}{100, 100, 100}
\definecolor{mygreen}{RGB}{0, 128, 0}
\definecolor{myorange}{RGB}{230, 126, 34}
\definecolor{mycodebackground}{RGB}{245, 245, 245}

% Set Theme Colors
\setbeamercolor{structure}{fg=myblue}
\setbeamercolor{frametitle}{fg=white, bg=myblue}
\setbeamercolor{title}{fg=myblue}
\setbeamercolor{section in toc}{fg=myblue}
\setbeamercolor{item projected}{fg=white, bg=myblue}
\setbeamercolor{block title}{bg=myblue!20, fg=myblue}
\setbeamercolor{block body}{bg=myblue!10}
\setbeamercolor{alerted text}{fg=myorange}

% Set Fonts
\setbeamerfont{title}{size=\Large, series=\bfseries}
\setbeamerfont{frametitle}{size=\large, series=\bfseries}
\setbeamerfont{caption}{size=\small}
\setbeamerfont{footnote}{size=\tiny}

% Custom Commands
\newcommand{\hilight}[1]{\colorbox{myorange!30}{#1}}
\newcommand{\concept}[1]{\textcolor{myblue}{\textbf{#1}}}

% Title Page Information
\title[Chapter: Mid-Term Review]{Week 8: Mid-Term Review \& Ethical Data Use}
\author[J. Smith]{John Smith, Ph.D.}
\institute[University Name]{
  Department of Computer Science\\
  University Name\\
  \vspace{0.3cm}
  Email: email@university.edu\\
  Website: www.university.edu
}
\date{\today}

% Document Start
\begin{document}

\frame{\titlepage}

\begin{frame}[fragile]
    \frametitle{Introduction to Mid-Term Review}
    \begin{block}{Overview of Key Topics}
        As we approach the mid-term, we reflect on key concepts in data processing and ethical considerations:
    \end{block}
\end{frame}

\begin{frame}[fragile]
    \frametitle{Data Processing Fundamentals}
    \begin{itemize}
        \item \textbf{Definition}: Data processing is the collection and manipulation of data to generate meaningful information.
        \item \textbf{Key Steps}:
        \begin{enumerate}
            \item \textbf{Data Collection}: Gathering raw data from various sources (surveys, sensors, databases).
            \item \textbf{Data Cleaning}: Identifying and correcting errors or inconsistencies to ensure accuracy.
            \item \textbf{Data Transformation}: Converting data into a suitable format using methods like normalization.
            \item \textbf{Data Analysis}: Applying statistical methods or algorithms to extract insights.
            \item \textbf{Data Visualization}: Presenting data graphically for easy interpretation.
        \end{enumerate}
    \end{itemize}
    \begin{block}{Example}
        Collecting survey data:
        \begin{itemize}
            \item Cleaning data to remove incomplete responses.
            \item Analyzing results to identify trends in consumer behavior.
        \end{itemize}
    \end{block}
\end{frame}

\begin{frame}[fragile]
    \frametitle{Ethical Considerations in Data Use}
    \begin{itemize}
        \item \textbf{Definition}: Ethical data use refers to responsible management and processing of data, ensuring respect for privacy, consent, and fairness.
        \item \textbf{Key Ethical Principles}:
        \begin{enumerate}
            \item \textbf{Informed Consent}: Users must be aware of data usage and give explicit permission.
            \item \textbf{Data Minimization}: Collect only necessary data to limit risks.
            \item \textbf{Transparency}: Being open about data processing builds trust.
            \item \textbf{Accountability}: Organizations must take responsibility for ethical data use.
        \end{enumerate}
    \end{itemize}
    \begin{block}{Example}
        A health app must inform users of data collection practices and ensure consent before gathering health data.
    \end{block}
\end{frame}

\begin{frame}[fragile]
    \frametitle{Importance of Ethical Data Use}
    Ethical data use refers to the principles and guidelines governing how data is collected, processed, stored, and shared. 
    It ensures that data handling respects stakeholders’ rights, fosters trust, and promotes social justice.
\end{frame}

\begin{frame}[fragile]
    \frametitle{Significance of Ethical Data Use}
    \begin{enumerate}
        \item \textbf{Trust and Transparency}
            \begin{itemize}
                \item Stakeholders are more likely to engage with organizations that demonstrate ethical practices.
                \item Committing to transparency builds and maintains trust.
            \end{itemize}
        
        \item \textbf{Accountability}
            \begin{itemize}
                \item Establishes accountability for data handlers.
                \item Reduces chances of abuse or mismanagement.
            \end{itemize}
        
        \item \textbf{Social Responsibility}
            \begin{itemize}
                \item Organizations must recognize their role in society.
                \item Ethical data use reflects commitment to social causes and prevents harm to marginalized communities.
            \end{itemize}
    \end{enumerate}
\end{frame}

\begin{frame}[fragile]
    \frametitle{Examples and Key Points}
    \begin{block}{Case Study}
        \textbf{Cambridge Analytica and Facebook} - This incident emphasizes the severe consequences of unethical data use.
        Personal data harvesting without consent for political advertising led to public outcry and loss of user trust.
    \end{block}
    
    \begin{block}{Key Points}
        \begin{itemize}
            \item \textbf{Protecting Privacy} - Ensures individuals have control over their data.
            \item \textbf{Equity and Fairness} - Promotes fairness and avoids bias in data use.
            \item \textbf{Compliance with Laws} - Adhering to regulations like GDPR for accountability.
        \end{itemize}
    \end{block}
\end{frame}

\begin{frame}[fragile]
    \frametitle{Key Ethical Principles}
    \begin{block}{Introduction}
        When handling data, it is critical to adhere to ethical principles. These principles lay the foundation for responsible data usage and ensure that the rights of individuals and the needs of society are respected.
    \end{block}
    \begin{itemize}
        \item Transparency
        \item Fairness
        \item Accountability
    \end{itemize}
\end{frame}

\begin{frame}[fragile]
    \frametitle{1. Transparency}
    \begin{block}{Definition}
        Transparency refers to the clarity and openness with which data practices are communicated. It involves providing stakeholders with accessible information about how data is collected, used, and shared.
    \end{block}
    \begin{itemize}
        \item Clear communication about data usage fosters trust.
        \item Organizations should disclose their data practices, including privacy policies and consent mechanisms.
    \end{itemize}
    \begin{block}{Example}
        A company developing an AI product should inform users about what data will be collected, how it will be used, and who will have access to it.
    \end{block}
\end{frame}

\begin{frame}[fragile]
    \frametitle{2. Fairness}
    \begin{block}{Definition}
        Fairness in data usage ensures that individuals and groups are treated equitably and without bias, particularly in automated systems that may influence decisions affecting people's lives.
    \end{block}
    \begin{itemize}
        \item Avoidance of discrimination in data algorithms is crucial.
        \item Continuous monitoring is necessary to detect bias and promote just outcomes.
    \end{itemize}
    \begin{block}{Example}
        In hiring algorithms, fairness entails ensuring that the data used does not lead to discriminatory outcomes based on race, gender, or socio-economic status.
    \end{block}
\end{frame}

\begin{frame}[fragile]
    \frametitle{3. Accountability}
    \begin{block}{Definition}
        Accountability involves taking responsibility for the actions taken with data and ensuring there are mechanisms for recourse if ethical standards are not met.
    \end{block}
    \begin{itemize}
        \item Organizations should establish policies and practices that hold individuals accountable for unethical data use.
        \item Implementing audits and evaluations can enhance accountability.
    \end{itemize}
    \begin{block}{Example}
        If a data breach occurs, the organization must inform affected individuals promptly and take steps to mitigate harm, thus demonstrating accountability.
    \end{block}
\end{frame}

\begin{frame}[fragile]
    \frametitle{Conclusion}
    Adhering to the principles of transparency, fairness, and accountability not only supports ethical data usage but also enhances the credibility and trustworthiness of the organization in the eyes of the public. These principles align with existing legal frameworks governing data privacy.
\end{frame}

\begin{frame}[fragile]
    \frametitle{Formulas and Diagrams}
    \begin{itemize}
        \item \textbf{Transparency} = Clear Communication + Open Practices
        \item \textbf{Fairness} = Equity in Data Representation + Ongoing Monitoring
        \item \textbf{Accountability} = Responsibility + Mechanisms for Recourse
    \end{itemize}
    By adhering to these principles, organizations can navigate the complex landscape of data ethics effectively.
\end{frame}

\begin{frame}[fragile]
  \frametitle{Legal Framework}
  \begin{block}{Overview of Data Privacy Laws and Regulations}
    Understanding the legal framework surrounding data privacy is essential for ensuring ethical data use. This framework includes various laws and regulations that dictate how organizations collect, store, process, and share personal data. Here we focus on two prominent privacy laws: 
    the General Data Protection Regulation (GDPR) and the California Consumer Privacy Act (CCPA).
  \end{block}
\end{frame}

\begin{frame}[fragile]
  \frametitle{General Data Protection Regulation (GDPR)}
  \begin{itemize}
    \item \textbf{Introduction:} Enacted in May 2018, GDPR protects personal data and privacy in the EU and EEA.
    \item \textbf{Key Aspects:}
      \begin{itemize}
        \item \textbf{Consent:} Explicit consent required for data processing.
        \item \textbf{Data Subject Rights:}
          \begin{itemize}
            \item Right to access: Request for personal data held.
            \item Right to rectification: Corrections of inaccurate data.
            \item Right to erasure: Request deletion under certain conditions.
          \end{itemize}
        \item \textbf{Data Breach Notifications:} Notify within 72 hours of a breach.
      \end{itemize}
    \item \textbf{Penalties:} Fines up to €20 million or 4\% of global revenue for non-compliance.
  \end{itemize}
\end{frame}

\begin{frame}[fragile]
  \frametitle{California Consumer Privacy Act (CCPA)}
  \begin{itemize}
    \item \textbf{Introduction:} Effective from January 2020, CCPA enhances privacy rights for California residents.
    \item \textbf{Key Aspects:}
      \begin{itemize}
        \item \textbf{Consumer Rights:}
          \begin{itemize}
            \item Right to know: Details on data collection and usage.
            \item Right to delete: Request deletion of personal information.
            \item Right to opt-out: Opt-out of the sale of personal information.
          \end{itemize}
        \item \textbf{Business Obligations:} Inform consumers about data practices and implement protection measures.
      \end{itemize}
    \item \textbf{Penalties:} Fines of $2,500 to $7,500 per violation.
  \end{itemize}
\end{frame}

\begin{frame}[fragile]
  \frametitle{Key Points and Conclusion}
  \begin{block}{Key Points to Emphasize}
    \begin{itemize}
      \item Importance of Compliance: GDPR and CCPA reflect a global movement towards stricter data privacy.
      \item Ethical Data Use: Legal frameworks are crucial for maintaining public trust and ethical responsibilities.
      \item Global Implications: Companies outside the EU/California may still be affected by these laws.
    \end{itemize}
  \end{block}
  
  \begin{block}{Conclusion}
    The landscape of data privacy laws illustrates the significance of ethical considerations in data utilization, emphasizing the need for organizations to navigate these complexities responsibly and uphold privacy rights.
  \end{block}
\end{frame}

\begin{frame}[fragile]
    \frametitle{Case Studies in Ethical Dilemmas}
    \begin{block}{Introduction}
        Ethical dilemmas in data use arise when decisions regarding data handling conflict with principles of privacy, consent, and transparency. Understanding these dilemmas through real-world examples is crucial.
    \end{block}
\end{frame}

\begin{frame}[fragile]
    \frametitle{Case Study Examples}
    \begin{enumerate}
        \item \textbf{Cambridge Analytica and Facebook}
            \begin{itemize}
                \item \textbf{Overview}: In 2016, Cambridge Analytica harvested the data of over 87 million Facebook users without informed consent.
                \item \textbf{Ethical Issues}: Lack of user awareness regarding data collection practices, and implications for democratic processes.
                \item \textbf{Outcome}: Widespread criticism, regulatory scrutiny, and calls for stricter data protection laws.
            \end{itemize}
        \item \textbf{Target's Predictive Analytics}
            \begin{itemize}
                \item \textbf{Overview}: Target predicted customer pregnancy status using purchasing patterns.
                \item \textbf{Ethical Issues}: Unintended privacy breaches as customers were targeted before disclosing their pregnancy.
                \item \textbf{Outcome}: Increased sales, but raised significant consumer privacy concerns.
            \end{itemize}
    \end{enumerate}
\end{frame}

\begin{frame}[fragile]
    \frametitle{Continued Case Study Examples}
    \begin{enumerate}
        \setcounter{enumi}{2}
        \item \textbf{Apple's Decision on User Privacy}
            \begin{itemize}
                \item \textbf{Overview}: Apple refuses to unlock phones for law enforcement, citing privacy.
                \item \textbf{Ethical Issues}: Balancing user privacy with national security interests.
                \item \textbf{Outcome}: Enhanced brand loyalty, but sparked debates about responsibilities to aid law enforcement.
            \end{itemize}
    \end{enumerate}
    \begin{block}{Key Points to Emphasize}
        \begin{itemize}
            \item Understanding ethical standards in data use, including consent and transparency, is critical.
            \item Consequences of ethical breaches can lead to legal repercussions and loss of consumer trust.
            \item Developing ethical guidelines for data use is vital for navigating dilemmas.
        \end{itemize}
    \end{block}
\end{frame}

\begin{frame}[fragile]
    \frametitle{Conclusion and Discussion}
    \begin{block}{Conclusion}
        The examination of these case studies reveals that ethical dilemmas in data use require careful consideration. Organizations must ensure that their practices align with ethical standards to protect user privacy and maintain trust.
    \end{block}
    
    \begin{block}{Discussion Questions}
        \begin{enumerate}
            \item How can organizations better align their data practices with ethical standards?
            \item What measures can enhance transparency in data usage among companies?
        \end{enumerate}
    \end{block}
\end{frame}

\begin{frame}[fragile]
    \frametitle{Responsible Data Handling Practices - Overview}
    \begin{itemize}
        \item Responsible data handling is essential for ethical data use.
        \item Protects individual privacy and builds trust.
        \item Key strategies:
        \begin{itemize}
            \item Data Minimization
            \item Anonymization
        \end{itemize}
    \end{itemize}
\end{frame}

\begin{frame}[fragile]
    \frametitle{Responsible Data Handling Practices - Data Minimization}
    \begin{block}{Definition}
        Data minimization involves limiting data collection and storage to only what is necessary for specific purposes.
    \end{block}
    \begin{itemize}
        \item \textbf{Key Principles:}
        \begin{itemize}
            \item Purpose Limitation: Collect only necessary data for specific goals.
            \item Relevance: Ensure data is relevant to tasks at hand.
        \end{itemize}
        \item \textbf{Example:} For a consumer habits survey, avoid collecting unnecessary sensitive information such as a full social security number.
    \end{itemize}
\end{frame}

\begin{frame}[fragile]
    \frametitle{Responsible Data Handling Practices - Anonymization}
    \begin{block}{Definition}
        Anonymization is the removal of personally identifiable information from datasets.
    \end{block}
    \begin{itemize}
        \item \textbf{Key Techniques:}
        \begin{itemize}
            \item Data Masking: Replace sensitive information with fake data.
            \item Aggregation: Summarize data to prevent identification of individuals.
        \end{itemize}
        \item \textbf{Example:} Use unique IDs instead of names in customer purchase studies.
        \item \textbf{Note:} Anonymization is crucial for compliance with laws like GDPR.
    \end{itemize}
\end{frame}

\begin{frame}[fragile]
    \frametitle{Responsible Data Handling Practices - Key Points}
    \begin{itemize}
        \item Assess the necessity of data collected.
        \item Implement anonymization methods to protect privacy.
        \item Regularly review data handling practices for ethical compliance.
    \end{itemize}
    \begin{block}{Additional Considerations}
        \begin{itemize}
            \item Documentation: Maintain records for accountability.
            \item Training: Educate team members on responsible data usage.
        \end{itemize}
    \end{block}
\end{frame}

\begin{frame}[fragile]
    \frametitle{Group Discussion: Ethical Scenarios}
    Engage students in discussing hypothetical scenarios to identify ethical dilemmas in data usage and propose solutions.

    \begin{block}{Concept Overview}
        Ethical dilemmas in data usage arise from collection, analysis, and sharing practices. It is important to identify these dilemmas to promote responsible data handling. 
    \end{block}
\end{frame}

\begin{frame}[fragile]
    \frametitle{Identifying Ethical Dilemmas}
    \begin{enumerate}
        \item \textbf{Informed Consent}
            \begin{itemize}
                \item \textit{Definition:} Individuals must be fully aware of data usage before consent.
                \item \textit{Example:} A health app collects user data but fails to inform users about its sale to advertisers.
            \end{itemize}
        
        \item \textbf{Data Privacy}
            \begin{itemize}
                \item \textit{Definition:} Respecting individuals' control over their personal information.
                \item \textit{Example:} A university collects student data but does not anonymize sensitive information, risking identity exposure.
            \end{itemize}
        
        \item \textbf{Data Ownership}
            \begin{itemize}
                \item \textit{Definition:} Determining the rights to collected data and its use decisions.
                \item \textit{Example:} A non-profit collects community data without involving the community in the usage decisions.
            \end{itemize}
    \end{enumerate}
\end{frame}

\begin{frame}[fragile]
    \frametitle{Group Discussion Prompts}
    Discuss the following scenarios:
    
    \begin{enumerate}
        \item \textbf{Scenario 1:} A company develops an AI tool using personal data to predict consumer behavior. What ethical considerations should they account for and how can they ensure ethical use of data?
        
        \item \textbf{Scenario 2:} A data scientist exploits a data set to predict sensitive outcomes (like health risks). How should they navigate the ethical implications?
        
        \item \textbf{Scenario 3:} A social media platform collects vast amounts of data 'anonymously' but can still correlate it back to individuals. Discuss the ethical boundaries of such practices.
    \end{enumerate}
    
    \begin{block}{Key Points to Emphasize}
        - Ethical frameworks in data use include fairness, accountability, and transparency.
        - Proactive solutions involve data governance, audits, and strong consent processes.
        - Collaboration with stakeholders is vital for creating ethical guidelines.
    \end{block}
\end{frame}

\begin{frame}[fragile]
    \frametitle{Reflection on Learning: Ethical Data Use}
    Wrap-up of key takeaways regarding ethical data use and how students can apply these principles in their future work.
\end{frame}

\begin{frame}[fragile]
    \frametitle{Key Concepts in Ethical Data Use}
    \begin{enumerate}
        \item \textbf{Informed Consent}
        \begin{itemize}
            \item \textit{Definition:} Ensuring individuals are aware of data use before consenting.
            \item \textit{Example:} A health app must explain how user data will be used.
        \end{itemize}

        \item \textbf{Data Privacy}
        \begin{itemize}
            \item \textit{Definition:} Protecting personal information and ensuring confidentiality.
            \item \textit{Example:} Using encryption to safeguard user data.
        \end{itemize}

        \item \textbf{Data Integrity}
        \begin{itemize}
            \item \textit{Definition:} Ensuring data accuracy and reliability.
            \item \textit{Example:} Conducting regular audits to catch data entry errors.
        \end{itemize}
    \end{enumerate}
\end{frame}

\begin{frame}[fragile]
    \frametitle{Key Concepts in Ethical Data Use (Continued)}
    \begin{enumerate}
        \setcounter{enumi}{3}  % Continue from the last number
        \item \textbf{Accountability}
        \begin{itemize}
            \item \textit{Definition:} Holding organizations responsible for data practices.
            \item \textit{Example:} Reporting data breaches to regulatory bodies.
        \end{itemize}

        \item \textbf{Transparency}
        \begin{itemize}
            \item \textit{Definition:} Clarity on data collection, usage, and sharing.
            \item \textit{Example:} Providing user-friendly data usage policies.
        \end{itemize}
    \end{enumerate}
\end{frame}

\begin{frame}[fragile]
    \frametitle{Applying Ethical Principles in Future Work}
    \begin{itemize}
        \item \textbf{Case Study Analysis:}
        \begin{itemize}
            \item Reflect on real-world case studies on ethical dilemmas.
            \item Ask: \textit{What went wrong? How could ethical considerations have changed the outcome?}
        \end{itemize}

        \item \textbf{Developing Ethically-Driven Projects:}
        \begin{itemize}
            \item Start by outlining ethical considerations alongside project goals.
            \item Include diverse perspectives to avoid biases in discussions.
        \end{itemize}

        \item \textbf{Continuous Learning:}
        \begin{itemize}
            \item Stay updated on regulations (GDPR, HIPAA, CCPA).
            \item Participate in workshops on data ethics.
        \end{itemize}
    \end{itemize}
\end{frame}

\begin{frame}[fragile]
    \frametitle{Key Takeaways}
    \begin{itemize}
        \item Ethical data use builds trust and ensures responsible data utilization.
        \item Analyze ethical implications of data strategies consistently.
        \item Advocate for ethical standards and educate teams about data practices.
    \end{itemize}

    By understanding and implementing these principles, you enhance your skills and contribute positively to the data-driven landscape in your future careers.
\end{frame}


\end{document}