\documentclass[aspectratio=169]{beamer}

% Theme and Color Setup
\usetheme{Madrid}
\usecolortheme{whale}
\useinnertheme{rectangles}
\useoutertheme{miniframes}

% Additional Packages
\usepackage[utf8]{inputenc}
\usepackage[T1]{fontenc}
\usepackage{graphicx}
\usepackage{booktabs}
\usepackage{listings}
\usepackage{amsmath}
\usepackage{amssymb}
\usepackage{xcolor}
\usepackage{tikz}
\usepackage{pgfplots}
\pgfplotsset{compat=1.18}
\usetikzlibrary{positioning}
\usepackage{hyperref}

% Custom Colors
\definecolor{myblue}{RGB}{31, 73, 125}
\definecolor{mygray}{RGB}{100, 100, 100}
\definecolor{mygreen}{RGB}{0, 128, 0}
\definecolor{myorange}{RGB}{230, 126, 34}
\definecolor{mycodebackground}{RGB}{245, 245, 245}

% Set Theme Colors
\setbeamercolor{structure}{fg=myblue}
\setbeamercolor{frametitle}{fg=white, bg=myblue}
\setbeamercolor{title}{fg=myblue}
\setbeamercolor{section in toc}{fg=myblue}
\setbeamercolor{item projected}{fg=white, bg=myblue}
\setbeamercolor{block title}{bg=myblue!20, fg=myblue}
\setbeamercolor{block body}{bg=myblue!10}
\setbeamercolor{alerted text}{fg=myorange}

% Set Fonts
\setbeamerfont{title}{size=\Large, series=\bfseries}
\setbeamerfont{frametitle}{size=\large, series=\bfseries}
\setbeamerfont{caption}{size=\small}
\setbeamerfont{footnote}{size=\tiny}

% Code Listing Style
\lstdefinestyle{customcode}{
  backgroundcolor=\color{mycodebackground},
  basicstyle=\footnotesize\ttfamily,
  breakatwhitespace=false,
  breaklines=true,
  commentstyle=\color{mygreen}\itshape,
  keywordstyle=\color{blue}\bfseries,
  stringstyle=\color{myorange},
  numbers=left,
  numbersep=8pt,
  numberstyle=\tiny\color{mygray},
  frame=single,
  framesep=5pt,
  rulecolor=\color{mygray},
  showspaces=false,
  showstringspaces=false,
  showtabs=false,
  tabsize=2,
  captionpos=b
}
\lstset{style=customcode}

% Custom Commands
\newcommand{\hilight}[1]{\colorbox{myorange!30}{#1}}
\newcommand{\source}[1]{\vspace{0.2cm}\hfill{\tiny\textcolor{mygray}{Source: #1}}}
\newcommand{\concept}[1]{\textcolor{myblue}{\textbf{#1}}}
\newcommand{\separator}{\begin{center}\rule{0.5\linewidth}{0.5pt}\end{center}}

% Footer and Navigation Setup
\setbeamertemplate{footline}{
  \leavevmode%
  \hbox{%
  \begin{beamercolorbox}[wd=.3\paperwidth,ht=2.25ex,dp=1ex,center]{author in head/foot}%
    \usebeamerfont{author in head/foot}\insertshortauthor
  \end{beamercolorbox}%
  \begin{beamercolorbox}[wd=.5\paperwidth,ht=2.25ex,dp=1ex,center]{title in head/foot}%
    \usebeamerfont{title in head/foot}\insertshorttitle
  \end{beamercolorbox}%
  \begin{beamercolorbox}[wd=.2\paperwidth,ht=2.25ex,dp=1ex,center]{date in head/foot}%
    \usebeamerfont{date in head/foot}
    \insertframenumber{} / \inserttotalframenumber
  \end{beamercolorbox}}%
  \vskip0pt%
}

% Turn off navigation symbols
\setbeamertemplate{navigation symbols}{}

% Title Page Information
\title[Chapter 12]{Chapter 12: Presentations and Feedback}
\author[J. Smith]{John Smith, Ph.D.}
\institute[University Name]{
  Department of Computer Science\\
  University Name\\
  \vspace{0.3cm}
  Email: email@university.edu\\
  Website: www.university.edu
}
\date{\today}

% Document Start
\begin{document}

\frame{\titlepage}

\begin{frame}[fragile]
    \frametitle{Introduction to Presentations and Feedback}
    \begin{block}{Significance of Presentations}
        Presentations are structured forms of communication that enable individuals to convey information and engage with an audience.
    \end{block}
    \begin{itemize}
        \item Verbal presentations, slideshows, posters, etc.
        \item Critical for skill development, engagement, and critical thinking.
    \end{itemize}
\end{frame}

\begin{frame}[fragile]
    \frametitle{Importance of Presentations}
    \begin{enumerate}
        \item \textbf{Skill Development:} Cultivates public speaking, organization, and information synthesis.
        \item \textbf{Engagement:} Encourages audience interaction through questions and discussions.
        \item \textbf{Critical Thinking:} Requires analysis to communicate complex ideas effectively.
    \end{enumerate}
    \begin{block}{Example}
        A group project on climate change solutions illustrates the need for data analysis and clear presentations, reinforcing understanding and communication skills.
    \end{block}
\end{frame}

\begin{frame}[fragile]
    \frametitle{Role of Feedback in Enhancing Communication Skills}
    \begin{block}{Understanding Feedback}
        Feedback is essential for improvement, providing constructive criticism on presentation content and delivery.
    \end{block}
    \begin{enumerate}
        \item \textbf{Identifies Strengths and Weaknesses:} Recognizes achievements and areas for improvement.
        \item \textbf{Encourages Reflection:} Promotes lifelong learning and adaptability.
        \item \textbf{Fosters Collaboration:} Creates a supportive environment for collective growth.
    \end{enumerate}
\end{frame}

\begin{frame}[fragile]
    \frametitle{Key Points and Conclusion}
    \begin{block}{Key Points}
        \begin{itemize}
            \item Presentations are vital for effective communication.
            \item Audience feedback is crucial for skill enhancement.
            \item Embracing feedback builds resilience in professional settings.
        \end{itemize}
    \end{block}
    \begin{block}{Conclusion}
        Developing presentation and feedback skills is key for academic success and future careers.
    \end{block}
    \begin{block}{Questions for Reflection}
        \begin{itemize}
            \item What aspects of your presentation style could benefit from feedback?
            \item How can you integrate audience interaction for engaging presentations?
        \end{itemize}
    \end{block}
\end{frame}

\begin{frame}[fragile]{Objectives of the Final Project Presentation - Part 1}
    \frametitle{Introduction}
    The final project presentation is a culmination of your learning experience throughout the course. 
    It is essential not only to convey your project content but also to effectively engage with your audience and respond to their feedback. 
    Here, we outline key objectives students should aim to achieve during their presentations.
\end{frame}

\begin{frame}[fragile]{Objectives of the Final Project Presentation - Part 2}
    \frametitle{Key Objectives}
    \begin{enumerate}
        \item \textbf{Effective Communication}
            \begin{itemize}
                \item \textbf{Clarity:} Present your ideas in a coherent and straightforward manner.
                    \begin{block}{Example}
                        Instead of saying "the implementation of the algorithm significantly improved performance metrics," clarify with 
                        "our new algorithm cut processing time by 30%, making it faster and more efficient."
                    \end{block}
                \item \textbf{Engagement:} Use strategies like storytelling and visuals.
                    \begin{block}{Illustration}
                        Start with a thought-provoking question related to your topic or share a brief, relevant personal story.
                    \end{block}
            \end{itemize}

        \item \textbf{Demonstrating Mastery of Content}
            \begin{itemize}
                \item \textbf{Depth of Knowledge:} Show that you are well-versed in your topic.
                    \begin{block}{Example}
                        Instead of a brief discussion, dive deeper into the research process and rationale behind your choices.
                    \end{block}
            \end{itemize}
    \end{enumerate}
\end{frame}

\begin{frame}[fragile]{Objectives of the Final Project Presentation - Part 3}
    \frametitle{Key Objectives Continued}
    \begin{enumerate}[resume]
        \item \textbf{Critical Engagement with Feedback}
            \begin{itemize}
                \item \textbf{Active Listening:} Be open to audience feedback and questions.
                    \begin{block}{Technique}
                        Repeat the question before answering to clarify and demonstrate understanding.
                    \end{block}
                \item \textbf{Iterative Improvement:} Use feedback to enhance future presentations.
                    \begin{block}{Statement}
                        "I appreciate your insight; we will consider this in our next stage of development."
                    \end{block}
            \end{itemize}

        \item \textbf{Demonstrating Confidence and Professionalism}
            \begin{itemize}
                \item \textbf{Body Language:} Maintain eye contact and use appropriate gestures.
                \item \textbf{Preparedness:} Practice to reduce anxiety and improve delivery.
            \end{itemize}
    \end{enumerate}
\end{frame}

\begin{frame}[fragile]{Objectives of the Final Project Presentation - Conclusion}
    \frametitle{Conclusion}
    By focusing on these objectives, you can elevate your presentation from mere information sharing to an interactive and reflective exchange of ideas. 
    Remember that the goal is not only to present but also to engage, inspire, and receive constructive feedback that can contribute to your learning journey. 

    \begin{block}{Key Points to Remember}
        \begin{itemize}
            \item Clarity and engagement are fundamental to effective communication.
            \item Show depth in content knowledge to demonstrate mastery.
            \item Embrace feedback as a tool for growth.
            \item Display confidence through body language and preparedness.
        \end{itemize}
    \end{block}
\end{frame}

\begin{frame}[fragile]
    \frametitle{Structure of the Presentations - Introduction}
    \begin{itemize}
        \item A well-structured presentation is essential for effective communication.
        \item Helps the audience absorb information clearly and maintains engagement.
        \item Key components will guide students towards clarity and coherence.
    \end{itemize}
\end{frame}

\begin{frame}[fragile]
    \frametitle{Structure of the Presentations - Key Components}
    \begin{enumerate}
        \item \textbf{Introduction}
        \begin{itemize}
            \item \textbf{Purpose}: Provide a roadmap and introduce the topic.
            \item \textbf{Key Elements}:
            \begin{itemize}
                \item \textbf{Hook}: Start with an engaging fact or story.
                \item \textbf{Thesis Statement}: Clearly state the main idea.
                \item \textbf{Overview}: Briefly outline what will be covered.
            \end{itemize}
            \item \textbf{Example}: "Did you know that over 80\% of people fear public speaking?"
        \end{itemize}
        
        \item \textbf{Body}
        \begin{itemize}
            \item \textbf{Purpose}: Deliver the main content and support your thesis.
            \item \textbf{Key Elements}:
            \begin{itemize}
                \item Organize into clear sections with 2-4 main points.
                \item Use supporting evidence: data, anecdotes, or quotes.
                \item Use transitions to guide the audience.
            \end{itemize}
        \end{itemize}
    \end{enumerate}
\end{frame}

\begin{frame}[fragile]
    \frametitle{Structure of the Presentations - Conclusion}
    \begin{enumerate}
        \item \textbf{Conclusion}
        \begin{itemize}
            \item \textbf{Purpose}: Summarizes key points and reinforces the message.
            \item \textbf{Key Elements}:
            \begin{itemize}
                \item Recap major points discussed.
                \item Call to action for audience engagement.
                \item Closing thought to leave a lasting impression.
            \end{itemize}
            \item \textbf{Example}: "As we’ve seen today, effective presentation skills are crucial."
        \end{itemize}

        \item \textbf{Tips for a Coherent Structure}
        \begin{itemize}
            \item Use visual aids: slides, charts, videos.
            \item Practice transitions for smooth flow.
            \item Manage time to cover content effectively.
        \end{itemize}
    \end{enumerate}
\end{frame}

\begin{frame}[fragile]
    \frametitle{Overview of Communication Skills}
    \begin{itemize}
        \item Effective communication is essential for delivering successful presentations.
        \item Influences how your ideas are received and shapes your credibility.
        \item Strategies for improvement:
            \begin{itemize}
                \item Public speaking tips
                \item Visual communication methods
            \end{itemize}
    \end{itemize}
\end{frame}

\begin{frame}[fragile]
    \frametitle{Public Speaking Tips}
    \begin{enumerate}
        \item \textbf{Practice Regularly}
            \begin{itemize}
                \item Rehearse multiple times to build comfort and confidence.
                \item Example: Practice in front of a mirror or record yourself to review.
            \end{itemize}
        \item \textbf{Know Your Audience}
            \begin{itemize}
                \item Tailor your message to resonate with the audience.
                \item Example: Adjust language for a science class vs. community event.
            \end{itemize}
        \item \textbf{Engage with the Audience}
            \begin{itemize}
                \item Start with a question or story to connect with your audience.
                \item Example: "Have you ever wondered how much waste we produce in a year?"
            \end{itemize}
    \end{enumerate}
\end{frame}

\begin{frame}[fragile]
    \frametitle{Additional Public Speaking Tips}
    \begin{enumerate}[resume]
        \item \textbf{Pace and Pausing}
            \begin{itemize}
                \item Speak at a moderate pace and use pauses for emphasis.
                \item Vary your tone to keep the audience engaged.
            \end{itemize}
        \item \textbf{Body Language}
            \begin{itemize}
                \item Appropriate gestures, eye contact, and open posture are key.
                \item Example: Stand straight and use gestures to convey enthusiasm.
            \end{itemize}
    \end{enumerate}
\end{frame}

\begin{frame}[fragile]
    \frametitle{Visual Communication Methods}
    \begin{enumerate}
        \item \textbf{Effective Slide Design}
            \begin{itemize}
                \item Keep slides simple with bullet points, high-contrast colors, and readable fonts.
                \item Principle: One main idea per slide.
            \end{itemize}
        \item \textbf{Images and Graphics}
            \begin{itemize}
                \item Use relevant visuals to support points, like graphs and charts.
                \item Example: A pie chart to show recycling rates.
            \end{itemize}
        \item \textbf{Consistency in Format}
            \begin{itemize}
                \item Uniformity in slide design enhances professionalism.
                \item Use templates that reflect your brand or presentation theme.
            \end{itemize}
    \end{enumerate}
\end{frame}

\begin{frame}[fragile]
    \frametitle{Key Points to Emphasize}
    \begin{itemize}
        \item \textbf{Preparation is Key}:
        The more prepared you are, the better your delivery.
        \item \textbf{Adaptability}:
        Adjust your presentation based on audience reactions.
        \item \textbf{Feedback}:
        Solicit and implement feedback to continuously improve skills.
    \end{itemize}
\end{frame}

\begin{frame}[fragile]
    \frametitle{Conclusion}
    Improving communication skills involves:
    \begin{itemize}
        \item Practice,
        \item Understanding your audience,
        \item Employing effective visual aids.
    \end{itemize}
    By incorporating these strategies, enhance your ability to convey ideas clearly and engagingly.
\end{frame}

\begin{frame}[fragile]
    \frametitle{Receiving and Implementing Feedback - Introduction}
    \begin{block}{Overview}
        Receiving feedback is a valuable opportunity for growth, particularly in the context of presentations. 
        The manner in which we accept and utilize feedback significantly influences our future performance. 
        Let's explore best practices for receiving feedback with grace and methods for implementing it to enhance our presentation skills.
    \end{block}
\end{frame}

\begin{frame}[fragile]
    \frametitle{Receiving Feedback - Best Practices}
    \begin{enumerate}
        \item \textbf{Graciously Receiving Feedback}
            \begin{itemize}
                \item \textbf{Listening Actively:} Give the speaker your full attention without interrupting.
                \item \textbf{Keeping an Open Mind:} Approach feedback as an opportunity for improvement.
                \item \textbf{Expressing Gratitude:} Thank the individual providing feedback, regardless of its content.
            \end{itemize}
    \end{enumerate}
\end{frame}

\begin{frame}[fragile]
    \frametitle{Implementing Feedback - Key Strategies}
    \begin{enumerate}
        \setcounter{enumi}{1}
        \item \textbf{Analyzing Feedback}
            \begin{itemize}
                \item \textbf{Differentiating Between Constructive Criticism and Negativity:} Focus on actionable insights.
                \item \textbf{Seeking Clarification:} Ask for examples if feedback is unclear.
            \end{itemize}
        \item \textbf{Implementing Feedback}
            \begin{itemize}
                \item \textbf{Identifying Key Themes:} Look for recurring suggestions.
                \item \textbf{Setting Specific Goals:} Transform feedback into actionable steps.
                \item \textbf{Practicing Changes:} Rehearse your presentation with the feedback integrated.
            \end{itemize}
    \end{enumerate}
\end{frame}

\begin{frame}[fragile]
    \frametitle{Conclusion and Key Takeaways}
    \begin{block}{Conclusion}
        Applying best practices for receiving and implementing feedback enhances your presentation skills and fosters a growth mindset. 
        Remember, feedback is about improvement and elevating your performance.
    \end{block}
    \begin{itemize}
        \item Actively listen and express gratitude for feedback.
        \item Differentiate between constructive feedback and negativity.
        \item Set specific goals based on feedback and practice implementing them.
    \end{itemize}
\end{frame}

\begin{frame}[fragile]
    \frametitle{Peer Feedback Mechanisms}
    \begin{block}{Understanding Peer Feedback}
        Peer feedback is an essential component of the learning process that enables students to learn from one another while enhancing their communication skills. This slide outlines how peer feedback will be conducted and details the criteria for constructive feedback.
    \end{block}
\end{frame}

\begin{frame}[fragile]
    \frametitle{Conducting Peer Feedback}
    \begin{enumerate}
        \item \textbf{Structured Peer Review Sessions} 
        \begin{itemize}
            \item Each student presents their work, and designated peers provide feedback.
            \item Feedback focuses on clarity, engagement, content accuracy, and organization.
        \end{itemize}
        
        \item \textbf{Feedback Forms}
        \begin{itemize}
            \item Use standardized forms for both qualitative and quantitative feedback.
            \item Example questions:
            \begin{itemize}
                \item What did you like most about the presentation?
                \item What specific areas could be improved?
                \item On a scale of 1-10, how effectively was the main idea communicated?
            \end{itemize}
        \end{itemize}

        \item \textbf{Peer Pairings}
        \begin{itemize}
            \item Rotate pairings for varied perspectives.
            \item Pair students of diverse backgrounds to encourage wide-ranging feedback.
        \end{itemize}
    \end{enumerate}
\end{frame}

\begin{frame}[fragile]
    \frametitle{Criteria for Constructive Feedback}
    \begin{enumerate}
        \item \textbf{Specificity}
        \begin{itemize}
            \item Feedback should target specific elements rather than general impressions.
            \item Example: Instead of "It was good," say "The visuals effectively illustrated your points."
        \end{itemize}
        
        \item \textbf{Positivity}
        \begin{itemize}
            \item Begin with positive observations before addressing areas for improvement.
            \item Example: "Your introduction was engaging, but the conclusion could be stronger."
        \end{itemize}

        \item \textbf{Actionable Suggestions}
        \begin{itemize}
            \item Provide feasible ideas or strategies for improvement.
            \item Example: "Consider slowing down your pace for enhanced clarity."
        \end{itemize}

        \item \textbf{Balanced Feedback}
        \begin{itemize}
            \item Balance positive comments with constructive criticism to maintain motivation.
        \end{itemize}
    \end{enumerate}
\end{frame}

\begin{frame}[fragile]
    \frametitle{Evaluating Communication Effectiveness}
    Evaluating communication effectiveness is essential for personal growth and development in both academic and professional settings. 
    It involves looking at how well messages are conveyed and received and understanding how to improve this process.
\end{frame}

\begin{frame}[fragile]
    \frametitle{Key Components of Evaluation}
    \begin{enumerate}
        \item \textbf{Feedback}:
          \begin{itemize}
              \item \textbf{Source of Feedback}: Gather insights from peers, instructors, or audience members.
              \item \textbf{Types of Feedback}:
                \begin{itemize}
                    \item \textbf{Constructive Feedback}: Focuses on strengths and areas for improvement.
                    \item \textbf{Positive Feedback}: Reinforces successful communication strategies.
                    \item \textbf{Critical Feedback}: Highlights weaknesses that require attention.
                \end{itemize}
          \end{itemize}
        \item \textbf{Self-Reflection}:
          \begin{itemize}
              \item \textbf{Self-Assessment}: After presenting, take time to analyze your performance.
                  \begin{itemize}
                      \item Questions to consider:
                      \begin{itemize}
                          \item Did I convey my message clearly?
                          \item How engaged was my audience?
                          \item What emotions did I observe in my audience? 
                      \end{itemize}
                  \end{itemize}
              \item \textbf{Journaling}: Keep a reflection journal to note down thoughts and feelings about your communication experiences.
          \end{itemize}
    \end{enumerate}
\end{frame}

\begin{frame}[fragile]
    \frametitle{Process of Evaluation}
    \begin{enumerate}
        \item \textbf{Collect Feedback}:
            \begin{itemize}
                \item Use structured forms to gather feedback based on criteria like clarity, engagement, content.
            \end{itemize}
        \item \textbf{Reflect}:
            \begin{itemize}
                \item Set aside time after each presentation to reflect on the feedback received.
                \item Compare the feedback against your self-assessment.
            \end{itemize}
        \item \textbf{Identify Trends}:
            \begin{itemize}
                \item Look for patterns in feedback. Common themes may indicate strengths or weaknesses that need attention.
            \end{itemize}
    \end{enumerate}
\end{frame}

\begin{frame}[fragile]
    \frametitle{Conclusion and Future Applications}
    \begin{block}{Importance of Presentation Skills}
        \begin{enumerate}
            \item \textbf{Communication Mastery}
            \item \textbf{Confidence Building}
            \item \textbf{Critical Thinking Development}
        \end{enumerate}
    \end{block}
\end{frame}

\begin{frame}[fragile]
    \frametitle{Importance of Presentation Skills - Details}
    \begin{itemize}
        \item \textbf{Communication Mastery:} 
        Developing presentation skills enhances clarity, conciseness, and engagement.
        \begin{itemize}
            \item \textit{Example:} Using storytelling techniques to connect with the audience emotionally.
        \end{itemize}
        \item \textbf{Confidence Building:} 
        Frequent presentations help cope with anxiety and improve public speaking.
        \begin{itemize}
            \item \textit{Illustration:} A student improves their delivery and tone through practice.
        \end{itemize}
        \item \textbf{Critical Thinking Development:} 
        Preparing presentations fosters organizing thoughts and crafting coherent arguments.
        \begin{itemize}
            \item \textit{Example:} Evaluating sources and synthesizing information on climate change.
        \end{itemize}
    \end{itemize}
\end{frame}

\begin{frame}[fragile]
    \frametitle{Future Applications}
    \begin{block}{Future Applications}
        \begin{itemize}
            \item \textbf{Academic Pursuits:} 
            Presenting research findings or defending theses leads to better grades and recognition.
            \item \textbf{Professional Settings:} 
            Essential for job interviews and team meetings; clear communication sets candidates apart.
            \item \textbf{Networking Opportunities:} 
            Lead to deeper connections and credibility at conferences and workshops.
        \end{itemize}
    \end{block}
\end{frame}

\begin{frame}[fragile]
    \frametitle{Key Takeaways and Questions for Reflection}
    \begin{block}{Key Takeaways}
        \begin{itemize}
            \item \textbf{Skill Building:} Essential across all walks of life.
            \item \textbf{Practice Makes Perfect:} Continual practice enhances confidence and effectiveness.
            \item \textbf{Feedback Is Essential:} Use feedback to refine presentation skills.
        \end{itemize}
    \end{block}
    
    \begin{block}{Questions for Reflection}
        \begin{itemize}
            \item How can you apply the skills from presentations to your future goals?
            \item What areas of your presentation skills need further development?
        \end{itemize}
    \end{block}
\end{frame}

\begin{frame}[fragile]
    \frametitle{Q\&A Session - Purpose}
    \begin{block}{Purpose of the Q\&A Session}
        To create an open platform for engaging discussions, clarifying doubts, and enhancing understanding of presentation and feedback processes. This interactive space allows for deeper learning through shared experiences.
    \end{block}
\end{frame}

\begin{frame}[fragile]
    \frametitle{Q\&A Session - Key Concepts}
    \begin{enumerate}
        \item \textbf{Understanding Presentations:}
            \begin{itemize}
                \item \textbf{Definition:} A structured way to communicate information to an audience, often enhanced with visual aids.
                \item \textbf{Importance:}
                    \begin{itemize}
                        \item Develops effective communication skills.
                        \item Facilitates sharing of ideas in academic and professional settings.
                    \end{itemize}
            \end{itemize}
        \item \textbf{Feedback Processes:}
            \begin{itemize}
                \item \textbf{Definition:} Providing evaluative information on a person's performance.
                \item \textbf{Importance:}
                    \begin{itemize}
                        \item Crucial for personal and professional growth.
                        \item Helps refine skills by addressing gaps.
                    \end{itemize}
            \end{itemize}
    \end{enumerate}
\end{frame}

\begin{frame}[fragile]
    \frametitle{Q\&A Session - Engaging Participation}
    \begin{block}{Preparation for Q\&A}
        \begin{itemize}
            \item \textbf{Encourage Participation:}
                \begin{itemize}
                    \item Reflect on main takeaways from previous presentations.
                    \item Think of scenarios involving uncertainties in presentations or feedback.
                \end{itemize}
            \item \textbf{Examples to Inspire Questions:}
                \begin{itemize}
                    \item \textit{Presentation Challenge:} Discuss a well-prepared presentation that did not go as expected.
                    \item \textit{Feedback Experience:} Share a helpful or upsetting feedback situation and the impact.
                \end{itemize}
        \end{itemize}
    \end{block}
\end{frame}


\end{document}