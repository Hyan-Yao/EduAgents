\documentclass[aspectratio=169]{beamer}

% Theme and Color Setup
\usetheme{Madrid}
\usecolortheme{whale}
\useinnertheme{rectangles}
\useoutertheme{miniframes}

% Additional Packages
\usepackage[utf8]{inputenc}
\usepackage[T1]{fontenc}
\usepackage{graphicx}
\usepackage{booktabs}
\usepackage{listings}
\usepackage{amsmath}
\usepackage{amssymb}
\usepackage{xcolor}
\usepackage{tikz}
\usepackage{pgfplots}
\pgfplotsset{compat=1.18}
\usetikzlibrary{positioning}
\usepackage{hyperref}

% Custom Colors
\definecolor{myblue}{RGB}{31, 73, 125}
\definecolor{mygray}{RGB}{100, 100, 100}
\definecolor{mygreen}{RGB}{0, 128, 0}
\definecolor{myorange}{RGB}{230, 126, 34}
\definecolor{mycodebackground}{RGB}{245, 245, 245}

% Set Theme Colors
\setbeamercolor{structure}{fg=myblue}
\setbeamercolor{frametitle}{fg=white, bg=myblue}
\setbeamercolor{title}{fg=myblue}
\setbeamercolor{section in toc}{fg=myblue}
\setbeamercolor{item projected}{fg=white, bg=myblue}
\setbeamercolor{block title}{bg=myblue!20, fg=myblue}
\setbeamercolor{block body}{bg=myblue!10}
\setbeamercolor{alerted text}{fg=myorange}

% Set Fonts
\setbeamerfont{title}{size=\Large, series=\bfseries}
\setbeamerfont{frametitle}{size=\large, series=\bfseries}
\setbeamerfont{caption}{size=\small}
\setbeamerfont{footnote}{size=\tiny}

% Document Start
\begin{document}

\frame{\titlepage}

\begin{frame}[fragile]
    \titlepage
\end{frame}

\begin{frame}[fragile]
    \frametitle{Introduction to Google AutoML}
    \begin{block}{What is Google AutoML?}
        Google AutoML is a suite of machine learning tools designed for users with limited programming expertise. It facilitates the creation of custom machine learning models by automating various stages of the ML workflow.
    \end{block}
\end{frame}

\begin{frame}[fragile]
    \frametitle{Why is Google AutoML Important?}
    \begin{enumerate}
        \item \textbf{Democratization of AI}: Empowers non-programmers to leverage AI without needing extensive technical knowledge.
        \item \textbf{Time Efficiency}: Reduces model development time from weeks to hours.
        \item \textbf{Optimized Performance}: Employs advanced algorithms for higher accuracy.
    \end{enumerate}
\end{frame}

\begin{frame}[fragile]
    \frametitle{Key Features of Google AutoML}
    \begin{itemize}
        \item \textbf{User-Friendly Interface}: Allows dataset uploads and model selection without coding.
        \item \textbf{Automatic Data Preprocessing}: Ensures datasets are clean and ready for training.
        \item \textbf{Model Selection and Tuning}: Selects the best model and tunes hyperparameters automatically.
        \item \textbf{Integration with Google Cloud}: Facilitates deployment and scaling through seamless integration.
    \end{itemize}
\end{frame}

\begin{frame}[fragile]
    \frametitle{Example Use Case}
    \begin{block}{Scenario: Retail Business}
        A retail company wants to predict customer preferences for product recommendations:
    \end{block}
    \begin{enumerate}
        \item \textbf{Data Collection}: Collect customer transaction data.
        \item \textbf{AutoML Process}: 
            \begin{itemize}
                \item Upload datasets to Google AutoML.
                \item AutoML analyzes data and trains a recommendation model.
            \end{itemize}
        \item \textbf{Outcome}: A model that predicts customer preferences enhances personalized marketing strategies.
    \end{enumerate}
\end{frame}

\begin{frame}[fragile]
    \frametitle{Key Points to Emphasize}
    \begin{itemize}
        \item \textbf{Accessibility}: Opens machine learning to non-technical users.
        \item \textbf{Efficiency}: Saves time and resources with automated workflows.
        \item \textbf{No Prior Experience Needed}: Users can participate without a coding or data science background.
    \end{itemize}
\end{frame}

\begin{frame}[fragile]
    \frametitle{Conclusion}
    Google AutoML represents a breakthrough in making machine learning accessible to everyone, promoting innovation and experimentation. Its powerful capabilities enable businesses and individuals to harness AI, regardless of their technical skills.
\end{frame}

\begin{frame}[fragile]
    \frametitle{Next Topic}
    This slide sets the stage for the next topic: \textbf{What is AutoML?} We will delve deeper into the definition and significance of AutoML in the context of machine learning automation.
\end{frame}

\begin{frame}[fragile]
    \frametitle{What is AutoML? - Part 1}
    \begin{block}{Definition of AutoML}
        AutoML, or Automated Machine Learning, refers to the process of automating the end-to-end workflow of applying machine learning to real-world problems. This includes stages such as:
        \begin{itemize}
            \item Data preprocessing
            \item Model selection
            \item Training
            \item Hyperparameter tuning
            \item Evaluation
        \end{itemize}
        The goal is to simplify complex and time-consuming tasks that often require expertise in coding and statistics.
    \end{block}
\end{frame}

\begin{frame}[fragile]
    \frametitle{What is AutoML? - Part 2}
    \begin{block}{Significance of AutoML}
        \begin{itemize}
            \item \textbf{Accessibility:} Enables non-technical users to build models via intuitive interfaces, democratizing machine learning.
            \item \textbf{Efficiency:} Automates repetitive tasks, accelerating the model development lifecycle.
            \item \textbf{Innovation:} Provides organizations the capability to apply machine learning in various domains, enhancing competitive advantages.
        \end{itemize}
    \end{block}
\end{frame}

\begin{frame}[fragile]
    \frametitle{What is AutoML? - Part 3}
    \begin{block}{Example Scenario}
        Consider a small business owner wanting to predict next month's sales. Typically, this involves:
        \begin{itemize}
            \item Data cleaning
            \item Choosing a model
            \item Fine-tuning parameters
        \end{itemize}
        With AutoML, they can simply upload their data, select a few parameters through an easy interface, and let the system create a predictive model effortlessly.
    \end{block}

    \begin{block}{Key Points to Emphasize}
        \begin{itemize}
            \item \textbf{Automated Pipeline:} Simplifies tasks from start to finish.
            \item \textbf{Visualization:} Provides visual insights into data and model performance for better comprehension.
            \item \textbf{Adaptability:} Supports various data types and problem domains.
        \end{itemize}
    \end{block}
\end{frame}

\begin{frame}[fragile]
    \frametitle{Key Features of Google AutoML - Introduction}
    Google AutoML is a suite of machine learning products designed to help developers with limited ML expertise to create high-quality models. By simplifying complex processes, it democratizes machine learning, allowing anyone to leverage its capabilities for real-world applications.
\end{frame}

\begin{frame}[fragile]
    \frametitle{Key Features of Google AutoML - Data Preparation}
    \begin{enumerate}
        \item \textbf{Data Preparation}:
        \begin{itemize}
            \item \textbf{Automatic Data Labeling}: Leveraging pre-trained models and human reviewers to swiftly label datasets.\\
            \textit{Example:} For image classification, AutoML can tag objects in pictures.
            
            \item \textbf{Data Augmentation}: Enhancing datasets through transformations (e.g., rotation, flipping) to improve model robustness.\\
            \textit{Example:} Augmenting 100 images of cats can generate variations, increasing size and diversity.
        \end{itemize}
    \end{enumerate}
\end{frame}

\begin{frame}[fragile]
    \frametitle{Key Features of Google AutoML - Model Training and Evaluation}
    \begin{enumerate}
        \setcounter{enumi}{2}
        \item \textbf{Model Training}:
        \begin{itemize}
            \item \textbf{Automated Model Architecture Search}: Selecting optimal model architectures automatically for the data types (e.g., image vs. text).
            \item \textbf{Transfer Learning}: Fine-tuning pre-trained models for specific datasets, crucial when data is limited.\\
            \textit{Example:} Adapting a general dog classifier to recognize specific breeds using your images.
        \end{itemize}

        \item \textbf{Model Evaluation}:
        \begin{itemize}
            \item \textbf{Built-in Performance Metrics}: Easy-to-understand metrics like accuracy, precision, recall, and F1 Score.
            \item \textbf{Feedback Loop}: Continuously refine models based on evaluation outcomes.\\
            \textit{Example:} Improving a sentiment analysis model's precision of 85% by optimizing data or model parameters.
        \end{itemize}
    \end{enumerate}
\end{frame}

\begin{frame}[fragile]
    \frametitle{Key Points and Conclusion}
    \begin{itemize}
        \item \textbf{User-Friendly Interface}: Simplifies engagement with machine learning for all users.
        \item \textbf{Scalability}: Easily scales from small experiments to large deployments with minimal adjustments.
        \item \textbf{Integration}: Seamless with Google Cloud services for enhanced functionality.
    \end{itemize}
    
    \textbf{Conclusion}: Google AutoML empowers users by simplifying the intricacies of machine learning, allowing potent model development even for those with minimal prior experience. By focusing on data preparation, automated training processes, and straightforward evaluation techniques, it opens doors to innovative solutions in countless domains.
\end{frame}

\begin{frame}[fragile]
    \frametitle{Getting Started with Google AutoML - Overview}
    \begin{block}{Overview}
        Google AutoML is a suite of machine learning products that enable developers with limited ML expertise to train high-quality models tailored to their needs. 
        This guide will walk you through setting up and accessing Google AutoML effectively for your projects.
    \end{block}
\end{frame}

\begin{frame}[fragile]
    \frametitle{Getting Started with Google AutoML - Step 1 to 3}
    \begin{enumerate}
        \item \textbf{Create a Google Cloud Account}
        \begin{itemize}
            \item Visit \textbf{Google Cloud Platform (GCP)}: 
            \href{https://cloud.google.com/}{https://cloud.google.com/}
            \item Sign up for free and receive credits.
            \item Access the Google Cloud Console.
        \end{itemize}

        \item \textbf{Create a New Project}
        \begin{itemize}
            \item In the console, click the dropdown at the top of the page.
            \item Select “New Project” and provide a unique name.
            \item Click “Create” to establish your project.
        \end{itemize}
        
        \item \textbf{Enable AutoML API}
        \begin{itemize}
            \item Go to “APIs \& Services” > “Library”.
            \item Search for “AutoML” and select the AutoML API.
            \item Click “Enable” to activate AutoML services for your project.
        \end{itemize}
    \end{enumerate}
\end{frame}

\begin{frame}[fragile]
    \frametitle{Getting Started with Google AutoML - Step 4 to 8}
    \begin{enumerate}
        \setcounter{enumi}{3}
        \item \textbf{Set Up Billing}
        \begin{itemize}
            \item Navigate to the “Billing” section in the menu.
            \item Link your billing account to cover AutoML usage.
        \end{itemize}

        \item \textbf{Navigate to AutoML}
        \begin{itemize}
            \item From the dashboard, select “AI \& Machine Learning”, then “Automated ML”.
            \item Choose the type of AutoML solution required (e.g., Vision or Natural Language).
        \end{itemize}

        \item \textbf{Create and Manage Datasets}
        \begin{itemize}
            \item Click “Create Dataset” and upload your data.
            \item Ensure your data is in the required format (CSV, JSON, etc.).
        \end{itemize}

        \item \textbf{Model Training}
        \begin{itemize}
            \item Select “Train Model” after configuring your dataset.
            \item Customize training settings as needed.
        \end{itemize}
        
        \item \textbf{Monitor and Evaluate}
        \begin{itemize}
            \item Review evaluation metrics like accuracy, precision, and recall post-training.
            \item Select the best-suited model based on results.
        \end{itemize}
    \end{enumerate}
\end{frame}

\begin{frame}[fragile]
    \frametitle{Key Points and Example Scenario}
    \begin{block}{Key Points to Emphasize}
        \begin{itemize}
            \item User-friendly interface for users with limited ML backgrounds.
            \item Importance of API integration for leveraging Google Cloud services.
            \item Continuous learning through model monitoring and retraining.
        \end{itemize}
    \end{block}

    \begin{block}{Example Scenario}
        Imagine developing an image classification application to identify plant species. 
        By following the steps outlined, you can create a tailored model with Google AutoML Vision, even with minimal ML knowledge.
    \end{block}
\end{frame}

\begin{frame}[fragile]
    \frametitle{Next Steps}
    \begin{block}{Next Steps}
        After training your model, focus on \textbf{Collecting and Preparing Data} effectively 
        to enhance model performance. This topic will be covered in the next slide.
    \end{block}
\end{frame}

\begin{frame}[fragile]
    \frametitle{Collecting and Preparing Data - Introduction}
    \begin{block}{Introduction}
        To effectively leverage Google AutoML, the quality of your data is paramount. This section discusses how to collect and prepare datasets that are suitable for training machine learning models.
    \end{block}
\end{frame}

\begin{frame}[fragile]
    \frametitle{Collecting and Preparing Data - Data Collection Methods}
    \begin{enumerate}
        \item \textbf{Surveys and Questionnaires:} 
        \begin{itemize}
            \item Gather data directly from users, useful for sentiment analysis or feedback.
            \item \textit{Example:} A survey collecting user preferences for a new product.
        \end{itemize}
        
        \item \textbf{Web Scraping:} 
        \begin{itemize}
            \item Extract data from websites using tools like Beautiful Soup or Scrapy.
            \item \textit{Example:} Collecting product reviews from an e-commerce site.
        \end{itemize}

        \item \textbf{APIs:} 
        \begin{itemize}
            \item Utilize existing data from different services through APIs.
            \item \textit{Example:} Using Twitter API to gather tweets for sentiment analysis.
        \end{itemize}

        \item \textbf{Public Datasets:} 
        \begin{itemize}
            \item Access freely available datasets from platforms like Kaggle or UCI Machine Learning Repository.
            \item \textit{Example:} The Iris dataset for classification tasks.
        \end{itemize}
    \end{enumerate}
\end{frame}

\begin{frame}[fragile]
    \frametitle{Collecting and Preparing Data - Data Cleaning Steps}
    \begin{enumerate}
        \item \textbf{Removing Duplicates:} Ensure unique entries to prevent bias.
        \begin{itemize}
            \item \textit{Tip:} Use commands like \texttt{drop\_duplicates()} in Python’s Pandas.
        \end{itemize}

        \item \textbf{Handling Missing Values:} Decide whether to fill missing values or remove entries.
        \begin{itemize}
            \item \textit{Example:} Filling missing survey responses with average values.
        \end{itemize}

        \item \textbf{Data Normalization:} Scale numeric values for uniformity.
        \begin{itemize}
            \item \textit{Tip:} Use Min-Max scaling or Standardization.
            \begin{lstlisting}[language=Python]
from sklearn.preprocessing import MinMaxScaler
scaler = MinMaxScaler()
normalized_data = scaler.fit_transform(raw_data)
            \end{lstlisting}
        \end{itemize}

        \item \textbf{Categorical Encoding:} Convert categorical variables into numerical formats.
        \begin{itemize}
            \item \textit{Example:} Transforming colors "Red", "Blue", and "Green" into binary columns.
        \end{itemize}
    \end{enumerate}
\end{frame}

\begin{frame}[fragile]
    \frametitle{Collecting and Preparing Data - Key Points and Conclusion}
    \begin{block}{Key Points to Emphasize}
        \begin{itemize}
            \item \textbf{Quality Over Quantity:} A smaller, well-prepared dataset often outperforms a larger, unclean one.
            \item \textbf{Iterative Process:} Data preparation is often iterative—expect to refine multiple times.
            \item \textbf{Understand Your Data:} Analyze the dataset's structure before modeling.
        \end{itemize}
    \end{block}

    \begin{block}{Conclusion}
        By carefully collecting and cleaning your datasets, you ensure that Google AutoML can learn effectively and yield reliable predictions. In the next slide, we will explore how to use this prepared data to build your first machine learning model.
    \end{block}
\end{frame}

\begin{frame}[fragile]
    \frametitle{Building Your First Model}
    \begin{block}{Overview}
        In this hands-on exercise, we will use Google AutoML to train your first machine learning model. 
        Google AutoML simplifies the model-building process by automating many complex tasks, allowing you to focus on your dataset.
    \end{block}
\end{frame}

\begin{frame}[fragile]
    \frametitle{Understanding Google AutoML}
    \begin{itemize}
        \item Google AutoML is a suite of machine learning tools.
        \item Allows users to train high-quality models without needing deep expertise in machine learning.
        \item Ideal for individuals or businesses wanting to leverage AI for their specific data-related challenges.
    \end{itemize}
\end{frame}

\begin{frame}[fragile]
    \frametitle{Step-by-Step Guide to Building Your First Model}
    \begin{enumerate}
        \item \textbf{Access Google AutoML:}
        \begin{itemize}
            \item Sign in to your Google Cloud account.
            \item Navigate to the AutoML section in the Google Cloud console.
        \end{itemize}
        
        \item \textbf{Select Your Use Case:}
        \begin{itemize}
            \item Choose the model type: Classification, Regression, or Translation.
            \item Example: Use classification for sentiment analysis of customer reviews.
        \end{itemize}
        
        \item \textbf{Import Your Dataset:}
        \begin{itemize}
            \item Import data from Google Cloud Storage or Google Sheets.
            \item Ensure your data is clean and well-labeled.
        \end{itemize}
        
        \item \textbf{Train Your Model:}
        \begin{itemize}
            \item Click on "Train your model."
            \item Google AutoML selects and tunes algorithms automatically.
            \item Monitor training for insights on performance.
        \end{itemize}
        
        \item \textbf{Evaluate the Model:} 
        \begin{itemize}
            \item After training, evaluate metrics such as accuracy and precision.
            \item A confusion matrix can help visualize performance:
            \end{itemize}
            \begin{block}{Confusion Matrix}
                \begin{equation}
                \begin{array}{|c|c|c|}
                    \hline
                    & \text{Predicted Positive} & \text{Predicted Negative} \\
                    \hline
                    \text{Actual Positive} & TP & FN \\
                    \hline
                    \text{Actual Negative} & FP & TN \\
                    \hline
                \end{array}
                \end{equation}
            \end{block}
        
        \item \textbf{Deploy Your Model:}
        \begin{itemize}
            \item Deploy for predicting once evaluation is satisfactory.
            \item Integrate using the Google AutoML API.
        \end{itemize}
    \end{enumerate}
\end{frame}

\begin{frame}[fragile]
    \frametitle{Understanding Model Outputs - Introduction}
    % Content goes here
    When using Google AutoML, understanding model outputs is crucial for interpreting performance and predictions.
    \begin{itemize}
        \item Model outputs inform decisions and adjustments to enhance your model.
        \item Key components include:
        \begin{itemize}
            \item Predictions
            \item Confidence Scores
            \item Probability Distribution
            \item Evaluation Metrics
        \end{itemize}
    \end{itemize}
\end{frame}

\begin{frame}[fragile]
    \frametitle{Key Components of Model Outputs}
    % Content goes here
    \begin{enumerate}
        \item \textbf{Predictions:}
        \begin{itemize}
            \item Output label based on learned patterns.
            \item Example: 
            \begin{lstlisting}
            Prediction: Dog
            Probability: 95%
            \end{lstlisting}
        \end{itemize}
        
        \item \textbf{Confidence Scores:}
        \begin{itemize}
            \item Indicates certainty in predictions (percentage).
            \item Example: 87% confidence means potential risk for decision.
        \end{itemize}
        
        \item \textbf{Probability Distribution:}
        \begin{itemize}
            \item Ranks predictions for multi-class classification.
            \item Example:
            \begin{lstlisting}
            Class Probabilities:
            - Cat: 0.10
            - Dog: 0.85
            - Others: 0.05
            \end{lstlisting}
        \end{itemize}
        
        \item \textbf{Evaluation Metrics:}
        \begin{itemize}
            \item Metrics like accuracy, precision, recall, F1 score will be discussed next.
        \end{itemize}
    \end{enumerate}
\end{frame}

\begin{frame}[fragile]
    \frametitle{Putting it All Together}
    % Content goes here
    Understanding outputs in context guides decision-making.
    \begin{itemize}
        \item Example Scenario: Predicting customer satisfaction.
        \begin{lstlisting}
        Prediction: Satisfied
        Probability: 92%
        \end{lstlisting}
        \item High confidence informs reliable decisions.
        \item Key Takeaways:
        \begin{itemize}
            \item Familiarize with outputs for better decision-making.
            \item Use confidence scores as a trust metric.
            \item Engage with visualizations for quick performance assessment.
        \end{itemize}
    \end{itemize}
\end{frame}

\begin{frame}[fragile]
    \frametitle{Evaluating Model Performance - Overview}
    \begin{block}{Importance of Model Evaluation}
        Evaluating the performance of machine learning models is crucial for ensuring that the systems we build make accurate predictions. This section discusses the key metrics used to assess the effectiveness of models created using Google AutoML.
    \end{block}
\end{frame}

\begin{frame}[fragile]
    \frametitle{Evaluating Model Performance - Key Metrics}
    \begin{enumerate}
        \item \textbf{Accuracy}
        \begin{itemize}
            \item \textbf{Definition:} Proportion of correct predictions.
            \item \textbf{Formula:}
            \[
            \text{Accuracy} = \frac{\text{True Positives} + \text{True Negatives}}{\text{Total Predictions}}
            \]
            \item \textbf{Example:} 80 out of 100 correct predictions = 80\% accuracy.
        \end{itemize}

        \item \textbf{Precision}
        \begin{itemize}
            \item \textbf{Definition:} Proportion of positive identifications that are correct.
            \item \textbf{Formula:}
            \[
            \text{Precision} = \frac{\text{True Positives}}{\text{True Positives} + \text{False Positives}}
            \]
            \item \textbf{Example:} Precision of 60\% if 30 out of 50 positive predictions are correct.
        \end{itemize}
    \end{enumerate}
\end{frame}

\begin{frame}[fragile]
    \frametitle{Evaluating Model Performance - More Metrics}
    \begin{enumerate}
        \setcounter{enumi}{2}
        \item \textbf{Recall (Sensitivity)}
        \begin{itemize}
            \item \textbf{Definition:} Proportion of actual positives that are correctly identified.
            \item \textbf{Formula:}
            \[
            \text{Recall} = \frac{\text{True Positives}}{\text{True Positives} + \text{False Negatives}}
            \]
            \item \textbf{Example:} Recall of 75\% if 30 out of 40 actual positives are identified.
        \end{itemize}

        \item \textbf{F1 Score}
        \begin{itemize}
            \item \textbf{Definition:} Harmonic mean of precision and recall.
            \item \textbf{Formula:}
            \[
            F1 = 2 \times \frac{\text{Precision} \times \text{Recall}}{\text{Precision} + \text{Recall}}
            \]
            \item \textbf{Example:} F1 Score of approximately 0.67 with precision of 0.6 and recall of 0.75.
        \end{itemize}
    \end{enumerate}
\end{frame}

\begin{frame}[fragile]
    \frametitle{Evaluating Model Performance - Additional Metrics}
    \begin{enumerate}
        \setcounter{enumi}{4}
        \item \textbf{ROC-AUC}
        \begin{itemize}
            \item \textbf{Definition:} Performance measurement of classification models at different thresholds.
            \item \textbf{Interpretation:} 1 indicates a perfect model, and 0.5 indicates random chance.
        \end{itemize}

        \item \textbf{Confusion Matrix}
        \begin{itemize}
            \item \textbf{Definition:} Visual representation of actual vs. predicted classifications.
            \item \textbf{Illustration:}
            \begin{center}
            \begin{tabular}{|c|c|c|}
                \hline
                & \textbf{Predicted Positive} & \textbf{Predicted Negative} \\
                \hline
                \textbf{Actual Positive} & TP & FN \\
                \hline
                \textbf{Actual Negative} & FP & TN \\
                \hline
            \end{tabular}
            \end{center}
        \end{itemize}
        
        \item \textbf{Conclusion}
        \begin{itemize}
            \item Evaluating models should not rely on a single metric. Multiple metrics provide a comprehensive view of performance.
        \end{itemize}
    \end{enumerate}
\end{frame}

\begin{frame}[fragile]
  \frametitle{Real-World Applications of AutoML - Introduction}
  Google AutoML is a suite of machine learning products that allows developers with limited machine learning expertise to train high-quality models tailored to their business needs. 
  \begin{itemize}
      \item \textbf{Automated Process}: Automates model building aspects such as data preprocessing, model selection, and hyperparameter tuning.
      \item \textbf{Accessibility}: Democratizes access to advanced machine learning techniques, enabling easier business adoption of AI solutions.
  \end{itemize}
\end{frame}

\begin{frame}[fragile]
  \frametitle{Real-World Applications of AutoML - Case Studies}
  \begin{enumerate}
    \item \textbf{Healthcare - Disease Prediction}
        \begin{itemize}
          \item \textbf{Application}: Predicting diabetes risk in patients.
          \item \textbf{Outcome}: Identified at-risk individuals for early interventions.
        \end{itemize}

    \item \textbf{Retail - Inventory Management}
        \begin{itemize}
          \item \textbf{Application}: Optimizing stock levels across stores.
          \item \textbf{Outcome}: Reduced overstock and increased sales by 15%.
        \end{itemize}

    \item \textbf{Finance - Fraud Detection}
        \begin{itemize}
          \item \textbf{Application}: Detecting fraudulent transactions in real time.
          \item \textbf{Outcome}: Improved detection rates by 30% and reduced false positives.
        \end{itemize}
  \end{enumerate}
\end{frame}

\begin{frame}[fragile]
  \frametitle{Real-World Applications of AutoML - Continued Case Studies}
  \begin{enumerate}\setcounter{enumi}{3}
    \item \textbf{Manufacturing - Predictive Maintenance}
        \begin{itemize}
          \item \textbf{Application}: Predict when machinery requires maintenance.
          \item \textbf{Outcome}: Reduced downtime by 25% and saved costs on unexpected failures.
        \end{itemize}

    \item \textbf{Marketing - Customer Segmentation}
        \begin{itemize}
          \item \textbf{Application}: Targeted customer segmentation for effective marketing.
          \item \textbf{Outcome}: Achieved a 20% increase in conversion rates.
        \end{itemize}
  \end{enumerate}
\end{frame}

\begin{frame}[fragile]
  \frametitle{Real-World Applications of AutoML - Conclusion}
  The integration of Google AutoML into various sectors showcases its versatility and effectiveness. 
  \begin{block}{Key Takeaway}
      Google AutoML simplifies machine-learning processes and leads to significant improvements in outcomes across industries. Embracing such technologies will be crucial for future innovation and competitiveness.
  \end{block}
\end{frame}

\begin{frame}[fragile]
    \frametitle{Ethical Considerations in AI and AutoML}
    \begin{itemize}
        \item **Defining Ethics in AI**: Morality and responsibilities in designing, using AI systems, including AutoML.
        \item **Importance of Ethics**: Ensures AI systems are responsible, transparent, and fair in their influence on daily life.
    \end{itemize}
\end{frame}

\begin{frame}[fragile]
    \frametitle{Data Privacy}
    \begin{itemize}
        \item **What is Data Privacy?** Protecting personal information from unauthorized access.
        \begin{block}{Example}
            When using AutoML for customer data analysis, anonymize sensitive information (like names and addresses) to safeguard privacy.
        \end{block}
        \item **Key Considerations:**
        \begin{itemize}
            \item **Consent**: Informed data subjects must consent to their data usage.
            \item **Data Management**: Strong data governance policies must be in place.
        \end{itemize}
    \end{itemize}
\end{frame}

\begin{frame}[fragile]
    \frametitle{Bias in Machine Learning}
    \begin{itemize}
        \item **Understanding Bias**: Prejudice ingrained in algorithms due to flawed training data or design choices.
        \begin{block}{Example}
            A recruitment tool trained on historical hiring data may perpetuate gender or racial biases if that data reflects past discrimination.
        \end{block}
        \item **Addressing Bias:**
        \begin{itemize}
            \item **Diverse Datasets**: Curate balanced datasets reflecting diverse populations.
            \item **Regular Audits**: Continuously evaluate AI models to identify and mitigate bias.
        \end{itemize}
    \end{itemize}
\end{frame}

\begin{frame}[fragile]
    \frametitle{Ethical AI Practices and Future Implications}
    \begin{itemize}
        \item **Ethical AI Practices:**
        \begin{itemize}
            \item **Transparency**: Clear AI decision-making processes.
            \item **Accountability**: Guidelines on responsibility for AI outcomes.
        \end{itemize}
        \item **Future Implications**:
        \begin{itemize}
            \item Ethical practices build public trust and acceptance of technology.
            \item **Inspiration for Students**: Consider your role in developing technologies with ethical frameworks. 
        \end{itemize}
    \end{itemize}
\end{frame}

\begin{frame}[fragile]
    \frametitle{Key Takeaways}
    \begin{itemize}
        \item Prioritize data privacy and obtain informed consent.
        \item Proactively address and mitigate bias in AI models.
        \item Foster transparency and accountability in AI development and deployment.
        \item Envision the ethical implications of your work in AI and AutoML.
    \end{itemize}
    
    \begin{block}{Conclusion}
        Ethical considerations help harness the potential of AutoML while ensuring technology serves everyone fairly and responsibly.
    \end{block}
\end{frame}

\begin{frame}[fragile]
  \frametitle{Future Trends in AutoML}
  \begin{itemize}
    \item AutoML (Automated Machine Learning) is rapidly evolving.
    \item Emerging trends enhance accessibility and capabilities in machine learning.
    \item Understanding these trends is crucial for students and practitioners.
  \end{itemize}
\end{frame}

\begin{frame}[fragile]
  \frametitle{Key Trends in AutoML}
  \begin{enumerate}
    \item \textbf{Integration of Deep Learning Architectures}
      \begin{itemize}
        \item Incorporating architectures like Transformers and U-Nets.
        \item Enhances performance in tasks like NLP and image segmentation.
      \end{itemize}
    
    \item \textbf{AutoML for Unstructured Data}
      \begin{itemize}
        \item Expanding capabilities to handle images, audio, and text.
        \item Example: Google AutoML Vision enables easy image classification.
      \end{itemize}
    
    \item \textbf{Improved Model Explainability}
      \begin{itemize}
        \item Focus on interpretability of models.
        \item Example: SHAP methods enhance trust and understanding.
      \end{itemize}
  \end{enumerate}
\end{frame}

\begin{frame}[fragile]
  \frametitle{Further Trends in AutoML}
  \begin{enumerate}[resume]
    \item \textbf{Federated Learning and Privacy-Preserving Techniques}
      \begin{itemize}
        \item Allows decentralized model training on sensitive data.
        \item Relevant in fields like healthcare and finance.
      \end{itemize}
    
    \item \textbf{No-Code and Low-Code Platforms}
      \begin{itemize}
        \item Empowering non-experts to build machine learning models.
        \item Example: Google’s AutoML provides intuitive interfaces for users.
      \end{itemize}
  \end{enumerate}
  
  \begin{block}{Key Points}
    \begin{itemize}
      \item AutoML democratizes access to AI technology for everyone.
      \item Ethical use and interpretability are becoming priorities.
    \end{itemize}
  \end{block}
  
  \begin{block}{Questions to Consider}
    \begin{itemize}
      \item How might improved model explainability change business usage?
      \item What challenges do you foresee with federated learning implementation?
      \item How can no-code platforms bridge the gap between experts and non-experts?
    \end{itemize}
  \end{block}
\end{frame}

\begin{frame}[fragile]
  \frametitle{Conclusion and Next Steps - Recap of Key Learnings}
  
  \begin{enumerate}
      \item \textbf{Understanding AutoML}: 
          \begin{itemize}
              \item AutoML simplifies the ML process by automating model selection, training, and hyperparameter tuning.
              \item Enables non-experts to leverage ML without extensive coding knowledge.
          \end{itemize}
      
      \item \textbf{Google AutoML Overview}: 
          \begin{itemize}
              \item Explored Google’s AutoML tools for images, text, and structured data.
              \item Key features include user-friendly interfaces and integration with Google Cloud services.
          \end{itemize}
      
      \item \textbf{Hands-On Exercises}: 
          \begin{itemize}
              \item Built machine learning models using sample datasets, solidifying understanding and insights into the model-building process.
          \end{itemize}
      
      \item \textbf{Model Evaluation and Understanding Metrics}: 
          \begin{itemize}
              \item Evaluated model performance through metrics like accuracy, precision, recall, and F1-score.
          \end{itemize}
      
      \item \textbf{Real-world Applications}: 
          \begin{itemize}
              \item Case studies illustrated how businesses implement AutoML for various problems.
          \end{itemize}
  \end{enumerate}
\end{frame}

\begin{frame}[fragile]
  \frametitle{Conclusion and Next Steps - Key Points to Emphasize}

  \begin{itemize}
      \item \textbf{Embrace Automation}: 
          \begin{itemize}
              \item Using AutoML reduces time and resources spent on model development.
          \end{itemize}

      \item \textbf{Experimentation is Key}: 
          \begin{itemize}
              \item Try different datasets and settings in AutoML for new insights.
              \item The iterative process is essential for learning and improvement.
          \end{itemize}

      \item \textbf{Continuous Learning}: 
          \begin{itemize}
              \item Stay updated with emerging trends like advances in neural networks (e.g., transformers).
              \item Utilize tutorials, forums, and Google’s documentation.
          \end{itemize}
  \end{itemize}
\end{frame}

\begin{frame}[fragile]
  \frametitle{Next Steps for Application}

  \begin{enumerate}
      \item \textbf{Identify a Problem}:
          \begin{itemize}
              \item Start with a real-world problem (e.g., predicting customer churn).
          \end{itemize}

      \item \textbf{Data Gathering}:
          \begin{itemize}
              \item Collect and preprocess relevant data ensuring cleanliness and structure.
          \end{itemize}
      
      \item \textbf{Utilize Google AutoML}:
          \begin{itemize}
              \item Build models using Google AutoML tools for specific data types.
          \end{itemize}
      
      \item \textbf{Evaluate and Iterate}:
          \begin{itemize}
              \item Assess model performance with discussed metrics and refine as needed.
          \end{itemize}

      \item \textbf{Deployment}:
          \begin{itemize}
              \item Deploy your model using Google Cloud for real-time predictions, considering maintenance.
          \end{itemize}

      \item \textbf{Engage with Community}:
          \begin{itemize}
              \item Join forums and communities focused on AutoML for sharing experiences.
          \end{itemize}
  \end{enumerate}
  
  \textbf{Encouraging Reflection:} 
  \begin{itemize}
      \item What challenges do you foresee in implementing AutoML in your projects?
      \item How can you leverage the skills acquired in future projects or your current role? 
  \end{itemize}
\end{frame}


\end{document}