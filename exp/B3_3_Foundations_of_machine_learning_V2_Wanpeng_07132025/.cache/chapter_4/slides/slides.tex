\documentclass[aspectratio=169]{beamer}

% Theme and Color Setup
\usetheme{Madrid}
\usecolortheme{whale}
\useinnertheme{rectangles}
\useoutertheme{miniframes}

% Additional Packages
\usepackage[utf8]{inputenc}
\usepackage[T1]{fontenc}
\usepackage{graphicx}
\usepackage{booktabs}
\usepackage{listings}
\usepackage{amsmath}
\usepackage{amssymb}
\usepackage{xcolor}
\usepackage{tikz}
\usepackage{pgfplots}
\pgfplotsset{compat=1.18}
\usetikzlibrary{positioning}
\usepackage{hyperref}

% Custom Colors
\definecolor{myblue}{RGB}{31, 73, 125}
\definecolor{mygray}{RGB}{100, 100, 100}
\definecolor{mygreen}{RGB}{0, 128, 0}
\definecolor{myorange}{RGB}{230, 126, 34}
\definecolor{mycodebackground}{RGB}{245, 245, 245}

% Set Theme Colors
\setbeamercolor{structure}{fg=myblue}
\setbeamercolor{frametitle}{fg=white, bg=myblue}
\setbeamercolor{title}{fg=myblue}
\setbeamercolor{section in toc}{fg=myblue}
\setbeamercolor{item projected}{fg=white, bg=myblue}
\setbeamercolor{block title}{bg=myblue!20, fg=myblue}
\setbeamercolor{block body}{bg=myblue!10}
\setbeamercolor{alerted text}{fg=myorange}

% Set Fonts
\setbeamerfont{title}{size=\Large, series=\bfseries}
\setbeamerfont{frametitle}{size=\large, series=\bfseries}
\setbeamerfont{caption}{size=\small}
\setbeamerfont{footnote}{size=\tiny}

% Code Listing Style
\lstdefinestyle{customcode}{
  backgroundcolor=\color{mycodebackground},
  basicstyle=\footnotesize\ttfamily,
  breakatwhitespace=false,
  breaklines=true,
  commentstyle=\color{mygreen}\itshape,
  keywordstyle=\color{blue}\bfseries,
  stringstyle=\color{myorange},
  numbers=left,
  numbersep=8pt,
  numberstyle=\tiny\color{mygray},
  frame=single,
  framesep=5pt,
  rulecolor=\color{mygray},
  showspaces=false,
  showstringspaces=false,
  showtabs=false,
  tabsize=2,
  captionpos=b
}
\lstset{style=customcode}

% Custom Commands
\newcommand{\hilight}[1]{\colorbox{myorange!30}{#1}}
\newcommand{\source}[1]{\vspace{0.2cm}\hfill{\tiny\textcolor{mygray}{Source: #1}}}
\newcommand{\concept}[1]{\textcolor{myblue}{\textbf{#1}}}
\newcommand{\separator}{\begin{center}\rule{0.5\linewidth}{0.5pt}\end{center}}

% Title Page Information
\title[Introduction to Supervised Learning]{Chapter 4: Introduction to Supervised Learning}
\author[J. Smith]{John Smith, Ph.D.}
\institute[University Name]{
  Department of Computer Science\\
  University Name\\
  \vspace{0.3cm}
  Email: email@university.edu\\
  Website: www.university.edu
}
\date{\today}

% Document Start
\begin{document}

\frame{\titlepage}

\begin{frame}[fragile]
    \frametitle{Introduction to Supervised Learning}
    \begin{block}{What is Supervised Learning?}
        Supervised learning involves training an algorithm on labeled data, where each training example is paired with an output label. The objective is to establish a mapping from input features to output labels, allowing predictions on unseen data.
    \end{block}
\end{frame}

\begin{frame}[fragile]
    \frametitle{Significance of Supervised Learning}
    \begin{itemize}
        \item \textbf{Predictive Power:} Useful for applications like disease diagnosis and stock market predictions.
        \item \textbf{Widespread Applications:} Integral to systems like personal assistants (Siri, Alexa) and fraud detection.
        \item \textbf{Data-Driven Decisions:} Helps businesses make informed decisions using predictive analytics.
    \end{itemize}
\end{frame}

\begin{frame}[fragile]
    \frametitle{Distinctions from Unsupervised Learning}
    \begin{itemize}
        \item \textbf{Labeled vs. Unlabeled Data:} 
            \begin{itemize}
                \item Supervised Learning: Uses labeled datasets (input-output pairs).
                \item Unsupervised Learning: Works with unlabeled data to find patterns.
            \end{itemize}
        \item \textbf{Learning Objectives:}
            \begin{itemize}
                \item Supervised Learning: Maps inputs to known outputs.
                \item Unsupervised Learning: Explores data structure without predefined outputs.
            \end{itemize}
    \end{itemize}
\end{frame}

\begin{frame}[fragile]
    \frametitle{Key Points to Remember}
    \begin{enumerate}
        \item Requires datasets with known outputs.
        \item Primarily used for classification and regression tasks.
        \item Common algorithms: linear regression, logistic regression, decision trees, and SVMs.
    \end{enumerate}
\end{frame}

\begin{frame}[fragile]
    \frametitle{Example Scenario}
    \begin{block}{Predicting Student Outcomes}
        - Collect data on study hours, test scores, and outcomes (pass/fail).\\
        - Train the model to identify successful study patterns.\\
        - Utilize the trained model to predict new student outcomes.
    \end{block}
\end{frame}

\begin{frame}[fragile]
    \frametitle{Engaging Questions for Reflection}
    \begin{itemize}
        \item How does knowing the "correct answer" during training improve the learning process?
        \item What types of real-world problems could be solved using supervised learning?
        \item In what ways might the lack of labeled data impact the effectiveness of a model?
    \end{itemize}
\end{frame}

\begin{frame}[fragile]
    \frametitle{What is Supervised Learning?}
    \begin{block}{Definition}
        Supervised learning is a type of machine learning where an algorithm is trained on a labeled dataset. Each input data point is paired with the correct output.
    \end{block}
    
    \begin{itemize}
        \item Inputs (features) are mapped to outputs (labels).
        \item Goal: Make predictions on new, unseen data.
    \end{itemize}
\end{frame}

\begin{frame}[fragile]
    \frametitle{Key Features of Supervised Learning}
    \begin{itemize}
        \item \textbf{Labeled Data:} Contains both inputs (features) and corresponding outputs (labels).
        \item \textbf{Training Process:} The model learns to recognize patterns from labeled data for future predictions.
    \end{itemize}
\end{frame}

\begin{frame}[fragile]
    \frametitle{Process Flow of Supervised Learning}
    \begin{enumerate}
        \item \textbf{Data Collection:} Gather input data and labels.
        \item \textbf{Data Preparation:} Clean and preprocess data; ensure diversity to avoid bias.
        \item \textbf{Model Selection:} Choose appropriate algorithm (e.g., linear regression, decision trees).
        \item \textbf{Training the Model:} Split dataset into training and testing sets; adjust parameters to minimize errors.
        \item \textbf{Evaluation:} Assess performance using metrics like accuracy, precision, and recall.
        \item \textbf{Deployment:} Use the trained model for predictions; monitor and retrain as needed.
    \end{enumerate}
\end{frame}

\begin{frame}[fragile]
    \frametitle{Key Points to Emphasize}
    \begin{itemize}
        \item Driven by relationships between labeled data and outputs.
        \item Widely applicable in fields like finance, healthcare, and marketing.
        \item Model performance depends heavily on the quality and quantity of labeled data.
    \end{itemize}
\end{frame}

\begin{frame}[fragile]
    \frametitle{Example: Email Classification}
    \begin{block}{Scenario}
        Predict whether an email is spam or not.
    \end{block}
    
    \begin{itemize}
        \item \textbf{Input Features:} Words in the email, sender’s address.
        \item \textbf{Label:} Spam (1) or Not-Spam (0).
    \end{itemize}
    
    Through training on labeled examples, the model learns to classify new emails based on learned patterns.
\end{frame}

\begin{frame}[fragile]{Key Terminologies - Introduction}
    \begin{block}{Introduction}
        In supervised learning, understanding the terminology is crucial to master the concepts effectively. Here are some essential terms that will help you grasp the foundational elements of this learning paradigm:
    \end{block}
\end{frame}

\begin{frame}[fragile]{Key Terminologies - Labels}
    \begin{block}{1. Labels}
        \begin{itemize}
            \item \textbf{Definition}: The labels are the output or the target variable that the model aims to predict. They are known values in the dataset.
            \item \textbf{Example}: In a dataset to predict house prices, the label would be the actual price of the house.
        \end{itemize}
    \end{block}
\end{frame}

\begin{frame}[fragile]{Key Terminologies - Features}
    \begin{block}{2. Features}
        \begin{itemize}
            \item \textbf{Definition}: Features are the input variables used by the model to make predictions. They provide the necessary information from which the model learns.
            \item \textbf{Example}: In the house price example, features might include the number of bedrooms, square footage, and location.
        \end{itemize}
    \end{block}
\end{frame}

\begin{frame}[fragile]{Key Terminologies - Training and Testing Sets}
    \begin{block}{3. Training Set}
        \begin{itemize}
            \item \textbf{Definition}: The training set is a subset of the data used to train the model. It includes both features and labels, allowing the model to learn patterns.
            \item \textbf{Example}: If you have 1,000 house listings, you might use 800 of them as the training set to teach your model how to estimate prices.
        \end{itemize}
    \end{block}
    
    \begin{block}{4. Testing Set}
        \begin{itemize}
            \item \textbf{Definition}: The testing set is a separate subset of the data that the model has never seen before. It is used to evaluate the model's performance and generalization capabilities.
            \item \textbf{Example}: Continuing with the house prices scenario, you could reserve 200 listings as your testing set to see how accurately your model can predict prices on unseen data.
        \end{itemize}
    \end{block}
\end{frame}

\begin{frame}[fragile]{Key Points to Emphasize}
    \begin{itemize}
        \item \textbf{Supervised Learning Basics}: Each of these components plays a pivotal role in the supervised learning framework, where the goal is to learn from labeled data.
        \item \textbf{Importance of Separation}: It's essential to keep your training and testing sets separate to ensure that your model's performance assessment is unbiased.
        \item \textbf{Real-World Application}: These terms are foundational in various applications, such as facial recognition, speech recognition, and medical diagnosis.
    \end{itemize}
\end{frame}

\begin{frame}[fragile]{Key Terminologies - Summary}
    \begin{block}{Summary}
        These key terms form the building blocks of supervised learning. By understanding and applying them, you can grasp how models learn from data and how to evaluate their performance effectively.
    \end{block}
\end{frame}

\begin{frame}[fragile]
    \frametitle{Types of Supervised Learning}
    \begin{block}{Introduction to Supervised Learning}
        Supervised learning is a branch of machine learning where models are trained on labeled data. This data consists of input-output pairs that allow the model to learn from examples.
    \end{block}
    \begin{itemize}
        \item Main types: Classification and Regression
    \end{itemize}
\end{frame}

\begin{frame}[fragile]
    \frametitle{Classification Problems}
    \begin{block}{Definition}
        Classification involves categorizing data into predefined classes or labels, predicting categories for new instances based on learned patterns.
    \end{block}
    \begin{itemize}
        \item Output: Discrete labels (e.g., yes/no, spam/not spam)
        \item Use: Identify group membership
    \end{itemize}
    \begin{block}{Examples}
        \begin{itemize}
            \item Email Classification: Spam detection
            \item Image Recognition: Classifying objects (e.g., cat, dog)
            \item Medical Diagnosis: Disease classification based on symptoms
        \end{itemize}
    \end{block}
    \begin{block}{Analogy}
        A sorting hat categorizing students into houses based on attributes!
    \end{block}
\end{frame}

\begin{frame}[fragile]
    \frametitle{Regression Problems}
    \begin{block}{Definition}
        Regression involves predicting a continuous output, estimating numeric values rather than categories.
    \end{block}
    \begin{itemize}
        \item Output: Continuous values (e.g., price, temperature)
        \item Use: Estimate quantities
    \end{itemize}
    \begin{block}{Examples}
        \begin{itemize}
            \item House Price Prediction: Estimating price based on features
            \item Weather Forecasting: Predicting temperature or rainfall
            \item Stock Market Predictions: Forecasting future stock prices
        \end{itemize}
    \end{block}
    \begin{block}{Analogy}
        A weather forecaster analyzing past patterns to predict tomorrow's temperature!
    \end{block}
\end{frame}

\begin{frame}[fragile]
    \frametitle{Summary Table}
    \begin{table}[ht]
        \centering
        \begin{tabular}{|l|l|l|}
            \hline
            Aspect          & Classification                               & Regression                 \\ \hline
            Output Type     & Discrete labels                             & Continuous values          \\ \hline
            Goal             & Assign to a category                      & Predict a numeric value    \\ \hline
            Examples         & Spam detection, image categorization      & Price forecasting, temperature prediction \\ \hline
        \end{tabular}
    \end{table}
\end{frame}

\begin{frame}[fragile]
    \frametitle{Conclusion and Questions}
    \begin{block}{Conclusion}
        Understanding the distinction between classification and regression is crucial for selecting the right supervised learning algorithm. Both types significantly enhance decision-making based on data.
    \end{block}
    \begin{itemize}
        \item Questions to Ponder:
            \begin{enumerate}
                \item In what other areas could classification be applied?
                \item How might regression improve our understanding of daily data trends?
            \end{enumerate}
    \end{itemize}
\end{frame}

\begin{frame}[fragile]
    \frametitle{Common Supervised Learning Algorithms - Introduction}
    \begin{block}{Overview}
        Supervised learning involves training models on labeled data. 
        Below, we introduce three widely-used algorithms:
        \begin{itemize}
            \item \textbf{Decision Trees}
            \item \textbf{Support Vector Machines (SVM)}
            \item \textbf{Neural Networks}
        \end{itemize}
    \end{block}
\end{frame}

\begin{frame}[fragile]
    \frametitle{Common Supervised Learning Algorithms - Decision Trees}
    \begin{block}{Concept}
        Decision trees use a flowchart-like structure to make decisions based on feature values. 
    \end{block}

    \begin{block}{Example}
        Predicting whether a customer will buy a product based on age and income. 
        \begin{itemize}
            \item Starts with: \textit{Is the income over \$50,000?}
            \item Leads to further questions until a final decision is reached (Buy/No-Buy).
        \end{itemize}
    \end{block}

    \begin{block}{Key Points}
        \begin{itemize}
            \item \textbf{Advantages:} Easy to understand; handles various data types.
            \item \textbf{Disadvantages:} Prone to overfitting; sensitive to data changes.
        \end{itemize}
    \end{block}
\end{frame}

\begin{frame}[fragile]
    \frametitle{Common Supervised Learning Algorithms - Support Vector Machines (SVM)}
    \begin{block}{Concept}
        SVMs classify data by finding an optimal hyperplane that separates different classes.
    \end{block}

    \begin{block}{Example}
        In a flower classification task based on petal length and width:
        \begin{itemize}
            \item Creates a line (2D) or plane (3D) to separate species.
            \item Maximizes the margin between classes and the hyperplane.
        \end{itemize}
    \end{block}

    \begin{block}{Key Points}
        \begin{itemize}
            \item \textbf{Advantages:} Effective in high dimensions; robust against overfitting.
            \item \textbf{Disadvantages:} Less effective on large datasets; kernel selection is complex.
        \end{itemize}
    \end{block}
\end{frame}

\begin{frame}[fragile]
    \frametitle{Common Supervised Learning Algorithms - Neural Networks}
    \begin{block}{Concept}
        Neural networks consist of interconnected nodes organized in layers, inspired by the human brain.
    \end{block}

    \begin{block}{Example}
        Used for image recognition (e.g., classifying cats vs. dogs):
        \begin{itemize}
            \item Transforms raw pixel data through layers, extracting features at each level.
        \end{itemize}
    \end{block}

    \begin{block}{Key Points}
        \begin{itemize}
            \item \textbf{Advantages:} Captures complex patterns; great for large datasets.
            \item \textbf{Disadvantages:} Requires large datasets; computationally intensive; difficult to interpret.
        \end{itemize}
    \end{block}
\end{frame}

\begin{frame}[fragile]
    \frametitle{Common Supervised Learning Algorithms - Conclusion}
    \begin{block}{Summary}
        Understanding these algorithms lays the foundation for advanced topics in supervised learning.
    \end{block}

    \begin{block}{Questions to Consider}
        \begin{itemize}
            \item What problems are best for Decision Trees, SVMs, or Neural Networks?
            \item Can you identify recent applications of neural networks (e.g., transformers, U-nets)?
        \end{itemize}
    \end{block}
\end{frame}

\begin{frame}[fragile]
    \frametitle{The Role of Data in Supervised Learning}
    \begin{block}{Understanding the Importance}
        Supervised learning relies heavily on the quality, quantity, and relevance of data for training algorithms. Let's explore these key aspects:
    \end{block}
\end{frame}

\begin{frame}[fragile]
    \frametitle{Data Quality}
    \begin{itemize}
        \item \textbf{Definition:} Accuracy, completeness, and reliability of training data.
        \item \textbf{Importance:} High-quality data leads to robust models and accurate predictions.
        \item \textbf{Example:} Erroneous entries, like negative prices in a house price prediction model, can compromise performance.
    \end{itemize}
    \begin{block}{Key Points}
        \begin{itemize}
            \item Clean data: Remove duplicates and correct errors.
            \item Valid data: Ensure labels are consistent and represent real-world scenarios.
        \end{itemize}
    \end{block}
\end{frame}

\begin{frame}[fragile]
    \frametitle{Data Quantity}
    \begin{itemize}
        \item \textbf{Definition:} The amount of data necessary for effective model training.
        \item \textbf{Importance:} More data can enhance model performance, particularly for complex algorithms.
        \item \textbf{Example:} A handwritten digit recognition model trained on thousands of images performs better than one trained on just a few dozen.
    \end{itemize}
    \begin{block}{Key Points}
        \begin{itemize}
            \item Balance: While more data is generally beneficial, it is crucial to maintain quality.
            \item Variety: Diverse data types can improve generalization.
        \end{itemize}
    \end{block}
\end{frame}

\begin{frame}[fragile]
    \frametitle{Data Relevance}
    \begin{itemize}
        \item \textbf{Definition:} How closely features in your dataset align with the problem being solved.
        \item \textbf{Importance:} Relevant data ensures that the model captures the right patterns for accurate predictions.
        \item \textbf{Example:} In spam detection, features like subject lines and sender addresses are relevant, while unrelated features (e.g., email length) may be less helpful.
    \end{itemize}
    \begin{block}{Key Points}
        \begin{itemize}
            \item Feature selection: Choose the most relevant features to enhance model interpretability and reduce overfitting.
            \item Domain knowledge: Utilize insights from the field to ensure data relevance.
        \end{itemize}
    \end{block}
\end{frame}

\begin{frame}[fragile]
    \frametitle{Conclusion}
    In supervised learning, the quality, quantity, and relevance of data are critical to your model's success. Investing time in curating and preprocessing data can lead to improved predictions and insights over time.
\end{frame}

\begin{frame}[fragile]
    \frametitle{Discussion Questions}
    \begin{enumerate}
        \item What are the potential consequences of using low-quality data?
        \item How does increasing the dataset size affect model training in real-world applications?
        \item What strategies can we employ to ensure data relevance when designing datasets for specific tasks?
    \end{enumerate}
\end{frame}

\begin{frame}[fragile]
    \frametitle{Evaluation Metrics - Introduction}
    \begin{itemize}
        \item Evaluation of algorithm performance is crucial in supervised learning.
        \item Various metrics quantify this performance.
        \item Focus on four key metrics:
        \begin{itemize}
            \item Accuracy
            \item Precision
            \item Recall
            \item F1-score
        \end{itemize}
    \end{itemize}
    \begin{block}{Overview}
        Each metric offers unique insights into the model's predictive capabilities.
    \end{block}
\end{frame}

\begin{frame}[fragile]
    \frametitle{Evaluation Metrics - Accuracy}
    \begin{itemize}
        \item \textbf{Definition}: Proportion of correct predictions made by the model.
        \item \textbf{Formula}:  
        \[
        \text{Accuracy} = \frac{\text{True Positives} + \text{True Negatives}}{\text{Total Predictions}}
        \]
        \item \textbf{Example}:
        \begin{itemize}
            \item True Positives (TP): 40
            \item True Negatives (TN): 50
            \item False Positives (FP): 10
            \item False Negatives (FN): 5
            \item Total Predictions = TP + TN + FP + FN = 105
        \end{itemize}
        \[
        \text{Accuracy} = \frac{40 + 50}{105} \approx 0.857 \text{ (or 85.7\%)}
        \]
        \item \textbf{Key Point}: Can be misleading for imbalanced datasets.
    \end{itemize}
\end{frame}

\begin{frame}[fragile]
    \frametitle{Evaluation Metrics - Precision, Recall, F1-Score}
    \begin{itemize}
        \item \textbf{Precision}:
        \begin{itemize}
            \item Definition: Quality of positive predictions.
            \item Formula:
            \[
            \text{Precision} = \frac{\text{True Positives}}{\text{True Positives} + \text{False Positives}}
            \]
            \item Example: 
            \[
            \text{Precision} = \frac{40}{40 + 10} = 0.8 \text{ (or 80\%)}
            \]
            \item Key Point: Important when false positives are costly.
        \end{itemize}
        
        \item \textbf{Recall}:
        \begin{itemize}
            \item Definition: Ability to identify all relevant instances.
            \item Formula:
            \[
            \text{Recall} = \frac{\text{True Positives}}{\text{True Positives} + \text{False Negatives}}
            \]
            \item Example: 
            \[
            \text{Recall} = \frac{40}{40 + 5} \approx 0.889 \text{ (or 88.9\%)}
            \]
            \item Key Point: Critical when false negatives are costly.
        \end{itemize}
        
        \item \textbf{F1-Score}:
        \begin{itemize}
            \item Definition: Harmonic mean of precision and recall.
            \item Formula:
            \[
            \text{F1-Score} = 2 \times \frac{\text{Precision} \times \text{Recall}}{\text{Precision} + \text{Recall}}
            \]
            \item Example: 
            \[
            \text{F1-Score} = 2 \times \frac{0.8 \times 0.889}{0.8 + 0.889} \approx 0.842 \text{ (or 84.2\%)}
            \]
            \item Key Point: Robust measure for imbalanced classes.
        \end{itemize}
    \end{itemize}
\end{frame}

\begin{frame}[fragile]
    \frametitle{Supervised Learning Use Cases - Overview}
    \begin{itemize}
        \item Supervised learning involves training algorithms on labeled datasets.
        \item Each training example has corresponding input and output values.
        \item The goal: to effectively map inputs to correct outputs.
    \end{itemize}
\end{frame}

\begin{frame}[fragile]
    \frametitle{Supervised Learning Use Cases - Applications}
    \begin{enumerate}
        \item \textbf{Healthcare}
            \begin{itemize}
                \item Diagnosis assistance using patient data.
                \item Example: Analyzing mammogram images to detect breast cancer.
            \end{itemize}
        \item \textbf{Finance}
            \begin{itemize}
                \item Fraud detection through historical transaction data.
                \item Example: Flagging unusual credit card transactions.
            \end{itemize}
        \item \textbf{Social Media}
            \begin{itemize}
                \item Content recommendation based on user interaction data.
                \item Example: Suggesting posts on platforms like Facebook or Instagram.
            \end{itemize}
    \end{enumerate}
\end{frame}

\begin{frame}[fragile]
    \frametitle{Supervised Learning Use Cases - Key Points}
    \begin{itemize}
        \item \textbf{Labeled Data Importance:} 
            \begin{itemize}
                \item Model effectiveness depends on data quality and size.
            \end{itemize}
        \item \textbf{Diverse Applications:}
            \begin{itemize}
                \item Widely applicable across sectors, including finance and social media.
            \end{itemize}
        \item \textbf{Performance Measurement:}
            \begin{itemize}
                \item Metrics like accuracy, precision, and recall are crucial.
            \end{itemize}
    \end{itemize}
\end{frame}

\begin{frame}[fragile]
    \frametitle{Supervised Learning Use Cases - Conclusion}
    \begin{itemize}
        \item Supervised learning is vital for real-world applications.
        \item It enhances efficiency, accuracy, and decision-making.
        \item Ethical considerations in deploying these models are essential.
    \end{itemize}
\end{frame}

\begin{frame}[fragile]
    \frametitle{Ethical Considerations in Supervised Learning}
    Supervised learning has vast applications but raises important ethical concerns. 
    This discussion focuses on:
    \begin{itemize}
        \item Bias in data and algorithms
        \item Need for fairness and transparency
    \end{itemize}
\end{frame}

\begin{frame}[fragile]
    \frametitle{Understanding Bias in Data and Algorithms}
    \begin{block}{Bias in Data}
        Data can reflect existing prejudices or stereotypes, leading to inequitable outcomes. 
    \end{block}
    \begin{block}{Bias in Algorithms}
        Algorithms can introduce bias through learning associations between features and outcomes.
    \end{block}
    
    \textbf{Example:} A credit scoring algorithm trained on data from high-income individuals may unfairly deny loans to lower-income applicants despite comparable creditworthiness.
\end{frame}

\begin{frame}[fragile]
    \frametitle{Emphasizing Fairness}
    \begin{itemize}
        \item \textbf{Fairness:} Algorithms must treat different groups equally.
        \item \textbf{Metrics for Fairness:} Consider demographic parity or equal opportunity.
        \item \textbf{Strategies to Mitigate Bias:}
        \begin{itemize}
            \item Diverse datasets to capture varied populations
            \item Regular bias audits for testing
        \end{itemize}
    \end{itemize}
    
    \textbf{Example:} Implementing fairness constraints in loan approvals can prevent discrimination.
\end{frame}

\begin{frame}[fragile]
    \frametitle{Importance of Transparency}
    \begin{itemize}
        \item \textbf{Transparency:} Clear documentation on how a model works fosters understanding.
        \item \textbf{Explainability:} Users should be able to interpret model decisions (e.g., using SHAP).
    \end{itemize}

    \textbf{Example:} An explainable loan approval system can provide applicants with rationales for decisions made.
\end{frame}

\begin{frame}[fragile]
    \frametitle{Key Points and Closing Thought}
    \begin{itemize}
        \item Bias in data and algorithms can lead to inequitable treatment.
        \item Strategies such as diversified datasets and bias audits promote fairness.
        \item Transparency in AI systems builds trust and enhances understanding.
    \end{itemize}

    \textbf{Closing Thought:} Fostering ethical practices in supervised learning is a responsibility for a just society.
\end{frame}

\begin{frame}[fragile]
    \frametitle{Summary \& Key Takeaways - Understanding Supervised Learning}
    \begin{block}{Definition}
        Supervised learning is a type of machine learning where an algorithm learns a mapping from input features to an output label by using labeled training data. Each input is associated with the correct output, allowing the model to learn patterns.
    \end{block}
    \begin{itemize}
        \item \textbf{Key Concept:} Labeled Data - Datasets that include input-output pairs.
        \item \textbf{Example:} Classifying emails as 'spam' or 'not spam' using labeled emails.
    \end{itemize}
\end{frame}

\begin{frame}[fragile]
    \frametitle{Summary \& Key Takeaways - Algorithms and Applications}
    \begin{itemize}
        \item \textbf{Common Algorithms:}
        \begin{itemize}
            \item \textbf{Linear Regression:} For predicting continuous outcomes (e.g., house prices).
            \item \textbf{Logistic Regression:} For binary classification tasks (e.g., disease prediction).
            \item \textbf{Decision Trees:} Structured to make decisions based on features (e.g., product purchase prediction).
        \end{itemize}
        
        \item \textbf{Real-World Applications:}
        \begin{itemize}
            \item Healthcare: Predicting patient outcomes.
            \item Finance: Credit scoring for loan eligibility.
            \item Marketing: Customer segmentation and targeting.
            \item Autonomous Vehicles: Object recognition.
        \end{itemize}
    \end{itemize}
\end{frame}

\begin{frame}[fragile]
    \frametitle{Summary \& Key Takeaways - Discussion and Conclusion}
    \begin{itemize}
        \item Supervised learning utilizes labeled data to train models.
        \item Various algorithms tackle different problems (regression vs classification).
        \item Applications enhance decision-making across various industries.
    \end{itemize}
    
    \begin{block}{Engagement Questions}
        \begin{itemize}
            \item How might supervised learning change the future of a specific field you are interested in?
            \item What ethical implications should we consider in its applications?
            \item Can potential biases arise in your examples or applications?
        \end{itemize}
    \end{block}
    
    \begin{block}{Conclusion}
        Reflect on how you would apply these concepts practically, and let's open the floor for further discussion and questions!
    \end{block}
\end{frame}


\end{document}