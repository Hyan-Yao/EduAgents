\documentclass[aspectratio=169]{beamer}

% Theme and Color Setup
\usetheme{Madrid}
\usecolortheme{whale}
\useinnertheme{rectangles}
\useoutertheme{miniframes}

% Additional Packages
\usepackage[utf8]{inputenc}
\usepackage[T1]{fontenc}
\usepackage{graphicx}
\usepackage{booktabs}
\usepackage{listings}
\usepackage{amsmath}
\usepackage{amssymb}
\usepackage{xcolor}
\usepackage{tikz}
\usepackage{pgfplots}
\pgfplotsset{compat=1.18}
\usetikzlibrary{positioning}
\usepackage{hyperref}

% Custom Colors
\definecolor{myblue}{RGB}{31, 73, 125}
\definecolor{mygray}{RGB}{100, 100, 100}
\definecolor{mygreen}{RGB}{0, 128, 0}
\definecolor{myorange}{RGB}{230, 126, 34}
\definecolor{mycodebackground}{RGB}{245, 245, 245}

% Set Theme Colors
\setbeamercolor{structure}{fg=myblue}
\setbeamercolor{frametitle}{fg=white, bg=myblue}
\setbeamercolor{title}{fg=myblue}
\setbeamercolor{section in toc}{fg=myblue}
\setbeamercolor{item projected}{fg=white, bg=myblue}
\setbeamercolor{block title}{bg=myblue!20, fg=myblue}
\setbeamercolor{block body}{bg=myblue!10}
\setbeamercolor{alerted text}{fg=myorange}

% Set Fonts
\setbeamerfont{title}{size=\Large, series=\bfseries}
\setbeamerfont{frametitle}{size=\large, series=\bfseries}
\setbeamerfont{caption}{size=\small}
\setbeamerfont{footnote}{size=\tiny}

% Code Listing Style
\lstdefinestyle{customcode}{
  backgroundcolor=\color{mycodebackground},
  basicstyle=\footnotesize\ttfamily,
  breakatwhitespace=false,
  breaklines=true,
  commentstyle=\color{mygreen}\itshape,
  keywordstyle=\color{blue}\bfseries,
  stringstyle=\color{myorange},
  numbers=left,
  numbersep=8pt,
  numberstyle=\tiny\color{mygray},
  frame=single,
  framesep=5pt,
  rulecolor=\color{mygray},
  showspaces=false,
  showstringspaces=false,
  showtabs=false,
  tabsize=2,
  captionpos=b
}
\lstset{style=customcode}

% Custom Commands
\newcommand{\hilight}[1]{\colorbox{myorange!30}{#1}}
\newcommand{\source}[1]{\vspace{0.2cm}\hfill{\tiny\textcolor{mygray}{Source: #1}}}
\newcommand{\concept}[1]{\textcolor{myblue}{\textbf{#1}}}
\newcommand{\separator}{\begin{center}\rule{0.5\linewidth}{0.5pt}\end{center}}

% Footer and Navigation Setup
\setbeamertemplate{footline}{
  \leavevmode%
  \hbox{%
  \begin{beamercolorbox}[wd=.3\paperwidth,ht=2.25ex,dp=1ex,center]{author in head/foot}%
    \usebeamerfont{author in head/foot}\insertshortauthor
  \end{beamercolorbox}%
  \begin{beamercolorbox}[wd=.5\paperwidth,ht=2.25ex,dp=1ex,center]{title in head/foot}%
    \usebeamerfont{title in head/foot}\insertshorttitle
  \end{beamercolorbox}%
  \begin{beamercolorbox}[wd=.2\paperwidth,ht=2.25ex,dp=1ex,center]{date in head/foot}%
    \usebeamerfont{date in head/foot}
    \insertframenumber{} / \inserttotalframenumber
  \end{beamercolorbox}}%
  \vskip0pt%
}

% Turn off navigation symbols
\setbeamertemplate{navigation symbols}{}

% Title Page Information
\title[Advanced Query Processing with Spark]{Week 5: Advanced Query Processing with Spark}
\author[J. Smith]{John Smith, Ph.D.}
\institute[University Name]{
  Department of Computer Science\\
  University Name\\
  \vspace{0.3cm}
  Email: email@university.edu\\
  Website: www.university.edu
}
\date{\today}

% Document Start
\begin{document}

\frame{\titlepage}

\begin{frame}[fragile]
    \titlepage
\end{frame}

\begin{frame}[fragile]
    \frametitle{Overview of Advanced Query Processing with Apache Spark}
    
    \begin{block}{What is Advanced Query Processing?}
        Advanced query processing refers to techniques and strategies used to execute complex queries over large datasets efficiently. In the context of Apache Spark, it involves:
        \begin{itemize}
            \item Leveraging distributed computing capabilities
            \item Optimizing execution plans
            \item Managing data intelligently
        \end{itemize}
    \end{block}
    
    \begin{block}{Significance in Big Data Analytics}
        \begin{itemize}
            \item \textbf{Scalability}: Handles large data volumes across multiple nodes.
            \item \textbf{Speed}: In-memory computing leads to faster query processing.
            \item \textbf{Flexibility}: Supports various data sources and query languages.
        \end{itemize}
    \end{block}
    
\end{frame}

\begin{frame}[fragile]
    \frametitle{Key Concepts in Advanced Query Processing}

    \begin{enumerate}
        \item \textbf{Catalyst Optimizer}
            \begin{itemize}
                \item \textbf{Purpose}: A powerful query optimization framework.
                \item \textbf{Function}: Analyzes query plans and optimizes execution strategies.
                \item \textbf{Benefit}: Increases query execution performance.
                \item \textbf{Example}: Rewriting joins and filtering data early.
            \end{itemize}
        
        \item \textbf{DataFrames and Datasets}
            \begin{itemize}
                \item \textbf{DataFrame}: Distributed collection of data organized into named columns.
                \item \textbf{Dataset}: Strong-typed interface for structured data.
                \item \textbf{Usage}: Efficient use of Spark's optimization features.
            \end{itemize}
    
        \item \textbf{Execution Plans}
            \begin{itemize}
                \item \textbf{Logical Plan}: Initial abstract representation of a query.
                \item \textbf{Physical Plan}: Optimized execution depiction of the query.
                \item \textbf{Example}: Choosing between different join strategies based on data size.
            \end{itemize}
    \end{enumerate}
\end{frame}

\begin{frame}[fragile]
    \frametitle{Example Use Case}
    
    \textbf{Scenario}: An e-commerce company wants to analyze customer purchase trends.
    
    \textbf{Query}: Retrieve the top 10 products sold in the last quarter.
    
    \begin{lstlisting}[language=Python]
df = spark.sql("""
SELECT product_id, COUNT(*) as sales_count
FROM sales
WHERE purchase_date >= '2023-07-01' AND purchase_date < '2023-10-01'
GROUP BY product_id
ORDER BY sales_count DESC
LIMIT 10
""")
    \end{lstlisting}
    
    \textbf{Significance}: This query showcases advanced filtering and aggregation, demonstrating how Spark handles complex queries seamlessly.
\end{frame}

\begin{frame}[fragile]
    \frametitle{Summary}
    
    \begin{itemize}
        \item Advanced query processing is essential for deriving insights from big data.
        \item Understanding components like the Catalyst Optimizer enhances performance.
        \item Apache Spark offers tools for executing complex queries efficiently, fostering data-driven decision-making.
    \end{itemize}
    
\end{frame}

\begin{frame}[fragile]
    \frametitle{Objectives of the Chapter - Part 1}
    \begin{block}{Learning Objectives}
        This chapter focuses on enhancing your understanding of Advanced Query Processing with Apache Spark, specifically through the use of Spark SQL. By the end of this chapter, you will be equipped with the skills necessary to perform complex data processing tasks.
    \end{block}

    \begin{enumerate}
        \item \textbf{Understand the Role of Spark SQL}
        \begin{itemize}
            \item Gain insight into how Spark SQL integrates with the Spark ecosystem for processing structured and semi-structured data.
            \item Recognize the benefits of using Spark SQL over traditional SQL engines, including performance optimization and scalability.
        \end{itemize}

        \item \textbf{Query Optimization Techniques}
        \begin{itemize}
            \item Learn about the various optimization techniques employed by Spark, such as predicate pushdown and query execution plans.
            \item Explore the use of the Catalyst Optimizer, which allows Spark to compile queries efficiently.
        \end{itemize}

        \item \textbf{DataFrame and Dataset APIs}
        \begin{itemize}
            \item Understand the differences between DataFrames and Datasets in Spark and how to use them effectively.
            \item Write and execute queries using these APIs to manipulate and analyze large datasets.
        \end{itemize}
    \end{enumerate}
\end{frame}

\begin{frame}[fragile]
    \frametitle{Objectives of the Chapter - Part 2}
    \begin{enumerate}[resume]
        \item \textbf{Working with Spark SQL Functions}
        \begin{itemize}
            \item Familiarize yourself with built-in functions available in Spark SQL for data transformation.
            \item Dive into UDFs (User Defined Functions) to extend Spark’s capabilities for specific tasks.
        \end{itemize}

        \item \textbf{Integration with Hive and Other Data Sources}
        \begin{itemize}
            \item Learn how Spark SQL can connect with Hive for reading and writing data, as well as accessing external data sources through JDBC.
            \item Understand the importance of Spark’s Data Source API in enabling connections to various data formats like Parquet, JSON, and CSV.
        \end{itemize}
    \end{enumerate}
\end{frame}

\begin{frame}[fragile]
    \frametitle{Objectives of the Chapter - Examples and Key Points}
    \begin{block}{Examples to Illustrate Concepts}
        \textbf{Example of Basic Spark SQL Query:}
        \begin{lstlisting}[language=Python]
from pyspark.sql import SparkSession

spark = SparkSession.builder.appName("Spark SQL Example").getOrCreate()
df = spark.read.json("data.json")  # Reading data from a JSON file
df.createOrReplaceTempView("people")

# SQL query to select names of people
result = spark.sql("SELECT name FROM people WHERE age > 21")
result.show()
        \end{lstlisting}

        \textbf{Query Optimization:}
        Utilize \texttt{explain()} on DataFrames to view the physical plan of the queries, allowing for optimization strategies.
    \end{block}

    \begin{block}{Key Points to Emphasize}
        \begin{itemize}
            \item \textbf{Performance}: Spark SQL significantly improves query performance through its in-memory computing and optimization techniques.
            \item \textbf{Scalability}: Scalability of Spark allows it to handle large datasets across distributed systems without compromising speed.
            \item \textbf{Flexibility}: The ability to work with various data sources and formats within a unified framework enhances the robustness of Spark in data analytics.
        \end{itemize}
    \end{block}
\end{frame}

\begin{frame}[fragile]
    \frametitle{What is Apache Spark?}
    % Overview of Apache Spark, its definition and significance
    Apache Spark is an open-source distributed computing system that enables programming entire clusters with implicit data parallelism and fault tolerance. 
    It is designed for large-scale data processing, making it an essential tool for big data analytics.
\end{frame}

\begin{frame}[fragile]
    \frametitle{Core Features of Apache Spark}
    \begin{enumerate}
        \item \textbf{Speed:}
        \begin{itemize}
            \item Processes data in-memory, making it much faster than disk-based systems like Hadoop MapReduce.
            \item Performance improvements can exceed 100 times on disk and 10 times in-memory for batch processing.
        \end{itemize}
        
        \item \textbf{Ease of Use:}
        \begin{itemize}
            \item High-level APIs available in languages like Java, Scala, Python, and R.
            \item User-friendly Spark SQL module for querying structured data.
        \end{itemize}
        
        \item \textbf{Flexibility:}
        \begin{itemize}
            \item Connects to various data sources (HDFS, Cassandra, HBase, S3, etc.).
            \item Handles batch processing, interactive queries, real-time streaming, and machine learning tasks.
        \end{itemize}
        
        \item \textbf{Unified Engine:}
        \begin{itemize}
            \item Unifies processing for various data formats and workloads.
            \item Integrates SQL, machine learning, and graph processing on the same dataset.
        \end{itemize}
        
        \item \textbf{Robust Ecosystem:}
        \begin{itemize}
            \item Comprehensive libraries for machine learning (MLlib), graph processing (GraphX), and structured data processing (Spark SQL).
        \end{itemize}
    \end{enumerate}
\end{frame}

\begin{frame}[fragile]
    \frametitle{Example Use Case: Analyzing User Behavior}
    \begin{itemize}
        \item Retailer analyzing customer purchasing patterns from various sources (sales transactions, website clicks).
        \begin{itemize}
            \item \textbf{Data Ingestion:} Spark pulls data from log files, databases, and streams.
            \item \textbf{Processing:} Analysts can perform transformations using Spark SQL without extensive coding.
            \item \textbf{Output:} Generate reports or train machine learning models to predict future behaviors.
        \end{itemize}
    \end{itemize}
\end{frame}

\begin{frame}[fragile]
    \frametitle{Simple Code Snippet (Spark SQL)}
    \begin{lstlisting}[language=Python]
# PySpark Example: Loading and querying data
from pyspark.sql import SparkSession

# Create Spark session
spark = SparkSession.builder.appName("Retail Analysis").getOrCreate()

# Load data from CSV
df = spark.read.csv("sales_data.csv", header=True, inferSchema=True)

# Query: Total sales by product
total_sales = df.groupBy("Product").agg({"Sales": "sum"}).show()
    \end{lstlisting}
\end{frame}

\begin{frame}[fragile]
    \frametitle{Conclusion}
    - Apache Spark is a revolutionary framework for big data processing.
    - It facilitates advanced analytics, machine learning, and streaming with high performance.
    - Understanding its core principles is crucial for effective real-world applications and analytics.
\end{frame}

\begin{frame}[fragile]
    \frametitle{Spark Architecture Overview}
    \begin{block}{Overview}
        Apache Spark operates on a robust architecture designed to efficiently process data in parallel across clusters of computers. Understanding its architecture is crucial for deploying and optimizing Spark applications.
    \end{block}
\end{frame}

\begin{frame}[fragile]
    \frametitle{Key Components of Spark Architecture}
    \begin{enumerate}
        \item \textbf{Driver Program}
        \item \textbf{Cluster Manager}
        \item \textbf{Worker Nodes}
    \end{enumerate}
\end{frame}

\begin{frame}[fragile]
    \frametitle{Driver Program}
    \begin{block}{Definition}
        The driver program is the component that runs the main function of your Spark application. It is responsible for:
    \end{block}
    \begin{itemize}
        \item Coordinating the execution of tasks.
        \item Scheduling the jobs and managing the overall application.
    \end{itemize}
    \begin{block}{Example}
        When you write a Spark application using Python or Scala, the script runs as a driver that initializes a \texttt{SparkContext}.
    \end{block}
\end{frame}

\begin{frame}[fragile]
    \frametitle{Cluster Manager}
    \begin{block}{Definition}
        The cluster manager is responsible for managing resources across the distributed computing environment:
    \end{block}
    \begin{itemize}
        \item Allocates resources to applications.
        \item Manages the lifecycle of tasks.
    \end{itemize}
    \begin{block}{Types}
        \begin{itemize}
            \item \textbf{Standalone}: Simplest option; Spark manages resources on its own.
            \item \textbf{Apache Mesos}: A general-purpose cluster manager.
            \item \textbf{Hadoop YARN}: Integrates Spark with existing Hadoop resources.
        \end{itemize}
    \end{block}
\end{frame}

\begin{frame}[fragile]
    \frametitle{Worker Nodes}
    \begin{block}{Definition}
        Worker nodes are the cluster machines that execute the tasks assigned by the driver.
    \end{block}
    \begin{itemize}
        \item Each worker has multiple executors that run tasks.
        \item Key functions include:
        \begin{itemize}
            \item Running individual tasks that process data.
            \item Storing data in memory or on disk (for shuffling or intermediate results).
        \end{itemize}
    \end{itemize}
    \begin{block}{Example}
        If a task involves filtering a dataset, the worker node will process those records and return the results to the driver.
    \end{block}
\end{frame}

\begin{frame}[fragile]
    \frametitle{Summary of Interactions}
    \begin{itemize}
        \item The driver program sends tasks to the cluster manager, which schedules them on worker nodes.
        \item The worker nodes process the data and report results back to the driver.
    \end{itemize}
    \begin{block}{Key Points to Emphasize}
        \begin{itemize}
            \item The driver orchestrates the tasks.
            \item The cluster manager is critical for resource allocation.
            \item Worker nodes perform the actual computations and data storage.
        \end{itemize}
    \end{block}
\end{frame}

\begin{frame}[fragile]
    \frametitle{Conclusion}
    Understanding the Spark architecture is crucial for leveraging the framework for big data processing tasks. In the next slide, we will explore different data processing models utilized in Spark, such as batch and stream processing.
\end{frame}

\begin{frame}
    \frametitle{Data Processing Models in Spark}
    \begin{block}{Introduction to Data Processing Models}
        Apache Spark is a powerful distributed computing framework that supports 
        various data processing models. Understanding these models is crucial 
        for efficiently utilizing Spark. This presentation introduces two primary 
        models: 
        \begin{itemize}
            \item Batch Processing
            \item Stream Processing
        \end{itemize}
    \end{block}
\end{frame}

\begin{frame}
    \frametitle{Batch Processing}
    \begin{block}{Definition}
        Batch processing involves processing large volumes of data collected 
        over a period (a batch) as a single unit. It is suitable for scenarios 
        where data is accumulated and processed at scheduled intervals rather 
        than in real-time.
    \end{block}

    \begin{itemize}
        \item \textbf{Latency}: High latency due to processing entire datasets 
        at once.
        \item \textbf{Use Cases}: Data warehousing, ETL processes, and large-scale 
        data analysis.
    \end{itemize}
    
    \begin{block}{Example}
        Imagine processing monthly sales data from a retail store. All 
        transaction records for the month are collected, and a Spark job 
        is executed at the end of the month to analyze the sales.
    \end{block}
\end{frame}

\begin{frame}[fragile]
    \frametitle{Batch Processing - Code Example}
    \begin{lstlisting}[language=Python,backgroundcolor=\color{lightgray}]
from pyspark.sql import SparkSession

# Create a Spark session
spark = SparkSession.builder.appName("Batch Processing Example").getOrCreate()

# Load a batch of data
sales_data = spark.read.csv("monthly_sales.csv", header=True)

# Perform analysis
sales_summary = sales_data.groupBy("product").agg({"amount": "sum"})
sales_summary.show()
    \end{lstlisting}
\end{frame}

\begin{frame}
    \frametitle{Stream Processing}
    \begin{block}{Definition}
        Stream processing entails continuous input, processing, and output of 
        data, allowing for real-time data analysis.
    \end{block}
    
    \begin{itemize}
        \item \textbf{Latency}: Low latency; data is processed as it is ingested.
        \item \textbf{Use Cases}: Real-time data analytics, fraud detection, 
        and live dashboards.
    \end{itemize}
    
    \begin{block}{Example}
        Consider a social media application analyzing tweets in real-time 
        to identify trending topics or sentiments.
    \end{block}
\end{frame}

\begin{frame}[fragile]
    \frametitle{Stream Processing - Code Example}
    \begin{lstlisting}[language=Python,backgroundcolor=\color{lightgray}]
from pyspark.sql import SparkSession
from pyspark.streaming import StreamingContext

# Create a Spark session and a Streaming context
spark = SparkSession.builder.appName("Stream Processing Example").getOrCreate()
ssc = StreamingContext(spark.sparkContext, 10)  # 10-second batch interval

# Create a DStream from a socket source
lines = ssc.socketTextStream("localhost", 9999)

# Process each RDD in the DStream
words = lines.flatMap(lambda line: line.split(" "))
word_counts = words.map(lambda word: (word, 1)).reduceByKey(lambda a, b: a + b)

# Print the counts to console
word_counts.pprint()

# Start the streaming context
ssc.start()
ssc.awaitTermination()
    \end{lstlisting}
\end{frame}

\begin{frame}
    \frametitle{Key Points}
    \begin{itemize}
        \item Batch Processing is ideal for large data volumes with 
        post-factum analysis.
        \item Stream Processing offers agility by processing data in  
        real-time, suitable for dynamic applications.
        \item Both models can be integrated within Spark’s unified framework.
    \end{itemize}
    Understanding these models helps in selecting the right approach for  
    Spark applications, enhancing data processing capabilities significantly.
\end{frame}

\begin{frame}[fragile]
    \frametitle{Introduction to Spark SQL}
    \begin{block}{What is Spark SQL?}
        Spark SQL is a component of Apache Spark that integrates relational data processing with Spark's functional programming capabilities. It allows users to execute SQL queries alongside data from Spark’s resilient distributed datasets (RDDs), making it easy to interact with structured data.
    \end{block}
\end{frame}

\begin{frame}[fragile]
    \frametitle{Key Features of Spark SQL}
    \begin{enumerate}
        \item \textbf{Unified Data Processing:}
        \begin{itemize}
            \item Supports querying data from various sources like Hive, Avro, Parquet, ORC, JSON, and JDBC.
            \item Allows seamless mixing of SQL queries with data processing operations.
        \end{itemize}
        
        \item \textbf{DataFrames:}
        \begin{itemize}
            \item A DataFrame is a distributed collection of data organized into named columns (similar to a table).
            \item Provides a high-level abstraction for structured data processing.
        \end{itemize}
        
        \item \textbf{Datasets:}
        \begin{itemize}
            \item A Dataset is an extension of DataFrames that provides compile-time type safety.
            \item Combines features of RDD and DataFrame, allowing both SQL and functional programming operations.
        \end{itemize}
    \end{enumerate}
\end{frame}

\begin{frame}[fragile]
    \frametitle{Key Features of Spark SQL (Continued)}
    \begin{enumerate}[resume]
        \item \textbf{Optimized Execution Engine:}
        \begin{itemize}
            \item Utilizes the Catalyst optimizer for efficient query optimization.
            \item Uses the Tungsten execution engine for performance improvements.
        \end{itemize}
        
        \item \textbf{Support for SQL:}
        \begin{itemize}
            \item Users can run SQL queries directly on DataFrames using Spark SQL.
        \end{itemize}
        
        \begin{block}{Example of a SQL query}
            \begin{lstlisting}[language=sql]
            SELECT name, age FROM people WHERE age > 20
            \end{lstlisting}
        \end{block}
    \end{enumerate}
\end{frame}

\begin{frame}[fragile]
    \frametitle{Advantages of Using Spark SQL}
    \begin{itemize}
        \item \textbf{Performance:} Enhanced performance due to optimization features and in-memory processing.
        \item \textbf{Scalability:} Highly scalable infrastructure for large datasets.
        \item \textbf{Ease of Use:} Friendly APIs for Python, Java, and Scala.
    \end{itemize}
\end{frame}

\begin{frame}[fragile]
    \frametitle{Summary and Next Steps}
    \begin{block}{Summary}
        Spark SQL bridges the gap between unstructured data processing and structured querying capabilities. Combining DataFrames and Datasets offers flexibility and performance for modern data processing needs.
    \end{block}
    
    \begin{block}{Next Steps}
        Explore how to create DataFrames from various data formats in the next slide.
    \end{block}
    
    \begin{center}
        \textit{Use Spark SQL to unlock the full potential of your data!}
    \end{center}
\end{frame}

\begin{frame}
    \frametitle{Creating DataFrames in Spark}
    \begin{block}{Introduction to DataFrames in Spark}
        \begin{itemize}
            \item **DataFrames**: Distributed collections of data organized into named columns.
            \item Higher-level abstraction than RDDs (Resilient Distributed Datasets).
            \item Similar to database tables or data frames in R/Pandas.
            \item Benefit from Spark SQL for easier manipulation and SQL-like queries.
        \end{itemize}
    \end{block}
\end{frame}

\begin{frame}
    \frametitle{Creating DataFrames from Various Data Sources}
    \begin{block}{1. Creating DataFrames from CSV Files}
        \begin{itemize}
            \item Spark supports CSV files for easy data import.
        \end{itemize}
        \begin{lstlisting}[language=python]
from pyspark.sql import SparkSession

# Create a Spark session
spark = SparkSession.builder.appName("DataFrameCreation").getOrCreate()

# Load CSV file into a DataFrame
df_csv = spark.read.csv("path/to/file.csv", header=True, inferSchema=True)

# Show the DataFrame content
df_csv.show()
        \end{lstlisting}
        \begin{itemize}
            \item \texttt{header=True}: First row contains column names.
            \item \texttt{inferSchema=True}: Automatically infers data types.
        \end{itemize}
    \end{block}
\end{frame}

\begin{frame}
    \frametitle{Creating DataFrames from Various Data Sources (cont.)}
    \begin{block}{2. Creating DataFrames from JSON Files}
        \begin{itemize}
            \item JSON is a common format for data interchange.
        \end{itemize}
        \begin{lstlisting}[language=python]
# Load JSON file into a DataFrame
df_json = spark.read.json("path/to/file.json")

# Show the DataFrame content
df_json.show()
        \end{lstlisting}
        \begin{itemize}
            \item JSON files are schema-less; Spark infers the schema automatically.
            \item Optimal for semi-structured data.
        \end{itemize}
    \end{block}
\end{frame}

\begin{frame}
    \frametitle{Creating DataFrames from Various Data Sources (cont.)}
    \begin{block}{3. Creating DataFrames from Parquet Files}
        \begin{itemize}
            \item Parquet is columnar storage for large datasets.
        \end{itemize}
        \begin{lstlisting}[language=python]
# Load Parquet file into a DataFrame
df_parquet = spark.read.parquet("path/to/file.parquet")

# Show the DataFrame content
df_parquet.show()
        \end{lstlisting}
        \begin{itemize}
            \item **Advantages of Parquet**:
                \begin{itemize}
                    \item Efficient storage and fast retrieval.
                    \item Supports complex nested data structures.
                    \item Optimized for use with Spark.
                \end{itemize}
        \end{itemize}
    \end{block}
\end{frame}

\begin{frame}
    \frametitle{Conclusion on DataFrames in Spark}
    \begin{itemize}
        \item DataFrames simplify data manipulation and analysis in Spark.
        \item They provide a structured format for various data sources.
        \item Understanding how to create DataFrames from CSV, JSON, and Parquet is crucial for effective big data processing.
    \end{itemize}
    \begin{block}{Next Steps}
        Next, we will explore basic operations on these DataFrames, such as filtering and aggregation.
    \end{block}
\end{frame}

\begin{frame}[fragile]
    \frametitle{Basic Operations with DataFrames - Introduction}
    \begin{block}{Introduction to DataFrames in Spark}
        DataFrames in Apache Spark are a distributed collection of 
        data organized into named columns, similar to a table in a 
        relational database or a data frame in R/Pandas. They are 
        designed for easy access and efficient manipulation of structured 
        and semi-structured data.
    \end{block}
\end{frame}

\begin{frame}[fragile]
    \frametitle{Basic Operations with DataFrames}
    \begin{enumerate}
        \item \textbf{Filtering DataFrames}
            \begin{itemize}
                \item Description: Selecting rows that meet certain conditions; similar to SQL's \texttt{WHERE} clause.
                \item Syntax:
                \begin{lstlisting}[language=Python]
filtered_df = df.filter(df['column_name'] > value)
                \end{lstlisting}
                \item Example:
                \begin{lstlisting}[language=Python]
# Suppose df contains a column 'age'
adults_df = df.filter(df['age'] >= 18)
                \end{lstlisting}
            \end{itemize}
        \item \textbf{Selecting Columns}
            \begin{itemize}
                \item Description: Extracting specific columns reduces dataset size and focuses on relevant information.
                \item Syntax:
                \begin{lstlisting}[language=Python]
selected_columns_df = df.select('column_name1', 'column_name2')
                \end{lstlisting}
                \item Example:
                \begin{lstlisting}[language=Python]
# From a DataFrame 'df', select 'name' and 'salary' columns
names_salaries_df = df.select('name', 'salary')
                \end{lstlisting}
            \end{itemize}
    \end{enumerate}
\end{frame}

\begin{frame}[fragile]
    \frametitle{Basic Operations with DataFrames - Aggregation}
    \begin{itemize}
        \item \textbf{Aggregation}
            \begin{itemize}
                \item Description: Summarizes data through functions such as \texttt{count()}, \texttt{sum()}, \texttt{avg()}, and \texttt{max()}.
                \item Syntax:
                \begin{lstlisting}[language=Python]
aggregated_df = df.groupBy('group_column').agg({'agg_column': 'agg_function'})
                \end{lstlisting}
                \item Example:
                \begin{lstlisting}[language=Python]
# Calculate average salary grouped by department
average_salary_df = df.groupBy('department').agg({'salary': 'avg'})
                \end{lstlisting}
            \end{itemize}
        \item Key Points
            \begin{itemize}
                \item DataFrames optimize data processing through lazy evaluation.
                \item Operations like \texttt{filter}, \texttt{select}, and \texttt{agg} are fundamental.
                \item Understanding these operations lays a foundation for advanced techniques.
            \end{itemize}
    \end{itemize}
\end{frame}

\begin{frame}[fragile]
    \frametitle{Code Snippet and Conclusion}
    \begin{block}{Combined Example}
    Here's a combined example demonstrating filtering, selection, and aggregation:
    \begin{lstlisting}[language=Python]
# Load DataFrame from a CSV
df = spark.read.csv('data/employees.csv', header=True, inferSchema=True)

# Filter, Select, and Aggregate
result_df = df.filter(df['age'] >= 30) \
               .select('name', 'age', 'salary') \
               .groupBy('age') \
               .agg({'salary': 'avg'}).show()
    \end{lstlisting}
    \end{block}

    \begin{block}{Conclusion}
        Mastering basic operations on DataFrames enables efficient data manipulation and prepares you for more complex analyses in Spark.
    \end{block}
\end{frame}

\begin{frame}[fragile]
    \frametitle{Advanced DataFrame Functions - Overview}
    In this section, we delve into advanced DataFrame functions in Apache Spark, focusing on:
    \begin{itemize}
        \item Joins
        \item Grouping
        \item Window Functions
    \end{itemize}
    These operations allow us to manipulate and analyze large datasets efficiently and effectively, enhancing our data processing capabilities.
\end{frame}

\begin{frame}[fragile]
    \frametitle{Advanced DataFrame Functions - Joins}
    Joins are used to combine two or more DataFrames based on a common key, enriching the dataset with related information.

    \begin{block}{Types of Joins}
        \begin{itemize}
            \item **Inner Join**: Returns only the rows with matching values in both DataFrames.
            \item **Outer Join (Full, Left, Right)**: Returns all rows from one or both DataFrames, filling in nulls for missing matches.
            \item **Cross Join**: Produces a Cartesian product of both DataFrames.
        \end{itemize}
    \end{block}

    \begin{block}{Example}
    \begin{lstlisting}[language=Python]
# Creating two DataFrames
df1 = spark.createDataFrame([(1, "Alice"), (2, "Bob")], ["id", "name"])
df2 = spark.createDataFrame([(1, "Sales"), (2, "Engineering")], ["id", "department"])

# Performing an Inner Join
result = df1.join(df2, df1.id == df2.id, "inner")
result.show()
    \end{lstlisting}
    \end{block}
\end{frame}

\begin{frame}[fragile]
    \frametitle{Advanced DataFrame Functions - Grouping and Window Functions}
    \textbf{Grouping} allows us to aggregate data, providing insights through summary statistics such as count and average.
    
    \begin{block}{Key Functions}
        \begin{itemize}
            \item \texttt{groupBy()}: Groups DataFrame rows based on columns.
            \item \texttt{agg()}: Performs aggregate calculations.
        \end{itemize}
    \end{block}

    \begin{block}{Example}
    \begin{lstlisting}[language=Python]
# Creating a DataFrame
sales_data = spark.createDataFrame([(1, "Alice", 100), (2, "Bob", 200), (1, "Alice", 150)], ["id", "name", "sales"])

# Grouping and aggregating
grouped_result = sales_data.groupBy("name").agg({"sales": "sum"})
grouped_result.show()
    \end{lstlisting}
    \end{block}

    \textbf{Window Functions} provide operations over a specified range of rows within a partition, offering advanced analytics capabilities.
\end{frame}

\begin{frame}[fragile]
    \frametitle{SQL Queries in Spark - Introduction}
    \begin{itemize}
        \item Spark SQL enables execution of SQL queries on structured data using the DataFrame API.
        \item Provides capability to run complex queries across large datasets.
        \item Operates in a distributed computing environment, enhancing performance for big data.
    \end{itemize}
\end{frame}

\begin{frame}[fragile]
    \frametitle{SQL Queries in Spark - Key Concepts}
    \begin{enumerate}
        \item \textbf{DataFrames}:
        \begin{itemize}
            \item Distributed collection of data organized into named columns.
            \item Enables seamless integration of SQL queries with Spark operations.
        \end{itemize}
        
        \item \textbf{Spark SQL Context}:
        \begin{itemize}
            \item Required to run SQL queries in Spark.
            \item Initialization example:
            \begin{lstlisting}[language=Python]
from pyspark.sql import SparkSession

spark = SparkSession.builder \
    .appName("Spark SQL Example") \
    .getOrCreate()
            \end{lstlisting}
        \end{itemize}
    \end{enumerate}
\end{frame}

\begin{frame}[fragile]
    \frametitle{SQL Queries in Spark - Executing Queries}
    \begin{itemize}
        \item SQL queries can be executed using the \texttt{sql()} method on a Spark session.
        \item Example:
        \begin{lstlisting}[language=Python]
df.createOrReplaceTempView("employees")

result_df = spark.sql("SELECT name, age FROM employees WHERE age > 30")
result_df.show()
        \end{lstlisting}
        \item Temporary Views: Use \texttt{createOrReplaceTempView()} to register DataFrames as temporary views.
    \end{itemize}
\end{frame}

\begin{frame}[fragile]
    \frametitle{SQL Queries in Spark - Differences with Traditional SQL}
    \begin{itemize}
        \item \textbf{Execution Model}:
        \begin{itemize}
            \item Spark SQL: Distributed processing, optimized for big data.
            \item Traditional SQL: Operates on single servers, limited scalability.
        \end{itemize}
        
        \item \textbf{Schema Management}:
        \begin{itemize}
            \item Spark SQL: Dynamic schema management, supports semi-structured data.
            \item Traditional SQL: Requires predefined schemas.
        \end{itemize}

        \item \textbf{Performance Optimization}:
        \begin{itemize}
            \item Spark SQL: Catalyst optimizer and Tungsten execution engine.
            \item Traditional SQL: Fixed query planner execution strategy.
        \end{itemize}
        
        \item \textbf{Interactivity and Ingestion}:
        \begin{itemize}
            \item Spark SQL: Supports interactive analysis and diverse data sources.
            \item Traditional SQL: Focuses on data in relational databases.
        \end{itemize}
    \end{itemize}
\end{frame}

\begin{frame}[fragile]
    \frametitle{Optimizing Queries in Spark - Introduction}
    \begin{itemize}
        \item Query optimization in Spark enhances data processing performance.
        \item Two primary components:
        \begin{itemize}
            \item \textbf{Catalyst Optimizer}: Handles query optimization using declarative rules.
            \item \textbf{Tungsten Execution Engine}: Improves memory management and CPU efficiency.
        \end{itemize}
    \end{itemize}
\end{frame}

\begin{frame}[fragile]
    \frametitle{Optimizing Queries in Spark - Catalyst Optimizer}
    \begin{itemize}
        \item Converts SQL queries into optimized logical plans.
        \begin{enumerate}
            \item \textbf{Logical Plan Generation}: Converts SQL into logical plan.
            \item \textbf{Analysis}: Validates query correctness.
            \item \textbf{Optimizations}: Applies transformation rules (e.g., constant folding, predicate pushdown).
            \item \textbf{Physical Plan Generation}: Translates optimized logical plan into a physical plan.
        \end{enumerate}
        \item \textbf{Example SQL Query}:
        \begin{lstlisting}[language=SQL]
SELECT * FROM employees WHERE salary > 50000;
        \end{lstlisting}
        \begin{itemize}
            \item Catalyst will validate existence of `employees` table.
            \item Optimize to scan only relevant partitions.
        \end{itemize}
    \end{itemize}
\end{frame}

\begin{frame}[fragile]
    \frametitle{Optimizing Queries in Spark - Tungsten Execution Engine}
    \begin{itemize}
        \item Focuses on enhancing execution performance:
        \begin{itemize}
            \item \textbf{Whole-Stage Code Generation}: Compiles entire query execution plan to Java bytecode.
            \item \textbf{Off-Heap Memory Management}: Efficiently manages large data using off-heap storage.
            \item \textbf{Data Locality Optimization}: Reduces data movement to enhance processing speeds.
        \end{itemize}
        \item \textbf{Example Spark SQL in Scala}:
        \begin{lstlisting}[language=Scala]
val df = spark.sql("SELECT * FROM employees WHERE salary > 50000")
df.explain(true) // Displays logical and physical plans
        \end{lstlisting}
    \end{itemize}
\end{frame}

\begin{frame}[fragile]
    \frametitle{Optimizing Queries in Spark - Key Techniques}
    \begin{itemize}
        \item \textbf{Broadcast Joins}:
        \begin{itemize}
            \item Speeds up joins by broadcasting small tables to all nodes.
            \item \textbf{Example}:
            \begin{lstlisting}[language=Scala]
val smallDF = spark.table("small_table")
val largeDF = spark.table("large_table")
val joinDF = largeDF.join(broadcast(smallDF), "id")
            \end{lstlisting}
        \end{itemize}
        \item \textbf{Partitioning and Bucketing}:
        \begin{itemize}
            \item Efficient data storage reducing data scanned during queries.
            \item \textbf{Example}:
            \begin{lstlisting}[language=SQL]
CREATE TABLE sales USING parquet PARTITIONED BY (year) 
CLUSTERED BY (country) INTO 10 BUCKETS;
            \end{lstlisting}
        \end{itemize}
    \end{itemize}
\end{frame}

\begin{frame}
    \frametitle{Optimizing Queries in Spark - Summary}
    \begin{itemize}
        \item Catalyst Optimizer and Tungsten Engine are crucial for optimization.
        \item Together, they convert high-level queries into optimized execution plans.
        \item Best practices (e.g., broadcast joins, partitioning) improve query performance.
    \end{itemize}
    \vspace{0.5cm}
    \begin{block}{Questions}
        Feel free to ask questions or engage in further discussions on applying these optimization techniques in real-world scenarios!
    \end{block}
\end{frame}

\begin{frame}[fragile]
    \frametitle{Integration with Other Data Tools - Overview}
    Apache Spark is a powerful open-source data processing engine designed for large-scale data processing. It allows for the integration with various data processing tools and platforms, enhancing its capacity and providing flexibility to data engineers and analysts.
\end{frame}

\begin{frame}[fragile]
    \frametitle{Integration with Hadoop}
    \begin{itemize}
        \item \textbf{Hadoop Ecosystem:} 
        Spark can run on Hadoop, utilizing HDFS (Hadoop Distributed File System) for storing large datasets, facilitating seamless data processing.
        
        \item \textbf{YARN:} 
        Spark can be deployed on a Hadoop cluster, leveraging YARN for resource management. Users can run Spark applications alongside traditional MapReduce applications.
        
        \item \textbf{Example:} 
        An organization could store large datasets on HDFS and utilize Spark for analytics, enabling faster data processing compared to traditional MapReduce jobs.
    \end{itemize}
\end{frame}

\begin{frame}[fragile]
    \frametitle{Integration with NoSQL Databases}
    \begin{itemize}
        \item \textbf{Cassandra:} 
        Spark can read and write from/to Cassandra databases, enhancing data analytics capabilities.
        
        \item \textbf{MongoDB:} 
        Integration allows Spark to perform real-time analytics on data in MongoDB using the Spark Connector for MongoDB.
        
        \item \textbf{HBase:} 
        Spark can connect to HBase, enabling batch and stream processing directly from this NoSQL database.
    \end{itemize}
    
    \begin{block}{Code Snippet}
    \begin{lstlisting}[language=python]
from pyspark.sql import SparkSession

# Creating a Spark session
spark = SparkSession.builder \
    .appName('MongoDBIntegration') \
    .config('spark.mongodb.input.uri', 'mongodb://127.0.0.1/mydb.mycollection') \
    .config('spark.mongodb.output.uri', 'mongodb://127.0.0.1/mydb.mycollection') \
    .getOrCreate()

# Reading data from MongoDB
df = spark.read.format('mongo').load()
    \end{lstlisting}
    *This code snippet demonstrates how to set up a Spark session and read data from a MongoDB collection using PySpark.*
    \end{block}
\end{frame}

\begin{frame}[fragile]
    \frametitle{Key Points and Conclusion}
    \begin{itemize}
        \item \textbf{Flexibility and Performance:} 
        Spark's integration with Hadoop and NoSQL databases enhances flexibility and performance in big data workloads.
        
        \item \textbf{Toolchain Compatibility:} 
        Spark serves as a bridge between batch processing and real-time analytics by integrating with various platforms.
        
        \item \textbf{Unified Data Processing:} 
        Organizations can unify data processing needs, reducing complexity and enhancing productivity.
    \end{itemize}
    
    \textbf{Conclusion:} \\
    The integration capabilities of Apache Spark with Hadoop and various NoSQL databases empower data professionals to process and analyze vast amounts of data efficiently. Understanding these integrations is vital for leveraging Spark's full potential in advanced data processing scenarios.
\end{frame}

\begin{frame}[fragile]
    \frametitle{Use Cases of Spark SQL - Introduction}
    \begin{block}{Overview}
        Spark SQL is a powerful component of Apache Spark that enables SQL queries on large datasets, leveraging distributed computation for efficient data processing.
    \end{block}
    \begin{block}{Use Cases}
        We explore several real-world applications in:
        \begin{itemize}
            \item Healthcare
            \item Finance
            \item Social Media
        \end{itemize}
    \end{block}
\end{frame}

\begin{frame}[fragile]
    \frametitle{Use Cases of Spark SQL - Healthcare}
    \begin{block}{Patient Data Analysis}
        Hospitals generate vast amounts of data, which can be analyzed using Spark SQL to improve patient outcomes.
    \end{block}
    \begin{exampleblock}{Example Query}
        \begin{lstlisting}[style=mystyle]
        SELECT patient_id, COUNT(readmission_id) as readmission_count 
        FROM patient_readmissions 
        GROUP BY patient_id 
        HAVING readmission_count > 1;
        \end{lstlisting}
    \end{exampleblock}
\end{frame}

\begin{frame}[fragile]
    \frametitle{Use Cases of Spark SQL - Finance and Social Media}
    \begin{block}{Finance}
        \begin{itemize}
            \item \textbf{Fraud Detection}: Monitoring transactions to identify suspicious patterns.
            \item \textbf{Risk Management}: Performing complex risk calculations on historical data for informed strategies.
        \end{itemize}
        \begin{exampleblock}{Example Query for Fraud Detection}
            \begin{lstlisting}[style=mystyle]
            SELECT user_id, COUNT(transaction_id) as suspicious_count 
            FROM transactions 
            WHERE amount > 10000 
            GROUP BY user_id 
            HAVING suspicious_count > 5;
            \end{lstlisting}
        \end{exampleblock}
    \end{block}
    
    \begin{block}{Social Media}
        \begin{itemize}
            \item \textbf{User Engagement}: Analyze user interactions for enhancing experiences.
            \item \textbf{Trend Analysis}: Identify trending topics for timely marketing.
        \end{itemize}
        \begin{exampleblock}{Example Query for Engagement}
            \begin{lstlisting}[style=mystyle]
            SELECT post_id, AVG(likes) as avg_likes 
            FROM posts 
            WHERE created_at >= '2023-01-01' 
            GROUP BY post_id;
            \end{lstlisting}
        \end{exampleblock}
    \end{block}
\end{frame}

\begin{frame}[fragile]
    \frametitle{Use Cases of Spark SQL - Key Points and Conclusion}
    \begin{block}{Key Points}
        \begin{itemize}
            \item \textbf{Scalability}: Suitable for big data environments.
            \item \textbf{Performance}: Enhanced query execution speed.
            \item \textbf{Flexibility}: Supports multiple data sources.
        \end{itemize}
    \end{block}
    \begin{block}{Conclusion}
        Spark SQL transforms how industries analyze large datasets, illustrating its importance in big data analytics.
    \end{block}
\end{frame}

\begin{frame}
    \frametitle{Takeaway}
    By understanding these applications, students can appreciate Spark SQL's strategic importance in decision-making across diverse industries.
\end{frame}

\begin{frame}[fragile]
    \frametitle{Challenges in Query Processing - Introduction}
    \begin{block}{Overview}
        In advanced query processing with Spark, several challenges may arise due to:
        \begin{itemize}
            \item Complexities of distributed processing
            \item Large data volume
            \item Diverse data sources
        \end{itemize}
        Understanding these challenges is essential for optimizing Spark's performance and ensuring efficient query execution.
    \end{block}
\end{frame}

\begin{frame}[fragile]
    \frametitle{Challenges in Query Processing - Key Challenges}
    \begin{enumerate}
        \item \textbf{Data Skew}
        \item \textbf{Complex Query Optimization}
        \item \textbf{Resource Allocation}
        \item \textbf{Latency in Processing}
        \item \textbf{Handling Schema Evolution}
    \end{enumerate}
\end{frame}

\begin{frame}[fragile]
    \frametitle{Challenges in Query Processing - Detailed Challenges}
    \begin{block}{Data Skew}
        \textbf{Explanation:} Uneven data distribution can create bottlenecks in task execution.
        
        \textbf{Example:} A dataset's "user_id" where one user holds 90\% of entries.
        
        \textbf{Solutions:}
        \begin{itemize}
            \item Use salting to evenly distribute workload.
            \item Re-partition datasets dynamically.
        \end{itemize}
    \end{block}

    \begin{block}{Complex Query Optimization}
        \textbf{Explanation:} Advanced queries can be complex for the Spark optimizer.
        
        \textbf{Example:} Joining three large tables may create massive intermediate datasets.
        
        \textbf{Solutions:}
        \begin{itemize}
            \item Break down queries into simpler sub-queries.
            \item Use broadcast joins for smaller datasets.
        \end{itemize}
    \end{block}
\end{frame}

\begin{frame}[fragile]
    \frametitle{Challenges in Query Processing - Further Challenges}
    \begin{block}{Resource Allocation}
        \textbf{Explanation:} Managing resources can be challenging, especially in cloud environments.
        
        \textbf{Example:} Insufficient executor memory may cause task failures.
        
        \textbf{Solutions:}
        \begin{itemize}
            \item Utilize dynamic resource allocation.
            \item Monitor and adjust Spark configurations.
        \end{itemize}
    \end{block}

    \begin{block}{Latency in Processing}
        \textbf{Explanation:} Data shuffling can increase latency, especially with large datasets.
        
        \textbf{Example:} Shuffling large partitions results in slow job execution.
        
        \textbf{Solutions:}
        \begin{itemize}
            \item Optimize data layout and partitioning.
            \item Use data caching to keep frequently accessed data in memory.
        \end{itemize}
    \end{block}
\end{frame}

\begin{frame}[fragile]
    \frametitle{Challenges in Query Processing - Final Challenges}
    \begin{block}{Handling Schema Evolution}
        \textbf{Explanation:} Changing data schemas can lead to processing challenges.
        
        \textbf{Example:} New columns in a table can break existing queries.
        
        \textbf{Solutions:}
        \begin{itemize}
            \item Implement schema evolution strategies (e.g., Avro, Parquet).
            \item Validate incoming data against expected schemas.
        \end{itemize}
    \end{block}

    \begin{block}{Conclusion}
        Addressing these challenges is vital for maximizing the efficiency and performance of Spark applications. Proactive strategies can optimize workloads and enhance overall performance.
    \end{block}
\end{frame}

\begin{frame}[fragile]
    \frametitle{Future of Spark in Data Processing - Introduction}
    \begin{block}{Overview}
        Apache Spark has been at the forefront of big data processing since its inception. As data volumes and complexity grow, Spark evolves to meet changing analytics demands.
    \end{block}
\end{frame}

\begin{frame}[fragile]
    \frametitle{Future of Spark in Data Processing - Key Drivers}
    \begin{itemize}
        \item \textbf{Scalability}: Enhancements through Kubernetes for better resource management.
        \item \textbf{Integration with AI \& ML}: Spark’s MLlib facilitates seamless AI/ML integration.
        \item \textbf{Stream Processing}: Spark Streaming enables real-time data analysis, essential for applications such as fraud detection.
    \end{itemize}
\end{frame}

\begin{frame}[fragile]
    \frametitle{Future of Spark in Data Processing - Innovations & Applications}
    \begin{itemize}
        \item \textbf{Adaptive Query Execution (AQE)}: Dynamically optimizes query performance based on runtime statistics.
        \item \textbf{Business Intelligence}: Real-time analytics enhance decision-making processes.
        \item \textbf{Healthcare Analytics}: Analyzing genomic data for improved patient care outcomes.
    \end{itemize}
\end{frame}

\begin{frame}[fragile]
    \frametitle{Future of Spark in Data Processing - Challenges and Conclusion}
    \begin{itemize}
        \item \textbf{Challenges}:
            \begin{itemize}
                \item Complexity of setup for new users
                \item Performance tuning difficulties
            \end{itemize}
        \item \textbf{Conclusion}: The future of Spark is promising, with continuous enhancements ensuring its role as a key tool for data processing and analytics.
    \end{itemize}
\end{frame}

\begin{frame}[fragile]
    \frametitle{Future of Spark in Data Processing - Example Code}
    \begin{lstlisting}[language=Python]
from pyspark.sql import SparkSession

# Create a Spark session
spark = SparkSession.builder \
    .appName("Future of Spark") \
    .getOrCreate()

# Load data and execute a query
data = spark.read.json("s3://your-bucket/your-data.json")
result = data.groupBy("category").count()
result.show()
    \end{lstlisting}
\end{frame}

\begin{frame}[fragile]
    \frametitle{Summary and Conclusion - Key Points}
    \begin{enumerate}
        \item \textbf{Advanced Query Processing in Spark}:
            \begin{itemize}
                \item Explored optimization techniques: predicate pushdown, column pruning, join optimizations.
                \item Introduced the Catalyst optimizer for dynamic query analysis.
            \end{itemize}
        \item \textbf{DataFrame and Dataset APIs}:
            \begin{itemize}
                \item User-friendly approach for structured data processing.
                \item Ease in expressing complex queries compared to RDDs.
            \end{itemize}
        \item \textbf{Spark SQL}:
            \begin{itemize}
                \item Integration of SQL with DataFrame API facilitates SQL and functional programming.
                \item Example usage: 
                \begin{lstlisting}[language=Python]
df = spark.sql("SELECT name, age FROM people WHERE age > 21")
                \end{lstlisting}
            \end{itemize}
    \end{enumerate}
\end{frame}

\begin{frame}[fragile]
    \frametitle{Summary and Conclusion - Continued}
    \begin{enumerate}[start=4]
        \item \textbf{Understanding Execution Plans}:
            \begin{itemize}
                \item Analyzed physical and logical execution plans with the \texttt{.explain()} method.
                \item Understanding execution leads to better query optimization insights.
            \end{itemize}
        \item \textbf{Performance Tuning}:
            \begin{itemize}
                \item Discussed tuning strategies: memory allocation, executor instances, and partitioning.
                \item These adjustments have direct impacts on query performance.
            \end{itemize}
    \end{enumerate}
\end{frame}

\begin{frame}[fragile]
    \frametitle{Importance of Mastering Advanced Query Processing}
    \begin{itemize}
        \item \textbf{Enhanced Performance}:
            \begin{itemize}
                \item Efficient queries lead to faster data retrieval and processing.
            \end{itemize}
        \item \textbf{Scalability}:
            \begin{itemize}
                \item Skills to build scalable solutions for growing datasets.
            \end{itemize}
        \item \textbf{Versatility}:
            \begin{itemize}
                \item Seamless integration with various data sources and formats.
            \end{itemize}
        \item \textbf{Real-World Applications}:
            \begin{itemize}
                \item Enhances decision-making capabilities across sectors like finance, healthcare, and e-commerce.
            \end{itemize}
    \end{itemize}
    
    \begin{block}{Conclusion}
        Mastering advanced query processing is essential for leveraging Spark's potential in data processing endeavors.
    \end{block}
\end{frame}


\end{document}