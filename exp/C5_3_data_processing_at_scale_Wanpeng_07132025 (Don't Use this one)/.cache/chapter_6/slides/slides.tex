\documentclass[aspectratio=169]{beamer}

% Theme and Color Setup
\usetheme{Madrid}
\usecolortheme{whale}
\useinnertheme{rectangles}
\useoutertheme{miniframes}

% Additional Packages
\usepackage[utf8]{inputenc}
\usepackage[T1]{fontenc}
\usepackage{graphicx}
\usepackage{booktabs}
\usepackage{listings}
\usepackage{amsmath}
\usepackage{amssymb}
\usepackage{xcolor}
\usepackage{tikz}
\usepackage{pgfplots}
\pgfplotsset{compat=1.18}
\usetikzlibrary{positioning}
\usepackage{hyperref}

% Custom Colors
\definecolor{myblue}{RGB}{31, 73, 125}
\definecolor{mygray}{RGB}{100, 100, 100}
\definecolor{mygreen}{RGB}{0, 128, 0}
\definecolor{myorange}{RGB}{230, 126, 34}
\definecolor{mycodebackground}{RGB}{245, 245, 245}

% Set Theme Colors
\setbeamercolor{structure}{fg=myblue}
\setbeamercolor{frametitle}{fg=white, bg=myblue}
\setbeamercolor{title}{fg=myblue}
\setbeamercolor{section in toc}{fg=myblue}
\setbeamercolor{item projected}{fg=white, bg=myblue}
\setbeamercolor{block title}{bg=myblue!20, fg=myblue}
\setbeamercolor{block body}{bg=myblue!10}
\setbeamercolor{alerted text}{fg=myorange}

% Set Fonts
\setbeamerfont{title}{size=\Large, series=\bfseries}
\setbeamerfont{frametitle}{size=\large, series=\bfseries}
\setbeamerfont{caption}{size=\small}
\setbeamerfont{footnote}{size=\tiny}

% Code Listing Style
\lstdefinestyle{customcode}{
  backgroundcolor=\color{mycodebackground},
  basicstyle=\footnotesize\ttfamily,
  breakatwhitespace=false,
  breaklines=true,
  commentstyle=\color{mygreen}\itshape,
  keywordstyle=\color{blue}\bfseries,
  stringstyle=\color{myorange},
  numbers=left,
  numbersep=8pt,
  numberstyle=\tiny\color{mygray},
  frame=single,
  framesep=5pt,
  rulecolor=\color{mygray},
  showspaces=false,
  showstringspaces=false,
  showtabs=false,
  tabsize=2,
  captionpos=b
}
\lstset{style=customcode}

% Custom Commands
\newcommand{\hilight}[1]{\colorbox{myorange!30}{#1}}
\newcommand{\source}[1]{\vspace{0.2cm}\hfill{\tiny\textcolor{mygray}{Source: #1}}}
\newcommand{\concept}[1]{\textcolor{myblue}{\textbf{#1}}}
\newcommand{\separator}{\begin{center}\rule{0.5\linewidth}{0.5pt}\end{center}}

% Footer and Navigation Setup
\setbeamertemplate{footline}{
  \leavevmode%
  \hbox{%
  \begin{beamercolorbox}[wd=.3\paperwidth,ht=2.25ex,dp=1ex,center]{author in head/foot}%
    \usebeamerfont{author in head/foot}\insertshortauthor
  \end{beamercolorbox}%
  \begin{beamercolorbox}[wd=.5\paperwidth,ht=2.25ex,dp=1ex,center]{title in head/foot}%
    \usebeamerfont{title in head/foot}\insertshorttitle
  \end{beamercolorbox}%
  \begin{beamercolorbox}[wd=.2\paperwidth,ht=2.25ex,dp=1ex,center]{date in head/foot}%
    \usebeamerfont{date in head/foot}
    \insertframenumber{} / \inserttotalframenumber
  \end{beamercolorbox}}%
  \vskip0pt%
}

% Turn off navigation symbols
\setbeamertemplate{navigation symbols}{}

% Title Page Information
\title[NoSQL Systems Overview]{Week 6: NoSQL Systems Overview}
\author[J. Smith]{John Smith, Ph.D.}
\institute[University Name]{
  Department of Computer Science\\
  University Name\\
  \vspace{0.3cm}
  Email: email@university.edu\\
  Website: www.university.edu
}
\date{\today}

% Document Start
\begin{document}

\frame{\titlepage}

\begin{frame}[fragile]
    \frametitle{Introduction to NoSQL Databases}
    \begin{block}{Overview}
        NoSQL databases signify an evolution in data handling just as modern applications demand flexibility, scalability, and speed in data management.
    \end{block}
    \begin{itemize}
        \item Traditional RDBMS face scalability limits with unstructured/semi-structured data.
        \item NoSQL databases adapt to the increasing volume, velocity, and variety of modern data.
    \end{itemize}
\end{frame}

\begin{frame}[fragile]
    \frametitle{Evolution of NoSQL}
    \begin{itemize}
        \item \textbf{Emergence (Late 2000s):}
        \begin{itemize}
            \item Term "NoSQL" coined around 2009 to identify non-SQL databases.
            \item Early innovations by Google (BigTable), Amazon (Dynamo), Facebook (Cassandra).
        \end{itemize}
        \item \textbf{Types of NoSQL Databases:}
        \begin{itemize}
            \item \textbf{Document Stores (e.g., MongoDB):} Flexible schemas using JSON-like documents.
            \item \textbf{Key-Value Stores (e.g., Redis):} Efficient key-value pair data retrieval.
            \item \textbf{Column Family Stores (e.g., Apache Cassandra):} Optimized for large-scale analytics via column storage.
            \item \textbf{Graph Databases (e.g., Neo4j):} Ideal for managing relationships in data, used in social networks.
        \end{itemize}
    \end{itemize}
\end{frame}

\begin{frame}[fragile]
    \frametitle{Significance of NoSQL}
    \begin{itemize}
        \item \textbf{Scalability:} 
        \begin{itemize}
            \item Designed for horizontal scaling; adding servers as data grows.
            \item Contrasts with RDBMS which typically requires vertical scaling.
        \end{itemize}
        \item \textbf{Flexibility:}
        \begin{itemize}
            \item Often schema-less, allowing adaptation to changing data structures.
        \end{itemize}
        \item \textbf{Performance:}
        \begin{itemize}
            \item High read/write speeds for real-time applications by distributing data.
        \end{itemize}
    \end{itemize}
    \begin{block}{Use Cases}
        Applications needing high throughput, such as social media, content management systems, and IoT.
    \end{block}
\end{frame}

\begin{frame}[fragile]
    \frametitle{Conclusion}
    NoSQL databases are integral to the modern data landscape, delivering vital capabilities to manage large, diverse datasets. Understanding the contrasts with traditional RDBMS is crucial for data architects and developers today.
    \begin{block}{Transition Note}
        As we move to the next slide, we will explore the defining characteristics that differentiate NoSQL databases from relational databases and their impact on data architecture.
    \end{block}
\end{frame}

\begin{frame}[fragile]{What is NoSQL?}
    \begin{block}{Definition of NoSQL}
        NoSQL (Not Only SQL) refers to a variety of database management systems that diverge from traditional relational database structures. Unlike relational databases, which utilize structured query language (SQL) for defining and manipulating data, NoSQL systems are designed to handle the needs of modern web applications and large-scale data storage more flexibly.
    \end{block}
\end{frame}

\begin{frame}[fragile]{Key Characteristics of NoSQL Databases}
    \begin{enumerate}
        \item \textbf{Schema Flexibility}
            \begin{itemize}
                \item NoSQL databases typically do not require a fixed schema or data structure, allowing for the storage of unstructured or semi-structured data.
                \item Example: A document-based NoSQL database, like MongoDB, can easily store documents of various shapes (fields) within the same collection.
            \end{itemize}
        
        \item \textbf{Horizontal Scalability}
            \begin{itemize}
                \item NoSQL databases can be expanded easily by adding more servers (nodes) to the database cluster, without requiring significant changes to the existing architecture.
                \item Example: A key-value store like Redis can scale out by adding more nodes to handle increasing loads during peak times.
            \end{itemize}
        
        \item \textbf{High Performance and Availability}
            \begin{itemize}
                \item NoSQL databases are built for high performance and availability, often confirming to CAP theorem—prioritizing availability and partition tolerance over consistency.
                \item Example: In eventual consistency models (like in Cassandra), updates will propagate across nodes eventually, allowing the system to be more available and resilient.
            \end{itemize}
    \end{enumerate}
\end{frame}

\begin{frame}[fragile]{Further Characteristics of NoSQL Databases}
    \begin{enumerate}
        \setcounter{enumi}{3} % Set the counter to continue numbering
        \item \textbf{Variety of Data Models}
            \begin{itemize}
                \item NoSQL databases utilize different data models such as document, key-value, column-family, and graph models, allowing for tailored solutions to specific data storage needs.
                \item Example: Graph databases like Neo4j excel in managing interconnected data, such as social networks.
            \end{itemize}
        
        \item \textbf{Unconventional Query Mechanisms}
            \begin{itemize}
                \item Unlike relational databases that use SQL, NoSQL databases may utilize unique query languages or APIs providing a programming-oriented interface for accessing and manipulating data.
                \item Example: In MongoDB, documents can be queried using JSON-like syntax, enhancing ease of use for developers familiar with JSON.
            \end{itemize}
    \end{enumerate}
\end{frame}

\begin{frame}[fragile]{Key Points and Summary}
    \begin{itemize}
        \item NoSQL is not simply an alternative to SQL; it offers diverse solutions for varying data storage needs.
        \item Flexibility and scalability make NoSQL databases suitable for big data applications and real-time web services.
        \item Selecting between NoSQL and relational databases involves understanding the specific requirements of the application, such as data structure, scale, and performance needs.
    \end{itemize}
    
    \begin{block}{Summary}
        NoSQL databases represent a paradigm shift in database management, catering to the complex data needs of modern applications. By understanding the features and benefits of NoSQL systems, developers can choose the appropriate database structure that aligns with their project requirements.
    \end{block}
\end{frame}

\begin{frame}[fragile]
  \frametitle{Types of NoSQL Databases - Overview}
  \begin{block}{Overview}
    NoSQL databases are designed for flexibility and scalability. They fall into several distinct categories, each serving specific use cases while offering varying features.
  \end{block}
\end{frame}

\begin{frame}[fragile]
  \frametitle{Types of NoSQL Databases - Part 1}
  \begin{enumerate}
    \item \textbf{Document Databases}
      \begin{itemize}
        \item \textbf{Definition:} Store data in documents (JSON, BSON, or XML formats).
        \item \textbf{Key Features:}
          \begin{itemize}
            \item Schema-less, allows storage of documents with different structures.
            \item Nested data support for complex representations.
          \end{itemize}
        \item \textbf{Example:} MongoDB
        \item \textbf{Use Cases:} Content management systems, real-time analytics.
      \end{itemize}
      
    \item \textbf{Key-Value Stores}
      \begin{itemize}
        \item \textbf{Definition:} Store data as key-value pairs where the key is unique.
        \item \textbf{Key Features:}
          \begin{itemize}
            \item High performance and speed, optimized for fast operations.
            \item Simple data model.
          \end{itemize}
        \item \textbf{Example:} Redis
        \item \textbf{Use Cases:} Caching solutions, user session management.
      \end{itemize}
  \end{enumerate}
\end{frame}

\begin{frame}[fragile]
  \frametitle{Types of NoSQL Databases - Part 2}
  \begin{enumerate}
    \setcounter{enumi}{2} % continue numbering
    \item \textbf{Column-Family Databases}
      \begin{itemize}
        \item \textbf{Definition:} Store data in columns organized into column families.
        \item \textbf{Key Features:}
          \begin{itemize}
            \item High scalability, efficient for big data workloads.
            \item Flexible schema design with wide rows.
          \end{itemize}
        \item \textbf{Example:} Apache Cassandra
        \item \textbf{Use Cases:} Time-series data, recommendation systems.
      \end{itemize}
    
    \item \textbf{Graph Databases}
      \begin{itemize}
        \item \textbf{Definition:} Focus on relationships between data points, stored in nodes connected by edges.
        \item \textbf{Key Features:}
          \begin{itemize}
            \item Powerful for relationship-intensive queries.
            \item Optimized for traversing connections between data.
          \end{itemize}
        \item \textbf{Example:} Neo4j
        \item \textbf{Use Cases:} Social networks, fraud detection.
      \end{itemize}
  \end{enumerate}
  
  \begin{block}{Key Points to Remember}
    - NoSQL databases are schema-less and allow flexible data structures.
    - Types cater to:
      \begin{itemize}
        \item Document databases for semi-structured data,
        \item Key-value stores for high-performance apps,
        \item Column-family stores for analytical workloads,
        \item Graph databases for complex relationships.
      \end{itemize}
  \end{block}
\end{frame}

\begin{frame}[fragile]
    \frametitle{Document Databases: MongoDB}
    \begin{block}{Introduction to MongoDB}
        MongoDB is a leading open-source document database system that utilizes a flexible, schema-less design. It allows for managing large volumes of unstructured data. Unlike traditional relational databases, MongoDB stores data in JSON-like documents, facilitating easier handling of disparate data types and structures.
    \end{block}
\end{frame}

\begin{frame}[fragile]
    \frametitle{Data Model}
    \begin{itemize}
        \item \textbf{Documents}: Fundamental units of data represented in BSON (Binary JSON).
            \begin{block}{Example:}
            \begin{lstlisting}[basicstyle=\small\ttfamily]
            {
              "name": "Alice",
              "age": 30,
              "email": "alice@example.com",
              "interests": ["reading", "traveling"]
            }
            \end{lstlisting}
            \end{block}
        \item \textbf{Collections}: Groups of documents like tables in relational databases, with no enforced structure.
        \item \textbf{Database}: Container for collections, forming the highest level in MongoDB's hierarchy.
    \end{itemize}
\end{frame}

\begin{frame}[fragile]
    \frametitle{Features and Use Cases of MongoDB}
    \begin{block}{Key Features}
        \begin{enumerate}
            \item Schema Flexibility: Allows documents to evolve over time.
            \item High Scalability: Supports horizontal scaling and sharding.
            \item Rich Query Language: Powerful queries including filtering, aggregations, and indexing.
            \item Replication and High Availability: Built-in support through Replica Sets.
            \item Integrated Aggregation Framework: Data processing and transformation within the database.
        \end{enumerate}
    \end{block}
    
    \begin{block}{Use Cases}
        \begin{itemize}
            \item Content Management Systems
            \item Real-time Analytics
            \item Internet of Things (IoT)
            \item Mobile Applications
        \end{itemize}
    \end{block}
\end{frame}

\begin{frame}[fragile]
    \frametitle{Code Snippet and Diagram}
    \begin{block}{Example of Inserting a Document}
    \begin{lstlisting}[basicstyle=\small\ttfamily]
    db.users.insertOne({
      name: "Bob",
      age: 25,
      email: "bob@example.com",
      interests: ["sports", "music"]
    });
    \end{lstlisting}
    \end{block}
    
    \begin{block}{Document Structure in MongoDB}
    \begin{lstlisting}[basicstyle=\small\ttfamily]
    [{
      "_id": ObjectId("..."),
      "name": "Alice",
      "age": 30,
      "email": "alice@example.com",
      "interests": ["reading", "traveling"]
    },
    {
      "_id": ObjectId("..."),
      "name": "Bob",
      "age": 25,
      "email": "bob@example.com",
      "interests": ["sports", "music"]
    }]
    \end{lstlisting}
    \end{block}
\end{frame}

\begin{frame}[fragile]
    \frametitle{Key-Value Stores - Overview}
    Key-value stores are a type of NoSQL database designed to store and retrieve data as a collection of key-value pairs. Each key is unique and acts as an identifier for its associated value, which can be simple data types (like strings or integers) or more complex data structures (like JSON objects).
\end{frame}

\begin{frame}[fragile]
    \frametitle{Key-Value Stores - Structure}
    \begin{itemize}
        \item \textbf{Key:} A unique identifier, often a string (e.g., "user123").
        \item \textbf{Value:} The data associated with the key, which can range from text to binary data.
    \end{itemize}

    \begin{block}{Example Structure}
        \begin{verbatim}
key: "user123"
value: {
    "name": "John Doe",
    "age": 30,
    "email": "johndoe@example.com"
}
        \end{verbatim}
    \end{block}
\end{frame}

\begin{frame}[fragile]
    \frametitle{Key-Value Stores - Benefits}    
    \begin{enumerate}
        \item \textbf{Simplicity:} The data model is straightforward.
        \item \textbf{Performance:} Optimized for fast read and write operations.
        \item \textbf{Scalability:} Can handle large volumes of data seamlessly.
        \item \textbf{Flexibility:} Supports various data types, offering versatility.
    \end{enumerate}
\end{frame}

\begin{frame}[fragile]
    \frametitle{Key-Value Stores - Examples}
    \begin{itemize}
        \item \textbf{Redis:} 
        An in-memory data structure store, commonly used as a database, cache, and message broker.
        \begin{block}{Basic Redis Example}
        \begin{verbatim}
SET user:1000 "{ 'name': 'John Doe', 'age': 30 }"
GET user:1000
        \end{verbatim}
        \end{block}

        \item \textbf{Amazon DynamoDB:} 
        A fully managed NoSQL database service providing fast and predictable performance.
    \end{itemize}
\end{frame}

\begin{frame}[fragile]
    \frametitle{Key-Value Stores - Key Points}
    \begin{itemize}
        \item Key-value stores are beneficial for applications requiring high speed and scalability.
        \item Suitable for caching, session storage, and real-time analytics.
        \item Important to understand the limitations, such as the lack of complex querying.
    \end{itemize}

    By recognizing the structure and benefits, students can appreciate the design principles and practical applications of key-value stores, leading to better decision-making in database selection.
\end{frame}

\begin{frame}[fragile]
    \frametitle{Introduction to Apache Cassandra}
    \begin{itemize}
        \item Highly scalable, distributed NoSQL database
        \item Handles large amounts of structured data across commodity servers
        \item Provides high availability and performance
        \item Suitable for fast and reliable data access applications
    \end{itemize}
\end{frame}

\begin{frame}[fragile]
    \frametitle{Key Concepts}
    \begin{enumerate}
        \item \textbf{Data Model}:
        \begin{itemize}
            \item \textbf{Column-Family Store}: Flexible data storage in column families
            \item \textbf{Rows and Columns}: Unique key identification for rows, flexible number of values in columns
            \item \textbf{Schema Flexibility}: Different column sets in the same column family for schema evolution
            \item \textbf{Illustration}: Users as rows with varying attributes
        \end{itemize}
        
        \item \textbf{Architecture}:
        \begin{itemize}
            \item \textbf{Peer-to-Peer}: All nodes equal, avoiding master node bottlenecks
            \item \textbf{Partitioning}: Automatic data distribution using partition keys
        \end{itemize}
    \end{enumerate}
\end{frame}

\begin{frame}[fragile]
    \frametitle{Features and Use Cases}
    \begin{itemize}
        \item \textbf{Features}:
        \begin{itemize}
            \item High Availability: No single point of failure
            \item Scalability: Horizontal scaling with no downtime
            \item Performance: High write and read throughput
            \item Tunable Consistency: Set consistency levels per query
        \end{itemize}
        
        \item \textbf{Use Cases}:
        \begin{itemize}
            \item Real-Time Analytics: Social media, IoT data
            \item Content Management: Fast retrieval for blogs, games
            \item Messaging Systems: Processing large volumes of messages
        \end{itemize}
    \end{itemize}
\end{frame}

\begin{frame}[fragile]
    \frametitle{Example of Data Structure}
    \begin{itemize}
        \item \textbf{Healthcare Application Example}:
        \begin{itemize}
            \item \textbf{Column Family}: Patients
            \item \textbf{Row Identifier}: PatientID
            \item \textbf{Columns}:
            \begin{itemize}
                \item Name
                \item Age
                \item Diagnosis
                \item TreatmentHistory
            \end{itemize}
        \end{itemize}
        
        \item \textbf{Key Points to Emphasize}:
        \begin{itemize}
            \item Designed for high availability and scalability
            \item Flexible data model allows easy adjustments
            \item Ideal for real-time access to large datasets
        \end{itemize}
    \end{itemize}
\end{frame}

\begin{frame}[fragile]
    \frametitle{Code Snippet for Data Insertion}
    \begin{block}{CQL Code}
    \begin{lstlisting}[language=sql]
INSERT INTO Patients (PatientID, Name, Age, Diagnosis)
VALUES ('12345', 'John Doe', 30, 'Hypertension');
    \end{lstlisting}
    \end{block}
    \begin{itemize}
        \item Example of inserting a new patient record
        \item Demonstrates schema flexibility
    \end{itemize}
\end{frame}

\begin{frame}[fragile]
    \frametitle{Comparing MongoDB and Cassandra - Introduction}
    \begin{itemize}
        \item Both MongoDB and Apache Cassandra are popular NoSQL databases.
        \item They manage large volumes of unstructured data.
        \item Key comparison areas:
            \begin{itemize}
                \item Scalability
                \item Performance
                \item Use Cases
            \end{itemize}
    \end{itemize}
\end{frame}

\begin{frame}[fragile]
    \frametitle{Comparing MongoDB and Cassandra - Scalability}
    \begin{block}{MongoDB}
        \begin{itemize}
            \item Utilizes \textbf{Horizontal Scaling} (sharding).
            \item Each shard improves performance as data grows.
            \item Supports automatic sharding and replication for high availability.
        \end{itemize}
    \end{block}

    \begin{block}{Cassandra}
        \begin{itemize}
            \item Offers \textbf{Linear Scalability}.
            \item Uses consistent hashing for even data distribution.
            \item Designed for a decentralized architecture, eliminating single points of failure.
        \end{itemize}
    \end{block}
\end{frame}

\begin{frame}[fragile]
    \frametitle{Comparing MongoDB and Cassandra - Performance}
    \begin{block}{MongoDB}
        \begin{itemize}
            \item Excellent read performance due to in-memory processing.
            \item Flexible schema allows easy changes to data structure.
            \item Performance may degrade with heavy write loads if not optimized.
        \end{itemize}
    \end{block}

    \begin{block}{Cassandra}
        \begin{itemize}
            \item Optimized for heavy write operations.
            \item Utilizes a log-structured storage system for fast operations.
            \item Tunable consistency levels provide flexibility.
        \end{itemize}
    \end{block}
\end{frame}

\begin{frame}[fragile]
    \frametitle{Comparing MongoDB and Cassandra - Use Cases}
    \begin{block}{MongoDB}
        \begin{itemize}
            \item Suitable for:
                \begin{itemize}
                    \item Rapid development with changing data schemas (e.g., content management systems).
                    \item Complex queries (e.g., social media applications).
                \end{itemize}
        \end{itemize}
    \end{block}

    \begin{block}{Cassandra}
        \begin{itemize}
            \item Ideal for:
                \begin{itemize}
                    \item High availability and fault tolerance (e.g., IoT applications).
                    \item Time-series data and write-heavy workloads (e.g., log data aggregation).
                \end{itemize}
        \end{itemize}
    \end{block}
\end{frame}

\begin{frame}[fragile]
    \frametitle{Key Points and Conclusion}
    \begin{itemize}
        \item \textbf{Data Model}: MongoDB (document-based), Cassandra (column-family).
        \item \textbf{Replication}: Cassandra’s approach minimizes downtime.
        \item \textbf{Consistency Models}: MongoDB - strong consistency, Cassandra - configurable consistency levels.
    \end{itemize}
    
    \begin{block}{Conclusion}
        Evaluate your project's requirements to choose between MongoDB and Cassandra, as they offer distinct advantages for different data management needs.
    \end{block}
\end{frame}

\begin{frame}[fragile]
    \frametitle{Example Code Snippet}
    \begin{block}{MongoDB}
        \begin{lstlisting}[language=JavaScript]
db.collection.insertOne({
   name: "John Doe", 
   age: 28, 
   interests: ["coding", "traveling"]
});
        \end{lstlisting}
    \end{block}

    \begin{block}{Cassandra}
        \begin{lstlisting}[language=CQL]
INSERT INTO users (username, age) VALUES ('JohnDoe', 28);
        \end{lstlisting}
    \end{block}
\end{frame}

\begin{frame}[fragile]
    \frametitle{Distributed Database Architecture}
    \begin{block}{What is Distributed Database Architecture?}
        Distributed Database Architecture refers to a model where data is stored across multiple physical locations. 
        This architecture is foundational in NoSQL databases, facilitating efficient data processing by distributing workloads, enhancing availability, and supporting scalability.
    \end{block}
\end{frame}

\begin{frame}[fragile]
    \frametitle{Key Concepts of Distributed Database Architecture}
    \begin{itemize}
        \item \textbf{Data Distribution:}
        \begin{itemize}
            \item Data is partitioned and spread across different nodes (servers).
            \item Each node handles a portion of the requests, improving performance.
        \end{itemize}
        
        \item \textbf{Replication:}
        \begin{itemize}
            \item Data is duplicated across nodes to enhance availability and fault tolerance.
            \item Other replicas can serve data without interruption in case of a node failure.
        \end{itemize}
        
        \item \textbf{Horizontal Scaling:}
        \begin{itemize}
            \item New nodes can be easily added, accommodating growth with minimal downtime.
            \item This approach is a significant advantage compared to traditional vertical scaling.
        \end{itemize}
    \end{itemize}
\end{frame}

\begin{frame}[fragile]
    \frametitle{Advantages of Distributed Architecture in NoSQL Databases}
    \begin{enumerate}
        \item \textbf{High Availability:}
        \begin{itemize}
            \item Systems remain operational even if one or more nodes fail due to data replication across several nodes.
            \item Example: Apache Cassandra's architecture continues functioning despite node failures.
        \end{itemize}
        
        \item \textbf{Scalability:}
        \begin{itemize}
            \item Scale out easily by adding more servers as data volume grows.
            \item Example: Amazon DynamoDB scales automatically for variable workloads.
        \end{itemize}
        
        \item \textbf{Performance:}
        \begin{itemize}
            \item Low latency through local data access, leading to more rapid data processing.
        \end{itemize}
        
        \item \textbf{Fault Tolerance:}
        \begin{itemize}
            \item The distributed nature helps maintain services during hardware or network failures, with techniques like gossip protocols.
        \end{itemize}
    \end{enumerate}
\end{frame}

\begin{frame}[fragile]
    \frametitle{Example: MongoDB's Sharding}
    \begin{block}{Sharding in MongoDB}
        \begin{itemize}
            \item \textbf{Sharding} distributes data across multiple servers:
            \begin{itemize}
                \item Collections are divided into shards, each on different servers.
                \item A shard key determines how documents are distributed.
            \end{itemize}
            \item This approach enables efficient management of large datasets and uniform performance.
        \end{itemize}
    \end{block}
    
    \begin{example}
        Example Sharding Key:
        \begin{lstlisting}
        {
            "userId": 12345
        }
        \end{lstlisting}
    \end{example}
\end{frame}

\begin{frame}[fragile]
    \frametitle{Key Takeaways}
    \begin{itemize}
        \item Distributed Database Architecture enhances performance, scalability, and fault tolerance in NoSQL systems.
        \item Understanding data distribution and replication is crucial for creating resilient applications.
        \item NoSQL databases like MongoDB and Cassandra leverage these architectures for efficient data handling.
    \end{itemize}
    
    \begin{block}{Next Steps}
        Prepare for a deeper understanding of how these architectures relate to the \textbf{CAP Theorem}, which discusses trade-offs in distributed environments!
    \end{block}
\end{frame}

\begin{frame}[fragile]
  \frametitle{CAP Theorem Explained - Overview}
  \begin{block}{Overview of the CAP Theorem}
    The CAP Theorem, proposed by computer scientist Eric Brewer, states that in a distributed data store, you can only guarantee two out of the following three characteristics at any given time:
  \end{block}
  \begin{enumerate}
    \item \textbf{Consistency (C)}: Every read receives the most recent write or an error. All nodes in the distributed system see the same data at the same time.
    
    \item \textbf{Availability (A)}: Every request (read/write) receives a response, even if the data returned might be stale. The system remains operational and responsive.
    
    \item \textbf{Partition Tolerance (P)}: The system continues to operate despite network partitions, meaning that some nodes can’t communicate with others.
  \end{enumerate}
\end{frame}

\begin{frame}[fragile]
  \frametitle{CAP Theorem Explained - Key Points}
  \begin{block}{Key Points}
    \begin{itemize}
      \item \textbf{Trade-offs}: NoSQL databases must choose between providing consistency, availability, or partition tolerance, particularly in scenarios involving network failures or massive data loads.
      \item \textbf{Real-World Applications}:
        \begin{itemize}
          \item \textbf{Cassandra}: Prioritizes availability and partition tolerance, allowing stale reads.
          \item \textbf{MongoDB}: Generally achieves a balance between consistency and availability, adapting based on configurations in distributed setups.
        \end{itemize}
    \end{itemize}
  \end{block}
\end{frame}

\begin{frame}[fragile]
  \frametitle{CAP Theorem Explained - Examples and Conclusion}
  \begin{block}{Examples}
    \textbf{Scenario}: Consider a shopping application where users place orders. If the network fails while updates are propagated:
    \begin{itemize}
      \item \textbf{Option 1 (CA)}: If we prioritize consistency, some users may not be able to place orders until the network is restored.
      \item \textbf{Option 2 (AP)}: If we prioritize availability, users can place orders, but they might receive outdated stock levels.
    \end{itemize}
  \end{block}

  \begin{block}{Conclusion}
    Understanding the CAP Theorem is critical for designing effective NoSQL databases. By recognizing the trade-offs between consistency, availability, and partition tolerance, developers can make informed decisions that align with application requirements and user expectations.
  \end{block}

  \begin{block}{Key Formula}
    Although the CAP theorem does not involve a mathematical formula, it can be represented symbolically as:
    \[
    \text{Choose 2 out of 3: } C, A, P 
    \]
  \end{block}
\end{frame}

\begin{frame}[fragile]
    \frametitle{Performance in NoSQL Databases}
    \begin{block}{Overview}
        Insights into performance considerations, such as read/write speeds, and choosing the right database.
    \end{block}
\end{frame}

\begin{frame}[fragile]
    \frametitle{Understanding Performance Considerations in NoSQL}
    \begin{itemize}
        \item Performance is crucial when selecting a NoSQL database.
        \item Focus on two main performance criteria:
            \begin{enumerate}
                \item \textbf{Read Speeds}: Data retrieval speed.
                \item \textbf{Write Speeds}: Data storage speed.
            \end{enumerate}
    \end{itemize}
\end{frame}

\begin{frame}[fragile]
    \frametitle{Factors Influencing Performance}
    \begin{itemize}
        \item \textbf{Data Model}: Different NoSQL databases have varied models affecting performance.
        \item \textbf{Consistency vs. Availability}: Trade-offs from the CAP theorem.
        \item \textbf{Sharding and Replication}:
            \begin{itemize}
                \item Sharding divides data for better performance.
                \item Replication enhances availability but may introduce latency.
            \end{itemize}
        \item \textbf{Indexing}: Proper indexing can significantly improve read performance.
        \item \textbf{Hardware and Network Latency}: Influences overall database performance.
    \end{itemize}
\end{frame}

\begin{frame}[fragile]
    \frametitle{Choosing the Right NoSQL Database for Performance}
    \begin{itemize}
        \item \textbf{Use Cases}: Assess your application's needs.
            \begin{itemize}
                \item Consider \textbf{Cassandra} for write-heavy applications.
                \item Consider \textbf{Redis} for high-speed in-memory data access.
            \end{itemize}
        \item \textbf{Benchmarking}: Test different databases based on workload.
    \end{itemize}
\end{frame}

\begin{frame}[fragile]
    \frametitle{Key Points to Remember}
    \begin{itemize}
        \item Performance is a balance between read and write speeds.
        \item Influenced by data model, consistency, sharding, replication, and indexing.
        \item Align your database choice with your application's specific requirements.
        \item Conduct regular performance testing as data grows.
    \end{itemize}
\end{frame}

\begin{frame}[fragile]
    \frametitle{Use Cases for NoSQL Databases}
    \begin{block}{Introduction to NoSQL Databases}
        NoSQL databases provide flexible data models and are designed to handle large volumes of unstructured data, making them ideal for various application scenarios across multiple industries. Unlike traditional relational databases, NoSQL databases can easily scale horizontally and often allow for rapid development cycles.
    \end{block}
\end{frame}

\begin{frame}[fragile]
    \frametitle{Key Use Cases of NoSQL Databases}
    \begin{enumerate}
        \item \textbf{Big Data Applications}
        \begin{itemize}
            \item \textbf{Example:} Social Media Platforms
            \item \textbf{Explanation:} Platforms like Facebook and Twitter deal with vast amounts of user-generated content, requiring databases that can handle petabytes of data.
        \end{itemize}

        \item \textbf{Real-Time Data Processing}
        \begin{itemize}
            \item \textbf{Example:} Online Gaming
            \item \textbf{Explanation:} Multiplayer online games manage multiple player interactions in real-time. NoSQL databases like Redis enable quick data management.
        \end{itemize}

        \item \textbf{Content Management Systems (CMS)}
        \begin{itemize}
            \item \textbf{Example:} Media Streaming Services
            \item \textbf{Explanation:} Services like Netflix utilize NoSQL databases to store diverse media metadata and user preferences.
        \end{itemize}
    \end{enumerate}
\end{frame}

\begin{frame}[fragile]
    \frametitle{Additional Use Cases and Conclusion}
    \begin{enumerate}
        \setcounter{enumi}{3} % Start numbering from 4
        \item \textbf{Internet of Things (IoT)}
        \begin{itemize}
            \item \textbf{Example:} Smart Home Devices
            \item \textbf{Explanation:} IoT devices generate continuous data that NoSQL databases can store for real-time retrieval.
        \end{itemize}

        \item \textbf{E-commerce Applications}
        \begin{itemize}
            \item \textbf{Example:} Retail Giants
            \item \textbf{Explanation:} Companies like Amazon use NoSQL to manage inventory and customer data efficiently.
        \end{itemize}
    \end{enumerate}

    \begin{block}{Conclusion}
        NoSQL databases excel in handling unstructured data, real-time processing needs, and flexible schema designs crucial for modern applications.
    \end{block}
    
    \begin{block}{Call to Action}
        Explore specific NoSQL databases relevant to your projects and consider how their features can meet your data challenges.
    \end{block}
\end{frame}

\begin{frame}
    \frametitle{Data Modeling in NoSQL}
    \begin{block}{Introduction}
        Data modeling is crucial for defining data structures and organization. In NoSQL systems, it varies significantly from relational databases, particularly in handling relationships among data entities. Here, we explore key differences, definitions, and examples that highlight these distinctions.
    \end{block}
\end{frame}

\begin{frame}
    \frametitle{Key Differences: Data Structure}
    \begin{itemize}
        \item \textbf{Relational Databases:} 
            \begin{itemize}
                \item Data is stored in structured tables with a predefined schema.
                \item Rows represent records and columns represent attributes.
            \end{itemize}
        \item \textbf{NoSQL Databases:} 
            \begin{itemize}
                \item Data can be stored in various formats (e.g., key-value pairs, documents, wide-column stores, graphs).
                \item Offers a dynamic schema.
            \end{itemize}
    \end{itemize}
    \textbf{Example:}  
    \begin{lstlisting}[basicstyle=\footnotesize]
    Customers Table:
    | CustomerID | Name     | Age | Email             |
    |-------------|----------|-----|-------------------|
    | 1           | Alice    | 30  | alice@example.com  |
    | 2           | Bob      | 25  | bob@example.com    |
    \end{lstlisting}
    \begin{lstlisting}[language=json, basicstyle=\footnotesize]
    [
      { "CustomerID": 1, "Name": "Alice", "Age": 30, "Email": "alice@example.com" },
      { "CustomerID": 2, "Name": "Bob", "Age": 25, "Email": "bob@example.com" }
    ]
    \end{lstlisting}
\end{frame}

\begin{frame}
    \frametitle{Key Differences: Schema Flexibility and Query Language}
    \begin{itemize}
        \item \textbf{Schema Flexibility:}
            \begin{itemize}
                \item \textbf{Relational:} Requires a fixed schema; changes may incur downtime.
                \item \textbf{NoSQL:} Allows schema-less design; new fields can be added without downtime.
            \end{itemize}
        \item \textbf{Normalization vs. Denormalization:}
            \begin{itemize}
                \item \textbf{Relational:} Emphasizes normalization to minimize redundancy (often uses JOIN operations).
                \item \textbf{NoSQL:} Typically denormalized for better read performance; related data stored together.
            \end{itemize}
            \textbf{Example:} 
            \begin{lstlisting}[language=json, basicstyle=\footnotesize]
            {
              "CustomerID": 1,
              "Name": "Alice",
              "Orders": [
                { "OrderID": 101, "Amount": 250 },
                { "OrderID": 102, "Amount": 150 }
              ]
            }
            \end{lstlisting}
        \item \textbf{Query Language:}
            \begin{itemize}
                \item \textbf{Relational:} Uses SQL with declarative syntax.
                \item \textbf{NoSQL:} Varies by database type; often offers more flexible query options.
            \end{itemize}
    \end{itemize}
\end{frame}

\begin{frame}
    \frametitle{Conclusion and Key Points}
    \begin{itemize}
        \item NoSQL databases provide scalability and performance benefits for large datasets and high-traffic applications.
        \item Flexibility in data modeling supports evolving business needs and diverse data types.
        \item Understanding foundational differences informs better decision-making for application design.
        \item Transitioning from relational to NoSQL databases requires rethinking data relationships and structures.
    \end{itemize}
\end{frame}

\begin{frame}[fragile]
    \frametitle{Scaling NoSQL Databases - Overview}
    \begin{itemize}
        \item NoSQL databases handle large volumes of data.
        \item Scalability and performance are critical for modern applications.
        \item Two primary techniques for scaling:
        \begin{itemize}
            \item \textbf{Sharding}
            \item \textbf{Replication}
        \end{itemize}
    \end{itemize}
\end{frame}

\begin{frame}[fragile]
    \frametitle{Scaling NoSQL Databases - Sharding}
    \begin{block}{Definition}
        Sharding is dividing a large database into smaller, manageable pieces called "shards."
    \end{block}
    
    \begin{block}{How It Works}
        \begin{itemize}
            \item \textbf{Horizontal Partitioning}: Data is split based on a shard key.
            \item Each shard operates independently, allowing for parallel processing of queries.
        \end{itemize}
    \end{block}
    
    \begin{block}{Example}
        Consider a social media application:
        \begin{itemize}
            \item IDs 1-100,000 go to Shard A.
            \item IDs 100,001-200,000 go to Shard B.
            \item New shards can be added as the user base grows.
        \end{itemize}
    \end{block}
\end{frame}

\begin{frame}[fragile]
    \frametitle{Scaling NoSQL Databases - Replication}
    \begin{block}{Definition}
        Replication creates copies of data across multiple servers for enhanced availability and reliability.
    \end{block}
    
    \begin{block}{How It Works}
        \begin{itemize}
            \item \textbf{Master-Slave Replication}: One node handles write requests; replicas sync for read requests.
            \item \textbf{Multi-Master Replication}: Multiple nodes accept writes, syncing changes while addressing conflict resolution.
        \end{itemize}
    \end{block}
    
    \begin{block}{Example}
        In a content delivery application:
        \begin{itemize}
            \item Master database stores blog posts.
            \item Replica databases serve read requests, improving read performance.
        \end{itemize}
    \end{block}
\end{frame}

\begin{frame}[fragile]
    \frametitle{Scaling NoSQL Databases - Conclusion}
    \begin{itemize}
        \item Sharding enhances performance by distributing data.
        \item Replication ensures high availability and fault tolerance.
        \item Align strategies with application needs and consider data integrity.
    \end{itemize}
\end{frame}

\begin{frame}[fragile]
    \frametitle{Challenges with NoSQL Systems}
    
    \begin{block}{Introduction}
        NoSQL databases provide flexibility and scalability for handling large volumes of unstructured and semi-structured data.
        However, their implementation introduces various challenges that must be navigated effectively.
    \end{block}
\end{frame}

\begin{frame}[fragile]
    \frametitle{Common Challenges with NoSQL}
    
    \begin{enumerate}
        \item \textbf{Data Consistency}
            \begin{itemize}
                \item Many NoSQL systems sacrifice consistency for availability and partition tolerance (CAP theorem).
                \item Example: An update to one node may not immediately reflect on another, leading to discrepancies.
            \end{itemize}
        
        \item \textbf{Complex Queries}
            \begin{itemize}
                \item Lack of advanced querying capabilities makes complex queries cumbersome.
                \item Example: Joins across collections may require pre-aggregation or application-level processing.
            \end{itemize}
    \end{enumerate}
\end{frame}

\begin{frame}[fragile]
    \frametitle{More Challenges with NoSQL}
    
    \begin{enumerate}[resume]
        \item \textbf{Data Modeling}
            \begin{itemize}
                \item Designing data models differs from relational databases and requires careful planning.
                \item Example: Choosing the right model (document, key-value, column-family, or graph) can be non-intuitive.
            \end{itemize}
        
        \item \textbf{Lack of Standardization}
            \begin{itemize}
                \item Diversity in NoSQL technologies leads to a steep learning curve.
                \item Example: Transitioning from MongoDB to Cassandra requires learning new query languages and data structures.
            \end{itemize}
        
        \item \textbf{Operational Management}
            \begin{itemize}
                \item Maintenance tasks (backups, monitoring, scaling) can be more complex.
                \item Key Point: Automated management tools may be necessary for smooth operations.
            \end{itemize}
    \end{enumerate}
\end{frame}

\begin{frame}[fragile]
    \frametitle{Final Challenges with NoSQL}
    
    \begin{enumerate}[resume]
        \item \textbf{Data Security}
            \begin{itemize}
                \item Many NoSQL databases lack built-in security features of traditional RDBMS.
                \item Example: Implementing authentication and encryption often depends on the specific system.
            \end{itemize}
        
        \item \textbf{Integration Challenges}
            \begin{itemize}
                \item Integrating NoSQL databases with existing legacy systems can be difficult.
                \item Example: ETL processes may need significant reworking for schema-less data.
            \end{itemize}
    \end{enumerate}
\end{frame}

\begin{frame}[fragile]
    \frametitle{Overcoming the Challenges}
    
    To effectively leverage the benefits of NoSQL while mitigating challenges, organizations can:
    \begin{itemize}
        \item \textbf{Invest in Training}: Educate teams on database selection, design, and maintenance.
        \item \textbf{Use Hybrid Solutions}: Combine SQL and NoSQL databases as per project requirements.
        \item \textbf{Implement Robust Monitoring and Security Measures}: Ensure data integrity and availability with proper tools.
    \end{itemize}
\end{frame}

\begin{frame}[fragile]
    \frametitle{Conclusion and Key Takeaways}
    
    \begin{block}{Conclusion}
        While NoSQL databases present unique challenges, understanding them can lead to informed decisions and successful implementations.
        Mastering these challenges is crucial as data continues to grow and evolve.
    \end{block}
    
    \begin{itemize}
        \item \textbf{Understand trade-offs}: How consistency, availability, and partition tolerance interact.
        \item \textbf{Choose wisely}: Select the right NoSQL database based on application needs.
        \item \textbf{Plan for scalability}: Design systems with future growth in mind.
    \end{itemize}
\end{frame}

\begin{frame}[fragile]
    \frametitle{Future Trends in NoSQL}
    \begin{block}{Overview}
        NoSQL databases have evolved significantly and continue to transform data processing and analytics. Emerging trends highlight the ongoing adaptation of NoSQL systems to meet the challenges of modern data demands, user needs, and technological advancements.
    \end{block}
\end{frame}

\begin{frame}[fragile]
    \frametitle{Key Future Trends in NoSQL - Part 1}
    \begin{enumerate}
        \item \textbf{Increased Adoption of Multi-Model Databases}
            \begin{itemize}
                \item \textbf{Explanation}: Multi-model databases support different data models (document, key-value, graph, etc.) within a single platform.
                \item \textbf{Example}: ArangoDB allows users to work with documents and graphs seamlessly.
            \end{itemize}
            
        \item \textbf{Integration with AI and Machine Learning}
            \begin{itemize}
                \item \textbf{Explanation}: NoSQL systems are becoming integral to AI/ML workflows by providing scalable storage and fast querying.
                \item \textbf{Example}: Using Apache Cassandra for efficient data retrieval in real-time machine learning applications.
            \end{itemize}
    \end{enumerate}
\end{frame}

\begin{frame}[fragile]
    \frametitle{Key Future Trends in NoSQL - Part 2}
    \begin{enumerate}
        \setcounter{enumi}{2} % continue enumeration
        \item \textbf{Serverless and Cloud-Native Architectures}
            \begin{itemize}
                \item \textbf{Explanation}: Many NoSQL databases are moving towards serverless and fully-managed cloud solutions.
                \item \textbf{Example}: AWS DynamoDB offers a serverless approach that scales based on demand.
            \end{itemize}

        \item \textbf{Enhanced Security Features}
            \begin{itemize}
                \item \textbf{Explanation}: Advanced security measures are being implemented due to increased data privacy regulations.
                \item \textbf{Example}: MongoDB Atlas provides features like VPC peering and end-to-end encryption.
            \end{itemize}

        \item \textbf{Focus on Real-Time Data Processing}
            \begin{itemize}
                \item \textbf{Explanation}: Growing requirements for real-time analytics are driving enhancements in NoSQL processing capabilities.
                \item \textbf{Example}: Apache Kafka, when integrated with NoSQL databases like MongoDB, enables real-time streaming analytics.
            \end{itemize}
    \end{enumerate}
\end{frame}

\begin{frame}[fragile]
    \frametitle{Key Points and Conclusion}
    \begin{block}{Key Points to Emphasize}
        \begin{itemize}
            \item \textbf{Flexibility and Scalability}: NoSQL databases can efficiently handle large volumes of diverse data.
            \item \textbf{Innovation in Data Processing}: Advancements in cloud technology, AI integration, and security improve NoSQL systems.
            \item \textbf{Community and Ecosystem Growth}: A thriving community supports rapid evolution and enhancements of NoSQL technologies.
        \end{itemize}
    \end{block}
    
    \begin{block}{Conclusion}
        As we move into the future, NoSQL databases will become even more crucial for businesses, enabling effective data management in a complex digital landscape.
    \end{block}
\end{frame}

\begin{frame}[fragile]
    \frametitle{Conclusion and Q\&A - Key Points Overview}
    \begin{itemize}
        \item \textbf{NoSQL Database Fundamentals}
            \begin{itemize}
                \item Definition: Designed for unstructured and semi-structured data.
                \item Types of NoSQL Databases:
                    \begin{itemize}
                        \item Document Stores (e.g., MongoDB)
                        \item Key-Value Stores (e.g., Redis)
                        \item Column Family Stores (e.g., Cassandra)
                        \item Graph Databases (e.g., Neo4j)
                    \end{itemize}
            \end{itemize}
        \item \textbf{Benefits of NoSQL Systems}
            \begin{itemize}
                \item Scalability
                \item Flexibility
                \item Performance
            \end{itemize}
    \end{itemize}
\end{frame}

\begin{frame}[fragile]
    \frametitle{Conclusion and Q\&A - Use Cases and Challenges}
    \begin{itemize}
        \item \textbf{Use Cases and Applications}
            \begin{itemize}
                \item Big Data Analytics
                \item Content Management Systems
                \item IoT Applications
            \end{itemize}
        \item \textbf{Challenges and Considerations}
            \begin{itemize}
                \item Consistency: Eventual consistency vs strong consistency
                \item Complex Queries: Limitations compared to SQL
                \item Ecosystem: Integration and understanding of data models
            \end{itemize}
    \end{itemize}
\end{frame}

\begin{frame}[fragile]
    \frametitle{Conclusion and Q\&A - Interaction}
    \begin{block}{Key Takeaways}
        \begin{itemize}
            \item NoSQL databases enable high scalability and performance.
            \item Choosing the right NoSQL type is crucial for system design.
            \item Stay updated on emerging NoSQL technologies.
        \end{itemize}
    \end{block}
    \begin{block}{Q\&A Session}
        \begin{itemize}
            \item Encourage Questions: Invite clarifications on NoSQL concepts and examples.
            \item Discussion Points:
                \begin{itemize}
                    \item How might NoSQL affect your current projects?
                    \item Challenges you've faced with NoSQL databases?
                \end{itemize}
        \end{itemize}
    \end{block}
\end{frame}


\end{document}