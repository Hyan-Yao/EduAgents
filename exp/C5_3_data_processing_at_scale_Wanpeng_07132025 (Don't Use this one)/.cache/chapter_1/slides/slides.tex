\documentclass[aspectratio=169]{beamer}

% Theme and Color Setup
\usetheme{Madrid}
\usecolortheme{whale}
\useinnertheme{rectangles}
\useoutertheme{miniframes}

% Additional Packages
\usepackage[utf8]{inputenc}
\usepackage[T1]{fontenc}
\usepackage{graphicx}
\usepackage{booktabs}
\usepackage{listings}
\usepackage{amsmath}
\usepackage{amssymb}
\usepackage{xcolor}
\usepackage{tikz}
\usepackage{pgfplots}
\pgfplotsset{compat=1.18}
\usetikzlibrary{positioning}
\usepackage{hyperref}

% Custom Colors
\definecolor{myblue}{RGB}{31, 73, 125}
\definecolor{mygray}{RGB}{100, 100, 100}
\definecolor{mygreen}{RGB}{0, 128, 0}
\definecolor{myorange}{RGB}{230, 126, 34}
\definecolor{mycodebackground}{RGB}{245, 245, 245}

% Set Theme Colors
\setbeamercolor{structure}{fg=myblue}
\setbeamercolor{frametitle}{fg=white, bg=myblue}
\setbeamercolor{title}{fg=myblue}
\setbeamercolor{section in toc}{fg=myblue}
\setbeamercolor{item projected}{fg=white, bg=myblue}
\setbeamercolor{block title}{bg=myblue!20, fg=myblue}
\setbeamercolor{block body}{bg=myblue!10}
\setbeamercolor{alerted text}{fg=myorange}

% Set Fonts
\setbeamerfont{title}{size=\Large, series=\bfseries}
\setbeamerfont{frametitle}{size=\large, series=\bfseries}
\setbeamerfont{caption}{size=\small}
\setbeamerfont{footnote}{size=\tiny}

% Document Start
\begin{document}

\frame{\titlepage}

\begin{frame}[fragile]
  \title{Week 1: Introduction to Data Models}
  \subtitle{Overview of the chapter and importance of data models in data processing.}
  \author{John Smith, Ph.D.}
  \date{\today}
  \maketitle
\end{frame}

\begin{frame}[fragile]
  \frametitle{Introduction to Data Models}
  \begin{block}{Overview of Data Models}
    Data models are foundational blueprints that define how data is stored, processed, and utilized. They are crucial in structuring databases and facilitating effective data management within various systems.
  \end{block}
\end{frame}

\begin{frame}[fragile]
  \frametitle{Importance of Data Models in Data Processing}
  \begin{itemize}
    \item \textbf{Framework for Data Organization:} 
    Data models provide a structured way to organize data, outlining relationships between data elements for efficient retrieval and manipulation.
    
    \item \textbf{Standardization:} 
    Using data models enhances standardization across organizations, crucial for data quality and integrity, allowing effective interpretation and use by multiple stakeholders.
    
    \item \textbf{Facilitating Communication:} 
    They act as a common language among developers, database administrators, and end-users, clarifying data interactions within a system.
    
    \item \textbf{Enhancing Data Analytics:} 
    Well-defined models support advanced analytics and aid in identifying trends and patterns for data-driven decision-making.
  \end{itemize}
\end{frame}

\begin{frame}[fragile]
  \frametitle{Examples of Data Models}
  \begin{enumerate}
    \item \textbf{Hierarchical Model:} 
    Organizes data in a tree-like structure with each record having a single parent. 
    \begin{itemize}
      \item \textit{Example:} An organizational chart representing employees and their reporting lines.
    \end{itemize}
    
    \item \textbf{Relational Model:} 
    Data is stored in tables with relationships established through foreign keys, widely used in modern databases. 
    \begin{itemize}
      \item \textit{Example:} A database table for customers linked to a separate table for orders via customer ID.
    \end{itemize}
    
    \item \textbf{Entity-Relationship (ER) Model:} 
    This framework illustrates how entities relate to each other, often visualized through ER diagrams. 
    \begin{itemize}
      \item \textit{Example:} A diagram showing relationships between customers, orders, and products.
    \end{itemize}
  \end{enumerate}
\end{frame}

\begin{frame}[fragile]
  \frametitle{Key Points to Emphasize}
  \begin{itemize}
    \item Data models are essential for effective data organization and management.
    \item They foster consistency and clarity across data-related tasks.
    \item Understanding different data models is crucial for selecting the appropriate model for various applications.
  \end{itemize}
\end{frame}

\begin{frame}[fragile]
  \frametitle{Summary}
  In this chapter, we will explore the various types of data models and their applications. By understanding these concepts, students will gain insight into building efficient data processing systems and enhancing data management strategies. This foundational knowledge is critical for anyone pursuing a career in data science, database administration, or analytics.

  \begin{block}{Conclusion}
    With this introduction, we encourage students to reflect on the importance of data models, setting the stage for further exploration in subsequent slides.
  \end{block}
\end{frame}

\begin{frame}[fragile]
    \frametitle{What is a Data Model? - Definition}
    \begin{block}{Definition}
        A \textbf{data model} is a conceptual framework for organizing and structuring data in a database. 
        It defines how data is connected, stored, and accessed. 
        Data models serve as blueprints for database design and dictate the rules for data interaction within a system.
    \end{block}
\end{frame}

\begin{frame}[fragile]
    \frametitle{What is a Data Model? - Significance}
    \begin{itemize}
        \item \textbf{Structure and Organization:} 
        Data models help to establish a structured framework for data, ensuring that it is organized logically, which facilitates efficient data retrieval and management.
        
        \item \textbf{Communication Tool:} 
        They serve as an effective communication tool among stakeholders (such as businesses, developers, and database administrators) by providing a visual representation of data relationships and entities.
        
        \item \textbf{Data Integrity and Quality:} 
        By defining the rules and constraints of data (e.g., data types, relationships), data models help maintain data integrity and quality, crucial for decision-making processes.
        
        \item \textbf{Facilitating Development:} 
        A well-defined data model guides developers in implementing and using APIs or queries consistently, ensuring that the system behaves as expected.
    \end{itemize}
\end{frame}

\begin{frame}[fragile]
    \frametitle{What is a Data Model? - Key Components}
    \begin{itemize}
        \item \textbf{Entities:} Objects or concepts represented in the database (e.g., Customer, Product).
        
        \item \textbf{Attributes:} Characteristics of each entity (e.g., Customer Name, Product Price).
        
        \item \textbf{Relationships:} Describes how entities are associated (e.g., a Customer “purchases” a Product).
    \end{itemize}
\end{frame}

\begin{frame}[fragile]
    \frametitle{What is a Data Model? - Example}
    \begin{block}{E-commerce Database}
        \begin{itemize}
            \item \textbf{Entities:}
            \begin{itemize}
                \item Customer (Attributes: customer\_id, name, email)
                \item Product (Attributes: product\_id, name, price)
            \end{itemize}
            
            \item \textbf{Relationship:} 
            A Customer can purchase multiple Products (One-to-Many).
        \end{itemize}
    \end{block}
\end{frame}

\begin{frame}[fragile]
    \frametitle{What is a Data Model? - Conclusion}
    \begin{block}{Conclusion}
        Understanding data models is foundational for database management and development. 
        They not only streamline data flow but also enhance data reliability and integrity. 
        As we delve deeper into this chapter, keep in mind how various types of data models (like Relational, NoSQL, and Graph) will build upon this foundational understanding.
    \end{block}
\end{frame}

\begin{frame}[fragile]{Types of Data Models - Overview}
    \begin{itemize}
        \item Data models provide a structured way to organize, manage, and manipulate data within a database.
        \item Understanding different types of data models is essential for:
        \begin{itemize}
            \item Selecting appropriate models based on application requirements
            \item Ensuring performance
            \item Maintaining scalability
        \end{itemize}
    \end{itemize}
\end{frame}

\begin{frame}[fragile]{Types of Data Models - 1. Relational Data Model}
    \begin{block}{Description}
        \begin{itemize}
            \item Organizes data into tables (relations) with rows and columns.
            \item Each table represents an entity, with columns as attributes.
            \item Relationships are established using foreign keys.
        \end{itemize}
    \end{block}

    \begin{block}{Key Characteristics}
        \begin{itemize}
            \item \textbf{ACID Compliance:} Ensures reliable processing of database transactions.
            \item \textbf{Schema-based:} Data structure needs definition before data insertion.
            \item \textbf{SQL:} Utilizes Structured Query Language for data management.
        \end{itemize}
    \end{block}
\end{frame}

\begin{frame}[fragile]{Types of Data Models - 1. Relational Data Model (cont.)}
    \begin{block}{Example}
        \begin{center}
            \begin{tabular}{|c|c|c|c|}
                \hline
                StudentID & Name   & Age & CourseID \\
                \hline
                1         & Alice  & 20  & C101     \\
                2         & Bob    & 22  & C102     \\
                \hline
            \end{tabular}
        \end{center}
    \end{block}

    \begin{block}{Key Point}
        Relational databases are suited for applications requiring complex transactions and strict data integrity.
    \end{block}
\end{frame}

\begin{frame}[fragile]{Types of Data Models - 2. NoSQL Data Model}
    \begin{block}{Description}
        \begin{itemize}
            \item NoSQL (Not only SQL) databases manage diverse data structures (e.g., document, key-value, graph).
            \item Schema-less, allowing flexibility in data storage.
        \end{itemize}
    \end{block}
    
    \begin{block}{Key Characteristics}
        \begin{itemize}
            \item \textbf{Scalability:} Horizontally scalable to handle data volume.
            \item \textbf{Variety:} Supports unstructured, semi-structured, and structured data.
            \item \textbf{Eventual Consistency:} Prefers availability and partition tolerance over immediate consistency.
        \end{itemize}
    \end{block}
\end{frame}

\begin{frame}[fragile]{Types of Data Models - 2. NoSQL Data Model (cont.)}
    \begin{block}{Example}
        \begin{lstlisting}[language=json]
{
  "StudentID": 1,
  "Name": "Alice",
  "Age": 20,
  "Courses": ["C101", "C102"]
}
        \end{lstlisting}
    \end{block}

    \begin{block}{Key Point}
        NoSQL databases are ideal for big data applications and real-time analytics, where flexible data models are necessary.
    \end{block}
\end{frame}

\begin{frame}[fragile]{Types of Data Models - 3. Graph Data Model}
    \begin{block}{Description}
        \begin{itemize}
            \item Designed to represent and query relationships between data points efficiently.
            \item Comprises nodes (entities) and edges (relationships).
        \end{itemize}
    \end{block}

    \begin{block}{Key Characteristics}
        \begin{itemize}
            \item \textbf{Interconnected Data:} Excellent for handling data with rich relationships.
            \item \textbf{Cypher Query Language:} Typical language for querying graph databases.
        \end{itemize}
    \end{block}
\end{frame}

\begin{frame}[fragile]{Types of Data Models - 3. Graph Data Model (cont.)}
    \begin{block}{Example}
        \begin{itemize}
            \item Nodes: Alice, Bob, Charlie
            \item Edges: (Alice - friends with -> Bob), (Alice - friends with -> Charlie)
        \end{itemize}
    \end{block}

    \begin{block}{Key Point}
        Graph databases are optimal for use cases such as social networks, recommendation systems, and fraud detection.
    \end{block}
\end{frame}

\begin{frame}[fragile]{Types of Data Models - Conclusion}
    \begin{itemize}
        \item Choosing the right data model is crucial according to application needs.
        \item Understanding relational, NoSQL, and graph data models helps in structuring and accessing data effectively.
    \end{itemize}
\end{frame}

\begin{frame}[fragile]{Types of Data Models - Next Steps}
    \begin{itemize}
        \item In the following slides, we will delve deeper into the \textbf{Relational Database Model}.
        \item We will cover its structure and operational mechanics in detail.
    \end{itemize}
\end{frame}

\begin{frame}
    \frametitle{Relational Database Model}
    \begin{block}{Overview}
        Relational databases organize data into tables or "relations" with a fixed schema, ensuring data integrity and consistency.
    \end{block}
\end{frame}

\begin{frame}
    \frametitle{Key Concepts}
    \begin{enumerate}
        \item \textbf{Tables (Relations)}
            \begin{itemize}
                \item Stored in rows and columns.
                \item Each table has a unique identifier called a \textbf{Primary Key}.
                \item Example: Library \textbf{Books Table}.
            \end{itemize}

        \item \textbf{Relationships}
            \begin{itemize}
                \item Linked through \textbf{Foreign Keys}.
                \item Types: One-to-One, One-to-Many, Many-to-Many.
                \item Example: Members Table linked to Books Table through a Checkout Table.
            \end{itemize}
    \end{enumerate}
\end{frame}

\begin{frame}[fragile]
    \frametitle{SQL and Data Operations}
    \begin{block}{Structured Query Language (SQL)}
        SQL is the standard language to interact with relational databases for CRUD operations.
    \end{block}

    \begin{block}{Example SQL Query}
        \begin{lstlisting}[language=SQL]
SELECT Title FROM Books WHERE Author = 'George Orwell';
        \end{lstlisting}
    \end{block}
    
    \begin{block}{Database Operations}
        \begin{itemize}
            \item Data Integrity through primary and foreign keys.
            \item Normalization to reduce redundancy.
            \item Transaction Management via ACID properties.
        \end{itemize}
    \end{block}
\end{frame}

\begin{frame}
    \frametitle{Key Points}
    \begin{itemize}
        \item Widely used for structured data, foundational for many systems.
        \item SQL offers powerful tools for data manipulation.
        \item Relationships and integrity measures enhance robustness.
    \end{itemize}
\end{frame}

\begin{frame}[fragile]
    \frametitle{Key Features of Relational Databases - Introduction}
    \begin{itemize}
        \item Relational databases manage structured data, essential in various fields.
        \item Explore two key features:
        \begin{itemize}
            \item ACID properties
            \item SQL usage
        \end{itemize}
    \end{itemize}
\end{frame}

\begin{frame}[fragile]
    \frametitle{Key Features - ACID Properties}
    \begin{block}{ACID Properties}
        ACID stands for \textbf{Atomicity, Consistency, Isolation, and Durability}, ensuring reliable database transactions.
    \end{block}
    \begin{itemize}
        \item \textbf{Atomicity}: Transaction is a single unit; either all or none of the operations succeed.
        \item \textbf{Consistency}: Transactions transform the database from one valid state to another.
        \item \textbf{Isolation}: Transactions do not interfere with each other, protecting data integrity.
        \item \textbf{Durability}: Committed transactions remain intact even in the event of a system crash.
    \end{itemize}
\end{frame}

\begin{frame}[fragile]
    \frametitle{Key Features - SQL Usage}
    \begin{block}{SQL (Structured Query Language)}
        SQL is the standard language for interacting with relational databases.
    \end{block}
    \begin{itemize}
        \item \textbf{Data Definition Language (DDL)}: Defines database structures.
        \begin{lstlisting}[language=SQL]
CREATE TABLE Employees (
    EmployeeID INT PRIMARY KEY,
    Name VARCHAR(100),
    HireDate DATE
);
        \end{lstlisting}
        
        \item \textbf{Data Manipulation Language (DML)}: Handles data operations.
        \begin{lstlisting}[language=SQL]
INSERT INTO Employees (EmployeeID, Name, HireDate)
VALUES (1, 'John Doe', '2023-01-15');
        \end{lstlisting}
        
        \item \textbf{Data Query Language (DQL)}: Retrieves data.
        \begin{lstlisting}[language=SQL]
SELECT * FROM Employees WHERE HireDate > '2023-01-01';
        \end{lstlisting}
    \end{itemize}
\end{frame}

\begin{frame}[fragile]
    \frametitle{Key Features - Conclusion and Takeaways}
    \begin{itemize}
        \item \textbf{Key Takeaways}:
        \begin{itemize}
            \item ACID properties ensure transaction reliability and data integrity.
            \item SQL is the core language for database interaction, enabling structured data operations.
        \end{itemize}
        \item Understanding these concepts is crucial for effective database management.
    \end{itemize}
\end{frame}

\begin{frame}[fragile]
    \frametitle{Use Cases for Relational Databases - Overview}
    \begin{block}{Overview of Relational Databases}
        Relational databases utilize a structured format to store data, allowing for efficient organization and retrieval through tables. Data is represented in rows and columns, enabling complex relationships through foreign keys and constraints.
    \end{block}
\end{frame}

\begin{frame}[fragile]
    \frametitle{Use Cases for Relational Databases - Key Use Cases}
    \begin{enumerate}
        \item \textbf{Transactional Systems:}
            \begin{itemize}
                \item \textit{Example:} Banking systems, e-commerce platforms.
                \item \textit{Why it excels:} Relational databases uphold ACID properties, ensuring data integrity and reliability in transactions. For instance, immediate balance updates during money transfers.
            \end{itemize}
        
        \item \textbf{Customer Relationship Management (CRM):}
            \begin{itemize}
                \item \textit{Example:} Salesforce, Zoho CRM.
                \item \textit{Why it excels:} Efficiently manages structured data such as customer details and transactions, allowing for insightful analyses of customer relationships.
            \end{itemize}
        
        \item \textbf{Inventory Management:}
            \begin{itemize}
                \item \textit{Example:} Manufacturing and retail businesses.
                \item \textit{Why it excels:} Tracks stock levels and orders in real-time, useful for generating inventory and sales reports.
            \end{itemize}
        
        \item \textbf{Human Resources Management:}
            \begin{itemize}
                \item \textit{Example:} Employee databases, payroll systems.
                \item \textit{Why it excels:} Efficient storage of organizational data, including personal details and performance reviews, ensuring easy accessibility.
            \end{itemize}
    \end{enumerate}
\end{frame}

\begin{frame}[fragile]
    \frametitle{Use Cases for Relational Databases - Advantages}
    \begin{itemize}
        \item \textbf{Structured Query Language (SQL):} A powerful and standardized way to manipulate and retrieve data.
        \item \textbf{Data Integrity:} Constraints and relationships enforce accuracy and validity of data.
        \item \textbf{Flexibility in Data Retrieval:} Complex queries can retrieve related data across multiple tables.
    \end{itemize}
\end{frame}

\begin{frame}[fragile]
    \frametitle{Use Cases for Relational Databases - SQL Query Demonstration}
    To illustrate how relational databases function, consider the following SQL query that retrieves all orders made by a specific customer:
    \begin{lstlisting}[language=SQL]
SELECT Orders.OrderID, Orders.OrderDate, Customers.CustomerName
FROM Orders
INNER JOIN Customers ON Orders.CustomerID = Customers.CustomerID
WHERE Customers.CustomerName = 'John Doe';
    \end{lstlisting}
    This example joins the Orders and Customers tables, filtering for orders belonging to 'John Doe', showcasing the power of relational databases in accessing related data.
\end{frame}

\begin{frame}[fragile]
    \frametitle{Use Cases for Relational Databases - Summary Points}
    \begin{itemize}
        \item Relational databases are optimal for applications requiring structured data, ACID compliance, and complex querying capabilities.
        \item They excel in scenarios where relationships between data points are paramount.
        \item Mastery of SQL is essential for effectively interacting with relational databases.
    \end{itemize}
\end{frame}

\begin{frame}[fragile]
    \frametitle{NoSQL Database Models - Overview}
    \begin{block}{Understanding NoSQL Databases}
        \begin{itemize}
            \item \textbf{Definition}: NoSQL (Not Only SQL) databases are designed to store and manage unstructured or semi-structured data.
            \item \textbf{Key Features}:
                \begin{itemize}
                    \item Flexibility in schema design
                    \item Suitable for scalability and rapid iteration
                \end{itemize}
        \end{itemize}
    \end{block}
\end{frame}

\begin{frame}[fragile]
    \frametitle{NoSQL vs. Relational Databases}
    \begin{block}{Differences}
        \begin{itemize}
            \item \textbf{Schema Flexibility}: NoSQL supports dynamic schema.
            \item \textbf{Data Storage}: NoSQL stores data as key-value, documents, wide-columns, or graphs.
            \item \textbf{Scalability}: NoSQL databases are designed for horizontal scaling across multiple servers.
        \end{itemize}
    \end{block}
\end{frame}

\begin{frame}[fragile]
    \frametitle{Key Characteristics of NoSQL Databases}
    \begin{itemize}
        \item \textbf{Schema-less}: Flexible structure for unstructured data.
        \item \textbf{High Performance}: Optimized for read and write operations in distributed systems.
        \item \textbf{Horizontal Scalability}: Expansion by adding servers is straightforward.
    \end{itemize}
\end{frame}

\begin{frame}[fragile]
    \frametitle{Use Cases and Examples}
    \begin{block}{Use Cases}
        \begin{itemize}
            \item \textbf{Social Media}: Handling diverse user-generated content.
            \item \textbf{Content Management}: Managing various types like videos and articles.
            \item \textbf{Real-Time Analytics}: Analyzing user behaviors for insights.
        \end{itemize}
    \end{block}
    
    \begin{block}{Comparison}
        \begin{itemize}
            \item \textbf{Relational Example (SQL)}:
            \begin{center}
                \begin{tabular}{|c|c|c|c|}
                    \hline
                    BookID & Title & Author & Price \\
                    \hline
                    1 & "1984" & George Orwell & 9.99 \\
                    2 & "To Kill a Mockingbird" & Harper Lee & 7.99 \\
                    \hline
                \end{tabular}
            \end{center}
            \item \textbf{NoSQL Example (JSON)}:
            \begin{lstlisting}
            {
              "BookID": 1,
              "Title": "1984",
              "Author": "George Orwell",
              "Price": 9.99,
              "Genres": ["Dystopian", "Political Fiction"]
            }
            \end{lstlisting}
        \end{itemize}
    \end{block}
\end{frame}

\begin{frame}[fragile]
    \frametitle{Conclusion}
    \begin{block}{Adoption in Modern Applications}
        \begin{itemize}
            \item NoSQL databases are increasingly popular due to their adaptability for evolving data needs.
            \item Common in cloud storage, real-time analytics, and mobile apps.
        \end{itemize}
    \end{block}
    
    \begin{block}{Next Steps}
        \begin{itemize}
            \item Explore the various types of NoSQL databases: document, key-value, column-family, and graph.
        \end{itemize}
    \end{block}
\end{frame}

\begin{frame}[fragile]
    \frametitle{Types of NoSQL Databases - Overview}
    \begin{itemize}
        \item NoSQL databases manage diverse data structures.
        \item Emphasize flexibility, scalability, and performance.
        \item Types we will cover:
        \begin{enumerate}
            \item Document Databases
            \item Key-Value Databases
            \item Column-Family Databases
            \item Graph Databases
        \end{enumerate}
    \end{itemize}
\end{frame}

\begin{frame}[fragile]
    \frametitle{Types of NoSQL Databases - Document Databases}
    \begin{block}{Definition}
        Document databases store data in document format, typically JSON or XML.
    \end{block}

    \begin{itemize}
        \item \textbf{Key Features:}
        \begin{itemize}
            \item Schema flexibility: No strict predefined schemas.
            \item Nesting: Documents can contain nested fields and arrays.
        \end{itemize}
        
        \item \textbf{Example:} MongoDB
        \begin{itemize}
            \item \textbf{Scenario:} A blog application storing posts as JSON documents.
        \end{itemize}
    \end{itemize}

    \begin{lstlisting}[language=json]
    {
        "title": "Introduction to NoSQL",
        "author": "Jane Doe",
        "content": "NoSQL is a database design that... ",
        "tags": ["NoSQL", "Databases", "Tech"]
    }
    \end{lstlisting}
\end{frame}

\begin{frame}[fragile]
    \frametitle{Types of NoSQL Databases - Key-Value Databases}
    \begin{block}{Definition}
        Key-value databases use a simple key-value method where each key maps to a specific value.
    \end{block}

    \begin{itemize}
        \item \textbf{Key Features:}
        \begin{itemize}
            \item Simplicity: Easy to use for basic retrieval.
            \item High performance: Optimized for read/write operations.
        \end{itemize}
        
        \item \textbf{Example:} Redis
        \begin{itemize}
            \item \textbf{Scenario:} Caching system for user sessions.
        \end{itemize}
    \end{itemize}

    \begin{lstlisting}[language=plaintext]
    Key: "session:12345"
    Value: {"userId": "abc123", "cart": ["item1", "item2"]}
    \end{lstlisting}
\end{frame}

\begin{frame}[fragile]
    \frametitle{Types of NoSQL Databases - Column-Family Databases}
    \begin{block}{Definition}
        Column-family databases store data in columns instead of rows for optimized querying.
    \end{block}

    \begin{itemize}
        \item \textbf{Key Features:}
        \begin{itemize}
            \item Efficient storage: Suitable for analytical queries.
            \item Flexible data model: Wide rows with varying columns.
        \end{itemize}
        
        \item \textbf{Example:} Apache Cassandra
        \begin{itemize}
            \item \textbf{Scenario:} Storing time-series sensor data.
        \end{itemize}
    \end{itemize}

    \begin{lstlisting}[language=sql]
    CREATE TABLE sensor_data (
        sensor_id UUID,
        timestamp TIMESTAMP,
        temperature FLOAT,
        humidity FLOAT,
        PRIMARY KEY (sensor_id, timestamp)
    );
    \end{lstlisting}
\end{frame}

\begin{frame}[fragile]
    \frametitle{Types of NoSQL Databases - Graph Databases}
    \begin{block}{Definition}
        Graph databases store data in graph structures with nodes and edges.
    \end{block}

    \begin{itemize}
        \item \textbf{Key Features:}
        \begin{itemize}
            \item Relationship-focused: Efficient for connected data.
            \item Schema flexibility: Models can evolve over time.
        \end{itemize}
        
        \item \textbf{Example:} Neo4j
        \begin{itemize}
            \item \textbf{Scenario:} Social networks representing friendships.
        \end{itemize}
    \end{itemize}

    \begin{lstlisting}[language=cypher]
    CREATE (a:User {name: "Alice"})-[:FRIENDS_WITH]->(b:User {name: "Bob"});
    \end{lstlisting}
\end{frame}

\begin{frame}[fragile]
    \frametitle{Key Points to Emphasize}
    \begin{itemize}
        \item \textbf{Flexibility and Scalability:}
        NoSQL databases address limitations of traditional relational models.
        
        \item \textbf{Use Cases:}
        Each database type excels in specific scenarios; choose based on data and access patterns.
        
        \item \textbf{Performance Benefits:}
        Significant improvements in read and write operations for large datasets.
    \end{itemize}
    
    Understanding these types provides a foundation for exploring their capabilities in modern data processing.
\end{frame}

\begin{frame}[fragile]
    \frametitle{Key Features of NoSQL Databases - Overview}
    \begin{block}{Objective}
        To understand the primary features that differentiate NoSQL databases from traditional relational databases, focusing on scalability, performance, and flexibility.
    \end{block}
\end{frame}

\begin{frame}[fragile]
    \frametitle{Key Features of NoSQL Databases - Scalability}
    \begin{itemize}
        \item \textbf{Definition:} Ability to handle increasing amounts of data and users efficiently.
        \item \textbf{Types of Scalability:}
        \begin{itemize}
            \item \textbf{Horizontal Scaling:} Adding more machines or nodes to distribute the load, essential for NoSQL databases.
            \item \textbf{Vertical Scaling:} Adding more power (CPU, RAM) to an existing machine; less common in NoSQL.
        \end{itemize}
        \item \textbf{Example:} A popular social media platform can scale horizontally to accommodate millions of active users and their posts without downtime.
    \end{itemize}
\end{frame}

\begin{frame}[fragile]
    \frametitle{Key Features of NoSQL Databases - Performance}
    \begin{itemize}
        \item \textbf{Definition:} Speed and efficiency of data read and write operations.
        \item \textbf{Key Points:}
        \begin{itemize}
            \item \textbf{Data Access:} Often faster due to key-value access patterns.
            \item \textbf{Caching Mechanisms:} In-memory storage capabilities to boost read performance.
        \end{itemize}
        \item \textbf{Example:} A real-time analytics platform ingests and processes large volumes of data (e.g., user interactions) with minimal latency.
    \end{itemize}
    
    \begin{block}{Performance Comparison}
        \begin{tabular}{|l|l|l|}
            \hline
            \textbf{Aspect} & \textbf{Traditional SQL Databases} & \textbf{NoSQL Databases} \\
            \hline
            Query Speed & Slower due to complex joins & Fast, especially for simple queries \\
            \hline
            Write Operations & Slower when scaled & Optimized for speed, allowing fast writes \\
            \hline
            Data Volume Handling & Limited by hardware & Efficient with large volumes \\
            \hline
        \end{tabular}
    \end{block}
\end{frame}

\begin{frame}[fragile]
    \frametitle{Key Features of NoSQL Databases - Flexibility}
    \begin{itemize}
        \item \textbf{Definition:} Ability to adapt the database schema and handle unstructured or semi-structured data.
        \item \textbf{Key Features:}
        \begin{itemize}
            \item \textbf{Schema-less Design:} No predefined schema needed for dynamic data modeling.
            \item \textbf{Variety of Data Types:} Can handle JSON, XML, images, etc.
        \end{itemize}
        \item \textbf{Example:} An e-commerce site can add new product attributes (like color or size) without downtime, thanks to flexible schema capabilities.
    \end{itemize}
\end{frame}

\begin{frame}[fragile]
    \frametitle{Summary and Conclusion}
    \begin{itemize}
        \item \textbf{Scalability:} Essential for handling growth; horizontal scaling allows seamless addition of resources.
        \item \textbf{Performance:} Optimized for speed in data access and handling large volumes efficiently.
        \item \textbf{Flexibility:} Lack of a rigid schema supports diverse and evolving data structures, conducive to rapid development.
    \end{itemize}
    
    \begin{block}{Conclusion}
        NoSQL databases provide significant advantages in scalability, performance, and flexibility, making them ideal for modern applications with large-scale, varied data environments.
    \end{block}
\end{frame}

\begin{frame}[fragile]
    \frametitle{Use Cases for NoSQL Databases}
    \begin{block}{Overview}
        NoSQL databases are designed for large unstructured and semi-structured data, enabling high scalability and rapid development.
    \end{block}
\end{frame}

\begin{frame}[fragile]
    \frametitle{Advantages of NoSQL Databases}
    \begin{itemize}
        \item **Scalability:** Can scale horizontally across many servers.
        \item **Flexibility:** Schema-less design allows diverse data structures.
        \item **High Performance:** Optimized for read/write operations, ideal for high transaction rates.
    \end{itemize}
\end{frame}

\begin{frame}[fragile]
    \frametitle{Examples of Use Cases}
    \begin{enumerate}
        \item \textbf{Social Media Platforms}
            \begin{itemize}
                \item Store user profiles, posts, comments, connections.
                \item Schema-less framework supports high traffic and data variety.
            \end{itemize}
        \item \textbf{Real-Time Analytics}
            \begin{itemize}
                \item Analyze streams like clickstreams or sensor data.
                \item Quick ingestion and processing for instant insights.
            \end{itemize}
        \item \textbf{Content Management Systems}
            \begin{itemize}
                \item Manage diverse content types: texts, videos, images.
                \item Document-oriented databases ease storage and retrieval.
            \end{itemize}
        \item \textbf{E-commerce Applications}
            \begin{itemize}
                \item Handle product catalogs, customer data, transaction histories.
                \item Efficiently manage large data volumes under fluctuating loads.
            \end{itemize}
        \item \textbf{Mobile Applications}
            \begin{itemize}
                \item Sync user data across multiple devices.
                \item Support offline access and real-time updates.
            \end{itemize}
    \end{enumerate}
\end{frame}

\begin{frame}[fragile]
    \frametitle{Example JSON Document Structure}
    \begin{lstlisting}[language=json]
{
  "user": {
    "id": "12345",
    "name": "Jane Doe",
    "posts": [
      {
        "post_id": "54321",
        "content": "Hello World!",
        "likes": 15
      }
    ],
    "friends": ["54321", "98765"]
  }
}
    \end{lstlisting}
    \begin{block}{Note}
        This JSON illustrates a common structure in document-based NoSQL databases.
    \end{block}
\end{frame}

\begin{frame}[fragile]
    \frametitle{Graph Database Model - Introduction}
    \begin{block}{Introduction to Graph Databases}
        Graph databases are a type of NoSQL database designed to represent and query data in the form of graphs. 
        This model emphasizes the relationships between data points, enabling powerful, flexible, and efficient data retrieval, especially when dealing with complex data structures.
    \end{block}
\end{frame}

\begin{frame}[fragile]
    \frametitle{Graph Database Model - Key Concepts}
    \begin{itemize}
        \item \textbf{Nodes}: Individual entities or objects in the graph.
            \begin{itemize}
                \item Example: In a social media application, a node may represent a user, a post, or a comment.
            \end{itemize}
        \item \textbf{Edges}: Connections between nodes, representing the relationships.
            \begin{itemize}
                \item Example: An edge might represent a user \textit{likes} a post or \textit{follows} another user.
            \end{itemize}
        \item \textbf{Properties}: Key-value pairs storing additional information for nodes or edges.
            \begin{itemize}
                \item Example: A user node might have properties such as name, age, city; an edge might indicate the date of a follow action.
            \end{itemize}
    \end{itemize}
\end{frame}

\begin{frame}[fragile]
    \frametitle{Graph Database Model - Visual Representation}
    \begin{block}{Simple Representation}
        \begin{lstlisting}
(User1) --follows--> (User2)
(User1) --likes--> (Post1)
        \end{lstlisting}
    \end{block}
    
    In this diagram:
    \begin{itemize}
        \item \textbf{User1} and \textbf{User2} are nodes representing users.
        \item \textbf{Post1} is a node representing a post created by User2.
        \item The arrows signify edges, indicating actions or relationships.
    \end{itemize}
\end{frame}

\begin{frame}[fragile]
    \frametitle{Graph Database Model - Advantages}
    \begin{enumerate}
        \item \textbf{Flexibility in Relationships}: Handle complex relationships, ideal for applications like social networks and recommendation systems.
        \item \textbf{Efficiency}: Quick graph traversals as relationships are inherently linked, contrasting with costly JOIN operations in relational databases.
        \item \textbf{Intuitive Data Modeling}: Reflects real-world connections, simplifying dynamic data interactions.
    \end{enumerate}
\end{frame}

\begin{frame}[fragile]
    \frametitle{Graph Database Model - Key Points}
    \begin{itemize}
        \item Graph databases excel where relationships are as important as the data itself.
        \item They provide a rich and intuitive way to model networks and relationships.
        \item The graph model meets the needs of various applications, including social media, network security, and supply chain management.
    \end{itemize}
\end{frame}

\begin{frame}[fragile]
    \frametitle{Graph Database Model - Conclusion}
    \begin{block}{Conclusion}
        Understanding the structure of graph databases—nodes, edges, and properties—sets the stage for more complex discussions on their features and use cases. 
        Next, we will explore key features of graph databases in detail.
    \end{block}
\end{frame}

\begin{frame}[fragile]
    \frametitle{Introduction to Key Concepts}
    Graph databases focus on the relationships between data points. Their core components include:
    \begin{enumerate}
        \item \textbf{Nodes}: Represent individual entities such as people, places, or events.
        \item \textbf{Edges}: Connect nodes and represent the relationships between them, often carrying properties.
        \item \textbf{Properties}: Attributes related to nodes and edges that provide additional context.
    \end{enumerate}
    
    Understanding these components is crucial for effective data modeling in graph databases.
\end{frame}

\begin{frame}[fragile]
    \frametitle{Nodes and Edges}
    \begin{block}{1. Nodes}
        \begin{itemize}
            \item \textbf{Definition}: Primary units of storage in graph databases, similar to rows in relational databases.
            \item \textbf{Example}: In a social network, each user is a node, e.g., "Alice" (Node A) and "Bob" (Node B).
        \end{itemize}
    \end{block}
    
    \begin{block}{2. Edges}
        \begin{itemize}
            \item \textbf{Definition}: Edges represent how nodes relate to each other, connecting two nodes.
            \item \textbf{Example}: An edge could represent a "Friend" relationship between Alice and Bob.
        \end{itemize}
        \begin{center}
            A---(Friend)--->B
        \end{center}
    \end{block}
\end{frame}

\begin{frame}[fragile]
    \frametitle{Understanding Relationships}
    \begin{block}{3. Relationships}
        \begin{itemize}
            \item \textbf{Importance}: Relationships highlight the interconnectedness of data, affecting retrieval and analysis.
            \item \textbf{Example}: In e-commerce, a product node can connect to a category node (e.g., "Electronics") via an edge labeled "Belongs To."
        \end{itemize}
    \end{block}

    \begin{center}
        \textbf{Visualizing Relationships:}
        \begin{verbatim}
          (Alice) --[Friend]--> (Bob)
               \
                \
              [Likes]
                   \
                    (Pizza)
        \end{verbatim}
    \end{center}
\end{frame}

\begin{frame}[fragile]
    \frametitle{Key Points and Conclusion}
    \begin{itemize}
        \item \textbf{Flexibility}: Graph databases efficiently handle complex data structures compared to traditional databases.
        \item \textbf{Query Power}: Graph queries, using languages like Cypher, can efficiently traverse relationships.
    \end{itemize}
    
    \begin{block}{Example Query}
    \begin{lstlisting}
    MATCH (a:User {name: 'Alice'})-[:Friend]->(f)
    RETURN f.name
    \end{lstlisting}
    \end{block}
    
    \begin{block}{Conclusion}
        Graph databases leverage nodes, edges, and relationships for a rich data model, enhancing insights and flexibility. 
        Understanding these features is essential for exploring practical use cases.
    \end{block}
\end{frame}

\begin{frame}[fragile]
    \frametitle{Use Cases for Graph Databases}
    Graph databases are designed to model and query relationships effectively, using nodes and edges to represent data elements. They are particularly useful in situations where relationships are complex or need to be analyzed deeply.
\end{frame}

\begin{frame}[fragile]
    \frametitle{Ideal Situations for Graph Databases}
    \begin{enumerate}
        \item \textbf{Social Networks}
        \begin{itemize}
            \item Example: Facebook, LinkedIn
            \item Use Case: Stores complex relationships between users.
        \end{itemize}

        \item \textbf{Recommendation Engines}
        \begin{itemize}
            \item Example: Amazon
            \item Use Case: Provides product recommendations based on user behavior.
        \end{itemize}

        \item \textbf{Fraud Detection}
        \begin{itemize}
            \item Example: Financial services
            \item Use Case: Identifies unusual transaction patterns between accounts.
        \end{itemize}
    \end{enumerate}
\end{frame}

\begin{frame}[fragile]
    \frametitle{More Use Cases}
    \begin{enumerate}[\setcounter{enumi}{3}]
        \item \textbf{Network and IT Operations}
        \begin{itemize}
            \item Example: Monitoring networked devices
            \item Use Case: Analyzes performance and dependencies of network connections.
        \end{itemize}

        \item \textbf{Knowledge Graphs}
        \begin{itemize}
            \item Example: Wikipedia's data structures
            \item Use Case: Organizes complex relationships for content discovery.
        \end{itemize}

        \item \textbf{Master Data Management}
        \begin{itemize}
            \item Example: Businesses with multiple data sources
            \item Use Case: Matches and merges disparate datasets for consistency.
        \end{itemize}
    \end{enumerate}
\end{frame}

\begin{frame}[fragile]
    \frametitle{Key Points and Summary}
    \begin{block}{Key Points}
        \begin{itemize}
            \item \textbf{Flexibility:} Graph databases adapt well to evolving relationships.
            \item \textbf{Performance:} Superior for deep traversal queries.
            \item \textbf{Real-World Modeling:} Natural fit for interconnected systems.
        \end{itemize}
    \end{block}
    
    \textbf{Summary:} Graph databases excel when relationships are key, enabling organizations to gain deeper insights and enhance decision-making.
\end{frame}

\begin{frame}[fragile]
    \frametitle{Comparative Analysis of Data Models}
    \begin{block}{Overview of Data Models}
        Data models are structural representations of data and how they interact. 
        Understanding different data models is critical for selecting the right one for specific applications. This presentation compares three predominant data models: 
        \textbf{Relational Databases}, \textbf{NoSQL Databases}, and \textbf{Graph Databases}.
    \end{block}
\end{frame}

\begin{frame}[fragile]
    \frametitle{Comparison Table of Data Models}
    \begin{table}[ht]
        \centering
        \begin{tabular}{|l|l|l|l|}
            \hline
            \textbf{Feature} & \textbf{Relational Databases} & \textbf{NoSQL Databases} & \textbf{Graph Databases} \\ \hline
            Data Structure & Tables with rows and columns & Document-oriented, key-value pairs, wide-column, or graph structures & Nodes (entities) and edges (relationships) \\ \hline
            Schema & Fixed schema & Flexible schema, often schemaless & Schema-less; relationships are integral to the structure \\ \hline
            Query Language & Structured Query Language (SQL) & Varies (e.g., MongoDB uses a query language) & Graph Query Languages like Cypher for Neo4j \\ \hline
            Transactions & ACID compliance & Typically BASE & Varies, but often focus on eventual consistency \\ \hline
            Scalability & Vertical scaling & Horizontal scaling & Horizontal scaling through distributed databases \\ \hline
            Use Cases & Traditional business applications, analytics & Big data applications, real-time data & Social networks, recommendation engines \\ \hline
            Performance & Slower with large, complex queries & Optimized for read/write performance & Optimized for traversing relationships \\ \hline
        \end{tabular}
    \end{table}
\end{frame}

\begin{frame}[fragile]
    \frametitle{Key Points and Examples}
    \begin{itemize}
        \item \textbf{Data Structure:} Helps in determining the best fit based on data complexity.
        \item \textbf{Schema Flexibility:} NoSQL's schema allows for rapid development; suitable for agile environments.
        \item \textbf{Query Methods:} Important to understand as they dictate data interaction efficacy.
        \item \textbf{Scalability:} Consider scaling needs; relational databases may struggle compared to NoSQL or graph databases.
        \item \textbf{Performance Considerations:} Performance varies significantly based on the data model and query types.
    \end{itemize}
    
    \begin{block}{Illustrative Examples}
        \begin{itemize}
            \item E-commerce platform: Uses relational database for transactions and inventory.
            \item Social media platform: Utilizes NoSQL to manage user profiles and posts.
            \item Recommendation engine: Employs graph database for analyzing user behavior and relationships.
        \end{itemize}
    \end{block}
\end{frame}

\begin{frame}[fragile]
    \frametitle{Conclusion}
    Selecting the right data model depends on your current needs and future growth. 
    Understanding the strengths and weaknesses of relational, NoSQL, and graph databases leads to better-informed architectural decisions. 
    By comparing these data models, you will be better prepared to evaluate which system aligns best with your application requirements.
\end{frame}

\begin{frame}[fragile]{Choosing the Right Data Model}
    \begin{block}{Overview}
        Selecting the appropriate data model is crucial for the success of your application. The choice influences performance, scalability, and efficiency. This guide provides practical guidelines to make an informed decision based on application requirements.
    \end{block}
\end{frame}

\begin{frame}[fragile]{Key Considerations When Choosing a Data Model - Part 1}
    \begin{enumerate}
        \item \textbf{Data Structure and Type of Relationships}
            \begin{itemize}
                \item \textbf{Relational Models}: Best for structured data with defined relationships (e.g., tables with primary and foreign keys).
                \item \textbf{NoSQL Models}: Ideal for semi-structured or unstructured data; flexible and accommodates varied data types.
                \item \textbf{Graph Databases}: Great for complex relationship mapping, like social networks, where entities are interconnected.
            \end{itemize}
        \item \textbf{Querying Requirements}
            \begin{itemize}
                \item Complex queries? Use relational databases. 
                \item For simple key-value retrieval, use document-based NoSQL.
            \end{itemize}
    \end{enumerate}
\end{frame}

\begin{frame}[fragile]{Key Considerations When Choosing a Data Model - Part 2}
    \begin{enumerate}
        \setcounter{enumi}{2}
        \item \textbf{Scalability Needs}
            \begin{itemize}
                \item \textbf{Vertical Scaling}: Suitable for relational databases with structured data.
                \item \textbf{Horizontal Scaling}: Preferred for NoSQL databases handling large volumes across distributed systems.
            \end{itemize}
        \item \textbf{Transaction Management}
            \begin{itemize}
                \item For strict data consistency (ACID properties), consider relational databases.
                \item For high-throughput and eventual consistency, NoSQL systems are more appropriate.
            \end{itemize}
        \item \textbf{Development Speed \& Flexibility}
            \begin{itemize}
                \item Rapid development favors NoSQL models due to their flexible schema.
                \item Relational databases require well-defined schema upfront, which can slow down development.
            \end{itemize}
    \end{enumerate}
\end{frame}

\begin{frame}[fragile]{Decision Framework}
    \begin{table}[ht]
        \centering
        \begin{tabular}{|l|c|c|c|}
            \hline
            \textbf{Criteria} & \textbf{Relational Databases} & \textbf{NoSQL Databases} & \textbf{Graph Databases} \\ \hline
            Data Structure & Structured & Semi-structured/Unstructured & Highly connected \\ \hline
            Query Complexity & High & Low to Moderate & High \\ \hline
            Scalability & Vertical & Horizontal & Horizontal \\ \hline
            Transaction Support & Strong (ACID) & Eventual consistency & Strong Relationships \\ \hline
            Flexibility & Low & High & Moderate \\ \hline
        \end{tabular}
    \end{table}
\end{frame}

\begin{frame}[fragile]{Conclusion and Key Points to Remember}
    \begin{block}{Conclusion}
        The right choice of data model should align with your application needs. By assessing criteria such as data structure, querying complexity, scalability, and consistency requirements, you can optimize performance and ensure long-term success.
    \end{block}
    
    \begin{itemize}
        \item Identify data structure and relationship complexity.
        \item Consider the type and frequency of queries executed.
        \item Evaluate scalability needs based on user base and load projections.
        \item Choose based on transaction requirements and development speed.
    \end{itemize}
\end{frame}

\begin{frame}[fragile]
    \frametitle{Summary and Key Takeaways - Part 1}
    \begin{block}{Overview}
        This chapter introduced the fundamental concepts of data models, highlighting their significance in database design and management. Understanding data models is crucial for aligning data structures with business requirements to enhance data integrity, accessibility, and analysis capabilities.
    \end{block}
    
    \begin{block}{Key Concepts Explored}
        \begin{enumerate}
            \item Definition of Data Models
            \item Types of Data Models
            \item Choosing the Right Data Model
        \end{enumerate}
    \end{block}
\end{frame}

\begin{frame}[fragile]
    \frametitle{Summary and Key Takeaways - Part 2}
    \begin{block}{Key Concepts Explored (cont.)}
        \begin{itemize}
            \item \textbf{Normalization and Denormalization:}
            \begin{itemize}
                \item Normalization to reduce redundancy.
                \item Denormalization for performance optimization.
            \end{itemize}
            \item \textbf{Real-world Applications:}
            \begin{itemize}
                \item E-commerce uses relational databases.
                \item Social networks utilize graph databases.
            \end{itemize}
        \end{itemize}
    \end{block}
\end{frame}

\begin{frame}[fragile]
    \frametitle{Summary and Key Takeaways - Part 3}
    \begin{block}{Future Topics}
        \begin{itemize}
            \item Advanced Data Modeling Techniques:
            \begin{itemize}
                \item Entity-relationship modeling, data warehousing, data lakes.
            \end{itemize}
            \item Data Model Implementation:
            \begin{itemize}
                \item Hands-on exercises with specific DBMS.
            \end{itemize}
            \item Impact of Emerging Technologies:
            \begin{itemize}
                \item Influence of machine learning and big data technologies on data modeling strategies.
            \end{itemize}
        \end{itemize}
    \end{block}
    
    \begin{block}{Key Points to Remember}
        \begin{itemize}
            \item A well-structured data model is crucial for efficient data management.
            \item The choice of data model impacts data integrity, performance, and scalability.
            \item Understanding data types and use cases is key to informed data modeling decisions.
        \end{itemize}
    \end{block}
\end{frame}


\end{document}