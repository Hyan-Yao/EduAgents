\documentclass[aspectratio=169]{beamer}

% Theme and Color Setup
\usetheme{Madrid}
\usecolortheme{whale}
\useinnertheme{rectangles}
\useoutertheme{miniframes}

% Additional Packages
\usepackage[utf8]{inputenc}
\usepackage[T1]{fontenc}
\usepackage{graphicx}
\usepackage{booktabs}
\usepackage{listings}
\usepackage{amsmath}
\usepackage{amssymb}
\usepackage{xcolor}
\usepackage{tikz}
\usepackage{pgfplots}
\pgfplotsset{compat=1.18}
\usetikzlibrary{positioning}
\usepackage{hyperref}

% Custom Colors
\definecolor{myblue}{RGB}{31, 73, 125}
\definecolor{mygray}{RGB}{100, 100, 100}
\definecolor{mygreen}{RGB}{0, 128, 0}
\definecolor{myorange}{RGB}{230, 126, 34}
\definecolor{mycodebackground}{RGB}{245, 245, 245}

% Set Theme Colors
\setbeamercolor{structure}{fg=myblue}
\setbeamercolor{frametitle}{fg=white, bg=myblue}
\setbeamercolor{title}{fg=myblue}
\setbeamercolor{section in toc}{fg=myblue}
\setbeamercolor{item projected}{fg=white, bg=myblue}
\setbeamercolor{block title}{bg=myblue!20, fg=myblue}
\setbeamercolor{block body}{bg=myblue!10}
\setbeamercolor{alerted text}{fg=myorange}

% Set Fonts
\setbeamerfont{title}{size=\Large, series=\bfseries}
\setbeamerfont{frametitle}{size=\large, series=\bfseries}
\setbeamerfont{caption}{size=\small}
\setbeamerfont{footnote}{size=\tiny}

% Document Start
\begin{document}

\frame{\titlepage}

\begin{frame}[fragile]
    \frametitle{Introduction to Ethical Data Practices}
    
    \begin{block}{Overview}
        Ethical data practices ensure that data in AI applications is handled with integrity, respect, and transparency.
    \end{block}
    
    \begin{block}{Importance of Ethics in Data Usage}
        Ethical considerations are critical due to the potential risks and responsibilities associated with data handling in AI contexts.
    \end{block}
\end{frame}

\begin{frame}[fragile]
    \frametitle{Key Aspects of Ethical Data Practices}
    
    \begin{enumerate}
        \item \textbf{Trust and Transparency}
            \begin{itemize}
                \item Fosters trust among users and stakeholders.
                \item Example: Health apps with clear data usage policies.
            \end{itemize}
        
        \item \textbf{Equity and Fairness}
            \begin{itemize}
                \item Prevents perpetuation of biases in AI systems.
                \item Illustration: Biased hiring algorithms disadvantaging underrepresented groups.
            \end{itemize}

        \item \textbf{Accountability}
            \begin{itemize}
                \item Organizations must be accountable for their data practices.
                \item Example: Google and Microsoft's ethical AI principles.
            \end{itemize}

        \item \textbf{Legal Compliance}
            \begin{itemize}
                \item Aligns with legal requirements like GDPR.
                \item Key Point: Non-compliance can lead to fines and reputational damage.
            \end{itemize}
    \end{enumerate}
\end{frame}

\begin{frame}[fragile]
    \frametitle{Engaging Questions and Conclusion}
    
    \begin{block}{Engaging Questions}
        \begin{itemize}
            \item How do we determine which ethical guidelines should govern our data use?
            \item What role does user consent play in ethical data usage?
            \item Can the pursuit of innovation in AI compromise ethical standards?
        \end{itemize}
    \end{block}

    \begin{block}{Conclusion}
        Adopting ethical practices is a foundational principle that enhances trust and promotes fairness in AI applications.
    \end{block}
\end{frame}

\begin{frame}[fragile]
    \frametitle{Types of Data - Introduction}
    \begin{block}{Understanding Data Types in AI}
        Data plays a crucial role in the functionality of Artificial Intelligence (AI). Two main categories can be identified when discussing data:
        \begin{itemize}
            \item \textbf{Structured Data}
            \item \textbf{Unstructured Data}
        \end{itemize}
        Understanding these types and their ethical implications is essential for responsible data usage.
    \end{block}
\end{frame}

\begin{frame}[fragile]
    \frametitle{Types of Data - Structured Data}
    \begin{block}{1. Structured Data}
        \textbf{Definition:} Structured data refers to highly organized information that is easily searchable in relational databases. It is typically formatted in tables with rows and columns.
        
        \textbf{Examples:}
        \begin{itemize}
            \item Customer information in a database (names, addresses, phone numbers)
            \item Financial transactions in spreadsheets
            \item Sensor data from Internet of Things (IoT) devices
        \end{itemize}
        
        \textbf{Ethical Implications:}
        \begin{itemize}
            \item \textbf{Privacy Concerns:} Handling personally identifiable information (PII) necessitates robust data protection measures.
            \item \textbf{Data Misuse:} Algorithms must ensure data is used for intended purposes only.
        \end{itemize}
    \end{block}
\end{frame}

\begin{frame}[fragile]
    \frametitle{Types of Data - Unstructured Data}
    \begin{block}{2. Unstructured Data}
        \textbf{Definition:} Unstructured data is information that does not have a predefined data model and exists in various formats.
        
        \textbf{Examples:}
        \begin{itemize}
            \item Social media posts and comments
            \item Emails and text messages
            \item Videos, images, and audio files
        \end{itemize}
        
        \textbf{Ethical Implications:}
        \begin{itemize}
            \item \textbf{Bias and Misrepresentation:} Unstructured data can reflect societal biases.
            \item \textbf{Data Ownership:} Clarity around data ownership and usage rights is crucial.
        \end{itemize}
    \end{block}
\end{frame}

\begin{frame}[fragile]
    \frametitle{Key Points and Discussion}
    \begin{block}{Key Points to Emphasize}
        \begin{itemize}
            \item \textbf{Data Type Impact:} The type of data used directly influences AI behavior and decision-making.
            \item \textbf{Ethical Responsibility:} Ethical data practices necessitate a focus on privacy, consent, and fairness.
            \item \textbf{Real-World Applications:} Diverse data sets can lead to innovative and trustworthy AI systems.
        \end{itemize}
    \end{block}
    
    \begin{block}{Discussion Questions}
        \begin{itemize}
            \item How can organizations ensure the ethical use of structured data?
            \item In what ways can biases be identified and corrected in unstructured data?
        \end{itemize}
    \end{block}
\end{frame}

\begin{frame}[fragile]
    \frametitle{Bias in Data}
    \begin{block}{Understanding Bias in Data}
        \textbf{Definition:} Bias in data refers to systematic errors that can lead to incorrect conclusions or unfair outcomes in machine learning models.
    \end{block}
    \begin{itemize}
        \item It can originate from data collection processes, feature selection, and representation of certain groups.
    \end{itemize}
\end{frame}

\begin{frame}[fragile]
    \frametitle{Sources of Bias}
    \begin{enumerate}
        \item \textbf{Sampling Bias:} Data collected is not representative of the population.
        \item \textbf{Measurement Bias:} Errors in data recording or collection (e.g. leading questions in surveys).
        \item \textbf{Label Bias:} Subjective ground truth labels in supervised learning (e.g. biases in sentiment analysis).
    \end{enumerate}
\end{frame}

\begin{frame}[fragile]
    \frametitle{Impact on AI Outcomes}
    \begin{itemize}
        \item \textbf{Decision-Making:} Biased training data leads to biased outputs in critical areas (e.g. hiring, lending).
        \item \textbf{User Trust:} Biased AI outputs can erode user trust and engagement.
        \item \textbf{Real-World Consequences:} Reinforces inequalities and propagates stereotypes.
    \end{itemize}
\end{frame}

\begin{frame}[fragile]
    \frametitle{Case Studies}
    \begin{block}{COMPAS Recidivism Algorithm}
        \begin{itemize}
            \item Exhibited racial bias, disproportionately flagging African American defendants as high risk.
            \item \textbf{Lesson Learned:} Data reflected systemic biases in arrest and sentencing practices.
        \end{itemize}
    \end{block}
\end{frame}

\begin{frame}[fragile]
    \frametitle{Case Studies (cont.)}
    \begin{block}{Amazon's Hiring Tool}
        \begin{itemize}
            \item Penalized resumes with the word "women," leading to bias against female candidates.
            \item \textbf{Lesson Learned:} Historical hiring data led to gender biases in the tech industry.
        \end{itemize}
    \end{block}
\end{frame}

\begin{frame}[fragile]
    \frametitle{Questions to Ponder}
    \begin{itemize}
        \item How can we ensure diversity in data collection to mitigate bias?
        \item What practices can be adopted to assess bias in AI systems?
        \item How might ethical considerations shape the future of data-driven decision-making?
    \end{itemize}
\end{frame}

\begin{frame}[fragile]
    \frametitle{Conclusion}
    \begin{block}{Summary}
        Understanding and addressing bias in data is crucial for the development of fair and trustworthy AI systems.
    \end{block}
    \begin{itemize}
        \item Diverse and representative training datasets can help mitigate bias.
        \item Transparent evaluation processes are essential.
    \end{itemize}
\end{frame}

\begin{frame}[fragile]
    \frametitle{Engage With This Topic}
    \begin{itemize}
        \item Reflect on examples in your life where you’ve noticed bias in decision-making driven by data.
        \item Consider how you can contribute to the creation of unbiased AI systems in your future career.
    \end{itemize}
\end{frame}

\begin{frame}[fragile]
    \frametitle{Next Slide}
    \begin{block}{Explore Privacy Considerations}
        In the context of ethical data usage and the implications of bias.
    \end{block}
\end{frame}

\begin{frame}[fragile]
    \frametitle{Privacy Considerations - Overview}
    \begin{block}{Understanding the Significance of Data Privacy}
        Data privacy refers to the proper handling, processing, storage, and usage of personal information to protect individuals' rights.
    \end{block}
\end{frame}

\begin{frame}[fragile]
    \frametitle{Why is Data Privacy Important?}
    \begin{enumerate}
        \item \textbf{Protection of Personal Information:} Individuals have the right to control their data, preserving dignity and autonomy.
        \item \textbf{Trust Building:} Organizations that prioritize data privacy build trust, essential for customer retention.
        \item \textbf{Regulatory Compliance:} Laws exist globally to protect data privacy, making compliance a legal obligation.
        \item \textbf{Risk Mitigation:} Inadequate data protection can lead to reputational damage and financial loss.
    \end{enumerate}
\end{frame}

\begin{frame}[fragile]
    \frametitle{Key Regulations Governing Data Privacy}
    \begin{enumerate}
        \item \textbf{General Data Protection Regulation (GDPR):}
            \begin{itemize}
                \item Requires consent before data collection.
                \item Grants rights to access, rectify, and erase data.
                \item Non-compliance can incur fines up to €20 million or 4\% of global turnover.
            \end{itemize}
        \item \textbf{California Consumer Privacy Act (CCPA):}
            \begin{itemize}
                \item Enhances privacy rights for California residents.
                \item Mandates disclosure of data collection practices.
                \item Allows consumers to opt-out of data selling.
            \end{itemize}
        \item \textbf{Health Insurance Portability and Accountability Act (HIPAA):}
            \begin{itemize}
                \item Establishes standards for the protection of health information.
                \item Ensures confidentiality and imposes penalties for breaches.
            \end{itemize}
    \end{enumerate}
\end{frame}

\begin{frame}[fragile]
    \frametitle{Ethical Frameworks in AI - Overview}
    Ethical frameworks provide structured approaches to addressing moral issues in various contexts, particularly in data usage in AI. They ensure that technology is developed and deployed with respect for human rights, fairness, and accountability.
\end{frame}

\begin{frame}[fragile]
    \frametitle{Ethical Frameworks in AI - Key Ethical Frameworks}
    \begin{enumerate}
        \item \textbf{Utilitarianism}
            \begin{itemize}
                \item \textit{Concept}: Maximizes overall happiness or utility.
                \item \textit{Example}: Transportation app optimizing routes for majority user satisfaction.
            \end{itemize}
        \item \textbf{Deontological Ethics}
            \begin{itemize}
                \item \textit{Concept}: Focuses on rules and duties over outcomes.
                \item \textit{Example}: Adhering strictly to data protection laws despite short-term gains.
            \end{itemize}
        \item \textbf{Virtue Ethics}
            \begin{itemize}
                \item \textit{Concept}: Emphasizes character and virtues of decision-makers.
                \item \textit{Example}: Data scientist mitigating biases in algorithms to ensure fairness.
            \end{itemize}
    \end{enumerate}
\end{frame}

\begin{frame}[fragile]
    \frametitle{Ethical Frameworks in AI - Established Guidelines}
    \begin{block}{The European General Data Protection Regulation (GDPR)}
        \begin{itemize}
            \item Ensures rights for data subjects over personal data.
            \item Mandates transparency, consent, and data minimization.
        \end{itemize}
    \end{block}
    
    \begin{block}{AI Ethics Guidelines by OECD}
        \begin{itemize}
            \item Promotes inclusive growth and human-centered values.
            \item Encourages AI use that respects democratic principles and rights.
        \end{itemize}
    \end{block}
    
    \begin{block}{IEEE Ethical Guidelines}
        \begin{itemize}
            \item Advocates accountability and transparency.
            \item Emphasizes considering societal impacts of AI technologies.
        \end{itemize}
    \end{block}
\end{frame}

\begin{frame}[fragile]
    \frametitle{Case Studies}
    \textbf{Understanding Ethical Issues in AI Data Usage}

    In this section, we will analyze specific case studies that emphasize the ethical intricacies of data usage within AI applications. Understanding these scenarios will help reinforce the importance of implementing ethical practices in data science and artificial intelligence. 

    Let's explore two significant cases:
\end{frame}

\begin{frame}[fragile]
    \frametitle{Case Study 1: Cambridge Analytica Scandal}
    
    \textbf{Overview:}
    In 2016, Cambridge Analytica harvested personal data from millions of Facebook users without their explicit consent. This information was used for targeted advertisements in political campaigns, raising significant ethical concerns about privacy, consent, and data manipulation.
    
    \textbf{Key Issues:}
    \begin{itemize}
        \item \textbf{Informed Consent:} Users were unaware their data was being collected and used for political advertising.
        \item \textbf{Manipulation and Influence:} Data was used to influence voting behavior, questioning the integrity of democratic processes.
        \item \textbf{Data Privacy Violations:} Personal data was exploited in ways that users did not foresee or agree to.
    \end{itemize}

    \textbf{Takeaway:} 
    Ethical data practices must prioritize user consent and transparency to prevent misuse of personal information.
\end{frame}

\begin{frame}[fragile]
    \frametitle{Case Study 2: Amazon's Facial Recognition Technology}
    
    \textbf{Overview:}
    Amazon's Rekognition software can identify and track people in real-time, adopted by law enforcement agencies. However, bias, accuracy, and privacy concerns have emerged.

    \textbf{Key Issues:}
    \begin{itemize}
        \item \textbf{Bias in AI Models:} Facial recognition technologies often misidentify people of color at higher rates compared to white individuals.
        \item \textbf{Privacy Concerns:} Potential for mass surveillance raises ethical questions about individual privacy and civil liberties.
        \item \textbf{Lack of Regulation:} The absence of regulations governing the use of such technology can lead to abuse.
    \end{itemize}

    \textbf{Takeaway:}
    AI applications must be developed with a focus on fairness, accountability, and transparency, ensuring they do not perpetuate existing biases or infringe on privacy rights.
\end{frame}

\begin{frame}[fragile]
    \frametitle{Key Points to Remember}
    
    \begin{itemize}
        \item Ethical considerations in AI and data usage are crucial to maintaining trust and safeguarding individual rights.
        \item Transparency, fairness, and informed consent are fundamental principles that must guide data practices.
        \item Real-world examples like Cambridge Analytica and Amazon's Rekognition illustrate the consequences of neglecting these principles.
    \end{itemize}
    
    \textbf{Conclusion:} 
    These case studies serve as reminders of the urgent need for ethical frameworks in AI applications. 

    By studying these examples, consider: \textit{How might we apply these lessons to current and future AI projects?}
\end{frame}

\begin{frame}[fragile]
    \frametitle{Implementing Ethical Practices - Introduction}
    \begin{itemize}
        \item Ethical data practices in machine learning (ML) are essential for:
        \begin{itemize}
            \item Responsible, fair, and transparent data use
            \item Protecting individuals' rights
            \item Enhancing trustworthiness and acceptability of AI systems
        \end{itemize}
        \item Importance: Not just regulatory compliance but also building integrity into AI innovations.
    \end{itemize}
\end{frame}

\begin{frame}[fragile]
    \frametitle{Implementing Ethical Practices - Strategies}
    \begin{enumerate}
        \item \textbf{Data Governance Framework}
        \begin{itemize}
            \item Establish guidelines for data collection, storage, and usage
            \item Assign responsibilities for compliance
        \end{itemize}

        \item \textbf{Informed Consent}
        \begin{itemize}
            \item Transparent communication about data collection and usage
            \item Opt-in and opt-out mechanisms for users
        \end{itemize}
    \end{enumerate}
\end{frame}

\begin{frame}[fragile]
    \frametitle{Implementing Ethical Practices - More Strategies}
    \begin{enumerate}[resume]
        \item \textbf{Bias Detection and Mitigation}
        \begin{itemize}
            \item Regular audits to evaluate algorithm biases
            \item Diverse data collection to minimize bias
        \end{itemize}

        \item \textbf{Transparency in Algorithms}
        \begin{itemize}
            \item Document decision-making processes
            \item User-facing explanations of algorithm outputs
        \end{itemize}

        \item \textbf{Regular Ethical Training}
        \begin{itemize}
            \item Workshops and training on ethical data usage
            \item Discussions on past ethical breaches for lessons learned
        \end{itemize}
    \end{enumerate}
\end{frame}

\begin{frame}[fragile]
    \frametitle{Implementing Ethical Practices - Key Points and Conclusion}
    \begin{itemize}
        \item Key points to emphasize:
        \begin{itemize}
            \item Ethical data practices foster trustworthy AI systems
            \item Regular audits and transparency enhance public trust
            \item Stakeholder engagement ensures alignment with societal values
        \end{itemize}
        \item Conclusion: 
        \begin{itemize}
            \item Ethical practices not only ensure regulatory compliance but also promote integrity and positive social impact.
        \end{itemize}
    \end{itemize}
\end{frame}

\begin{frame}[fragile]
    \frametitle{Future Directions in Ethical Practices for Data Usage}
    \begin{block}{Overview}
        Speculating on future trends in ethical practices for data usage in AI, this section emphasizes ongoing research and development needs driven by technology advancement, regulatory pressures, and demand for accountability.
    \end{block}
\end{frame}

\begin{frame}[fragile]
    \frametitle{Key Future Trends}
    \begin{enumerate}
        \item \textbf{Increased Regulatory Frameworks}
            \begin{itemize}
                \item More comprehensive legislation is expected to protect personal data.
                \item Examples like GDPR set a precedent for global regulations.
            \end{itemize}
        \item \textbf{Ethics-by-Design Approach}
            \begin{itemize}
                \item Ethical considerations will be integrated at every stage of AI development.
                \item Focus on data management from collection to implementation.
            \end{itemize}
        \item \textbf{Advancements in Explainable AI (XAI)}
            \begin{itemize}
                \item Push for AI models that clarify their decision processes.
                \item Enhances transparency and addresses biases effectively.
            \end{itemize}
    \end{enumerate}
\end{frame}

\begin{frame}[fragile]
    \frametitle{Collaboration and Ethical Marketplaces}
    \begin{enumerate}
        \setcounter{enumi}{3}
        \item \textbf{Collaborative Governance}
            \begin{itemize}
                \item Multi-stakeholder collaboration among tech companies, governments, and civil society.
                \item Establishing equitable AI systems through diverse perspectives.
            \end{itemize}
        \item \textbf{Ethical Data Marketplaces}
            \begin{itemize}
                \item Consumers could share data securely and receive compensation.
                \item Enhances control over personal information and creates mutual benefits.
            \end{itemize}
    \end{enumerate}
\end{frame}

\begin{frame}[fragile]
    \frametitle{Conclusion and Discussion}
    \begin{block}{Conclusion}
        Future trends will highlight regulatory compliance, ethical design, transparency, collaborative governance, and innovative data-sharing models. Continuous research is crucial for a responsible AI ecosystem.
    \end{block}
    \begin{block}{Questions for Discussion}
        \begin{itemize}
            \item How can businesses prepare for upcoming regulations?
            \item What role do individuals play in ensuring ethical data practices? 
            \item How do we balance innovation with ethical responsibilities in AI?
        \end{itemize}
    \end{block}
\end{frame}

\begin{frame}[fragile]
    \frametitle{Conclusion and Q\&A}
    \begin{block}{Key Takeaways from Chapter 10: Ethical Practices in Data Usage}
        \begin{enumerate}
            \item Understanding Data Ethics
            \item Importance of Transparency
            \item Fairness and Non-Discrimination
            \item Data Privacy and Protection
            \item Accountability
        \end{enumerate}
    \end{block}
\end{frame}

\begin{frame}[fragile]
    \frametitle{Understanding Data Ethics}
    \begin{itemize}
        \item Data ethics refers to moral principles guiding data collection, processing, and usage.
        \item Ethical practices ensure individuals' rights are protected, preventing harm or discrimination.
        \item {\bf Example:} Data collection for AI models requires user consent and understanding of data usage.
    \end{itemize}
\end{frame}

\begin{frame}[fragile]
    \frametitle{Importance of Transparency}
    \begin{itemize}
        \item Transparency builds trust between organizations and individuals.
        \item Organizations should be open about data sources, methodologies, and processes.
        \item {\bf Example:} A clear communication of data policies by a social media platform showcases ethical transparency.
    \end{itemize}
\end{frame}

\begin{frame}[fragile]
    \frametitle{Fairness, Privacy, and Accountability}
    \begin{itemize}
        \item Fairness and Non-Discrimination:
            \begin{itemize}
                \item AI must be designed to prevent bias and favoring of specific groups.
                \item {\bf Example:} AI tools must regularly audit to avoid unintended discrimination in recruitment.
            \end{itemize}
        \item Data Privacy and Protection:
            \begin{itemize}
                \item Protecting privacy is crucial; organizations must comply with regulations like GDPR.
                \item {\bf Example:} Using two-factor authentication and encryption for transactions shows strong data protection.
            \end{itemize}
        \item Accountability:
            \begin{itemize}
                \item Ensuring stakeholders are responsible for their data practices is essential for ethical usage.
                \item {\bf Example:} Appointing a Chief Data Ethics Officer ensures organizational accountability.
            \end{itemize}
    \end{itemize}
\end{frame}

\begin{frame}[fragile]
    \frametitle{Engaging Questions for Discussion}
    \begin{itemize}
        \item What challenges do organizations face in implementing ethical data practices?
        \item Can you think of situations where lack of ethical practices in data led to negative outcomes?
        \item How can emerging technologies improve transparency and accountability in data usage?
        \item In what ways can individuals advocate for better data ethics in organizations they interact with?
    \end{itemize}
\end{frame}

\begin{frame}[fragile]{Thank You}
    \begin{center}
        {\Large Thank You}\\[1.em]
        {\large Questions and Discussion}
    \end{center}
\end{frame}


\end{document}