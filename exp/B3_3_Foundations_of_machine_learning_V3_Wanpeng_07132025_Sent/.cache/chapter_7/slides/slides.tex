\documentclass[aspectratio=169]{beamer}

% Theme and Color Setup
\usetheme{Madrid}
\usecolortheme{whale}
\useinnertheme{rectangles}
\useoutertheme{miniframes}

% Additional Packages
\usepackage[utf8]{inputenc}
\usepackage[T1]{fontenc}
\usepackage{graphicx}
\usepackage{booktabs}
\usepackage{listings}
\usepackage{amsmath}
\usepackage{amssymb}
\usepackage{xcolor}
\usepackage{tikz}
\usepackage{pgfplots}
\pgfplotsset{compat=1.18}
\usetikzlibrary{positioning}
\usepackage{hyperref}

% Custom Colors
\definecolor{myblue}{RGB}{31, 73, 125}
\definecolor{mygray}{RGB}{100, 100, 100}
\definecolor{mygreen}{RGB}{0, 128, 0}
\definecolor{myorange}{RGB}{230, 126, 34}
\definecolor{mycodebackground}{RGB}{245, 245, 245}

% Set Theme Colors
\setbeamercolor{structure}{fg=myblue}
\setbeamercolor{frametitle}{fg=white, bg=myblue}
\setbeamercolor{title}{fg=myblue}
\setbeamercolor{section in toc}{fg=myblue}
\setbeamercolor{item projected}{fg=white, bg=myblue}
\setbeamercolor{block title}{bg=myblue!20, fg=myblue}
\setbeamercolor{block body}{bg=myblue!10}
\setbeamercolor{alerted text}{fg=myorange}

% Set Fonts
\setbeamerfont{title}{size=\Large, series=\bfseries}
\setbeamerfont{frametitle}{size=\large, series=\bfseries}
\setbeamerfont{caption}{size=\small}
\setbeamerfont{footnote}{size=\tiny}

% Footer and Navigation Setup
\setbeamertemplate{footline}{
  \leavevmode%
  \hbox{%
  \begin{beamercolorbox}[wd=.3\paperwidth,ht=2.25ex,dp=1ex,center]{author in head/foot}%
    \usebeamerfont{author in head/foot}\insertshortauthor
  \end{beamercolorbox}%
  \begin{beamercolorbox}[wd=.5\paperwidth,ht=2.25ex,dp=1ex,center]{title in head/foot}%
    \usebeamerfont{title in head/foot}\insertshorttitle
  \end{beamercolorbox}%
  \begin{beamercolorbox}[wd=.2\paperwidth,ht=2.25ex,dp=1ex,center]{date in head/foot}%
    \usebeamerfont{date in head/foot}
    \insertframenumber{} / \inserttotalframenumber
  \end{beamercolorbox}}%
  \vskip0pt%
}

% Turn off navigation symbols
\setbeamertemplate{navigation symbols}{}

% Title Page Information
\title[Supervised vs Unsupervised Learning]{Chapter 7: Supervised vs Unsupervised Learning}
\author[J. Smith]{John Smith, Ph.D.}
\institute[University Name]{
  Department of Computer Science\\
  University Name\\
  \vspace{0.3cm}
  Email: email@university.edu\\
  Website: www.university.edu
}
\date{\today}

% Document Start
\begin{document}

\frame{\titlepage}

\begin{frame}[fragile]
    \frametitle{Introduction to Supervised and Unsupervised Learning}
    \begin{block}{Overview}
        Machine learning, a subset of artificial intelligence (AI), depends on algorithms that learn from data. The distinction between \textbf{supervised} and \textbf{unsupervised learning} is essential for effectively leveraging machine learning techniques.
    \end{block}
\end{frame}

\begin{frame}[fragile]
    \frametitle{What is Supervised Learning?}
    \begin{itemize}
        \item \textbf{Definition:} Models are trained on a labeled dataset, where each training sample is paired with an output label. The goal is to learn a mapping from inputs to the desired outputs.
        \item \textbf{Example:} Think of a teacher grading exams based on correct answers. When predicting if an email is spam, the model learns from labeled emails (spam/not spam).
    \end{itemize}
\end{frame}

\begin{frame}[fragile]
    \frametitle{What is Unsupervised Learning?}
    \begin{itemize}
        \item \textbf{Definition:} Unsupervised learning deals with unlabeled data, allowing the system to identify patterns or groupings without explicit instructions.
        \item \textbf{Example:} Consider a detective analyzing evidence without prior knowledge of a case. A common application is clustering customers based on purchasing behavior.
    \end{itemize}
\end{frame}

\begin{frame}[fragile]
    \frametitle{Significance of Understanding Their Differences}
    \begin{itemize}
        \item \textbf{Choice of Technique:} Use supervised learning for predicting outcomes and unsupervised for discovering hidden patterns.
        \item \textbf{Real-World Applications:} 
        \begin{itemize}
            \item Supervised: Credit scoring, medical diagnosis, image recognition.
            \item Unsupervised: Market basket analysis, recommendation systems, dimensionality reduction.
        \end{itemize}
    \end{itemize}
\end{frame}

\begin{frame}[fragile]
    \frametitle{Key Points to Emphasize}
    \begin{itemize}
        \item \textbf{Data Requirements:} 
            \begin{itemize}
                \item Supervised learning needs labeled data; unsupervised does not.
            \end{itemize}
        \item \textbf{Outcome Focus:} 
            \begin{itemize}
                \item Supervised is about predicting outcomes accurately; unsupervised is about discovering structure.
            \end{itemize}
        \item \textbf{Common Algorithms:} 
            \begin{itemize}
                \item Supervised: Linear Regression, Decision Trees, Neural Networks.
                \item Unsupervised: K-Means Clustering, Hierarchical Clustering, PCA.
            \end{itemize}
    \end{itemize}
\end{frame}

\begin{frame}[fragile]
    \frametitle{Engaging Questions to Ponder}
    \begin{itemize}
        \item How would you approach a problem with access to labeled versus unlabeled data?
        \item Can you think of a scenario where unsupervised learning might outperform supervised learning?
        \item What challenges might arise when using these techniques in real-life scenarios?
    \end{itemize}
\end{frame}

\begin{frame}[fragile]
    \frametitle{Conclusion}
    Grasping the differences and suitable contexts for supervised and unsupervised learning is foundational for effective machine learning practices. As we progress, consider the practical applications and theoretical underpinnings that make these techniques powerful.
\end{frame}

\begin{frame}[fragile]{Definition of Supervised Learning - Overview}
    \begin{block}{What is Supervised Learning?}
        Supervised learning is a fundamental technique in machine learning where a model learns from labeled data. The central concept involves using input-output pairs to train the model, allowing it to make predictions or classifications based on new, unseen data.
    \end{block}
\end{frame}

\begin{frame}[fragile]{Definition of Supervised Learning - Key Characteristics}
    \begin{enumerate}
        \item \textbf{Labeled Data:} Each training example is paired with an outcome label. For instance, in spam detection, emails (input) are labeled as "spam" or "not spam" (output).
        \item \textbf{Predictive Modeling:} The aim is to make predictions based on past observations, capturing the relationship between inputs and outputs.
        \item \textbf{Feedback Loop:} Model predictions can be compared to true labels, allowing for corrections and improvements through iterative training.
    \end{enumerate}
\end{frame}

\begin{frame}[fragile]{Definition of Supervised Learning - Common Usage}
    \begin{itemize}
        \item \textbf{Classification Tasks:} Model predicts categorical labels such as spam detection or handwritten digit recognition.
        \item \textbf{Regression Tasks:} Model predicts continuous outputs, like house prices based on features such as size and location.
    \end{itemize}
\end{frame}

\begin{frame}[fragile]{Definition of Supervised Learning - Examples of Common Algorithms}
    \begin{enumerate}
        \item \textbf{Linear Regression:} Predicts continuous values (e.g. car prices based on features).
        \item \textbf{Logistic Regression:} Used for binary classification tasks (e.g. tumor classification).
        \item \textbf{Support Vector Machines (SVM):} Classifies data and finds a hyperplane to separate different classes (e.g. voice recognition).
        \item \textbf{Decision Trees:} Tree-like models used for classification and regression (e.g. customer purchase evaluation).
        \item \textbf{Neural Networks:} Deep learning models that capture complex relationships (e.g. image recognition tasks).
    \end{enumerate}
\end{frame}

\begin{frame}[fragile]{Definition of Supervised Learning - Conclusion}
    \begin{block}{Final Thoughts}
        Supervised learning is a robust framework for solving predictive problems with labeled data. Its extensive applications span multiple industries, paving the way for innovative technology and solutions.
    \end{block}
    
    \begin{block}{Key Takeaway}
        Supervised learning utilizes labeled data to train models for classification and regression tasks, making it essential for predictive analytics in various real-world applications.
    \end{block}
\end{frame}

\begin{frame}[fragile]
  \frametitle{Applications of Supervised Learning - Introduction}
  \begin{block}{Overview}
    Supervised learning is a powerful machine learning approach that leverages labeled data to train models and make predictions. 
  \end{block}
  \begin{itemize}
    \item Exploration of impactful applications in real-world scenarios.
    \item Importance of supervised learning across various domains.
  \end{itemize}
\end{frame}

\begin{frame}[fragile]
  \frametitle{Applications of Supervised Learning - Fraud Detection}
  \begin{block}{Overview}
    Financial institutions use supervised learning to identify unusual patterns indicative of fraudulent activities.
  \end{block}
  \begin{itemize}
    \item \textbf{Example}: Credit card companies analyze historical transaction data (labeled as ‘fraudulent’ or ‘non-fraudulent’) to train models.
    \item Model assesses new transactions in real-time, flagging those resembling past fraudulent patterns.
    \item \textbf{Key Point}: Accurate fraud detection saves millions and enhances customer trust.
  \end{itemize}
\end{frame}

\begin{frame}[fragile]
  \frametitle{Applications of Supervised Learning - Image Recognition}
  \begin{block}{Overview}
    Supervised learning algorithms excel in identifying objects within images.
  \end{block}
  \begin{itemize}
    \item \textbf{Example}: Social media platforms use convolutional neural networks (CNNs) trained on millions of labeled images to recognize faces.
    \item Model facilitates auto-tagging of friends.
    \item \textbf{Key Point}: Image recognition enables applications in security (e.g., facial recognition) and healthcare (e.g., anomaly detection in medical scans).
  \end{itemize}
\end{frame}

\begin{frame}[fragile]
  \frametitle{Applications of Supervised Learning - Medical Diagnosis}
  \begin{block}{Overview}
    Supervised learning aids healthcare professionals in diagnosing diseases based on patient symptoms and medical histories.
  \end{block}
  \begin{itemize}
    \item \textbf{Example}: Models trained with patient data (known outcomes) predict diseases like diabetes or cancer.
    \item Analyzing patterns in X-ray images labeled as ‘normal’ or ‘abnormal’ helps assist radiologists.
    \item \textbf{Key Point}: Early and accurate diagnosis improves patient outcomes and treatments.
  \end{itemize}
\end{frame}

\begin{frame}[fragile]
  \frametitle{Applications of Supervised Learning - Sentiment Analysis}
  \begin{block}{Overview}
    Companies use supervised learning to gauge public sentiment by analyzing customer reviews or social media posts.
  \end{block}
  \begin{itemize}
    \item \textbf{Example}: Training models on labeled positive and negative reviews helps businesses determine overall sentiment.
    \item \textbf{Key Point}: Insights from sentiment analysis enable companies to fine-tune marketing strategies and improve customer satisfaction.
  \end{itemize}
\end{frame}

\begin{frame}[fragile]
  \frametitle{Applications of Supervised Learning - Summary}
  \begin{block}{Summary}
    Supervised learning finds applications across diverse fields such as finance, healthcare, and marketing.
  \end{block}
  \begin{itemize}
    \item Its ability to analyze past data and apply learnings to new scenarios makes it an invaluable tool for informed decision-making.
  \end{itemize}
\end{frame}

\begin{frame}[fragile]
  \frametitle{Applications of Supervised Learning - Thought Provoker}
  \begin{questionblock}{Discussion}
    How might advancements in supervised learning algorithms shape the future of industries such as autonomous driving or personalized medicine? 
    Consider the implications of larger datasets and sophisticated models like transformers in these fields.
  \end{questionblock}
\end{frame}

\begin{frame}[fragile]
    \frametitle{Definition of Unsupervised Learning}
    \begin{block}{What is Unsupervised Learning?}
        Unsupervised learning is a type of machine learning where algorithms learn from \textbf{unlabeled data}. The algorithm must discover patterns and structures without explicit outputs or categories.
    \end{block}
\end{frame}

\begin{frame}[fragile]
    \frametitle{Key Features of Unsupervised Learning}
    \begin{itemize}
        \item \textbf{No Labeled Data}: Algorithms operate on input data without known outcomes or classes.
        \item \textbf{Data Exploration}: Primarily used for exploratory purposes, organizing data based on inherent traits.
        \item \textbf{Automated Pattern Recognition}: Identifies hidden structures and correlations in the data autonomously.
    \end{itemize}
\end{frame}

\begin{frame}[fragile]
    \frametitle{Main Types of Unsupervised Learning}
    \begin{columns}
        \begin{column}{0.5\textwidth}
            \textbf{Clustering}
            \begin{itemize}
                \item Groups data points based on similarity.
                \item \textit{Example:} Customer segmentation for personalized marketing strategies.
            \end{itemize}
        \end{column}
        \begin{column}{0.5\textwidth}
            \textbf{Association}
            \begin{itemize}
                \item Discovers rules describing large data portions.
                \item \textit{Example:} Market basket analysis showing buying patterns (e.g., bread and butter).
            \end{itemize}
        \end{column}
    \end{columns}
\end{frame}

\begin{frame}[fragile]
    \frametitle{How Unsupervised Learning Differs from Supervised Learning}
    \begin{itemize}
        \item \textbf{Supervised Learning}: Works with labeled datasets to predict outcomes (e.g., spam classification).
        \item \textbf{Purpose of Unsupervised Learning}: Focuses on discovering hidden patterns rather than predicting known outcomes.
    \end{itemize}
\end{frame}

\begin{frame}[fragile]
    \frametitle{Conclusions and Applications}
    \begin{itemize}
        \item \textbf{Key Points}:
        \begin{itemize}
            \item Unsupervised learning is vital for extracting insights from unlabeled data.
            \item Applications include customer insights, anomaly detection, marketing, healthcare, and quality control.
        \end{itemize}
        \item \textbf{Interactive Element}:
            \begin{block}{Think-Pair-Share Question}
                How could you use unsupervised learning to improve customer service in a business? Discuss with a partner and share your ideas!
            \end{block}
    \end{itemize}
\end{frame}

\begin{frame}[fragile]
    \frametitle{Applications of Unsupervised Learning - Overview}
    \begin{itemize}
        \item Unsupervised learning discovers patterns in unlabeled data.
        \item Key applications:
        \begin{itemize}
            \item Marketing Analysis
            \item Customer Segmentation
            \item Anomaly Detection
        \end{itemize}
    \end{itemize}
\end{frame}

\begin{frame}[fragile]
    \frametitle{Applications of Unsupervised Learning - Marketing Analysis}
    \begin{block}{Concept}
        Unsupervised learning helps marketers understand consumer behavior without predefined labels.
    \end{block}
    
    \begin{block}{Example}
        Companies use clustering algorithms like K-Means to group customers based on purchasing behaviors.
        \begin{itemize}
            \item E.g., Customers can be segmented into clusters: “bargain hunters,” “brand loyalists,” “occasional shoppers.”
        \end{itemize}
    \end{block}
    
    \begin{block}{Key Takeaway}
        Personalized marketing strategies are developed by optimizing campaigns for different customer segments.
    \end{block}
\end{frame}

\begin{frame}[fragile]
    \frametitle{Applications of Unsupervised Learning - Customer Segmentation}
    \begin{block}{Concept}
        Identifying distinct groups helps companies target specific audiences more effectively.
    \end{block}

    \begin{block}{Example}
        A travel agency can cluster clients based on travel preferences, e.g., adventure seekers vs luxury travelers.
    \end{block}
    
    \begin{block}{Key Takeaway}
        Effective segmentation enhances customer experiences and increases sales via targeted promotions.
    \end{block}
\end{frame}

\begin{frame}[fragile]
    \frametitle{Applications of Unsupervised Learning - Anomaly Detection}
    \begin{block}{Concept}
        Useful for spotting unusual patterns indicating fraud or system failures.
    \end{block}

    \begin{block}{Example}
        Credit card companies employ anomaly detection algorithms to monitor transactions.
        \begin{itemize}
            \item Flags unusual spending patterns indicating potential fraud.
        \end{itemize}
    \end{block}

    \begin{block}{Key Takeaway}
        Early detection of anomalies mitigates risks, ensuring customer safety and company integrity.
    \end{block}
\end{frame}

\begin{frame}[fragile]
    \frametitle{Applications of Unsupervised Learning - Summary}
    \begin{itemize}
        \item Unsupervised learning reveals insights from unlabeled data, driving strategic decisions.
        \item Key Applications:
        \begin{itemize}
            \item Marketing Analysis: Tailored messaging for client segments.
            \item Customer Segmentation: Targeted products/services based on behavior.
            \item Anomaly Detection: Protecting against fraud and monitoring operations.
        \end{itemize}
    \end{itemize}
\end{frame}

\begin{frame}[fragile]
    \frametitle{Comparison of Supervised and Unsupervised Learning - Overview}
    Supervised and unsupervised learning are two fundamental types of machine learning methods that serve different purposes.  
    Understanding their differences helps you choose the right approach for your data analysis needs.
\end{frame}

\begin{frame}[fragile]
    \frametitle{Comparison of Supervised and Unsupervised Learning - Key Differences}
    \begin{table}[h]
        \centering
        \begin{tabular}{|l|l|l|}
            \hline
            \textbf{Feature} & \textbf{Supervised Learning} & \textbf{Unsupervised Learning} \\
            \hline
            Data Requirements & Requires labeled datasets (input-output pairs) & Uses unlabeled datasets (input only) \\
            \hline
            Problem Types & Classification and regression tasks & Clustering and association tasks \\
            \hline
            Learning Objective & Predict outcomes based on input data & Discover patterns and relationships in data \\
            \hline
            Example Algorithms & Linear Regression, Decision Trees, Support Vector Machines & K-means Clustering, Hierarchical Clustering, PCA \\
            \hline
            Output & Provides specific predictions or classifications & Generates insights about data structure without specific predictions \\
            \hline
        \end{tabular}
    \end{table}
\end{frame}

\begin{frame}[fragile]
    \frametitle{Comparison of Supervised and Unsupervised Learning - Explanations}
    \begin{block}{Supervised Learning}
      In supervised learning, the model is trained on a dataset where each input is paired with the correct output.
      This enables the algorithm to learn relationships and predict outcomes.  
      \textbf{Example:} A spam detection model classifies emails as "spam" or "not spam".
    \end{block}
    
    \begin{block}{Unsupervised Learning}
      Unsupervised learning seeks to identify patterns or structures from input data alone, without pre-labeled outputs.  
      \textbf{Example:} Customer segmentation to group customers based on purchasing behavior without predefined categories.
    \end{block}
\end{frame}

\begin{frame}[fragile]
    \frametitle{Comparison of Supervised and Unsupervised Learning - Key Points}
    \begin{enumerate}
        \item \textbf{Purpose and Goals:}  
            Supervised learning focuses on making predictions while unsupervised learning explores data for hidden relationships.
        
        \item \textbf{Applications:}  
            Supervised learning is used in finance (credit scoring) and healthcare (disease prediction). 
            Unsupervised learning excels in exploratory tasks like market basket analysis.
        
        \item \textbf{Choosing Techniques:}  
            Choose supervised learning for labeled datasets and unsupervised learning for exploring insights from unlabeled data.
    \end{enumerate}
\end{frame}

\begin{frame}[fragile]
    \frametitle{Comparison of Supervised and Unsupervised Learning - Conclusion}
    Both supervised and unsupervised learning are essential in data science.  
    Recognizing the distinctions and applications of each method empowers you to select the appropriate approach  
    based on your specific needs and dataset characteristics.
\end{frame}

\begin{frame}[fragile]
    \frametitle{Choosing the Right Approach}
    \begin{block}{Introduction to Selection Criteria}
        When determining whether to use supervised or unsupervised learning, it’s essential to consider three core factors: 
        \begin{enumerate}
            \item Project requirements
            \item Data type
            \item Business needs
        \end{enumerate}
        Understanding these will guide you in making the correct choice for your project.
    \end{block}
\end{frame}

\begin{frame}[fragile]
    \frametitle{1. Understanding Project Requirements}
    \begin{block}{Questions to Consider}
        \begin{itemize}
            \item \textbf{Are labeled data available?}
                \begin{itemize}
                    \item \textbf{Supervised Learning}: Requires datasets with input-output pairs (e.g., labels).
                    \item \textbf{Use Case}: Predicting house prices with known prices.
                \end{itemize}
            \item \textbf{What is your objective?}
                \begin{itemize}
                    \item \textbf{Supervised Learning}: For classification and regression tasks.
                    \item \textbf{Unsupervised Learning}: To discover patterns within unlabeled data.
                \end{itemize}
        \end{itemize}
    \end{block}

    \begin{block}{Example}
        \begin{itemize}
            \item \textbf{Supervised}: Predicting customer churn using historical data.
            \item \textbf{Unsupervised}: Segmenting customers based on buying habits.
        \end{itemize}
    \end{block}
\end{frame}

\begin{frame}[fragile]
    \frametitle{2. Data Type Considerations}
    \begin{block}{Key Points}
        \begin{itemize}
            \item \textbf{Type of Data}:
                \begin{itemize}
                    \item \textbf{Numerical}: Can be both ‒ predicting sales (supervised) or clustering sales data (unsupervised).
                    \item \textbf{Categorical}: Classifying reviews as positive/negative (supervised) vs. grouping products (unsupervised).
                \end{itemize}
            \item \textbf{Volume of Data}:
                \begin{itemize}
                    \item Larger datasets enhance supervised model reliability, while unsupervised can reveal insights from smaller datasets.
                \end{itemize}
        \end{itemize}
    \end{block}

    \begin{block}{Illustration}
        \begin{itemize}
            \item \textbf{Supervised}: Predicting house prices based on features such as size.
            \item \textbf{Unsupervised}: Finding customer segments based on purchasing behavior.
        \end{itemize}
    \end{block}
\end{frame}

\begin{frame}[fragile]
    \frametitle{3. Aligning with Business Needs}
    \begin{block}{Considerations}
        \begin{itemize}
            \item \textbf{Immediate Use}:
                \begin{itemize}
                    \item Supervised Learning provides quick insights for immediate decisions (e.g., fraud detection).
                \end{itemize}
            \item \textbf{Exploratory Analysis}:
                \begin{itemize}
                    \item Unsupervised Learning supports exploration to drive future strategies (e.g., discovering new user personas).
                \end{itemize}
        \end{itemize}
    \end{block}
    
    \begin{block}{Example}
        \begin{itemize}
            \item \textbf{Supervised}: A bank predicting loan defaults based on past data.
            \item \textbf{Unsupervised}: An e-commerce company identifying purchasing trends across demographics.
        \end{itemize}
    \end{block}
\end{frame}

\begin{frame}[fragile]
    \frametitle{Summary of Key Points}
    \begin{block}{}
        \begin{itemize}
            \item \textbf{Supervised Learning}: Use with labeled data and specific targets.
            \item \textbf{Unsupervised Learning}: Useful for discovering patterns without labels.
            \item \textbf{Alignment}: Always consider project goals, data availability, and business impact.
        \end{itemize}
    \end{block}
\end{frame}

\begin{frame}[fragile]
  \frametitle{Summary - Understanding Learning Approaches}
  \begin{block}{Overview}
    This chapter explored two fundamental machine learning approaches: 
    \textbf{Supervised Learning} and \textbf{Unsupervised Learning}.
    Both play essential roles in data analysis and predictions.
  \end{block}
\end{frame}

\begin{frame}[fragile]
  \frametitle{Summary - Key Points Recap}
  \begin{enumerate}
    \item \textbf{Supervised Learning:}
    \begin{itemize}
      \item \textbf{Definition:} Model trained on labeled data.
      \item \textbf{Examples:} 
        \begin{itemize}
          \item Classification (e.g., spam detection)
          \item Regression (e.g., price prediction)
        \end{itemize}
      \item \textbf{Use Cases:} Ideal for historical data predictions.
    \end{itemize}
    
    \item \textbf{Unsupervised Learning:}
    \begin{itemize}
      \item \textbf{Definition:} Model trained on unlabeled data to identify patterns.
      \item \textbf{Examples:}
        \begin{itemize}
          \item Clustering (e.g., customer segmentation)
          \item Dimensionality Reduction (e.g., PCA)
        \end{itemize}
      \item \textbf{Use Cases:} Useful for data exploration.
    \end{itemize}
  \end{enumerate}
\end{frame}

\begin{frame}[fragile]
  \frametitle{Summary - Importance of Learning Both Approaches}
  \begin{itemize}
    \item \textbf{Versatility in Problem-Solving:} Different problems need different methods.
    \item \textbf{Complementary Insights:} Combining both learning types can enhance model performance.
    \item \textbf{Real-World Applications:}
    \begin{itemize}
      \item Healthcare (supervised) vs. Market Segmentation (unsupervised).
    \end{itemize}
  \end{itemize}

  \begin{block}{Engaging Questions}
    \begin{itemize}
      \item How might a business benefit from predictive models?
      \item In what ways can unsupervised learning enhance consumer understanding?
    \end{itemize}
  \end{block}
\end{frame}


\end{document}