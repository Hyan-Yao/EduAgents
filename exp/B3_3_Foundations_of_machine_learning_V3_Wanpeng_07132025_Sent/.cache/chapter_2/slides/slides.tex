\documentclass[aspectratio=169]{beamer}

% Theme and Color Setup
\usetheme{Madrid}
\usecolortheme{whale}
\useinnertheme{rectangles}
\useoutertheme{miniframes}

% Additional Packages
\usepackage[utf8]{inputenc}
\usepackage[T1]{fontenc}
\usepackage{graphicx}
\usepackage{booktabs}
\usepackage{listings}
\usepackage{amsmath}
\usepackage{amssymb}
\usepackage{xcolor}
\usepackage{tikz}
\usepackage{pgfplots}
\pgfplotsset{compat=1.18}
\usetikzlibrary{positioning}
\usepackage{hyperref}

% Custom Colors
\definecolor{myblue}{RGB}{31, 73, 125}
\definecolor{mygray}{RGB}{100, 100, 100}
\definecolor{mygreen}{RGB}{0, 128, 0}
\definecolor{myorange}{RGB}{230, 126, 34}
\definecolor{mycodebackground}{RGB}{245, 245, 245}

% Title Page Information
\title[Data Types]{Chapter 2: Types of Data}
\author[J. Smith]{John Smith, Ph.D.}
\institute[University Name]{
  Department of Computer Science\\
  University Name\\
  \vspace{0.3cm}
  Email: email@university.edu\\
  Website: www.university.edu
}
\date{\today}

\begin{document}

\frame{\titlepage}

\begin{frame}[fragile]
    \frametitle{Introduction to Types of Data}
    \begin{block}{Significance of Data Types in AI}
        Understanding data types is crucial in AI, determining how data can be used, processed, and analyzed.
    \end{block}
\end{frame}

\begin{frame}[fragile]
    \frametitle{Why Data Types Matter}
    \begin{itemize}
        \item \textbf{Decision-Making:} 
        Different AI algorithms require specific data types. 
        \begin{itemize}
            \item Image recognition - structured images
            \item Natural language processing - unstructured text
        \end{itemize}
        
        \item \textbf{Model Performance:} 
        The quality of data influences AI model accuracy; poorly labeled data leads to inaccurate predictions.
        
        \item \textbf{Scalability and Efficiency:} 
        Knowing data type aids in selecting tools and techniques for storage, processing, and analysis.
    \end{itemize}
\end{frame}

\begin{frame}[fragile]
    \frametitle{Common Data Types in AI}
    \begin{itemize}
        \item \textbf{Structured Data:} 
        Highly organized and easily searchable.
        \begin{itemize}
            \item Example: Sales transactions table with product ID, quantity sold, and price.
        \end{itemize}

        \item \textbf{Unstructured Data:} 
        Lacks a predefined format, complex to analyze.
        \begin{itemize}
            \item Example: Customer reviews with varied styles.
        \end{itemize}

        \item \textbf{Semi-structured Data:} 
        Partially organized, often uses tags for separation.
        \begin{itemize}
            \item Example: JSON or XML data files.
        \end{itemize}
    \end{itemize}
\end{frame}

\begin{frame}[fragile]
    \frametitle{Key Takeaways and Questions}
    \begin{itemize}
        \item Identifying appropriate data types is essential for AI success.
        \item AI methodologies are better suited for specific data types.
        \item Understanding diverse data types is vital as data generation grows.
    \end{itemize}

    \begin{block}{Questions to Consider}
        \begin{itemize}
            \item How do different types of data influence machine learning model training?
            \item What challenges arise from using unstructured data in AI?
            \item When is semi-structured data more beneficial than other types?
        \end{itemize}
    \end{block}
\end{frame}

\begin{frame}[fragile]{Presentation Overview}
  \tableofcontents[hideallsubsections]
\end{frame}

\begin{frame}[fragile]{Understanding Structured Data}
  \frametitle{Structured Data}
  
  \begin{block}{Definition}
    Structured data refers to information that is organized into a well-defined format, making it easily searchable and analyzable.
  \end{block}
  
  \begin{itemize}
    \item \textbf{Key Characteristics:}
    \begin{itemize}
      \item Highly Organized (e.g., databases)
      \item Data Types are Defined (e.g., integers, strings)
      \item Easily Processed (queries with SQL)
    \end{itemize}
    
    \item \textbf{Common Examples:}
    \begin{itemize}
      \item Relational Databases (MySQL, PostgreSQL)
      \item Spreadsheets (Excel, Google Sheets)
    \end{itemize}
  \end{itemize}
\end{frame}

\begin{frame}[fragile]{Understanding Unstructured Data}
  \frametitle{Unstructured Data}
  
  \begin{block}{Definition}
    Unstructured data is information that does not conform to a pre-defined data model, presenting challenges for analysis.
  \end{block}
  
  \begin{itemize}
    \item \textbf{Key Characteristics:}
    \begin{itemize}
      \item Lack of Organization
      \item Diverse Formats (text documents, images, audio, videos)
      \item Requires Advanced Processing (NLP, computer vision)
    \end{itemize}
    
    \item \textbf{Common Examples:}
    \begin{itemize}
      \item Text Data (emails, social media posts)
      \item Images and Videos (YouTube, Instagram content)
    \end{itemize}
  \end{itemize}
\end{frame}

\begin{frame}[fragile]{Comparing Structured and Unstructured Data}
  \frametitle{Key Points to Emphasize}
  
  \begin{itemize}
    \item \textbf{Structured Data:}
    \begin{itemize}
      \item Ideal for quantitative analysis
      \item Supports well-defined queries for reporting and analytics
    \end{itemize}
    
    \item \textbf{Unstructured Data:}
    \begin{itemize}
      \item Offers qualitative insights
      \item Represents the majority of data produced today
    \end{itemize}
  \end{itemize}
  
  \begin{block}{Conclusion}
    Both data types are vital for leveraging insights in a data-driven world, especially in AI applications.
  \end{block}
\end{frame}

\begin{frame}[fragile]{Illustrative Examples and Questions}
  \frametitle{Examples and Discussion}
  
  \begin{itemize}
    \item \textbf{Example of Online Retail Business:}
    \begin{itemize}
      \item \textbf{Structured:} Customer data (Name, Address, Purchase History) in a relational database.
      \item \textbf{Unstructured:} Customer reviews on the website or product images uploaded by users.
    \end{itemize}
    
    \item \textbf{Engaging Questions:}
    \begin{itemize}
      \item How might unstructured data inform business decisions differently than structured data?
      \item Can you think of examples in your daily life where you interact with both data types?
    \end{itemize}
  \end{itemize}
  
  This understanding will set the stage for exploring how these data types influence machine learning in the next chapter.
\end{frame}

\begin{frame}[fragile]
    \frametitle{Significance of Data Types in AI - Introduction}
    In the realm of Artificial Intelligence (AI), the type of data we work with significantly impacts how well our machine learning models perform. 
    \begin{itemize}
        \item Understanding the nuances between various data types is crucial.
        \item Effective AI applications require tailored approaches based on the data type.
    \end{itemize}
\end{frame}

\begin{frame}[fragile]
    \frametitle{Significance of Data Types in AI - Key Data Categories}
    \begin{enumerate}
        \item \textbf{Structured Data}
            \begin{itemize}
                \item \textbf{Definition}: Organized information typically found in databases/spreadsheets (rows and columns).
                \item \textbf{Examples}: Customer databases, transaction records, sensor data.
                \item \textbf{Impact on AI}: Easier to analyze; facilitates classical algorithms (e.g., linear regression).
            \end{itemize}
        
        \item \textbf{Unstructured Data}
            \begin{itemize}
                \item \textbf{Definition}: Information without a predefined format, challenging to process.
                \item \textbf{Examples}: Text documents, images, videos, social media posts.
                \item \textbf{Impact on AI}: Requires advanced algorithms (e.g., neural networks); enables applications like NLP.
            \end{itemize}
        
        \item \textbf{Semi-Structured Data}
            \begin{itemize}
                \item \textbf{Definition}: Data with some organizational properties but not in a traditional database.
                \item \textbf{Examples}: JSON and XML files, web pages with embedded tags.
                \item \textbf{Impact on AI}: Provides flexibility while retaining some structure for processing.
            \end{itemize}
    \end{enumerate}
\end{frame}

\begin{frame}[fragile]
    \frametitle{Significance of Data Types in AI - Model Performance}
    \begin{block}{Role of Data Types}
        \begin{itemize}
            \item \textbf{Feature Engineering}: Different data types need tailored preprocessing techniques.
            \begin{itemize}
                \item Text: Tokenization or vectorization.
                \item Image: Resizing or normalization.
            \end{itemize}
            
            \item \textbf{Model Selection and Complexity}:
            \begin{itemize}
                \item \textbf{Structured Data}: Simpler models often yield high performance.
                \item \textbf{Unstructured Data}: Requires complex models (e.g., Convolutional Neural Networks).
            \end{itemize}

            \item \textbf{Data Quantity and Quality}:
            \begin{itemize}
                \item High-quality structured data leads to better baseline models.
                \item Unstructured data may require larger datasets for effective training.
            \end{itemize}
        \end{itemize}
    \end{block}
\end{frame}

\begin{frame}[fragile]
    \frametitle{Significance of Data Types in AI - Examples and Conclusion}
    \begin{block}{Illustrative Example}
        \begin{itemize}
            \item \textbf{Using Structured Data}:
                \begin{itemize}
                    \item Database tracking purchases can yield insights easily.
                \end{itemize}
            \item \textbf{Using Unstructured Data}:
                \begin{itemize}
                    \item Analyzing customer reviews requires NLP techniques.
                \end{itemize}
        \end{itemize}
    \end{block}

    \begin{block}{Key Points to Remember}
        \begin{itemize}
            \item The type of data influences the approach to AI model development.
            \item Structured data is suited for simpler models; unstructured data allows for richer insights with complexity.
            \item Understanding data strengths/weaknesses guides machine learning strategy selection.
        \end{itemize}
    \end{block}

    \begin{block}{Conclusion}
        Recognizing the significance of data types is foundational in AI development, optimizing model performance based on available data.
    \end{block}
\end{frame}

\begin{frame}[fragile]
    \frametitle{Types of Machine Learning}
    \begin{block}{Overview}
        Machine Learning (ML) is a subset of artificial intelligence that allows systems to learn from data and improve their performance over time without explicit programming. The three primary types of machine learning are:
        \begin{itemize}
            \item Supervised Learning
            \item Unsupervised Learning
            \item Reinforcement Learning
        \end{itemize}
        Each type has unique applications, learning mechanisms, and use cases.
    \end{block}
\end{frame}

\begin{frame}[fragile]
    \frametitle{1. Supervised Learning}
    \begin{itemize}
        \item \textbf{Definition}: Trained on a labeled dataset; each training example has an output label.
        \item \textbf{Mechanism}: Learns to map inputs to outputs using a training set. Makes predictions on new data.
        \item \textbf{Examples}:
        \begin{itemize}
            \item Classification: Predicting if an email is spam.
            \item Regression: Forecasting housing prices based on features.
        \end{itemize}
        \item \textbf{Key Point}: Effective when output is known and labeled data is available.
    \end{itemize}
\end{frame}

\begin{frame}[fragile]
    \frametitle{2. Unsupervised Learning}
    \begin{itemize}
        \item \textbf{Definition}: Trained on data without labeled responses; aims to find patterns.
        \item \textbf{Mechanism}: Analyzes input data to identify clusters or associations.
        \item \textbf{Examples}:
        \begin{itemize}
            \item Clustering: Grouping customers by purchasing behavior.
            \item Dimensionality Reduction: Techniques like PCA to reduce variables.
        \end{itemize}
        \item \textbf{Key Point}: Useful for exploratory data analysis and discovering hidden patterns.
    \end{itemize}
\end{frame}

\begin{frame}[fragile]
    \frametitle{3. Reinforcement Learning}
    \begin{itemize}
        \item \textbf{Definition}: An agent learns to make decisions by performing actions to maximize cumulative reward.
        \item \textbf{Mechanism}: Learns from feedback (rewards/penalties) and adapts strategies.
        \item \textbf{Examples}:
        \begin{itemize}
            \item Gaming: Algorithms in AlphaGo that learn strategies through self-play.
            \item Robotics: Robots navigated through environments by trial and error.
        \end{itemize}
        \item \textbf{Key Point}: Useful in adaptive decision-making in complex environments.
    \end{itemize}
\end{frame}

\begin{frame}[fragile]
    \frametitle{Summary of Differences}
    \begin{table}[h!]
        \centering
        \begin{tabular}{|c|c|c|c|}
            \hline
            \textbf{Feature} & \textbf{Supervised Learning} & \textbf{Unsupervised Learning} & \textbf{Reinforcement Learning} \\
            \hline
            Data Type & Labeled data & Unlabeled data & Environment feedback \\
            \hline
            Learning Objective & Predict outcomes & Discover patterns & Maximize cumulative rewards \\
            \hline
            Use Cases & Email classification, price prediction & Customer segmentation, anomaly detection & Game AI, robot navigation \\
            \hline
        \end{tabular}
    \end{table}
\end{frame}

\begin{frame}[fragile]
    \frametitle{Conclusion}
    Understanding these types of machine learning is crucial for selecting the appropriate approach based on data availability and the problem to be solved. Each type utilizes different data characteristics and learning paradigms, leading to various applications across diverse fields such as finance, healthcare, marketing, and robotics. 

    Next, we will delve into exploring data relationships and visualization techniques, reinforcing concepts of data types and machine learning further.
\end{frame}

\begin{frame}[fragile]
    \frametitle{Data Relationships and Visualization - Introduction}
    \begin{itemize}
        \item Understanding data relationships is crucial for analysis.
        \item Visualization helps to uncover patterns, trends, and insights.
        \item Effective communication of data relationships is key for decision-making.
    \end{itemize}
\end{frame}

\begin{frame}[fragile]
    \frametitle{Key Concepts}
    \begin{block}{1. Data Relationships}
        \begin{itemize}
            \item \textbf{Correlation}: Examines associations between two variables.
            \item \textbf{Causation vs. Correlation}: Correlation does not imply causation; deeper investigation is required.
        \end{itemize}
    \end{block}

    \begin{block}{2. Types of Relationships}
        \begin{itemize}
            \item \textbf{Positive Relationship}: As one variable increases, the other does too.
            \item \textbf{Negative Relationship}: As one variable increases, the other decreases.
        \end{itemize}
    \end{block}
\end{frame}

\begin{frame}[fragile]
    \frametitle{Visualization Techniques}
    \begin{itemize}
        \item \textbf{Scatter Plots}:
        \begin{itemize}
            \item Useful for two continuous variables.
            \item Example: Relationship between years of experience and salary.
        \end{itemize}
        
        \item \textbf{Bar Charts}:
        \begin{itemize}
            \item Effective for comparing categorical data.
            \item Example: Sales figures across product categories.
        \end{itemize}

        \item \textbf{Heatmaps}:
        \begin{itemize}
            \item Visualizes complex data relationships.
            \item Example: User interactions across hours of the day.
        \end{itemize}
    \end{itemize}
    
    \begin{block}{Importance of Visualization}
        \begin{itemize}
            \item Promotes understanding of complex relationships.
            \item Drives insights that raw data may not reveal.
            \item Facilitates effective communication of findings.
        \end{itemize}
    \end{block}
\end{frame}

\begin{frame}[fragile]
    \frametitle{Basic Machine Learning Models - Overview}
    Machine Learning (ML) involves creating algorithms that allow computers to learn from data to make predictions or decisions. Understanding basic ML models is crucial for grasping data-driven solutions. This slide focuses on three foundational models:
    \begin{itemize}
        \item Linear Regression
        \item Decision Trees
        \item k-Nearest Neighbors (k-NN)
    \end{itemize}
\end{frame}

\begin{frame}[fragile]
    \frametitle{Basic Machine Learning Models - Linear Regression}
    \begin{block}{1. Linear Regression}
        \textbf{Concept:} Predicts a continuous outcome through a linear relationship between input features and the target variable.
    \end{block}
    \textbf{Example:} Predicting house prices based on size (square footage):
    \begin{equation}
        \text{Price} = m \cdot \text{Size} + b
    \end{equation}
    where:
    \begin{itemize}
        \item \( m \): slope (increment in price per square foot)
        \item \( b \): y-intercept (base price)
    \end{itemize}
    \textbf{Key Point:} Good for initial predictions when relationships are simple.
\end{frame}

\begin{frame}[fragile]
    \frametitle{Basic Machine Learning Models - Decision Trees and k-NN}
    \begin{block}{2. Decision Trees}
        \textbf{Concept:} Splits data into subsets based on feature values, forming a tree-like model.
    \end{block}
    \textbf{Example:} Classifying customer purchase behavior:
    \begin{itemize}
        \item Is the customer under 30?
        \begin{itemize}
            \item Yes: further questions
            \item No: 'No' branch
        \end{itemize}
    \end{itemize}
    \textbf{Key Point:} Easy to interpret, ideal for exploration.

    \begin{block}{3. k-Nearest Neighbors (k-NN)}
        \textbf{Concept:} Classifies data points based on nearest neighbors in feature space.
    \end{block}
    \textbf{Example:} Is a fruit an apple or orange based on weight and color intensity? It predicts based on majority class of neighbors.
    \textbf{Key Point:} Simple but can be slow with large datasets.
\end{frame}

\begin{frame}[fragile]
    \frametitle{Implementing and Evaluating Machine Learning Models}
    \textbf{Implementation Steps:}
    \begin{enumerate}
        \item \textbf{Data Preparation:} Clean and preprocess data.
        \item \textbf{Model Selection:} Choose based on data and task.
        \item \textbf{Training:} Train model with part of the dataset.
        \item \textbf{Testing:} Evaluate model on unseen data.
    \end{enumerate}
    
    \textbf{Evaluation Metrics:}
    \begin{itemize}
        \item \textbf{Accuracy:} Correct predictions out of total instances.
        \item \textbf{Mean Squared Error (MSE):} Average of squared errors for regression.
    \end{itemize}
\end{frame}

\begin{frame}[fragile]
    \frametitle{Basic Machine Learning Models - Conclusion and Questions}
    Basic machine learning models form the foundation for advanced algorithms. Understanding these methods is the first step in building predictive models that leverage data insights.
    
    \textbf{Questions to Consider:}
    \begin{itemize}
        \item How might these models apply to everyday decision-making?
        \item In which scenarios could one model outperform another, and why?
    \end{itemize}
\end{frame}

\begin{frame}[fragile]
    \frametitle{Exploring Data Sources - Introduction}
    In the realm of artificial intelligence (AI), data is the backbone of every algorithm and machine learning model. Understanding where to obtain high-quality data and how to leverage it effectively can lead to innovative solutions and insights. This slide discusses various data sources relevant to AI, showcasing practical examples of data-driven solutions.
\end{frame}

\begin{frame}[fragile]
    \frametitle{Exploring Data Sources - Key Data Sources}
    
    \begin{enumerate}
        \item \textbf{Structured Data}
        \begin{itemize}
            \item \textbf{Definition}: Organized and easily searchable information (e.g., databases, spreadsheets).
            \item \textbf{Example}: Customer databases storing names, contact details, transaction history.
            \item \textbf{Application}: Predictive analytics for forecasting customer behavior and personalized marketing recommendations.
        \end{itemize}
        
        \item \textbf{Unstructured Data}
        \begin{itemize}
            \item \textbf{Definition}: Lacks a predefined format (e.g., text, images, videos).
            \item \textbf{Example}: Social media posts, customer reviews, and emails.
            \item \textbf{Application}: Sentiment analysis informing businesses about public perception of their brand.
        \end{itemize}
        
        \item \textbf{Semi-Structured Data}
        \begin{itemize}
            \item \textbf{Definition}: Data that has organizational properties but doesn't reside in a relational database (e.g., JSON, XML).
            \item \textbf{Example}: Web log files recording user interactions on a website.
            \item \textbf{Application}: Web analytics tools analyzing site traffic patterns to enhance user experience.
        \end{itemize}
    \end{enumerate}
\end{frame}

\begin{frame}[fragile]
    \frametitle{Exploring Data Sources - Key Data Sources (Continued)}
    
    \begin{enumerate}
        \setcounter{enumi}{3} % Continue numbering from 3
        \item \textbf{Open Data}
        \begin{itemize}
            \item \textbf{Definition}: Publicly available data that can be accessed and used freely (e.g., government datasets).
            \item \textbf{Example}: Data from the World Health Organization (WHO) on global health statistics.
            \item \textbf{Application}: Analyzing public data to track disease outbreaks and improve health services.
        \end{itemize}
        
        \item \textbf{Synthetic Data}
        \begin{itemize}
            \item \textbf{Definition}: Data generated artificially rather than drawn from real-world events.
            \item \textbf{Example}: Virtual environments simulating realistic movements for training autonomous vehicles.
            \item \textbf{Application}: Building robust models in scenarios where real data is scarce (e.g., medical imaging).
        \end{itemize}
    \end{enumerate}
\end{frame}

\begin{frame}[fragile]
    \frametitle{Practical Examples of Data-Driven Solutions}
    
    \begin{itemize}
        \item \textbf{Recommendation Systems}: Companies like Netflix and Amazon analyze both structured and unstructured data to recommend products or content tailored to individual preferences.
        \item \textbf{Fraud Detection}: Financial institutions leverage transaction data (structured) combined with behavioral data (semi-structured) to identify unusual patterns and prevent fraud in real-time.
        \item \textbf{Healthcare Analytics}: Medical researchers employ open and structured datasets to develop predictive models for forecasting patient outcomes based on historical data.
    \end{itemize}
\end{frame}

\begin{frame}[fragile]
    \frametitle{Key Points to Emphasize}
    \begin{itemize}
        \item \textbf{Diversity of Data Sources}: Different data types provide unique insights; combining them enhances model accuracy.
        \item \textbf{Importance of Data Quality}: High-quality data leads to better AI outcomes; focus on reliability and cleanliness.
        \item \textbf{Innovative Uses of Data}: Creative applications of diverse data sources are revolutionizing industries, from healthcare to entertainment.
    \end{itemize}
\end{frame}

\begin{frame}[fragile]
    \frametitle{Ethical Considerations in Data Use}
    \begin{block}{Introduction to Ethical Considerations}
        Ethical considerations ensure that data is handled responsibly and respectfully, minimizing harm while maximizing benefits of data-driven decisions.
    \end{block}
\end{frame}

\begin{frame}[fragile]
    \frametitle{Key Ethical Issues in Data Usage}
    \begin{itemize}
        \item \textbf{Bias in Data}
            \begin{itemize}
                \item \textit{Definition:} Bias occurs when data samples do not accurately represent the population.
                \item \textit{Example:} Hiring algorithms trained on historical data may favor certain demographics, like race or gender.
            \end{itemize}
        
        \item \textbf{Privacy Concerns}
            \begin{itemize}
                \item \textit{Definition:} Privacy issues arise when personal data is collected without consent.
                \item \textit{Example:} The Cambridge Analytica scandal exploited Facebook users' data for targeted ads without knowledge.
            \end{itemize}
    \end{itemize}
\end{frame}

\begin{frame}[fragile]
    \frametitle{Real-World Case Studies}
    \begin{itemize}
        \item \textbf{Case Study 1: COMPAS Algorithm}
            \begin{itemize}
                \item \textit{Context:} Used in criminal justice to assess reoffending risk.
                \item \textit{Issue:} Exhibited racial bias, unfairly labeling Black defendants as higher risk.
                \item \textit{Lesson:} Necessity for transparent and fair data practices.
            \end{itemize}

        \item \textbf{Case Study 2: GDPR Implementation}
            \begin{itemize}
                \item \textit{Context:} EU's GDPR sets guidelines for data collection.
                \item \textit{Issue:} Requires explicit user consent, emphasizing privacy and data security.
                \item \textit{Lesson:} Regulatory frameworks can enforce ethical standards.
            \end{itemize}
    \end{itemize}
\end{frame}

\begin{frame}[fragile]
    \frametitle{Key Points to Emphasize}
    \begin{itemize}
        \item \textbf{Ethical Data Collection:} Seek informed consent and maintain transparency.
        \item \textbf{Addressing Bias:} Identify and minimize bias in data sources and algorithms.
        \item \textbf{Protecting Privacy:} Implement robust security measures and comply with legal standards.
        \item \textbf{Continuous Evaluation:} Regularly audit data processes for ethical compliance.
    \end{itemize}
\end{frame}

\begin{frame}[fragile]
    \frametitle{Concluding Thoughts}
    Ethics in data usage is a fundamental responsibility that shapes trust in data-driven solutions. As you explore data types, consider how these ethical considerations influence your work and decisions.
\end{frame}

\begin{frame}[fragile]
    \frametitle{Conclusion and Summary}
    \begin{block}{Key Points Discussed on Types of Data}
        1. Definition of Data Types
        \begin{itemize}
            \item \textbf{Quantitative Data}: Numerical data (e.g., height, weight).
            \item \textbf{Qualitative Data}: Categorical data (e.g., colors, names).
        \end{itemize}
    \end{block}
\end{frame}

\begin{frame}[fragile]
    \frametitle{Conclusion and Summary - Part 2}
    \begin{block}{Importance in Machine Learning}
        \begin{itemize}
            \item Data type influences algorithm choice and model performance.
            \item Quantitative data allows statistical analysis; qualitative data often needs techniques like natural language processing.
        \end{itemize}
    \end{block}
    
    \begin{block}{Data Collection and Preparation}
        \begin{itemize}
            \item Quality of data is crucial for model integrity.
            \item Preprocessing (cleaning, normalizing, encoding) is essential.
        \end{itemize}
    \end{block}
\end{frame}

\begin{frame}[fragile]
    \frametitle{Conclusion and Summary - Part 3}
    \begin{block}{Real-World Applications}
        \begin{itemize}
            \item \textbf{Healthcare}: Predicting health risks from quantitative data.
            \item \textbf{Retail}: Sales data informs inventory; customer reviews guide development.
        \end{itemize}
    \end{block}

    \begin{block}{Ethical Considerations}
        \begin{itemize}
            \item Consider biases and privacy in data usage.
        \end{itemize}
    \end{block}
    
    \begin{block}{Engaging Reflection Questions}
        \begin{itemize}
            \item How do different types of data affect model outcomes?
            \item Examples of qualitative data providing insights not captured by quantitative data?
        \end{itemize}
    \end{block}
    
    \begin{block}{Final Thoughts}
        Recognizing the richness of data types is vital for those entering machine learning.
    \end{block}
\end{frame}

\begin{frame}[fragile]
    \frametitle{Discussion Questions - Overview}
    In this interactive session, we aim to explore the various types of data discussed in Chapter 2, emphasizing their practical applications in real-world scenarios. 
    \begin{itemize}
        \item Connect theoretical knowledge to everyday experiences.
        \item Understand the importance of data in decision-making processes.
    \end{itemize}
\end{frame}

\begin{frame}[fragile]
    \frametitle{Discussion Questions - Key Types of Data}
    \begin{enumerate}
        \item \textbf{Quantitative Data}
            \begin{itemize}
                \item Numerical data that can be measured and expressed mathematically. 
                \item \textit{Example:} Height of students, sales figures.
                \item \textit{Discussion Prompt:} How do businesses use sales data to forecast future sales trends?
            \end{itemize}
            
        \item \textbf{Qualitative Data}
            \begin{itemize}
                \item Descriptive data that cannot be easily measured.
                \item \textit{Example:} Student feedback, customer reviews.
                \item \textit{Discussion Prompt:} Why might qualitative data be crucial for improving customer satisfaction?
            \end{itemize}
            
        \item \textbf{Categorical Data}
            \begin{itemize}
                \item Data that can be categorized into groups or labels.
                \item \textit{Example:} Types of fruits (e.g., apple, banana).
                \item \textit{Discussion Prompt:} How could categorical data be used in market segmentation?
            \end{itemize}
            
        \item \textbf{Time-Series Data}
            \begin{itemize}
                \item Data points collected at specific time intervals.
                \item \textit{Example:} Daily temperature readings, stock prices.
                \item \textit{Discussion Prompt:} How can time-series analysis help businesses in supply chain management?
            \end{itemize}
    \end{enumerate}
\end{frame}

\begin{frame}[fragile]
    \frametitle{Discussion Questions - Engaging Questions}
    \begin{itemize}
        \item \textbf{Real-Life Application:} What types of data influenced your decision to purchase a recent product?
        \item \textbf{Prediction Focus:} How can quantitative and qualitative data work together in predicting consumer behavior?
        \item \textbf{Emerging Technologies:} How do companies like Netflix or Spotify use data to enhance user experience? What types of data are involved in their recommendation systems?
        \item \textbf{Personal Experience:} Describe a situation where you used data for a decision. What type of data did you rely on?
    \end{itemize}
\end{frame}


\end{document}