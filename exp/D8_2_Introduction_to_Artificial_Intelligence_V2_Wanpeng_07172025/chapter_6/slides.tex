\documentclass[aspectratio=169]{beamer}

% Theme and Color Setup
\usetheme{Madrid}
\usecolortheme{whale}
\useinnertheme{rectangles}
\useoutertheme{miniframes}

% Additional Packages
\usepackage[utf8]{inputenc}
\usepackage[T1]{fontenc}
\usepackage{graphicx}
\usepackage{booktabs}
\usepackage{listings}
\usepackage{amsmath}
\usepackage{amssymb}
\usepackage{xcolor}
\usepackage{tikz}
\usepackage{pgfplots}
\pgfplotsset{compat=1.18}
\usetikzlibrary{positioning}
\usepackage{hyperref}

% Custom Colors
\definecolor{myblue}{RGB}{31, 73, 125}
\definecolor{mygray}{RGB}{100, 100, 100}
\definecolor{mygreen}{RGB}{0, 128, 0}
\definecolor{myorange}{RGB}{230, 126, 34}
\definecolor{mycodebackground}{RGB}{245, 245, 245}

% Set Theme Colors
\setbeamercolor{structure}{fg=myblue}
\setbeamercolor{frametitle}{fg=white, bg=myblue}
\setbeamercolor{title}{fg=myblue}
\setbeamercolor{section in toc}{fg=myblue}
\setbeamercolor{item projected}{fg=white, bg=myblue}
\setbeamercolor{block title}{bg=myblue!20, fg=myblue}
\setbeamercolor{block body}{bg=myblue!10}
\setbeamercolor{alerted text}{fg=myorange}

% Set Fonts
\setbeamerfont{title}{size=\Large, series=\bfseries}
\setbeamerfont{frametitle}{size=\large, series=\bfseries}
\setbeamerfont{caption}{size=\small}
\setbeamerfont{footnote}{size=\tiny}

% Document Start
\begin{document}

\frame{\titlepage}

\begin{frame}[fragile]
    \frametitle{Introduction to Decision Making - Overview}
    \begin{block}{Overview of Decision-Making in AI}
        Decision making is a critical aspect of artificial intelligence (AI) that enables machines and systems to make choices based on data, objectives, and set criteria. It involves evaluating various possibilities and selecting the best action to achieve specified goals.
    \end{block}
\end{frame}

\begin{frame}[fragile]
    \frametitle{Introduction to Decision Making - Significance}
    \begin{block}{Significance of Decision Making}
        \begin{itemize}
            \item \textbf{Autonomy:} AI systems can operate independently by assessing situations and determining the best course of action without human intervention.
            \item \textbf{Adaptability:} AI algorithms can adapt to changing environments by continuously learning from new information, improving their decision-making over time.
            \item \textbf{Efficiency:} AI-driven decision-making processes can significantly reduce time and resource costs compared to traditional methods.
        \end{itemize}
    \end{block}
\end{frame}

\begin{frame}[fragile]
    \frametitle{Introduction to Decision Making - Applications}
    \begin{block}{Applications in Various Fields}
        \begin{enumerate}
            \item \textbf{Healthcare}
                \begin{itemize}
                    \item Diagnosis: AI systems analyze patient data and symptoms to suggest possible diagnoses.
                    \item Treatment Planning: AI optimizes treatment plans based on patient history and current medical guidelines.
                \end{itemize}
            \item \textbf{Finance}
                \begin{itemize}
                    \item Risk Assessment: AI evaluates credit applications by analyzing financial behaviors and conditions.
                    \item Algorithmic Trading: AI makes real-time trading decisions to maximize profits based on market data.
                \end{itemize}
            \item \textbf{Robotics}
                \begin{itemize}
                    \item Autonomous Navigation: Robots utilize decision-making algorithms to navigate environments safely.
                    \item Task Allocation: AI assigns tasks to robots in warehouses based on efficiency metrics.
                \end{itemize}
            \item \textbf{Gaming}
                \begin{itemize}
                    \item Game AI: AI governs non-player characters (NPCs) to create engaging, realistic gameplay experiences by making strategic decisions.
                \end{itemize}
            \item \textbf{Natural Language Processing}
                \begin{itemize}
                    \item Chatbots: AI chatbots make decisions about responses based on user queries and context.
                \end{itemize}
        \end{enumerate}
    \end{block}
\end{frame}

\begin{frame}[fragile]
    \frametitle{Key Points and Code Snippet}
    \begin{block}{Key Points to Emphasize}
        \begin{itemize}
            \item The importance of decision-making in enabling AI systems to function autonomously and efficiently.
            \item Diverse applications demonstrate the versatility and impact of effective decision-making algorithms across different domains.
        \end{itemize}
    \end{block}

    \begin{block}{Basic Decision-Making Algorithm}
        \begin{lstlisting}[language=Python]
function makeDecision(input_data):
    options = evaluateOptions(input_data)
    best_choice = selectBestOption(options)
    return best_choice
        \end{lstlisting}
    \end{block}
\end{frame}

\begin{frame}[fragile]
    \frametitle{Conclusion}
    \begin{block}{Conclusion}
        By understanding the fundamental principles and significance of decision-making in AI, we can appreciate how these technologies empower systems to function intelligently and autonomously across various industries.
    \end{block}
\end{frame}

\begin{frame}[fragile]{Markov Decision Processes (MDPs) - Introduction}
    \begin{block}{What are MDPs?}
        Markov Decision Processes (MDPs) provide a mathematical framework for modeling decision-making in situations where outcomes are partly random and partly under the control of a decision-maker. MDPs are widely used in areas such as robotics, economics, and artificial intelligence, particularly in reinforcement learning.
    \end{block}
\end{frame}

\begin{frame}[fragile]{Markov Decision Processes (MDPs) - Key Components}
    \begin{enumerate}
        \item \textbf{States (S)}:
            \begin{itemize}
                \item \textbf{Definition:} A set of all possible situations or configurations that the decision-maker can encounter.
                \item \textbf{Example:} In a chess game, each arrangement of pieces on the board represents a different state.
            \end{itemize}
            
        \item \textbf{Actions (A)}:
            \begin{itemize}
                \item \textbf{Definition:} A set of all possible actions that can be taken by the decision-maker.
                \item \textbf{Example:} In chess, possible actions include moving a piece or capturing an opponent's piece.
            \end{itemize}

        \item \textbf{Transition Probabilities (P)}:
            \begin{itemize}
                \item \textbf{Definition:} A function that describes the likelihood of transitioning from one state to another given a specific action.
                \item \textbf{Notation:} \( P(s'|s, a) \), probability of moving to state \( s' \) from state \( s \) after taking action \( a \).
                \item \textbf{Example:} Moving a piece in chess has transition probabilities based on the opponent's responses.
            \end{itemize}
    \end{enumerate}
\end{frame}

\begin{frame}[fragile]{Markov Decision Processes (MDPs) - Rewards and Policies}
    \begin{enumerate}
        \setcounter{enumi}{3}
        \item \textbf{Rewards (R)}:
            \begin{itemize}
                \item \textbf{Definition:} A function that assigns a numerical value (reward) based on the state-action pair.
                \item \textbf{Notation:} \( R(s, a) \), expected reward after taking action \( a \) in state \( s \).
                \item \textbf{Example:} Capturing a piece may provide a positive reward, while losing a piece yields a negative reward.
            \end{itemize}

        \item \textbf{Policies (π)}:
            \begin{itemize}
                \item \textbf{Definition:} A mapping from states to actions defining the decision-maker's behavior.
                \item \textbf{Notation:} \( π(s) \) denotes the action chosen in state \( s \) according to policy \( π \).
                \item \textbf{Example:} A chess strategy that favors aggressive moves over defensive ones.
            \end{itemize}
    \end{enumerate}
\end{frame}

\begin{frame}
  \frametitle{Key Components of MDPs}
  \begin{block}{Introduction to MDPs}
    Markov Decision Processes (MDPs) provide a mathematical framework for modeling decision-making where outcomes are partly random and partly under the control of a decision maker. Understanding the key components of MDPs is essential to grasp how decisions impact states and rewards.
  \end{block}
\end{frame}

\begin{frame}
  \frametitle{Key Components of MDPs - States}
  \begin{itemize}
    \item \textbf{States (S)}:
      \begin{itemize}
        \item \textbf{Definition}: States represent the various situations or configurations in which an agent can find itself within the environment.
        \item \textbf{Example}: In a robot navigation scenario, states could be different locations on a grid (e.g., position (1,1), (1,2)).
        \item \textbf{Key Point}: Each state contains sufficient information to determine the potential action decisions.
      \end{itemize}
  \end{itemize}
\end{frame}

\begin{frame}
  \frametitle{Key Components of MDPs - Actions and Transition Probabilities}
  \begin{itemize}
    \item \textbf{Actions (A)}:
      \begin{itemize}
        \item \textbf{Definition}: Actions are the choices available to an agent that can change its state.
        \item \textbf{Example}: For our robot, possible actions might include moving up, down, left, or right.
        \item \textbf{Key Point}: The set of actions can vary depending on the current state—fewer options at edges of the grid.
      \end{itemize}
    
    \item \textbf{Transition Probabilities (T)}:
      \begin{itemize}
        \item \textbf{Definition}: Transition probabilities quantify the likelihood of moving from one state to another after performing a specific action.
        \item \textbf{Mathematically Represented}:
        \begin{equation}
        T(s, a, s') = P(s' \mid s, a)
        \end{equation}
        \item \textbf{Example}: For the robot moving right from (1,1), 70% chance it goes to (1,2), 30% it stays at (1,1).
        \item \textbf{Key Point}: These probabilities model uncertainty in action outcomes.
      \end{itemize}
  \end{itemize}
\end{frame}

\begin{frame}
  \frametitle{Key Components of MDPs - Reward Function}
  \begin{itemize}
    \item \textbf{Reward Function (R)}:
      \begin{itemize}
        \item \textbf{Definition}: Assigns a numerical reward for transitioning from one state to another via an action.
        \item \textbf{Mathematically Represented}:
        \begin{equation}
        R(s, a, s')
        \end{equation}
        \item \textbf{Example}: Reaching (2,2) gives +10 reward; colliding with a wall gives -1.
        \item \textbf{Key Point}: The goal of the agent is to maximize cumulative rewards over time.
      \end{itemize}
  \end{itemize}
\end{frame}

\begin{frame}
  \frametitle{Summary and Code Snippet}
  \begin{itemize}
    \item MDPs consist of states, actions, transition probabilities, and reward functions, defining the decision-making process under uncertainty.
    \item Understanding these components is critical for advancing to topics like value functions in reinforcement learning.
  \end{itemize}

  \begin{block}{Code Snippet for MDP Setup}
  \begin{lstlisting}[language=Python]
class MDP:
    def __init__(self, states, actions, transition_probabilities, rewards):
        self.states = states
        self.actions = actions
        self.transition_probabilities = transition_probabilities
        self.rewards = rewards
        
    def get_reward(self, state, action, next_state):
        return self.rewards[state][action][next_state]

# Example Initialization
states = ['S1', 'S2', 'S3']
actions = ['left', 'right']
transition_probabilities = {...}  # Define transition probabilities
rewards = {...}  # Define rewards
  \end{lstlisting}
  \end{block}
\end{frame}

\begin{frame}[fragile]
    \frametitle{Value Functions - Definition and Significance}
    \begin{block}{Definition}
        In the context of Markov Decision Processes (MDPs), a \textbf{value function} evaluates the desirability of states or actions under a specific policy.
    \end{block}
    \begin{block}{Significance}
        - It guides decision-making in reinforcement learning by indicating how good it is to be in a particular state or to perform an action.
    \end{block}
\end{frame}

\begin{frame}[fragile]
    \frametitle{Value Functions - Types}
    \begin{enumerate}
        \item \textbf{State-Value Function ($V$)}:
        \begin{itemize}
            \item \textbf{Definition}: $V(s)$ represents expected future rewards from state $s$ under policy $\pi$.
            \item \textbf{Formula}:
            \begin{equation}
                V_{\pi}(s) = \mathbb{E} [R_t \mid S_t = s, \pi] = \sum_{a \in A} \pi(a|s) \sum_{s'} P(s'|s,a) \left[R(s,a,s') + \gamma V_{\pi}(s')\right]
            \end{equation}
            \item \textbf{Key Concept}: Evaluates favorability of being in state $s$ under policy $\pi$.
        \end{itemize}
        
        \item \textbf{Action-Value Function ($Q$)}:
        \begin{itemize}
            \item \textbf{Definition}: $Q(s, a)$ represents expected return when taking action $a$ in state $s$ and following policy $\pi$.
            \item \textbf{Formula}:
            \begin{equation}
                Q_{\pi}(s, a) = \mathbb{E} [R_t \mid S_t = s, A_t = a, \pi] = \sum_{s'} P(s'|s,a) \left[R(s,a,s') + \gamma V_{\pi}(s')\right]
            \end{equation}
            \item \textbf{Key Concept}: Evaluates quality of actions $a$ in states $s$ for maximizing rewards.
        \end{itemize}
    \end{enumerate}
\end{frame}

\begin{frame}[fragile]
    \frametitle{Value Functions - Significance and Example}
    \begin{block}{Significance of Value Functions}
        - \textbf{Guiding Policy Improvement}: Refines decision-making by comparing values of states and actions.
        - \textbf{Foundation for Algorithms}: Used in Q-learning, SARSA, etc., to update policies based on experiences.
    \end{block}

    \begin{block}{Example Illustration}
        Consider a grid world MDP with actions: UP, DOWN, LEFT, RIGHT.
        \begin{itemize}
            \item If in state $s_1$, moving to state $s_2$ gives $V(s_1) = 3$, reward = $5$, and $V(s_2) = 6$.
            \item The movement from $s_1$ to $s_2$ is favorable since $V(s_1) + \text{reward} + V(s_2) = 3 + 5 + 6$.
        \end{itemize}
    \end{block}
\end{frame}

\begin{frame}[fragile]
    \frametitle{Bellman Equations - Introduction}
    \begin{block}{Overview}
        The Bellman equations are fundamental for solving Markov Decision Processes (MDPs) and are essential in reinforcement learning. 
        They provide a recursive way to compute the value of states (state-value functions) and actions (action-value functions), enabling the discovery of optimal policies.
    \end{block}
\end{frame}

\begin{frame}[fragile]
    \frametitle{Bellman Equations - Key Concepts}
    \begin{itemize}
        \item \textbf{Markov Decision Processes (MDPs)}:
            \begin{itemize}
                \item Defined by states \( S \), actions \( A \), transition probabilities \( P(s'|s, a) \), rewards \( R(s, a) \), and discount factor \( \gamma \in [0, 1] \)
                \item Objective: Find a policy \( \pi \) that maximizes expected cumulative reward
            \end{itemize}
        \item \textbf{Value Functions}:
            \begin{itemize}
                \item \textbf{State-Value Function \( V(s) \)}: The expected return starting from state \( s \) following policy \( \pi \)
                \item \textbf{Action-Value Function \( Q(s,a) \)}: The expected return starting from state \( s \), taking action \( a \), then following policy \( \pi \)
            \end{itemize}
    \end{itemize}
\end{frame}

\begin{frame}[fragile]
    \frametitle{Bellman Equations - Mathematical Forms}
    \begin{block}{Bellman Equation for State-Value Function}
        \begin{equation}
            V(s) = \sum_{a \in A} \pi(a|s) \sum_{s' \in S} P(s'|s, a) \left( R(s, a, s') + \gamma V(s') \right)
        \end{equation}
        \begin{itemize}
            \item Describes the value of state \( s \) in terms of expected future rewards
        \end{itemize}
    \end{block}

    \begin{block}{Bellman Equation for Action-Value Function}
        \begin{equation}
            Q(s, a) = \sum_{s' \in S} P(s'|s, a) \left( R(s, a, s') + \gamma \sum_{a' \in A} \pi(a'|s') Q(s', a') \right)
        \end{equation}
        \begin{itemize}
            \item Estimates the quality of taking a specific action in a given state
        \end{itemize}
    \end{block}
\end{frame}

\begin{frame}[fragile]
    \frametitle{Importance and Example of Bellman Equations}
    \begin{itemize}
        \item \textbf{Importance of Bellman Equations}:
            \begin{itemize}
                \item Recursive structure enables breaking down decisions into simpler subproblems
                \item Solving these equations provides optimal policies maximizing expected returns
            \end{itemize}

        \item \textbf{Example}:
            \begin{itemize}
                \item An agent in a grid world with states as positions and actions as movements (up, down, left, right)
                \item Policy may favor moving right, influencing future state values and updates to the value function reflecting agent behavior
            \end{itemize}

        \item \textbf{Key Points to Emphasize}:
            \begin{itemize}
                \item Systematic computation of value functions in MDPs
                \item Foundation for many reinforcement learning algorithms
                \item Understanding these equations is critical for policy evaluation and improvement
            \end{itemize}
    \end{itemize}
\end{frame}

\begin{frame}[fragile]
    \frametitle{Closing Remarks}
    \begin{block}{Looking Ahead}
        As we transition to the next slide on policies in MDPs, remember:
        \begin{itemize}
            \item The Bellman equations described here guide our understanding of optimal actions based on learned experiences in various states.
        \end{itemize}
    \end{block}
\end{frame}

\begin{frame}[fragile]
    \frametitle{Policies in MDPs - Overview}
    In Markov Decision Processes (MDPs), decisions are guided by policies. Understanding the distinction between deterministic and stochastic policies is crucial as it directly influences the decision-making and outcomes of agents in various environments.
\end{frame}

\begin{frame}[fragile]
    \frametitle{What is a Policy?}
    A policy is a strategy that maps states to actions and represents how an agent decides based on the current state of the environment.
    
    \begin{block}{Mathematically:}
        A policy \( \pi \) can be defined as:
        \begin{itemize}
            \item \textbf{Deterministic Policy:} \( \pi: S \rightarrow A \)
            \item \textbf{Stochastic Policy:} \( \pi: S \rightarrow \text{Probabilities over } A \)
        \end{itemize}
    \end{block}
\end{frame}

\begin{frame}[fragile]
    \frametitle{Types of Policies}
    \begin{enumerate}
        \item \textbf{Deterministic Policies:}
        \begin{itemize}
            \item Specifies a single action for each state.
            \item \textit{Example:} In chess, the policy indicates the exact move based on the board position.
        \end{itemize}

        \item \textbf{Stochastic Policies:}
        \begin{itemize}
            \item Assigns a probability distribution over possible actions for each state.
            \item \textit{Example:} A robot might move left with a probability of 0.7 and right with a probability of 0.3 from the same state.
        \end{itemize}
    \end{enumerate}
\end{frame}

\begin{frame}[fragile]
    \frametitle{Pros and Cons of Policies}
    \begin{block}{Deterministic Policies}
        \begin{itemize}
            \item \textbf{Pros:}
            \begin{itemize}
                \item Predictable and easy to analyze.
                \item Simplicity: each state leads to a definitive action.
            \end{itemize}
            \item \textbf{Cons:}
            \begin{itemize}
                \item Limited flexibility in uncertain environments.
            \end{itemize}
        \end{itemize}
    \end{block}
    
    \begin{block}{Stochastic Policies}
        \begin{itemize}
            \item \textbf{Pros:}
            \begin{itemize}
                \item Adaptable to uncertainty and variability.
                \item Facilitates exploration, which may lead to better long-term policies.
            \end{itemize}
            \item \textbf{Cons:}
            \begin{itemize}
                \item More complex and potentially harder to analyze.
            \end{itemize}
        \end{itemize}
    \end{block}
\end{frame}

\begin{frame}[fragile]
    \frametitle{Impact on Decision-Making}
    \begin{itemize}
        \item \textbf{Optimality:} The choice of policy significantly influences the value (returns) an agent can achieve.
        \item \textbf{Exploration vs. Exploitation:}
        \begin{itemize}
            \item Stochastic policies support exploration.
            \item Deterministic policies focus on exploitation of known strategies.
        \end{itemize}
    \end{itemize}
    
    \begin{block}{Example Scenario}
        In a grid world with state \( S_1 \):
        \begin{itemize}
            \item \textbf{Deterministic Policy:} Always moves right to \( S_2 \).
            \item \textbf{Stochastic Policy:} Moves to \( S_2 \) with a probability of 0.8, and to \( S_3 \) with a probability of 0.2.
        \end{itemize}
    \end{block}
\end{frame}

\begin{frame}[fragile]
    \frametitle{Key Points and Conclusion}
    \begin{itemize}
        \item Policies guide an agent's behavior in MDPs.
        \item Deterministic policies offer clarity; stochastic policies provide flexibility in uncertain environments.
        \item The choice of policy depends on task and environment requirements.
    \end{itemize}
    
    \textbf{Conclusion:} Understanding the differences between deterministic and stochastic policies is essential for effective decision-making in MDPs.

    \textbf{Next Steps:} Prepare for the slide on \textit{Solving MDPs}, which will explore techniques such as value iteration and policy iteration.
\end{frame}

\begin{frame}[fragile]
    \frametitle{Solving MDPs - Overview}
    \begin{block}{Markov Decision Processes (MDPs)}
        MDPs provide a mathematical framework for decision-making where outcomes are partly random and partly under the control of a decision-maker. Solving an MDP involves finding an optimal policy that maximizes the expected cumulative reward over time.
    \end{block}

    \begin{itemize}
        \item \textbf{States (S):} Possible situations an agent may face.
        \item \textbf{Actions (A):} Choices the agent can make in each state.
        \item \textbf{Transitions (P):} Probabilities of moving from one state to another, given an action.
        \item \textbf{Rewards (R):} Immediate gain after transitioning from one state to another.
    \end{itemize}
\end{frame}

\begin{frame}[fragile]
    \frametitle{Solving MDPs - Dynamic Programming}
    \begin{block}{Methods for Solving MDPs}
        \begin{itemize}
            \item \textbf{Dynamic Programming:} Algorithms that simplify complex problems by breaking them into subproblems, requiring complete knowledge of the MDP.
        \end{itemize}
    \end{block}

    \begin{enumerate}
        \item \textbf{Value Iteration:}
            \begin{itemize}
                \item \textbf{Concept:} Iteratively updating state values until convergence.
                \item \textbf{Procedure:}
                    \begin{enumerate}
                        \item Initialize values \( V(s) \) (e.g., \( V(s)=0 \)).
                        \item Update values based on the Bellman equation:
                        \begin{equation}
                        V_{new}(s) = \max_{a} \left( R(s, a) + \sum_{s'} P(s'|s, a) V(s') \right)
                        \end{equation}
                        \item Repeat until values stabilize.
                    \end{enumerate}
            \end{itemize}
        \item \textbf{Policy Iteration:}
            \begin{itemize}
                \item Concept: Alternates between policy evaluation and improvement.
            \end{itemize}
    \end{enumerate}
\end{frame}

\begin{frame}[fragile]
    \frametitle{Solving MDPs - Policy Iteration}
    \begin{block}{Policy Iteration Procedure}
        \begin{enumerate}
            \item Start with an arbitrary policy \( \pi \).
            \item \textbf{Policy Evaluation:} Calculate the value function using:
            \begin{equation}
            V^\pi(s) = R(s, \pi(s)) + \sum_{s'} P(s'|s, \pi(s)) V^\pi(s')
            \end{equation}
            \item \textbf{Policy Improvement:} Update by choosing actions that maximize expected value:
            \begin{equation}
            \pi_{new}(s) = \arg\max_{a} \left( R(s, a) + \sum_{s'} P(s'|s, a) V^\pi(s') \right)
            \end{equation}
            \item Repeat until the policy stops changing.
        \end{enumerate}
    \end{block}
    
    \begin{itemize}
        \item \textbf{Conclusion:} MDP solutions via dynamic programming are crucial for decision-making in various fields including AI and robotics.
        \item \textbf{Next Topic:} Transition to Reinforcement Learning and its connection to MDPs.
    \end{itemize}
\end{frame}

\begin{frame}[fragile]
    \frametitle{Understanding Reinforcement Learning (RL)}
    \begin{block}{Definition}
        Reinforcement Learning (RL) is a type of machine learning where an agent learns to make decisions by interacting with an environment to maximize cumulative rewards. Unlike supervised learning, RL relies on trial and error based on feedback obtained from actions.
    \end{block}
\end{frame}

\begin{frame}[fragile]
    \frametitle{Relation to Markov Decision Processes (MDPs)}
    \begin{itemize}
        \item MDPs provide a mathematical framework for modeling decision-making in environments with randomness and control.
        \item An MDP is defined by:
        \begin{itemize}
            \item \textbf{States (S)}: All possible situations the agent can be in.
            \item \textbf{Actions (A)}: Choices available to the agent at each state.
            \item \textbf{Transition Probability (P)}: Probability of moving from one state to another given an action.
            \item \textbf{Rewards (R)}: Immediate feedback received after taking an action.
            \item \textbf{Discount Factor ($\gamma$)}: Represents the importance of future rewards (0 < $\gamma$ < 1).
        \end{itemize}
    \end{itemize}
    \begin{block}{Connection to RL}
        RL is used to solve MDPs by improving a policy based on rewards received from actions, without needing full information on transition probabilities.
    \end{block}
\end{frame}

\begin{frame}[fragile]
    \frametitle{Key Concepts in Reinforcement Learning}
    \begin{enumerate}
        \item \textbf{Agent}: Learner or decision maker interacting with the environment.
        \item \textbf{Environment}: Everything the agent interacts with (e.g., a video game).
        \item \textbf{Rewards}: Numerical values guiding the agent's learning process.
        \item \textbf{Exploration vs. Exploitation}:
        \begin{itemize}
            \item \textbf{Exploration}: Trying new actions.
            \item \textbf{Exploitation}: Leveraging known actions for high rewards.
        \end{itemize}
    \end{enumerate}
\end{frame}

\begin{frame}[fragile]
    \frametitle{Example Illustration}
    \begin{block}{Example Scenario}
        A robot learning to navigate a maze:
        \begin{itemize}
            \item \textbf{States (S)}: Positions in the maze.
            \item \textbf{Actions (A)}: Move forward, turn left, turn right.
            \item \textbf{Rewards (R)}: +10 for reaching the exit, -1 for hitting walls.
        \end{itemize}
        Through trials, the robot improves its path based on received rewards.
    \end{block}
\end{frame}

\begin{frame}[fragile]
    \frametitle{Key Points to Emphasize}
    \begin{itemize}
        \item RL differs from supervised learning, focusing on feedback rather than labeled data.
        \item The relationship between RL and MDPs is foundational for solving MDPs with incomplete information.
        \item Understanding exploration vs. exploitation is crucial for effective RL application.
    \end{itemize}
\end{frame}

\begin{frame}[fragile]
  \frametitle{Key Concepts in Reinforcement Learning - Overview}
  \begin{itemize}
    \item Reinforcement Learning (RL) involves agents interacting with environments.
    \item Key components:
      \begin{itemize}
        \item Agents
        \item Environments
        \item Rewards
        \item Exploration vs. Exploitation
      \end{itemize}
    \item Understanding these concepts is critical for grasping RL techniques.
  \end{itemize}
\end{frame}

\begin{frame}[fragile]
  \frametitle{Key Concepts in Reinforcement Learning - Agents and Environments}
  \textbf{1. Agents}
  \begin{block}{Definition}
    An agent is an entity that makes decisions by taking actions in an environment to achieve specific goals.
  \end{block}
  \begin{itemize}
    \item \textbf{Example}: In a video game, the player is the agent, making choices to win the game.
    \item \textbf{Key Point}: Agents learn from their experiences through interactions with the environment.
  \end{itemize}

  \textbf{2. Environments}
  \begin{block}{Definition}
    The environment is everything that the agent interacts with, encompassing all possible states and scenarios.
  \end{block}
  \begin{itemize}
    \item \textbf{Example}: In chess, the chessboard and pieces form the environment where the agent operates.
    \item \textbf{Key Point}: Agents navigate to maximize cumulative rewards.
  \end{itemize}
\end{frame}

\begin{frame}[fragile]
  \frametitle{Key Concepts in Reinforcement Learning - Rewards and Exploration vs. Exploitation}
  \textbf{3. Rewards}
  \begin{block}{Definition}
    Rewards are feedback signals received by the agent after taking actions, indicating success in achieving goals.
  \end{block}
  \begin{itemize}
    \item \textbf{Example}: A self-driving car receives positive rewards for reaching a destination and negative rewards for violations.
    \item \textbf{Key Point}: RL aims to maximize cumulative rewards over time.
  \end{itemize}

  \textbf{4. Exploration vs. Exploitation}
  \begin{block}{Definition}
    This concept describes the trade-off between two strategies:
  \end{block}

  \begin{itemize}
    \item \textbf{Exploration}: Trying new actions for better understanding.
      \begin{itemize}
        \item \textbf{Example}: A robot learns its layout by random movement in a maze.
      \end{itemize}
      
    \item \textbf{Exploitation}: Choosing the best-known actions for highest reward.
      \begin{itemize}
        \item \textbf{Example}: A poker player betting on the most successful hand.
      \end{itemize}
    
    \item \textbf{Key Point}: Balancing exploration and exploitation is crucial for effective learning.
  \end{itemize}
\end{frame}

\begin{frame}[fragile]
  \frametitle{Summary and Formulas}
  \textbf{Summary}
  \begin{itemize}
    \item Agents act in environments guided by rewards.
    \item Key concepts include the role of each component.
    \item The balance between exploration and exploitation is essential.
  \end{itemize}

  \textbf{Cumulative Reward Formula}
  \begin{equation}
    R = r_1 + r_2 + r_3 + ... + r_n
  \end{equation}
  where \( r_i \) is the reward received at time \( i \).

  \textbf{Code Snippet - Simple Reward Structure}
  \begin{lstlisting}[language=Python]
class Agent:
    def __init__(self):
        self.total_reward = 0
        
    def receive_reward(self, reward):
        self.total_reward += reward
        return self.total_reward
  \end{lstlisting}

  Understanding these concepts provides a foundation for exploring reinforcement learning techniques like Q-Learning.
\end{frame}

\begin{frame}[fragile]
  \frametitle{Q-Learning Overview}
  \begin{block}{What is Q-Learning?}
    Q-Learning is a model-free reinforcement learning algorithm 
    used to find an optimal action-selection policy for an agent 
    interacting with an environment.
  \end{block}
  
  \begin{itemize}
    \item Enables the agent to learn how to act optimally in any given state.
    \item Does not require complete understanding of the environment.
  \end{itemize}
\end{frame}

\begin{frame}[fragile]
  \frametitle{Key Concepts of Q-Learning}
  \begin{itemize}
    \item **Agent:** The learner or decision-maker in the environment.
    \item **Environment:** The setting offering states and rewards.
    \item **State (s):** A specific situation at a particular time.
    \item **Action (a):** Choices available to the agent.
    \item **Reward (r):** Feedback based on the action taken by the agent.
  \end{itemize}
\end{frame}

\begin{frame}[fragile]
  \frametitle{How Q-Learning Works}
  \begin{enumerate}
    \item **Q-Values**: Core of Q-Learning is the Q-value, \( Q(s, a) \), representing expected future rewards.
    \item **Updating Q-Values**: 
      \[
      Q(s, a) \leftarrow Q(s, a) + \alpha \left(R + \gamma \max_a Q(s', a) - Q(s, a)\right)
      \]
      Where:
      \begin{itemize}
        \item \( \alpha \): Learning rate (0 < \( \alpha \) ≤ 1)
        \item \( R \): Reward after transitioning from s to s' via action a.
        \item \( \gamma \): Discount factor (0 ≤ \( \gamma \) < 1)
        \item \( s' \): New state after taking action a.
      \end{itemize}
    \item **Exploration vs. Exploitation**: Balancing new actions vs. best-known actions (e.g., using ε-greedy strategy).
  \end{enumerate}
\end{frame}

\begin{frame}[fragile]
  \frametitle{Applications and Conclusion}
  \begin{itemize}
    \item **Applications of Q-Learning:**
    \begin{itemize}
      \item Game Playing (e.g., chess, Go)
      \item Robotics (optimal paths in navigation)
      \item Resource Management (cloud computing)
    \end{itemize}
    \item **Key Points:**
    \begin{itemize}
      \item Model-Free Learning: Q-Learning learns from experience.
      \item Convergence: Given sufficient exploration, Q-values converge to optimal policy.
      \item Simplicity: Straightforward to implement and effective in many scenarios.
    \end{itemize}
  \end{itemize}
\end{frame}

\begin{frame}[fragile]
  \frametitle{Example Code Snippet}
  \begin{lstlisting}[language=Python]
initialize Q(s, a) arbitrarily
for each episode:
    s = initial state
    while not terminal state:
        a = choose action from s using policy derived from Q (e.g., ε-greedy)
        take action a, observe reward R and next state s'
        Q(s, a) = Q(s, a) + α * (R + γ * max Q(s', a) - Q(s, a))
        s = s'  # update state
  \end{lstlisting}
\end{frame}

\begin{frame}[fragile]
    \frametitle{Deep Reinforcement Learning - Introduction}
    \begin{block}{What is DRL?}
        Deep Reinforcement Learning (DRL) is an advanced machine learning paradigm that integrates reinforcement learning (RL) principles with deep learning techniques. It empowers agents to learn optimal behaviors in high-dimensional environments using neural networks.
    \end{block}
\end{frame}

\begin{frame}[fragile]
    \frametitle{Key Concepts in Deep Reinforcement Learning}
    \begin{itemize}
        \item \textbf{Reinforcement Learning (RL)}: 
        \begin{itemize}
            \item Agents learn by interacting with their environment, receiving rewards or penalties.
        \end{itemize}
        \item \textbf{Deep Learning}: 
        \begin{itemize}
            \item A subset of machine learning that uses multi-layered neural networks for feature representation.
        \end{itemize}
        \item \textbf{Combining RL with Deep Learning}:
        \begin{enumerate}
            \item Function Approximation
            \item Policy-Based vs. Value-Based methods
        \end{enumerate}
    \end{itemize}
\end{frame}

\begin{frame}[fragile]
    \frametitle{Deep Q-Network (DQN) Example}
    \begin{block}{Architecture}
        A neural network processes raw input states (e.g., frames from a game) and outputs Q-values for possible actions.
    \end{block}
    
    \begin{block}{Training}
        Updates are made using the Bellman equation:
        \begin{equation}
        Q(s, a) \leftarrow Q(s, a) + \alpha \left[ r + \gamma \max_{a'} Q(s', a') - Q(s, a) \right]
        \end{equation}
        where:
        \begin{itemize}
            \item $\alpha$: Learning Rate
            \item $\gamma$: Discount Factor
            \item $r$: Reward
            \item $s$: Current State
            \item $a$: Action Taken
            \item $s'$: Resultant State
        \end{itemize}
    \end{block}
\end{frame}

\begin{frame}[fragile]
    \frametitle{Applications of MDPs and Reinforcement Learning}
    \begin{block}{Key Concepts}
        \begin{itemize}
            \item \textbf{Markov Decision Processes (MDPs)}: Provide a mathematical framework for modeling decision-making, where an agent maximizes cumulative rewards.
            \item \textbf{Reinforcement Learning (RL)}: A type of machine learning where an agent learns decisions through trial and error, typically using MDPs.
        \end{itemize}
    \end{block}
\end{frame}

\begin{frame}[fragile]
    \frametitle{Real-World Applications of MDPs and RL}
    \begin{enumerate}
        \item \textbf{Robotics}
            \begin{itemize}
                \item \textit{Autonomous Navigation:} e.g., a robot vacuum learning optimal paths.
                \item \textit{Manipulation Tasks:} e.g., a robotic arm learning to pick and place objects.
            \end{itemize}
        \item \textbf{Gaming}
            \begin{itemize}
                \item \textit{Game AI Development:} e.g., OpenAI’s Dota 2 bot enhancing strategic decisions.
                \item \textit{Procedural Content Generation:} e.g., dynamic game level adjustment based on player performance.
            \end{itemize}
        \item \textbf{Autonomous Systems}
            \begin{itemize}
                \item \textit{Self-Driving Cars:} e.g., real-time decision-making for traffic navigation.
                \item \textit{Drones:} e.g., optimizing flight paths based on conditions.
            \end{itemize}
    \end{enumerate}
\end{frame}

\begin{frame}[fragile]
    \frametitle{Key Points and Visuals}
    \begin{block}{Key Points}
        \begin{itemize}
            \item Versatility of MDPs and RL in solving dynamic decision-making problems.
            \item Importance of balance between exploration and exploitation in RL.
            \item Advancements in algorithms enhancing RL effectiveness.
        \end{itemize}
    \end{block}
    
    \begin{block}{Visual Elements}
        Suggest including:
        \begin{itemize}
            \item A flowchart of the MDP framework.
            \item A code snippet using OpenAI’s Gym for an MDP problem.
        \end{itemize}
    \end{block}
\end{frame}

\begin{frame}[fragile]
    \frametitle{Sample Code for RL with MDP}
    \begin{lstlisting}[language=python]
import gym

env = gym.make("CartPole-v1")
state = env.reset()

for _ in range(1000):
    env.render()
    action = env.action_space.sample()  # Sample a random action
    state, reward, done, _ = env.step(action)
    if done:
        state = env.reset()

env.close()
    \end{lstlisting}
\end{frame}

\begin{frame}[fragile]
    \frametitle{Challenges in Reinforcement Learning - Overview}
    \begin{block}{Overview of Challenges}
        Reinforcement Learning (RL) presents several unique challenges that complicate the decision-making process. 
        Understanding these challenges is crucial for developing effective RL algorithms.
    \end{block}
\end{frame}

\begin{frame}[fragile]
    \frametitle{Challenges in Reinforcement Learning - Sample Inefficiency}
    \begin{itemize}
        \item \textbf{Definition}: Sample inefficiency refers to the need for a large amount of data or interactions with the environment to learn effective policies.
        \item \textbf{Explanation}: 
        Unlike supervised learning, where models learn from a fixed dataset, RL agents (agents) must explore their environments, which can lead to slow learning progress.
        \item \textbf{Example}: In a complex game, an RL agent might require thousands of game plays to develop strategies that a human could learn in a few attempts.
    \end{itemize}
    
    \begin{block}{Key Point}
        Efficient algorithms, such as experience replay or transfer learning, help reduce sample inefficiency by utilizing past experiences.
    \end{block}
\end{frame}

\begin{frame}[fragile]
    \frametitle{Challenges in Reinforcement Learning - Exploration vs. Exploitation}
    \begin{itemize}
        \item \textbf{Definition}: The exploration-exploitation dilemma is the trade-off between exploring new actions (exploration) and leveraging known actions that yield high rewards (exploitation).
        \item \textbf{Explanation}: An agent must balance trying new strategies to discover their potential against using strategies that are already known to work well.
    \end{itemize}
    
    \begin{block}{Illustration}
        \begin{itemize}
            \item \textbf{Exploration}: Trying a new path in a maze.
            \item \textbf{Exploitation}: Repeating the path that previously led to success.
        \end{itemize}
    \end{block}
    
    \begin{itemize}
        \item \textbf{Example}: An agent in a maze must decide whether to explore unknown paths or continue down a known path that previously resulted in success.
    \end{itemize}
    
    \begin{block}{Key Point}
        Techniques like $\epsilon$-greedy and Upper Confidence Bound (UCB) are often employed to manage this trade-off efficiently.
    \end{block}
\end{frame}

\begin{frame}[fragile]
    \frametitle{Challenges in Reinforcement Learning - Stability of Learning}
    \begin{itemize}
        \item \textbf{Definition}: Stability refers to the ability of an RL algorithm to converge to a stable solution or policy during training.
        \item \textbf{Explanation}: Many RL algorithms, especially those that utilize deep neural networks (deep RL), can suffer from instability and divergence due to the complex interplay between changing policies and value estimates.
        \item \textbf{Example}: If an agent modifies its policy too drastically based on recent experiences, it may start to oscillate between different policies without converging.
    \end{itemize}

    \begin{block}{Key Points}
        \begin{itemize}
            \item Techniques like target networks and experience replay can enhance stability.
            \item Regularization methods can also help in managing this challenge.
        \end{itemize}
    \end{block}
\end{frame}

\begin{frame}[fragile]
    \frametitle{Challenges in Reinforcement Learning - Conclusion}
    \begin{itemize}
        \item The challenges of sample inefficiency, exploration issues, and stability are integral to the development of robust Reinforcement Learning algorithms.
        \item Understanding these issues allows for the design of better learning strategies and enhancements to existing algorithms.
    \end{itemize}
    
    \begin{block}{Additional Insights}
        \begin{itemize}
            \item Implementing advanced algorithms like Proximal Policy Optimization (PPO) and Trust Region Policy Optimization (TRPO) offers solutions to many of the discussed challenges.
            \item Remaining informed about these challenges prepares us for the subsequent exploration of ethical considerations in the context of Reinforcement Learning.
        \end{itemize}
    \end{block}
\end{frame}

\begin{frame}[fragile]
    \frametitle{Ethical Considerations - Overview}
    As we explore the deployment of Markov Decision Processes (MDPs) and Reinforcement Learning (RL), it is crucial to address the ethical implications that arise. 
    \begin{itemize}
        \item The application of these techniques can significantly impact individuals and society.
        \item Key aspects to analyze include:
        \begin{itemize}
            \item \textbf{Bias}
            \item \textbf{Fairness}
            \item \textbf{Decision Transparency}
        \end{itemize}
    \end{itemize}
\end{frame}

\begin{frame}[fragile]
    \frametitle{Ethical Considerations - Bias}
    \begin{block}{1. Bias}
        \begin{itemize}
            \item \textbf{Definition:} Bias in AI models refers to systematic favoritism, where certain groups are either advantaged or disadvantaged based on attributes like race, gender, or socio-economic status.
            \item \textbf{Example:} 
                \begin{itemize}
                    \item Consider an RL-based hiring system. If the data used for training contains historical biases (e.g., fewer female candidates in tech roles), the model may perpetuate and amplify this bias in its decision-making process.
                \end{itemize}
            \item \textbf{Key Point:} Ensure diverse and representative training data to minimize bias during the training phase.
        \end{itemize}
    \end{block}
\end{frame}

\begin{frame}[fragile]
    \frametitle{Ethical Considerations - Fairness and Decision Transparency}
    \begin{block}{2. Fairness}
        \begin{itemize}
            \item \textbf{Definition:} Fairness involves ensuring that decisions made by MDPs and RL are equitable and justifiable across different demographic groups.
            \item \textbf{Illustration:} 
                \begin{itemize}
                    \item Use a fairness metric like \textbf{Demographic Parity}, which checks if the decision outcomes (e.g., job offers) are independent of protected attributes (e.g., gender). 
                    \item Fairness-aware algorithms can help balance outcomes.
                \end{itemize}
            \item \textbf{Key Point:} Implement fairness-aware algorithms that assess and mitigate disparities in decision-making outcomes.
        \end{itemize}
    \end{block}

    \begin{block}{3. Decision Transparency}
        \begin{itemize}
            \item \textbf{Definition:} Transparency in AI describes how understandable and interpretable the decision-making process is for end-users and stakeholders.
            \item \textbf{Example:}
                \begin{itemize}
                    \item An RL agent managing healthcare resource allocation must provide insights into how it arrived at certain decisions.
                    \item Explainable AI techniques, such as SHAP (SHapley Additive exPlanations), can illuminate the decision paths taken by these models.
                \end{itemize}
            \item \textbf{Key Point:} Foster transparency by employing models that are interpretable and offer explanations for their decisions.
        \end{itemize}
    \end{block}
\end{frame}

\begin{frame}[fragile]
    \frametitle{Ethical Considerations - Conclusion}
    The ethical deployment of MDPs and RL systems is vital for building trust, accountability, and societal acceptance. 
    \begin{itemize}
        \item Integrating bias mitigation strategies, promoting fairness, and enhancing decision transparency are essential steps toward responsible AI.
    \end{itemize}
    
    \begin{block}{Additional Considerations}
        \begin{itemize}
            \item Encourage interdisciplinary collaboration, involving ethicists and domain experts, to holistically address ethical issues.
            \item Regular audits and assessments of the algorithms post-deployment can catch biases and fairness issues that may arise in real-world applications.
        \end{itemize}
    \end{block}

    \begin{block}{References}
        Consider exploring research papers and case studies focused on ethical AI practices to delve deeper into each topic discussed.
    \end{block}
\end{frame}

\begin{frame}[fragile]
    \frametitle{Future Directions in Decision Making}
    \begin{block}{Overview}
        In the rapidly evolving field of artificial intelligence and machine learning, decision-making algorithms, particularly Markov Decision Processes (MDPs) and reinforcement learning (RL), are at the forefront of potential advancements. This slide highlights innovative trends and areas for future development that could enhance these techniques.
    \end{block}
\end{frame}

\begin{frame}[fragile]
    \frametitle{Future Directions: Concepts}
    \begin{enumerate}
        \item Hybrid Models
        \item Improved Exploration Strategies
        \item Transfer Learning in RL
        \item Explainable AI in Decision Making
        \item Addressing Ethical Concerns
    \end{enumerate}
\end{frame}

\begin{frame}[fragile]
    \frametitle{Hybrid Models}
    \begin{itemize}
        \item \textbf{Explanation:} Combining MDPs with neural networks (i.e., Deep Reinforcement Learning) creates more powerful models that can handle complex environments.
        \item \textbf{Example:} AlphaGo, which uses neural networks to evaluate board positions in a way that traditional MDPs cannot.
    \end{itemize}
\end{frame}

\begin{frame}[fragile]
    \frametitle{Improved Exploration Strategies}
    \begin{itemize}
        \item \textbf{Explanation:} Developing more effective exploration techniques can help agents learn efficiently in environments with sparse reward signals.
        \item \textbf{Example:} Implementing curiosity-driven learning, where agents explore based on a novelty metric, leading to faster learning in unknown environments.
    \end{itemize}
\end{frame}

\begin{frame}[fragile]
    \frametitle{Transfer Learning and Explainable AI}
    \begin{itemize}
        \item \textbf{Transfer Learning in RL:}
        \begin{itemize}
            \item \textbf{Explanation:} Leveraging knowledge gained in one task to improve performance in another similar task enhances flexibility and efficiency.
            \item \textbf{Example:} A robot trained to manipulate one type of object could use that knowledge to adapt to handle a different object quickly.
        \end{itemize}
        
        \item \textbf{Explainable AI in Decision Making:}
        \begin{itemize}
            \item \textbf{Explanation:} As RL systems are deployed across critical sectors, the need for transparency and interpretability grows. Future algorithms may focus on being able to explain their decision-making process.
            \item \textbf{Example:} An RL agent used in healthcare must provide rationales for its treatment decisions to gain trust from medical professionals and patients.
        \end{itemize}
    \end{itemize}
\end{frame}

\begin{frame}[fragile]
    \frametitle{Addressing Ethical Concerns and Future Research Areas}
    \begin{itemize}
        \item \textbf{Addressing Ethical Concerns:}
        \begin{itemize}
            \item \textbf{Explanation:} Advancing algorithms to mitigate bias while ensuring fairness in decision outcomes is crucial. Research into fair reinforcement learning is underway to create equitable solutions.
            \item \textbf{Example:} Developing algorithms that adjust outcomes based on demographic fairness criteria ensures health resource allocations do not favor one group over another.
        \end{itemize}
        
        \item \textbf{Potential Future Research Areas:}
        \begin{enumerate}
            \item Robustness against adversarial attacks
            \item Real-time decision making
            \item Multi-agent systems
        \end{enumerate}
    \end{itemize}
\end{frame}

\begin{frame}[fragile]
    \frametitle{Key Points to Emphasize}
    \begin{itemize}
        \item Integration with advanced AI offers significant potential.
        \item Novel exploration strategies can enhance learning efficiency.
        \item Sharing learned knowledge across tasks leads to smarter, faster agents.
        \item The growing demands for ethical AI and transparency cannot be overlooked.
    \end{itemize}
\end{frame}

\begin{frame}[fragile]
  \frametitle{Summary and Key Takeaways - Overview of Decision-Making}
  \begin{itemize}
    \item \textbf{Markov Decision Processes (MDPs)}
      \begin{itemize}
        \item Framework for modeling decision-making with randomness and control.
        \item Key components: States (S), Actions (A), Transition Function (T), Reward Function (R), and Discount Factor ($\gamma$).
      \end{itemize}
      
    \item \textbf{Reinforcement Learning (RL)}
      \begin{itemize}
        \item Learning paradigm where agents take actions to maximize cumulative rewards.
        \item Important concepts: Policy ($\pi$), Value Function ($V$), and Q-function ($Q$).
      \end{itemize}
  \end{itemize}
\end{frame}

\begin{frame}[fragile]
  \frametitle{Summary and Key Takeaways - Real-World Applications}
  \begin{itemize}
    \item \textbf{Autonomous Vehicles}
      \begin{itemize}
        \item MDPs assist in navigation and decision-making under uncertainty.
      \end{itemize}
      
    \item \textbf{Healthcare}
      \begin{itemize}
        \item RL is used for creating personalized treatment plans based on patient feedback.
      \end{itemize}

    \item \textbf{Robotics}
      \begin{itemize}
        \item Robots employ RL for learning tasks through exploration and feedback.
      \end{itemize}
  \end{itemize}
\end{frame}

\begin{frame}[fragile]
  \frametitle{Summary and Key Takeaways - Key Concepts}
  \begin{itemize}
    \item \textbf{Relation Between MDPs and RL}
      \begin{itemize}
        \item MDPs provide the theoretical foundation; RL is its application in practical scenarios.
      \end{itemize}
      
    \item \textbf{Exploration vs. Exploitation}
      \begin{itemize}
        \item Agents must balance exploring new actions versus exploiting known successful actions.
      \end{itemize}
  \end{itemize}

  \begin{block}{Conclusion}
    Understanding MDPs and RL allows the modeling and solving of complex decision-making tasks across various domains, advancing AI and automation.
  \end{block}
\end{frame}


\end{document}