\documentclass[aspectratio=169]{beamer}

% Theme and Color Setup
\usetheme{Madrid}
\usecolortheme{whale}
\useinnertheme{rectangles}
\useoutertheme{miniframes}

% Additional Packages
\usepackage[utf8]{inputenc}
\usepackage[T1]{fontenc}
\usepackage{graphicx}
\usepackage{booktabs}
\usepackage{listings}
\usepackage{amsmath}
\usepackage{amssymb}
\usepackage{xcolor}
\usepackage{tikz}
\usepackage{pgfplots}
\pgfplotsset{compat=1.18}
\usetikzlibrary{positioning}
\usepackage{hyperref}

% Custom Colors
\definecolor{myblue}{RGB}{31, 73, 125}
\definecolor{mygray}{RGB}{100, 100, 100}
\definecolor{mygreen}{RGB}{0, 128, 0}
\definecolor{myorange}{RGB}{230, 126, 34}
\definecolor{mycodebackground}{RGB}{245, 245, 245}

% Set Theme Colors
\setbeamercolor{structure}{fg=myblue}
\setbeamercolor{frametitle}{fg=white, bg=myblue}
\setbeamercolor{title}{fg=myblue}
\setbeamercolor{section in toc}{fg=myblue}
\setbeamercolor{item projected}{fg=white, bg=myblue}
\setbeamercolor{block title}{bg=myblue!20, fg=myblue}
\setbeamercolor{block body}{bg=myblue!10}
\setbeamercolor{alerted text}{fg=myorange}

% Set Fonts
\setbeamerfont{title}{size=\Large, series=\bfseries}
\setbeamerfont{frametitle}{size=\large, series=\bfseries}
\setbeamerfont{caption}{size=\small}
\setbeamerfont{footnote}{size=\tiny}

% Document Start
\begin{document}

\frame{\titlepage}

\begin{frame}[fragile]
    \frametitle{Overview of Logic Reasoning in Artificial Intelligence}
    Logic reasoning is the process of using formal logical principles to derive conclusions from premises. 
    It plays a crucial role in artificial intelligence (AI) by enabling machines to perform tasks that require understanding, inference, and decision-making.
\end{frame}

\begin{frame}[fragile]
    \frametitle{Importance of Logic Reasoning in AI}
    \begin{itemize}
        \item \textbf{Knowledge Representation}
        \begin{itemize}
            \item AI systems represent knowledge about the world using a formal language.
            \item Example: "If symptoms A and B are present, then disease X is likely."
        \end{itemize}

        \item \textbf{Automated Reasoning}
        \begin{itemize}
            \item Automatically deduces new information essential for problem-solving.
            \item Example: "All humans are mortal; Socrates is a human; therefore, Socrates is mortal."
        \end{itemize}

        \item \textbf{Consistency and Validity}
        \begin{itemize}
            \item Ensures conclusions are consistent and valid, critical in legal and ethical reasoning.
        \end{itemize}

        \item \textbf{Inferences and Predictions}
        \begin{itemize}
            \item Enables making inferences and generating predictions.
            \item Example: "If User A likes Book 1 and Book 2, then User A will likely enjoy Book 3."
        \end{itemize}
    \end{itemize}
\end{frame}

\begin{frame}[fragile]
    \frametitle{Types of Logic in AI}
    \begin{enumerate}
        \item \textbf{Propositional Logic}
        \begin{itemize}
            \item Deals with true or false propositions using logical operators.
            \item Example: Let \( P \) be "It is raining" and \( Q \) be "I will take an umbrella." 
            The expression \( P \Rightarrow Q \) reads: "If it is raining, then I will take an umbrella."
        \end{itemize}

        \item \textbf{First-Order Logic (Predicate Logic)}
        \begin{itemize}
            \item Extends propositional logic with predicates and quantifiers.
            \item Example: "All humans are mortal" can be expressed as \( \forall x (Human(x) \Rightarrow Mortal(x)) \).
        \end{itemize}
    \end{enumerate}
\end{frame}

\begin{frame}[fragile]
    \frametitle{Conclusion and Key Takeaways}
    Logic reasoning is integral to reasoning, decision-making, and knowledge representation in AI:
    \begin{itemize}
        \item Enables structured knowledge representation and reasoning.
        \item Provides consistency and validity checks for AI conclusions.
        \item Facilitates inferences and predictions using propositional and first-order logic.
    \end{itemize}

    \textbf{Further Reading:} Explore resources on propositional and first-order logic for a deeper understanding of logical structures in AI systems.
\end{frame}

\begin{frame}[fragile]{Propositional Logic - Definition}
    \begin{block}{Definition of Propositional Logic}
        Propositional logic, also known as propositional calculus or logic, deals with propositions, which are statements that can be classified as either true or false. Propositions are represented by variables, and logical operations are applied to these variables to form compound propositions.
    \end{block}
\end{frame}

\begin{frame}[fragile]{Propositional Logic - Structure}
    \begin{block}{Structure of Propositional Logic}
        \begin{enumerate}
            \item \textbf{Propositions}:
            \begin{itemize}
                \item A proposition is a declarative sentence that asserts a fact. 
                \item \textbf{Examples}:
                \begin{itemize}
                    \item "The sky is blue" (True)
                    \item "It is raining" (True/False depending on the situation)
                \end{itemize}
            \end{itemize}
            \item \textbf{Logical Operators}:
            Propositional logic employs several key logical operators:
            \begin{itemize}
                \item \textbf{Negation ($\neg$)}: Inverts the truth value of a proposition.
                \begin{itemize}
                    \item \textbf{Example}: If $P = $ "The sky is blue", then $\neg P = $ "The sky is not blue".
                \end{itemize}
                \item \textbf{Conjunction ($\land$)}: Represents "and"; true only if both propositions are true.
                \begin{itemize}
                    \item \textbf{Example}: $P \land Q$ ("The sky is blue AND it is raining").
                \end{itemize}
                \item \textbf{Disjunction ($\lor$)}: Represents "or"; true if at least one proposition is true.
                \begin{itemize}
                    \item \textbf{Example}: $P \lor Q$ ("The sky is blue OR it is raining").
                \end{itemize}
                \item \textbf{Implication ($\rightarrow$)}: Represents "if...then"; false only when the first is true and the second is false.
                \begin{itemize}
                    \item \textbf{Example}: $P \rightarrow Q$ ("If the sky is blue, then it is daytime").
                \end{itemize}
                \item \textbf{Biconditional ($\leftrightarrow$)}: True when both propositions are either true or false simultaneously.
                \begin{itemize}
                    \item \textbf{Example}: $P \leftrightarrow Q$ ("The sky is blue if and only if it is daytime").
                \end{itemize}
            \end{itemize}
        \end{enumerate}
    \end{block}
\end{frame}

\begin{frame}[fragile]{Truth Tables - Introduction}
    \begin{block}{Introduction to Truth Tables}
        A \textbf{truth table} is a mathematical table used in logic—specifically in propositional logic—to determine the truth value of logical expressions based on the truth values of their variables. Each row in the table represents a possible combination of truth values for the variables involved.
    \end{block}
\end{frame}

\begin{frame}[fragile]{Truth Tables - Components}
    \begin{block}{Components of Truth Tables}
        \begin{itemize}
            \item \textbf{Variables}: Propositional variables (e.g., P, Q) that can be either true (T) or false (F).
            \item \textbf{Operators}: Logical connectives that combine variables to form compound statements. Common operators include:
            \begin{itemize}
                \item AND ($\land$)
                \item OR ($\lor$)
                \item NOT ($\neg$)
            \end{itemize}
        \end{itemize}
    \end{block}
\end{frame}

\begin{frame}[fragile]{Truth Tables - Example 1}
    \frametitle{Truth Table for $P \land Q$}
    
    \begin{block}{Example of a Truth Table}
        Consider the logical expression involving two variables $P$ and $Q$:
        \[
            P \land Q \quad (\text{P and Q})
        \]
        \begin{table}[h]
            \centering
            \begin{tabular}{|c|c|c|}
                \hline
                P & Q & $P \land Q$ \\ 
                \hline
                T & T & T \\ 
                T & F & F \\ 
                F & T & F \\ 
                F & F & F \\ 
                \hline
            \end{tabular}
        \end{table}
        \textbf{Explanation}: The expression $P \land Q$ is true only when both $P$ and $Q$ are true (T).
    \end{block}
\end{frame}

\begin{frame}[fragile]{Truth Tables - Example 2}
    \frametitle{Truth Table for $P \lor \neg Q$}
    
    \begin{block}{Another Example}
        Consider the logical expression:
        \[
            P \lor \neg Q \quad (\text{P or not Q})
        \]
        \begin{table}[h]
            \centering
            \begin{tabular}{|c|c|c|c|}
                \hline
                P & Q & $\neg Q$ & $P \lor \neg Q$ \\ 
                \hline
                T & T & F & T \\ 
                T & F & T & T \\ 
                F & T & F & F \\ 
                F & F & T & T \\ 
                \hline
            \end{tabular}
        \end{table}
        \textbf{Explanation}: The column for $\neg Q$ shows the negation of $Q$. The expression $P \lor \neg Q$ evaluates to true if either $P$ is true or $Q$ is false.
    \end{block}
\end{frame}

\begin{frame}[fragile]{Truth Tables - Conclusion}
    \begin{block}{Conclusion}
        Truth tables are essential for accurately determining the validity of logical statements in propositional logic. They provide a structured way to analyze relationships between statements and are foundational for understanding more complex logical systems like first-order logic.
    \end{block}
\end{frame}

\begin{frame}[fragile]{Logical Connectives - Overview}
    \begin{block}{Overview of Logical Connectives}
        In propositional logic, logical connectives are symbols used to combine or modify propositions. Understanding these connectives is essential for evaluating complex logical statements and is foundational for reasoning in formal logic.
    \end{block}
\end{frame}

\begin{frame}[fragile]{Logical Connectives - AND (Conjunction)}
    \begin{itemize}
        \item \textbf{Symbol:} $\land$
        \item \textbf{Definition:} The conjunction of two propositions is true only if both propositions are true.
    \end{itemize}
    \begin{block}{Truth Table}
        \begin{tabular}{|c|c|c|}
            \hline
            $P$ & $Q$ & $P \land Q$ \\
            \hline
            T & T & T \\
            T & F & F \\
            F & T & F \\
            F & F & F \\
            \hline
        \end{tabular}
    \end{block}
    \begin{block}{Example}
        If $P$: "It is raining" and $Q$: "It is cold", then $P \land Q$: "It is raining AND it is cold" is true if both conditions are true.
    \end{block}
\end{frame}

\begin{frame}[fragile]{Logical Connectives - OR (Disjunction)}
    \begin{itemize}
        \item \textbf{Symbol:} $\lor$
        \item \textbf{Definition:} The disjunction of two propositions is true if at least one of the propositions is true.
    \end{itemize}
    \begin{block}{Truth Table}
        \begin{tabular}{|c|c|c|}
            \hline
            $P$ & $Q$ & $P \lor Q$ \\
            \hline
            T & T & T \\
            T & F & T \\
            F & T & T \\
            F & F & F \\
            \hline
        \end{tabular}
    \end{block}
    \begin{block}{Example}
        If $P$: "You will go to the park" and $Q$: "You will go to the mall", then $P \lor Q$: "You will go to the park OR you will go to the mall" is true if at least one activity occurs.
    \end{block}
\end{frame}

\begin{frame}[fragile]{Logical Connectives - NOT (Negation) and Implication}
    \begin{itemize}
        \item \textbf{NOT (Negation)}
            \begin{itemize}
                \item \textbf{Symbol:} $\neg$
                \item \textbf{Definition:} The negation of a proposition is true if the proposition is false and vice versa.
            \end{itemize}
            \begin{block}{Truth Table}
                \begin{tabular}{|c|c|}
                    \hline
                    $P$ & $\neg P$ \\
                    \hline
                    T & F \\
                    F & T \\
                    \hline
                \end{tabular}
            \end{block}
            \begin{block}{Example}
                If $P$: "It is sunny", then $\neg P$: "It is NOT sunny" is true if "It is sunny" is false.
            \end{block}
        
        \item \textbf{Implication (If...Then)}
            \begin{itemize}
                \item \textbf{Symbol:} $\rightarrow$
                \item \textbf{Definition:} The implication indicates that if the first proposition (antecedent) is true, then the second proposition (consequent) must also be true.
            \end{itemize}
            \begin{block}{Truth Table}
                \begin{tabular}{|c|c|c|}
                    \hline
                    $P$ & $Q$ & $P \rightarrow Q$ \\
                    \hline
                    T & T & T \\
                    T & F & F \\
                    F & T & T \\
                    F & F & T \\
                    \hline
                \end{tabular}
            \end{block}
            \begin{block}{Example}
                If $P$: "You study hard" and $Q$: "You will pass", then $P \rightarrow Q$: "If you study hard, then you will pass" is false only if you study hard and fail to pass.
            \end{block}
    \end{itemize}
\end{frame}

\begin{frame}[fragile]{Logical Connectives - Biconditional}
    \begin{itemize}
        \item \textbf{Biconditional (If and Only If)}
            \begin{itemize}
                \item \textbf{Symbol:} $\leftrightarrow$
                \item \textbf{Definition:} The biconditional is true if both propositions are either true or false.
            \end{itemize}
            \begin{block}{Truth Table}
                \begin{tabular}{|c|c|c|}
                    \hline
                    $P$ & $Q$ & $P \leftrightarrow Q$ \\
                    \hline
                    T & T & T \\
                    T & F & F \\
                    F & T & F \\
                    F & F & T \\
                    \hline
                \end{tabular}
            \end{block}
            \begin{block}{Example}
                If $P$: "You are a citizen" and $Q$: "You have the right to vote", then $P \leftrightarrow Q$: "You are a citizen IF AND ONLY IF you have the right to vote" is true when both situations hold the same truth value.
            \end{block}
    \end{itemize}
\end{frame}

\begin{frame}[fragile]{Key Points and Application}
    \begin{itemize}
        \item Logical connectives allow us to build complex statements from simpler ones.
        \item Understanding the structure of truth tables is pivotal in evaluating the truth values of logical expressions.
        \item Mastering these connectives is critical for progressing into more complex topics such as first-order logic.
    \end{itemize}
    \begin{block}{Application}
        Logical connectives form the basis for constructing valid arguments and analyzing their structure. As you advance in the study of logic, keep these foundational elements in mind as you encounter more complex reasoning patterns.
    \end{block}
\end{frame}

\begin{frame}[fragile]{Valid Arguments and Logical Equivalence - Part 1}
    \frametitle{Understanding Valid Arguments}
    
    \begin{block}{Definition}
        A valid argument is a reasoning structure where if the premises are true, the conclusion must also be true. 
        This doesn't necessarily mean the premises are factually correct; rather, it's about the form of the argument.
    \end{block}
    
    \begin{exampleblock}{Example of a Valid Argument}
        \begin{enumerate}
            \item \textbf{Premise 1:} If it rains, the ground will be wet. $(P \rightarrow Q)$
            \item \textbf{Premise 2:} It is raining. $(P)$
            \item \textbf{Conclusion:} Therefore, the ground is wet. $(Q)$
        \end{enumerate}
    \end{exampleblock}
    
    \begin{block}{Conclusion}
        This argument is valid because if both premises are true, the conclusion must also be true.
    \end{block}
\end{frame}

\begin{frame}[fragile]{Valid Arguments and Logical Equivalence - Part 2}
    \frametitle{Understanding Logical Equivalence}

    \begin{block}{Definition}
        Two propositions are logically equivalent if they always have the same truth value in any given situation. 
        This means that no matter what the values of the propositions are, they will yield the same truth output.
    \end{block}

    \begin{block}{Examples of Logical Equivalence}
        \begin{itemize}
            \item \textbf{Contraposition:} $P \rightarrow Q$ is logically equivalent to $\neg Q \rightarrow \neg P$
            \item \textbf{De Morgan's Laws:}
                \begin{itemize}
                    \item $\neg (P \land Q)$ is logically equivalent to $\neg P \lor \neg Q$
                    \item $\neg (P \lor Q)$ is logically equivalent to $\neg P \land \neg Q$
                \end{itemize}
        \end{itemize}
    \end{block}
\end{frame}

\begin{frame}[fragile]{Valid Arguments and Logical Equivalence - Part 3}
    \frametitle{Equivalence Demonstration}

    \begin{itemize}
        \item Let $P$: "It is raining." 
        \item Let $Q$: "The ground is wet."
    \end{itemize}

    \begin{block}{Equivalence Demonstration}
        \begin{itemize}
            \item \textbf{Original Statement:} "If it is raining, then the ground is wet." $(P \rightarrow Q)$
            \item \textbf{Contrapositive:} "If the ground is not wet, then it is not raining." $(\neg Q \rightarrow \neg P)$
        \end{itemize}
    \end{block}

    \begin{block}{Truth Table}
        \centering
        \begin{tabular}{|c|c|c|c|c|c|}
            \hline
            $P$ & $Q$ & $P \rightarrow Q$ & $\neg Q$ & $\neg P$ & $\neg Q \rightarrow \neg P$ \\
            \hline
            T & T & T & F & F & T \\
            T & F & F & T & F & F \\
            F & T & T & F & T & T \\
            F & F & T & T & T & T \\
            \hline
        \end{tabular}
    \end{block}

    \begin{block}{Conclusion}
        Both $P \rightarrow Q$ and $\neg Q \rightarrow \neg P$ have identical truth values, confirming they are logically equivalent.
    \end{block}
\end{frame}

\begin{frame}[fragile]{Recap and Next Steps}
    \frametitle{Key Points and Next Steps}

    \begin{itemize}
        \item A \textbf{valid argument} has a logical structure that guarantees the truth of the conclusion based on the premises.
        \item \textbf{Logical equivalence} ensures that two propositions hold the same truth value across all scenarios.
    \end{itemize}

    \begin{block}{Next Steps}
        We will explore First-Order Logic, which includes predicates, quantifiers, and how to extend propositional logic to more complex statements.
    \end{block}
\end{frame}

\begin{frame}[fragile]{First-Order Logic - Introduction}
    \begin{block}{Introduction to First-Order Logic}
        First-Order Logic (FOL) is a powerful framework used in mathematics, philosophy, linguistics, and computer science to express statements about objects and their relationships. 
        It extends propositional logic by including quantifiers and predicates.
    \end{block}
\end{frame}

\begin{frame}[fragile]{First-Order Logic - Key Components}
    \begin{block}{Key Components of First-Order Logic}
        \begin{enumerate}
            \item \textbf{Predicates}
                \begin{itemize}
                    \item A predicate is a function that takes objects as input and returns a truth value (true/false).
                    \item Example: Let \( P(x) \) denote "x is a cat".
                    \item For specific objects: \( P(Whiskers) \) is true if Whiskers is indeed a cat.
                \end{itemize}
                
            \item \textbf{Quantifiers}
                \begin{itemize}
                    \item \textbf{Universal Quantifier} (\( \forall \)): 
                        \begin{itemize}
                            \item Notation: \( \forall x \, P(x) \) means "For every x, P(x) is true."
                            \item Example: \( \forall x \, (P(x) \rightarrow Q(x)) \): "For every x, if x is a cat, then x is an animal."
                        \end{itemize}
                    \item \textbf{Existential Quantifier} (\( \exists \)): 
                        \begin{itemize}
                            \item Notation: \( \exists x \, P(x) \) means "There exists at least one x such that P(x) is true."
                            \item Example: \( \exists x \, P(x) \): "There exists at least one x such that x is a cat."
                        \end{itemize}
                \end{itemize}
            
            \item \textbf{Terms}
                \begin{itemize}
                    \item Represent objects within a domain; can be constants, variables, or functions.
                    \item Constants: Specific objects (e.g., \( a \) for a specific cat named Fluffy).
                    \item Variables: Symbols that can represent any object (e.g., \( x, y \)).
                    \item Functions: Mappings from objects to objects (e.g., \( f(x) \) could represent the mother of x).
                \end{itemize}
        \end{enumerate}
    \end{block}
\end{frame}

\begin{frame}[fragile]{First-Order Logic - Summary and Example}
    \begin{block}{Summary of Key Points}
        \begin{itemize}
            \item FOL allows for more expressive statements than propositional logic by incorporating predicates, quantifiers, and terms.
            \item Predicates enable statements about properties or relations of objects.
            \item Quantifiers provide a means to express generality or existence within the domain.
        \end{itemize}
    \end{block}

    \begin{block}{Example Formulation}
        Consider the statement: "All dogs bark." 
        In FOL, this can be expressed as:
        \[
        \forall x \, (Dog(x) \rightarrow Barks(x))
        \]
        Here, \( Dog(x) \) is a predicate identifying dogs, and \( Barks(x) \) is a predicate stating that the object x barks.
    \end{block}
\end{frame}

\begin{frame}[fragile]{First-Order Logic - Conclusion}
    \begin{block}{Conclusion}
        Understanding the components of First-Order Logic is foundational for reasoning in mathematics, formal proofs, and artificial intelligence applications 
        where reasoning about entities, their properties, and relationships is essential.
    \end{block}
\end{frame}

\begin{frame}[fragile]
    \frametitle{Predicates and Quantifiers - Introduction}
    \begin{itemize}
        \item \textbf{Predicate:} A statement expressing a property or relation among objects with variables from a specific domain.
        \item \textbf{Notation:} A predicate \( P(x) \) is true for a specific value of \( x \).
        \item \textbf{Example:} 
          \begin{itemize}
              \item Let \( P(x) \) mean "x is a cat".
              \item If \( x = \text{"Whiskers"} \), then \( P(\text{"Whiskers"}) \) is true; if \( x = \text{"Rover"} \), then \( P(\text{"Rover"}) \) is false.
          \end{itemize}
    \end{itemize}
\end{frame}

\begin{frame}[fragile]
    \frametitle{Predicates and Quantifiers - Types of Quantifiers}
    \begin{block}{Universal Quantifier ( \( \forall \) )}
        \begin{itemize}
            \item \textbf{Symbol:} \( \forall \)
            \item \textbf{Meaning:} "For all..." or "For every..."
            \item \textbf{Usage:} Indicates a property holds for all elements in a set.
            \item \textbf{Example:} 
              \begin{itemize}
                  \item If \( P(x) \) represents "x is a bird", then \( \forall x \, P(x) \) means "All x are birds".
              \end{itemize}
        \end{itemize}
    \end{block}

    \begin{block}{Existential Quantifier ( \( \exists \) )}
        \begin{itemize}
            \item \textbf{Symbol:} \( \exists \)
            \item \textbf{Meaning:} "There exists..." or "For some..."
            \item \textbf{Usage:} States at least one element makes the predicate true.
            \item \textbf{Example:} 
              \begin{itemize}
                  \item If \( Q(x) \) denotes "x is a blue car", then \( \exists x \, Q(x) \) means "There exists at least one x such that x is a blue car".
              \end{itemize}
        \end{itemize}
    \end{block}
\end{frame}

\begin{frame}[fragile]
    \frametitle{Predicates and Quantifiers - Key Points}
    \begin{itemize}
        \item A \textbf{predicate} asserts information about objects.
        \item The \textbf{universal quantifier} (\( \forall \)) applies to every member of a domain.
        \item The \textbf{existential quantifier} (\( \exists \)) confirms at least one member satisfies the statement.
        \item Mastery of these concepts is essential for logical arguments and translating natural language into logical forms.
    \end{itemize}

    \begin{block}{Formulas}
        \begin{itemize}
            \item Universal Quantifier: \( \forall x \, P(x) \)
            \item Existential Quantifier: \( \exists x \, P(x) \)
        \end{itemize}
    \end{block}
\end{frame}

\begin{frame}[fragile]{Syntax and Semantics of First-Order Logic}
    \begin{block}{Overview}
        First-Order Logic (FOL) is a powerful framework used in mathematics, computer science, and philosophy to express statements about objects and their relationships. The two critical components of FOL are \textbf{syntax} and \textbf{semantics}.
    \end{block}
\end{frame}

\begin{frame}[fragile]{Syntax of First-Order Logic}
    \begin{block}{1. Syntax of First-Order Logic}
        \begin{itemize}
            \item \textbf{Definition}: The formal structure or rules that govern how symbols can be combined to create valid statements.
            \item \textbf{Basic Components}:
            \begin{itemize}
                \item \textbf{Constants}: Symbols denoting specific objects (e.g., \(a, b, c\)).
                \item \textbf{Variables}: Symbols representing any object (e.g., \(x, y, z\)).
                \item \textbf{Predicates}: Express properties or relationships (e.g., \(P(x)\), \(Loves(a, b)\)).
                \item \textbf{Functions}: Map objects to other objects (e.g., \(F(x)\) denotes the father of \(x\)).
                \item \textbf{Logical Connectives}: Combine statements (e.g., \(\land\), \(\lor\), \(\neg\), \(\implies\)).
                \item \textbf{Quantifiers}: Indicate scope (e.g., \(\forall\) for "for all", \(\exists\) for "there exists").
            \end{itemize}
            \item \textbf{Well-Formed Formula (WFF)}: Example WFF - \(\forall x (P(x) \implies \exists y (Q(y, x)))\)
        \end{itemize}
    \end{block}
\end{frame}

\begin{frame}[fragile]{Semantics of First-Order Logic}
    \begin{block}{2. Semantics of First-Order Logic}
        \begin{itemize}
            \item \textbf{Definition}: Deals with the meaning of statements in FOL and how truth values are assigned.
            \item \textbf{Domains}: The collection of objects that variables can refer to (e.g., "all humans").
            \item \textbf{Interpretation}: Assigns meaning to symbols:
            \begin{itemize}
                \item Constants map to specific objects in the domain.
                \item Predicates map to relations or properties.
                \item Functions assign an output object to input objects.
            \end{itemize}
            \item \textbf{Truth Assignment}: Determines truth values based on interpretation:
            \begin{itemize}
                \item Let \(D = \{Alice, Bob\}\) be the domain.
                \item \(Loves(Alice, Bob)\) is **True**; \(Loves(Bob, Alice)\) is **False**.
            \end{itemize}
        \end{itemize}
    \end{block}
\end{frame}

\begin{frame}[fragile]{Key Points and Example}
    \begin{block}{Key Points to Emphasize}
        \begin{itemize}
            \item Syntax and semantics work together: Syntax gives us the rules for forming statements; semantics provides meanings.
            \item Understanding these components is essential for further studies in logic and reasoning applications.
        \end{itemize}
    \end{block}
    \begin{block}{Example for Clarity}
        Consider the statement: 
        \[
        \forall x (Cat(x) \rightarrow HasWhiskers(x))
        \]
        \begin{itemize}
            \item \textbf{Syntax}: A well-formed formula using a universal quantifier, a predicate, and an implication.
            \item \textbf{Semantics}: If interpreted in the domain of all cats, it asserts that every cat has whiskers.
        \end{itemize}
    \end{block}
\end{frame}

\begin{frame}[fragile]{Conclusion}
    \begin{block}{Conclusion}
        Understanding the syntax and semantics of First-Order Logic is crucial in constructing logical arguments, automating reasoning in computer systems, and developing foundational knowledge for more advanced logic studies.
        
        By grasping these concepts, students will be well-prepared to apply FOL in various contexts, which will be outlined in the neighboring slides.
    \end{block}
\end{frame}

\begin{frame}[fragile]{Inference Rules in First-Order Logic - Introduction}
    \begin{block}{Introduction to Inference Rules}
        Inference rules are the fundamental building blocks used to derive new statements (conclusions) from existing statements (premises) in first-order logic (FOL). Understanding these rules is essential for drawing valid conclusions, which is a crucial aspect of logical reasoning.
    \end{block}
\end{frame}

\begin{frame}[fragile]{Inference Rules in First-Order Logic - Key Rules}
    \begin{block}{Key Inference Rules}
        \begin{enumerate}
            \item \textbf{Universal Instantiation (UI)}
                \begin{itemize}
                    \item \textbf{Form:} From $\forall x P(x)$, infer $P(a)$ for any specific instance $a$.
                    \item \textbf{Example:} From $\forall x \ (\text{Human}(x) \rightarrow \text{Mortal}(x))$ and $\text{Human}(\text{Socrates})$, conclude $\text{Mortal}(\text{Socrates})$.
                \end{itemize}
                
            \item \textbf{Existential Instantiation (EI)}
                \begin{itemize}
                    \item \textbf{Form:} From $\exists x \ P(x)$, infer $P(a)$ for a specific instance $a$.
                    \item \textbf{Example:} From $\exists x \ \text{Happy}(x)$, we can state $\text{Happy}(a)$ for a new instance $a$.
                \end{itemize}
                
            \item \textbf{Universal Generalization (UG)}
                \begin{itemize}
                    \item \textbf{Form:} From $P(a)$ holding for an arbitrary $a$, conclude $\forall x P(x)$.
                    \item \textbf{Example:} If $\text{Friend}(a)$ is true for any person $a$, conclude $\forall x \ \text{Friend}(x)$.
                \end{itemize}
        \end{enumerate}
    \end{block}
\end{frame}

\begin{frame}[fragile]{Inference Rules in First-Order Logic - More Key Rules}
    \begin{block}{Continued Key Inference Rules}
        \begin{enumerate}
            \setcounter{enumi}{3}
            \item \textbf{Existential Generalization (EG)}
                \begin{itemize}
                    \item \textbf{Form:} From $P(a)$, conclude $\exists x P(x)$.
                    \item \textbf{Example:} If $\text{Philosopher}(\text{Socrates})$, we can conclude $\exists x \ \text{Philosopher}(x)$.
                \end{itemize}

            \item \textbf{Modus Ponens}
                \begin{itemize}
                    \item \textbf{Form:} From $P \rightarrow Q$ and $P$, infer $Q$.
                    \item \textbf{Example:} From $P: \text{Rains} \rightarrow \text{Wet}$ and $P: \text{Rains}$, conclude $Q: \text{Wet}$.
                \end{itemize}
        \end{enumerate}
    \end{block}
\end{frame}

\begin{frame}[fragile]{Inference Rules in First-Order Logic - Importance and Conclusion}
    \begin{block}{Key Points to Emphasize}
        \begin{itemize}
            \item \textbf{Importance:} Inference rules are essential for formal proofs in FOL and enhance logical reasoning capabilities.
            \item \textbf{Application:} Applicable in mathematics, computer science, and artificial intelligence.
            \item \textbf{Validity:} Correct application is critical for maintaining logical argument validity.
        \end{itemize}
    \end{block}

    \begin{block}{Conclusion}
        Developing an understanding of inference rules in first-order logic helps in constructing valid arguments and implications, forming the backbone of logical reasoning used in various disciplines.
    \end{block}
\end{frame}

\begin{frame}[fragile]
    \frametitle{Resolution in First-Order Logic}
    \begin{block}{Overview}
        Resolution is a powerful inference technique in first-order logic (FOL) used to derive conclusions from premises by converting them into clausal form.
    \end{block}
\end{frame}

\begin{frame}[fragile]
    \frametitle{Key Concepts}
    \begin{enumerate}
        \item \textbf{Clausal Form}:
        \begin{itemize}
            \item A conjunction of one or more clauses; a clause is a disjunction of literals.
        \end{itemize}
        
        \item \textbf{Literal}:
        \begin{itemize}
            \item An atomic proposition (e.g., $P$) or its negation (e.g., $\neg P$).
        \end{itemize}
        
        \item \textbf{Resolution Rule}:
        \begin{itemize}
            \item From two clauses:
            \begin{itemize}
                \item $A \lor B$ (Clause 1) and $¬B \lor C$ (Clause 2)
            \end{itemize}
            \item We can derive a new clause: $A \lor C$.
        \end{itemize}
    \end{enumerate}
\end{frame}

\begin{frame}[fragile]
    \frametitle{Steps to Apply Resolution}
    \begin{enumerate}
        \item \textbf{Convert to Clausal Form}:
        \begin{itemize}
            \item Use logical equivalences to transform premises.
        \end{itemize}
        
        \item \textbf{Negate the Conclusion}:
        \begin{itemize}
            \item Negate the conclusion and add it to the set of premises.
        \end{itemize}
        
        \item \textbf{Apply Resolution}:
        \begin{itemize}
            \item Generate new clauses iteratively until:
            \begin{itemize}
                \item A contradiction (empty clause) is derived.
                \item No new clauses can be generated.
            \end{itemize}
        \end{itemize}
    \end{enumerate}
\end{frame}

\begin{frame}[fragile]
    \frametitle{Example}
    \textbf{Premises:}
    \begin{enumerate}
        \item $ \forall x (P(x) \rightarrow Q(x)) $
        \item $ P(a) $
    \end{enumerate}
    
    \textbf{Negate Conclusion:} Assume $ ¬Q(a) $

    \textbf{Convert to Clausal Form:}
    \begin{itemize}
        \item $ ¬P(x) \lor Q(x) $ 
        \item $ P(a) $ 
        \item $ Q(a) $ 
    \end{itemize}

    \textbf{Apply Resolution:}
    \begin{itemize}
        \item From $ P(a) $ and $ ¬P(x) \lor Q(x) $, derive $ Q(a) $.
        \item This contradicts $ ¬Q(a) $, thus proving $ Q(a) $ must be true.
    \end{itemize}
\end{frame}

\begin{frame}[fragile]
    \frametitle{Key Points to Remember}
    \begin{itemize}
        \item \textbf{Soundness and Completeness}:
        \begin{itemize}
            \item Resolution is sound and complete; if a conclusion follows, it will be proved.
        \end{itemize}
        
        \item \textbf{Efficiency}:
        \begin{itemize}
            \item While simple in concept, resolution can require significant computational resources.
        \end{itemize}
    \end{itemize}
\end{frame}

\begin{frame}[fragile]
    \frametitle{Applications and Conclusion}
    \begin{itemize}
        \item Resolution is essential in:
        \begin{itemize}
            \item Automated theorem proving
            \item Logic programming
            \item Artificial intelligence
        \end{itemize}
        
        \item Understanding resolution is crucial for using first-order logic to derive conclusions in various fields, especially in AI and knowledge representation.
    \end{itemize}
\end{frame}

\begin{frame}[fragile]
    \frametitle{Applications of Logic in AI}
    Logic plays a crucial role in artificial intelligence (AI) by providing a formal framework for reasoning and knowledge representation.
    This is achieved through propositional logic and first-order logic, which enable machines to infer conclusions from given facts and rules.
\end{frame}

\begin{frame}[fragile]
    \frametitle{Introduction to Propositional Logic}
    \begin{block}{Definition}
        Propositional logic deals with statements that can be either true or false but not both. It uses propositional variables (e.g., P, Q) to represent these statements.
    \end{block}
    
    \begin{block}{Application}
        \textbf{Example:}
        If P = "It is raining" and Q = "Take an umbrella", a rule could be: 
        \begin{equation}
            P \rightarrow Q
        \end{equation}
        which means "If it is raining, then take an umbrella."
    \end{block}
\end{frame}

\begin{frame}[fragile]
    \frametitle{First-Order Logic (FOL)}
    \begin{block}{Definition}
        FOL extends propositional logic by allowing quantified variables and predicates, making it more expressive.
    \end{block}
    
    \begin{itemize}
        \item \textbf{Predicates:} Function-like entities that return true or false (e.g., Loves(John, Mary)).
        \item \textbf{Quantifiers:}
        \begin{itemize}
            \item Existential ($\exists$): "There exists."
            \item Universal ($\forall$): "For all."
        \end{itemize}
    \end{itemize}

    \begin{block}{Application}
        Used in knowledge representation to create statements about objects and their relationships.
        \textbf{Example:} "Everyone loves Mary."
        \begin{equation}
            \forall x \, (Human(x) \rightarrow Loves(x, Mary))
        \end{equation}
    \end{block}
\end{frame}

\begin{frame}[fragile]
    \frametitle{Practical Applications in AI}
    \begin{enumerate}
        \item \textbf{Expert Systems:}
            \begin{itemize}
                \item Use logic to emulate human expert decision-making (e.g., MYCIN).
            \end{itemize}
        \item \textbf{Automated Theorem Proving:}
            \begin{itemize}
                \item Logic is used to prove mathematical theorems through algorithms based on first-order logic.
            \end{itemize}
        \item \textbf{Natural Language Processing (NLP):}
            \begin{itemize}
                \item FOL aids in understanding and processing the semantics of sentences by representing their meanings logically.
            \end{itemize}
        \item \textbf{Robotics:}
            \begin{itemize}
                \item Logic-based AI helps robots make decisions based on their environment.
            \end{itemize}
    \end{enumerate}
\end{frame}

\begin{frame}[fragile]
    \frametitle{Key Points to Emphasize}
    \begin{itemize}
        \item Logic provides a structured way to tackle complex reasoning tasks in AI.
        \item Propositional logic is suited for straightforward decisions, while first-order logic handles more complex relationships and quantifications.
        \item Applications range from medical diagnosis to language understanding, showcasing logic’s versatility in the AI field.
    \end{itemize}
\end{frame}

\begin{frame}[fragile]
    \frametitle{Summary}
    The integration of propositional and first-order logic in AI creates powerful systems capable of reasoning, problem-solving, and learning.
    By understanding these logical frameworks, we can better comprehend how AI interprets information and makes decisions.
\end{frame}

\begin{frame}[fragile]
    \frametitle{Logic-Based AI Systems}
    \begin{block}{Introduction}
        Logic-Based AI systems use formal logic to represent knowledge and facilitate reasoning. They enable machines to derive conclusions from known facts, supporting complex decision-making and problem-solving.
    \end{block}
\end{frame}

\begin{frame}[fragile]
    \frametitle{Key Components}
    \begin{enumerate}
        \item \textbf{Knowledge Representation:}
        \begin{itemize}
            \item Information is stored in formats that AI can use.
            \item Examples:
                \begin{itemize}
                    \item Propositional Logic: "It is raining" as \( P \).
                    \item First-Order Logic: "All humans are mortal" as \( \forall x (Human(x) \rightarrow Mortal(x)) \).
                \end{itemize}
        \end{itemize}
        
        \item \textbf{Inference Engines:}
        \begin{itemize}
            \item Apply logical rules to derive new information.
            \item Example: Given "All birds can fly" and "A penguin is a bird," conclude "A penguin can fly" (incorrect but illustrates reasoning).
        \end{itemize}
    \end{enumerate}
\end{frame}

\begin{frame}[fragile]
    \frametitle{Examples of Logic-Based AI Systems}
    \begin{enumerate}
        \item \textbf{Expert Systems:}
        \begin{itemize}
            \item Mimics human decision-making.
            \item Example: MYCIN for diagnosing bacterial infections using rules.
        \end{itemize}
        
        \item \textbf{Automated Theorem Provers:}
        \begin{itemize}
            \item Prove mathematical theorems automatically.
            \item Example: Prover9 uses first-order logic to deduce statements.
        \end{itemize}
        
        \item \textbf{Semantic Web Technologies:}
        \begin{itemize}
            \item Enhance web content using logic-based principles.
            \item Example: RDF and OWL utilize first-order logic for relationships.
        \end{itemize}
    \end{enumerate}
\end{frame}

\begin{frame}[fragile]
    \frametitle{Mathematical Representation and Conclusion}
    \begin{block}{Logical Representation}
        \begin{itemize}
            \item Propositional Logic: \( P \lor Q \) (P or Q)
            \item First-Order Logic: \( \exists y \, (Bird(y) \land Flies(y)) \) (There exists a y such that y is a bird and y flies)
        \end{itemize}
    \end{block}
    
    \begin{block}{Conclusion}
        Logic-based AI systems are vital for replicating human reasoning. Their applications span many fields, demonstrating the practical use of logical reasoning in technology.
    \end{block}
\end{frame}

\begin{frame}[fragile]
    \frametitle{Challenges in Logic Reasoning - Overview}
    Implementing logic reasoning in AI systems presents several challenges. 
    These challenges stem from:
    \begin{itemize}
        \item Limitations of logic frameworks
        \item Complexities in real-world applications
        \item The dynamic nature of knowledge and reasoning processes
    \end{itemize}
\end{frame}

\begin{frame}[fragile]
    \frametitle{Challenges in Logic Reasoning - Key Challenges}
    \begin{enumerate}
        \item \textbf{Expressiveness vs. Decidability}
            \begin{itemize}
                \item Balancing ability to represent complex statements with the ability to determine their truth.
                \item \textit{Example:} First-order logic is more expressive than propositional logic but can be undecidable.
            \end{itemize}

        \item \textbf{Scalability}
            \begin{itemize}
                \item Complexity of logical relationships increases with data growth.
                \item \textit{Example:} Large knowledge bases in expert systems can cause delays in processing.
            \end{itemize}
    \end{enumerate}
\end{frame}

\begin{frame}[fragile]
    \frametitle{Challenges in Logic Reasoning - Continued}
    \begin{enumerate}
        \setcounter{enumi}{2} % Continue from previous enumeration
        \item \textbf{Knowledge Representation}
            \begin{itemize}
                \item Accurately representing knowledge for processing by logic systems is challenging.
                \item \textit{Illustration:} Representing "All cats are mammals" must allow for variations such as "some feline species."
            \end{itemize}

        \item \textbf{Ambiguity and Vagueness}
            \begin{itemize}
                \item Natural language often contains terms that are difficult to define in logical terms.
                \item \textit{Example:} "Tall person" lacks a defined metric for logical interpretation.
            \end{itemize}

        \item \textbf{Dynamic Knowledge}
            \begin{itemize}
                \item Knowledge evolves; keeping logical systems up-to-date while ensuring consistency is challenging.
                \item \textit{Example:} New medical research may alter existing treatment protocols.
            \end{itemize}
    \end{enumerate}
\end{frame}

\begin{frame}[fragile]
    \frametitle{Challenges in Logic Reasoning - Inference Limitations}
    \begin{itemize}
        \item \textbf{Inference Limitations}
            \begin{itemize}
                \item Incorrect application of inference rules can lead to errors in reasoning.
                \item \textit{Illustration:} The premise "All birds can fly" leads to incorrect conclusions.
            \end{itemize}
        \item \textbf{Summary}
            \begin{itemize}
                \item Addressing these challenges is crucial for effective design and deployment of logic-based AI systems.
                \item Solutions may include hybrid approaches, improved representation techniques, and robust updating mechanisms.
            \end{itemize}
    \end{itemize}
\end{frame}

\begin{frame}[fragile]{Ethical Considerations in Logic Applications - Part 1}
    \frametitle{Understanding Ethics in Logic Applications}
    As artificial intelligence (AI) becomes increasingly integrated into decision-making processes, the ethical implications of employing logic reasoning in these systems become crucial. 
    \begin{itemize}
        \item Focus on bias and consequences of automated decision-making.
    \end{itemize}
\end{frame}

\begin{frame}[fragile]{Ethical Considerations in Logic Applications - Part 2}
    \frametitle{Key Concepts}
    \begin{enumerate}
        \item \textbf{Bias in Logic Systems}
        \begin{itemize}
            \item \textbf{Definition}: Disproportionate favoritism or prejudice leading to unjust consequences.
            \item \textbf{Sources of Bias}:
            \begin{itemize}
                \item Data: Biased datasets perpetuate existing prejudices.
                \item Algorithms: Flawed logic can enforce biases.
            \end{itemize}
            \item \textbf{Example}: A hiring algorithm favoring candidates from certain educational backgrounds due to systemic prejudices in historical data.
        \end{itemize}
        
        \item \textbf{Decision-Making Processes}
        \begin{itemize}
            \item \textbf{Transparency}: Logic reasoning needs to be understandable to end-users impacting accountability.
            \item \textbf{Responsibility}: Complexities arise regarding accountability for decisions made by AI systems.
            \item \textbf{Example}: AI in criminal justice predicting recidivism rates based on flawed data raises accountability questions.
        \end{itemize}
    \end{enumerate}
\end{frame}

\begin{frame}[fragile]{Ethical Considerations in Logic Applications - Part 3}
    \frametitle{Implications and Conclusion}
    \begin{itemize}
        \item \textbf{Implications of Unethical Logic Use}
        \begin{itemize}
            \item Fairness: AI biases undermine fairness in societal applications (e.g., hiring, law enforcement).
            \item Trust: Public trust erosion hinders technological adoption.
        \end{itemize}
        
        \item \textbf{Key Points to Emphasize}
        \begin{itemize}
            \item Awareness: Continuous scrutiny of AI training data and algorithms is vital.
            \item Interdisciplinary Collaboration: Engaging diverse experts can mitigate bias.
            \item Regulations: Advocating for fairness and transparency legislation.
        \end{itemize}
        
        \item \textbf{Conclusion}: Ethical considerations in logic reasoning in AI must not be overlooked for the development of equitable systems.
    \end{itemize}
\end{frame}

\begin{frame}[fragile]
    \frametitle{Future Directions in Logic Reasoning}
    Exploring future trends and improvements in logic reasoning technologies for AI.
\end{frame}

\begin{frame}[fragile]
    \frametitle{Introduction to Future Trends in Logic Reasoning}
    \begin{itemize}
        \item As AI technologies evolve, the methodologies in logic reasoning will transform.
        \item This section outlines emerging trends and advancements in propositional and first-order logic.
    \end{itemize}
\end{frame}

\begin{frame}[fragile]
    \frametitle{Integration of Logic and Machine Learning}
    \begin{itemize}
        \item \textbf{Concept:} Combine classical logic with machine learning to enhance decision-making.
        \item \textbf{Example:} Utilizing logic to interpret machine learning models, enabling transparency.
        \item \textbf{Key Point:} This hybrid model clarifies and justifies AI inferences, overcoming black-box challenges.
    \end{itemize}
\end{frame}

\begin{frame}[fragile]
    \frametitle{Enhanced Knowledge Representation}
    \begin{itemize}
        \item \textbf{Concept:} Develop richer and more expressive knowledge representation systems.
        \item \textbf{Example:} Advanced ontologies capturing complex relations and dependencies.
        \item \textbf{Key Point:} Enhanced representation improves algorithms in interpreting context and semantics.
    \end{itemize}
\end{frame}

\begin{frame}[fragile]
    \frametitle{Automated Theorem Proving Across Disciplines}
    \begin{itemize}
        \item \textbf{Concept:} Automated reasoning tools adapting to various domains (e.g., mathematics, law).
        \item \textbf{Example:} Theorem provers assisting in legal reasoning to verify contract validity.
        \item \textbf{Key Point:} Expanding applicability empowers systems to solve complex real-world problems.
    \end{itemize}
\end{frame}

\begin{frame}[fragile]
    \frametitle{Logic in Natural Language Processing (NLP)}
    \begin{itemize}
        \item \textbf{Concept:} Bridging logic reasoning with NLP to enhance AI comprehension.
        \item \textbf{Example:} Using logical frameworks to resolve language ambiguities in dialogue systems.
        \item \textbf{Key Point:} Enhances interaction and understanding between AI and users.
    \end{itemize}
\end{frame}

\begin{frame}[fragile]
    \frametitle{Addressing Ethical and Bias Concerns}
    \begin{itemize}
        \item \textbf{Concept:} Incorporating logic to tackle ethical implications and biases in decision-making.
        \item \textbf{Example:} Logic frameworks analyzing decision trails to identify bias.
        \item \textbf{Key Point:} Ensures fairness and provides structured evaluation methods against biases.
    \end{itemize}
\end{frame}

\begin{frame}[fragile]
    \frametitle{Conclusion and Key Takeaways}
    The future of logic reasoning technologies in AI is promising, focusing on:
    \begin{itemize}
        \item \textbf{Integration:} Merging logic and machine learning.
        \item \textbf{Knowledge Representation:} More expressive models.
        \item \textbf{Automated Theorem Proving:} Broadening application across fields.
        \item \textbf{NLP:} Using logic to enhance communication.
        \item \textbf{Ethics:} Tackling bias and promoting fairness.
    \end{itemize}
\end{frame}

\begin{frame}[fragile]
    \frametitle{Further Reading}
    For those interested in deeper exploration, consider:
    \begin{itemize}
        \item "Explainable AI and its Role in Logical Reasoning"
        \item "The Ethics of Automating Decision-Making: A Logical Perspective"
    \end{itemize}
\end{frame}

\begin{frame}[fragile]
    \frametitle{Summary and Recap - Key Points}
    \begin{enumerate}
        \item \textbf{Propositional Logic Basics}
        \begin{itemize}
            \item \textbf{Definition}: Deals with statements (propositions) that can be either true or false.
            \item \textbf{Key Components}:
            \begin{itemize}
                \item \textbf{Propositions}: Simple statements (e.g., "It is raining").
                \item \textbf{Connectives}: Logical operators (AND, OR, NOT, IMPLIES).
            \end{itemize}
            \item \textbf{Truth Tables}: Evaluating truth values under different scenarios.
        \end{itemize}
    \end{enumerate}
\end{frame}

\begin{frame}[fragile]
    \frametitle{Summary and Recap - Examples}
    \textbf{Example:} 
    \begin{itemize}
        \item Proposition: $P$: "It is raining."
        \item Proposition: $Q$: "I will take an umbrella."
        \item Compound Statement: "If P, then Q" ($P \rightarrow Q$).
    \end{itemize}
    
    \begin{enumerate}
        \item \textbf{First-Order Logic (FOL)}
        \begin{itemize}
            \item \textbf{Definition}: Extends propositional logic with quantifiers and predicates.
            \item \textbf{Key Components}:
            \begin{itemize}
                \item \textbf{Predicates}: Functions that return true or false (e.g., $Loves(John, Mary)$).
                \item \textbf{Quantifiers}: 
                \begin{itemize}
                    \item Universal ($\forall$): "For all."
                    \item Existential ($\exists$): "There exists."
                \end{itemize}
            \end{itemize}
        \end{itemize}
        
        \item \textbf{Example in FOL}:
        \begin{itemize}
            \item Statement: $\forall x (Human(x) \rightarrow Mortal(x))$, meaning "All humans are mortal."
        \end{itemize}
    \end{enumerate}
\end{frame}

\begin{frame}[fragile]
    \frametitle{Summary and Recap - Applications and Relevance}
    \begin{enumerate}
        \item \textbf{Semantics and Inference}
        \begin{itemize}
            \item \textbf{Semantics}: Meaning behind propositions.
            \item \textbf{Inference Rules}: Logical rules for deriving new truths (e.g., Modus Ponens).
        \end{itemize}
        
        \item \textbf{Applications in AI}
        \begin{itemize}
            \item \textbf{Knowledge Representation}: Structuring and storing knowledge using logic.
            \item \textbf{Reasoning}: AI uses logic for reasoning about facts and relationships.
        \end{itemize}
        
        \item \textbf{Relevance to AI}
        \begin{itemize}
            \item Understanding of logic is critical for AI algorithms needing reasoning and decision-making.
            \item Enhances ability to process data in fields like natural language processing and automated theorem proving.
        \end{itemize}
    \end{enumerate}
\end{frame}


\end{document}