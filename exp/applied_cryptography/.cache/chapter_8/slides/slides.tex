\documentclass{beamer}

% Theme choice
\usetheme{Madrid} % You can change to e.g., Warsaw, Berlin, CambridgeUS, etc.

% Encoding and font
\usepackage[utf8]{inputenc}
\usepackage[T1]{fontenc}

% Graphics and tables
\usepackage{graphicx}
\usepackage{booktabs}

% Code listings
\usepackage{listings}
\lstset{
    basicstyle=\ttfamily\small,
    keywordstyle=\color{blue},
    commentstyle=\color{gray},
    stringstyle=\color{red},
    breaklines=true,
    frame=single
}

% Math packages
\usepackage{amsmath}
\usepackage{amssymb}

% Colors
\usepackage{xcolor}

% TikZ and PGFPlots
\usepackage{tikz}
\usepackage{pgfplots}
\pgfplotsset{compat=1.18}
\usetikzlibrary{positioning}

% Hyperlinks
\usepackage{hyperref}

% Title information
\title{Chapter 8: Implementing Cryptography in Java}
\author{Your Name}
\institute{Your Institution}
\date{\today}

\begin{document}

\frame{\titlepage}

\begin{frame}[fragile]
    \frametitle{Introduction to Cryptography in Java}
    \begin{block}{Overview of Cryptography}
        Cryptography is a vital aspect of computer security that enables the protection of data through encoding and decoding processes. 
        In the context of Java programming, cryptography allows developers to secure data during transmission and storage to ensure confidentiality, integrity, and authenticity.
    \end{block}
\end{frame}

\begin{frame}[fragile]
    \frametitle{Importance of Cryptography in Java}
    \begin{enumerate}
        \item \textbf{Data Security}: Protects sensitive information (like passwords, personal information, etc.) from unauthorized access.
        \item \textbf{Secure Communication}: Protocols like HTTPS rely on cryptographic techniques to ensure secure connections over the internet.
        \item \textbf{Data Integrity}: Cryptographic hash functions ensure that data has not been altered or tampered with during storage or transmission.
        \item \textbf{Authentication}: Supports methods for verifying users' identities (e.g., digital signatures).
    \end{enumerate}
\end{frame}

\begin{frame}[fragile]
    \frametitle{Key Concepts in Cryptography}
    \begin{itemize}
        \item \textbf{Encryption}: Converting plaintext into ciphertext to prevent unauthorized access.
        \item \textbf{Decryption}: Converting ciphertext back into plaintext, making it readable again.
        \item \textbf{Hashing}: A one-way function that converts data into a fixed-size string of characters, used for integrity checks.
    \end{itemize}
\end{frame}

\begin{frame}[fragile]
    \frametitle{Example: Symmetric Encryption in Java}
    \begin{block}{Using AES for Encryption}
        \begin{lstlisting}[language=Java]
import javax.crypto.Cipher;
import javax.crypto.KeyGenerator;
import javax.crypto.SecretKey;

public class AESEncryption {
    public static void main(String[] args) throws Exception {
        KeyGenerator keyGen = KeyGenerator.getInstance("AES");
        keyGen.init(128); // Key size
        SecretKey secretKey = keyGen.generateKey();

        Cipher cipher = Cipher.getInstance("AES");
        cipher.init(Cipher.ENCRYPT_MODE, secretKey);
        
        String plaintext = "Hello, Secure World!";
        byte[] ciphertext = cipher.doFinal(plaintext.getBytes());
        // ciphertext is now encrypted
    }
}
        \end{lstlisting}
    \end{block}
\end{frame}

\begin{frame}[fragile]
    \frametitle{Key Points to Emphasize}
    \begin{itemize}
        \item Java provides robust libraries (\texttt{javax.crypto}, \texttt{java.security}) for implementing cryptographic features.
        \item Understand various encryption algorithms (like AES, RSA) and when to use each to ensure effective security.
        \item Stay updated with best practices in cryptography to maintain the highest security standards.
    \end{itemize}
\end{frame}

\begin{frame}[fragile]
    \frametitle{Summary}
    \begin{block}{Key Takeaway}
        Cryptography is essential in Java for securing data, ensuring communication integrity, and verifying authentication. 
        Familiarity with cryptographic principles and Java libraries equips developers with the necessary skills to protect sensitive information effectively.
    \end{block}
\end{frame}

\begin{frame}[fragile]
    \frametitle{Understanding Hash Functions - Definition}
    A \textbf{hash function} is a mathematical algorithm that transforms an input (or 'message') into a fixed-length string of characters, referred to as a \textbf{hash value}. This unique representation of data is essential in various cryptographic protocols.

    \begin{block}{Key Properties of Hash Functions}
        \begin{enumerate}
            \item \textbf{Deterministic:} The same input always produces the same output.
            \item \textbf{Fast computation:} Quick to compute the hash value for any input.
            \item \textbf{Pre-image resistance:} Infeasible to retrieve the original input from a hash value.
            \item \textbf{Small changes produce drastic changes:} Minor alterations in input lead to significantly different hash values.
            \item \textbf{Collision resistance:} Difficult to find two different inputs that yield the same hash value.
        \end{enumerate}
    \end{block}
\end{frame}

\begin{frame}[fragile]
    \frametitle{Understanding Hash Functions - Role in Data Integrity}
    Hash functions are crucial for verifying data integrity. The process involves generating a hash value from original data when it is created or transmitted.

    \begin{block}{Data Creation and Verification}
        \begin{itemize}
            \item \textbf{Original Data:} "HelloWorld"
            \item \textbf{Hash Value (SHA-256):} 
            \texttt{a591a6d40bf420404a011733cfb7b190d62c65bf0bcda190458f19750121c2b0}
            \item \textbf{Verification:}
                \begin{itemize}
                    \item Received Data: "HelloWorld"
                    \item Recomputed Hash Value: 
                    \texttt{a591a6d40bf420404a011733cfb7b190d62c65bf0bcda190458f19750121c2b0}
                \end{itemize}
        \end{itemize}
    \end{block}

    If the hash values match, data integrity is confirmed; otherwise, it indicates data alteration.
\end{frame}

\begin{frame}[fragile]
    \frametitle{Understanding Hash Functions - Role in Authentication}
    Hash functions are key in authentication processes, particularly in password storage methods.

    \begin{itemize}
        \item Plain-text passwords are not stored; instead, hashed versions are kept.
        \item When a user enters their password, the system hashes the input and compares it to the stored hash.
    \end{itemize}

    \begin{block}{Example Code Snippet in Java}
        \begin{lstlisting}[language=Java]
import java.security.MessageDigest;

public class HashExample {
    public static void main(String[] args) throws Exception {
        String input = "HelloWorld";
        
        MessageDigest md = MessageDigest.getInstance("SHA-256");
        byte[] hash = md.digest(input.getBytes());

        StringBuilder hexString = new StringBuilder();
        for (byte b : hash) {
            String hex = Integer.toHexString(0xff & b);
            if (hex.length() == 1) hexString.append('0');
            hexString.append(hex);
        }
        
        System.out.println("Hash Value: " + hexString.toString());
    }
}
        \end{lstlisting}
    \end{block}
\end{frame}

\begin{frame}[fragile]
    \frametitle{Java Libraries for Cryptography - Introduction}
    \begin{block}{Overview}
        Cryptography is essential for securing data, ensuring privacy, and verifying identity in applications. Java provides robust libraries for implementing cryptographic functions, notably:
    \end{block}
    \begin{enumerate}
        \item Java Cryptography Architecture (JCA)
        \item Bouncy Castle
    \end{enumerate}
\end{frame}

\begin{frame}[fragile]
    \frametitle{Java Libraries for Cryptography - JCA}
    \begin{block}{Java Cryptography Architecture (JCA)}
        \begin{itemize}
            \item **Overview**: JCA provides a framework for accessing and implementing cryptographic operations such as encryption and key generation.
            \item **Key Features**:
                \begin{itemize}
                    \item Provider Architecture: Multiple cryptographic service providers can be used.
                    \item Algorithm Support: Supports standard algorithms (e.g., AES, RSA).
                \end{itemize}
            \item **Basic Concepts**:
                \begin{itemize}
                    \item \textbf{Key}: A secret value used in cryptographic operations.
                    \item \textbf{Cipher}: An object for performing encryption and decryption.
                \end{itemize}
        \end{itemize}
    \end{block}
\end{frame}

\begin{frame}[fragile]
    \frametitle{Java Libraries for Cryptography - JCA Example}
    \begin{block}{Example Usage}
    \begin{lstlisting}[language=Java]
    import javax.crypto.Cipher;
    import javax.crypto.KeyGenerator;
    import javax.crypto.SecretKey;

    // Generate a secret key
    KeyGenerator keyGen = KeyGenerator.getInstance("AES");
    SecretKey secretKey = keyGen.generateKey();

    // Create a Cipher for encryption
    Cipher cipher = Cipher.getInstance("AES");
    cipher.init(Cipher.ENCRYPT_MODE, secretKey);
    byte[] encryptedData = cipher.doFinal("Hello, World!".getBytes());
    \end{lstlisting}
    \end{block}
\end{frame}

\begin{frame}[fragile]
    \frametitle{Java Libraries for Cryptography - Bouncy Castle}
    \begin{block}{Overview of Bouncy Castle}
        \begin{itemize}
            \item **Overview**: Bouncy Castle is an open-source library offering additional cryptographic algorithms not found in JCA.
            \item **Key Features**:
                \begin{itemize}
                    \item Extensive Algorithm Support: Includes many symmetric/asymmetric algorithms.
                    \item Lightweight API: Provides advanced features while being easy to implement.
                \end{itemize}
            \item **Use Cases**: Ideal for projects requiring specialized cryptographic algorithms.
        \end{itemize}
    \end{block}
\end{frame}

\begin{frame}[fragile]
    \frametitle{Java Libraries for Cryptography - Bouncy Castle Example}
    \begin{block}{Example Usage}
    \begin{lstlisting}[language=Java]
    import org.bouncycastle.jce.provider.BouncyCastleProvider;
    import java.security.Security;
    import org.bouncycastle.crypto.digests.SHA256Digest;

    // Add Bouncy Castle as a security provider
    Security.addProvider(new BouncyCastleProvider());

    // Create an instance of SHA-256
    SHA256Digest sha256 = new SHA256Digest();
    byte[] input = "Hello, World!".getBytes();
    sha256.update(input, 0, input.length);
    byte[] hash = new byte[sha256.getDigestSize()];
    sha256.doFinal(hash, 0);
    \end{lstlisting}
    \end{block}
\end{frame}

\begin{frame}[fragile]
    \frametitle{Java Libraries for Cryptography - Key Points}
    \begin{itemize}
        \item **JCA**: Built into Java, provides a wide range of cryptographic functions.
        \item **Bouncy Castle**: Offers a rich set of additional algorithms for complex needs.
        \item **Selection**: Choice of library depends on specific cryptographic requirements and standards.
    \end{itemize}
\end{frame}

\begin{frame}
    \frametitle{Implementing Hash Functions in Java}
    \begin{block}{Overview of Hash Functions}
        \begin{itemize}
            \item \textbf{Definition}: A hash function transforms an input into a fixed-length string of bytes, producing a unique digest for each unique input.
            \item \textbf{Purpose}: Used for data integrity verification, password storage, and digital signatures.
        \end{itemize}
    \end{block}
\end{frame}

\begin{frame}
    \frametitle{Common Hash Functions}
    \begin{itemize}
        \item \textbf{SHA-256}:
        \begin{itemize}
            \item Part of the SHA-2 family.
            \item Produces a 256-bit (32-byte) hash value.
            \item Commonly used in security applications like SSL/TLS and cryptocurrency.
        \end{itemize}
    \end{itemize}
\end{frame}

\begin{frame}[fragile]
    \frametitle{Step-by-Step Implementation of SHA-256 in Java}
    \textbf{Step 1: Import Required Classes}
    \begin{lstlisting}[language=Java]
import java.security.MessageDigest;
import java.security.NoSuchAlgorithmException;
    \end{lstlisting}
\end{frame}

\begin{frame}[fragile]
    \frametitle{Implementing SHA-256 - Part 2}
    \textbf{Step 2: Create a Method to Compute the Hash}
    \begin{lstlisting}[language=Java]
public class HashUtil {
    public static String sha256(String input) {
        try {
            MessageDigest digest = MessageDigest.getInstance("SHA-256");
            byte[] hashBytes = digest.digest(input.getBytes("UTF-8"));
            StringBuilder hexString = new StringBuilder();
            for (byte b : hashBytes) {
                String hex = Integer.toHexString(0xff & b);
                if (hex.length() == 1) hexString.append('0');
                hexString.append(hex);
            }
            return hexString.toString(); 
        } catch (NoSuchAlgorithmException | java.io.UnsupportedEncodingException e) {
            throw new RuntimeException(e);
        }
    }
}
    \end{lstlisting}
\end{frame}

\begin{frame}[fragile]
    \frametitle{Using the Hash Method}
    \textbf{Step 3: Example of Using the Hash Method}
    \begin{lstlisting}[language=Java]
public class Main {
    public static void main(String[] args) {
        String data = "Hello, World!";
        String hash = HashUtil.sha256(data);
        System.out.println("SHA-256 Hash: " + hash);
    }
}
    \end{lstlisting}
\end{frame}

\begin{frame}
    \frametitle{Key Points to Emphasize}
    \begin{itemize}
        \item \textbf{Input/Output}: Input can be any length; output is a fixed 64-character string for SHA-256.
        \item \textbf{Uniqueness}: A minor change in input results in a significantly different hash.
        \item \textbf{Security}: Hash functions are one-way; recovering the original input from a hash is infeasible.
    \end{itemize}
\end{frame}

\begin{frame}
    \frametitle{Use Cases of Hash Functions}
    \begin{itemize}
        \item \textbf{Data Integrity}: Verifying data has not changed (e.g., file verification).
        \item \textbf{Password Storage}: Storing hashes instead of plaintext for security.
        \item \textbf{Digital Signatures}: Ensuring document integrity and authenticity.
    \end{itemize}
    This slide provides a thorough understanding of implementing hash functions in Java, particularly focusing on SHA-256, helping students grasp both conceptual and practical aspects of cryptographic hash functions.
\end{frame}

\begin{frame}[fragile]
    \frametitle{Overview of Symmetric vs Asymmetric Cryptography}
    \begin{block}{Summary}
        Cryptography safeguards information using two primary types:
        \begin{itemize}
            \item \textbf{Symmetric Cryptography:} Uses a single key for encryption and decryption.
            \item \textbf{Asymmetric Cryptography:} Utilizes a pair of keys (public and private).
        \end{itemize}
        This presentation highlights their definitions, characteristics, common algorithms, usage scenarios, and examples in Java.
    \end{block}
\end{frame}

\begin{frame}[fragile]
    \frametitle{1. Symmetric Cryptography}
    \begin{block}{Definition}
        Symmetric cryptography (secret-key cryptography) uses a single key for both encryption and decryption.
    \end{block}
    \begin{itemize}
        \item \textbf{Key Characteristics:}
            \begin{itemize}
                \item \textbf{Speed:} Generally faster due to simpler computations.
                \item \textbf{Key Management:} Requires secure sharing and storage of the same key.
            \end{itemize}
        \item \textbf{Common Algorithms:}
            \begin{itemize}
                \item AES (Advanced Encryption Standard)
                \item DES (Data Encryption Standard)
                \item 3DES (Triple DES)
            \end{itemize}
        \item \textbf{Usage Scenarios:}
            \begin{itemize}
                \item Encrypting large volumes of data (e.g., files, databases).
                \item Scenarios with secure key exchange (e.g., internal networks).
            \end{itemize}
    \end{itemize}
\end{frame}

\begin{frame}[fragile]
    \frametitle{Example of Symmetric Cryptography in Java}
    \begin{lstlisting}[language=Java]
    import javax.crypto.Cipher;
    import javax.crypto.KeyGenerator;
    import javax.crypto.SecretKey;

    KeyGenerator keyGen = KeyGenerator.getInstance("AES");
    SecretKey secretKey = keyGen.generateKey();

    Cipher cipher = Cipher.getInstance("AES");
    cipher.init(Cipher.ENCRYPT_MODE, secretKey);
    byte[] encryptedData = cipher.doFinal("Sensitive Data".getBytes());
    \end{lstlisting}
\end{frame}

\begin{frame}[fragile]
    \frametitle{2. Asymmetric Cryptography}
    \begin{block}{Definition}
        Asymmetric cryptography (public-key cryptography) uses a pair of keys: a public key for encryption and a private key for decryption.
    \end{block}
    \begin{itemize}
        \item \textbf{Key Characteristics:}
            \begin{itemize}
                \item \textbf{Security:} Public key can be shared freely, enhancing key exchange security.
                \item \textbf{Speed:} Generally slower due to complex computations.
            \end{itemize}
        \item \textbf{Common Algorithms:}
            \begin{itemize}
                \item RSA (Rivest–Shamir–Adleman)
                \item DSA (Digital Signature Algorithm)
                \item ECC (Elliptic Curve Cryptography)
            \end{itemize}
        \item \textbf{Usage Scenarios:}
            \begin{itemize}
                \item Secure communications over insecure channels (e.g., emails, online transactions).
                \item Digital signatures and certificate validation.
            \end{itemize}
    \end{itemize}
\end{frame}

\begin{frame}[fragile]
    \frametitle{Example of Asymmetric Cryptography in Java}
    \begin{lstlisting}[language=Java]
    import java.security.KeyPairGenerator;
    import java.security.PrivateKey;
    import java.security.PublicKey;
    import javax.crypto.Cipher;

    KeyPairGenerator keyPairGen = KeyPairGenerator.getInstance("RSA");
    var keyPair = keyPairGen.generateKeyPair();
    PublicKey publicKey = keyPair.getPublic();
    PrivateKey privateKey = keyPair.getPrivate();

    Cipher cipher = Cipher.getInstance("RSA");
    cipher.init(Cipher.ENCRYPT_MODE, publicKey);
    byte[] encryptedData = cipher.doFinal("Sensitive Data".getBytes());
    \end{lstlisting}
\end{frame}

\begin{frame}[fragile]
    \frametitle{Conclusion and Key Points}
    \begin{itemize}
        \item \textbf{Symmetric vs. Asymmetric:}
            \begin{itemize}
                \item Symmetric: Same key for encryption/decryption; ideal for large data.
                \item Asymmetric: Different keys (public/private); optimal for secure key exchange.
            \end{itemize}
        \item \textbf{Performance Considerations:}
            \begin{itemize}
                \item Symmetric algorithms are faster for bulk data.
                \item Asymmetric algorithms provide enhanced security but are slower.
            \end{itemize}
        \item \textbf{Real-World Applications:}
            \begin{itemize}
                \item Use symmetric encryption for data at rest (files, databases).
                \item Use asymmetric encryption for key exchanges and securing communications.
            \end{itemize}
    \end{itemize}
\end{frame}

\begin{frame}[fragile]
    \frametitle{Implementing Symmetric Encryption - Introduction}
    \begin{itemize}
        \item Symmetric encryption uses a single key for both encryption and decryption.
        \item It is efficient and suitable for encrypting large volumes of data.
        \item \textbf{Common algorithms}: 
        \begin{itemize}
            \item Advanced Encryption Standard (AES) 
            \item Data Encryption Standard (DES) 
            \item Triple DES
        \end{itemize}
    \end{itemize}
\end{frame}

\begin{frame}[fragile]
    \frametitle{Implementing Symmetric Encryption - Why Use AES?}
    \begin{itemize}
        \item \textbf{Security}: AES is widely regarded as secure due to its strong encryption and key sizes (128, 192, 256 bits).
        \item \textbf{Efficiency}: AES is optimized for performance across different hardware and software platforms.
    \end{itemize}
\end{frame}

\begin{frame}[fragile]
    \frametitle{Implementing AES in Java - Dependencies}
    To use AES encryption in Java, ensure you have the following imports:
    \begin{lstlisting}[language=java]
    import javax.crypto.Cipher;
    import javax.crypto.KeyGenerator;
    import javax.crypto.SecretKey;
    import javax.crypto.spec.SecretKeySpec;
    import javax.crypto.spec.IvParameterSpec;
    import java.util.Base64;
    \end{lstlisting}
\end{frame}

\begin{frame}[fragile]
    \frametitle{Implementing AES in Java - Code Walkthrough}
    \textbf{1. Generate a Secret Key}
    \begin{lstlisting}[language=java]
    KeyGenerator keyGen = KeyGenerator.getInstance("AES");
    keyGen.init(256); // Key size
    SecretKey secretKey = keyGen.generateKey();
    \end{lstlisting}
    
    \textbf{2. Encrypt Data}
    \begin{lstlisting}[language=java]
    public static String encrypt(String plainText, SecretKey secretKey) throws Exception {
        Cipher cipher = Cipher.getInstance("AES/CBC/PKCS5Padding");
        IvParameterSpec ivParams = new IvParameterSpec(new byte[16]); // Initialization vector
        cipher.init(Cipher.ENCRYPT_MODE, secretKey, ivParams);
        byte[] encryptedBytes = cipher.doFinal(plainText.getBytes("UTF-8"));
        return Base64.getEncoder().encodeToString(encryptedBytes);
    }
    \end{lstlisting}
\end{frame}

\begin{frame}[fragile]
    \frametitle{Implementing AES in Java - Continued}
    \textbf{3. Decrypt Data}
    \begin{lstlisting}[language=java]
    public static String decrypt(String encryptedText, SecretKey secretKey) throws Exception {
        Cipher cipher = Cipher.getInstance("AES/CBC/PKCS5Padding");
        IvParameterSpec ivParams = new IvParameterSpec(new byte[16]); // Same IV used for encryption
        cipher.init(Cipher.DECRYPT_MODE, secretKey, ivParams);
        byte[] decryptedBytes = cipher.doFinal(Base64.getDecoder().decode(encryptedText));
        return new String(decryptedBytes, "UTF-8");
    }
    \end{lstlisting}
\end{frame}

\begin{frame}[fragile]
    \frametitle{Example Usage}
    Example code for using AES encryption and decryption:
    \begin{lstlisting}[language=java]
    public static void main(String[] args) throws Exception {
        SecretKey secretKey = KeyGenerator.getInstance("AES").generateKey();
        String originalText = "Hello, World!";
        
        String encryptedText = encrypt(originalText, secretKey);
        System.out.println("Encrypted: " + encryptedText);
        
        String decryptedText = decrypt(encryptedText, secretKey);
        System.out.println("Decrypted: " + decryptedText);
    }
    \end{lstlisting}
\end{frame}

\begin{frame}[fragile]
    \frametitle{Implementing Symmetric Encryption - Key Points}
    \begin{itemize}
        \item \textbf{Key Management}: Properly store and manage keys; exposing keys compromises security.
        \item \textbf{IV Usage}: Always use a unique IV for each encryption operation to enhance security.
        \item \textbf{Padding Scheme}: CBC mode with PKCS5Padding is widely used; understand the implications of your padding scheme.
    \end{itemize}
\end{frame}

\begin{frame}[fragile]
    \frametitle{Best Practices in Symmetric Encryption}
    \begin{itemize}
        \item \textbf{Key Size}: Use at least AES-128 for production; AES-256 offers further security.
        \item \textbf{Configuration}: Review and configure the encryption parameters carefully (mode, padding).
        \item \textbf{Handling Exceptions}: Implement robust exception handling, especially for encryption and decryption processes.
    \end{itemize}
\end{frame}

\begin{frame}[fragile]
    \frametitle{Conclusion}
    Mastering symmetric encryption, particularly with AES, is critical for secure data handling in Java applications. By following best practices and understanding the mechanics behind the code, developers can effectively implement secure encryption solutions.
\end{frame}

\begin{frame}[fragile]
    \frametitle{Implementing Asymmetric Encryption - Introduction}
    \begin{itemize}
        \item \textbf{Definition}: Asymmetric encryption uses a pair of keys—one public and one private.
        \item \textbf{Key Properties}:
        \begin{itemize}
            \item \textbf{Confidentiality}: Only someone with the private key can decrypt messages encrypted with the public key.
            \item \textbf{Authentication}: Digital signatures can validate the authenticity of the sender.
        \end{itemize}
    \end{itemize}
\end{frame}

\begin{frame}[fragile]
    \frametitle{Implementing Asymmetric Encryption - RSA Algorithm}
    \begin{itemize}
        \item \textbf{Overview}: RSA is the most widely used asymmetric encryption algorithm based on large prime numbers.
        \item \textbf{Key Generation}:
        \begin{enumerate}
            \item Choose two large prime numbers, \( p \) and \( q \).
            \item Compute \( n = p \times q \) (modulus for both keys).
            \item Compute Euler's totient: \( \phi(n) = (p-1)(q-1) \).
            \item Choose an integer \( e \) such that \( 1 < e < \phi(n) \) and \( \gcd(e, \phi(n)) = 1 \).
            \item Compute \( d \) as the modular inverse of \( e \mod \phi(n) \).
        \end{enumerate}
    \end{itemize}
\end{frame}

\begin{frame}[fragile]
    \frametitle{Implementing Asymmetric Encryption - RSA in Java}
    \begin{block}{Example: RSA Algorithm Implementation}
    \begin{lstlisting}[language=Java]
import java.security.KeyPair;
import java.security.KeyPairGenerator;
import java.security.PrivateKey;
import java.security.PublicKey;
import javax.crypto.Cipher;

public class RSAExample {
    public static void main(String[] args) throws Exception {
        // Key Generation
        KeyPairGenerator keyGen = KeyPairGenerator.getInstance("RSA");
        keyGen.initialize(2048); // Key size
        KeyPair pair = keyGen.generateKeyPair();
        PublicKey publicKey = pair.getPublic();
        PrivateKey privateKey = pair.getPrivate();

        // Encryption
        Cipher cipher = Cipher.getInstance("RSA");
        cipher.init(Cipher.ENCRYPT_MODE, publicKey);
        String message = "Hello, this is a secret message!";
        byte[] encryptedMessage = cipher.doFinal(message.getBytes());

        // Decryption
        cipher.init(Cipher.DECRYPT_MODE, privateKey);
        byte[] decryptedMessage = cipher.doFinal(encryptedMessage);
        System.out.println("Decrypted Message: " + new String(decryptedMessage));
    }
}
    \end{lstlisting}
    \end{block}
\end{frame}

\begin{frame}[fragile]
    \frametitle{Implementing Asymmetric Encryption - Key Points and Conclusion}
    \begin{itemize}
        \item \textbf{Security Strength}: RSA is secure based on the difficulty of factoring large numbers.
        \item \textbf{Performance}: Asymmetric encryption is slower than symmetric encryption and is often paired with symmetric techniques.
        \item \textbf{Use of Libraries}: Java provides built-in libraries (e.g., \texttt{java.security}, \texttt{javax.crypto}) for cryptographic implementations.
    \end{itemize}
    \begin{block}{Conclusion}
        Asymmetric encryption, especially RSA, is critical for secure communications. Understanding its implementation in Java empowers developers to build secure applications.
    \end{block}
\end{frame}

\begin{frame}
    \frametitle{Cryptographic Protocols in Java}
    \begin{itemize}
        \item Importance of cryptographic protocols for secure communication
        \item Overview of TLS and SSL
        \item Implementation in Java using libraries like JSSE and Bouncy Castle
    \end{itemize}
\end{frame}

\begin{frame}
    \frametitle{Understanding Cryptographic Protocols}
    \begin{block}{Definition}
        Cryptographic protocols are essential for securing communications over networks, providing:
        \begin{itemize}
            \item Confidentiality
            \item Integrity
            \item Authentication
            \item Non-repudiation
        \end{itemize}
    \end{block}
    \begin{block}{Protocols Overview}
        \begin{itemize}
            \item \textbf{SSL}: Original protocol, now largely deprecated.
            \item \textbf{TLS}: Successful protocol with enhanced security, primarily used today.
        \end{itemize}
    \end{block}
\end{frame}

\begin{frame}
    \frametitle{Implementing TLS/SSL in Java}
    \begin{itemize}
        \item \textbf{Java Secure Socket Extension (JSSE)}:
            \begin{itemize}
                \item API for secure communication
                \item Supports client and server-side
            \end{itemize}
        \item \textbf{Bouncy Castle}:
            \begin{itemize}
                \item Lightweight API for cryptography
                \item Extends Java's cryptographic capabilities
            \end{itemize}
    \end{itemize}
\end{frame}

\begin{frame}[fragile]
    \frametitle{Example: Simple SSL Server}
    \begin{lstlisting}[language=Java]
import javax.net.ssl.*;
import java.security.KeyStore;

public class SSLServer {
    public static void main(String[] args) throws Exception {
        KeyStore ks = KeyStore.getInstance("JKS");
        ks.load(new FileInputStream("keystore.jks"), "password".toCharArray());
        
        KeyManagerFactory kmf = KeyManagerFactory.getInstance(KeyManagerFactory.getDefaultAlgorithm());
        kmf.init(ks, "password".toCharArray());

        SSLContext sslContext = SSLContext.getInstance("TLS");
        sslContext.init(kmf.getKeyManagers(), null, null);
        
        SSLServerSocketFactory ssf = sslContext.getServerSocketFactory();
        SSLServerSocket s = (SSLServerSocket) ssf.createServerSocket(8443);

        while (true) {
            SSLSocket socket = (SSLSocket) s.accept();
            System.out.println("Client connected!");
            socket.close();
        }
    }
}
    \end{lstlisting}
\end{frame}

\begin{frame}
    \frametitle{Key Points to Emphasize}
    \begin{itemize}
        \item \textbf{Security Best Practices}:
            \begin{itemize}
                \item Use the latest TLS version
                \item Regularly update certificates and use strong algorithms
            \end{itemize}
        \item \textbf{Performance Considerations}:
            \begin{itemize}
                \item Encryption can introduce latency
                \item Balance security and performance needs
            \end{itemize}
        \item \textbf{Error Handling}:
            \begin{itemize}
                \item Implement robust handling during SSL connections
            \end{itemize}
    \end{itemize}
\end{frame}

\begin{frame}
    \frametitle{Conclusion}
    Understanding and implementing cryptographic protocols like TLS/SSL in Java is crucial for securing communications. Utilizing libraries such as JSSE simplifies this process while adhering to security best practices. Continuous education and strong protocol implementation remain vital as threats evolve.
\end{frame}

\begin{frame}[fragile]
    \frametitle{Testing and Validating Cryptographic Implementations - Introduction}
    \begin{block}{Introduction to Cryptographic Testing}
        Cryptography is foundational to security in modern applications. Given its high stakes, it’s crucial to test and validate cryptographic implementations thoroughly. This ensures that the algorithms behave as expected and provide the necessary level of security.
    \end{block}
\end{frame}

\begin{frame}[fragile]
    \frametitle{Testing and Validating Cryptographic Implementations - Importance}
    \begin{block}{Why is Testing Crucial?}
        \begin{itemize}
            \item \textbf{Vulnerabilities}: Even small mistakes in cryptographic code can introduce vulnerabilities that may be exploited by attackers.
            \item \textbf{Compliance}: Certain industries and standards (like PCI-DSS) require regular testing of cryptographic implementations.
            \item \textbf{Standards and Regulations}: Adhering to standards helps maintain trust and security throughout the software lifecycle.
        \end{itemize}
    \end{block}
\end{frame}

\begin{frame}[fragile]
    \frametitle{Testing and Validating Cryptographic Implementations - Key Methods}
    \begin{block}{Key Testing Methods}
        \begin{enumerate}
            \item \textbf{Unit Testing}
            \begin{itemize}
                \item \textbf{Purpose}: Validate individual components of cryptographic functions (e.g., key generation, encryption, decryption).
                \item \textbf{Example}: Test that decryption produces the original plaintext.
                \begin{lstlisting}[language=java]
                @Test
                public void testAESEncryption() {
                    String plainText = "Hello World";
                    String key = "1234567890123456"; // 16 bytes key for AES
                    String encrypted = AESEncrypt(plainText, key);
                    String decrypted = AESDecrypt(encrypted, key);
                    assertEquals(plainText, decrypted); // Validate that original text is retrieved
                }
                \end{lstlisting}
            \end{itemize}
            \item \textbf{Integration Testing}
            \begin{itemize}
                \item \textbf{Purpose}: Ensure components work together correctly within the cryptographic framework.
            \end{itemize}
            \item \textbf{Fuzz Testing}
            \begin{itemize}
                \item \textbf{Purpose}: Use random data to test inputs on cryptographic functions to detect unexpected behavior.
            \end{itemize}
            \item \textbf{Formal Verification}
            \begin{itemize}
                \item \textbf{Purpose}: Mathematically prove that an algorithm adheres to its specification.
            \end{itemize}
        \end{enumerate}
    \end{block}
\end{frame}

\begin{frame}[fragile]
    \frametitle{Testing and Validating Cryptographic Implementations - Secure Coding}
    \begin{block}{Secure Coding Practices}
        \begin{itemize}
            \item \textbf{Use Established Libraries}: Rely on vetted libraries (like Bouncy Castle or Java’s built-in libraries) rather than implementing algorithms from scratch.
            \item \textbf{Avoid Hardcoding Secrets}: Never store cryptographic keys or secrets directly in code.
            \item \textbf{Peer Review and Code Audits}: Have multiple eyes on critical code components to surface potential vulnerabilities.
        \end{itemize}
    \end{block}
\end{frame}

\begin{frame}[fragile]
    \frametitle{Testing and Validating Cryptographic Implementations - Conclusion}
    \begin{block}{Conclusion and Key Takeaways}
        \begin{itemize}
            \item \textbf{Thorough Testing is Essential}: High-stakes environments require rigorous testing to ensure security.
            \item \textbf{Diverse Test Methods}: Employ multiple testing strategies for comprehensive validation.
            \item \textbf{Adopt Secure Coding Practices}: Trust in established libraries and prioritize secure coding to minimize risk.
        \end{itemize}
        By following these principles of rigorous testing and secure coding, developers can bolster the integrity of their cryptographic implementations and enhance the security of their applications.
    \end{block}
\end{frame}

\begin{frame}[fragile]
    \frametitle{Best Practices for Implementing Cryptography in Java - Part 1}
    \begin{enumerate}
        \item \textbf{Use Established Libraries}:
        \begin{itemize}
            \item Prefer well-reviewed libraries like Bouncy Castle or Java Cryptography Extension (JCE).
            \item These libraries undergo constant scrutiny and updates for security vulnerabilities.
        \end{itemize}
        
        \item \textbf{Implement Strong Key Management}:
        \begin{itemize}
            \item Use secure storage solutions for keys (e.g., KeyStore).
            \item Avoid hardcoding keys in the source code.
        \end{itemize}        
    \end{enumerate}
\end{frame}

\begin{frame}[fragile]
    \frametitle{Best Practices for Implementing Cryptography in Java - Part 2}
    \begin{enumerate}
        \setcounter{enumi}{2} % Continue numbering from the previous frame
        \item \textbf{Adopt Secure Coding Practices}:
        \begin{itemize}
            \item Validate input thoroughly to prevent injection attacks.
            \item Use well-known patterns and avoid custom cryptographic algorithms.
        \end{itemize}
        
        \item \textbf{Adopt Proper Padding for Block Ciphers}:
        \begin{itemize}
            \item Use padding schemes like PKCS5 or PKCS7 for secure data handling.
            \item Ensure block sizes match and handle exceptions appropriately.
        \end{itemize}
    \end{enumerate}
\end{frame}

\begin{frame}[fragile]
    \frametitle{Future Directions in Cryptography}
    \begin{enumerate}
        \item \textbf{Post-Quantum Cryptography}:
        \begin{itemize}
            \item Traditional algorithms (RSA, ECC) may become vulnerable to quantum computing.
            \item Research is focusing on developing new algorithms resistant to quantum attacks.
        \end{itemize}
        
        \item \textbf{Homomorphic Encryption}:
        \begin{itemize}
            \item Enables computations on encrypted data without decryption.
            \item Applications include secure data processing in cloud services.
        \end{itemize}

        \item \textbf{Blockchain and Cryptography}:
        \begin{itemize}
            \item Cryptographic principles support secure transactions and decentralized verification.
            \item Emerging trends in smart contracts that utilize cryptography.
        \end{itemize}

        \item \textbf{Zero-Knowledge Proofs}:
        \begin{itemize}
            \item Allow one party to prove to another that they know a value, without revealing the value.
            \item Important for privacy and security in identity verification.
        \end{itemize}
    \end{enumerate}
\end{frame}

\begin{frame}[fragile]
    \frametitle{Key Points to Remember}
    \begin{itemize}
        \item \textbf{Security is an Ongoing Process}: Regular updates, code reviews, and assessments are essential to maintain cryptographic security.
        \item \textbf{Education and Awareness}: Keeping informed about the latest threats and advancements in cryptography is crucial for effective implementation.
        \item \textbf{Collaboration}: Engage with the community to share findings, receive feedback, and improve security measures collectively.
    \end{itemize}
\end{frame}


\end{document}