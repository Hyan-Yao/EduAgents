\documentclass{beamer}

% Theme choice
\usetheme{Madrid} % You can change to e.g., Warsaw, Berlin, CambridgeUS, etc.

% Encoding and font
\usepackage[utf8]{inputenc}
\usepackage[T1]{fontenc}

% Graphics and tables
\usepackage{graphicx}
\usepackage{booktabs}

% Code listings
\usepackage{listings}
\lstset{
    basicstyle=\ttfamily\small,
    keywordstyle=\color{blue},
    commentstyle=\color{gray},
    stringstyle=\color{red},
    breaklines=true,
    frame=single
}

% Math packages
\usepackage{amsmath}
\usepackage{amssymb}

% Colors
\usepackage{xcolor}

% TikZ and PGFPlots
\usepackage{tikz}
\usepackage{pgfplots}
\pgfplotsset{compat=1.18}
\usetikzlibrary{positioning}

% Hyperlinks
\usepackage{hyperref}

% Title information
\title{Chapter 11: Emerging Technologies in Cryptography}
\author{Your Name}
\institute{Your Institution}
\date{\today}

\begin{document}

\frame{\titlepage}

\begin{frame}[fragile]
    \frametitle{Introduction to Emerging Technologies in Cryptography}
    \begin{block}{Overview}
        Cryptography is essential for modern security, ensuring privacy, integrity, and authenticity of information. With advancements in technology, especially in computing and networking, new cryptographic technologies have significantly enhanced security measures.
    \end{block}
\end{frame}

\begin{frame}[fragile]
    \frametitle{Key Concepts in Cryptography - Part 1}
    \begin{enumerate}
        \item \textbf{Post-Quantum Cryptography}
            \begin{itemize}
                \item Traditional algorithms (e.g., RSA, ECC) are at risk from quantum computers.
                \item Designed to be secure against both classical and quantum computer attacks.
                \item \textit{Example:} Lattice-based cryptography (e.g., NTRU, Learning With Errors (LWE)).
            \end{itemize}
        \item \textbf{Homomorphic Encryption}
            \begin{itemize}
                \item Allows computations on encrypted data without decryption.
                \item \textit{Example:} Cloud services perform calculations on encrypted user data, preserving privacy.
            \end{itemize}
    \end{enumerate}
\end{frame}

\begin{frame}[fragile]
    \frametitle{Key Concepts in Cryptography - Part 2}
    \begin{enumerate}
        \setcounter{enumi}{2}
        \item \textbf{Zero-Knowledge Proofs}
            \begin{itemize}
                \item Allows a prover to convince a verifier of knowledge without revealing the value.
                \item \textit{Example:} Used in blockchain technologies to validate transactions while keeping details confidential.
            \end{itemize}
        \item \textbf{Blockchain and Decentralized Identity}
            \begin{itemize}
                \item Enhances data integrity and security using a decentralized platform.
                \item Users control their identity data without relying on centralized authorities.
            \end{itemize}
    \end{enumerate}
\end{frame}

\begin{frame}[fragile]
    \frametitle{Significance of Recent Advances}
    \begin{itemize}
        \item \textbf{Enhanced Security:} New technologies provide higher security in the face of rising cyber threats.
        \item \textbf{Privacy Preservation:} Homomorphic encryption and zero-knowledge proofs allow data analysis without compromising privacy.
        \item \textbf{Trust in Transactions:} Blockchain enhances trust through transparency and immutability.
    \end{itemize}
\end{frame}

\begin{frame}[fragile]
    \frametitle{Key Points and Closing Note}
    \begin{itemize}
        \item Cryptography is evolving to counter emerging threats, especially from quantum computing.
        \item New technologies empower users, giving them control over their data and identities.
        \item Understanding these advancements is vital for cybersecurity professionals and organizations.
    \end{itemize}
    \begin{block}{Closing Note}
        Next, we will explore \textbf{Quantum Cryptography}, examining its principles, key distribution methods, and advantages over traditional systems.
    \end{block}
\end{frame}

\begin{frame}[fragile]
    \frametitle{Quantum Cryptography - Overview}
    \begin{block}{Introduction to Quantum Cryptography}
        Quantum cryptography leverages the principles of quantum mechanics to secure data. 
        Unlike classical cryptographic methods, which rely on the computational complexity 
        of certain mathematical problems, quantum cryptography provides a fundamentally secure 
        way to communicate using quantum bits (qubits).
    \end{block}
\end{frame}

\begin{frame}[fragile]
    \frametitle{Quantum Cryptography - Key Principles}
    \begin{itemize}
        \item \textbf{Quantum Superposition:} Qubits can exist in multiple states, allowing for richer information encoding.
        \item \textbf{Quantum Entanglement:} Entangled qubits influence each other instantaneously, aiding in eavesdropping detection.
        \item \textbf{Heisenberg Uncertainty Principle:} Measuring a quantum state changes it, revealing the presence of an eavesdropper.
    \end{itemize}
\end{frame}

\begin{frame}[fragile]
    \frametitle{Quantum Key Distribution (QKD)}
    \begin{block}{QKD Overview}
        Quantum Key Distribution is a method for secure key exchange. A common protocol used is \textbf{BB84}.
    \end{block}
    \begin{enumerate}
        \item Alice sends randomly polarized photons to Bob.
        \item Bob randomly measures the photons' polarization.
        \item Both compare results to detect eavesdropping.
        \item If secure, they create a shared key.
    \end{enumerate}
\end{frame}

\begin{frame}[fragile]
    \frametitle{Quantum Cryptography - Advantages}
    \begin{itemize}
        \item \textbf{Unconditional Security:} Security is retained even with advancements in computing (e.g., quantum computers).
        \item \textbf{Eavesdropping Detection:} Any interception attempt is detectable, allowing for immediate key discarding.
        \item \textbf{No Mathematical Assumptions:} Provides stronger security guarantees without relying on hard mathematical problems.
    \end{itemize}
\end{frame}

\begin{frame}[fragile]
    \frametitle{Quantum Cryptography - Conclusion}
    \begin{block}{Key Takeaways}
        Quantum cryptography represents a shift in secure communications using principles like superposition and entanglement.
        \begin{itemize}
            \item Emerging real-world applications are advancing as technology progresses.
            \item Integration with classical networks is an area of ongoing research.
        \end{itemize}
    \end{block}
    \begin{block}{Next Steps}
        Dive deeper into QKD protocols in the upcoming slides.
    \end{block}
\end{frame}

\begin{frame}[fragile]
    \frametitle{Quantum Key Distribution (QKD)}
    \begin{block}{Understanding Quantum Key Distribution}
        Quantum Key Distribution (QKD) is a revolutionary method of securing communication by leveraging quantum mechanics principles. Unlike classical key exchange methods, QKD allows for the generation and sharing of encryption keys securely, guaranteeing protection against potential eavesdroppers.
    \end{block}
\end{frame}

\begin{frame}[fragile]
    \frametitle{Key Mechanisms of QKD: The BB84 Protocol}
    \begin{block}{Introduction to BB84}
        Developed by Bennett and Brassard in 1984, the BB84 protocol is the first and most well-known QKD protocol. It allows two parties (Alice and Bob) to share a secret key while detecting any eavesdropping.
    \end{block}
    
    \begin{block}{Basic Principles}
        \begin{itemize}
            \item Quantum Bits (Qubits): Information is encoded using the quantum states of photons.
            \item Polarization States: Qubits can represent quantum states like vertical/horizontal (0/1) and diagonal (superposition).
        \end{itemize}
    \end{block}
\end{frame}

\begin{frame}[fragile]
    \frametitle{BB84 Protocol Steps}
    \begin{enumerate}
        \item \textbf{Preparation:} Alice sends qubits to Bob, chosen randomly from two basis sets (rectilinear or diagonal).
        \item \textbf{Measurement:} Bob measures qubits using randomly selected bases and records results and bases.
        \item \textbf{Basis Reconciliation:} Alice and Bob share their chosen bases, discarding mismatched measurements.
        \item \textbf{Key Generation:} Matched basis results provide the shared secret key.
        \item \textbf{Eavesdropping Detection:} Eavesdropping attempts disturb quantum states, detectable via error rates.
    \end{enumerate}
\end{frame}

\begin{frame}[fragile]
    \frametitle{Real-World Applications of QKD}
    \begin{itemize}
        \item \textbf{Government and Military Communications:} Secure channels for sensitive information protection.
        \item \textbf{Financial Sector:} Banks utilize QKD for transaction security and customer data protection.
        \item \textbf{Future Internet Security:} Research into QKD for a quantum internet ensuring global secure communication.
    \end{itemize}
\end{frame}

\begin{frame}[fragile]
    \frametitle{Key Points to Emphasize}
    \begin{itemize}
        \item QKD offers unconditional security, unlike classical cryptography reliant on computational hardness.
        \item Success hinges on quantum mechanics principles such as superposition and entanglement.
        \item Robust infrastructure and advanced technology are essential for real-world implementations.
    \end{itemize}
\end{frame}

\begin{frame}[fragile]
    \frametitle{Conclusion}
    By understanding and implementing QKD mechanisms like BB84, we pave the way for a secure future in digital communication, ensuring that our data remains confidential against advanced threats.
\end{frame}

\begin{frame}[fragile]
    \frametitle{Impacts of Quantum Computing on Cryptography - Introduction}
    \begin{itemize}
        \item \textbf{Quantum Computing Concept}: Utilizes principles of quantum mechanics for information processing.
        \item \textbf{Key Feature}: Qubits can exist in multiple states, providing immense parallel processing power.
    \end{itemize}
\end{frame}

\begin{frame}[fragile]
    \frametitle{Impacts of Quantum Computing on Cryptography - Threat to Traditional Algorithms}
    \begin{itemize}
        \item \textbf{Vulnerability of RSA and ECC}:
        \begin{itemize}
            \item \textbf{RSA (Rivest–Shamir–Adleman)}:
            \begin{itemize}
                \item Current security relies on difficulty of factoring large primes.
                \item \textbf{Quantum Risk}: Shor's algorithm can efficiently break RSA.
            \end{itemize}
            \item \textbf{ECC (Elliptic Curve Cryptography)}:
            \begin{itemize}
                \item Based on the difficulty of the discrete logarithm problem.
                \item \textbf{Quantum Risk}: Also vulnerable to Shor's algorithm.
            \end{itemize}
        \end{itemize}
        \item \textbf{Example}: RSA-2048 could be broken in a few hours by a powerful quantum computer.
    \end{itemize}
\end{frame}

\begin{frame}[fragile]
    \frametitle{Impacts of Quantum Computing on Cryptography - Post-Quantum Cryptography Necessity}
    \begin{itemize}
        \item \textbf{Post-Quantum Cryptography Definition}: Algorithms secure against quantum attacks.
        \item \textbf{Key Considerations}:
        \begin{itemize}
            \item Development of new algorithms: Lattice-based, hash-based, code-based.
            \item Standardization efforts: NIST working on post-quantum algorithms.
        \end{itemize}
        \item \textbf{Key Examples}:
        \begin{itemize}
            \item \textbf{Lattice-based}: NTRUEncrypt - Hard problems in lattice structures.
            \item \textbf{Hash-based}: XMSS - Utilizes hash functions for signing messages.
        \end{itemize}
    \end{itemize}
\end{frame}

\begin{frame}[fragile]
    \frametitle{Impacts of Quantum Computing on Cryptography - Key Points and Conclusion}
    \begin{itemize}
        \item Quantum computing threatens RSA and ECC security.
        \item Transition to post-quantum cryptography is essential for data security.
        \item Development of quantum-resistant algorithms requires global collaboration.
    \end{itemize}

    \begin{block}{Conclusion}
        Preparing for a post-quantum world is crucial for securing digital communications globally.
    \end{block}
\end{frame}

\begin{frame}[fragile]
    \frametitle{Blockchain Technology - Overview}
    \begin{block}{Definition}
        Blockchain is a decentralized, distributed ledger technology that securely records transactions across many computers. 
        This ensures that no single entity has control, and all transactions are transparent and immutable.
    \end{block}
\end{frame}

\begin{frame}[fragile]
    \frametitle{Blockchain Technology - Structure}
    \begin{itemize}
        \item \textbf{Blocks:}
        \begin{itemize}
            \item \textbf{Header:} Includes metadata such as the block version, timestamp, and a reference (hash) to the previous block.
            \item \textbf{Transaction Data:} Stores the actual transactions or data being validated.
            \item \textbf{Hash:} A unique fingerprint of the block's contents generated by a cryptographic hash function, ensuring data integrity.
        \end{itemize}
        
        \item \textbf{Chain:}
        \begin{itemize}
            \item Blocks are linked together in chronological order, each referencing the previous block's hash.
        \end{itemize}
        
        \item \textbf{Nodes:}
        \begin{itemize}
            \item Participants in the blockchain network, each holding a copy of the entire blockchain, working together to validate new transactions.
        \end{itemize}
    \end{itemize}
\end{frame}

\begin{frame}[fragile]
    \frametitle{Blockchain Technology - Role in Security}
    \begin{itemize}
        \item \textbf{Decentralization:} Eliminates single points of failure, enhancing resilience against attacks or data breaches.
        
        \item \textbf{Immutability:} Data added to the blockchain cannot be modified or deleted without network consensus.
        
        \item \textbf{Transparency:} Transactions are visible on the ledger to all participants, fostering trust without needing intermediaries.
        
        \item \textbf{Security:} Cryptographic techniques secure transactions while maintaining anonymity through public keys.
    \end{itemize}
\end{frame}

\begin{frame}[fragile]
    \frametitle{Blockchain Technology - Example}
    \begin{enumerate}
        \item Alice wants to send 1 Bitcoin to Bob.
        \item The transaction is bundled with others into a block.
        \item Nodes validate this block using cryptographic consensus mechanisms (e.g., Proof of Work).
        \item Once validated, the block is added to the chain, and the transaction is secure and immutable.
    \end{enumerate}
\end{frame}

\begin{frame}[fragile]
    \frametitle{Blockchain Technology - Key Points}
    \begin{itemize}
        \item Blockchain is a \textbf{distributed ledger} that enhances security and trust.
        \item The \textbf{chain structure} supports integrity and validity of transactions through cryptographic hash linking.
        \item Applicable in various sectors beyond cryptocurrencies, including supply chain management, healthcare, and voting systems.
    \end{itemize}
\end{frame}

\begin{frame}[fragile]
    \frametitle{Cryptographic Algorithms in Blockchain - Overview}
    Cryptography is the backbone of blockchain technology, ensuring the integrity, authenticity, and security of transactions. The two primary cryptographic techniques employed in blockchain are:
    \begin{itemize}
        \item \textbf{Hashing Functions}
        \item \textbf{Digital Signatures}
    \end{itemize}
\end{frame}

\begin{frame}[fragile]
    \frametitle{Cryptographic Algorithms in Blockchain - Hashing Functions}
    \textbf{Definition}: A hashing function transforms input data of any size into a fixed-size output, known as a hash value or digest. This process is crucial for encapsulating transaction data within a blockchain.

    \textbf{Key Properties}:
    \begin{itemize}
        \item \textbf{Deterministic}: Same input produces the same hash output.
        \item \textbf{Fast Computation}: Quick to compute the hash for any given data.
        \item \textbf{Pre-image Resistance}: Difficult to find the original input from a hash.
        \item \textbf{Small Changes, Big Impact}: Minor alterations in input result in different hashes.
        \item \textbf{Collision Resistance}: Unlikely for two different inputs to yield the same output.
    \end{itemize}
    
    \textbf{Example}: 
    \begin{itemize}
        \item \textbf{SHA-256}: Used in Bitcoin.
            \begin{itemize}
                \item Input: "Hello, World!"
                \item SHA-256 Hash: \texttt{a591a6d40bf420404a11d194f1f191c5e89b7b8e10d1e9fabe77b2b55b0f1729}
            \end{itemize}
    \end{itemize}
\end{frame}

\begin{frame}[fragile]
    \frametitle{Cryptographic Algorithms in Blockchain - Digital Signatures}
    \textbf{Definition}: Digital signatures are cryptographic equivalents of handwritten signatures, providing proof of authenticity and integrity of digital messages.

    \textbf{Key Properties}:
    \begin{itemize}
        \item \textbf{Authenticity}: Confirms the sender’s identity.
        \item \textbf{Integrity}: Ensures message is unaltered during transit.
        \item \textbf{Non-repudiation}: A signer cannot deny having signed the transaction.
    \end{itemize}

    \textbf{Process}:
    \begin{enumerate}
        \item \textbf{Key Pair Generation}: Private key (secret) and public key (shared).
        \item \textbf{Signing Process}: Sender hashes transaction data, encrypts the hash using the private key.
        \item \textbf{Verification Process}: Recipient uses the public key to confirm the sender’s identity and message integrity.
    \end{enumerate}

    \textbf{Example}: Alice sends a transaction to Bob:
    \begin{itemize}
        \item Creates a hash of transaction data, signs it, and sends both to Bob.
        \item Bob verifies the signature using Alice’s public key.
    \end{itemize}
\end{frame}

\begin{frame}[fragile]
    \frametitle{Cryptographic Algorithms in Blockchain - Key Points and Summary}
    \textbf{Key Points to Emphasize}:
    \begin{itemize}
        \item \textbf{Security Foundation}: Vital for securing cryptocurrencies and ensuring transaction authenticity.
        \item \textbf{Immutable Ledger}: Hash functions create an immutable ledger; altering transactions requires rehashing subsequent blocks.
        \item \textbf{Public and Private Keys}: Critical for secure identity verification across the blockchain.
    \end{itemize}

    \textbf{Summary}: 
    In summary, hashing functions and digital signatures are essential components of blockchain technology that ensure data integrity, authentication, and overall security within decentralized systems. Understanding these algorithms is crucial for comprehending blockchain operations and maintaining trust among users.
\end{frame}

\begin{frame}[fragile]
    \frametitle{Implications of Blockchain for Security}
    Analysis of the security benefits and challenges posed by blockchain technology in various sectors.
\end{frame}

\begin{frame}[fragile]
    \frametitle{Overview of Blockchain Security}
    Blockchain is revolutionizing the realm of cybersecurity by providing a decentralized and immutable ledger system. 
    \begin{itemize}
        \item Each block contains:
        \begin{itemize}
            \item A cryptographic hash of the previous block
            \item A timestamp
            \item Transaction data
        \end{itemize}
        \item Establishing a secure chain of information.
        \item Offers several security benefits alongside notable challenges.
    \end{itemize}
\end{frame}

\begin{frame}[fragile]
    \frametitle{Security Benefits of Blockchain}
    \begin{enumerate}
        \item \textbf{Decentralization}
            \begin{itemize}
                \item Reduces risk of single points of failure.
                \item Example: Decentralized finance (DeFi) facilitates peer-to-peer transactions without intermediaries.
            \end{itemize}
        
        \item \textbf{Immutability}
            \begin{itemize}
                \item Data cannot be altered or deleted once recorded.
                \item Illustration: Supply chain monitoring logs product journeys transparently.
            \end{itemize}
        
        \item \textbf{Transparency}
            \begin{itemize}
                \item Transactions are publicly visible on public blockchains.
                \item Example: Charities can provide transparent donation records.
            \end{itemize}
        
        \item \textbf{Cryptographic Security}
            \begin{itemize}
                \item Uses advanced cryptographic techniques for securing data.
                \item Key Point: Hash functions (e.g., SHA-256) detect unauthorized alterations.
            \end{itemize}
    \end{enumerate}
\end{frame}

\begin{frame}[fragile]
    \frametitle{Challenges of Blockchain Security}
    \begin{enumerate}
        \item \textbf{51\% Attack}
            \begin{itemize}
                \item Control of over half of network's power can manipulate the blockchain.
                \item More pronounced in smaller blockchains.
            \end{itemize}

        \item \textbf{Smart Contract Vulnerabilities}
            \begin{itemize}
                \item Bugs in self-executing contracts can be exploited.
                \item Case Study: Ethereum's DAO hack revealed such vulnerabilities.
            \end{itemize}

        \item \textbf{User Mismanagement}
            \begin{itemize}
                \item Security depends on user practices (e.g., safeguarding private keys).
                \item Example: Loss of private keys results in irreversible assets loss.
            \end{itemize}

        \item \textbf{Regulatory Uncertainty}
            \begin{itemize}
                \item Evolving regulations pose legal risks.
                \item Compliance with laws like AML and KYC is a major concern.
            \end{itemize}
    \end{enumerate}
\end{frame}

\begin{frame}[fragile]
    \frametitle{Conclusion and Key Takeaway}
    While blockchain offers profound security benefits through:
    \begin{itemize}
        \item Decentralization
        \item Immutability
        \item Transparency
        \item Cryptographic defenses
    \end{itemize}
    it still faces challenges that require careful consideration.
    
    \textbf{Key Takeaway:} Engaging with blockchain necessitates balancing its transformative security features with awareness of risks and taking appropriate mitigation measures.
\end{frame}

\begin{frame}[fragile]
    \frametitle{Comparative Analysis: Quantum Cryptography vs. Blockchain}
    In this slide, we will compare \textbf{Quantum Cryptography} and \textbf{Blockchain} based on:
    \begin{itemize}
        \item Security Features
        \item Use Cases
        \item Future Prospects
    \end{itemize}
    Both technologies represent significant advancements in cryptography but operate on fundamentally different principles.
\end{frame}

\begin{frame}[fragile]
    \frametitle{1. Security Features}
    \textbf{Quantum Cryptography:}
    \begin{itemize}
        \item \textbf{Principle of Quantum Mechanics:} Utilizes quantum mechanics for data security, primarily through Quantum Key Distribution (QKD).
        \item \textbf{Unconditional Security:} Security is theoretically unbreakable; any eavesdropping disturbs quantum states, alerting parties involved.
    \end{itemize}

    \textbf{Blockchain:}
    \begin{itemize}
        \item \textbf{Decentralization:} Uses a distributed ledger architecture, making it resistant to record alterations by malicious actors.
        \item \textbf{Cryptography in Use:} Employs cryptographic hashing and digital signatures, secured via consensus mechanisms (e.g., Proof of Work, Proof of Stake).
    \end{itemize}
    
    \textit{Key Point:} Quantum Cryptography offers theoretical unbreakability, whereas Blockchain relies on decentralization and cryptographic methods for security.
\end{frame}

\begin{frame}[fragile]
    \frametitle{2. Use Cases and Future Prospects}
    \textbf{Quantum Cryptography Use Cases:}
    \begin{itemize}
        \item \textbf{Secure Communication:} Optimal for government and military communications requiring high security.
        \item \textbf{Financial Transactions:} Can protect sensitive transactions from quantum computing threats.
    \end{itemize}

    \textbf{Blockchain Use Cases:}
    \begin{itemize}
        \item \textbf{Cryptocurrencies:} Major application in cryptocurrencies like Bitcoin and Ethereum facilitating peer-to-peer transactions.
        \item \textbf{Supply Chain Management:} Improves transparency and traceability, enhancing accountability.
        \item \textbf{Smart Contracts:} Automates processes across various industries with self-executing contracts.
    \end{itemize}

    \textit{Key Point:} Quantum Cryptography excels in high-security applications, while Blockchain offers decentralized solutions across diverse sectors.
    
    \vspace{1em} % Add space before the next section

    \textbf{Future Prospects:}
    \begin{itemize}
        \item Quantum Cryptography aims for global standards for quantum-safe cryptographic protocols.
        \item Blockchain continues to see wider adoption and strives for interoperability and scalability improvements.
    \end{itemize}
\end{frame}

\begin{frame}[fragile]
    \frametitle{Conclusion}
    Quantum Cryptography and Blockchain each possess unique strengths and applications:
    \begin{itemize}
        \item Quantum Cryptography offers unparalleled security based on quantum physics.
        \item Blockchain enhances transparency and security in digital transactions with practical decentralized solutions.
    \end{itemize}

    \textit{Note:} For further discussions, consider exploring the ethical implications of these technologies, which will be addressed in the next slide.
\end{frame}

\begin{frame}[fragile]
    \frametitle{Ethical Considerations - Introduction}
    Emerging technologies in cryptography, such as quantum cryptography and advanced blockchain applications, bring forward not only innovations but also ethical dilemmas and legal challenges. 
    \begin{itemize}
        \item Discussion on the ethical implications of these technologies 
        \item Highlighting the necessity for a robust legal framework to ensure responsible use
    \end{itemize}
\end{frame}

\begin{frame}[fragile]
    \frametitle{Ethical Considerations - Key Aspects}
    \begin{enumerate}
        \item \textbf{Privacy vs. Security}
            \begin{itemize}
                \item Balancing Act: Enhances security but can violate privacy rights
                \item Example: Encryption protects whistleblowers vs. potential misuse by criminals
            \end{itemize}
        \item \textbf{Access and Inclusivity}
            \begin{itemize}
                \item Digital Divide: Limited access leads to information security inequalities
                \item Example: Small businesses vs. larger corporations in adopting secure practices
            \end{itemize}
    \end{enumerate}
\end{frame}

\begin{frame}[fragile]
    \frametitle{Ethical Considerations - Accountability to Conclusion}
    \begin{enumerate}[resume]
        \item \textbf{Accountability and Traceability}
            \begin{itemize}
                \item Anonymous Transactions can hinder accountability
                \item Example: Cryptocurrencies in illegal activities
            \end{itemize}
        \item \textbf{Trust and Transparency}
            \begin{itemize}
                \item Need for transparency in encryption practices
                \item Example: Organizations' disclosure about data encryption
            \end{itemize}
        \item \textbf{Potential for Abuse}
            \begin{itemize}
                \item Surveillance and Control: Risks of using cryptography for mass surveillance
                \item Example: National security justifications for monitoring
            \end{itemize}
    \end{enumerate}
    \begin{block}{Legal Frameworks}
        \begin{itemize}
            \item Regulations and Standards: Countries developing legal frameworks for responsible use
            \item International Cooperation: Need for uniform standards globally
        \end{itemize}
    \end{block}
\end{frame}

\begin{frame}[fragile]
    \frametitle{Future Trends in Cryptography - Overview}
    Cryptography is adapting to societal needs and technological threats. 
    Key emerging trends include:
    \begin{itemize}
        \item Post-Quantum Cryptography
        \item Homomorphic Encryption
        \item Zero-Knowledge Proofs
        \item Decentralized Cryptography
        \item Integration with AI and Machine Learning
    \end{itemize}
    
    Understanding these trends is crucial for ensuring data privacy and security.
\end{frame}

\begin{frame}[fragile]
    \frametitle{Future Trends in Cryptography - Post-Quantum Cryptography}
    \begin{block}{Concept}  
    With the rise of quantum computing, traditional systems (like RSA and ECC) face significant risks. Post-quantum cryptography seeks to develop new algorithms resilient to quantum attacks.
    \end{block}
    
    \begin{block}{Examples}
    \begin{itemize}
        \item Lattice-based cryptography (e.g., NTRU)
        \item Code-based cryptography
    \end{itemize}
    \end{block}
    
    \textbf{Key Point:} Organizations must prepare for the transition to post-quantum standards to protect sensitive data.
\end{frame}

\begin{frame}[fragile]
    \frametitle{Future Trends in Cryptography - Other Key Concepts}
    \begin{enumerate}
        \item \textbf{Homomorphic Encryption}
        \begin{itemize}
            \item Allows computations on encrypted data, enhancing data privacy during processing.
            \item Example: Cloud services can analyze financial records without accessing raw data.
        \end{itemize}

        \item \textbf{Zero-Knowledge Proofs}
        \begin{itemize}
            \item Enables one party to prove knowledge without revealing the value.
            \item Example: Proving one's age without disclosing the birth date.
        \end{itemize}

        \item \textbf{Decentralized Cryptography}
        \begin{itemize}
            \item Distributes control across nodes, offering enhanced security.
            \item Example: Cryptocurrencies like Bitcoin utilize decentralized networks.
        \end{itemize}
    \end{enumerate}
\end{frame}

\begin{frame}[fragile]
    \frametitle{Future Trends in Cryptography - AI and Conclusion}
    \begin{block}{Integration with AI and ML}
    \begin{itemize}
        \item AI and ML help identify patterns and threats.
        \item Example: AI predicting cyberattacks and allowing preemptive measures.
    \end{itemize}
    \end{block}
    
    \begin{block}{Conclusion}
    The future of cryptography is driven by advancements that enhance data privacy and security in a connected world. Organizations must be aware of these trends to effectively safeguard their information.
    \end{block}
    
    \textbf{Informational Takeaway:} Understanding these trends prepares organizations for future digital challenges.
\end{frame}


\end{document}