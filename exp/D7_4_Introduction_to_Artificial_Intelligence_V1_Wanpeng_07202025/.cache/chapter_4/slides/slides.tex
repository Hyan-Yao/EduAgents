\documentclass[aspectratio=169]{beamer}

% Theme and Color Setup
\usetheme{Madrid}
\usecolortheme{whale}
\useinnertheme{rectangles}
\useoutertheme{miniframes}

% Additional Packages
\usepackage[utf8]{inputenc}
\usepackage[T1]{fontenc}
\usepackage{graphicx}
\usepackage{booktabs}
\usepackage{listings}
\usepackage{amsmath}
\usepackage{amssymb}
\usepackage{xcolor}
\usepackage{tikz}
\usepackage{pgfplots}
\pgfplotsset{compat=1.18}
\usetikzlibrary{positioning}
\usepackage{hyperref}

% Custom Colors
\definecolor{myblue}{RGB}{31, 73, 125}
\definecolor{mygray}{RGB}{100, 100, 100}
\definecolor{mygreen}{RGB}{0, 128, 0}
\definecolor{myorange}{RGB}{230, 126, 34}
\definecolor{mycodebackground}{RGB}{245, 245, 245}

% Set Theme Colors
\setbeamercolor{structure}{fg=myblue}
\setbeamercolor{frametitle}{fg=white, bg=myblue}
\setbeamercolor{title}{fg=myblue}
\setbeamercolor{section in toc}{fg=myblue}
\setbeamercolor{item projected}{fg=white, bg=myblue}
\setbeamercolor{block title}{bg=myblue!20, fg=myblue}
\setbeamercolor{block body}{bg=myblue!10}
\setbeamercolor{alerted text}{fg=myorange}

% Set Fonts
\setbeamerfont{title}{size=\Large, series=\bfseries}
\setbeamerfont{frametitle}{size=\large, series=\bfseries}
\setbeamerfont{caption}{size=\small}
\setbeamerfont{footnote}{size=\tiny}

% Code Listing Style
\lstdefinestyle{customcode}{
  backgroundcolor=\color{mycodebackground},
  basicstyle=\footnotesize\ttfamily,
  breakatwhitespace=false,
  breaklines=true,
  commentstyle=\color{mygreen}\itshape,
  keywordstyle=\color{blue}\bfseries,
  stringstyle=\color{myorange},
  numbers=left,
  numbersep=8pt,
  numberstyle=\tiny\color{mygray},
  frame=single,
  framesep=5pt,
  rulecolor=\color{mygray},
  showspaces=false,
  showstringspaces=false,
  showtabs=false,
  tabsize=2,
  captionpos=b
}
\lstset{style=customcode}

% Custom Commands
\newcommand{\hilight}[1]{\colorbox{myorange!30}{#1}}
\newcommand{\source}[1]{\vspace{0.2cm}\hfill{\tiny\textcolor{mygray}{Source: #1}}}
\newcommand{\concept}[1]{\textcolor{myblue}{\textbf{#1}}}
\newcommand{\separator}{\begin{center}\rule{0.5\linewidth}{0.5pt}\end{center}}

% Footer and Navigation Setup
\setbeamertemplate{footline}{
  \leavevmode%
  \hbox{%
  \begin{beamercolorbox}[wd=.3\paperwidth,ht=2.25ex,dp=1ex,center]{author in head/foot}%
    \usebeamerfont{author in head/foot}\insertshortauthor
  \end{beamercolorbox}%
  \begin{beamercolorbox}[wd=.5\paperwidth,ht=2.25ex,dp=1ex,center]{title in head/foot}%
    \usebeamerfont{title in head/foot}\insertshorttitle
  \end{beamercolorbox}%
  \begin{beamercolorbox}[wd=.2\paperwidth,ht=2.25ex,dp=1ex,center]{date in head/foot}%
    \usebeamerfont{date in head/foot}
    \insertframenumber{} / \inserttotalframenumber
  \end{beamercolorbox}}%
  \vskip0pt%
}

% Turn off navigation symbols
\setbeamertemplate{navigation symbols}{}

% Title Page Information
\title[Logic Reasoning]{Chapter 7: Logic Reasoning: Propositional and First-Order Logic}
\author[J. Smith]{John Smith, Ph.D.}
\institute[University Name]{
  Department of Computer Science\\
  University Name\\
  \vspace{0.3cm}
  Email: email@university.edu\\
  Website: www.university.edu
}
\date{\today}

% Document Start
\begin{document}

\frame{\titlepage}

\begin{frame}[fragile]
    \frametitle{Introduction to Logic Reasoning}
    \begin{block}{Overview}
        Logic reasoning is a systematic method of drawing conclusions based on premises or facts. It is crucial in Artificial Intelligence (AI) as it enables machines to mimic human thought processes, assess information, infer conclusions, and make decisions.
    \end{block}
\end{frame}

\begin{frame}[fragile]
    \frametitle{Types of Logic}
    \begin{itemize}
        \item \textbf{Propositional Logic:}
        \begin{itemize}
            \item Involves declarative statements (propositions) that can be true or false but not both. 
            \item \textbf{Example:}
            \begin{itemize}
                \item Proposition A: ``It is raining.''
                \item Proposition B: ``The ground is wet.''
            \end{itemize}
            \item Logical Statement: $A \rightarrow B$ (If it is raining, then the ground is wet)
        \end{itemize}
        
        \item \textbf{First-Order Logic:}
        \begin{itemize}
            \item Extends propositional logic by including quantifiers and relations, allowing for more complex representations.
            \item \textbf{Example:}
            \begin{itemize}
                \item ``For all $x$, if $x$ is a cat, then $x$ is a mammal.''
                \item This can be symbolized as: $\forall x \, (Cat(x) \rightarrow Mammal(x))$
            \end{itemize}
        \end{itemize}
    \end{itemize}
\end{frame}

\begin{frame}[fragile]
    \frametitle{Importance of Logic in AI}
    \begin{itemize}
        \item \textbf{Automated Reasoning:} 
        Logic reasoning enables automated theorem proving and systems like expert systems, which replicate human decision-making.
        \item \textbf{Natural Language Processing:} 
        Understanding human language involves logic to analyze structure and context accurately.
        \item \textbf{Knowledge Representation:} 
        Logic allows AI systems to represent complex knowledge in an understandable manner, facilitating reasoning and conclusions.
    \end{itemize}
    
    \begin{block}{Conclusion}
        Logic reasoning underpins many AI methods, equipping machines to process information intelligently. Understanding its components and applications is crucial for exploring deeper aspects of AI.
    \end{block}
\end{frame}

\begin{frame}[fragile]{Types of Logic}
\begin{block}{Introduction}
Logic is a foundational concept in mathematical reasoning and artificial intelligence, primarily divided into two types: 
\textbf{Propositional Logic} and \textbf{First-Order Logic} (FOL). Each type serves unique applications and has distinct characteristics.
\end{block}
\end{frame}

\begin{frame}[fragile]{Propositional Logic - Definition and Basics}
\begin{block}{Definition}
Propositional Logic deals with propositions that can be either true or false, represented by variables that can take on the value of True (T) or False (F).
\end{block}

\begin{itemize}
    \item \textbf{Propositions:} Statements that can be true or false (e.g., "It is raining.").
    \item \textbf{Logical Connectives:}
    \begin{itemize}
        \item AND ($\land$): True if both propositions are true.
        \item OR ($\lor$): True if at least one proposition is true.
        \item NOT ($\neg$): True if the proposition is false.
        \item IMPLIES ($\rightarrow$): True unless a true proposition implies a false one.
    \end{itemize}
\end{itemize}

\begin{block}{Example}
Let $p$: "It is raining." and $q$: "I will carry an umbrella."\\
The compound proposition, $p \rightarrow q$, translates to "If it is raining, then I will carry an umbrella."
\end{block}
\end{frame}

\begin{frame}[fragile]{First-Order Logic (FOL) - Definition and Key Components}
\begin{block}{Definition}
First-Order Logic expands upon Propositional Logic by including quantifiers and predicates, allowing for more complex statements about objects and their relationships.
\end{block}

\begin{itemize}
    \item \textbf{Predicates:} Functions that express properties of objects (e.g., $Loves(John, Mary)$).
    \item \textbf{Quantifiers:}
    \begin{itemize}
        \item Universal Quantifier ($\forall$): Holds for all members of a domain (e.g., "For all $x$, $x$ is a human").
        \item Existential Quantifier ($\exists$): States there exists at least one member that meets the condition (e.g., "There exists an $x$ such that $x$ is a dog").
    \end{itemize}
\end{itemize}

\begin{block}{Example}
\begin{itemize}
    \item $\forall x (Human(x) \rightarrow Mortal(x))$ translates to "All humans are mortal."
    \item $\exists y (Dog(y) \land Barks(y))$ translates to "There exists at least one dog that barks."
\end{itemize}
\end{block}
\end{frame}

\begin{frame}[fragile]{Key Differences and Applications}
\begin{block}{Key Differences}
\begin{tabular}{|c|c|c|}
\hline
Aspect                  & Propositional Logic          & First-Order Logic               \\
\hline
Nature of Statements    & Deals with whole propositions & Deals with objects and their properties \\
Complexity             & Simpler structures           & More expressive and complex      \\
Quantification         & No quantifiers               & Supports quantification         \\
\hline
\end{tabular}
\end{block}

\begin{block}{Applications}
\begin{itemize}
    \item \textbf{Propositional Logic:} Used in circuit design, simple algorithms, and reasoning about events.
    \item \textbf{First-Order Logic:} Used in knowledge representation, natural language processing, and formal verification in software engineering.
\end{itemize}
\end{block}
\end{frame}

\begin{frame}[fragile]
    \frametitle{Propositional Logic: Definition}
    
    \begin{block}{Definition}
        Propositional Logic, also known as propositional calculus or sentential logic, is a branch of logic that deals with propositions—statements that can either be true or false but not both.
    \end{block}
    
    \begin{itemize}
        \item Formal framework for analyzing logical relationships
        \item Constructs valid arguments
    \end{itemize}
\end{frame}

\begin{frame}[fragile]
    \frametitle{Propositional Logic: Basic Building Blocks}
    
    \begin{enumerate}
        \item \textbf{Propositions:}
            \begin{itemize}
                \item Declarative sentences with a truth value.
                \item \textbf{Examples:}
                    \begin{itemize}
                        \item “The sky is blue.” (True)
                        \item “2 + 2 = 5.” (False)
                    \end{itemize}
            \end{itemize}
        
        \item \textbf{Logical Connectives:}
            \begin{itemize}
                \item \textbf{Conjunction (∧):} Represents "and".
                \item \textbf{Disjunction (∨):} Represents "or".
                \item \textbf{Negation (¬):} Represents "not".
                \item \textbf{Implication (→):} Represents "if...then".
                \item \textbf{Biconditional (↔):} Represents "if and only if".
            \end{itemize}
    \end{enumerate}
\end{frame}

\begin{frame}[fragile]
    \frametitle{Truth Values and Key Points}
    
    \begin{itemize}
        \item \textbf{Truth Values:}
            \begin{itemize}
                \item True (T) or False (F)
                \item For compound propositions, the truth value depends on the components and logical connectives.
            \end{itemize}
        
        \item \textbf{Key Points to Emphasize:}
            \begin{itemize}
                \item Foundation for understanding complex logic like first-order logic.
                \item Goal: Create logical expressions and determine validity.
                \item Understanding components leads to truth tables.
            \end{itemize}
    \end{itemize}
    
    \textbf{Illustration: Basic Connectives}
    \begin{itemize}
        \item Example for \( P \land Q \):
            \begin{itemize}
                \item If \( P = \text{True} \) and \( Q = \text{True} \), then \( P \land Q = \text{True} \).
                \item If \( P = \text{True} \) and \( Q = \text{False} \), then \( P \land Q = \text{False} \).
            \end{itemize}
    \end{itemize}
    
    \textbf{Engage the students:} 
    \begin{itemize}
        \item "Can you think of two statements that can be combined using 'and'?"
        \item "How does negation affect the truth of a proposition?"
    \end{itemize}
\end{frame}

\begin{frame}[fragile]
    \frametitle{Truth Tables - Definition}
    \begin{block}{Definition of Truth Tables}
        A truth table is a mathematical table used in logic—specifically in propositional and first-order logic—that lists all possible truth values for a set of logical variables. This visual representation helps evaluate the validity of logical expressions systematically.
    \end{block}
    
    \begin{block}{Purpose of Truth Tables}
        Truth tables are essential tools for:
        \begin{itemize}
            \item Determining the truth value of complex expressions.
            \item Understanding the behavior of logical connectives.
            \item Testing the validity of logical arguments.
        \end{itemize}
    \end{block}
\end{frame}

\begin{frame}[fragile]
    \frametitle{Truth Tables - Components}
    \begin{block}{Components of a Truth Table}
        \begin{enumerate}
            \item **Variables**: Each statement or proposition is represented by a letter (e.g., P, Q, R).
            \item **Rows**: Each row of the table represents a possible combination of truth values for the variables.
            \item **Columns**: Columns show each logical expression evaluated for the combinations of truth values.
        \end{enumerate}
    \end{block}
\end{frame}

\begin{frame}[fragile]
    \frametitle{Truth Tables - Basic Example}
    Consider two propositions:  
    \begin{itemize}
        \item P: "It is raining."  
        \item Q: "It is daytime."
    \end{itemize}

    The truth table for P AND Q ($P \land Q$) and P OR Q ($P \lor Q$) is:

    \begin{center}
        \begin{tabular}{|c|c|c|c|}
            \hline
            P     & Q     & $P \land Q$ & $P \lor Q$ \\
            \hline
            True  & True  & True  & True  \\
            True  & False & False & True  \\
            False & True  & False & True  \\
            False & False & False & False \\
            \hline
        \end{tabular}
    \end{center}
    
    \begin{block}{Key Points}
        \begin{itemize}
            \item **$P \land Q$** is true only when both P and Q are true.
            \item **$P \lor Q$** is true when at least one of P or Q is true.
        \end{itemize}
    \end{block}
\end{frame}

\begin{frame}[fragile]
    \frametitle{Truth Tables - Constructing Steps}
    \begin{block}{Constructing a Truth Table: Steps}
        \begin{enumerate}
            \item Identify the propositions and logical connectives in the expression.
            \item List all combinations of truth values for the propositions.
            \item Evaluate the expression for each combination and write the results in the corresponding columns.
        \end{enumerate}
    \end{block}
\end{frame}

\begin{frame}[fragile]
    \frametitle{Truth Tables - Complex Example}
    Let’s explore $P \rightarrow Q$ (P implies Q) and $\neg P$ (not P):

    \begin{center}
        \begin{tabular}{|c|c|c|c|}
            \hline
            P     & Q     & $\neg P$    & $P \rightarrow Q$ \\
            \hline
            True  & True  & False & True  \\
            True  & False & False & False \\
            False & True  & True  & True  \\
            False & False & True  & True  \\
            \hline
        \end{tabular}
    \end{center}

    \begin{block}{Understanding IMPLIES}
        $P \rightarrow Q$ is false only when P is true and Q is false; otherwise, it is true.
    \end{block}
\end{frame}

\begin{frame}[fragile]
    \frametitle{Truth Tables - Summary}
    \begin{block}{Summary}
        Truth tables provide a clear method for evaluating logical expressions. They help visualize the relationships between propositions and their connected symbols. Mastery of truth tables leads to improved understanding in more complex areas of logic, such as quantifier expressions and logical proofs.
    \end{block}
    
    \begin{block}{Practice}
        Students are encouraged to practice creating truth tables for various logical expressions to solidify their understanding of propositional logic!
    \end{block}
\end{frame}

\begin{frame}[fragile]
    \frametitle{Logical Connectives}
    \begin{block}{Overview of Logical Connectives}
        Logical connectives are fundamental components in propositional logic that connect simple propositions to form compound statements. 
        Understanding these connectives is crucial for evaluating logical expressions and building truth tables.
    \end{block}
\end{frame}

\begin{frame}[fragile]
    \frametitle{Types of Logical Connectives - Part 1}
    \begin{enumerate}
        \item \textbf{AND (Conjunction)} 
            % Description of AND
            \begin{itemize}
                \item \textbf{Symbol:} $\wedge$
                \item \textbf{Definition:} Returns true only if both propositions are true.
                \item \textbf{Example:}
                \begin{itemize}
                    \item Let $P$: "It is raining."
                    \item Let $Q$: "I have an umbrella."
                    \item The statement "$P$ AND $Q$" ($P \wedge Q$) is true only if both $P$ and $Q$ are true.
                \end{itemize}
            \end{itemize}
            \textbf{Truth Table:}
            \begin{tabular}{|c|c|c|}
                \hline
                $P$ & $Q$ & $P \wedge Q$ \\
                \hline
                True & True  & True  \\
                True & False & False \\
                False & True  & False \\
                False & False & False \\
                \hline
            \end{tabular}
    \end{enumerate}
\end{frame}

\begin{frame}[fragile]
    \frametitle{Types of Logical Connectives - Part 2}
    \begin{enumerate}
        \setcounter{enumi}{1} % Start from the second item
        \item \textbf{OR (Disjunction)} 
            % Description of OR
            \begin{itemize}
                \item \textbf{Symbol:} $\vee$
                \item \textbf{Definition:} Returns true if at least one proposition is true.
                \item \textbf{Example:}
                \begin{itemize}
                    \item For $P$: "I will go to the park."
                    \item For $Q$: "I will go to the mall."
                    \item The statement "$P$ OR $Q$" ($P \vee Q$) is true if either $P$ or $Q$ (or both) is true.
                \end{itemize}
            \end{itemize}
            \textbf{Truth Table:}
            \begin{tabular}{|c|c|c|}
                \hline
                $P$ & $Q$ & $P \vee Q$ \\
                \hline
                True & True  & True  \\
                True & False & True  \\
                False & True  & True  \\
                False & False & False \\
                \hline
            \end{tabular}
    \end{enumerate}
\end{frame}

\begin{frame}[fragile]
    \frametitle{Types of Logical Connectives - Part 3}
    \begin{enumerate}
        \setcounter{enumi}{2} % Continue from the last item
        \item \textbf{NOT (Negation)} 
            % Description of NOT
            \begin{itemize}
                \item \textbf{Symbol:} $\neg$
                \item \textbf{Definition:} Inverts the truth value of a proposition.
                \item \textbf{Example:}
                \begin{itemize}
                    \item For $P$: "It is sunny."
                    \item The statement "NOT $P$" ($\neg P$) is true if $P$ is false.
                \end{itemize}
            \end{itemize}
            \textbf{Truth Table:}
            \begin{tabular}{|c|c|}
                \hline
                $P$ & $\neg P$ \\
                \hline
                True  & False \\
                False & True  \\
                \hline
            \end{tabular}
        \item \textbf{IMPLIES (Conditional)}
            % Description of IMPLIES
            \begin{itemize}
                \item \textbf{Symbol:} $\rightarrow$
                \item \textbf{Definition:} Returns false when the first proposition is true and the second is false.
                \item \textbf{Example:}
                \begin{itemize}
                    \item For $P$: "I study hard."
                    \item For $Q$: "I will pass the exam."
                    \item The statement "$P$ IMPLIES $Q$" ($P \rightarrow Q$) is false only if $P$ is true and $Q$ is false.
                \end{itemize}
            \end{itemize}
            \textbf{Truth Table:}
            \begin{tabular}{|c|c|c|}
                \hline
                $P$ & $Q$ & $P \rightarrow Q$ \\
                \hline
                True  & True  & True  \\
                True  & False & False \\
                False & True  & True  \\
                False & False & True  \\
                \hline
            \end{tabular}
    \end{enumerate}
\end{frame}

\begin{frame}[fragile]
    \frametitle{Types of Logical Connectives - Part 4}
    \begin{enumerate}
        \setcounter{enumi}{4} % Continue from the last item
        \item \textbf{BICONDITIONAL (If and Only If)}
            % Description of BICONDITIONAL
            \begin{itemize}
                \item \textbf{Symbol:} $\leftrightarrow$
                \item \textbf{Definition:} Returns true if both propositions are either true or false.
                \item \textbf{Example:}
                \begin{itemize}
                    \item For $P$: "You can take the bus."
                    \item For $Q$: "You can take the train."
                    \item The statement "$P$ BICONDITIONAL $Q$" ($P \leftrightarrow Q$) is true if both $P$ and $Q$ are either true or false.
                \end{itemize}
            \end{itemize}
            \textbf{Truth Table:}
            \begin{tabular}{|c|c|c|}
                \hline
                $P$ & $Q$ & $P \leftrightarrow Q$ \\
                \hline
                True  & True  & True  \\
                True  & False & False \\
                False & True  & False \\
                False & False & True  \\
                \hline
            \end{tabular}
    \end{enumerate}
\end{frame}

\begin{frame}[fragile]
    \frametitle{Key Points}
    \begin{itemize}
        \item Each logical connective alters the relationship between propositions.
        \item Understanding their truth tables is essential for evaluating logical statements.
        \item Logical connectives form the foundation for more complex logical expressions and reasoning in propositional logic.
    \end{itemize}
\end{frame}

\begin{frame}[fragile]
    \frametitle{Learning Objectives}
    \begin{itemize}
        \item Understand how propositional logic is applied in AI problem-solving.
        \item Identify real-world scenarios where propositional logic enhances decision-making.
        \item Analyze examples to illustrate the use of propositional logic in various fields.
    \end{itemize}
\end{frame}

\begin{frame}[fragile]
    \frametitle{Introduction to Propositional Logic}
    Propositional logic is the branch of logic dealing with propositions that can be either true or false. It uses logical connectives (AND, OR, NOT, IMPLIES, BICONDITIONAL) to form complex statements.  
    These elements are foundational in AI, as they facilitate reasoning and decision-making processes.
\end{frame}

\begin{frame}[fragile]
    \frametitle{Real-World Applications of Propositional Logic - Part 1}
    \begin{enumerate}
        \item \textbf{Expert Systems:}
            \begin{itemize}
                \item Mimic the decision-making ability of a human expert.
                \item Example: A medical diagnosis system.
                \item Logical Form: If a patient has a fever (P), then they may have an infection (Q). $\Rightarrow P \rightarrow Q$
            \end{itemize}

        \item \textbf{Automated Reasoning:}
            \begin{itemize}
                \item Backbone for automated theorem proving.
                \item Example: SAT Solvers for Boolean satisfiability.
                \item Problem Statement: Is $(P \land \neg Q) \lor (Q)$ satisfiable?
            \end{itemize}
    \end{enumerate}
\end{frame}

\begin{frame}[fragile]
    \frametitle{Real-World Applications of Propositional Logic - Part 2}
    \begin{enumerate}
        \item \textbf{Robotics:}
            \begin{itemize}
                \item Used for navigation and obstacle avoidance.
                \item Example: 
                \begin{itemize}
                    \item If the path is clear (P), then move forward (Q). $\Rightarrow P \rightarrow Q$
                    \item If an obstacle is detected (R), then stop (S). $\Rightarrow R \rightarrow S$
                \end{itemize}
            \end{itemize}

        \item \textbf{Game AI:}
            \begin{itemize}
                \item Makes decisions based on game states.
                \item Example:
                \begin{itemize}
                    \item If the player is in proximity (P), then the enemy should attack (Q). $\Rightarrow P \rightarrow Q$
                    \item If health is low (R), then retreat (S). $\Rightarrow R \rightarrow S$
                \end{itemize}
            \end{itemize}
    \end{enumerate}
\end{frame}

\begin{frame}[fragile]
    \frametitle{Real-World Applications of Propositional Logic - Part 3}
    \begin{enumerate}
        \item \textbf{Natural Language Processing (NLP):}
            \begin{itemize}
                \item Used to understand and interpret human languages.
                \item Example:
                \begin{itemize}
                    \item If a user commands, "Turn on the lights" (P), then the system should execute the action (Q). $\Rightarrow P \rightarrow Q$
                \end{itemize}
            \end{itemize}
    \end{enumerate}
\end{frame}

\begin{frame}[fragile]
    \frametitle{Key Points and Conclusion}
    \begin{block}{Key Points to Emphasize}
        \begin{itemize}
            \item Propositional logic structures thought processes and decision-making in AI systems.
            \item Foundational for more complex reasoning systems, such as First-Order Logic.
            \item Enables clarity in reasoning through well-defined rules and propositions.
        \end{itemize}
    \end{block}
    
    \textbf{Conclusion:}  
    Propositional logic plays a vital role in various domains of AI, ensuring systems can reason and respond to real-world scenarios effectively. Understanding its applications prepares us for more advanced concepts in first-order logic.
\end{frame}

\begin{frame}[fragile]
    \frametitle{First-Order Logic: Definition - Overview}
    \begin{block}{What is First-Order Logic?}
        First-Order Logic (FOL), also known as Predicate Logic, extends propositional logic by introducing **predicates** and **quantifiers**. 
        It allows for a richer expression of relationships between objects and their properties.
    \end{block}
\end{frame}

\begin{frame}[fragile]
    \frametitle{First-Order Logic: Structure}
    \begin{itemize}
        \item \textbf{Predicates:}
        \begin{itemize}
            \item A predicate expresses a property of objects or a relationship between them.
            \item Example: Let \( P(x) \) represent "x is a human." Here, \( P \) is the predicate and \( x \) is a variable.
        \end{itemize}
        
        \item \textbf{Quantifiers:}
        \begin{itemize}
            \item \textbf{Universal Quantifier} (\( \forall \)): Holds for all elements in a domain.
                \begin{itemize}
                    \item Example: \( \forall x \, P(x) \): "For all x, x is a human."
                \end{itemize}
            \item \textbf{Existential Quantifier} (\( \exists \)): At least one element in the domain satisfies the predicate.
                \begin{itemize}
                    \item Example: \( \exists x \, P(x) \): "There exists an x such that x is a human."
                \end{itemize}
        \end{itemize}
    \end{itemize}
\end{frame}

\begin{frame}[fragile]
    \frametitle{First-Order Logic: Applications and Examples}
    \begin{block}{Real-World Relation}
        FOL is essential for representing complex statements in fields such as AI, natural language processing, and formal reasoning.
    \end{block}

    \begin{block}{Combined Usage Example}
        Consider the statement: "All humans are mortal."
        \begin{equation}
            \forall x \, (P(x) \rightarrow Q(x))
        \end{equation}
        where \( P(x) \) represents "x is a human" and \( Q(x) \) represents "x is mortal."
    \end{block}
    
    \begin{block}{Summary}
        FOL enhances propositional logic by incorporating predicates and quantifiers, facilitating more nuanced representations in formal systems.
    \end{block}
\end{frame}

\begin{frame}[fragile]
    \frametitle{Quantifiers in First-Order Logic}
    \begin{block}{Learning Objectives}
        \begin{itemize}
            \item Understand definitions and roles of existential and universal quantifiers.
            \item Explore the significance of quantifiers in logical statements.
            \item Analyze practical examples illustrating quantifier usage.
        \end{itemize}
    \end{block}
\end{frame}

\begin{frame}[fragile]
    \frametitle{Introduction to Quantifiers}
    In first-order logic, quantifiers are essential for expressing statements involving variables.
    They allow us to make generalizations or assertions about the existence or properties of objects in a domain.
    
    \begin{block}{Types of Quantifiers}
        \begin{enumerate}
            \item \textbf{Universal Quantifier ( $\forall$ )}
                \begin{itemize}
                    \item \textbf{Definition:} Asserts a property holds for all elements in a domain.
                    \item \textbf{Symbol:} \( \forall x \) reads as "for all \( x \)".
                    \item \textbf{Example:} \( \forall x (P(x)) \): "For all \( x \), \( P(x) \) is true."
                    \item \textbf{Illustration:} If \( P(x) \) means "x is a student", then \( \forall x (P(x)) \) means "Every individual in the domain is a student".
                \end{itemize}
            \item \textbf{Existential Quantifier ( $\exists$ )}
                \begin{itemize}
                    \item \textbf{Definition:} Asserts there exists at least one element in a domain where a property holds.
                    \item \textbf{Symbol:} \( \exists x \) reads as "there exists an \( x \)".
                    \item \textbf{Example:} \( \exists x (P(x)) \): "There exists at least one \( x \) such that \( P(x) \) is true."
                    \item \textbf{Illustration:} If \( P(x) \) means "x is a cat", then \( \exists x (P(x)) \) means "There is at least one individual in the domain that is a cat".
                \end{itemize}
        \end{enumerate}
    \end{block}
\end{frame}

\begin{frame}[fragile]
    \frametitle{Functions and Significance of Quantifiers}
    \begin{block}{Key Functions}
        \begin{itemize}
            \item \textbf{Expressing General Knowledge:} 
            Universal quantifiers express universally accepted rules and truths.
            \item \textbf{Existential Assertions:} 
            Existential quantifiers validate the existence of examples needed in proofs and algorithms.
            \item \textbf{Combining Quantifiers:}
            Quantifiers can be combined, e.g., \( \forall x \exists y \, (P(x, y)) \) indicates "for every \( x \), there exists a \( y \) such that \( P(x, y) \) holds".
        \end{itemize}
    \end{block}
    
    \begin{block}{Key Points to Emphasize}
        \begin{itemize}
            \item Distinguish between universal and existential quantifiers.
            \item Recognize their pivotal roles in logical reasoning.
            \item Practice multiple examples to solidify understanding.
        \end{itemize}
    \end{block}
\end{frame}

\begin{frame}[fragile]
    \frametitle{Inference in First-Order Logic}
    % Overview and significance of inference in first-order logic.
    Inference is a process of drawing conclusions from premises using established rules. 
    In first-order logic (FOL), inference allows us to deduce new statements from existing ones through formal reasoning.
\end{frame}

\begin{frame}[fragile]
    \frametitle{Key Inference Rules in First-Order Logic}
    % Explanation of key inference rules
    \begin{enumerate}
        \item \textbf{Universal Instantiation (UI)}:
        \begin{itemize}
            \item If $\forall x \, P(x)$ is true, then $P(a)$ is also true for any particular element $a$.
        \end{itemize}

        \item \textbf{Existential Instantiation (EI)}:
        \begin{itemize}
            \item If $\exists x \, P(x)$ is true, we can introduce a new constant $c$ such that $P(c)$ holds.
        \end{itemize}

        \item \textbf{Universal Generalization (UG)}:
        \begin{itemize}
            \item If $P(a)$ is proven true for an arbitrary element $a$, then we conclude $\forall x \, P(x)$.
        \end{itemize}

        \item \textbf{Existential Generalization (EG)}:
        \begin{itemize}
            \item If $P(a)$ holds for a specific $a$, we can infer $\exists x \, P(x)$.
        \end{itemize}
    \end{enumerate}
\end{frame}

\begin{frame}[fragile]
    \frametitle{Applications of Inference in AI}
    % Overview of applications of inference rules in AI.
    \begin{itemize}
        \item \textbf{Knowledge Representation}: 
        Uses FOL to represent facts, enabling automated reasoning.
        
        \item \textbf{Natural Language Processing (NLP)}: 
        Helps in understanding semantics, resolving ambiguities, and generating meanings.
        
        \item \textbf{Automated Theorem Proving}: 
        Systems like Prolog use inference to prove or disprove theorems.
        
        \item \textbf{Expert Systems}: 
        Inference engines evaluate information to mimic human expertise in decision-making.
    \end{itemize}
\end{frame}

\begin{frame}[fragile]
    \frametitle{Applications of First-Order Logic}
    \begin{block}{What is First-Order Logic (FOL)?}
        First-Order Logic is a powerful framework that extends propositional logic to include quantifiers and predicates, allowing for more expressive statements about objects and their relationships.
    \end{block}
    \begin{block}{Importance in AI}
        It is crucial in formalizing reasoning processes, especially in Artificial Intelligence (AI).
    \end{block}
\end{frame}

\begin{frame}[fragile]
    \frametitle{Key Applications in AI Systems - Part 1}
    \begin{enumerate}
        \item \textbf{Knowledge Representation:}
            \begin{itemize}
                \item FOL allows the representation of complex relationships in a structured format.
                \item Example: "All humans are mortal" is expressed in FOL as: 
                \begin{equation}
                    \forall x \, \text{(Human}(x) \rightarrow \text{Mortal}(x))
                \end{equation}
            \end{itemize}
        
        \item \textbf{Natural Language Processing (NLP):}
            \begin{itemize}
                \item FOL is used in parsing and understanding human languages.
                \item Example: Transforming "Every student in the class passed the exam" into FOL helps derive meaning from text.
            \end{itemize}
    \end{enumerate}
\end{frame}

\begin{frame}[fragile]
    \frametitle{Key Applications in AI Systems - Part 2}
    \begin{enumerate}
        \setcounter{enumi}{2}
        \item \textbf{Automated Theorem Proving:}
            \begin{itemize}
                \item Utilizes FOL to prove mathematical theorems and validate logical statements.
                \item Inference rules like modus ponens help derive new knowledge.
            \end{itemize}

        \item \textbf{Expert Systems:}
            \begin{itemize}
                \item FOL encodes expert knowledge, like medical conditions based on symptoms.
                \item Example: "If a patient has a fever and a cough, they might have the flu" expressed in FOL.
            \end{itemize}
        
        \item \textbf{Robotics and Planning:}
            \begin{itemize}
                \item Employed to plan actions based on the environment.
                \item Example: "If there is an obstacle (O) in front of it, then it should turn left" expressed in FOL:
                \begin{equation}
                    \forall x \, \text{(Obstacle}(x) \rightarrow \text{TurnLeft)}
                \end{equation}
            \end{itemize}
    \end{enumerate}
\end{frame}

\begin{frame}[fragile]
    \frametitle{Key Points and Conclusion}
    \begin{block}{Key Points to Emphasize:}
        \begin{itemize}
            \item \textbf{Expressiveness:} FOL expresses complex statements enabling richer knowledge representations.
            \item \textbf{Inference:} AI systems use FOL for inferring new information, enhancing decision-making.
            \item \textbf{Interdisciplinary Utilization:} FOL has applications in various domains such as robotics, computational linguistics, and expert systems.
        \end{itemize}
    \end{block}
    \begin{block}{Conclusion}
        First-Order Logic is foundational for advanced AI systems, enabling knowledge representation, inference, and complex reasoning across various fields.
    \end{block}
\end{frame}

\begin{frame}[fragile]
    \frametitle{Comparison: Propositional vs First-Order Logic}
    \begin{block}{Introduction to Logic}
        Logic provides a framework for reasoning and making deductions, evaluating whether statements (propositions) are true or false. This presentation focuses on **Propositional Logic** and **First-Order Logic**.
    \end{block}
\end{frame}

\begin{frame}[fragile]
    \frametitle{Key Concepts}
    \begin{itemize}
        \item \textbf{Propositional Logic}:
        \begin{itemize}
            \item \textbf{Definition}: Deals with propositions—statements that can either be true or false.
            \item \textbf{Elements}:
            \begin{itemize}
                \item Propositions: Simple statements (e.g., "It is raining.")
                \item Connectives: Operators forming compound statements (e.g., AND, OR, NOT).
            \end{itemize}
            \item \textbf{Expressiveness}: Limited to simple facts; unable to express relationships.
        \end{itemize}

        \item \textbf{First-Order Logic (FOL)}:
        \begin{itemize}
            \item \textbf{Definition}: An expressive system that incorporates quantifiers and predicates.
            \item \textbf{Elements}:
            \begin{itemize}
                \item Predicates: Functions that return true/false based on inputs (e.g., "Loves(x, y)").
                \item Quantifiers: 
                \begin{itemize}
                    \item Universal ( ∀ ) – "For all"
                    \item Existential ( ∃ ) – "There exists"
                \end{itemize}
            \end{itemize}
            \item \textbf{Expressiveness}: Can express properties of objects and their relationships.
        \end{itemize}
    \end{itemize}
\end{frame}

\begin{frame}[fragile]
    \frametitle{Comparative Analysis}
    \begin{tabular}{|l|l|l|}
        \hline
        \textbf{Feature} & \textbf{Propositional Logic} & \textbf{First-Order Logic} \\
        \hline
        Expressiveness & Limited to simple statements & Richer; can express relationships and properties \\
        \hline
        Components & Propositions and connectives & Predicates, quantifiers, and objects \\
        \hline
        Example Statement & "It is raining AND it is cold." & "∀x (Human(x) → Mortal(x))" \\
        \hline
        Applications & Basic circuit design; simple logical statements & AI reasoning; natural language processing; databases \\
        \hline
    \end{tabular}
\end{frame}

\begin{frame}[fragile]
    \frametitle{Illustrative Examples}
    \begin{itemize}
        \item \textbf{Propositional Logic}:
        \begin{itemize}
            \item Example: Let P = "It is raining" and Q = "The ground is wet."
            \item Statement: P AND Q (If it is raining, then the ground is wet).
        \end{itemize}
        
        \item \textbf{First-Order Logic}:
        \begin{itemize}
            \item Example: Let Loves(x, y) = "x loves y"
            \item Statement: $\exists x \exists y \; (Loves(x, y))$ (There exists someone who loves someone).
        \end{itemize}
    \end{itemize}
\end{frame}

\begin{frame}[fragile]
    \frametitle{Key Points to Emphasize}
    \begin{itemize}
        \item \textbf{Expressiveness}: FOL’s ability to represent complex statements is vital for AI and knowledge representation.
        \item \textbf{Complexity}: FOL introduces complexity in reasoning, which will be explored in subsequent slides.
    \end{itemize}
\end{frame}

\begin{frame}[fragile]
    \frametitle{Conclusion}
    \begin{itemize}
        \item Understanding the differences between propositional and first-order logic is crucial in logic reasoning—especially in AI.
        \item Mastery of these concepts equips students with foundational tools for advanced studies in logic and its applications.
    \end{itemize}
    \begin{block}{Q\&A}
        Feel free to ask questions or seek clarification on any of these concepts!
    \end{block}
\end{frame}

\begin{frame}[fragile]
    \frametitle{Complexity of Logical Reasoning}
    % Discussion on the computational complexity of logical reasoning in AI systems.
    \begin{block}{Understanding Computational Complexity}
        Computational complexity in logical reasoning refers to the amount of resources (time and space) required to solve problems using logical systems. It focuses on how the complexity of reasoning grows as the problem size increases.
    \end{block}
\end{frame}

\begin{frame}[fragile]
    \frametitle{Key Complexity Classes}
    % Overview of different complexity classes related to logical reasoning.
    \begin{enumerate}
        \item \textbf{P (Polynomial Time):} Problems solvable in polynomial time by a deterministic Turing machine. Example: Propositional satisfiability (SAT) for certain forms.
        \item \textbf{NP (Non-deterministic Polynomial Time):} Problems for which a proposed solution can be verified in polynomial time. Example: General SAT problems.
        \item \textbf{PSPACE:} Problems solvable using a polynomial amount of memory. First-order logic reasoning often falls into this category.
        \item \textbf{EXP (Exponential Time):} Problems that require exponential time to solve. Example: General first-order logic reasoning can be in EXP.
    \end{enumerate}
\end{frame}

\begin{frame}[fragile]
    \frametitle{Propositional Logic vs. First-Order Logic}
    \begin{itemize}
        \item \textbf{Propositional Logic Complexity:}
        \begin{itemize}
            \item Deciding satisfiability is NP-complete; no polynomial-time solution is known for every case.
            \item \textit{Example:} The formula \( P \land (Q \lor \neg R) \) can be evaluated quickly, but determining its satisfiability is complex.
        \end{itemize}
        
        \item \textbf{First-Order Logic Complexity:}
        \begin{itemize}
            \item Reasoning in first-order logic is often PSPACE-complete, requiring significant space as problems scale.
            \item \textit{Example:} The sentence “For every person \(x\), there exists a pet \(y\) such that \(y\) is owned by \(x\)” introduces quantifiers and increases evaluation complexity.
        \end{itemize}
    \end{itemize}
\end{frame}

\begin{frame}[fragile]
    \frametitle{Why Complexity Matters in AI Systems}
    \begin{enumerate}
        \item \textbf{Scalability:} More variables and rules lead to increased computational demands, causing inefficiencies in tasks like automated theorem proving.
        \item \textbf{Practical Applications:} Understanding complexity helps in algorithm design and optimization, essential for applications such as natural language processing and robotics.
        \item \textbf{Algorithm Selection:} Awareness of problem complexity guides the selection of computational strategies, such as using heuristics for NP-hard problems.
    \end{enumerate}
\end{frame}

\begin{frame}[fragile]
    \frametitle{Key Takeaways}
    % Summary of important points discussed in the presentation
    \begin{itemize}
        \item Complexity is crucial for configuring AI systems leveraging logical reasoning.
        \item Propositional logic is simpler but limited in expressiveness; first-order logic offers greater expressiveness but involves more complexity.
        \item Recognizing classifications (P, NP, PSPACE, EXP) aids in understanding required computational resources.
    \end{itemize}
\end{frame}

\begin{frame}[fragile]
    \frametitle{Summary}
    % Overview of the importance of understanding logical reasoning complexity
    Comprehending the complexity of logical reasoning not only satisfies academic curiosity but also informs practical machine learning and AI application design, bridging theoretical knowledge with real-world implementation challenges.
\end{frame}

\begin{frame}[fragile]
    \frametitle{Challenges in Logical Reasoning - Overview}
    % Overview of the challenges in implementing logic systems in AI
    Implementing logical reasoning processes within AI systems presents a variety of challenges. 
    As AI aims to model complex reasoning akin to human thought, identifying and overcoming these challenges is crucial for the development of effective AI applications.
\end{frame}

\begin{frame}[fragile]
    \frametitle{Challenges in Logical Reasoning - Key Challenges}
    % Outline of key challenges
    \begin{enumerate}
        \item \textbf{Computational Complexity}
        \begin{itemize}
            \item Many logical systems, particularly those using First-Order Logic (FOL), suffer from high computational complexity.
            \item Example: Satisfiability (SAT) is NP-complete and can become infeasible with larger sets of propositions.
        \end{itemize}

        \item \textbf{Knowledge Representation}
        \begin{itemize}
            \item Structuring knowledge effectively is a major hurdle; concepts must be precisely encoded.
            \item Example: Representing "All humans are mortal" and "Socrates is a human" requires careful syntax for deduction.
        \end{itemize}
    \end{enumerate}
\end{frame}

\begin{frame}[fragile]
    \frametitle{Challenges in Logical Reasoning - Key Challenges Continued}
    % Continued discussion of the challenges
    \begin{enumerate}
        \setcounter{enumi}{2} % Continuing the enumeration
        \item \textbf{Handling Uncertainty}
        \begin{itemize}
            \item Real-world data can be vague or incomplete; traditional logical systems struggle with precision.
            \item Example: In medical diagnoses, AI faces uncertainties despite having numerous cases.
        \end{itemize}

        \item \textbf{Scalability Issues}
        \begin{itemize}
            \item Logical reasoning systems often struggle to scale with data volume.
            \item Example: Natural Language Processing faces difficulties in understanding nuanced meanings in large texts.
        \end{itemize}

        \item \textbf{Integration with Other AI Techniques}
        \begin{itemize}
            \item Logical reasoning often conflicts with probabilistic reasoning and machine learning approaches.
            \item Example: Logical outputs are deterministic, while machine learning relies on generalized patterns.
        \end{itemize}
    \end{enumerate}
\end{frame}

\begin{frame}[fragile]
    \frametitle{Challenges in Logical Reasoning - Practical Implications}
    % Discussion on implications and future directions
    \begin{block}{Real-World AI Applications}
        Consider intelligent assistants, robotics, and automated decision-making systems. Each faces limitations when solely relying on logic without accounting for uncertainty or ambiguity.
    \end{block}

    \begin{block}{Future Directions}
        Ongoing efforts aim to create hybrid systems that integrate logical reasoning with probabilistic methods to better manage complexity and uncertainty.
    \end{block}
\end{frame}

\begin{frame}[fragile]
    \frametitle{Challenges in Logical Reasoning - Conclusion}
    % Conclusion of the challenges in logical reasoning
    Addressing these challenges is essential for harnessing the full potential of logical reasoning in AI. Overcoming limitations like computational complexity, representation issues, and integration capabilities will lead us to more robust and versatile AI systems.

    \begin{block}{Final Note}
        Logical reasoning is just one facet of AI; understanding its challenges enhances our ability to build smarter applications capable of tackling real-world problems.
    \end{block}
\end{frame}

\begin{frame}
    \frametitle{Tools for Logic Reasoning in AI}
    \begin{block}{Overview}
        Logic reasoning is pivotal in AI for making inferences and solving problems. 
        Tools like Prolog help implement logical systems, facilitating reasoning in AI applications.
    \end{block}
\end{frame}

\begin{frame}
    \frametitle{What is Logic Reasoning?}
    Logic reasoning is the process of deducing new knowledge from existing facts using logical principles.
    
    \begin{itemize}
        \item Allows informed decision-making and problem-solving based on data and rules.
    \end{itemize}
    
    \begin{block}{Key Types of Logic}
        \begin{itemize}
            \item \textbf{Propositional Logic}: Deals with true or false propositions using connectives (AND, OR, NOT).
            \item \textbf{First-Order Logic (FOL)}: Extends propositional logic with objects, relations, and quantifiers.
        \end{itemize}
    \end{block}
\end{frame}

\begin{frame}
    \frametitle{Key Logic Reasoning Tools in AI}
    
    \textbf{Prolog}
    \begin{itemize}
        \item \textbf{Description}: Logic programming language for expressing complex relationships and rules.
        \item \textbf{Functionality}: Uses facts and rules for deduction, based on backward chaining and unification.
    \end{itemize}
    
    \begin{block}{Example in Prolog}
    \begin{lstlisting}[language=prolog]
    % Facts
    parent(john, mary).
    parent(mary, lucas).

    % Rule
    grandparent(X, Y) :- parent(X, Z), parent(Z, Y).

    % Query
    ?- grandparent(john, lucas).
    % Expected Output: true.
    \end{lstlisting}
    \end{block}
\end{frame}

\begin{frame}
    \frametitle{Other Logic Tools}
    \begin{itemize}
        \item \textbf{SAT Solvers}: Determine the satisfiability of propositional logic formulas.
        \item \textbf{Answer Set Programming (ASP)}: Solves problems defined by rules and constraints.
        \item \textbf{Datalog}: Declarative programming for databases, exploring queries based on facts and rules.
    \end{itemize}
\end{frame}

\begin{frame}
    \frametitle{Role of Tools in Logic Reasoning}
    \begin{itemize}
        \item \textbf{Automated Inference}: Automatically infers new knowledge, reducing human error.
        \item \textbf{Data Representation}: Represents complex relationships structurally for analysis.
        \item \textbf{Problem Solving}: Tackles complex problems like puzzles and planning.
    \end{itemize}
\end{frame}

\begin{frame}
    \frametitle{Key Points to Emphasize}
    \begin{itemize}
        \item Logic reasoning tools are essential for intelligent systems capable of natural reasoning.
        \item Prolog demonstrates effective implementation of reasoning in various AI applications.
        \item Tool choice depends on logic requirements and the nature of the AI issue being solved.
    \end{itemize}
\end{frame}

\begin{frame}
    \frametitle{Conclusion}
    The role of logic reasoning tools, like Prolog, is crucial in AI. These tools automate reasoning processes and enhance the capability to manage complex queries and relationships, significantly contributing to future advancements in AI.

    \begin{block}{Next Steps}
        Explore a real-world case study on effective use of logic reasoning in AI solutions.
    \end{block}
\end{frame}

\begin{frame}[fragile]
    \frametitle{Case Study: Logic in AI Solutions}
    % Introduction to Logic in AI
    Logic reasoning is foundational in artificial intelligence (AI), enabling machines to draw conclusions from facts or premises. 
    \begin{itemize}
        \item Grounded in formal systems: propositional and first-order logic.
        \item Allows structured decision-making.
    \end{itemize}
\end{frame}

\begin{frame}[fragile]
    \frametitle{Real-World Application: Automated Theorem Proving}
    % Application of logic in automated theorem proving
    \textbf{Context:} Automated theorem proving is key in AI, useful for software verification and formal methods.
    
    \textbf{Example:} A software team ensures code meets safety properties using an AI theorem prover like Coq.
    \begin{enumerate}
        \item Formalize properties the software must satisfy.
        \item Generate proofs validating code fulfillment of these properties automatically.
    \end{enumerate}
    
    % Key Points
    \begin{itemize}
        \item AI uses logical rules to explore proof paths.
        \item Reduces human error and increases reliability.
    \end{itemize}
\end{frame}

\begin{frame}[fragile]
    \frametitle{Logic Reasoning in Knowledge Representation}
    % Logic in knowledge representation
    Logic is crucial in knowledge representation within AI systems.
    \begin{itemize}
        \item Logical languages represent complex relationships and rules.
        \item Example: An AI managing hospital appointments can infer based on:
        \begin{itemize}
            \item "All patients must be registered."
            \item "If a patient has an appointment, they must be seen by a doctor."
        \end{itemize}
        \item AI answers queries on appointment availability with precision.
    \end{itemize}
\end{frame}

\begin{frame}[fragile]
    \frametitle{Case Study: IBM Watson in Healthcare}
    % IBM Watson and logical reasoning
    \textbf{Context:} IBM Watson employs logic reasoning for interpreting unstructured data in healthcare.
    
    \textbf{Key Features:}
    \begin{itemize}
        \item Natural Language Processing (NLP): Transforms patient data into logical statements.
        \item Inference Engine: Uses first-order logic to derive conclusions from medical literature.
    \end{itemize}
    
    \textbf{Impact:}
    \begin{itemize}
        \item Recommends treatment options based on evidence.
        \item Explains logic behind recommendations, aiding clinical decision-making.
    \end{itemize}
\end{frame}

\begin{frame}[fragile]
    \frametitle{Conclusion and Key Takeaways}
    % Conclusion for the case study
    Logic reasoning equips AI systems to handle ambiguity and automate complex decision-making.
    \begin{itemize}
        \item Enhances accuracy in applications like theorem proving and healthcare.
        \item Integrating logical frameworks enables effective decision-making.
    \end{itemize}
    
    \textbf{Key Takeaways:}
    \begin{itemize}
        \item Logic is fundamental to AI decisions.
        \item Automated theorem proving is vital in software verification.
        \item Knowledge representation allows efficient querying and decision-making.
    \end{itemize}
    
    This case study highlights the practical importance of logical reasoning in contemporary AI solutions.
\end{frame}

\begin{frame}[fragile]
    \frametitle{Conclusion and Future Directions - Importance of Logic in AI}
    % Summarizing the importance of logic reasoning in AI
    \begin{itemize}
        \item \textbf{Foundation of Intelligent Systems}:
        Logic reasoning is crucial for structured decision-making in AI systems, enabling machines to derive conclusions from premises.
        \begin{itemize}
            \item \textit{Example:} In natural language processing, logic assists in grammar parsing and understanding context.
        \end{itemize}
        
        \item \textbf{Problem Solving and Automation}:
        Logical frameworks support automated reasoning, aiding AI in efficiently solving complex problems.
        \begin{itemize}
            \item \textit{Illustration:} An AI robot in a smart home uses logic to infer when to turn on lights based on commands.
        \end{itemize}
        
        \item \textbf{Explainability and Transparency}:
        Logical reasoning is key for creating explainable AI systems, offering clear justifications for decisions.
        \begin{itemize}
            \item \textit{Key Point:} This transparency is vital in fields like healthcare to understand AI decision rationale.
        \end{itemize}
    \end{itemize}
\end{frame}

\begin{frame}[fragile]
    \frametitle{Conclusion and Future Directions - Future Trends in Logic Reasoning}
    % Exploring future trends in logic reasoning
    \begin{itemize}
        \item \textbf{Integration of Logic with Machine Learning}:
        The future involves blending traditional logical reasoning with machine learning to improve reasoning under uncertainty.
        \begin{itemize}
            \item \textit{Example:} Logic-based models that can learn from data while maintaining consistency will enhance AI applications.
        \end{itemize}
        
        \item \textbf{Advances in Automated Theorem Proving}:
        Research will focus on making automated theorem proving systems more efficient and complex.
        \begin{itemize}
            \item \textit{Implication:} This will lead to more reliable software systems with fewer vulnerabilities.
        \end{itemize}

        \item \textbf{Logic in Quantum Computing}:
        As quantum computing progresses, logic will be crucial for developing algorithms based on quantum principles.
        \begin{itemize}
            \item \textit{Consideration:} This intersection may transform problem-solving in cryptography and optimization.
        \end{itemize}
        
        \item \textbf{Incorporating Non-Monotonic Logic}:
        Understanding non-monotonic reasoning enables AI models to adapt dynamically to new evidence.
        \begin{itemize}
            \item \textit{Example:} A self-driving car can update its understanding of traffic rules to ensure safe driving decisions.
        \end{itemize}
    \end{itemize}
\end{frame}

\begin{frame}[fragile]
    \frametitle{Conclusion and Future Directions - Key Points and References}
    % Key points to emphasize and references for further study
    \begin{itemize}
        \item \textbf{Key Points to Emphasize}:
        \begin{itemize}
            \item Logic is essential for structured problem-solving and decision-making in AI.
            \item Future trends involve improved integration of logic with machine learning and advances in reasoning techniques.
            \item The evolution of logical reasoning will enhance transparency and interactivity in AI systems.
        \end{itemize}
        
        \item \textbf{References for Further Study}:
        \begin{itemize}
            \item *Artificial Intelligence: A Modern Approach* by Stuart Russell and Peter Norvig
            \item Research articles on the integration of logic and machine learning methodologies.
        \end{itemize}
    \end{itemize}
\end{frame}


\end{document}