\documentclass[aspectratio=169]{beamer}

% Theme and Color Setup
\usetheme{Madrid}
\usecolortheme{whale}
\useinnertheme{rectangles}
\useoutertheme{miniframes}

% Additional Packages
\usepackage[utf8]{inputenc}
\usepackage[T1]{fontenc}
\usepackage{graphicx}
\usepackage{booktabs}
\usepackage{listings}
\usepackage{amsmath}
\usepackage{amssymb}
\usepackage{xcolor}
\usepackage{tikz}
\usepackage{pgfplots}
\pgfplotsset{compat=1.18}
\usetikzlibrary{positioning}
\usepackage{hyperref}

% Custom Colors
\definecolor{myblue}{RGB}{31, 73, 125}
\definecolor{mygray}{RGB}{100, 100, 100}
\definecolor{mygreen}{RGB}{0, 128, 0}
\definecolor{myorange}{RGB}{230, 126, 34}
\definecolor{mycodebackground}{RGB}{245, 245, 245}

% Set Theme Colors
\setbeamercolor{structure}{fg=myblue}
\setbeamercolor{frametitle}{fg=white, bg=myblue}
\setbeamercolor{title}{fg=myblue}
\setbeamercolor{section in toc}{fg=myblue}
\setbeamercolor{item projected}{fg=white, bg=myblue}
\setbeamercolor{block title}{bg=myblue!20, fg=myblue}
\setbeamercolor{block body}{bg=myblue!10}
\setbeamercolor{alerted text}{fg=myorange}

% Set Fonts
\setbeamerfont{title}{size=\Large, series=\bfseries}
\setbeamerfont{frametitle}{size=\large, series=\bfseries}
\setbeamerfont{caption}{size=\small}
\setbeamerfont{footnote}{size=\tiny}

% Document Start
\begin{document}

\frame{\titlepage}

\begin{frame}[fragile]
    \frametitle{Introduction to Artificial Intelligence}
    \begin{block}{What is Artificial Intelligence (AI)?}
        Artificial Intelligence (AI) is a branch of computer science focused on creating systems that perform tasks requiring human intelligence. This includes:
    \end{block}
    \begin{itemize}
        \item Decision-making
        \item Problem-solving
        \item Natural language understanding
        \item Pattern recognition
    \end{itemize}
\end{frame}

\begin{frame}[fragile]
    \frametitle{Importance of AI in Today's World}
    \begin{enumerate}
        \item \textbf{Widespread Application:}
        \begin{itemize}
            \item \textbf{Healthcare:} Revolutionizing diagnostics and patient care.
            \item \textbf{Finance:} Real-time monitoring for fraud detection and investment management.
            \item \textbf{Transportation:} Enabling autonomous navigation and traffic optimization.
        \end{itemize}
        \item \textbf{Transforming Industries:} AI impacts agriculture, manufacturing, and retail.
        \item \textbf{Data-Driven Decision Making:} Rapid analysis of large data sets for informed decisions.
    \end{enumerate}
\end{frame}

\begin{frame}[fragile]
    \frametitle{Key Concepts in AI}
    \begin{itemize}
        \item \textbf{Machine Learning:} Systems learn from data and improve without explicit programming.
        \begin{itemize}
            \item Example: Email spam detection.
        \end{itemize}
        \item \textbf{Natural Language Processing (NLP):} Computers interpret and process human language.
        \begin{itemize}
            \item Example: Virtual assistants like Siri or Alexa.
        \end{itemize}
        \item \textbf{Computer Vision:} Interpreting visual data to make decisions.
        \begin{itemize}
            \item Example: Facial recognition technology.
        \end{itemize}
    \end{itemize}
\end{frame}

\begin{frame}[fragile]
    \frametitle{History of AI - Introduction}
    \begin{block}{Introduction to AI Development}
        The field of Artificial Intelligence (AI) has experienced remarkable growth and transformation since its inception. This slide summarizes the key milestones that have shaped AI, illustrating its evolution from early theoretical concepts to sophisticated systems.
    \end{block}
\end{frame}

\begin{frame}[fragile]
    \frametitle{History of AI - Key Milestones}
    \begin{enumerate}
        \item {\bf 1943 - Neural Networks:}
        \begin{itemize}
            \item Key Point: Warren McCulloch and Walter Pitts introduced a simple neural network model.
            \item Explanation: This laid the groundwork for later AI research through a mathematical framework for understanding neural processes.
        \end{itemize}

        \item {\bf 1950 - Turing Test:}
        \begin{itemize}
            \item Key Point: Alan Turing published "Computing Machinery and Intelligence".
            \item Explanation: The Turing Test assessed a machine's ability to exhibit intelligent behavior indistinguishable from a human.
        \end{itemize}

        \item {\bf 1956 - Dartmouth Conference:}
        \begin{itemize}
            \item Key Point: Birthplace of AI.
            \item Explanation: Organizers proposed that aspects of learning and intelligence could be accurately described for machine simulation.
        \end{itemize}
    \end{enumerate}
\end{frame}

\begin{frame}[fragile]
    \frametitle{History of AI - Continued Milestones}
    \begin{enumerate}
        \setcounter{enumi}{3} % Start from 4
        \item {\bf 1960s - Early AI Programs:}
        \begin{itemize}
            \item Key Point: Programs like ELIZA and SHRDLU.
            \item Explanation: ELIZA illustrated natural language understanding, while SHRDLU handled object manipulation in a virtual setting.
        \end{itemize}

        \item {\bf 1970s - AI Winter:}
        \begin{itemize}
            \item Key Point: Reduced funding and interest in AI.
            \item Explanation: Disappointment with AI capabilities led to slow research progress amid skepticism.
        \end{itemize}

        \item {\bf 1980s - Expert Systems:}
        \begin{itemize}
            \item Key Point: Rebirth of AI through expert systems like MYCIN and XCON.
            \item Explanation: These systems utilized rule-based logic to replicate human decision-making in specific domains.
        \end{itemize}

        \item {\bf 1997 - Deep Blue:}
        \begin{itemize}
            \item Key Point: IBM's Deep Blue defeated chess champion Garry Kasparov.
            \item Explanation: This demonstrated algorithm-driven analysis in complex problem-solving.
        \end{itemize}
    \end{enumerate}
\end{frame}

\begin{frame}[fragile]
    \frametitle{History of AI - Recent Milestones and Conclusion}
    \begin{enumerate}
        \setcounter{enumi}{8} % Start from 9
        \item {\bf 2010s - Deep Learning Era:}
        \begin{itemize}
            \item Key Point: Breakthroughs in neural networks enabled advancements in image and speech recognition.
            \item Explanation: Techniques such as CNNs have driven progress in various fields, including healthcare and autonomous vehicles.
        \end{itemize}

        \item {\bf Present - AI in Everyday Life:}
        \begin{itemize}
            \item Key Point: Pervasive AI technologies in applications like Google Assistant and smart devices.
            \item Explanation: AI is transforming industries from finance to healthcare, significantly impacting daily life.
        \end{itemize}
    \end{enumerate}
    
    \begin{block}{Conclusion}
        The journey of AI illustrates resilience and adaptability, with each milestone contributing to our understanding of intelligence and its machine implementation.
    \end{block}
\end{frame}

\begin{frame}[fragile]
    \frametitle{What is Artificial Intelligence (AI)?}
    \begin{block}{Definition}
        Artificial Intelligence (AI) is a branch of computer science dedicated to creating systems that can mimic human intelligence. These systems are designed to perform tasks that typically require human cognitive abilities, such as reasoning, learning from experience, understanding natural language, and perception.
    \end{block}
\end{frame}

\begin{frame}[fragile]
    \frametitle{Various Definitions of AI}
    \begin{enumerate}
        \item \textbf{General Definition}:
        \begin{itemize}
            \item \textit{AI as Intelligence}: "The ability of a digital computer or computer-controlled robot to perform tasks commonly associated with intelligent beings."
            \item \textbf{Example}: Autonomous vehicles navigating through traffic.
        \end{itemize}
        
        \item \textbf{Functional Perspective}:
        \begin{itemize}
            \item \textit{AI as Task Performance}: "Systems that can perform tasks with human-like proficiency."
            \item \textbf{Example}: Chatbots that can engage in meaningful conversations with users.
        \end{itemize}
    \end{enumerate}
\end{frame}

\begin{frame}[fragile]
    \frametitle{Continuing Definitions of AI}
    \begin{enumerate}
        \setcounter{enumii}{2}
        \item \textbf{Cognitive Science Angle}:
        \begin{itemize}
            \item \textit{AI as Cognitive Simulation}: "AI attempts to simulate human thinking processes."
            \item \textbf{Example}: Expert systems that provide medical diagnoses based on a database of information.
        \end{itemize}
        
        \item \textbf{Technological Approach}:
        \begin{itemize}
            \item \textit{AI as Technology Utilization}: "Systems designed using algorithms and neural networks to analyze large datasets."
            \item \textbf{Example}: Recommendation systems on platforms like Netflix or Amazon.
        \end{itemize}
        
        \item \textbf{Philosophical Perspective}:
        \begin{itemize}
            \item \textit{AI as Simulation of Human Mind}: "Is it possible for machines to truly think or feel?"
            \item Discussed in the context of the Turing Test, proposed by Alan Turing, which assesses whether a machine can exhibit intelligent behavior indistinguishable from a human.
        \end{itemize}
    \end{enumerate}
\end{frame}

\begin{frame}[fragile]
    \frametitle{Key Points and Illustrations}
    \begin{itemize}
        \item \textbf{Diverse Interpretations}: AI does not have a singular definition; it varies across different contexts including philosophy, technology, and cognitive science.
        \item \textbf{Core Functions}: Focus on its ability to learn, reason, and adapt—elements crucial for intelligent behavior.
        \item \textbf{Impactful Examples}: Real-world applications illustrate how AI enhances user experiences in fields like healthcare and entertainment.
    \end{itemize}
    
    \begin{block}{Illustrations}
        \begin{itemize}
            \item Turing Test Diagram
            \item Infographic of AI applications (Autonomous vehicles, Virtual assistants, Fraud detection systems)
        \end{itemize}
    \end{block}
\end{frame}

\begin{frame}[fragile]
    \frametitle{Conclusion and Next Steps}
    \begin{block}{Conclusion}
        Understanding AI's multifaceted definitions allows us to recognize its diverse applications and implications in our world. 
    \end{block}
    
    \textbf{Next Step}: We will delve into the core concepts in AI, focusing on elements such as machine learning and neural networks that drive these intelligent systems.
\end{frame}

\begin{frame}[fragile]
    \frametitle{Core Concepts in AI - Overview}
    \begin{block}{Objective}
        To understand the fundamental concepts of Artificial Intelligence (AI), focusing on:
        \begin{itemize}
            \item Machine Learning (ML)
            \item Neural Networks (NN)
            \item Natural Language Processing (NLP)
        \end{itemize}
    \end{block}
\end{frame}

\begin{frame}[fragile]
    \frametitle{Core Concepts in AI - Machine Learning}
    \begin{block}{1. Machine Learning (ML)}
        \begin{itemize}
            \item \textbf{Definition}: A subset of AI enabling systems to learn from data and make decisions with minimal human intervention.
            \item \textbf{Types of Machine Learning}:
                \begin{itemize}
                    \item \textbf{Supervised Learning}: Training on labeled data (e.g., predicting house prices).
                    \item \textbf{Unsupervised Learning}: Finding hidden patterns in unlabeled data (e.g., customer segmentation).
                    \item \textbf{Reinforcement Learning}: Learning optimal actions through trial and error (e.g., game-playing AI).
                \end{itemize}
        \end{itemize}
        
        \textbf{Key Point}: Machine learning drives modern AI applications, from recommendation systems to self-driving cars.
    \end{block}
\end{frame}

\begin{frame}[fragile]
    \frametitle{Core Concepts in AI - Neural Networks}
    \begin{block}{2. Neural Networks (NN)}
        \begin{itemize}
            \item \textbf{Definition}: Computational models inspired by the human brain, composed of interconnected nodes (neurons).
            \item \textbf{How They Work}:
                \begin{itemize}
                    \item \textbf{Input Layer}: Receives input data.
                    \item \textbf{Hidden Layers}: Process data through transformations.
                    \item \textbf{Output Layer}: Provides final output.
                \end{itemize}
        \end{itemize}
        
        \begin{equation}
            y = f\left( \sum (w_i \cdot x_i) + b \right)
        \end{equation}
        
        Where:
        \begin{itemize}
            \item $y$ is the output
            \item $x_i$ are the inputs
            \item $w_i$ are the weights
            \item $b$ is the bias
            \item $f$ is the activation function
        \end{itemize}
        \textbf{Example}: Image recognition systems that classify images, such as distinguishing between dogs and cats.
    \end{block}
\end{frame}

\begin{frame}[fragile]
    \frametitle{Core Concepts in AI - Natural Language Processing}
    \begin{block}{3. Natural Language Processing (NLP)}
        \begin{itemize}
            \item \textbf{Definition}: A field at the intersection of AI and linguistics, focusing on the interaction between computers and human language.
            \item \textbf{Applications}:
                \begin{itemize}
                    \item \textbf{Chatbots}: Automated conversational agents.
                    \item \textbf{Sentiment Analysis}: Determining emotional tone of text (e.g., reviews).
                    \item \textbf{Text Analytics}: Extracting insights from text data (e.g., keyword extraction).
                \end{itemize}
        \end{itemize}
        \textbf{Key Point}: NLP enables machines to understand and respond to human language, facilitating seamless technology interactions.
    \end{block}
\end{frame}

\begin{frame}[fragile]
    \frametitle{Conclusion and Reflection}
    \begin{block}{Conclusion}
        Understanding core concepts—Machine Learning, Neural Networks, and Natural Language Processing—provides a solid foundation for further AI studies.
    \end{block}

    \begin{block}{Example Questions for Reflection}
        \begin{itemize}
            \item How might machine learning impact industries such as healthcare or finance?
            \item In what ways do neural networks differ from traditional statistical methods?
            \item What ethical considerations arise with natural language processing technologies?
        \end{itemize}
    \end{block}
    
    \begin{block}{Additional Note}
        For practical engagement, explore tools like Scikit-learn for ML, TensorFlow/Keras for NN, and NLTK/SpaCy for NLP.
    \end{block}
\end{frame}

\begin{frame}[fragile]
    \frametitle{Foundational Theories}
    \begin{block}{Overview}
        AI draws upon foundational theories for simulating human-like reasoning and knowledge processing.
        Two essential theories are \textbf{Knowledge Representation} and \textbf{Reasoning}.
    \end{block}
\end{frame}

\begin{frame}[fragile]
    \frametitle{Knowledge Representation}
    \begin{block}{Definition}
        Knowledge representation refers to how information is structured and encoded for AI systems.
    \end{block}
    \begin{itemize}
        \item \textbf{Key Components:}
        \begin{itemize}
            \item \textbf{Entities:} Objects or concepts (e.g., "Dog", "New York City").
            \item \textbf{Attributes:} Characteristics (e.g., "Breed: Labrador").
            \item \textbf{Relations:} Connections (e.g., "is a pet of").
        \end{itemize}
    
        \item \textbf{Types of Knowledge Representation:}
        \begin{itemize}
            \item \textbf{Semantic Networks:} Graph structures representing knowledge.
            \item \textbf{Frames:} Data structures similar to object-oriented programming.
            \item \textbf{Rules:} Logical statements defining relationships (IF...THEN).
        \end{itemize}
    \end{itemize}
\end{frame}

\begin{frame}[fragile]
    \frametitle{Reasoning}
    \begin{block}{Definition}
        Reasoning is the process through which AI systems derive conclusions or predictions from knowledge.
    \end{block}
    \begin{itemize}
        \item \textbf{Types of Reasoning:}
        \begin{enumerate}
            \item \textbf{Deductive Reasoning:} Specific conclusions from general principles.
            \item \textbf{Inductive Reasoning:} General principles from specific examples.
            \item \textbf{Abductive Reasoning:} Inferring the most likely cause from an observation.
        \end{enumerate}
    \end{itemize}
    
    \begin{block}{Key Points}
        \begin{itemize}
            \item Knowledge Representation is crucial for AI's understanding of the world.
            \item Reasoning enables decision-making and problem-solving.
            \item The choice of representation and reasoning impacts AI application efficiency.
        \end{itemize}
    \end{block}
\end{frame}

\begin{frame}[fragile]
    \frametitle{Example Illustration}
    \begin{block}{Semantic Network Example}
    Consider the following representation:
    \begin{lstlisting}
    [Animal] --is a--> [Dog]
    [Dog] --has--> [Name: Max]
    [Dog] --behaves--> [Barks]
    \end{lstlisting}
    This illustrates relationships among entities, aiding AI in processing knowledge.
    \end{block}
\end{frame}

\begin{frame}[fragile]
    \frametitle{Agent Architecture Types}
    \begin{block}{Introduction to Agent Architectures}
        Agent architectures are fundamental structures that define how artificial intelligence (AI) agents perceive their environment, make decisions, and act to achieve specified goals. Selecting the appropriate architecture is crucial for implementing effective AI systems.
    \end{block}
\end{frame}

\begin{frame}[fragile]
    \frametitle{Types of Agent Architectures}
    \begin{enumerate}
        \item \textbf{Simple Reflex Agents}
        \begin{itemize}
            \item \textbf{Concept:} React to current environmental conditions without memory of past states.
            \item \textbf{Example:} A thermostat turns on heating when temperature drops below a set threshold.
        \end{itemize}
        
        \item \textbf{Model-Based Reflex Agents}
        \begin{itemize}
            \item \textbf{Concept:} Utilize an internal model of the world to maintain state information over time.
            \item \textbf{Example:} An autonomous vacuum cleaner maps the room and recalls areas cleaned.
        \end{itemize}
        
        \item \textbf{Goal-Based Agents}
        \begin{itemize}
            \item \textbf{Concept:} Make decisions based on achieving specific goals.
            \item \textbf{Example:} A chess-playing program analyzes moves to maximize winning chances.
        \end{itemize}
    \end{enumerate}
\end{frame}

\begin{frame}[fragile]
    \frametitle{Types of Agent Architectures (Continued)}
    \begin{enumerate}
        \setcounter{enumi}{3} % continues numbering from the previous frame
        \item \textbf{Utility-Based Agents}
        \begin{itemize}
            \item \textbf{Concept:} Assess choices by measuring the utility (value) of outcomes.
            \item \textbf{Example:} Assistant software suggests travel routes based on time, cost, and user preferences.
        \end{itemize}
        
        \item \textbf{Learning Agents}
        \begin{itemize}
            \item \textbf{Concept:} Evolve and improve performance through experience and learning from interactions.
            \item \textbf{Example:} A recommendation system learns user preferences over time to suggest products.
        \end{itemize}
    \end{enumerate}

    \begin{block}{Key Points to Emphasize}
        \begin{itemize}
            \item Flexibility and range of architectures.
            \item Critical role of decision-making in intelligent behavior.
            \item Increased complexity may necessitate sophisticated architectures.
        \end{itemize}
    \end{block}
\end{frame}

\begin{frame}[fragile]
    \frametitle{Reactive Agents - Overview}
    \begin{block}{Definition}
        Reactive agents operate based on the current environment without maintaining an internal model or engaging in complex reasoning. Their decisions are made in real-time.
    \end{block}
\end{frame}

\begin{frame}[fragile]
    \frametitle{Reactive Agents - Key Characteristics}
    \begin{itemize}
        \item \textbf{Simplicity:} Straightforward decision-making processes.
        \item \textbf{Immediate Responses:} Quick reactions to environmental changes.
        \item \textbf{No Planning Required:} Decisions based on fixed rules or learned behaviors.
    \end{itemize}
\end{frame}

\begin{frame}[fragile]
    \frametitle{Reactive Agents - Decision Making}
    \begin{enumerate}
        \item \textbf{Sensors:} Perceive the environment and gather current state information.
        \item \textbf{Actuators:} Perform actions based on sensor input.
        \item \textbf{Condition-Action Rules:} Predefined rules dictating responses to specific stimuli.
    \end{enumerate}
\end{frame}

\begin{frame}[fragile]
    \frametitle{Reactive Agents - Example}
    \begin{block}{Reactive Robot Scenario}
        \begin{itemize}
            \item \textbf{Sensors:} Infrared or sonar for obstacle detection.
            \item \textbf{Rules:}
            \begin{itemize}
                \item If an obstacle is detected on the left, turn right.
                \item If an obstacle is detected in front, reverse and then turn left.
            \end{itemize}
            \item \textbf{Behavior:} Governed by a finite set of rules for immediate reactions.
        \end{itemize}
    \end{block}
\end{frame}

\begin{frame}[fragile]
    \frametitle{Reactive Agents - Advantages and Limitations}
    \begin{columns}
        \begin{column}{0.5\textwidth}
            \textbf{Advantages:}
            \begin{itemize}
                \item Speed: Quick reaction times for real-time applications.
                \item Ease of Implementation: Simple architectures reduce complexity.
            \end{itemize}
        \end{column}
        \begin{column}{0.5\textwidth}
            \textbf{Limitations:}
            \begin{itemize}
                \item Lack of Learning: Minimal adaptation capability.
                \item Limited Complexity: Inability to perform multi-step reasoning.
            \end{itemize}
        \end{column}
    \end{columns}
\end{frame}

\begin{frame}[fragile]
    \frametitle{Reactive Agents - Summary}
    \begin{itemize}
        \item Reactive agents respond immediately to environmental states.
        \item Utilize condition-action rules for effective performance in simple tasks.
        \item While excelling in speed and simplicity, they struggle with complex scenarios requiring advanced planning or learning.
    \end{itemize}
\end{frame}

\begin{frame}[fragile]
    \frametitle{Deliberative Agents - Overview}
    \begin{block}{Overview}
        Deliberative agents are intelligent agents in artificial intelligence focused on planning and reasoning based on knowledge and environment. They differ from reactive agents by engaging in thoughtful processes to achieve goals through a sequence of actions.
    \end{block}
\end{frame}

\begin{frame}[fragile]
    \frametitle{Deliberative Agents - Key Concepts}
    \begin{enumerate}
        \item \textbf{Definition}:
        \begin{itemize}
            \item Deliberative agents reason about actions before executing them, considering knowledge and expected consequences.
        \end{itemize}
        
        \item \textbf{Components}:
        \begin{itemize}
            \item \textbf{Knowledge Base}: Repository of facts/rules used to understand the environment.
            \item \textbf{Reasoning Module}: Processes information to draw inferences and make decisions.
            \item \textbf{Planning Module}: Generates action sequences to achieve goals considering constraints.
        \end{itemize}
        
        \item \textbf{Types of Reasoning}:
        \begin{itemize}
            \item Deductive: Specific conclusions from general facts.
            \item Inductive: General rules from specific instances.
            \item Abductive: Best explanation from available data.
        \end{itemize}
    \end{enumerate}
\end{frame}

\begin{frame}[fragile]
    \frametitle{Deliberative Agents - Example and Key Points}
    \begin{block}{Example: Autonomous Robot}
        Imagine a robot navigating to deliver a package. It:
        \begin{enumerate}
            \item Locates its position.
            \item Identifies the destination.
            \item Avoids obstacles based on coordinates.
        \end{enumerate}
        The robot calculates the shortest path while considering blockages and alternatives.
    \end{block}
    
    \begin{block}{Key Points}
        \begin{itemize}
            \item Require more computational resources for planning.
            \item Excel in complex environments needing careful consideration.
            \item Effectiveness hinges on the completeness of the knowledge base.
        \end{itemize}
    \end{block}
\end{frame}

\begin{frame}[fragile]
    \frametitle{Deliberative Agents - Code Snippet and Conclusion}
    \begin{block}{Pseudocode Example}
        \begin{lstlisting}[language=Python]
def plan_route(current_location, destination, obstacles):
    possible_paths = generate_paths(current_location, destination)
    feasible_paths = filter_paths(possible_paths, obstacles)
    return choose_best_path(feasible_paths)
        \end{lstlisting}
        This snippet illustrates a planning approach, generating and evaluating paths based on environmental knowledge.
    \end{block}
    
    \begin{block}{Conclusion}
        Deliberative agents enhance AI by integrating reasoning and planning, solving complex problems, and achieving goals systematically. Understanding their operation is crucial for developing intelligent systems capable of autonomous behavior.
    \end{block}
\end{frame}

\begin{frame}[fragile]
    \frametitle{Hybrid Agents - Overview}
    \begin{block}{Definition}
        Hybrid agents combine the strengths of reactive and deliberative approaches in artificial intelligence (AI) to enhance performance in dynamic environments.
    \end{block}
    \begin{itemize}
        \item Adapt to complex environments
        \item Respond to dynamic changes
        \item Reason about actions based on accumulated knowledge
    \end{itemize}
\end{frame}

\begin{frame}[fragile]
    \frametitle{Hybrid Agents - Key Concepts}
    \begin{columns}
        \begin{column}{0.5\textwidth}
            \textbf{Reactive Agents:}
            \begin{itemize}
                \item Operate on a stimulus-response basis
                \item Quick and efficient for real-time environments
                \item Lack internal knowledge representation and planning
            \end{itemize}
        \end{column}

        \begin{column}{0.5\textwidth}
            \textbf{Deliberative Agents:}
            \begin{itemize}
                \item Utilize models for action planning
                \item Perform calculations for future states
                \item Suited for complex, foresight-required tasks
            \end{itemize}
        \end{column}
    \end{columns}
\end{frame}

\begin{frame}[fragile]
    \frametitle{How Hybrid Agents Work}
    Hybrid agents integrate the strengths of both reactive and deliberative agents:
    \begin{itemize}
        \item Make quick decisions in urgent situations
        \item Plan for long-term goals
        \item Seamlessly transition between reactive and deliberative modes based on context
    \end{itemize}
\end{frame}

\begin{frame}[fragile]
    \frametitle{Example of Hybrid Agents}
    Consider a \textbf{robotic vacuum cleaner}:
    \begin{itemize}
        \item \textbf{Reactive Functions:} Reacts to obstacles using simple sensor inputs (e.g., stopping, navigating around furniture).
        \item \textbf{Deliberative Functions:} Plans effective cleaning paths based on its map of the house.
    \end{itemize}
\end{frame}

\begin{frame}[fragile]
    \frametitle{Architecture of Hybrid Agents}
    \begin{enumerate}
        \item \textbf{Sensors:} Gather real-time data from the environment.
        \item \textbf{Reactive Layer:} Immediate data processing for prompt responses.
        \item \textbf{Deliberative Layer:} Updates knowledge base and conducts planning.
        \item \textbf{Decision-Making Module:} Chooses between reactive or deliberative actions based on conditions.
    \end{enumerate}
\end{frame}

\begin{frame}[fragile]
    \frametitle{Advantages and Challenges of Hybrid Agents}
    \textbf{Advantages:}
    \begin{itemize}
        \item Flexibility to handle routine and complex tasks
        \item Adaptability to unexpected events
        \item Efficient by mitigating limitations of each approach
    \end{itemize}

    \textbf{Challenges:}
    \begin{itemize}
        \item Increased complexity in design and implementation
        \item Potential conflicts between reactive and deliberative decisions may require resolution mechanisms
    \end{itemize}
\end{frame}

\begin{frame}[fragile]
    \frametitle{Conclusion}
    Hybrid agents are a significant advancement in AI, allowing systems to function effectively across diverse scenarios.
    \begin{itemize}
        \item Respond quickly to changes while understanding broader objectives
        \item Valuable in dynamic environments
    \end{itemize}
\end{frame}

\begin{frame}[fragile]
    \frametitle{Key Points to Remember}
    \begin{itemize}
        \item Hybrid agents combine strengths of reactive and deliberative models
        \item Major advantages include flexibility and adaptability
        \item Use cases like robotic vacuum cleaners illustrate functionality
        \item Architecture includes reactive and deliberative layers
    \end{itemize}
\end{frame}

\begin{frame}[fragile]
    \frametitle{Search Algorithms in AI - Introduction}
    \begin{block}{Introduction to Search Algorithms}
        Search algorithms are foundational components in artificial intelligence (AI) that help find solutions to problems by exploring potential paths and options. 
        They enable agents to navigate through complex problem spaces to identify optimal or satisfactory solutions.
    \end{block}
\end{frame}

\begin{frame}[fragile]
    \frametitle{Search Algorithms in AI - Significance}
    \begin{block}{Significance of Search Algorithms}
        \begin{itemize}
            \item \textbf{Problem Solving}: Essential for AI applications across various domains, including robotics, game playing, and natural language processing.
            \item \textbf{Efficiency}: Critical for optimizing the performance of AI systems by minimizing resource usage like time and memory.
            \item \textbf{Exploration}: Aid in exploring unknown environments or solution spaces, making them integral to decision-making processes.
        \end{itemize}
    \end{block}
\end{frame}

\begin{frame}[fragile]
    \frametitle{Search Algorithms in AI - Categories}
    \begin{block}{Categories of Search Algorithms}
        \begin{enumerate}
            \item \textbf{Uninformed Search Algorithms}:
            \begin{itemize}
                \item \textbf{Definition}: Explore the search space without additional information about the goal state.
                \item \textbf{Examples}:
                \begin{itemize}
                    \item \textbf{Breadth-First Search (BFS)}: Explores all nodes at the present depth before moving on to nodes at the next depth level.
                    \begin{itemize}
                        \item \textbf{Use Case}: Finding the shortest path in unweighted graphs.
                    \end{itemize}
                    \item \textbf{Depth-First Search (DFS)}: Explores as far down a branch as possible before backtracking.
                    \begin{itemize}
                        \item \textbf{Use Case}: Searching for solutions in puzzle games like Sudoku.
                    \end{itemize}
                \end{itemize}
            \end{itemize}
            \item \textbf{Informed Search Algorithms (Heuristic)}:
            \begin{itemize}
                \item \textbf{Definition}: Use domain knowledge to find solutions more efficiently.
                \item \textbf{Examples}:
                \begin{itemize}
                    \item \textbf{A* Search Algorithm}: 
                    \begin{equation}
                    f(n) = g(n) + h(n)
                    \end{equation}
                    where \( g(n) \) is the cost to reach node \( n \) and \( h(n) \) is the estimated cost from \( n \) to the goal.
                    \begin{itemize}
                        \item \textbf{Use Case}: Pathfinding in maps (like Google Maps).
                    \end{itemize}
                    \item \textbf{Greedy Best-First Search}: 
                    \begin{itemize}
                        \item Expands the node that appears to be the closest to the goal, as estimated by a heuristic.
                        \item \textbf{Use Case}: Game AI where immediate rewards are prioritized.
                    \end{itemize}
                \end{itemize}
            \end{itemize}
        \end{enumerate}
    \end{block}
\end{frame}

\begin{frame}[fragile]
    \frametitle{Reinforcement Learning: Learning Objectives}
    \begin{itemize}
        \item Understand the core principles of reinforcement learning (RL).
        \item Identify key components of RL systems.
        \item Recognize applications of RL in decision-making processes.
    \end{itemize}
\end{frame}

\begin{frame}[fragile]
    \frametitle{What is Reinforcement Learning?}
    Reinforcement Learning (RL) is a type of machine learning where an agent learns to make decisions by taking actions in an environment to maximize cumulative rewards. 
    \begin{itemize}
        \item Unlike supervised learning, RL uses trial-and-error interactions for learning outcomes.
    \end{itemize}
\end{frame}

\begin{frame}[fragile]
    \frametitle{Key Concepts in Reinforcement Learning}
    \begin{enumerate}
        \item \textbf{Agent}: The learner or decision maker.
        \item \textbf{Environment}: Everything the agent interacts with; it provides feedback based on the agent's actions.
        \item \textbf{Action (A)}: Choices made by the agent to interact with the environment.
        \item \textbf{State (S)}: A representation of the current situation of the agent in the environment.
        \item \textbf{Reward (R)}: Feedback from the environment to evaluate the effectiveness of an action taken by the agent.
    \end{enumerate}
\end{frame}

\begin{frame}[fragile]
    \frametitle{How Reinforcement Learning Works}
    \begin{block}{Exploration vs. Exploitation}
        The agent must balance exploring new actions (exploration) and utilizing known actions that yield the highest rewards (exploitation).
    \end{block}
    \begin{block}{Learning Process}
        \begin{itemize}
            \item The agent receives a state from the environment.
            \item It selects an action based on its policy.
            \item The agent receives a reward from the environment based on the action taken.
            \item The state changes based on the action, and the process repeats.
        \end{itemize}
    \end{block}
\end{frame}

\begin{frame}[fragile]
    \frametitle{Example: Robot Vacuum Cleaner}
    \begin{itemize}
        \item \textbf{States}: Different locations in a room (e.g., clean, dirty).
        \item \textbf{Actions}: Move forward, turn left, turn right, or start cleaning.
        \item \textbf{Rewards}:
            \begin{itemize}
                \item +1 reward for cleaning a dirty spot.
                \item -1 penalty if it hits a wall.
            \end{itemize}
    \end{itemize}
    The vacuum learns to navigate the room by receiving feedback on its cleaning actions, adjusting its strategy to maximize the cleaned area and minimize obstacles.
\end{frame}

\begin{frame}[fragile]
    \frametitle{Key Algorithms in RL}
    \textbf{Q-learning}: A model-free RL algorithm that learns the value of actions in states without needing a model of the environment.
    \begin{equation}
        Q(S, A) \leftarrow Q(S, A) + \alpha \left[R + \gamma \max_{A'} Q(S', A') - Q(S, A)\right]
    \end{equation}
    where:
    \begin{itemize}
        \item $\alpha$ = learning rate
        \item $\gamma$ = discount factor
        \item $S'$ = next state
    \end{itemize}
\end{frame}

\begin{frame}[fragile]
    \frametitle{Applications of Reinforcement Learning}
    \begin{itemize}
        \item \textbf{Game Playing}: Algorithms like AlphaGo use RL to learn strategies.
        \item \textbf{Robotics}: Autonomous robots learn to navigate complex environments and complete tasks.
        \item \textbf{Finance}: Portfolio management systems optimize investment strategies over time.
        \item \textbf{Healthcare}: Personalized treatment plans can be adjusted using RL based on patient data.
    \end{itemize}
\end{frame}

\begin{frame}[fragile]
    \frametitle{Conclusion}
    Reinforcement Learning equips agents to learn optimal behaviors through interaction with their environment, making it invaluable in various domains requiring decision-making processes. By understanding its principles, students can grasp how machines learn and adapt to challenges in real time.
\end{frame}

\begin{frame}[fragile]
    \frametitle{Next Slide Preview}
    Delve into Markov Decision Processes and their role in formalizing decision-making models in RL contexts.
\end{frame}

\begin{frame}[fragile]
    \frametitle{Markov Decision Processes - Introduction}
    \begin{block}{Overview}
        Markov Decision Processes (MDPs) are a mathematical framework for modeling decision-making scenarios involving uncertainty and control by an agent.
    \end{block}
    
    \begin{itemize}
        \item MDPs formalize decision-making through defined states, actions, and rewards.
        \item They establish a foundation for reinforcement learning applications.
    \end{itemize}
\end{frame}

\begin{frame}[fragile]
    \frametitle{Key Components of MDPs}
    \begin{enumerate}
        \item \textbf{States (S)}: All possible situations the agent can be in.
        \item \textbf{Actions (A)}: Moves the agent can make from each state.
        \item \textbf{Transition Function (P)}: 
            \begin{equation}
                P(s' | s, a) = \text{Probability of moving to state } s' \text{ from state } s \text{ by taking action } a
            \end{equation}
        \item \textbf{Rewards (R)}:
            \begin{equation}
                R(s, a) = \text{Reward obtained for taking action } a \text{ in state } s
            \end{equation}
        \item \textbf{Discount Factor ($\gamma$)}: 
        A value between 0 and 1 determining future reward importance.
    \end{enumerate}
\end{frame}

\begin{frame}[fragile]
    \frametitle{MDP Formalization and Example}
    \begin{block}{MDP Formalization}
        An MDP is defined as a tuple (S, A, P, R, $\gamma$). The objective is to find a policy ($\pi$) that maximizes expected cumulative rewards.
    \end{block}
    
    \begin{block}{Example: Robot in a Grid World}
        \begin{itemize}
            \item **States (S)**: Each grid cell.
            \item **Actions (A)**: Move up, down, left, right.
            \item **Transition Function (P)**:
                \begin{itemize}
                    \item E.g., 80\% chance to move right to cell (1,2), 20\% chance to slip to (2,2).
                \end{itemize}
            \item **Rewards (R)**: 
                \begin{itemize}
                    \item +10 for reaching the goal, -1 for each step.
                \end{itemize}
            \item **Discount Factor ($\gamma$)**: 0.9 emphasizes future rewards.
        \end{itemize}
    \end{block}
\end{frame}

\begin{frame}[fragile]
    \frametitle{Overview of Evaluation Metrics and Methods}
    Evaluating the performance of Artificial Intelligence (AI) systems is crucial for understanding their effectiveness, reliability, and applicability in real-world scenarios. This evaluation helps developers and researchers improve their models and make informed decisions based on quantitative and qualitative metrics.
\end{frame}

\begin{frame}[fragile]
    \frametitle{Key Metrics for Evaluation - Part 1}
    \begin{enumerate}
        \item \textbf{Accuracy}  
        \begin{itemize}
            \item Measures the proportion of true results (both true positives and true negatives) among the total number of cases examined.
            \item \textbf{Formula:}  
            \[
            \text{Accuracy} = \frac{TP + TN}{TP + TN + FP + FN}
            \]
            \item \textbf{Example:} If an AI model predicts 80 out of 100 instances correctly, the accuracy is 80\%.
        \end{itemize}

        \item \textbf{Precision}  
        \begin{itemize}
            \item Indicates the accuracy of positive predictions.
            \item \textbf{Formula:}  
            \[
            \text{Precision} = \frac{TP}{TP + FP}
            \]
            \item \textbf{Example:} If a model identifies 30 true positives and makes 10 false positives, then the precision is \( \frac{30}{30+10} = 0.75 \) or 75\%.
        \end{itemize}
    \end{enumerate}
\end{frame}

\begin{frame}[fragile]
    \frametitle{Key Metrics for Evaluation - Part 2}
    \begin{enumerate}
        \setcounter{enumi}{2} % continue from the previous frame
        \item \textbf{Recall (Sensitivity)}  
        \begin{itemize}
            \item Measures the ability of a model to find all relevant cases (true positives).
            \item \textbf{Formula:}  
            \[
            \text{Recall} = \frac{TP}{TP + FN}
            \]
            \item \textbf{Example:} If there are 50 actual positive cases and the model identifies 40, recall is \( \frac{40}{50} = 0.8 \) or 80\%.
        \end{itemize}

        \item \textbf{F1 Score}  
        \begin{itemize}
            \item Combines precision and recall into a single metric.
            \item \textbf{Formula:}  
            \[
            \text{F1 Score} = 2 \times \frac{\text{Precision} \times \text{Recall}}{\text{Precision} + \text{Recall}}
            \]
            \item \textbf{Example:} If precision is 75\% and recall is 80\%, the F1 score would be approximately 77.5\%.
        \end{itemize}

        \item \textbf{ROC-AUC}  
        \begin{itemize}
            \item Evaluates the trade-off between true positive rates and false positive rates.
            \item A score of 1 represents perfect classification; a score of 0.5 indicates random guessing.
        \end{itemize}
    \end{enumerate}
\end{frame}

\begin{frame}[fragile]
    \frametitle{Methods of Evaluation}
    \begin{enumerate}
        \item \textbf{Cross-Validation}  
        \begin{itemize}
            \item Involves dividing the dataset into multiple parts (folds) for training and testing.
            \item Common methods: k-fold and stratified k-fold cross-validation.
        \end{itemize}

        \item \textbf{Holdout Method}  
        \begin{itemize}
            \item Splits the dataset into training and test sets.
            \item The model is trained on the training subset and evaluated on the test subset.
        \end{itemize}

        \item \textbf{Confusion Matrix}  
        \begin{itemize}
            \item A table that describes the performance of a classification model.
            \item Helps analyze aspects like TP, TN, FP, and FN.
        \end{itemize}
    \end{enumerate}
\end{frame}

\begin{frame}[fragile]
    \frametitle{Key Takeaways}
    \begin{itemize}
        \item Accurate evaluation of AI systems is essential for trust and improvement.
        \item Selection of metrics depends on the specific application and costs associated with false positives and false negatives.
        \item Combining multiple metrics provides a better overall performance view than relying on a single metric.
    \end{itemize}
\end{frame}

\begin{frame}[fragile]
    \frametitle{Ethical Considerations in AI - Introduction}
    \begin{block}{Overview}
        As Artificial Intelligence (AI) continues to permeate various aspects of society, understanding the ethical implications and societal impacts becomes crucial.
    \end{block}
    \begin{itemize}
        \item Addressing core ethical principles
        \item Exploring potential benefits and challenges
    \end{itemize}
\end{frame}

\begin{frame}[fragile]
    \frametitle{Ethical Considerations in AI - Core Principles}
    \begin{enumerate}
        \item \textbf{Fairness}
            \begin{itemize}
                \item Definition: Ensuring AI does not reinforce biases or perpetuate discrimination.
                \item Example: Algorithms in hiring must be scrutinized for biases against demographic groups.
            \end{itemize}
        \item \textbf{Accountability}
            \begin{itemize}
                \item Definition: Establishing responsibility for AI decisions.
                \item Example: Liability in an accident involving an autonomous vehicle.
            \end{itemize}
        \item \textbf{Transparency}
            \begin{itemize}
                \item Definition: AI systems should allow scrutiny for decision-making processes.
                \item Example: Disclosure of credit scoring methods in AI lending systems.
            \end{itemize}
        \item \textbf{Privacy}
            \begin{itemize}
                \item Definition: Protecting user data and ensuring informed consent.
                \item Example: Regulations for facial recognition systems to prevent data misuse.
            \end{itemize}
    \end{enumerate}
\end{frame}

\begin{frame}[fragile]
    \frametitle{Ethical Considerations in AI - Societal Impacts}
    \begin{enumerate}
        \item \textbf{Job Displacement}
            \begin{itemize}
                \item Concern: Automation may displace jobs across industries.
                \item Example: AI tools in customer service reducing the need for human agents.
            \end{itemize}
        \item \textbf{Security Risks}
            \begin{itemize}
                \item Concern: Exploitation of AI for malicious purposes.
                \item Example: Deepfake videos misleading viewers and influencing elections.
            \end{itemize}
        \item \textbf{Ethical Dilemmas in Decision-Making}
            \begin{itemize}
                \item Concern: Autonomous systems facing moral choices.
                \item Example: Self-driving cars making decisions in emergency situations.
            \end{itemize}
    \end{enumerate}
\end{frame}

\begin{frame}[fragile]
    \frametitle{AI Applications in Real World - Introduction}
    \begin{block}{Overview}
        AI is transforming numerous domains, impacting how we live and work. This slide explores concrete applications across various industries showcasing AI's real-world significance.
    \end{block}
\end{frame}

\begin{frame}[fragile]
    \frametitle{AI Applications in Key Domains}
    \begin{enumerate}
        \item \textbf{Healthcare}
            \begin{itemize}
                \item \textbf{Use Case:} Diagnostic Systems
                \item \textbf{Impact:} Enhanced diagnostic speed and accuracy.
                \item \textit{Example:} Google's DeepMind's AI detects diseases from retinal scans.
            \end{itemize}
        
        \item \textbf{Finance}
            \begin{itemize}
                \item \textbf{Use Case:} Fraud Detection
                \item \textbf{Impact:} Increased security in digital transactions.
                \item \textit{Example:} PayPal's AI flags suspicious transactions.
            \end{itemize}
    \end{enumerate}
\end{frame}

\begin{frame}[fragile]
    \frametitle{Continued Applications in AI}
    \begin{enumerate}
        \setcounter{enumi}{2} % Continue numbering from previous frame
        \item \textbf{Transportation}
            \begin{itemize}
                \item \textbf{Use Case:} Autonomous Vehicles
                \item \textbf{Impact:} Reduces traffic accidents and improves transport efficiency.
                \item \textit{Example:} Waymo's self-driving cars.
            \end{itemize}
        
        \item \textbf{Retail}
            \begin{itemize}
                \item \textbf{Use Case:} Personalized Shopping Experiences
                \item \textbf{Impact:} Increased sales through tailored marketing.
                \item \textit{Example:} Amazon's product recommendation systems.
            \end{itemize}

        \item \textbf{Manufacturing}
            \begin{itemize}
                \item \textbf{Use Case:} Predictive Maintenance
                \item \textbf{Impact:} Reduced operational costs and increased reliability.
                \item \textit{Example:} General Electric's failure predictions in jet engines.
            \end{itemize}
    \end{enumerate}
\end{frame}

\begin{frame}[fragile]
    \frametitle{Conclusion and Future Directions - Current State of AI}
    \begin{itemize}
        \item \textbf{Widespread Adoption:} AI technologies are extensively implemented in healthcare (e.g., diagnostic systems), finance (e.g., fraud detection), and transportation (e.g., autonomous vehicles).
        \item \textbf{Deep Learning Dominance:} Deep learning remains the cornerstone of modern AI, enabling breakthroughs in NLP and computer vision (e.g., OpenAI's GPT-3, Google's BERT).
        \item \textbf{Agent Architectures:} Various architectures such as rule-based systems, decision trees, and reinforcement learning agents support the development of intelligent systems.
    \end{itemize}
\end{frame}

\begin{frame}[fragile]
    \frametitle{Conclusion and Future Directions - Potential Future Trends}
    \begin{itemize}
        \item \textbf{Explainable AI (XAI):} Future AI systems will prioritize transparency and accountability, especially in regulated sectors like finance and healthcare.
        \item \textbf{General AI:} Efforts are underway to develop General AI, capable of performing any intellectual task a human can do, necessitating advanced algorithms and cognitive architectures.
        \item \textbf{AI Ethics and Governance:} Ethical considerations regarding bias, privacy, and job displacement will become crucial, leading to structured governance.
        \item \textbf{Interdisciplinary Integration:} AI's integration with fields like neuroscience and psychology will enhance solutions to complex global problems.
    \end{itemize}
\end{frame}

\begin{frame}[fragile]
    \frametitle{Conclusion and Future Directions - Key Takeaways}
    \begin{itemize}
        \item AI is reshaping industries and driven by advancements in deep learning.
        \item Future trends will likely focus on explainable AI, General AI, ethical governance, and cross-disciplinary collaboration.
        \item Continuous education is essential for responsible innovation and application in AI.
    \end{itemize}
\end{frame}


\end{document}