\documentclass[aspectratio=169]{beamer}

% Theme and Color Setup
\usetheme{Madrid}
\usecolortheme{whale}
\useinnertheme{rectangles}
\useoutertheme{miniframes}

% Additional Packages
\usepackage[utf8]{inputenc}
\usepackage[T1]{fontenc}
\usepackage{graphicx}
\usepackage{booktabs}
\usepackage{listings}
\usepackage{amsmath}
\usepackage{amssymb}
\usepackage{xcolor}
\usepackage{tikz}
\usepackage{pgfplots}
\pgfplotsset{compat=1.18}
\usetikzlibrary{positioning}
\usepackage{hyperref}

% Custom Colors
\definecolor{myblue}{RGB}{31, 73, 125}
\definecolor{mygray}{RGB}{100, 100, 100}
\definecolor{mygreen}{RGB}{0, 128, 0}
\definecolor{myorange}{RGB}{230, 126, 34}
\definecolor{mycodebackground}{RGB}{245, 245, 245}

% Set Theme Colors
\setbeamercolor{structure}{fg=myblue}
\setbeamercolor{frametitle}{fg=white, bg=myblue}
\setbeamercolor{title}{fg=myblue}
\setbeamercolor{section in toc}{fg=myblue}
\setbeamercolor{item projected}{fg=white, bg=myblue}
\setbeamercolor{block title}{bg=myblue!20, fg=myblue}
\setbeamercolor{block body}{bg=myblue!10}
\setbeamercolor{alerted text}{fg=myorange}

% Set Fonts
\setbeamerfont{title}{size=\Large, series=\bfseries}
\setbeamerfont{frametitle}{size=\large, series=\bfseries}
\setbeamerfont{caption}{size=\small}
\setbeamerfont{footnote}{size=\tiny}

% Code Listing Style
\lstdefinestyle{customcode}{
  backgroundcolor=\color{mycodebackground},
  basicstyle=\footnotesize\ttfamily,
  breakatwhitespace=false,
  breaklines=true,
  commentstyle=\color{mygreen}\itshape,
  keywordstyle=\color{blue}\bfseries,
  stringstyle=\color{myorange},
  numbers=left,
  numbersep=8pt,
  numberstyle=\tiny\color{mygray},
  frame=single,
  framesep=5pt,
  rulecolor=\color{mygray},
  showspaces=false,
  showstringspaces=false,
  showtabs=false,
  tabsize=2,
  captionpos=b
}
\lstset{style=customcode}

% Custom Commands
\newcommand{\hilight}[1]{\colorbox{myorange!30}{#1}}
\newcommand{\source}[1]{\vspace{0.2cm}\hfill{\tiny\textcolor{mygray}{Source: #1}}}
\newcommand{\concept}[1]{\textcolor{myblue}{\textbf{#1}}}
\newcommand{\separator}{\begin{center}\rule{0.5\linewidth}{0.5pt}\end{center}}

% Title Page Information
\title[Week 9: Bayesian Networks]{Week 9: Bayesian Networks}
\author[J. Smith]{John Smith, Ph.D.}
\institute[University Name]{
  Department of Computer Science\\
  University Name\\
  \vspace{0.3cm}
  Email: email@university.edu\\
  Website: www.university.edu
}
\date{\today}

% Document Start
\begin{document}

\frame{\titlepage}

\begin{frame}[fragile]
    \frametitle{Introduction to Bayesian Networks}
    \begin{block}{What Are Bayesian Networks?}
        Bayesian networks are powerful graphical models that represent the probabilistic relationships among a set of variables. 
        Each variable can affect others, capturing complex dependencies in uncertain environments.
    \end{block}
\end{frame}

\begin{frame}[fragile]
    \frametitle{Key Characteristics of Bayesian Networks}
    \begin{itemize}
        \item \textbf{Directed Acyclic Graph (DAG):} 
        Bayesian networks consist of nodes (variables) connected by directed edges (links), illustrating conditional dependencies without cycles.
        
        \item \textbf{Probabilistic Inference:} 
        They allow us to compute the probabilities of certain outcomes based on prior knowledge.
    \end{itemize}
\end{frame}

\begin{frame}[fragile]
    \frametitle{Relevance in AI}
    \begin{enumerate}
        \item \textbf{Handling Uncertainty:} 
        Many AI applications operate in uncertain environments. Bayesian networks provide a structured way to model uncertainty, making them valuable for decision-making and predictive analytics.
        
        \item \textbf{Applications Across Domains:} 
        \begin{itemize}
            \item \textbf{Medical Diagnosis:} 
            Help model symptoms and diseases to support diagnostics.
            \item \textbf{Machine Learning:} 
            Useful in tasks like classification and regression, enhancing algorithms' capability to reason about uncertain inputs.
            \item \textbf{Natural Language Processing:} 
            Assist in tasks such as sentiment analysis by connecting various linguistic features to understand context.
        \end{itemize}
    \end{enumerate}
\end{frame}

\begin{frame}[fragile]
    \frametitle{Examples of Bayesian Networks}
    \begin{itemize}
        \item \textbf{Medical Diagnosis:} 
        A Bayesian network could link symptoms (e.g., fever, cough) to diseases (e.g., flu, COVID-19), allowing doctors to calculate the probability of a disease given observed symptoms.
        
        \item \textbf{Weather Prediction:} 
        By modeling factors like temperature, humidity, and wind direction, a Bayesian network can provide insights about the likelihood of rain based on current conditions.
    \end{itemize}
\end{frame}

\begin{frame}[fragile]
    \frametitle{Why We Use Bayesian Networks}
    \begin{itemize}
        \item \textbf{Transparency:} 
        Offers a clear visual representation of complex relationships.
        
        \item \textbf{Modularity:} 
        New evidence can be incorporated easily, updating beliefs without a complete overhaul.
        
        \item \textbf{Calculative Efficiency:} 
        The underlying algorithms allow for efficient computation even as the model scales in complexity.
    \end{itemize}
\end{frame}

\begin{frame}[fragile]
    \frametitle{Conclusion}
    Bayesian networks provide a robust framework for dealing with uncertainty in AI, enabling more informed decisions across various fields. 
    In the upcoming slides, we'll dive deeper into the definitions and principles of how these networks function, including their structure and the mathematics behind them.
\end{frame}

\begin{frame}[fragile]
    \frametitle{What is a Bayesian Network?}
    
    \begin{block}{Definition}
        A Bayesian network is a graphical model that represents a set of variables and their probabilistic dependencies. 
        These networks utilize directed acyclic graphs (DAGs) where nodes represent random variables and edges signify the conditional dependencies between them.
    \end{block}
    
    \begin{itemize}
        \item Intuitive visualization of variable relationships
        \item Representation of influence among variables
    \end{itemize}
\end{frame}

\begin{frame}[fragile]
    \frametitle{Graphical Representation and Conditional Probability}

    \begin{block}{Graphical Representation}
        \begin{itemize}
            \item **Nodes**: Represent random variables (discrete or continuous)
            \item **Edges**: Indicate direct dependencies between nodes
        \end{itemize}
    \end{block}
    
    \begin{block}{Conditional Probability}
        \begin{itemize}
            \item Built on conditional probabilities
            \item If A is a parent of B, then the probability of B given A is represented as $P(B|A)$
        \end{itemize}
    \end{block}
\end{frame}

\begin{frame}[fragile]
    \frametitle{Key Features of Bayesian Networks}
    
    \begin{itemize}
        \item **Acyclic**: No cycles in the graph; you cannot return to a node once you leave it.
        \item **Local Independence**: Each node is independent of its non-descendants given its parents, facilitating simpler joint probability computation.
        \item **Inference**: Enables efficient computation of marginal probabilities of certain variables given observations.
    \end{itemize}
\end{frame}

\begin{frame}[fragile]
    \frametitle{Example of a Bayesian Network}

    \begin{block}{Simple Bayesian Network with 3 Variables}
        Consider the variables: Rain, Traffic Jam, and Accident.
    \end{block}
    
    \begin{itemize}
        \item **Nodes**: 
            \begin{itemize}
                \item Rain (R)
                \item Traffic Jam (T)
                \item Accident (A)
            \end{itemize}
        \item **Probabilities**:
            \begin{itemize}
                \item $P(Rain) = 0.1$ (10\% chance of rain)
                \item $P(Traffic Jam | Rain) = 0.8$ (80\% chance of jam if it’s raining)
                \item $P(Traffic Jam | \neg Rain) = 0.2$ (20\% chance of jam if not raining)
                \item $P(Accident | Traffic Jam) = 0.9$ (90\% chance of accident if there is a jam)
            \end{itemize}
    \end{itemize}
\end{frame}

\begin{frame}[fragile]
    \frametitle{Key Points to Emphasize}

    \begin{itemize}
        \item Bayesian networks provide a systematic way to represent and reason about uncertainty.
        \item They are powerful tools in medical diagnosis, risk assessment, and machine learning.
        \item Understanding their structure and dependencies is crucial for effective model building and inference.
    \end{itemize}
    
    \begin{block}{Conclusion}
        By understanding Bayesian networks, students will be equipped to visualize complex probabilistic relationships and make informed decisions based on probabilistic reasoning.
    \end{block}
\end{frame}

\begin{frame}[fragile]
    \frametitle{Components of Bayesian Networks - Overview}
    \begin{block}{Overview of Bayesian Network Components}
        Bayesian networks are structured as graphical models composed of two primary components: \textbf{nodes} and \textbf{edges}. 
    \end{block}
\end{frame}

\begin{frame}[fragile]
    \frametitle{Components of Bayesian Networks - Nodes}
    \begin{enumerate}
        \item \textbf{Nodes}
        \begin{itemize}
            \item \textbf{Definition}: Each node corresponds to a variable or random variable. 
            \item \textbf{Types}:
            \begin{itemize}
                \item \textbf{Discrete Nodes}: Have a finite number of states (e.g., Weather: \{Sunny, Rainy, Cloudy\}).
                \item \textbf{Continuous Nodes}: Can take any value within a range (e.g., Temperature).
            \end{itemize}
        \end{itemize}
        \item \textbf{Example}: In a medical diagnosis network, nodes can represent symptoms (e.g., Coughing, Fever) and diseases (e.g., Flu, Cold).
    \end{enumerate}
\end{frame}

\begin{frame}[fragile]
    \frametitle{Components of Bayesian Networks - Edges and Key Points}
    \begin{enumerate}
        \setcounter{enumi}{1}
        \item \textbf{Edges}
        \begin{itemize}
            \item \textbf{Definition}: Illustrate dependencies between nodes; directed edges imply influence from one variable to another.
            \item \textbf{Characteristics}:
            \begin{itemize}
                \item \textbf{Directed}: Each edge has a direction shown by an arrow.
                \item \textbf{Dependency}: Absence of an edge suggests conditional independence given their parents.
            \end{itemize}
        \end{itemize}
        \item \textbf{Key Points to Emphasize}
        \begin{itemize}
            \item Nodes represent variables while edges represent dependencies.
            \item The structure allows efficient representation of joint probability distributions.
            \item Each edge signifies a direction of influence.
        \end{itemize}
    \end{enumerate}
\end{frame}

\begin{frame}[fragile]
    \frametitle{Illustrative Example and Conclusion}
    \begin{block}{Illustrative Example}
        Consider a simple Bayesian network with three nodes: 
        \begin{itemize}
            \item A: Weather (Sunny, Rainy)
            \item B: Sprinkler (On, Off)
            \item C: Grass Wet (Wet, Dry)
        \end{itemize}
        \textbf{Network Representation}:
        \begin{center}
        \begin{verbatim}
          A (Weather)
           ↓
          B (Sprinkler) 
           ↓
        C (Grass Wet)
        \end{verbatim}
        \end{center}
        In this example: 
        \begin{itemize}
            \item The edge from \textbf{A} to \textbf{B} indicates weather affects the sprinkler status.
            \item The edge from \textbf{B} to \textbf{C} indicates the sprinkler influences grass wetness.
        \end{itemize}
    \end{block}

    \begin{block}{Conclusion}
        Understanding the components of Bayesian networks—nodes and edges—is fundamental for grasping how these networks function. They encapsulate complex relationships and dependencies.
    \end{block}
\end{frame}

\begin{frame}[fragile]
    \frametitle{Probabilities and Conditional Independence}
    \begin{block}{Understanding Conditional Independence}
        Conditional independence is a fundamental concept in probability theory that states:
        \[
        P(X \cap Y | Z) = P(X | Z) \times P(Y | Z)
        \]
        This means knowing \(Z\) makes \(X\) irrelevant to predicting \(Y\) and vice versa.
    \end{block}
\end{frame}

\begin{frame}[fragile]
    \frametitle{Importance in Bayesian Networks}
    \begin{itemize}
        \item \textbf{Structure Representation:} 
        Each node represents a random variable and edges signify dependencies. Conditional independence helps simplify the network structure by removing unnecessary edges.
        
        \item \textbf{Simplifying Calculations:} 
        Conditioned variables allow us to focus on relevant dependencies, reducing complexity in inference tasks.
    \end{itemize}
\end{frame}

\begin{frame}[fragile]
    \frametitle{Examples of Conditional Independence}
    \begin{enumerate}
        \item \textbf{Medical Diagnosis:}
        \begin{itemize}
            \item Let \(A\) be "Having the flu," \(B\) be "Coughing," \(C\) be "Presence of fever."
            \item Knowing \(C\) (fever) makes \(A\) irrelevant to predicting \(B\) (coughing):
            \[
            A \perp\!\!\!\perp B | C
            \]
        \end{itemize}
        
        \item \textbf{Weather and Activity:}
        \begin{itemize}
            \item Let \(W\) be "It’s raining," \(S\) be "People carry umbrellas," \(T\) be "People go jogging."
            \item Knowing \(W\) (raining) makes \(S\) irrelevant to predicting \(T\) (jogging):
            \[
            S \perp\!\!\!\perp T | W
            \]
        \end{itemize}
    \end{enumerate}
\end{frame}

\begin{frame}[fragile]
    \frametitle{Key Points and Conclusion}
    \begin{itemize}
        \item Conditional independence reduces complexity in Bayesian networks.
        \item Essential for separating complex interdependencies, enhancing efficiency in Bayesian reasoning.
        \item Crucial for accurately modeling real-world scenarios.
    \end{itemize}
    \textbf{Next Steps:} In the following slide, we will explore the practical aspects of constructing a Bayesian Network.
\end{frame}

\begin{frame}[fragile]
    \frametitle{Constructing a Bayesian Network - Overview}
    \begin{block}{Overview}
        Constructing a Bayesian Network involves a structured approach to represent a set of variables and their conditional dependencies through a directed acyclic graph (DAG).
        \begin{itemize}
            \item The process ensures clarity and accuracy.
            \item Steps are essential for effective representation.
        \end{itemize}
    \end{block}
\end{frame}

\begin{frame}[fragile]
    \frametitle{Constructing a Bayesian Network - Steps}
    \begin{block}{Steps to Construct a Bayesian Network}
        \begin{enumerate}
            \item Define the Variables
            \item Determine the Relationships
            \item Draw the Directed Acyclic Graph (DAG)
            \item Specify Conditional Probability Distributions (CPDs)
            \item Validate the Model
            \item Refine the Network
        \end{enumerate}
    \end{block}
\end{frame}

\begin{frame}[fragile]
    \frametitle{Constructing a Bayesian Network - Details}
    \begin{itemize}
        \item \textbf{Define the Variables:}
            \begin{itemize}
                \item Identify relevant variables (e.g., Symptom, Disease, Test Result).
            \end{itemize}
        
        \item \textbf{Determine the Relationships:}
            \begin{itemize}
                \item Establish cause-effect relationships (e.g., Disease impacts Symptom).
            \end{itemize}
        
        \item \textbf{Draw the DAG:}
            \begin{itemize}
                \item Nodes: Disease, Symptom, Test Result
                \item Edges: Disease $\rightarrow$ Symptom, Disease $\rightarrow$ Test Result
            \end{itemize}
        
        \item \textbf{Specify CPDs:}
            \begin{itemize}
                \item Define conditional probabilities.
                \item E.g., $P(X | \text{Parent Variables})$ for binary variable \(X\).
            \end{itemize}
        
        \item \textbf{Validate and Refine:}
            \begin{itemize}
                \item Validate the model with real data.
                \item Adjust based on feedback for accuracy.
            \end{itemize}
    \end{itemize}
\end{frame}

\begin{frame}[fragile]
    \frametitle{D-separation and Independence}
    \begin{block}{Understanding D-separation}
        D-separation is a graphical criterion in Bayesian networks that helps determine whether a set of variables (nodes) is conditionally independent from another set of variables given a third set.
    \end{block}
    It aids probabilistic inference, simplifying complex models by identifying which variables can be excluded from consideration when predicting others.
\end{frame}

\begin{frame}[fragile]
    \frametitle{Key Concepts}
    \begin{itemize}
        \item \textbf{Graph Structure}:
            \begin{itemize}
                \item A Bayesian network consists of nodes (random variables) and directed edges (probabilistic dependencies).
                \item The absence of an edge signifies conditional independence, where d-separation is applicable.
            \end{itemize}
        \item \textbf{Paths in Bayesian Networks}:
            \begin{itemize}
                \item A path is a sequence of edges connecting nodes.
                \item We categorize paths based on directionality (head-to-head or tail-to-tail).
            \end{itemize}
        \item \textbf{Conditioning Variables}:
            \begin{itemize}
                \item Two nodes \(A\) and \(B\) are conditionally independent given \(C\) if knowing \(C\) does not provide information about \(A\) and \(B\).
            \end{itemize}
    \end{itemize}
\end{frame}

\begin{frame}[fragile]
    \frametitle{D-Separation Criterion}
    The d-separation rules can be summarized as follows:
    
    \begin{enumerate}
        \item \textbf{Rule 1 (Chain Structure)}:
            \begin{itemize}
                \item If \(A \rightarrow B \rightarrow C\), then \(A\) d-separates \(B\) and \(C\) if conditioned on \(B\).
            \end{itemize}
        \item \textbf{Rule 2 (Fork Structure)}:
            \begin{itemize}
                \item If \(A \leftarrow B \rightarrow C\), then \(B\) d-separates \(A\) and \(C\) if conditioned on itself or its descendants.
            \end{itemize}
        \item \textbf{Rule 3 (Collider Structure)}:
            \begin{itemize}
                \item If \(A \rightarrow B \leftarrow C\), then \(A\) and \(C\) are independent unless \(B\) or its descendants are conditioned on.
            \end{itemize}
    \end{enumerate}
\end{frame}

\begin{frame}[fragile]
    \frametitle{Inference in Bayesian Networks - Overview}
    \begin{block}{Overview}
        Inference in Bayesian Networks involves calculating the posterior probabilities of certain variables given evidence about others. 
        This process enables decision-making and predictions based on the relationship between variables represented in a network.
    \end{block}
\end{frame}

\begin{frame}[fragile]
    \frametitle{Inference in Bayesian Networks - Key Concepts}
    \begin{itemize}
        \item \textbf{Bayesian Networks:}
        \begin{itemize}
            \item Directed acyclic graph (DAG) where nodes represent random variables.
            \item Edges signify probabilistic dependencies.
            \item Each node has a conditional probability table (CPT) quantifying the effect of parent nodes.
        \end{itemize}
        
        \item \textbf{Inference Process:}
        \begin{itemize}
            \item Compute the probability of query variables given known evidence variables.
            \item Employ Bayes' theorem to update probability as evidence becomes available.
        \end{itemize}
    \end{itemize}
    
    \begin{block}{Bayes' Theorem}
        \[
        P(H|E) = \frac{P(E|H) \cdot P(H)}{P(E)}
        \]
    \end{block}
\end{frame}

\begin{frame}[fragile]
    \frametitle{Inference in Bayesian Networks - Probabilistic Query Process}
    \begin{itemize}
        \item \textbf{Probabilistic Query Process:}
        \begin{enumerate}
            \item \textbf{Finding Marginal Probabilities:}
            \begin{itemize}
                \item Compute the probability distribution for a subset of variables.
                \item Example: 
                \[
                P(X) = \sum_{Y} P(X, Y)
                \]
            \end{itemize}
            
            \item \textbf{Conditional Probabilities:}
            \begin{itemize}
                \item Obtain the probability of a variable given certain evidence.
                \item Example: 
                \[
                P(X|E) = \frac{P(E|X) \cdot P(X)}{P(E)}
                \]
            \end{itemize}
        \end{enumerate}
    \end{itemize}
\end{frame}

\begin{frame}[fragile]
    \frametitle{Inference in Bayesian Networks - Example}
    \begin{block}{Example}
        Consider a simple Bayesian network containing three variables:
        \begin{itemize}
            \item Rain (R)
            \item Sprinkler (S)
            \item Wet Grass (W)
        \end{itemize}
        
        \textbf{CPTs:}
        \begin{itemize}
            \item \( P(R) = 0.2 \)
            \item \( P(S|R) \):
            \begin{itemize}
                \item \( P(S|R=true) = 0.01 \)
                \item \( P(S|R=false) = 0.4 \)
            \end{itemize}
            \item \( P(W|S,R) \):
            \begin{itemize}
                \item \( P(W|S=true,R=true) = 0.99 \)
                \item \( P(W|S=true,R=false) = 0.9 \)
                \item \( P(W|S=false,R=true) = 0.8 \)
                \item \( P(W|S=false,R=false) = 0.1 \)
            \end{itemize}
        \end{itemize}
        
        \textbf{Query Example:} If \( W=true \) (grass is wet), compute \( P(R=true|W=true) \) using Bayes' theorem and the law of total probability.
    \end{block}
\end{frame}

\begin{frame}[fragile]
    \frametitle{Inference in Bayesian Networks - Key Takeaways}
    \begin{itemize}
        \item Inference supports reasoning under uncertainty, combining prior knowledge with observed evidence.
        \item Understanding the structure and dependencies in a Bayesian network is crucial for accurate inference.
        \item Different inference techniques (exact and approximate) may be necessary based on the network's complexity.
    \end{itemize}
    
    \begin{block}{Applications}
        Bayesian Inference is useful in varied applications, from risk assessment to decision-making in uncertain environments.
    \end{block}
\end{frame}

\begin{frame}[fragile]
    \frametitle{Exact Inference Algorithms}
    \begin{block}{Overview}
        Exact inference algorithms are techniques used to compute the exact posterior probabilities in Bayesian networks.
        \begin{itemize}
            \item Variable Elimination
            \item Junction Tree algorithms
        \end{itemize}
    \end{block}
\end{frame}

\begin{frame}[fragile]
    \frametitle{Variable Elimination}
    \begin{block}{Description}
        Variable Elimination is a systematic way of summing out variables from a joint probability distribution to compute marginal probabilities.
    \end{block}

    \begin{block}{Key Steps}
        \begin{enumerate}
            \item Factorization: Start with the joint probability distribution of the variables.
            \item Elimination order: Choose an order to remove variables.
            \item Marginalization: For each variable to be eliminated, sum over its possible values.
        \end{enumerate}
    \end{block}

    \begin{block}{Example}
        To compute \( P(A | B = \text{true}) \):
        \begin{itemize}
            \item Start with joint distribution: \( P(A, B, C) \)
            \item Eliminate \( C \): \( P(A, B) = \sum_C P(A, B, C) \)
            \item Normalize to find \( P(A | B = \text{true}) \)
        \end{itemize}
    \end{block}

    \begin{block}{Mathematical Notation}
        \begin{equation}
            P(X) = \sum_{Y \in \text{Neighbors}(X)} P(X, Y)
        \end{equation}
    \end{block}
\end{frame}

\begin{frame}[fragile]
    \frametitle{Junction Tree Algorithm}
    \begin{block}{Description}
        The Junction Tree algorithm transforms a Bayesian network into a tree structure to facilitate efficient inference.
    \end{block}

    \begin{block}{Key Steps}
        \begin{enumerate}
            \item Moralization: Convert directed graph into undirected.
            \item Triangulation: Ensure no cycles of four or more nodes.
            \item Construct Junction Tree: Create clusters of nodes sharing common variables.
            \item Message Passing: Perform belief propagation to compute probabilities.
        \end{enumerate}
    \end{block}

    \begin{block}{Example}
        For moral graph of variables \( X, Y, Z \):
        \begin{itemize}
            \item Create clusters: {X, Y}, {Y, Z}.
            \item Pass messages until beliefs converge.
        \end{itemize}
    \end{block}

    \begin{block}{Key Formula}
        \begin{equation}
            P(C) = \frac{\prod_{i} P(C_i)}{P(\text{separator})}
        \end{equation}
    \end{block}
\end{frame}

\begin{frame}[fragile]
    \frametitle{Key Points}
    \begin{itemize}
        \item \textbf{Efficiency:} Variable Elimination is straightforward but computationally intensive for high-dimensional datasets.
        \item \textbf{Scalability:} Junction Tree is more efficient for larger networks using message passing.
        \item \textbf{Exact Results:} Both methods provide exact probabilities with different approaches.
    \end{itemize}
\end{frame}

\begin{frame}[fragile]
    \frametitle{Summary}
    In summary, exact inference techniques like Variable Elimination and Junction Tree algorithms are essential tools for reasoning in Bayesian networks, allowing for precise computation of probabilities. Understanding these methods is crucial for effectively applying Bayesian models in various applications.
\end{frame}

\begin{frame}[fragile]
    \frametitle{Next Slide Preview}
    \begin{block}{Approximate Inference Algorithms}
        Learn about methods to compute approximate solutions when exact methods are too computationally expensive, including:
        \begin{itemize}
            \item Markov Chain Monte Carlo (MCMC)
            \item Variational Inference techniques
        \end{itemize}
    \end{block}
\end{frame}

\begin{frame}[fragile]
    \frametitle{Approximate Inference Algorithms}
    
    \begin{block}{Introduction to Approximate Inference Methods}
        Approximate inference algorithms are crucial for making predictions and decisions using Bayesian networks when exact inference becomes computationally infeasible.
        
        We'll explore two prominent methods:
        \begin{itemize}
            \item Markov Chain Monte Carlo (MCMC)
            \item Variational Inference
        \end{itemize}
    \end{block}
\end{frame}

\begin{frame}[fragile]
    \frametitle{Markov Chain Monte Carlo (MCMC)}
    
    \begin{block}{Concept}
        MCMC involves constructing a Markov chain with a desired distribution as its equilibrium distribution. Sampling from this chain helps approximate properties of the target distribution.
    \end{block}
    
    \begin{block}{Key Steps}
        \begin{enumerate}
            \item \textbf{Initialize}: Start with an arbitrary point in the state space.
            \item \textbf{Transition}: Use a proposal distribution to randomly move to a new state.
            \item \textbf{Acceptance}: Accept or reject the new state based on an acceptance criterion (e.g., Metropolis-Hastings).
        \end{enumerate}
    \end{block}
\end{frame}

\begin{frame}[fragile]
    \frametitle{Markov Chain Monte Carlo (MCMC) - Example and Formula}
    
    \begin{block}{Example}
        To estimate the integral of a function \( f(x) \) over a region, MCMC allows sampling to generate a sequence of samples from the target distribution. The mean of these samples provides an estimate of our integral.
    \end{block}
    
    \begin{block}{Key Formula}
        The acceptance probability \( \alpha \) is defined as:
        \begin{equation}
            \alpha = \min\left(1, \frac{p(x') \cdot q(x|x')}{p(x) \cdot q(x'|x)}\right)
        \end{equation}
        where \( p(x) \) is the target distribution, and \( q(x'|x) \) is the proposal distribution.
    \end{block}
\end{frame}

\begin{frame}[fragile]
    \frametitle{Variational Inference}
    
    \begin{block}{Concept}
        Variational inference approximates complex posterior distributions with simpler, tractable distributions by optimizing parameters in a lower-dimensional space.
    \end{block}
    
    \begin{block}{Key Steps}
        \begin{enumerate}
            \item \textbf{Choose} a family of distributions \( Q \) that is easier to work with.
            \item \textbf{Optimize}: Maximize the Evidence Lower Bound (ELBO) defined as:
            \begin{equation}
                \text{ELBO} = \mathbb{E}_{q(z)}[\log p(x, z)] - \mathbb{E}_{q(z)}[\log q(z)]
            \end{equation}
            where \( z \) represents latent variables.
        \end{enumerate}
    \end{block}
\end{frame}

\begin{frame}[fragile]
    \frametitle{Variational Inference - Example}
    
    \begin{block}{Example}
        In a classification problem, instead of directly inferring the posterior distribution of class labels given features, one could use a simpler Gaussian distribution. Variational inference finds the parameters of this Gaussian that best approximate the true posterior.
    \end{block}
    
    \begin{block}{Key Points to Emphasize}
        \begin{itemize}
            \item Approximate inference methods are crucial for large-scale Bayesian networks.
            \item MCMC is stochastic, providing samples to estimate distributions.
            \item Variational Inference is deterministic, providing an optimization approach.
            \item Selection between these techniques depends on specific applications and computational resources.
        \end{itemize}
    \end{block}
\end{frame}

\begin{frame}[fragile]
    \frametitle{Conclusion}
    
    \begin{block}{Conclusion}
        Understanding MCMC and Variational Inference allows for efficient handling of probabilities and predictions in Bayesian networks, even when dealing with large datasets. Next, we will explore practical applications of Bayesian networks and how these inference methods are applied in real-world scenarios.
    \end{block}
\end{frame}

\begin{frame}[fragile]
    \frametitle{Applications of Bayesian Networks}
    \begin{itemize}
        \item Bayesian Networks (BNs) are powerful probabilistic graphical models.
        \item They represent variables and their conditional dependencies using directed acyclic graphs (DAGs).
        \item Key applications include:
        \begin{itemize}
            \item Medical Diagnosis
            \item Forecasting
            \item Decision Making
            \item Risk Assessment
        \end{itemize}
    \end{itemize}
\end{frame}

\begin{frame}[fragile]
    \frametitle{1. Medical Diagnosis}
    \begin{itemize}
        \item \textbf{Concept:} Model probabilities of diseases based on symptoms and patient data.
        \item \textbf{Example:} Determine likelihood of flu, pneumonia, or COVID-19 from symptoms like fever and cough.
        \item \textbf{Key Point:} BNs integrate expert knowledge and uncertain information in diagnostics.
    \end{itemize}
\end{frame}

\begin{frame}[fragile]
    \frametitle{2. Forecasting}
    \begin{itemize}
        \item \textbf{Concept:} Integrate data sources for predictions about future events.
        \item \textbf{Example:} Predict stock market trends by analyzing economic indicators and historical data.
        \item \textbf{Key Point:} Improve accuracy of forecasts by considering probabilistic relationships and updating with new information.
    \end{itemize}
\end{frame}

\begin{frame}[fragile]
    \frametitle{3. Decision Making}
    \begin{itemize}
        \item \textbf{Concept:} Evaluate probabilities of outcomes based on decisions under uncertainty.
        \item \textbf{Example:} Analyze best product launch strategies considering market trends and customer preferences.
        \item \textbf{Key Point:} Help decision-makers maximize expected utility or minimize risks.
    \end{itemize}
\end{frame}

\begin{frame}[fragile]
    \frametitle{4. Risk Assessment}
    \begin{itemize}
        \item \textbf{Concept:} Assess risks in various fields like cybersecurity and environmental science.
        \item \textbf{Example:} Model flood risk based on rainfall levels, soil saturation, and topography.
        \item \textbf{Key Point:} Analyze how different factors contribute to overall risk for comprehensive evaluation.
    \end{itemize}
\end{frame}

\begin{frame}[fragile]
    \frametitle{Summary and Formula Highlight}
    \begin{itemize}
        \item BNs are versatile AI tools used in practical applications like medical diagnosis, forecasting, decision making, and risk assessment.
        \item They model complex relationships under uncertainty, combining qualitative and quantitative data.
    \end{itemize}
    
    \begin{block}{Bayes' Theorem}
        \begin{equation}
        P(A | B) = \frac{P(B | A) \cdot P(A)}{P(B)}
        \end{equation}
        Where:
        \begin{itemize}
            \item \( P(A | B) \): Posterior probability of event A given B.
            \item \( P(B | A) \): Likelihood of event B given A.
            \item \( P(A) \): Prior probability of event A.
            \item \( P(B) \): Marginal probability of event B.
        \end{itemize}
    \end{block}
\end{frame}

\begin{frame}[fragile]
    \frametitle{Challenges in Using Bayesian Networks - Overview}
    Bayesian Networks (BNs) are powerful tools for probabilistic reasoning and decision-making in uncertain environments. However, several challenges and limitations hinder their effective construction and utilization. This slide discusses these challenges to enhance your understanding and prepare you for practical applications.
\end{frame}

\begin{frame}[fragile]
    \frametitle{Challenges in Using Bayesian Networks - Complexity of Structure}
    \begin{itemize}
        \item \textbf{Challenge}: Designing the structure of a Bayesian Network can be inherently complex, especially for systems with many variables and interdependencies.
        \item \textbf{Example}: In a medical diagnosis scenario, the relationships between diseases and symptoms can be intricate, requiring careful analysis to represent them accurately.
    \end{itemize}
\end{frame}

\begin{frame}[fragile]
    \frametitle{Challenges in Using Bayesian Networks - Data Limitations}
    \begin{itemize}
        \item \textbf{Challenge}: BNs require accurate probability distributions. Incomplete or biased data can lead to incorrect inferences.
        \item \textbf{Example}: A BN based on insufficient historical medical records may misrepresent the probabilities of certain conditions, impacting diagnosis accuracy.
    \end{itemize}
\end{frame}

\begin{frame}[fragile]
    \frametitle{Challenges in Using Bayesian Networks - Computational Complexity}
    \begin{itemize}
        \item \textbf{Challenge}: Inference in Bayesian Networks can be computationally intensive, particularly as the number of variables increases.
        \item \textbf{Key Point}: Exact inference methods exhibit exponential time complexity. Approximations or sampling methods (e.g., MCMC) may be necessary but can compromise accuracy.
    \end{itemize}
\end{frame}

\begin{frame}[fragile]
    \frametitle{Challenges in Using Bayesian Networks - Parameter Estimation}
    \begin{itemize}
        \item \textbf{Challenge}: Estimating parameters (conditional probabilities) requires expert knowledge or extensive data, making it difficult in practice.
        \item \textbf{Method}: Techniques like Maximum Likelihood Estimation or Bayesian Estimation can be employed but may require strong assumptions about the data.
    \end{itemize}
\end{frame}

\begin{frame}[fragile]
    \frametitle{Challenges in Using Bayesian Networks - Difficulty in Updating Models}
    \begin{itemize}
        \item \textbf{Challenge}: Incorporating new evidence into an established Bayesian Network can be challenging and may require re-evaluation of model structure and parameters.
        \item \textbf{Example}: In dynamic environments (e.g., stock market predictions), frequent updates are necessary, making model maintenance complex.
    \end{itemize}
\end{frame}

\begin{frame}[fragile]
    \frametitle{Challenges in Using Bayesian Networks - Interpretability}
    \begin{itemize}
        \item \textbf{Challenge}: While BNs make probabilistic relationships explicit, the complexity of large networks can hinder interpretability.
        \item \textbf{Key Point}: Stakeholders who are not statistically trained may find it challenging to understand the implications of the network’s probabilities.
    \end{itemize}
\end{frame}

\begin{frame}[fragile]
    \frametitle{Challenges in Using Bayesian Networks - Formulas \& Techniques}
    \begin{block}{Bayes' Theorem}
        A fundamental principle used for updating probability estimates:
        \begin{equation}
            P(A|B) = \frac{P(B|A) \cdot P(A)}{P(B)}
        \end{equation}
    \end{block}
\end{frame}

\begin{frame}[fragile]
    \frametitle{Challenges in Using Bayesian Networks - Conclusion}
    Awareness of these challenges is essential for effectively constructing and using Bayesian Networks. Overcoming these hurdles requires a blend of statistical knowledge, domain expertise, and computational resources. Future sections will explore tools and libraries that can aid in addressing some of these challenges.
\end{frame}

\begin{frame}
    \frametitle{Tools and Libraries for Bayesian Networks}
    \begin{block}{Introduction}
        Bayesian Networks (BNs) are graphical models that represent dependencies among variables using directed acyclic graphs (DAGs). 
        Several Python libraries are available for creating, manipulating, and analyzing these networks, each with unique features.
    \end{block}
\end{frame}

\begin{frame}[fragile]
    \frametitle{Key Libraries}
    \begin{enumerate}
        \item \textbf{pgmpy}
            \begin{itemize}
                \item \textbf{Overview:} A popular library for Probabilistic Graphical Models.
                \item \textbf{Features:}
                    \begin{itemize}
                        \item Supports inference using algorithms like Variable Elimination and Belief Propagation.
                        \item Handles both discrete and continuous random variables.
                    \end{itemize}
                \item \textbf{Example Code:}
                \begin{lstlisting}[language=Python]
from pgmpy.models import BayesianModel
from pgmpy.inference import VariableElimination

# Defining a simple Bayesian Network
model = BayesianModel([('Rain', 'Traffic'), ('Accident', 'Traffic')])
model.add_cpds(cpd_rain, cpd_traffic, cpd_accident)

# Inference
infer = VariableElimination(model)
result = infer.query(variables=['Traffic'], evidence={'Rain': 1})
print(result)
                \end{lstlisting}
            \end{itemize}
        
        \item \textbf{bnlearn}
            \begin{itemize}
                \item \textbf{Overview:} A user-friendly library for learning the structure from data.
                \item \textbf{Features:}
                    \begin{itemize}
                        \item Algorithms like PC, Hill Climbing, and Tabu Search for structure learning.
                        \item Easy integration with pandas.
                    \end{itemize}
                \item \textbf{Example Code:}
                \begin{lstlisting}[language=Python]
import bnlearn as bn

# Load data
df = bn.make_Euclidean_table()
# Learn the structure
model = bn.bnlearn(df)
                \end{lstlisting}
            \end{itemize}
    \end{enumerate}
\end{frame}

\begin{frame}[fragile]
    \frametitle{Additional Libraries}
    \begin{enumerate}
        \setcounter{enumi}{2} % Continue the enumeration
        \item \textbf{TensorFlow Probability (TFP)}
            \begin{itemize}
                \item \textbf{Overview:} Extends TensorFlow to support probabilistic reasoning.
                \item \textbf{Features:}
                    \begin{itemize}
                        \item Enables creation of complex distributions and scalable Bayesian inference.
                    \end{itemize}
                \item \textbf{Example Code:}
                \begin{lstlisting}[language=Python]
import tensorflow_probability as tfp

# Define a Bayesian model
model = tfp.distributions.Beta()
# Perform inference
samples = model.sample(1000)
                \end{lstlisting}
            \end{itemize}
    \end{enumerate}
\end{frame}

\begin{frame}
    \frametitle{Key Points and Conclusion}
    \begin{itemize}
        \item \textbf{Flexibility:} Libraries allow complex model definitions and inference based on data.
        \item \textbf{Community Support:} Active communities provide extensive documentation for learning.
        \item \textbf{Applicability:} Useful in various fields like healthcare, finance, and AI for decision-making.
    \end{itemize}
    \begin{block}{Conclusion}
        Familiarizing yourself with these tools enhances the ability to model uncertainties and make informed decisions based on probabilistic outcomes. 
    \end{block}
\end{frame}

\begin{frame}[fragile]
    \frametitle{Case Study: Real-World Example}
    \begin{block}{Introduction to Bayesian Networks}
        Bayesian Networks (BNs) are graphical models that encode probabilistic relationships among variables. They allow reasoning under uncertainty, making them highly useful for decision-making processes in various fields.
    \end{block}
\end{frame}

\begin{frame}[fragile]
    \frametitle{Real-World Example: Medical Diagnosis}
    \begin{itemize}
        \item \textbf{Case Study Overview}
        \begin{itemize}
            \item \textbf{Problem:} Early diagnosis of diseases based on patient symptoms and test results.
            \item \textbf{Objective:} Increase prediction accuracy and provide decision support for clinicians.
        \end{itemize}
        
        \item \textbf{Key Variables}
        \begin{itemize}
            \item Symptoms: Fever, Cough, Fatigue
            \item Diseases: Influenza, COVID-19, Common Cold
            \item Diagnostic Tests: PCR Test, Rapid Antigen Test
        \end{itemize}
        
        \item \textbf{Structure of the Bayesian Network}
        \begin{itemize}
            \item Nodes represent Symptoms, Diseases, and Diagnostic Tests.
            \item Arrows indicate dependencies; e.g., symptoms influenced by specific diseases.
        \end{itemize}
    \end{itemize}
\end{frame}

\begin{frame}[fragile]
    \frametitle{Inference with Bayesian Networks}
    \begin{block}{Example of the Network}
        \begin{itemize}
            \item \textbf{Graph Structure}
            \begin{itemize}
                \item Diseases → Symptoms (e.g., Influenza causes Fever and Cough)
                \item Tests → Diseases (e.g., PCR Test gives a higher probability of diagnosing COVID-19)
            \end{itemize}
            
            \item \textbf{Probabilities}
            \begin{itemize}
                \item \( P(Fever | Influenza) = 0.9 \)
                \item \( P(Cough | Common Cold) = 0.75 \)
                \item \( P(Positive\, PCR | COVID-19) = 0.95 \)
            \end{itemize}
        \end{itemize}
        
        \begin{block}{Inference Process}
            Using the Bayesian Network, clinicians can input observed symptoms to infer the probabilities of various diseases:
            \begin{itemize}
                \item \textbf{Input:} Patient has Fever and Cough.
                \item \textbf{Query:} What is the probability of COVID-19?
                \item \textbf{Calculation:} Use Bayes' Rule:
                \begin{equation}
                    P(Disease | Symptoms) = \frac{P(Symptoms | Disease) \cdot P(Disease)}{P(Symptoms)}
                \end{equation}
            \end{itemize}
        \end{block}
        
        \begin{block}{Advantages}
            \begin{itemize}
                \item Handles Uncertainty
                \item Dynamic Updating
                \item Interpretable Results
            \end{itemize}
        \end{block}
    \end{block}
\end{frame}

\begin{frame}[fragile]
    \frametitle{Best Practices for Building Bayesian Networks - Introduction}
    \begin{block}{Introduction to Bayesian Networks}
        Bayesian Networks (BNs) are graphical models that represent a set of variables and their conditional dependencies using a directed acyclic graph (DAG). They transform complex probabilistic reasoning into a structure that is manageable. Building an effective Bayesian Network is essential for accurate inference and analysis.
    \end{block}
\end{frame}

\begin{frame}[fragile]
    \frametitle{Best Practices for Building Bayesian Networks - Part 1}
    \begin{enumerate}
        \item \textbf{Define the Scope and Purpose}
        \begin{itemize}
            \item Clarify Objectives: Determine what questions the BN should answer or what decisions it should support.
            \item Identify Key Variables: Focus on variables that directly affect the outcomes of interest.
            \item \textit{Example}: In a medical diagnosis BN, prioritize symptoms and diseases that are most relevant.
        \end{itemize}
        
        \item \textbf{Collect and Analyze Data}
        \begin{itemize}
            \item Quality Data Collection: Ensure that data used for constructing the network is high-quality, relevant, and sufficient.
            \item Statistical Analysis: Use statistical methods to estimate the probabilities and dependencies among variables.
            \item \textit{Example}: For weather prediction, analyze historical data on temperature, humidity, and weather conditions.
        \end{itemize}
    \end{enumerate}
\end{frame}

\begin{frame}[fragile]
    \frametitle{Best Practices for Building Bayesian Networks - Part 2}
    \begin{enumerate}[start=3]
        \item \textbf{Design the Structure}
        \begin{itemize}
            \item Node and Edge Definition: Define nodes (variables) and directed edges (relationships) based on domain knowledge and data analysis.
            \item Avoid Over-Complexity: Keep the model as simple as possible to facilitate understanding and computational efficiency.
            \item \textit{Illustration}: Map out relationships among variables; e.g., Rain $\rightarrow$ Wet Grass.
        \end{itemize}
        
        \item \textbf{Parameterization}
        \begin{itemize}
            \item Conditional Probability Tables (CPTs): Define CPTs for each node that represent the probability of each outcome given its parents.
            \item Consistency Checks: Verify that the probabilities in the CPTs sum to 1.
            \item \textit{Example}: A CPT for a 'Flu' node may look like:
            \begin{itemize}
                \item If ‘Weather = Cold’, P(Flu=True)=0.8, P(Flu=False)=0.2.
            \end{itemize}
        \end{itemize}
    \end{enumerate}
\end{frame}

\begin{frame}[fragile]
    \frametitle{Best Practices for Building Bayesian Networks - Part 3}
    \begin{enumerate}[start=5]
        \item \textbf{Validation and Testing}
        \begin{itemize}
            \item Model Testing: Validate the model by comparing predictions against real-world data.
            \item Sensitivity Analysis: Assess how changes in one variable affect predictions to identify critical variables.
            \item \textit{Example}: Check how altering symptom probabilities affects disease predictions.
        \end{itemize}
        
        \item \textbf{Iterative Refinement}
        \begin{itemize}
            \item Feedback Loop: Revise the model based on performance, emerging data, and domain feedback.
            \item Continuous Learning: As more data is collected, update the network for improved accuracy over time.
        \end{itemize}
    \end{enumerate}
\end{frame}

\begin{frame}[fragile]
    \frametitle{Best Practices for Building Bayesian Networks - Conclusion}
    \begin{enumerate}[start=7]
        \item \textbf{User-Friendly Interface}
        \begin{itemize}
            \item Visual Representation: Use clear diagrams to represent the network for easier interpretation by non-experts.
            \item Guided Decision-Making: Provide tools to assist users in querying the network and interpreting results.
        \end{itemize}
        
        \item \textbf{Documentation and Communication}
        \begin{itemize}
            \item Document Assumptions: Clearly record assumptions made during model construction for transparency.
            \item Effective Communication: Present findings and uncertainties clearly to stakeholders.
        \end{itemize}
    \end{enumerate}
    \begin{block}{Key Takeaways}
        \begin{itemize}
            \item Start with a clear purpose and relevant data.
            \item Carefully design the structure and parameterization of the BN.
            \item Validate and iterate to enhance the model’s reliability.
            \item Communicate findings effectively to ensure the model's utility in decision-making processes.
        \end{itemize}
    \end{block}
\end{frame}

\begin{frame}[fragile]
    \frametitle{Future of Bayesian Networks in AI - Overview}
    \begin{block}{Overview of Bayesian Networks}
        Bayesian Networks (BNs) are graphical models that represent a set of variables and their probabilistic dependencies. 
        They are based on Bayes' theorem and are used for reasoning under uncertainty. 
        As AI evolves, the application and innovation in BNs present numerous opportunities and challenges.
    \end{block}
\end{frame}

\begin{frame}[fragile]
    \frametitle{Future of Bayesian Networks in AI - Key Areas of Advancement}
    \begin{itemize}
        \item \textbf{Integration with Deep Learning}
            \begin{itemize}
                \item \textit{Concept}: Combining BNs with deep learning models enhances interpretability while maintaining accuracy.
                \item \textit{Example}: Using BNs to model uncertainties in data inputs for a neural network, providing a way to assess the trustworthiness of predictions.
            \end{itemize}
        \item \textbf{Scalable Algorithms}
            \begin{itemize}
                \item \textit{Concept}: Development of new algorithms for large-scale BNs that handle high-dimensional data efficiently.
                \item \textit{Example}: Utilizing variational inference techniques to allow real-time data processing in dynamic environments.
            \end{itemize}
        \item \textbf{Advancements in Inference Techniques}
            \begin{itemize}
                \item \textit{Concept}: Improved algorithms for probabilistic inference that provide faster results.
                \item \textit{Example}: Using message-passing techniques to optimize computations in networks with a large number of variables.
            \end{itemize}
    \end{itemize}
\end{frame}

\begin{frame}[fragile]
    \frametitle{Future of Bayesian Networks in AI - Applications and Challenges}
    \begin{block}{Applications in Emerging Fields}
        \begin{enumerate}
            \item \textbf{Healthcare}
                \begin{itemize}
                    \item \textit{Usage}: BNs can be used for personalized medicine by incorporating diverse data points (genomics, lifestyle).
                    \item \textit{Impact}: Enhanced decision-making in clinical practices and predictive analytics for patient outcomes.
                \end{itemize}
            \item \textbf{Autonomous Systems}
                \begin{itemize}
                    \item \textit{Usage}: Integration in robotics for decision-making under uncertainty.
                    \item \textit{Impact}: Improved navigation and obstacle avoidance systems in autonomous vehicles.
                \end{itemize}
            \item \textbf{Cybersecurity}
                \begin{itemize}
                    \item \textit{Usage}: Real-time threat assessment and risk management using BNs.
                    \item \textit{Impact}: More effective identification of vulnerabilities and proactive defense strategies.
                \end{itemize}
        \end{enumerate}
    \end{block}
    
    \begin{block}{Challenges and Considerations}
        \begin{itemize}
            \item Complexity of models: Trade-offs between model complexity and interpretability.
            \item Data Scarcity: Difficulty in acquiring sufficient data in many domains.
            \item Ethical Considerations: Privacy and bias issues in decision-making processes.
        \end{itemize}
    \end{block}
\end{frame}

\begin{frame}[fragile]
    \frametitle{Future of Bayesian Networks in AI - Conclusion}
    \begin{block}{Conclusion}
        The future of Bayesian Networks in AI is promising, with ongoing research and advancements that bridge traditional probabilistic inference with modern machine learning techniques. 
        These developments not only enrich the AI landscape but also create more reliable and interpretable systems, capable of solving complex real-world problems.
    \end{block}
    
    \begin{block}{Key Takeaway}
        Bayes' theorem continues to play a vital role in the evolution of AI, offering insights into the relationships between uncertainty and decision-making.
    \end{block}
\end{frame}

\begin{frame}[fragile]
    \frametitle{Conclusion and Key Takeaways - Overview}
    \begin{itemize}
        \item Summary of key points discussed in the chapter.
        \item Implications for AI applications.
    \end{itemize}
\end{frame}

\begin{frame}[fragile]
    \frametitle{Conclusion and Key Takeaways - Bayesian Networks}
    \begin{enumerate}
        \item \textbf{Overview of Bayesian Networks}
        \begin{itemize}
            \item Definition: Probabilistic graphical models representing variables and their dependencies via a directed acyclic graph (DAG).
            \item Components:
            \begin{itemize}
                \item \textbf{Nodes}: Represent random variables.
                \item \textbf{Edges}: Indicate dependencies between variables.
            \end{itemize}
        \end{itemize}
        
        \item \textbf{Importance in AI}
        \begin{itemize}
            \item Interpretable models allow for transparency.
            \item Effective in uncertain environments (e.g., medical diagnosis).
        \end{itemize}
    \end{enumerate}
\end{frame}

\begin{frame}[fragile]
    \frametitle{Key Concepts and Applications}
    \begin{enumerate}
        \setcounter{enumi}{2}
        \item \textbf{Key Concepts Reviewed}
        
        \begin{itemize}
            \item \textbf{Conditional Probability}
            \item \textbf{Bayes' Theorem}:
            \begin{equation}
            P(H|E) = \frac{P(E|H) \cdot P(H)}{P(E)}
            \end{equation}
            \begin{itemize}
                \item Posterior probability: $P(H|E)$
                \item Likelihood: $P(E|H)$
                \item Prior probability: $P(H)$
                \item Marginal likelihood: $P(E)$
            \end{itemize}
            \item \textbf{Inference Algorithms}: Belief propagation and variable elimination.
        \end{itemize}
        
        \item \textbf{Real-World Applications}
        \begin{itemize}
            \item Healthcare: Diagnosis support.
            \item Finance: Credit scoring and risk management.
            \item Robotics: Navigation in uncertain environments.
        \end{itemize}
    \end{enumerate}
\end{frame}

\begin{frame}[fragile]
    \frametitle{Future Implications}
    \begin{itemize}
        \item Integration with machine learning for enhanced prediction accuracy.
        \item Advancements leading to more robust algorithms.
    \end{itemize}

    \begin{block}{Key Points to Remember}
        \begin{itemize}
            \item Bayesian networks provide clear reasoning under uncertainty.
            \item Their ability to integrate new evidence efficiently makes them essential.
            \item Utilizing Bayesian networks can lead to smarter AI systems.
        \end{itemize}
    \end{block}
\end{frame}


\end{document}