\documentclass[aspectratio=169]{beamer}

% Theme and Color Setup
\usetheme{Madrid}
\usecolortheme{whale}
\useinnertheme{rectangles}
\useoutertheme{miniframes}

% Additional Packages
\usepackage[utf8]{inputenc}
\usepackage[T1]{fontenc}
\usepackage{graphicx}
\usepackage{booktabs}
\usepackage{listings}
\usepackage{amsmath}
\usepackage{amssymb}
\usepackage{xcolor}
\usepackage{tikz}
\usepackage{pgfplots}
\pgfplotsset{compat=1.18}
\usetikzlibrary{positioning}
\usepackage{hyperref}

% Custom Colors
\definecolor{myblue}{RGB}{31, 73, 125}
\definecolor{mygray}{RGB}{100, 100, 100}
\definecolor{mygreen}{RGB}{0, 128, 0}
\definecolor{myorange}{RGB}{230, 126, 34}
\definecolor{mycodebackground}{RGB}{245, 245, 245}

% Set Theme Colors
\setbeamercolor{structure}{fg=myblue}
\setbeamercolor{frametitle}{fg=white, bg=myblue}
\setbeamercolor{title}{fg=myblue}
\setbeamercolor{section in toc}{fg=myblue}
\setbeamercolor{item projected}{fg=white, bg=myblue}
\setbeamercolor{block title}{bg=myblue!20, fg=myblue}
\setbeamercolor{block body}{bg=myblue!10}
\setbeamercolor{alerted text}{fg=myorange}

% Set Fonts
\setbeamerfont{title}{size=\Large, series=\bfseries}
\setbeamerfont{frametitle}{size=\large, series=\bfseries}
\setbeamerfont{caption}{size=\small}
\setbeamerfont{footnote}{size=\tiny}

% Document Start
\begin{document}

\frame{\titlepage}

\begin{frame}[fragile]
    \frametitle{Introduction to Multi-Agent Search and Game Playing}
    % Overview of Multi-Agent Systems
    \begin{block}{Overview of Multi-Agent Systems}
        A multi-agent system (MAS) involves multiple interacting intelligent agents that can be software or physical robots capable of autonomous actions.
    \end{block}
    
    \begin{block}{Importance in Artificial Intelligence (AI)}
        \begin{itemize}
            \item \textbf{Complex Problem Solving:} MAS collaboratively tackle complex problems that are challenging for a single agent.
            \item \textbf{Diverse Applications:}
            \begin{itemize}
                \item Robotics (e.g., search and rescue)
                \item Traffic Management (e.g., optimizing vehicle flow)
                \item Gaming (e.g., chess, video games)
            \end{itemize}
        \end{itemize}
    \end{block}
\end{frame}

\begin{frame}[fragile]
    \frametitle{Key Concepts}
    % Key concepts in multi-agent systems
    \begin{block}{Agent}
        An agent is an entity that perceives its environment and acts upon it. Agents can be:
        \begin{itemize}
            \item \textbf{Reactive:} Respond to changes in the environment.
            \item \textbf{Proactive:} Take initiative to achieve goals.
        \end{itemize}
    \end{block}
    
    \begin{block}{Collaboration vs. Competition}
        \begin{itemize}
            \item \textbf{Collaborative agents:} Work together towards a common goal (e.g., team robots).
            \item \textbf{Competitive agents:} Compete against each other for individual goals (e.g., game players).
        \end{itemize}
    \end{block}
\end{frame}

\begin{frame}[fragile]
    \frametitle{Examples and Key Takeaways}
    % Examples of Multi-Agent Systems
    \begin{block}{Examples}
        \begin{itemize}
            \item \textbf{Game Playing (e.g., Chess):} Competing agents use strategies like minimax and alpha-beta pruning.
            \item \textbf{Robotics:} 
            \begin{itemize}
                \item \textit{Swarm Robotics:} Groups of simple robots collaborate to execute complex tasks similar to ant colonies.
            \end{itemize}
        \end{itemize}
    \end{block}
    
    \begin{block}{Key Takeaways}
        \begin{itemize}
            \item \textbf{Inter-agent Communication:} Essential for coordinating actions and sharing information.
            \item \textbf{Learning and Adaptation:} Employing ML techniques for adaptive strategies based on interactions and experiences.
        \end{itemize}
    \end{block}
\end{frame}

\begin{frame}[fragile]{Defining Multi-Agent Systems}
  \frametitle{Overview}
  \begin{itemize}
    \item Definition and characteristics of multi-agent systems (MAS)
    \item Real-world applications of MAS
    \item Importance of interactions among agents
  \end{itemize}
\end{frame}

\begin{frame}[fragile]{What is a Multi-Agent System (MAS)?}
  \begin{block}{Definition}
    A multi-agent system is a system consisting of multiple interacting agents, which can be software programs or physical entities. Each agent operates autonomously and makes decisions based on its perception of the environment.
  \end{block}
  
  \begin{block}{Key Characteristics}
    \begin{enumerate}
      \item \textbf{Autonomy:} Agents operate independently based on their own logic.
      \item \textbf{Interaction:} Agents communicate, cooperate, or compete with each other.
      \item \textbf{Adaptability:} Agents can adapt and learn from their experiences.
      \item \textbf{Locally Perceived Environment:} Each agent has limited perception, influencing its decisions.
    \end{enumerate}
  \end{block}
\end{frame}

\begin{frame}[fragile]{Examples of Multi-Agent Systems}
  \begin{block}{Types of Agents}
    \begin{itemize}
      \item \textbf{Reactive Agents:} Respond to stimuli without using an internal model (e.g., thermostat).
      \item \textbf{Deliberative Agents:} Utilize internal models for decision-making (e.g., self-driving car).
      \item \textbf{Learning Agents:} Learn from experiences and improve over time (e.g., recommendation systems).
    \end{itemize}
  \end{block}
  
  \begin{block}{Applications}
    \begin{itemize}
      \item \textbf{Robotics:} Swarm robotics coordinating UAVs for surveillance.
      \item \textbf{Transportation:} Traffic management systems optimizing vehicle interactions.
      \item \textbf{Game Playing:} Competitive environments in games like chess or poker.
      \item \textbf{Market Simulation:} Agents simulating buyer/seller interactions in economics.
    \end{itemize}
  \end{block}
\end{frame}

\begin{frame}[fragile]
    \frametitle{Strategies for Multi-Agent Search - Learning Objectives}
    \begin{block}{Learning Objectives}
        \begin{itemize}
            \item Understand various strategies applied in multi-agent search scenarios.
            \item Identify the advantages and limitations of these strategies.
            \item Develop insight into their applications in real-world problems.
        \end{itemize}
    \end{block}
\end{frame}

\begin{frame}[fragile]
    \frametitle{Strategies for Multi-Agent Search - Key Concepts}
    \begin{block}{Definition of Multi-Agent Search}
        Multi-agent search involves multiple autonomous agents working in a shared environment to explore, gather information, or solve problems. These agents can either cooperate, compete, or a combination of both.
    \end{block}
    
    \begin{enumerate}
        \item \textbf{Cooperative Search}
            \begin{itemize}
                \item Agents work together to achieve a common goal, sharing information and tasks.
                \item \textit{Example}: In a robot team searching for victims in a disaster scenario, robots coordinate to cover more ground efficiently.
                \item \textit{Key Point}: Information sharing can significantly reduce search time.
            \end{itemize}
        
        \item \textbf{Competitive Search}
            \begin{itemize}
                \item Agents compete for resources or objectives, often leading to adversarial dynamics.
                \item \textit{Example}: In online games, players compete against each other to achieve victory.
                \item \textit{Key Point}: Strategies focus on predicting opponents' moves and countering them effectively.
            \end{itemize}
        
        \item \textbf{Distributed Search}
            \begin{itemize}
                \item Each agent follows independent policies, leading to varied search strategies.
                \item \textit{Example}: In distributed sensor networks, each sensor may have its own control algorithm to detect phenomena based on local data.
                \item \textit{Key Point}: Parallel processing can enhance search efficiency.
            \end{itemize}
        
        \item \textbf{Hybrid Strategies}
            \begin{itemize}
                \item Combining elements of cooperation and competition.
                \item \textit{Example}: Multiple delivery drones cooperate to map the most efficient delivery routes while competing to complete their tasks.
                \item \textit{Key Point}: Balancing cooperation and competition can enhance overall efficiency.
            \end{itemize}
    \end{enumerate}
\end{frame}

\begin{frame}[fragile]
    \frametitle{Strategies for Multi-Agent Search - Advantages & Limitations}
    \begin{block}{Advantages}
        \begin{itemize}
            \item Increased coverage and exploration speed.
            \item Parallelism can lead to faster problem-solving.
            \item Diverse strategies allow flexibility and adaptability in dynamic environments.
        \end{itemize}
    \end{block}
    
    \begin{block}{Limitations}
        \begin{itemize}
            \item Potential for conflict and resource wastage in competitive settings.
            \item Coordination overhead in cooperative tasks may slow progress.
            \item Complexity of strategy implementation can increase with the number of agents.
        \end{itemize}
    \end{block}
    
    \begin{block}{Conclusion}
        The choice of strategy in multi-agent search scenarios depends on the goals of the agents, the environment, and the tasks. Understanding these strategies enhances the design of effective multi-agent systems.
    \end{block}
\end{frame}

\begin{frame}[fragile]
    \frametitle{Example Code Snippet}
    \begin{lstlisting}[language=Python]
class Agent:
    def __init__(self, id):
        self.id = id
        self.position = None  # Agent's position in the environment

    def share_information(self, partner):
        # Share current position and mission status
        return f"Agent {self.id} shares info with Agent {partner.id}"

    def search(self, area):
        # Simple search algorithm
        print(f"Agent {self.id} searching area: {area}")
        # Update position logic goes here

# Example usage
agent1 = Agent(1)
agent2 = Agent(2)
print(agent1.share_information(agent2))
agent1.search("Zone A")
    \end{lstlisting}
\end{frame}

\begin{frame}[fragile]
    \frametitle{Cooperative Problem Solving - Introduction}
    \begin{block}{Definition}
        Cooperative problem solving involves multiple agents working collaboratively to achieve a common goal or solve a shared problem. 
    \end{block}
    \begin{itemize}
        \item Useful in scenarios where individual agents lack knowledge, resources, or capabilities.
    \end{itemize}
\end{frame}

\begin{frame}[fragile]
    \frametitle{Cooperative Problem Solving - Key Concepts}
    \begin{itemize}
        \item \textbf{Agents:} Autonomous entities capable of perceiving their environment and taking actions.
        \item \textbf{Collaboration:} Coordinating actions and sharing information to enhance effectiveness.
        \item \textbf{Shared Goals:} A common objective that drives agents to work together.
    \end{itemize}
\end{frame}

\begin{frame}[fragile]
    \frametitle{Cooperative Problem Solving - Examples}
    \begin{enumerate}
        \item \textbf{Robotic Teams:} 
            \begin{itemize}
                \item Search and rescue robots navigating a disaster zone.
                \item Communicate findings to optimize the search process.
            \end{itemize}
        \item \textbf{Distributed Sensor Networks:} 
            \begin{itemize}
                \item Multiple sensors monitoring environmental conditions.
                \item Sharing data for comprehensive understanding of changes.
            \end{itemize}
    \end{enumerate}
\end{frame}

\begin{frame}[fragile]
    \frametitle{Cooperative Problem Solving - Mechanisms}
    \begin{itemize}
        \item \textbf{Communication Protocols:} Ensuring agents share information effectively.
        \item \textbf{Consensus Algorithms:} Achieving agreement on state/decisions in uncertain environments.
        \item \textbf{Resource Sharing:} Efficient algorithms for allocating resources to avoid conflict.
    \end{itemize}
\end{frame}

\begin{frame}[fragile]
    \frametitle{Cooperative Problem Solving - Challenges}
    \begin{itemize}
        \item \textbf{Coordination Complexity:} Increased number of agents complicates coordination.
        \item \textbf{Conflict Resolution:} Differentiating collaborative and competitive situations.
        \item \textbf{Decentralization vs. Centralization:} Impact on performance and efficiency.
    \end{itemize}
\end{frame}

\begin{frame}[fragile]
    \frametitle{Cooperative Problem Solving - Illustrative Example}
    \begin{block}{Cooperative Search Problem}
        Imagine multiple drones deployed to survey a large forest for wildfires. Each drone:
        \begin{itemize}
            \item Uses its sensors to detect heat and smoke.
            \item Communicates location and findings to other drones.
            \item Adjusts flight path based on reports to cover more ground efficiently.
        \end{itemize}
    \end{block}
\end{frame}

\begin{frame}[fragile]
    \frametitle{Cooperative Problem Solving - Key Takeaways}
    \begin{itemize}
        \item Cooperation significantly improves problem-solving capabilities.
        \item Effective communication and shared objectives are crucial.
        \item Addressing cooperation challenges enhances multi-agent systems.
    \end{itemize}
\end{frame}

\begin{frame}[fragile]
    \frametitle{Cooperative Problem Solving - Conclusion}
    \begin{block}{Summary}
        Cooperative problem solving is essential for multi-agent systems, allowing agents to leverage their strengths to address complex challenges. 
        \begin{itemize}
            \item Fostering collaboration leads to outcomes unattainable individually.
        \end{itemize}
    \end{block}
\end{frame}

\begin{frame}[fragile]
    \frametitle{Game Playing in AI - Overview}
    \begin{itemize}
        \item Game playing is a fundamental domain in AI due to:
        \begin{enumerate}
            \item **Complex Problem Solving**: Encapsulates decision-making environments with varied strategies. Examples: chess, Go.
            \item **Adversarial Environment**: Involves competing agents with conflicting objectives, simulating real-world problems.
            \item **Structured Framework**: Clear rules and objectives facilitate problem formulation and evaluation.
            \item **Measurement of Performance**: Game outcomes provide immediate feedback, allowing performance comparisons.
        \end{enumerate}
    \end{itemize}
\end{frame}

\begin{frame}[fragile]
    \frametitle{Game Playing in AI - Key Points}
    \begin{itemize}
        \item **Historical Significance**: Landmark events such as:
        \begin{itemize}
            \item IBM's Deep Blue vs. Garry Kasparov (1997)
            \item Google's AlphaGo vs. Lee Sedol (2016)
        \end{itemize}
        \item **Algorithm Development**: Led to algorithms like minimax and reinforcement learning.
        \item **Generalization to Real-World Applications**: Insights applicable in economics, robotics, strategic planning.
    \end{itemize}
\end{frame}

\begin{frame}[fragile]
    \frametitle{Game Playing in AI - Example and Conclusion}
    \begin{itemize}
        \item **Illustrative Example**: 
        \begin{itemize}
            \item Chess as a Case Study: \( 10^{120} \) possible positions (Shannon's Number).
            \item Minimax Algorithm: Use of Alpha-Beta pruning to optimize decision-making.
        \end{itemize}
        \item **Conclusion**: Game playing in AI refines algorithms and decision-making strategies.
        \begin{itemize}
            \item Insights from games impact various sectors, making it a pivotal AI study area.
        \end{itemize}
    \end{itemize}
\end{frame}

\begin{frame}[fragile]
    \frametitle{Code Snippet for Minimax Algorithm}
    \begin{block}{Minimax Algorithm}
    \begin{lstlisting}[language=Python]
def minimax(node, depth, is_maximizing_player):
    if depth == 0 or game_over(node):
        return evaluate(node)
    
    if is_maximizing_player:
        max_eval = float('-inf')
        for child in get_children(node):
            eval = minimax(child, depth - 1, False)
            max_eval = max(max_eval, eval)
        return max_eval
    else:
        min_eval = float('inf')
        for child in get_children(node):
            eval = minimax(child, depth - 1, True)
            min_eval = min(min_eval, eval)
        return min_eval
    \end{lstlisting}
    \end{block}
\end{frame}

\begin{frame}[fragile]
    \frametitle{Adversarial Search Algorithms - Overview}
    \begin{block}{Overview}
        Adversarial search algorithms are essential in AI for making decisions in competitive environments, where multiple agents have opposing goals. These algorithms aim to find the optimal move for an agent in a game-like scenario, considering the potential moves of the opponent.
    \end{block}
\end{frame}

\begin{frame}[fragile]
    \frametitle{Adversarial Search Algorithms - Minimax Algorithm}
    \begin{block}{1. Minimax Algorithm}
        \begin{itemize}
            \item \textbf{Concept}: A recursive search algorithm used for minimizing the possible loss in a worst-case scenario. It assumes that the opponent also plays optimally.
            \item \textbf{Mechanism}:
              \begin{itemize}
                  \item \textbf{Decision Tree}: Represents all possible moves, alternating between "Max" (player's turn) and "Min" (opponent's turn).
                  \item \textbf{Evaluating Nodes}: Leaf nodes are evaluated using a utility function.
              \end{itemize}
            \item \textbf{Example}: In a Tic-Tac-Toe game, if it's 'X's turn (Max), the algorithm evaluates all possible moves to maximize X's chances of winning.
        \end{itemize}
    \end{block}
\end{frame}

\begin{frame}[fragile]
    \frametitle{Adversarial Search Algorithms - Minimax Code}
    \begin{lstlisting}[language=Python]
def minimax(node, depth, isMax):
    if depth == 0 or is_terminal(node):
        return evaluate(node)
    
    if isMax:
        best = -float('inf')
        for child in generate_children(node):
            best = max(best, minimax(child, depth - 1, False))
        return best
    else:
        best = float('inf')
        for child in generate_children(node):
            best = min(best, minimax(child, depth - 1, True))
        return best
    \end{lstlisting}
\end{frame}

\begin{frame}[fragile]
    \frametitle{Adversarial Search Algorithms - Alpha-Beta Pruning}
    \begin{block}{2. Alpha-Beta Pruning}
        \begin{itemize}
            \item \textbf{Concept}: An optimization technique for the Minimax algorithm.
            \item \textbf{Mechanism}:
              \begin{itemize}
                  \item Introduces two values: Alpha (best value that the maximizer can guarantee) and Beta (best value the minimizer can guarantee).
                  \item If a move is worse than previously examined, that branch is pruned.
              \end{itemize}
            \item \textbf{Example}: In Tic-Tac-Toe, Alpha-Beta pruning allows the algorithm to ignore branches that do not need exploration.
        \end{itemize}
    \end{block}
\end{frame}

\begin{frame}[fragile]
    \frametitle{Adversarial Search Algorithms - Alpha-Beta Code}
    \begin{lstlisting}[language=Python]
def alpha_beta(node, depth, alpha, beta, isMax):
    if depth == 0 or is_terminal(node):
        return evaluate(node)

    if isMax:
        best = -float('inf')
        for child in generate_children(node):
            best = max(best, alpha_beta(child, depth - 1, alpha, beta, False))
            alpha = max(alpha, best)
            if beta <= alpha:
                break  # Beta cut-off
        return best
    else:
        best = float('inf')
        for child in generate_children(node):
            best = min(best, alpha_beta(child, depth - 1, alpha, beta, True))
            beta = min(beta, best)
            if beta <= alpha:
                break  # Alpha cut-off
        return best
    \end{lstlisting}
\end{frame}

\begin{frame}[fragile]
    \frametitle{Adversarial Search Algorithms - Key Points}
    \begin{block}{Key Points to Remember}
        \begin{itemize}
            \item \textbf{Optimal Play}: Both Minimax and Alpha-Beta Pruning assume that both players play optimally.
            \item \textbf{Efficiency}: Alpha-Beta pruning significantly improves the efficiency of the Minimax algorithm.
            \item \textbf{Use in Games}: Commonly applied in two-player games like Chess, Checkers, and Tic-Tac-Toe.
        \end{itemize}
    \end{block}
\end{frame}

\begin{frame}[fragile]
    \frametitle{Adversarial Search Algorithms - Conclusion}
    \begin{block}{Conclusion}
        Understanding adversarial search algorithms provides insights into how AI can effectively operate in competitive environments. Mastery of algorithms like Minimax and Alpha-Beta pruning lays the foundation for advanced game-playing AI systems.
    \end{block}
\end{frame}

\begin{frame}[fragile]
    \frametitle{Evaluation Functions in Game Playing}
    \begin{block}{Understanding Evaluation Functions}
        Evaluation functions are essential components in game playing and AI.
        They help agents assess game states to determine the best course of action.
    \end{block}
\end{frame}

\begin{frame}[fragile]
    \frametitle{What is an Evaluation Function?}
    \begin{itemize}
        \item An \textbf{evaluation function} (denoted as $f$ or $E$) 
        takes a game state as input and returns a quantitative value.
        \item The aim is to maximize the value for the player and minimize it for the opponent.
    \end{itemize}
    \begin{block}{Example}
        In chess, a simple evaluation function might assign values based on piece material:
        \begin{itemize}
            \item Pawns: 1 point
            \item Knights/Bishops: 3 points
            \item Rooks: 5 points
            \item Queens: 9 points
        \end{itemize}
        The function totals the values of all pieces for position strength.
    \end{block}
\end{frame}

\begin{frame}[fragile]
    \frametitle{Designing Evaluation Functions}
    \begin{enumerate}
        \item \textbf{Feature Selection:} Identify relevant features for the game's strategy.
        \item \textbf{Weight Assignment:} Assign weights based on feature importance.
        \item \textbf{Combining Features:} Use a linear equation for the evaluation score:
        \begin{equation}
        E(state) = w_1 \cdot f_1(state) + w_2 \cdot f_2(state) + \ldots + w_n \cdot f_n(state)
        \end{equation}
    \end{enumerate}
\end{frame}

\begin{frame}[fragile]
    \frametitle{Importance of Evaluation Functions}
    \begin{itemize}
        \item \textbf{Improved Decision Making:} They guide the AI in selecting optimal moves.
        \item \textbf{Efficiency:} Help agents avoid exhaustive searches for quicker response times.
        \item \textbf{Adaptability:} Can be fine-tuned for different game situations and strategies.
    \end{itemize}
\end{frame}

\begin{frame}[fragile]
    \frametitle{Key Takeaways}
    \begin{itemize}
        \item Evaluation functions simplify complex game states into numerical values.
        \item The quality of these functions directly impacts AI performance.
        \item Applications extend beyond games; they're used in finance, scheduling, and decision-making.
    \end{itemize}
    \begin{block}{Conclusion}
        Understanding and optimizing evaluation functions is crucial for a strategic advantage in competitive environments.
    \end{block}
\end{frame}

\begin{frame}[fragile]
    \frametitle{Reinforcement Learning in Game Playing - Introduction}
    \begin{block}{Definition of Reinforcement Learning}
        Reinforcement Learning (RL) is a type of machine learning where an agent learns to make decisions by receiving feedback from its actions through rewards or penalties.
    \end{block}
    
    \begin{block}{Goal}
        The goal of reinforcement learning is to develop a strategy (policy) that maximizes the cumulative reward over time in various environments, including games.
    \end{block}
\end{frame}

\begin{frame}[fragile]
    \frametitle{Reinforcement Learning in Game Playing - Key Concepts}
    \begin{itemize}
        \item **Agent:** The learner or decision-maker (e.g., a character in a game).
        \item **Environment:** Everything the agent interacts with (game world).
        \item **State (S):** A specific configuration of the environment (e.g., the current board position).
        \item **Actions (A):** Possible moves the agent can take in a given state (e.g., move left, shoot).
        \item **Reward (R):** Feedback from the environment based on the agent's actions.
        \item **Policy ($\pi$):** A strategy that the agent employs to determine actions based on states.
    \end{itemize}
\end{frame}

\begin{frame}[fragile]
    \frametitle{The RL Process}
    \begin{enumerate}
        \item **Initialization:** Start with a random or predefined policy.
        \item **Interaction:** The agent takes an action, moving to a new state and receiving a reward.
        \item **Learning:** The agent updates its policy based on feedback from rewards using techniques like Q-learning or Policy Gradient methods.
        \item **Iteration:** The process repeats, refining the agent's decisions over time.
    \end{enumerate}
\end{frame}

\begin{frame}[fragile]
    \frametitle{Illustrative Example: AlphaGo}
    \begin{block}{Description}
        AlphaGo, developed by DeepMind, used reinforcement learning to master the game of Go.
    \end{block}
    \begin{block}{Technique}
        It combined RL with deep learning, where:
        \begin{itemize}
            \item The agent evaluated board positions (state) using neural networks.
            \item Predicted the best moves (actions).
        \end{itemize}
    \end{block}
    \begin{block}{Outcome}
        AlphaGo defeated world champions, showcasing the power of RL in complex strategic decision-making.
    \end{block}
\end{frame}

\begin{frame}[fragile]
    \frametitle{Important Algorithms in RL}
    \begin{itemize}
        \item **Q-learning:** A value-based method where:
        \begin{itemize}
            \item The agent learns a value function (Q-values) that estimates future rewards.
            \item **Update Equation:** 
            \begin{equation}
                Q(S, A) \leftarrow Q(S, A) + \alpha \left( R + \gamma \max_{A'} Q(S', A') - Q(S, A) \right)
            \end{equation}
            \item Where:
            \begin{itemize}
                \item $\alpha$ = learning rate
                \item $\gamma$ = discount factor
            \end{itemize}
        \end{itemize}
        
        \item **Policy Gradients:** Directly optimize the policy instead of estimating value functions.
    \end{itemize}
\end{frame}

\begin{frame}[fragile]
    \frametitle{Key Points and Conclusion}
    \begin{itemize}
        \item **Exploration vs. Exploitation:** Balancing the need to discover new actions (exploration) and using known high-reward actions (exploitation) is crucial in RL.
        \item **Transfer of Learning:** Once an agent masters one game, its strategies may apply to similar environments, enhancing learning efficiency.
    \end{itemize}
    \begin{block}{Conclusion}
        Reinforcement learning offers powerful techniques for decision-making in games. By learning from interactions with the environment, agents can develop sophisticated strategies, demonstrating adaptability and improved performance in complex, dynamic situations.
    \end{block}
\end{frame}

\begin{frame}[fragile]
    \frametitle{Case Studies in Multi-Agent Search}
    \begin{block}{Introduction to Multi-Agent Search}
        Multi-Agent Systems (MAS) involve multiple autonomous agents interacting and collaborating to solve problems. These systems are applied in various domains, demonstrating multi-agent search principles in complex environments.
    \end{block}
\end{frame}

\begin{frame}[fragile]
    \frametitle{Key Concepts}
    \begin{itemize}
        \item \textbf{Multi-Agent Systems (MAS):} Composed of multiple interacting intelligent agents that sense their environment, make decisions, and act, often collaboratively or competitively.
        \item \textbf{Search Algorithms:} Techniques to explore a problem space to find solutions; in multi-agent settings, agents coordinate search strategies for efficiency.
    \end{itemize}
\end{frame}

\begin{frame}[fragile]
    \frametitle{Case Study Examples}
    \begin{enumerate}
        \item \textbf{Robotics in Warehouse Management:}
            \begin{itemize}
                \item \textbf{Example:} Amazon Robotics
                \item \textbf{Application:} Autonomous systems in warehouses use multi-agent search algorithms for navigation and coordination.
                \item \textbf{Outcome:} Enhanced efficiency in order processing and inventory management.
            \end{itemize}
        
        \item \textbf{Traffic Management Systems:}
            \begin{itemize}
                \item \textbf{Example:} Adaptive Traffic Signal Control
                \item \textbf{Application:} Traffic signals act as agents optimizing flow with real-time data, minimizing congestion.
                \item \textbf{Outcome:} Reduced travel time and improved public safety.
            \end{itemize}
        
        \item \textbf{Distributed AI in Gaming:}
            \begin{itemize}
                \item \textbf{Example:} Multiplayer Online Games (e.g., Dota 2)
                \item \textbf{Application:} AI agents cooperate with players, using multi-agent search strategies for tactical decisions.
                \item \textbf{Outcome:} Enhanced gaming experience through dynamic interactions.
            \end{itemize}
    \end{enumerate}
\end{frame}

\begin{frame}[fragile]
    \frametitle{Key Points to Emphasize}
    \begin{itemize}
        \item \textbf{Collaboration vs. Competition:} Agents may collaborate towards common goals or compete for resources.
        \item \textbf{Algorithm Adaptation:} Different scenarios require varied search algorithms (e.g., cooperative algorithms like A* vs. competitive algorithms like Minimax).
        \item \textbf{Scalability:} MAS can scale from small networks to large systems, making them versatile across applications.
    \end{itemize}
\end{frame}

\begin{frame}[fragile]
    \frametitle{Conclusion}
    Case studies demonstrate the practical application of multi-agent systems, highlighting their benefits in efficiency and decision-making across various industries. Recognizing the underlying mechanisms is vital for future advancements in the field.
\end{frame}

\begin{frame}[fragile]
    \frametitle{Additional Resources}
    \begin{itemize}
        \item \textbf{Textbooks:} In-depth theories and applications of Multi-Agent Systems.
        \item \textbf{Online Courses:} Platforms like Coursera or edX for courses on SCMs and AI.
        \item \textbf{Research Papers:} Stay updated with the latest studies in the field.
    \end{itemize}
\end{frame}

\begin{frame}[fragile]
    \frametitle{References}
    \begin{enumerate}
        \item "Multi-Agent Systems: A Modern Approach to Distributed Artificial Intelligence," by G. Weiss.
        \item "Artificial Intelligence for Autonomous Vehicles: Agent-Based Techniques," by A. A. Aziz.
    \end{enumerate}
\end{frame}

\begin{frame}[fragile]
    \frametitle{Call to Action}
    Reflect on how multi-agent systems can integrate into your field of interest. What challenges and benefits can you envision in their application?
\end{frame}

\begin{frame}[fragile]
    \frametitle{Challenges in Multi-Agent Systems - Introduction}
    \begin{itemize}
        \item Multi-agent systems (MAS) consist of multiple interacting agents.
        \item These agents can be autonomous software programs or robots.
        \item MAS possess great potential for solving complex problems.
        \item Key challenges in their development and implementation need to be addressed.
    \end{itemize}
\end{frame}

\begin{frame}[fragile]
    \frametitle{Challenges in Multi-Agent Systems - Key Challenges}
    \begin{enumerate}
        \item Coordination and Communication
            \begin{itemize}
                \item Agents must share information, negotiate tasks, and coordinate actions.
                \item \textbf{Example:} Agents in a robotic soccer game need to communicate effectively.
            \end{itemize}
        \item Conflict Resolution
            \begin{itemize}
                \item Competing goals can lead to conflicts.
                \item \textbf{Example:} Agents competing for limited resources in allocation scenarios.
            \end{itemize}
    \end{enumerate}
\end{frame}

\begin{frame}[fragile]
    \frametitle{Challenges in Multi-Agent Systems - Continued}
    \begin{enumerate}[resume]
        \item Scalability
            \begin{itemize}
                \item Increased number of agents makes coordination challenging.
                \item \textbf{Example:} Hundreds of autonomous vehicles in smart city management can lead to data overload.
            \end{itemize}
        \item Robustness and Fault Tolerance
            \begin{itemize}
                \item Agents must function reliably despite failures.
                \item \textbf{Example:} In multi-drone delivery, failure of one drone shouldn't disrupt operations.
            \end{itemize}
        \item Learning and Adaptation
            \begin{itemize}
                \item Agents should learn and adapt their behavior through experience.
                \item \textbf{Example:} Agents in competitive gaming evolving strategies based on opponents.
            \end{itemize}
        \item Ethical and Security Issues
            \begin{itemize}
                \item Privacy and fairness concerns in decision-making processes.
                \item \textbf{Example:} Autonomous agents managing personal data must ensure compliance with regulations.
            \end{itemize}
    \end{enumerate}
\end{frame}

\begin{frame}[fragile]
    \frametitle{Challenges in Multi-Agent Systems - Conclusion}
    \begin{itemize}
        \item Addressing these challenges is crucial for successful MAS development.
        \item Innovating techniques to overcome obstacles can harness the potential of multi-agent cooperation.
        \item Solutions can lead to more sophisticated applications in robotics and smart environments.
    \end{itemize}
    \begin{block}{Illustration Suggestion}
        Consider including a flowchart on agent interactions or a diagram of a decentralized agent system for visual reference.
    \end{block}
\end{frame}

\begin{frame}[fragile]
    \frametitle{Future Trends in Multi-Agent Search}
    \begin{block}{Introduction to Future Trends}
        Multi-agent systems (MAS) are increasingly integrated into various applications, enabling better decision-making and cooperative problem-solving.
        This slide delves into significant trends that are shaping the evolution of multi-agent search systems and their functionalities.
    \end{block}
\end{frame}

\begin{frame}[fragile]
    \frametitle{Key Trends Affecting Multi-Agent Systems}
    \begin{enumerate}
        \item \textbf{Enhanced Collaboration through AI}
        \begin{itemize}
            \item AI-driven systems optimize collaboration by learning from shared experiences.
            \item \textit{Example:} Autonomous vehicles in smart cities sharing traffic data.
        \end{itemize}
        
        \item \textbf{Decentralized Learning}
        \begin{itemize}
            \item Agents improve performance using local data without central control.
            \item \textit{Example:} Robots in swarm robotics adapting in real-time search-and-rescue.
        \end{itemize}
    \end{enumerate}
\end{frame}

\begin{frame}[fragile]
    \frametitle{Continued Trends in Multi-Agent Systems}
    \begin{enumerate}
        \setcounter{enumi}{2}
        \item \textbf{Integration with Blockchain Technology}
        \begin{itemize}
            \item Secure and transparent information sharing via smart contracts.
            \item \textit{Example:} Decentralized marketplaces enhancing trust in transactions.
        \end{itemize}
        
        \item \textbf{Increased Focus on Emotional Intelligence}
        \begin{itemize}
            \item Future agents will interpret emotions to respond better.
            \item \textit{Example:} Customer service bots analyzing sentiment in dialogues.
        \end{itemize}

        \item \textbf{Cross-domain Applications}
        \begin{itemize}
            \item Expanding applications in healthcare, finance, and environmental management.
            \item \textit{Example:} Coordinating patient care through real-time data sharing.
        \end{itemize}
        
        \item \textbf{Use of Genetic Algorithms}
        \begin{itemize}
            \item Evolving strategies to solve complex problems using natural selection concepts.
            \item \textit{Example:} Agents developing strategies in competitive gaming scenarios.
        \end{itemize}
        
        \item \textbf{Advancements in Natural Language Processing (NLP)}
        \begin{itemize}
            \item Enhanced communication between agents and humans via improved NLP.
            \item \textit{Example:} Customer support systems leveraging NLP for effective query resolution.
        \end{itemize}
    \end{enumerate}
\end{frame}

\begin{frame}[fragile]
    \frametitle{Conclusion}
    \begin{block}{The Future is Collaborative and Intelligent}
        As multi-agent systems evolve, they will adopt advanced technologies that enhance collaboration, efficiency, and user-centered interactions. 
        The trends discussed will redefine agent functionalities and their impact across various industries.
    \end{block}
    
    \begin{itemize}
        \item AI and decentralized learning are crucial for enhancing collaboration.
        \item Blockchain and emotional intelligence will revolutionize interaction dynamics.
        \item MAS applications are broadening significantly, impacting diverse domains.
    \end{itemize}
\end{frame}

\begin{frame}[fragile]
    \frametitle{Ethical Implications - Overview}
    \begin{block}{Understanding the Ethical Considerations in Multi-Agent Systems}
        This section discusses key ethical dilemmas posed by deploying multi-agent systems. 
    \end{block}
\end{frame}

\begin{frame}[fragile]
    \frametitle{Ethical Implications - Key Concepts}
    \begin{enumerate}
        \item \textbf{Responsibility \& Accountability}
        \begin{itemize}
            \item Who is responsible when agents act autonomously?
            \item Ethical dilemmas may arise, particularly if decisions lead to harm.
        \end{itemize}
        
        \item \textbf{Bias and Fairness}
        \begin{itemize}
            \item Agents can perpetuate or amplify biases from training data.
            \item Example: AI hiring systems may favor certain demographic groups due to biased historical data.
        \end{itemize}
        
        \item \textbf{Privacy Concerns}
        \begin{itemize}
            \item Multi-agent systems often entail data collection, raising privacy issues.
            \item Focus on how data is gathered, stored, and used to protect user information.
        \end{itemize}

        \item \textbf{Autonomy vs Control}
        \begin{itemize}
            \item Balancing agent autonomy with human oversight is crucial.
            \item Example: Autonomous drones may operate efficiently but raise ethical concerns without human intervention protocols.
        \end{itemize}
    \end{enumerate}
\end{frame}

\begin{frame}[fragile]
    \frametitle{Ethical Implications - Illustrative Example}
    \begin{block}{Autonomous Vehicles}
        A self-driving car (a multi-agent system) must make crucial decisions:
        \begin{itemize}
            \item In an unavoidable accident, it must decide whether to protect its passengers or pedestrians.
            \item Raises questions about the ethical framework guiding the decision.
            \item Who bears responsibility for the AI's choices?
        \end{itemize}
    \end{block}
\end{frame}

\begin{frame}[fragile]
    \frametitle{Ethical Implications - Key Points}
    \begin{itemize}
        \item \textbf{Ethical frameworks} like utilitarianism and deontological ethics guide complex decision making.
        \item Emphasize the \textbf{importance of transparency} in algorithms to build trust.
        \item \textbf{Stakeholder engagement} is vital for shaping responsible AI development.
    \end{itemize}
\end{frame}

\begin{frame}[fragile]
    \frametitle{Ethical Implications - Final Thoughts}
    \begin{block}{Conclusion}
        As multi-agent systems integrate further into daily life, addressing ethical implications becomes a moral imperative, alongside technical challenges.
    \end{block}
\end{frame}

\begin{frame}[fragile]
    \frametitle{Collaborative Learning - Introduction}
    \begin{block}{What is Collaborative Learning?}
        Collaborative learning refers to a dynamic educational approach where individuals engage with one another in groups to enhance their understanding and problem-solving abilities. This model is particularly effective in multi-agent systems, where multiple intelligent agents interact within a common environment.
    \end{block}
\end{frame}

\begin{frame}[fragile]
    \frametitle{Collaborative Learning - Key Concepts}
    \begin{itemize}
        \item \textbf{Interdependence:} Success relies on the collaboration of agents who share goals.
        \item \textbf{Communication:} Agents must effectively share information and strategies among themselves.
        \item \textbf{Collective Problem Solving:} Solutions are developed through shared insights and strategies.
    \end{itemize}
\end{frame}

\begin{frame}[fragile]
    \frametitle{Enhancing Understanding in Multi-Agent Systems}
    \begin{enumerate}
        \item \textbf{Knowledge Sharing:}
        \begin{itemize}
            \item Agents exchange experiences and strategies, leading to improved performance.
            \item \textit{Example:} In a game-playing scenario, agents can learn from each other's successful moves, enhancing the game strategy.
        \end{itemize}
        \item \textbf{Collective Intelligence:}
        \begin{itemize}
            \item The group's performance often exceeds that of individual members due to diverse strategies and perspectives.
            \item \textit{Example:} In cooperative robot navigation, different agents may contribute unique routes, optimizing overall navigation efficiency.
        \end{itemize}
        \item \textbf{Role Allocation:}
        \begin{itemize}
            \item Agents can assume specialized roles based on their strengths, promoting efficiency.
            \item \textit{Example:} In a rescue mission, some agents can focus on information gathering while others coordinate actions.
        \end{itemize}
    \end{enumerate}
\end{frame}

\begin{frame}[fragile]
    \frametitle{Benefits of Collaborative Learning}
    \begin{itemize}
        \item \textbf{Increased Robustness:} By working together, agents can tackle complex challenges more efficiently.
        \item \textbf{Adaptability:} Collaborative learning allows systems to adapt to dynamic environments through shared experiences.
        \item \textbf{Enhanced Learning Rates:} Agents can accelerate their learning curves by leveraging the knowledge of their peers.
    \end{itemize}
\end{frame}

\begin{frame}[fragile]
    \frametitle{Illustration of Collaborative Learning}
    \begin{block}{Drone Communication Example}
    Consider a scenario where multiple autonomous drones are tasked with mapping an area. By using collaborative learning:
    \begin{itemize}
        \item Drones can communicate findings about terrain types to each other.
        \item They use shared algorithms to adjust paths based on real-time data from multiple drones.
    \end{itemize}
    \end{block}
    \begin{center}
    \begin{verbatim}
[Drone 1] <--Share Info-- [Drone 2]
       \               /
        \             /
         [Collaborative Learning]
        /             \
       /               \
[Drone 3] <--Share Info-- [Drone 4]
    \end{verbatim}
    \end{center}
\end{frame}

\begin{frame}[fragile]
    \frametitle{Key Takeaways}
    \begin{itemize}
        \item Collaborative learning is essential in multi-agent systems to improve individual and group performance.
        \item Effective communication and role allocation significantly enhance problem-solving.
        \item Applying collaborative methodologies can lead to innovative solutions in complex environments.
    \end{itemize}
\end{frame}

\begin{frame}[fragile]
    \frametitle{Tools and Technologies in Multi-Agent Systems - Overview}
    \begin{block}{Overview}
        In the realm of Artificial Intelligence (AI), multi-agent systems (MAS) involve the interaction of multiple agents, which can be either collaborative or competitive. Various tools and technologies are employed to design, implement, and evaluate these systems effectively. 
    \end{block}
\end{frame}

\begin{frame}[fragile]
    \frametitle{Tools and Technologies in Multi-Agent Systems - Key Tools}
    \begin{block}{Key Tools and Technologies}
        \begin{enumerate}
            \item \textbf{TensorFlow}  
                \begin{itemize}
                    \item \textbf{Description}: Open-source machine learning framework developed by Google. Ideal for building neural networks in multi-agent settings.
                    \item \textbf{Use Case}: Training reinforcement learning agents.
                \end{itemize}
                
            \item \textbf{PyTorch}  
                \begin{itemize}
                    \item \textbf{Description}: Open-source library from Facebook's AI Research, known for dynamic computation graphs.
                    \item \textbf{Use Case}: Developing agent policies for Multi-Agent Reinforcement Learning (MARL).
                \end{itemize}
                
            \item \textbf{OpenAI Gym}  
                \begin{itemize}
                    \item \textbf{Description}: Toolkit for developing and comparing reinforcement learning algorithms.
                    \item \textbf{Use Case}: Experiments with multi-agent environments.
                \end{itemize}
                
            \item \textbf{RLlib}  
                \begin{itemize}
                    \item \textbf{Description}: Library for reinforcement learning on top of Ray.
                    \item \textbf{Use Case}: Scaling training across clusters.
                \end{itemize}
                
            \item \textbf{MATLAB/Simulink}  
                \begin{itemize}
                    \item \textbf{Description}: Programming environment for algorithm development and data analysis.
                    \item \textbf{Use Case}: Simulation and design of real-time multi-agent systems.
                \end{itemize}
        \end{enumerate}
    \end{block}
\end{frame}

\begin{frame}[fragile]
    \frametitle{Tools and Technologies in Multi-Agent Systems - Highlights}
    \begin{block}{Key Points to Emphasize}
        \begin{itemize}
            \item \textbf{Flexibility}: Tools facilitate both simple agent behaviors and complex interactions.
            \item \textbf{Scalability}: Technologies enable scaling from single-agent to large-scale systems.
            \item \textbf{Experimental Frameworks}: Specialized environments for efficient testing of strategies and algorithms.
        \end{itemize}
    \end{block}
\end{frame}

\begin{frame}[fragile]
    \frametitle{Project and Evaluation - Overview}
    In this chapter, we dive into the exciting realm of multi-agent systems, exploring how various agents can interact, compete, or collaborate within complex environments. Course projects are designed to help you apply theoretical concepts in practical scenarios while honing your problem-solving skills.
\end{frame}

\begin{frame}[fragile]
    \frametitle{Project Topics}
    \begin{enumerate}
        \item \textbf{Game Playing Agents}
            \begin{itemize}
                \item \textbf{Objective}: Create an AI agent to play a classic board game (e.g., Chess, Checkers).
                \item \textbf{Focus}: Implementation of algorithms such as Minimax, Alpha-Beta Pruning, or Monte Carlo Tree Search.
                \item \textbf{Example}: Develop an agent that can beat a weak human player or another AI with basic strategies.
            \end{itemize}
        
        \item \textbf{Collaborative Search}
            \begin{itemize}
                \item \textbf{Objective}: Design a multi-agent system where agents collaborate to achieve a shared goal (e.g., search and rescue in a simulated environment).
                \item \textbf{Focus}: Use of communication protocols among agents and shared knowledge for efficient search.
                \item \textbf{Example}: Simulate a drone swarm for locating missing persons in a grid-based environment, allowing agents to share information about searched areas.
            \end{itemize}
        
        \item \textbf{Competitive Multi-Agent Systems}
            \begin{itemize}
                \item \textbf{Objective}: Create agents that compete in a resource acquisition game (e.g., Trading, Resource Management).
                \item \textbf{Focus}: Development of strategies for resource allocation and monopolization while managing competition.
                \item \textbf{Example}: Implement a marketplace simulation where agents compete to maximize profit while responding to market changes.
            \end{itemize}
    \end{enumerate}
\end{frame}

\begin{frame}[fragile]
    \frametitle{Evaluation Criteria}
    \begin{enumerate}
        \item \textbf{Functionality} (40\%)
            \begin{itemize}
                \item Does the agent perform the intended tasks?
                \item Are the implemented algorithms functioning correctly in various scenarios?
            \end{itemize}
        
        \item \textbf{Efficiency} (30\%)
            \begin{itemize}
                \item Evaluate the performance of the agent regarding computation time and resource usage.
                \item Is the agent able to reach decisions quickly enough to play effectively in real-time situations?
            \end{itemize}
        
        \item \textbf{Innovation} (20\%)
            \begin{itemize}
                \item How creative or original are your strategies or algorithmic approaches?
                \item Did you implement unique features or optimizations that enhance your agent's capabilities?
            \end{itemize}
        
        \item \textbf{Documentation and Presentation} (10\%)
            \begin{itemize}
                \item Is there a clear explanation of your design choices and methodologies?
                \item How well did you present your project findings, including overall insights and learned lessons?
            \end{itemize}
    \end{enumerate}
\end{frame}

\begin{frame}[fragile]
    \frametitle{Key Points to Emphasize}
    \begin{itemize}
        \item \textbf{Collaboration and Competition}: Understand the dynamics that drive interactions in multi-agent environments.
        \item \textbf{Algorithm Choice}: The choice of algorithms can greatly impact your agent's performance; explore and analyze their applicability.
        \item \textbf{Iterative Process}: Development involves testing, evaluating, and refining your approach based on results and feedback.
    \end{itemize}
    \textbf{Conclusion:} This project will not only reinforce your knowledge of multi-agent systems but also give you practical skills in AI development, collaboration, and critical thinking. Embrace this opportunity to innovate and learn through hands-on experience!
\end{frame}

\begin{frame}[fragile]
    \frametitle{Summary and Q\&A - Overview of Multi-Agent Search and Game Playing Concepts}
    
    \begin{block}{1. Multi-Agent Systems}
        \begin{itemize}
            \item \textbf{Definition:} Multiple autonomous entities (agents) interacting to achieve goals.
            \item \textbf{Key Characteristics:} Autonomy, cooperation, competition, communication.
        \end{itemize}
    \end{block}

    \begin{block}{2. Game Theory in AI}
        \begin{itemize}
            \item \textbf{Concept:} Framework for analyzing decision-making in competitive contexts.
            \item \textbf{Key Components:}
            \begin{itemize}
                \item Players: The agents.
                \item Strategies: Decisions available to players.
                \item Payoffs: Outcomes of strategies.
            \end{itemize}
            \item \textbf{Example:} Prisoner's Dilemma.
        \end{itemize}
    \end{block}
\end{frame}

\begin{frame}[fragile]
    \frametitle{Summary and Q\&A - Search Algorithms and Coordination}

    \begin{block}{3. Search Algorithms for MAS}
        \begin{itemize}
            \item \textbf{Adversarial Search:} 
            \begin{itemize}
                \item Minimax Algorithm: Finds optimal strategy by minimizing potential losses.
                \item Formula:
                \begin{equation}
                    V(A) = 
                    \begin{cases} 
                    \max_{s \in S} V(s) & \text{if player A's turn} \\
                    \min_{s \in S} V(s) & \text{if player B's turn}
                    \end{cases}
                \end{equation}
                \item Alpha-Beta Pruning: Optimization technique that eliminates non-essential branches.
            \end{itemize}
        \end{itemize}
    \end{block}

    \begin{block}{4. Coordination and Collaboration}
        \begin{itemize}
            \item Agents may need to work together for efficiency.
            \item Techniques: negotiation, coalition formation, shared resources.
        \end{itemize}
    \end{block}
\end{frame}

\begin{frame}[fragile]
    \frametitle{Summary and Q\&A - Applications and Key Points}

    \begin{block}{5. Applications of Multi-Agent Systems}
        \begin{itemize}
            \item \textbf{Robotics:} Autonomous robots for search and rescue.
            \item \textbf{Traffic Management:} Agents optimizing traffic flow.
            \item \textbf{Game Playing:} AI in strategic board games like chess and Go.
        \end{itemize}
    \end{block}

    \begin{block}{Key Points to Emphasize}
        \begin{itemize}
            \item Understanding dynamics between competitive and cooperative agents is crucial.
            \item Algorithm choice significantly impacts outcomes.
            \item Real-world applications illustrate practical utility.
        \end{itemize}
    \end{block}

    \begin{block}{Q\&A Section}
        \begin{itemize}
            \item Open the floor for student questions on multi-agent concepts.
            \item Encourage sharing of examples and challenges faced.
        \end{itemize}
    \end{block}
\end{frame}


\end{document}