\documentclass[aspectratio=169]{beamer}

% Theme and Color Setup
\usetheme{Madrid}
\usecolortheme{whale}
\useinnertheme{rectangles}
\useoutertheme{miniframes}

% Additional Packages
\usepackage[utf8]{inputenc}
\usepackage[T1]{fontenc}
\usepackage{graphicx}
\usepackage{booktabs}
\usepackage{listings}
\usepackage{amsmath}
\usepackage{amssymb}
\usepackage{xcolor}
\usepackage{tikz}
\usepackage{pgfplots}
\pgfplotsset{compat=1.18}
\usetikzlibrary{positioning}
\usepackage{hyperref}

% Custom Colors
\definecolor{myblue}{RGB}{31, 73, 125}
\definecolor{mygray}{RGB}{100, 100, 100}
\definecolor{mygreen}{RGB}{0, 128, 0}
\definecolor{myorange}{RGB}{230, 126, 34}
\definecolor{mycodebackground}{RGB}{245, 245, 245}

% Set Theme Colors
\setbeamercolor{structure}{fg=myblue}
\setbeamercolor{frametitle}{fg=white, bg=myblue}
\setbeamercolor{title}{fg=myblue}
\setbeamercolor{section in toc}{fg=myblue}
\setbeamercolor{item projected}{fg=white, bg=myblue}
\setbeamercolor{block title}{bg=myblue!20, fg=myblue}
\setbeamercolor{block body}{bg=myblue!10}
\setbeamercolor{alerted text}{fg=myorange}

% Set Fonts
\setbeamerfont{title}{size=\Large, series=\bfseries}
\setbeamerfont{frametitle}{size=\large, series=\bfseries}
\setbeamerfont{caption}{size=\small}
\setbeamerfont{footnote}{size=\tiny}

% Custom Commands
\newcommand{\hilight}[1]{\colorbox{myorange!30}{#1}}
\newcommand{\concept}[1]{\textcolor{myblue}{\textbf{#1}}}

% Title Page Information
\title[Chapter 8: Ethical Implications of AI]{Chapter 8: Ethical Implications of AI}
\author[J. Smith]{John Smith, Ph.D.}
\institute[University Name]{
  Department of Computer Science\\
  University Name\\
  \vspace{0.3cm}
  Email: email@university.edu\\
  Website: www.university.edu
}
\date{\today}

% Document Start
\begin{document}

\frame{\titlepage}

\begin{frame}[fragile]
    \titlepage
\end{frame}

\begin{frame}[fragile]
    \frametitle{Introduction to Ethical Implications of AI}
    \begin{block}{Overview of Ethical Challenges}
        Ethical challenges related to Artificial Intelligence include privacy, bias, and societal impacts.
    \end{block}
\end{frame}

\begin{frame}[fragile]
    \frametitle{Definition of Ethical Implications in AI}
    Ethical implications in AI refer to the moral considerations that arise during the development, deployment, and use of AI systems. 
    These issues emerge due to the profound impact AI has on individuals and society as a whole.
\end{frame}

\begin{frame}[fragile]
    \frametitle{Key Ethical Challenges}
    \begin{enumerate}
        \item \textbf{Privacy}
        \begin{itemize}
            \item AI systems often require vast amounts of data, including sensitive personal information.
            \item \textbf{Example}: Facial recognition technology can identify individuals in public spaces without their consent, leading to surveillance concerns.
        \end{itemize}
        
        \item \textbf{Bias}
        \begin{itemize}
            \item AI algorithms can perpetuate or exacerbate biases present in training data, leading to unfair treatment of certain groups.
            \item \textbf{Example}: An AI hiring tool trained on historical data may favor candidates from specific demographics, disadvantaging others.
        \end{itemize}
        
        \item \textbf{Societal Impact}
        \begin{itemize}
            \item The deployment of AI can lead to shifts in employment, human interaction, and social structures, raising questions about economic inequality.
            \item \textbf{Example}: Automation in industries like manufacturing can lead to job losses, affecting entire communities.
        \end{itemize}
    \end{enumerate}
\end{frame}

\begin{frame}[fragile]
    \frametitle{Additional Considerations}
    \begin{itemize}
        \item \textbf{Accountability}: Who is responsible for decisions made by AI systems? Establishing accountability is crucial for addressing the consequences of AI actions.
        \item \textbf{Transparency}: AI systems should be transparent to ensure that stakeholders understand how decisions are made, promoting trust and accountability.
    \end{itemize}
\end{frame}

\begin{frame}[fragile]
    \frametitle{Key Points to Emphasize}
    \begin{itemize}
        \item The importance of ethical frameworks in guiding the development of AI.
        \item Ongoing dialogues among stakeholders (developers, policymakers, and the public) are essential for addressing these challenges.
        \item Regulatory frameworks may be necessary to protect individuals and society from potential harms of AI.
    \end{itemize}
\end{frame}

\begin{frame}[fragile]
    \frametitle{Conclusion}
    As AI technology evolves, so too must our understanding and approach to its ethical implications. 
    Engaging with these challenges is vital for creating an equitable and responsible AI-driven future.
\end{frame}

\begin{frame}[fragile]
    \frametitle{Discussion Points}
    Encourage students to consider how they can contribute to ethical AI practices in their own work and study. 
    Inviting discussions and debates can enhance their understanding and prepare them to address these critical issues in real-world contexts.
\end{frame}

\begin{frame}[fragile]
    \frametitle{Understanding AI Ethics}
    AI Ethics refers to the set of principles, values, and guidelines that govern the development, deployment, and use of Artificial Intelligence technologies. It encompasses a range of moral considerations related to the impact of AI on individuals, society, and the environment.
\end{frame}

\begin{frame}[fragile]
    \frametitle{Key Principles of AI Ethics}
    \begin{enumerate}
        \item \textbf{Transparency}
            \begin{itemize}
                \item AI systems should be understandable and explainable to users and stakeholders. 
                \item Example: An algorithm that determines loan approvals should provide reasons for its decisions.
            \end{itemize}
        \item \textbf{Fairness}
            \begin{itemize}
                \item AI should be designed to operate without bias, ensuring equitable treatment across all user groups.
                \item Example: Avoiding biased outcomes in hiring algorithms that may favor certain demographics.
            \end{itemize}
        \item \textbf{Accountability}
            \begin{itemize}
                \item Developers and organizations must take responsibility for AI systems' actions and decisions.
                \item Example: If an AI makes a harmful decision, there should be a clear process for redress and correction.
            \end{itemize}
        \item \textbf{Privacy}
            \begin{itemize}
                \item Respect user privacy by safeguarding personal data and obtaining informed consent before data collection.
                \item Example: Users should be informed about what data is collected when using an AI-enabled service.
            \end{itemize}
        \item \textbf{Safety and Security}
            \begin{itemize}
                \item AI systems must be reliable, safe, and secure against malicious use or unintentional harm.
                \item Example: Autonomous vehicles must be programmed to prioritize human safety.
            \end{itemize}
    \end{enumerate}
\end{frame}

\begin{frame}[fragile]
    \frametitle{Why AI Ethics Matters}
    Ethical considerations in AI are crucial for fostering trust, preventing harm, and encouraging responsible innovation. Addressing ethical implications is essential for:
    \begin{itemize}
        \item Protecting individuals and groups.
        \item Promoting sustainability in the development of AI technologies.
    \end{itemize}
    
    \textbf{Key Takeaways:}
    \begin{itemize}
        \item Understanding AI ethics is foundational for responsible AI development.
        \item Implementing ethical principles helps address critical issues like bias, privacy, and accountability.
        \item Successful AI deployment requires a balance between technological advancement and ethical responsibility. 
    \end{itemize}
    
    \textbf{Conclusion:} 
    The ethical implications of AI are profound and widespread. It is vital to embed ethical reflection into AI practices to ensure technologies benefit society as a whole.
\end{frame}

\begin{frame}[fragile]{Privacy Concerns in AI - Overview}
    \begin{block}{Key Concepts to Understand}
        \begin{itemize}
            \item \textbf{Data Privacy:} Proper handling and control of sensitive personal information.
            \item \textbf{Data Collection Methods:}
            \begin{enumerate}
                \item Surveillance Techniques
                \item User-Submitted Data
                \item Third-Party Data Aggregation
            \end{enumerate}
            \item \textbf{User Consent:} Informed consent, opt-in vs. opt-out mechanisms.
        \end{itemize}
    \end{block}
\end{frame}

\begin{frame}[fragile]{Privacy Concerns in AI - Example & Key Points}
    \begin{block}{Illustrative Example}
        Consider a smartphone app that tracks user location:
        \begin{itemize}
            \item \textbf{Data Collection:} Tracks location continuously.
            \item \textbf{User Consent:} Users may not read terms fully.
            \item \textbf{Implications:} Selling data without user awareness creates ethical issues.
        \end{itemize}
    \end{block}

    \begin{block}{Key Points to Emphasize}
        \begin{itemize}
            \item \textbf{Transparency:} Clear communication of data policies.
            \item \textbf{Data Minimization:} Collect only necessary data.
            \item \textbf{Security Measures:} Implement robust data protection protocols.
        \end{itemize}
    \end{block}
\end{frame}

\begin{frame}[fragile]{Privacy Concerns in AI - Conclusion}
    \begin{block}{Conclusion}
        Understanding data privacy in AI is crucial for building trust. 
        \begin{itemize}
            \item Foster informed consent and maintain transparency.
            \item Contribute to ethical AI use and user compliance.
            \item Respect and uphold individual rights and societal norms.
        \end{itemize}
    \end{block}
\end{frame}

\begin{frame}[fragile]
    \frametitle{Bias in AI Systems - Introduction}
    \begin{block}{Definition of Bias}
        Bias in AI refers to systematic and unfair discrimination in decision-making processes due to prejudiced assumptions made during algorithm design or data collection. This can lead to unjust outcomes for certain individuals or groups.
    \end{block}

    \begin{itemize}
        \item Introduction to Bias in AI
    \end{itemize}
\end{frame}

\begin{frame}[fragile]
    \frametitle{Bias in AI Systems - Manifestation}
    \begin{enumerate}
        \item Data Bias
        \begin{itemize}
            \item \textbf{Example:} Facial recognition systems trained predominantly on light-skinned images may perform poorly on darker-skinned individuals, leading to false negatives.
            \item \textbf{Types:}
            \begin{itemize}
                \item Historical Bias
                \item Sampling Bias
            \end{itemize}
        \end{itemize}
        
        \item Algorithmic Bias
        \begin{itemize}
            \item Algorithms may unintentionally prioritize certain features.
            \item \textbf{Example:} AI recruitment tools may overlook diverse talent.
        \end{itemize}

        \item User Bias
        \begin{itemize}
            \item AI systems reflect the biases of their developers.
            \item \textbf{Example:} Content recommendation systems may amplify developer/user biases.
        \end{itemize}
    \end{enumerate}
\end{frame}

\begin{frame}[fragile]
    \frametitle{Bias in AI Systems - Consequences and Key Points}
    \begin{block}{Consequences}
        \begin{itemize}
            \item Biased decision-making can lead to unfair practices in hiring, lending, law enforcement, and healthcare.
            \item Trust in AI systems can decrease with evident biases.
        \end{itemize}
    \end{block}

    \begin{block}{Key Points}
        \begin{itemize}
            \item Addressing bias is crucial for fair AI outcomes.
            \item Continuous monitoring and diverse datasets are essential.
            \item Ethical considerations must be integrated throughout the AI lifecycle.
        \end{itemize}
    \end{block}
\end{frame}

\begin{frame}[fragile]
    \frametitle{Bias in AI Systems - Illustrative Examples and Conclusion}
    \begin{itemize}
        \item \textbf{Criminal Justice:} Biased algorithms for parole risk assessments may unfairly target minority populations.
    \end{itemize}

    \begin{block}{Summary and Action Items}
        \begin{itemize}
            \item Encourage examination of datasets for representativity.
            \item Promote transparency in AI model training.
            \item Advocate for regulatory frameworks for bias mitigation.
        \end{itemize}
    \end{block}

    \begin{block}{Conclusion}
        Understanding and addressing bias in AI systems is essential for ethical AI applications.
    \end{block}
\end{frame}

\begin{frame}[fragile]
    \frametitle{Suggested Activity}
    \begin{block}{Workshop}
        Conduct a data analysis session to identify and discuss potential biases in datasets related to hiring or loan approvals.
    \end{block}
\end{frame}

\begin{frame}[fragile]
    \frametitle{Case Studies on Ethical Dilemmas}
    \begin{block}{Overview}
        This section examines real-world examples showcasing the ethical implications of artificial intelligence (AI). 
        The case studies highlight both successful applications and failures, emphasizing the importance of ethics in AI.
    \end{block}
\end{frame}

\begin{frame}[fragile]
    \frametitle{Key Case Studies}
    \begin{enumerate}
        \item \textbf{AI in Hiring Practices: Amazon Recruitment Tool}
        \begin{itemize}
            \item \textbf{Scenario:} Developed to streamline hiring, but biased against women.
            \item \textbf{Outcome:} Trained on a male-dominated dataset, devalued female resumes.
            \item \textbf{Emphasis:} Importance of diverse training data and evaluation of AI outputs.
        \end{itemize}
        
        \item \textbf{Facial Recognition Technology: Clearview AI}
        \begin{itemize}
            \item \textbf{Scenario:} Uses scraped images for facial recognition by law enforcement.
            \item \textbf{Concerns:} Privacy violations and discriminatory practices.
            \item \textbf{Outcome:} Legal scrutiny and lawsuits unfolded highlighting ethical dilemmas.
            \item \textbf{Emphasis:} Balancing public safety with individual privacy rights.
        \end{itemize}

        \item \textbf{Predictive Policing: PredPol}
        \begin{itemize}
            \item \textbf{Scenario:} AI tool forecasting potential crime areas based on historical data.
            \item \textbf{Challenges:} Reflection of existing biases leading to over-policing.
            \item \textbf{Outcome:} Multiple cities paused use due to effectiveness and fairness concerns.
            \item \textbf{Emphasis:} Need for algorithm analysis in law enforcement applications.
        \end{itemize}
    \end{enumerate}
\end{frame}

\begin{frame}[fragile]
    \frametitle{Key Points to Highlight}
    \begin{itemize}
        \item \textbf{Bias and Fairness:} AI systems can mirror societal biases; diverse datasets are needed to ensure equitable outcomes.
        \item \textbf{Transparency and Accountability:} Organizations should disclose AI operation methods and be accountable for their societal impacts.
        \item \textbf{Legal and Ethical Frameworks:} Need for evolving frameworks as AI technology advances to safeguard individual rights.
    \end{itemize}
    \begin{block}{Conclusion}
        These case studies remind us that while AI enhances decision-making and efficiency, prioritizing ethical considerations is crucial for beneficial and fair outcomes.
    \end{block}
\end{frame}

\begin{frame}[fragile]
    \frametitle{Proposed Solutions to Ethical Issues}
    
    As artificial intelligence (AI) technologies become increasingly integrated into our daily lives, addressing their ethical implications is essential. This slide presents key strategies to mitigate the negative impacts of AI, focusing on three main areas: 
    \begin{itemize}
        \item Transparency
        \item Accountability
        \item Inclusive design practices
    \end{itemize}
\end{frame}

\begin{frame}[fragile]
    \frametitle{Transparency}
    
    \begin{block}{Definition}
        Transparency in AI refers to the clarity around how AI systems operate, including how they make decisions and use data.
    \end{block}

    \begin{itemize}
        \item \textbf{Understandability}: Users should be able to comprehend the AI model's functioning (e.g., data utilization and decision logic).
        \item \textbf{Documentation}: Provide detailed documentation of algorithms and data sources to stakeholders.
    \end{itemize}

    \begin{block}{Example}
        Transparent algorithms in healthcare AI help clinicians understand patient risk assessments, allowing explanations to patients about AI-driven decisions.
    \end{block}

\end{frame}

\begin{frame}[fragile]
    \frametitle{Accountability and Inclusive Design}

    \begin{block}{Accountability}
        \begin{itemize}
            \item \textbf{Clear Ownership}: Establish responsibility for AI system outcomes – developers, companies, or users.
            \item \textbf{Auditing Mechanisms}: Implement regular audits of AI systems to evaluate decision-making processes and results.
        \end{itemize}
        \begin{block}{Example}
            In the event of an autonomous vehicle accident, accountability must clarify liability (manufacturer, developer, or owner).
        \end{block}
    \end{block}

    \begin{block}{Inclusive Design Practices}
        \begin{itemize}
            \item \textbf{Diverse Data Representation}: Ensure training data reflects user population diversity to minimize biases.
            \item \textbf{User Involvement}: Engage a diverse group of stakeholders in the design process to identify potential uses and misuses of AI technologies.
        \end{itemize}
        \begin{block}{Example}
            AI speech recognition systems trained on diverse accents improve accessibility for users from various linguistic backgrounds.
        \end{block}
    \end{block}
\end{frame}

\begin{frame}[fragile]
    \frametitle{Conclusion and Additional Considerations}

    \begin{block}{Conclusion}
        Addressing the ethical implications of AI requires proactive strategies. By prioritizing transparency, ensuring accountability, and employing inclusive design practices, we can create AI technologies that promote fairness, trust, and usability.
    \end{block}

    \begin{itemize}
        \item \textbf{Stakeholder Engagement}: Foster discussions among developers, ethicists, and users to shape ethical AI practices.
        \item \textbf{Regulatory Compliance}: Align solutions with existing regulations and global standards to enhance public trust.
    \end{itemize}

    By focusing on these proposed solutions, we can navigate the ethical landscape of AI more effectively, balancing innovation with responsibility.
\end{frame}

\begin{frame}[fragile]{The Role of Regulation and Policy - Introduction}
    \begin{block}{Introduction}
        As artificial intelligence (AI) technology becomes more integrated into various sectors of society, the necessity for clear regulations and policies grows immensely. This slide discusses the importance of establishing frameworks to govern AI, ensuring ethical standards are upheld and potential harms are minimized.
    \end{block}
\end{frame}

\begin{frame}[fragile]{The Role of Regulation and Policy - Why Regulation is Necessary}
    \begin{enumerate}
        \item \textbf{Rapid Development}:
        \begin{itemize}
            \item AI is evolving at an unprecedented pace, often outstripping existing legal and ethical frameworks.
            \item This can lead to unintended consequences, such as biased algorithms or privacy violations.
        \end{itemize}
        
        \item \textbf{Complex Interactions}:
        \begin{itemize}
            \item Technology interacts with sensitive data and decisions that significantly impact individuals, such as job automation, law enforcement, and healthcare.
        \end{itemize}
        
        \item \textbf{Global Reach}:
        \begin{itemize}
            \item AI technologies transcend national boundaries, making international cooperation and regulatory alignment essential for managing risks effectively.
        \end{itemize}
    \end{enumerate}
\end{frame}

\begin{frame}[fragile]{The Role of Regulation and Policy - Key Areas of Focus}
    \begin{enumerate}
        \item \textbf{Transparency}:
        \begin{itemize}
            \item Regulations should mandate explanations for AI decision-making processes.
            \item \textit{Example}: A bank using AI to approve loans must provide clear information on how decisions are made to ensure fairness.
        \end{itemize}
        
        \item \textbf{Accountability}:
        \begin{itemize}
            \item Establishing who is responsible when AI systems cause harm or fail is vital.
            \item \textit{Example}: In autonomous vehicles, the question of liability for accidents can involve manufacturers, developers, or owners.
        \end{itemize}
        
        \item \textbf{Data Privacy}:
        \begin{itemize}
            \item Policies should safeguard personal information processed by AI and ensure compliance with regulations such as GDPR.
            \item \textit{Example}: AI applications in healthcare must anonymize patient data and obtain consent for data usage.
        \end{itemize}
    \end{enumerate}
\end{frame}

\begin{frame}[fragile]
    \frametitle{Future Trends in AI Ethics - Introduction}
    \begin{itemize}
        \item As AI evolves, so do the ethical considerations in its development and deployment.
        \item Understanding future trends in AI ethics is essential for societal benefits and minimizing harms.
    \end{itemize}
\end{frame}

\begin{frame}[fragile]
    \frametitle{Future Trends in AI Ethics - Key Trends}
    \begin{enumerate}
        \item Increased Regulation and Oversight
            \begin{itemize}
                \item Stricter regulations for AI's use globally.
                \item Example: The European Union's AI Act.
            \end{itemize}
        \item Focus on Fairness and Bias Mitigation
            \begin{itemize}
                \item Enhanced fairness and inclusivity in algorithms.
                \item Example: IBM’s AI Fairness 360 toolkit.
            \end{itemize}
        \item Explainability and Transparency
            \begin{itemize}
                \item Demand for understanding AI decision-making.
                \item Example: LIME technique for interpretable models.
            \end{itemize}
    \end{enumerate}
\end{frame}

\begin{frame}[fragile]
    \frametitle{Future Trends in AI Ethics - Further Considerations}
    \begin{enumerate}
        \setcounter{enumi}{3}
        \item AI and Societal Impact Assessments
            \begin{itemize}
                \item Conduct assessments for potential societal consequences.
                \item Example: Evaluating AI in hiring tools.
            \end{itemize}
        \item Collaboration Between Stakeholders
            \begin{itemize}
                \item Involvement of technologists, ethicists, and policy-makers.
                \item Example: Partnership on AI for best practices in ethics.
            \end{itemize}
    \end{enumerate}

    \begin{block}{Conclusion}
        \begin{itemize}
            \item Proactive in addressing ethical implications of AI as technology progresses.
            \item Continuous discussions and collaborations are crucial for ethical AI.
        \end{itemize}
    \end{block}
\end{frame}

\begin{frame}[fragile]
    \frametitle{Key Takeaways}
    \begin{itemize}
        \item AI ethics are shifting towards stricter regulations, fairness, and transparency.
        \item Collaboration amongst stakeholders is essential for responsible AI development.
        \item Ethical considerations must adapt to technological advancements.
    \end{itemize}
\end{frame}

\begin{frame}[fragile]
    \frametitle{Conclusion on Ethical Implications of AI}
    As we conclude our exploration of the ethical implications surrounding artificial intelligence (AI), 
    it is imperative to synthesize the key concepts and learnings. 
    AI technologies entail a myriad of ethical considerations, from data privacy to algorithmic bias, 
    necessitating a commitment to responsible practices to harness their potential for societal good.
\end{frame}

\begin{frame}[fragile]
    \frametitle{Key Takeaways - Part 1}
    \begin{enumerate}
        \item \textbf{Understanding AI Ethics}:
        \begin{itemize}
            \item AI ethics encompasses the moral principles guiding AI development and application.
            \item Emphasizes the importance of fairness, accountability, transparency, and respect for user privacy.
        \end{itemize}
        
        \item \textbf{Recognizing Bias}:
        \begin{itemize}
            \item AI systems can reflect biases in training data, potentially favoring certain demographics.
            \item Example: An AI hiring model trained on biased historical data may perpetuate inequality.
        \end{itemize}
    \end{enumerate}
\end{frame}

\begin{frame}[fragile]
    \frametitle{Key Takeaways - Part 2}
    \begin{enumerate}
        \setcounter{enumi}{2}
        
        \item \textbf{Data Privacy and Consent}:
        \begin{itemize}
            \item Significant concerns arise from the collection and use of personal data.
            \item Users must be informed about data usage and provide consent.
            \item Example: The Cambridge Analytica scandal illustrates the dangers of data misuse.
        \end{itemize}
        
        \item \textbf{Accountability in Automation}:
        \begin{itemize}
            \item Determining accountability for AI decisions is crucial as systems become more autonomous.
            \item Example: In autonomous vehicles, clarity is needed on liability in case of accidents.
        \end{itemize}
        
        \item \textbf{Redefining Human-AI Collaboration}:
        \begin{itemize}
            \item Responsible AI practices should focus on augmenting human capabilities, not replacing them.
            \item Example: AI enhances diagnostic accuracy in healthcare while preserving human judgment.
        \end{itemize}
    \end{enumerate}
\end{frame}

\begin{frame}[fragile]
    \frametitle{Key Takeaways - Part 3}
    \begin{enumerate}
        \setcounter{enumi}{5}
        
        \item \textbf{Regulating AI Technologies}:
        \begin{itemize}
            \item A call for regulatory frameworks governing AI development is emerging to ensure ethical standards.
            \item This includes initiatives for enforcing ethical guidelines and promoting diversity in AI teams.
        \end{itemize}
    \end{enumerate}
    
    \begin{block}{Final Thoughts}
        As AI technology evolves, ethical vigilance is paramount. Stakeholders—technologists, policymakers, 
        and the public—must collaborate to ensure ethical, equitable, 
        and value-aligned AI development and implementation.
    \end{block}
\end{frame}

\begin{frame}[fragile]
    \frametitle{Engagement Exercise}
    Reflect on a recent AI technology you have interacted with. 
    Consider the ethical implications of its use. 
    What measures would you advocate to enhance its ethical standing?
\end{frame}


\end{document}