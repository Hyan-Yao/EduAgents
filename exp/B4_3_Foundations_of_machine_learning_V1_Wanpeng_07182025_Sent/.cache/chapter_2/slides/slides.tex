\documentclass[aspectratio=169]{beamer}

% Theme and Color Setup
\usetheme{Madrid}
\usecolortheme{whale}
\useinnertheme{rectangles}
\useoutertheme{miniframes}

% Additional Packages
\usepackage[utf8]{inputenc}
\usepackage[T1]{fontenc}
\usepackage{graphicx}
\usepackage{booktabs}
\usepackage{listings}
\usepackage{amsmath}
\usepackage{amssymb}
\usepackage{xcolor}
\usepackage{tikz}
\usepackage{pgfplots}
\pgfplotsset{compat=1.18}
\usetikzlibrary{positioning}
\usepackage{hyperref}

% Custom Colors
\definecolor{myblue}{RGB}{31, 73, 125}
\definecolor{mygray}{RGB}{100, 100, 100}
\definecolor{mygreen}{RGB}{0, 128, 0}
\definecolor{myorange}{RGB}{230, 126, 34}
\definecolor{mycodebackground}{RGB}{245, 245, 245}

% Set Theme Colors
\setbeamercolor{structure}{fg=myblue}
\setbeamercolor{frametitle}{fg=white, bg=myblue}
\setbeamercolor{title}{fg=myblue}
\setbeamercolor{section in toc}{fg=myblue}
\setbeamercolor{item projected}{fg=white, bg=myblue}
\setbeamercolor{block title}{bg=myblue!20, fg=myblue}
\setbeamercolor{block body}{bg=myblue!10}
\setbeamercolor{alerted text}{fg=myorange}

% Set Fonts
\setbeamerfont{title}{size=\Large, series=\bfseries}
\setbeamerfont{frametitle}{size=\large, series=\bfseries}
\setbeamerfont{caption}{size=\small}
\setbeamerfont{footnote}{size=\tiny}

% Code Listing Style
\lstdefinestyle{customcode}{
  backgroundcolor=\color{mycodebackground},
  basicstyle=\footnotesize\ttfamily,
  breakatwhitespace=false,
  breaklines=true,
  commentstyle=\color{mygreen}\itshape,
  keywordstyle=\color{blue}\bfseries,
  stringstyle=\color{myorange},
  numbers=left,
  numbersep=8pt,
  numberstyle=\tiny\color{mygray},
  frame=single,
  framesep=5pt,
  rulecolor=\color{mygray},
  showspaces=false,
  showstringspaces=false,
  showtabs=false,
  tabsize=2,
  captionpos=b
}
\lstset{style=customcode}

% Document Start
\begin{document}

\frame{\titlepage}

\begin{frame}[fragile]
    \titlepage
\end{frame}

\begin{frame}[fragile]
    \frametitle{What is Supervised Learning?}
    \begin{itemize}
        \item \textbf{Definition}: A type of machine learning where a model is trained on a labeled dataset, pairing input data with correct outputs.
        \item \textbf{Objective}: Learn a function that accurately predicts outputs given new inputs.
    \end{itemize}
\end{frame}

\begin{frame}[fragile]
    \frametitle{Classification in Supervised Learning}
    \begin{itemize}
        \item \textbf{Definition}: A task where the output variable is a category. The model predicts a discrete label based on input features.
        \item \textbf{Common Use Cases}:
            \begin{itemize}
                \item Email filtering (spam vs. not spam)
                \item Medical diagnosis (disease vs. no disease)
                \item Sentiment analysis (positive, negative, neutral)
            \end{itemize}
    \end{itemize}
\end{frame}

\begin{frame}[fragile]
    \frametitle{Key Concepts in Classification}
    \begin{itemize}
        \item \textbf{Training Data}: Dataset with features (inputs) and corresponding labels (outputs).
        \item \textbf{Features}: Input variables for predictions (e.g., keywords, sender).
        \item \textbf{Labels}: Target variable to predict (e.g., "spam" or "not spam").
    \end{itemize}
\end{frame}

\begin{frame}[fragile]
    \frametitle{Popular Classification Algorithms}
    \begin{enumerate}
        \item \textbf{Logistic Regression}
            \begin{itemize}
                \item Models the probability of a given input belonging to a class.
                \item \textbf{Formula}:
                \begin{equation}
                    P(Y=1|X) = \frac{1}{1 + e^{-(\beta_0 + \beta_1X_1 + \ldots + \beta_nX_n)}}
                \end{equation}
                \item \textbf{Application}: Binary outcomes.
            \end{itemize}
        
        \item \textbf{Support Vector Machines (SVM)}
            \begin{itemize}
                \item Finds the optimal hyperplane maximizing the margin between classes.
            \end{itemize}
        
        \item \textbf{Decision Trees}
            \begin{itemize}
                \item A tree structure representing decisions based on feature values.
            \end{itemize}
        
        \item \textbf{Random Forest}
            \begin{itemize}
                \item Ensemble of decision trees, enhances accuracy and reduces overfitting.
            \end{itemize}
    \end{enumerate}
\end{frame}

\begin{frame}[fragile]
    \frametitle{Evaluation Metrics for Classification}
    \begin{itemize}
        \item \textbf{Accuracy}: Proportion of correctly predicted instances.
        \item \textbf{Precision}: True positives / (True positives + False positives).
        \item \textbf{Recall (Sensitivity)}: True positives / (True positives + False negatives).
        \item \textbf{F1 Score}: Harmonic mean of precision and recall.
    \end{itemize}
\end{frame}

\begin{frame}[fragile]
    \frametitle{Key Points to Remember}
    \begin{itemize}
        \item Supervised learning requires labeled data for training.
        \item Classification is a fundamental task with discrete labels.
        \item Familiarize yourself with various algorithms and applications.
        \item Evaluating model performance ensures reliability.
    \end{itemize}
    Understanding these concepts paves the way for deeper exploration into classification algorithms and applications.
\end{frame}

\begin{frame}[fragile]
    \frametitle{Decision Trees - Overview}
    \begin{block}{Definition}
        A decision tree is a flowchart-like structure used for classification and regression tasks in supervised learning. It represents decisions based on a series of rules derived from the training data.
    \end{block}
\end{frame}

\begin{frame}[fragile]
    \frametitle{Decision Trees - Structure}
    \begin{itemize}
        \item \textbf{Root Node}: The top node that represents the entire dataset.
        \item \textbf{Internal Nodes}: Represent features or attributes, acting as decision points.
        \item \textbf{Leaf Nodes}: Terminal nodes that represent outcomes or class labels.
    \end{itemize}
    
    \begin{block}{Diagram Representation}
        % Representation of a simple tree
        \begin{center}
            \texttt{
                [Root Node] \\
                /      \textbackslash \\
            [Internal]  [Internal] \\
                |          | \\
            [Leaf]     [Leaf]
            }
        \end{center}
    \end{block}
\end{frame}

\begin{frame}[fragile]
    \frametitle{Decision Trees - How They Work}
    \begin{enumerate}
        \item \textbf{Splitting}: The dataset is split recursively to improve node purity.
        \item \textbf{Selecting the Best Split}:
            \begin{itemize}
                \item \textbf{Gini Impurity}: Measures the likelihood an element is incorrectly labeled.
                \item \textbf{Entropy}: Measures randomness; reducing entropy after a split is the goal.
            \end{itemize}
    \end{enumerate}

    \begin{block}{Formulas}
        \begin{itemize}
            \item \textbf{Gini Impurity}: 
            \begin{equation}
                Gini(D) = 1 - \sum_{i=1}^{C} (p_i)^2 
            \end{equation}
            \item \textbf{Entropy}: 
            \begin{equation}
                Entropy(D) = -\sum_{i=1}^{C} p_i \log_2(p_i) 
            \end{equation}
        \end{itemize}
    \end{block}
\end{frame}

\begin{frame}[fragile]
    \frametitle{Decision Trees - Advantages}
    \begin{itemize}
        \item \textbf{Easy to Understand and Interpret}: Visual representation is intuitive.
        \item \textbf{No Need for Data Normalization}: No feature scaling required.
        \item \textbf{Handles Both Numerical and Categorical Data}: Can manage various data types seamlessly.
        \item \textbf{Handles Missing Values}: Naturally manages missing data during tree construction.
    \end{itemize}
\end{frame}

\begin{frame}[fragile]
    \frametitle{Decision Trees - Disadvantages}
    \begin{itemize}
        \item \textbf{Overfitting}: Risk of overly complex trees that do not generalize well.
        \item \textbf{Instability}: Small data variations can greatly change the tree structure.
        \item \textbf{Bias Towards Dominant Classes}: May skew predictions if one class is overly dominant.
    \end{itemize}
\end{frame}

\begin{frame}[fragile]
    \frametitle{Key Takeaways and Next Steps}
    \begin{itemize}
        \item Decision trees clarify the decision-making process in supervised learning.
        \item Understanding their structure and functionality is crucial for model building.
        \item Awareness of their limitations aids in improving performance (e.g., pruning, ensemble methods).
    \end{itemize}
    
    \begin{block}{Next Steps}
        In the upcoming slide, we will explore how to implement decision trees using Python and Scikit-learn, including practical code snippets and examples.
    \end{block}
\end{frame}

\begin{frame}[fragile]
    \frametitle{Implementing Decision Trees - Overview}
    \begin{block}{Overview}
        Implementing a decision tree in Python using the Scikit-learn library involves several straightforward steps, including data preparation, model training, and evaluation.
    \end{block}
    
    \begin{block}{Key Points}
        \begin{itemize}
            \item Data Preparation is Crucial
            \item Choosing Parameters to avoid overfitting
            \item Model Evaluation is necessary for performance assessment
        \end{itemize}
    \end{block}
\end{frame}

\begin{frame}[fragile]
    \frametitle{Implementing Decision Trees - Steps 1-4}
    \begin{enumerate}
        \item \textbf{Import Necessary Libraries}
        \begin{lstlisting}
import pandas as pd
from sklearn.model_selection import train_test_split
from sklearn.tree import DecisionTreeClassifier
from sklearn import metrics
        \end{lstlisting}

        \item \textbf{Load the Dataset}
        \begin{lstlisting}
data = pd.read_csv('your_dataset.csv')
        \end{lstlisting}

        \item \textbf{Prepare the Data}
        \begin{lstlisting}
X = data.drop('target', axis=1)  # features
y = data['target']                # labels
        \end{lstlisting}

        \item \textbf{Split the Data}
        \begin{lstlisting}
X_train, X_test, y_train, y_test = train_test_split(X, y, test_size=0.3, random_state=42)
        \end{lstlisting}
    \end{enumerate}
\end{frame}

\begin{frame}[fragile]
    \frametitle{Implementing Decision Trees - Steps 5-8}
    \begin{enumerate}
        \setcounter{enumi}{4} % Continue numbering from the previous frame
        \item \textbf{Initialize the Decision Tree Classifier}
        \begin{lstlisting}
classifier = DecisionTreeClassifier(random_state=42)
        \end{lstlisting}

        \item \textbf{Train the Model}
        \begin{lstlisting}
classifier.fit(X_train, y_train)
        \end{lstlisting}

        \item \textbf{Make Predictions}
        \begin{lstlisting}
y_pred = classifier.predict(X_test)
        \end{lstlisting}

        \item \textbf{Evaluate the Model}
        \begin{lstlisting}
accuracy = metrics.accuracy_score(y_test, y_pred)
confusion_matrix = metrics.confusion_matrix(y_test, y_pred)
print("Accuracy:", accuracy)
print("Confusion Matrix:\n", confusion_matrix)
        \end{lstlisting}
    \end{enumerate}
\end{frame}

\begin{frame}[fragile]
    \frametitle{Evaluating Decision Trees - Introduction}
    \begin{block}{Introduction to Evaluation Metrics}
        When implementing decision trees, it is vital to evaluate their performance to ensure they make accurate predictions. Evaluation metrics allow us to quantify how well our model performs and identify areas for improvement. Key metrics include:
        \begin{itemize}
            \item Accuracy
            \item Precision
            \item Recall
        \end{itemize}
    \end{block}
\end{frame}

\begin{frame}[fragile]
    \frametitle{Evaluating Decision Trees - Accuracy}
    \begin{block}{1. Accuracy}
        \begin{itemize}
            \item \textbf{Definition}: Measures the proportion of correct predictions (both true positives and true negatives) made by the model out of all predictions.
            \item \textbf{Formula}:
            \begin{equation}
            \text{Accuracy} = \frac{\text{True Positives} + \text{True Negatives}}{\text{Total Predictions}}
            \end{equation}
            \item \textbf{Example}: If a decision tree correctly classifies 80 out of 100 instances:
            \begin{equation}
            \text{Accuracy} = \frac{80}{100} = 0.8 \, (or \, 80\%)
            \end{equation}
            \item \textbf{Key Point}: While accuracy provides a quick overview, it may not be sufficient in cases of class imbalance.
        \end{itemize}
    \end{block}
\end{frame}

\begin{frame}[fragile]
    \frametitle{Evaluating Decision Trees - Precision and Recall}
    \begin{block}{2. Precision}
        \begin{itemize}
            \item \textbf{Definition}: Measures the accuracy of the positive predictions made; quantifies how many predicted positive cases were actually positive.
            \item \textbf{Formula}:
            \begin{equation}
            \text{Precision} = \frac{\text{True Positives}}{\text{True Positives} + \text{False Positives}}
            \end{equation}
            \item \textbf{Example}: If a model predicts 50 instances as positive but only 30 are truly positive:
            \begin{equation}
            \text{Precision} = \frac{30}{30 + 20} = \frac{30}{50} = 0.6 \, (or \, 60\%)
            \end{equation}
            \item \textbf{Key Point}: High precision indicates a low false positive rate, crucial in applications like fraud detection or medical diagnosis.
        \end{itemize}
    \end{block}

    \begin{block}{3. Recall (Sensitivity)}
        \begin{itemize}
            \item \textbf{Definition}: Measures the ability of the model to identify all relevant instances; reflects the true positive rate.
            \item \textbf{Formula}:
            \begin{equation}
            \text{Recall} = \frac{\text{True Positives}}{\text{True Positives} + \text{False Negatives}}
            \end{equation}
            \item \textbf{Example}: If there are 40 actual positive instances, and the model identifies 30 of them correctly:
            \begin{equation}
            \text{Recall} = \frac{30}{30 + 10} = \frac{30}{40} = 0.75 \, (or \, 75\%)
            \end{equation}
            \item \textbf{Key Point}: Recall is crucial in contexts where missing a positive instance has significant consequences, such as disease screenings.
        \end{itemize}
    \end{block}
\end{frame}

\begin{frame}[fragile]
    \frametitle{Evaluating Decision Trees - Confusion Matrix and Conclusion}
    \begin{block}{Visual Insight: Confusion Matrix}
        To visualize these metrics, we can use a confusion matrix, which summarizes the results of classifications:
        \begin{table}[]
        \centering
        \begin{tabular}{|c|c|c|}
        \hline
                          & Predicted Positive & Predicted Negative \\ \hline
        Actual Positive    & True Positives (TP) & False Negatives (FN) \\ \hline
        Actual Negative    & False Positives (FP) & True Negatives (TN)  \\ \hline
        \end{tabular}
        \end{table}
    \end{block}

    \begin{block}{Conclusion}
        Evaluating decision trees using these metrics provides insight into model effectiveness and informs improvements. By balancing accuracy, precision, and recall, we can tailor models suited for specific applications.
    \end{block}
\end{frame}

\begin{frame}[fragile]
    \frametitle{Evaluating Decision Trees - Code Snippet}
    \begin{block}{Code Snippet (Python with Scikit-learn)}
        Here’s a simple code example to compute accuracy, precision, and recall using Scikit-learn:
        \begin{lstlisting}[language=Python]
from sklearn.metrics import accuracy_score, precision_score, recall_score

# True labels and predicted labels
y_true = [1, 0, 1, 1, 0, 1, 0]
y_pred = [1, 0, 1, 0, 0, 1, 1]

# Calculating metrics
accuracy = accuracy_score(y_true, y_pred)
precision = precision_score(y_true, y_pred)
recall = recall_score(y_true, y_pred)

print(f'Accuracy: {accuracy:.2f}')
print(f'Precision: {precision:.2f}')
print(f'Recall: {recall:.2f}')
        \end{lstlisting}
    \end{block}
\end{frame}

\begin{frame}[fragile]
    \frametitle{Key Takeaway}
    Understanding and applying accuracy, precision, and recall is essential for effectively evaluating decision tree models in classification tasks. Evaluating these metrics allows us to enhance the model's capabilities for real-world applications.
\end{frame}

\begin{frame}[fragile]
  \frametitle{Logistic Regression - Overview}
  \begin{block}{What is Logistic Regression?}
    Logistic Regression is a statistical method used for binary classification tasks—where the outcome variable is categorical with two possible outcomes (0 and 1).
  \end{block}
  \begin{itemize}
    \item Unlike linear regression, which predicts continuous values, logistic regression predicts the probability of an event occurring.
    \item It is utilized in various fields such as healthcare, finance, and marketing.
  \end{itemize}
\end{frame}

\begin{frame}[fragile]
  \frametitle{Logistic Regression - How It Works}
  \begin{block}{Logit Function}
    Logistic regression uses the logistic function to model the probability of the binary outcome:
    \begin{equation}
      P(Y=1|X) = \frac{1}{1 + e^{-(\beta_0 + \beta_1 X_1 + \beta_2 X_2 + ... + \beta_n X_n)}}
    \end{equation}
    where:
    \begin{itemize}
      \item \(P(Y=1|X)\) is the predicted probability,
      \item \(\beta_0\) is the intercept,
      \item \(\beta_1, \beta_2, ..., \beta_n\) are the coefficients for the independent variables.
    \end{itemize}
  \end{block}
  
  \begin{block}{Decision Boundary}
    Logistic regression models the log-odds, allowing us to find a decision boundary that separates the classes.
  \end{block}
\end{frame}

\begin{frame}[fragile]
  \frametitle{Logistic Regression - Example and Applications}
  \begin{block}{Example}
    Consider predicting whether a student will pass (1) or fail (0) based on hours studied:
    \begin{itemize}
      \item Independent Variable: Hours Studied (X)
      \item Dependent Variable: Pass/Fail (Y)
      \item Model: \(Logit(P) = -2 + 0.5X\)
    \end{itemize}
    For a student studying 6 hours:
    \begin{equation}
      P(Y=1|X=6) = \frac{1}{1 + e^{-(-2 + 0.5 \times 6)}} \approx 0.88
    \end{equation}
    This indicates an 88\% probability of passing.
  \end{block}

  \begin{block}{Key Applications}
    \begin{itemize}
      \item Healthcare: Predicting likelihood of disease occurrence.
      \item Finance: Credit scoring for evaluating default risk.
      \item Marketing: Predicting customer response to campaigns.
    \end{itemize}
  \end{block}
\end{frame}

\begin{frame}[fragile]
    \frametitle{Implementing Logistic Regression}
    
    \begin{block}{What is Logistic Regression?}
        Logistic regression is a statistical method for binary classification. It predicts one of two possible outcomes (e.g., success/failure).
    \end{block}
    
    \begin{block}{Key Components}
        \begin{itemize}
            \item \textbf{Sigmoid Function}: Converts linear output to a probability between 0 and 1.
            \item \textbf{Cost Function}: Uses log loss to quantify model prediction performance.
        \end{itemize}
    \end{block}
\end{frame}

\begin{frame}[fragile]
    \frametitle{Implementing Logistic Regression - Python Example}

    \textbf{1. Import Necessary Libraries}
    \begin{lstlisting}[language=Python]
import numpy as np
import pandas as pd
import matplotlib.pyplot as plt
from sklearn.model_selection import train_test_split
from sklearn.linear_model import LogisticRegression
from sklearn.metrics import confusion_matrix, accuracy_score
    \end{lstlisting}

    \textbf{2. Load and Prepare Data}
    \begin{lstlisting}[language=Python]
from sklearn.datasets import load_iris
iris = load_iris()
X = iris.data[:100, :2]  # First 100 samples, first two features
y = iris.target[:100]     # Binary target variable

X_train, X_test, y_train, y_test = train_test_split(X, y, test_size=0.2, random_state=42)
    \end{lstlisting}
\end{frame}

\begin{frame}[fragile]
    \frametitle{Implementing Logistic Regression - Training and Evaluation}

    \textbf{3. Instantiate and Train the Model}
    \begin{lstlisting}[language=Python]
model = LogisticRegression()
model.fit(X_train, y_train)
    \end{lstlisting}

    \textbf{4. Make Predictions}
    \begin{lstlisting}[language=Python]
y_pred = model.predict(X_test)
print("Predictions:", y_pred)
    \end{lstlisting}

    \textbf{5. Evaluate the Model}
    \begin{lstlisting}[language=Python]
accuracy = accuracy_score(y_test, y_pred)
print("Accuracy:", accuracy)
cm = confusion_matrix(y_test, y_pred)
print("Confusion Matrix:\n", cm)
    \end{lstlisting}
  
    \begin{block}{Key Points}
        \begin{itemize}
            \item Logistic Regression is foundational for binary classification due to interpretability.
            \item The Sigmoid function transforms outputs into probabilities.
            \item Scikit-learn simplifies implementation and evaluation.
        \end{itemize}
    \end{block}
\end{frame}

\begin{frame}
    \frametitle{Evaluating Logistic Regression}
    \begin{block}{Overview of Evaluation Metrics}
        In supervised learning, particularly for classification tasks, evaluating model performance is critical. This slide focuses on two essential metrics for logistic regression: 
        \begin{itemize}
            \item Confusion Matrix
            \item AUC-ROC Curve
        \end{itemize}
    \end{block}
\end{frame}

\begin{frame}
    \frametitle{Confusion Matrix}
    A confusion matrix visually represents the performance of a classification model by comparing actual versus predicted classifications. It consists of four components:
    \begin{itemize}
        \item \textbf{True Positives (TP)}: Correctly predicted positive cases.
        \item \textbf{True Negatives (TN)}: Correctly predicted negative cases.
        \item \textbf{False Positives (FP)}: Incorrectly predicted positive cases (Type I Error).
        \item \textbf{False Negatives (FN)}: Incorrectly predicted negative cases (Type II Error).
    \end{itemize}

    \begin{block}{Confusion Matrix Structure}
    \begin{tabular}{|c|c|c|}
    \hline
                & Predicted Positive & Predicted Negative \\
    \hline
    Actual Positive & TP                  & FN                 \\
    \hline
    Actual Negative & FP                  & TN                 \\
    \hline
    \end{tabular}
    \end{block}
\end{frame}

\begin{frame}
    \frametitle{Key Metrics from Confusion Matrix}
    Key metrics derived from the confusion matrix include:
    
    \begin{itemize}
        \item \textbf{Accuracy:} 
        \begin{equation}
        \text{Accuracy} = \frac{TP + TN}{TP + TN + FP + FN}
        \end{equation}
        \item \textbf{Precision:} 
        \begin{equation}
        \text{Precision} = \frac{TP}{TP + FP}
        \end{equation}
        \item \textbf{Recall (Sensitivity):} 
        \begin{equation}
        \text{Recall} = \frac{TP}{TP + FN}
        \end{equation}
        \item \textbf{F1 Score:} 
        \begin{equation}
        F1 = 2 \cdot \frac{\text{Precision} \times \text{Recall}}{\text{Precision} + \text{Recall}}
        \end{equation}
    \end{itemize}
\end{frame}

\begin{frame}
    \frametitle{AUC-ROC Curve}
    The ROC (Receiver Operating Characteristic) Curve illustrates how the true positive rate (Recall) varies with the false positive rate across different thresholds. The \textbf{Area Under the Curve (AUC)} quantifies the model's ability to discriminate between classes.

    \begin{block}{AUC Value Interpretation}
        \begin{itemize}
            \item 0.5: No discrimination capability (like random guessing).
            \item 0.7 - 0.8: Reasonable ability to distinguish classes.
            \item 0.8 - 0.9: Strong model.
            \item Above 0.9: Excellent model.
        \end{itemize}
    \end{block}
\end{frame}

\begin{frame}[fragile]
    \frametitle{Implementation Example}
    Here's a code snippet for evaluating a logistic regression model:
    
    \begin{lstlisting}[language=Python]
from sklearn.metrics import confusion_matrix, RocCurveDisplay, accuracy_score
import matplotlib.pyplot as plt

# Assuming y_true and y_pred are your actual and predicted labels
cm = confusion_matrix(y_true, y_pred)
print('Confusion Matrix:\n', cm)

# AUC-ROC
RocCurveDisplay.from_predictions(y_true, y_scores)  # y_scores are the probabilities from the model
plt.show()

# Accuracy Calculation
accuracy = accuracy_score(y_true, y_pred)
print('Accuracy:', accuracy)
    \end{lstlisting}
\end{frame}

\begin{frame}
    \frametitle{Summary}
    By understanding these evaluation metrics, you can better assess the effectiveness of your logistic regression model and make informed decisions regarding model improvements:
    \begin{itemize}
        \item Confusion matrix and AUC-ROC are crucial for understanding model performance.
        \item Accuracy can be misleading, especially in imbalanced datasets.
        \item AUC-ROC provides a comprehensive view across various thresholds.
    \end{itemize}
\end{frame}

\begin{frame}[fragile]
    \frametitle{K-Nearest Neighbors (KNN) - Overview}
    \begin{block}{Overview}
        K-Nearest Neighbors (KNN) is a simple, intuitively appealing algorithm used in supervised learning for classification tasks.
        It operates on the principle of finding the "k" closest data points (neighbors) in the feature space and classifies a data point based on the majority class among those neighbors.
    \end{block}
\end{frame}

\begin{frame}[fragile]
    \frametitle{KNN - How It Works}
    \begin{enumerate}
        \item \textbf{Distance Calculation:}
        \begin{itemize}
            \item KNN employs distance metrics (like Euclidean, Manhattan, or Minkowski) to determine how “close” data points are.
            \item \textbf{Euclidean Distance Formula:}
            \begin{equation}
            d = \sqrt{(x_2 - x_1)^2 + (y_2 - y_1)^2}
            \end{equation}
        \end{itemize}
        
        \item \textbf{Selecting 'k':}
        \begin{itemize}
            \item 'k' is a predefined constant, indicating how many neighbors will be considered.
            \item It is common to use odd values for k to prevent ties in classification.
        \end{itemize}

        \item \textbf{Majority Voting:}
        \begin{itemize}
            \item KNN assigns the class label that is most frequent among the ‘k’ neighbors to the new data point.
        \end{itemize}
    \end{enumerate}
\end{frame}

\begin{frame}[fragile]
    \frametitle{KNN - Advantages and Scenarios for Use}
    \begin{block}{Advantages of KNN}
        \begin{itemize}
            \item \textbf{Simplicity:} Easy to understand and implement.
            \item \textbf{No Training Phase:} KNN is a lazy learner, thus no training is required.
            \item \textbf{Robustness:} Effective for small to medium-sized datasets and multi-class classification.
        \end{itemize}
    \end{block}

    \begin{block}{Scenarios for Use}
        \begin{itemize}
            \item \textbf{Recommendation Systems:} Identifying similar products.
            \item \textbf{Image Recognition:} Classifying images based on visual similarities.
            \item \textbf{Medical Diagnosis:} Classifying patient data based on symptoms to predict diseases.
        \end{itemize}
    \end{block}
\end{frame}

\begin{frame}[fragile]
    \frametitle{KNN - Key Points and Example}
    \begin{block}{Key Points}
        \begin{itemize}
            \item The effectiveness of KNN relies on the choice of ‘k’ and the distance metric used.
            \item It can be computationally expensive with large datasets due to distance calculations.
            \item Normalization of data may be necessary, especially with features on different scales.
        \end{itemize}
    \end{block}

    \begin{block}{Example Illustration}
        Consider a scatter plot where different classes (e.g., circles and squares) are represented. When a new data point (e.g., a triangle) is introduced, KNN will find the nearest neighbors and classify the triangle based on the majority class of those neighbors.
    \end{block}
\end{frame}

\begin{frame}[fragile]
    \frametitle{Implementing KNN - Overview}
    \begin{block}{Overview}
        K-Nearest Neighbors (KNN) is a simple yet powerful classification algorithm that makes predictions based on the distances between data points. The primary steps for implementing KNN involve:
    \end{block}
    \begin{itemize}
        \item Data preparation
        \item Splitting the dataset
        \item Creating a KNN model
        \item Making predictions
    \end{itemize}
\end{frame}

\begin{frame}[fragile]
    \frametitle{Implementing KNN - Step-by-Step Implementation}
    \begin{enumerate}
        \item \textbf{Import Necessary Libraries}
        \begin{lstlisting}[language=Python]
import numpy as np
import pandas as pd
from sklearn.model_selection import train_test_split
from sklearn.neighbors import KNeighborsClassifier
from sklearn.metrics import accuracy_score, classification_report, confusion_matrix
        \end{lstlisting}
    
        \item \textbf{Load and Prepare the Dataset}
        \begin{lstlisting}[language=Python]
from sklearn.datasets import load_iris

iris = load_iris()
X = iris.data  # features
y = iris.target  # target classes
        \end{lstlisting}
    
        \item \textbf{Split the Data into Training and Testing Sets}
        \begin{lstlisting}[language=Python]
X_train, X_test, y_train, y_test = train_test_split(X, y, test_size=0.2, random_state=42)
        \end{lstlisting}
    \end{enumerate}
\end{frame}

\begin{frame}[fragile]
    \frametitle{Implementing KNN - Model Creation and Evaluation}
    \begin{enumerate}[resume]
        \item \textbf{Create the KNN Model}
        \begin{lstlisting}[language=Python]
knn = KNeighborsClassifier(n_neighbors=3)
        \end{lstlisting}

        \item \textbf{Fit the Model to the Training Data}
        \begin{lstlisting}[language=Python]
knn.fit(X_train, y_train)
        \end{lstlisting}

        \item \textbf{Make Predictions}
        \begin{lstlisting}[language=Python]
y_pred = knn.predict(X_test)
        \end{lstlisting}

        \item \textbf{Evaluate the Model Performance}
        \begin{lstlisting}[language=Python]
accuracy = accuracy_score(y_test, y_pred)
print(f'Accuracy: {accuracy:.2f}')
print('Classification Report:\n', classification_report(y_test, y_pred))
print('Confusion Matrix:\n', confusion_matrix(y_test, y_pred))
        \end{lstlisting}

        \item \textbf{Key Points}
        \begin{itemize}
            \item Choosing K: Impacts model performance
            \item Distance Metrics: Determines closest neighbors
            \item Scikit-learn: Simplifies KNN implementation
        \end{itemize}
    \end{enumerate}
\end{frame}

\begin{frame}[fragile]
  \frametitle{Evaluating KNN - Overview}
  \begin{block}{Overview of KNN Evaluation}
    K-Nearest Neighbors (KNN) is a powerful classification algorithm. Its performance heavily relies on:
    \begin{itemize}
      \item The choice of neighbors (the parameter `K`)
      \item The distance metric used to determine proximity between instances
    \end{itemize}
  \end{block}
\end{frame}

\begin{frame}[fragile]
  \frametitle{Evaluating KNN - Metrics}
  \begin{block}{Metrics for Evaluating KNN Performance}
    The effectiveness of a KNN model is assessed using various evaluation metrics:
    
    \begin{itemize}
      \item \textbf{Accuracy:}
        \begin{equation}
        \text{Accuracy} = \frac{\text{Number of Correct Predictions}}{\text{Total Predictions}} \times 100
        \end{equation}
        Example: If out of 100 test instances, 85 are correctly classified, accuracy is 85\%.

      \item \textbf{Precision, Recall, and F1-Score:}
        \begin{equation}
        \text{Precision} = \frac{\text{True Positives}}{\text{True Positives} + \text{False Positives}}
        \end{equation}
        \begin{equation}
        \text{Recall} = \frac{\text{True Positives}}{\text{True Positives} + \text{False Negatives}}
        \end{equation}
        \begin{equation}
        F1 = 2 \times \frac{\text{Precision} \times \text{Recall}}{\text{Precision} + \text{Recall}}
        \end{equation}
        Example: In medical diagnostics, high precision ensures fewer false positives while high recall identifies most actual sick patients.
    \end{itemize}
  \end{block}
\end{frame}

\begin{frame}[fragile]
  \frametitle{Evaluating KNN - Neighbor Selection and Metrics}
  \begin{block}{Neighbor Selection (`K`)}
    \begin{itemize}
      \item \textbf{Choosing the Right K:}
        \begin{itemize}
          \item A small K leads to high variance (overfitting)
          \item A large K can introduce bias (underfitting)
          \item \textit{Rule of Thumb:} Start with \( K = \sqrt{N} \) where \( N \) is the number of training samples.
        \end{itemize}
        
      \item \textbf{Cross-Validation:}
        Use techniques like k-fold cross-validation to systematically determine the optimal K by assessing performance across different folds of the dataset.
    \end{itemize}
  \end{block}
  
  \begin{block}{Distance Metrics}
    KNN is sensitive to the distance metric used. Common distance metrics include:

    \begin{itemize}
      \item \textbf{Euclidean Distance:}
        \begin{equation}
        d(p, q) = \sqrt{\sum_{i=1}^{n}(p_i - q_i)^2}
        \end{equation}

      \item \textbf{Manhattan Distance:}
        \begin{equation}
        d(p, q) = \sum_{i=1}^{n}|p_i - q_i|
        \end{equation}

      \item \textbf{Minkowski Distance:}
        \begin{equation}
        d(p, q) = \left(\sum_{i=1}^{n} |p_i - q_i|^m \right)^{1/m}
        \end{equation}
    \end{itemize}
  \end{block}
\end{frame}

\begin{frame}[fragile]
  \frametitle{Evaluating KNN - Key Points and Example}
  \begin{block}{Key Points to Emphasize}
    \begin{itemize}
      \item \textbf{Impact of K:} Finding the optimal number of neighbors is crucial for model balance.
      \item \textbf{Distance Sensitivity:} The choice of distance metric can drastically change classification results.
      \item \textbf{Evaluation Importance:} Employing various metrics gives a holistic view of model performance beyond mere accuracy.
    \end{itemize}
  \end{block}

  \begin{block}{Practical Example (Python Code Snippet)}
  \begin{lstlisting}[language=Python]
from sklearn.neighbors import KNeighborsClassifier
from sklearn.metrics import accuracy_score, classification_report

# Assuming X_train, X_test, y_train, y_test are predefined
knn = KNeighborsClassifier(n_neighbors=5)  # Selecting K=5
knn.fit(X_train, y_train)
y_pred = knn.predict(X_test)

# Evaluation
print("Accuracy:", accuracy_score(y_test, y_pred))
print("Classification Report:\n", classification_report(y_test, y_pred))
  \end{lstlisting}
  \end{block}
\end{frame}

\begin{frame}[fragile]
    \frametitle{Model Evaluation Techniques}
    \begin{block}{Overview}
        Model evaluation is a critical step in the machine learning pipeline, allowing us to assess the performance of classification algorithms. This slide discusses various evaluation techniques universally applicable to classification models.
    \end{block}
\end{frame}

\begin{frame}[fragile]
    \frametitle{Key Evaluation Metrics - Part 1}
    \begin{enumerate}
        \item \textbf{Accuracy}:
        \begin{itemize}
            \item Definition: The ratio of correctly predicted instances to the total instances.
            \item Formula: 
            \begin{equation}
            \text{Accuracy} = \frac{\text{TP} + \text{TN}}{\text{TP} + \text{TN} + \text{FP} + \text{FN}}
            \end{equation}
            where:
            \begin{itemize}
                \item TP = True Positives
                \item TN = True Negatives
                \item FP = False Positives
                \item FN = False Negatives
            \end{itemize}
            \item Example: If a model correctly classifies 90 out of 100 emails as spam or not, its accuracy is 90\%.
        \end{itemize}

        \item \textbf{Precision}:
        \begin{itemize}
            \item Definition: The ratio of correctly predicted positive observations to the total predicted positives.
            \item Formula: 
            \begin{equation}
            \text{Precision} = \frac{\text{TP}}{\text{TP} + \text{FP}}
            \end{equation}
            \item Example: In a medical diagnosis test, if 30 out of 50 positive tests are correct, Precision = 0.60 (or 60\%).
        \end{itemize}
    \end{enumerate}
\end{frame}

\begin{frame}[fragile]
    \frametitle{Key Evaluation Metrics - Part 2}
    \begin{enumerate}
        \setcounter{enumi}{2} % continue numbering
        \item \textbf{Recall (Sensitivity)}:
        \begin{itemize}
            \item Definition: The ratio of correctly predicted positive observations to all actual positives.
            \item Formula: 
            \begin{equation}
            \text{Recall} = \frac{\text{TP}}{\text{TP} + \text{FN}}
            \end{equation}
            \item Example: If in a test for a disease, 80 out of 100 patients with the disease are identified, Recall = 0.80 (or 80\%).
        \end{itemize}

        \item \textbf{F1 Score}:
        \begin{itemize}
            \item Definition: The harmonic mean of Precision and Recall, useful for imbalanced datasets.
            \item Formula: 
            \begin{equation}
            \text{F1 Score} = 2 \times \frac{\text{Precision} \times \text{Recall}}{\text{Precision} + \text{Recall}}
            \end{equation}
            \item Example: If Precision = 0.75 and Recall = 0.60, then F1 Score ≈ 0.67.
        \end{itemize}
        
        \item \textbf{Confusion Matrix}:
        \begin{itemize}
            \item A 2x2 matrix summarizing the performance of a classification algorithm.
            \item Example layout:
            \begin{verbatim}
            |              | Predicted Negative | Predicted Positive |
            |--------------|-------------------|--------------------|
            | Actual Negative | True Negative (TN) | False Positive (FP) |
            | Actual Positive | False Negative (FN) | True Positive (TP)  |
            \end{verbatim}
        \end{itemize}
    \end{enumerate}
\end{frame}

\begin{frame}[fragile]
    \frametitle{Additional Techniques}
    \begin{itemize}
        \item \textbf{ROC Curve} (Receiver Operating Characteristic):
        \begin{itemize}
            \item A graphical representation of a model’s performance across all classification thresholds.
            \item Key Point: The area under the ROC curve (AUC) indicates how well the model distinguishes between classes.
        \end{itemize}

        \item \textbf{Cross-Validation}:
        \begin{itemize}
            \item A technique to assess how the results of a statistical analysis will generalize to an independent dataset.
            \item Common method: k-Fold Cross-Validation.
            \item Key Point: Helps mitigate overfitting and provides a more robust estimate of model performance.
        \end{itemize}
    \end{itemize}
\end{frame}

\begin{frame}[fragile]
    \frametitle{Key Takeaways}
    Evaluating models through accuracy, precision, recall, F1 score, ROC curve, and confusion matrix provides comprehensive insights into their performance. Choosing the right metric depends on the specific problem (e.g., imbalanced datasets) and the costs associated with false positives and false negatives.

    By using these evaluation techniques consistently, we can ensure that our classification models are not only accurate but also reliable and applicable in real-world scenarios.
\end{frame}

\begin{frame}[fragile]
    \frametitle{Comparison of Algorithms}
    \begin{block}{Title}
        Comparative Analysis of Decision Trees, Logistic Regression, and K-Nearest Neighbors (KNN)
    \end{block}
    \begin{itemize}
        \item Supervised learning algorithms are crucial for classification tasks.
        \item This slide compares three prominent algorithms: 
        Decision Trees, Logistic Regression, and KNN.
    \end{itemize}
\end{frame}

\begin{frame}[fragile]
    \frametitle{1. Decision Trees}
    \begin{block}{Concept}
        Decision Trees split the data into subsets based on feature values, leading to predictions at the leaf nodes.
    \end{block}
    \begin{itemize}
        \item \textbf{Advantages}:
            \begin{itemize}
                \item Easy to interpret with visual representation.
                \item Handles numerical and categorical data well.
                \item Works effectively with non-linear relationships.
            \end{itemize}
        \item \textbf{Disadvantages}:
            \begin{itemize}
                \item Prone to overfitting, especially with noise.
                \item Sensitive to data changes, leading to different structures.
            \end{itemize}
        \item \textbf{Use Cases}: Ideal for interpretability tasks, e.g., credit scoring, medical diagnosis.
    \end{itemize}
\end{frame}

\begin{frame}[fragile]
    \frametitle{2. Logistic Regression}
    \begin{block}{Concept}
        Logistic Regression predicts the probability of a binary outcome using a logistic function.
    \end{block}
    \begin{itemize}
        \item \textbf{Advantages}:
            \begin{itemize}
                \item Efficient for linearly separable classes.
                \item Provides probabilities and feature importance insights.
                \item Requires less data preprocessing.
            \end{itemize}
        \item \textbf{Disadvantages}:
            \begin{itemize}
                \item Assumes linearity between variables.
                \item Ineffective for complex relationships.
            \end{itemize}
        \item \textbf{Use Cases}: Marketing response predictions and scenarios with approximately linear relationships.
        \item \textbf{Formula}:
        \begin{equation}
            P(Y=1|X) = \frac{1}{1 + e^{-(\beta_0 + \beta_1X_1 + \beta_2X_2 + ... + \beta_nX_n)}}
        \end{equation}
    \end{itemize}
\end{frame}

\begin{frame}[fragile]
    \frametitle{3. K-Nearest Neighbors (KNN)}
    \begin{block}{Concept}
        KNN classifies cases based on the majority class among the K nearest neighbors in the feature space.
    \end{block}
    \begin{itemize}
        \item \textbf{Advantages}:
            \begin{itemize}
                \item Simple implementation; instance-based without a training phase.
                \item Handles multi-class cases naturally.
                \item Adapts well to complex datasets.
            \end{itemize}
        \item \textbf{Disadvantages}:
            \begin{itemize}
                \item Computationally expensive with distance calculations.
                \item Sensitive to the choice of K and data scale.
            \end{itemize}
        \item \textbf{Use Cases}: Recommendation systems, image recognition, pattern recognition.
    \end{itemize}
\end{frame}

\begin{frame}[fragile]
    \frametitle{Key Points to Emphasize}
    \begin{itemize}
        \item \textbf{Interpretability}: Decision Trees provide high interpretability, in contrast to KNN and Logistic Regression which may require further explanation.
        \item \textbf{Computational Efficiency}: Logistic Regression generally has faster training and inference than KNN, especially with larger datasets.
        \item \textbf{Data Structure}: Decision Trees can naturally handle missing values; Logistic Regression and KNN need preprocessing.
        \item \textbf{Algorithm Selection}: Choose based on data characteristics, problem complexity, and requirements.
    \end{itemize}
\end{frame}

\begin{frame}[fragile]
    \frametitle{Ethical Considerations in Supervised Learning - Overview}
    \begin{block}{Overview}
        In the context of supervised learning, particularly classification, ethical considerations are paramount to ensure:
        \begin{itemize}
            \item Fairness
            \item Accountability
            \item Transparency
        \end{itemize}
        This involves understanding and mitigating biases that arise during data collection, model training, and deployment.
    \end{block}
\end{frame}

\begin{frame}[fragile]
    \frametitle{Key Concepts - Bias and Fairness}
    \begin{block}{Bias in Data}
        \begin{itemize}
            \item \textbf{Definition}: Systematic errors in data leading to unfair treatment of groups.
            \item \textbf{Example}: A loan approval model trained on biased data rejecting underrepresented applicants.
        \end{itemize}
    \end{block}
    
    \begin{block}{Sources of Bias}
        \begin{itemize}
            \item Historical Bias: Reflects societal inequalities.
            \item Representation Bias: Underrepresented groups in training data.
            \item Measurement Bias: Inaccurate data collection methods.
        \end{itemize}
    \end{block}
\end{frame}

\begin{frame}[fragile]
    \frametitle{Fairness and Real-World Implications}
    \begin{block}{Fairness in Classification}
        \begin{itemize}
            \item Goal: Model predictions must not discriminate based on attributes like race or gender.
            \item Frameworks: 
            \begin{itemize}
                \item Equal Opportunity: Same true positive rates across groups.
                \item Demographic Parity: Equal distribution of predictions among groups.
            \end{itemize}
        \end{itemize}
    \end{block}
    
    \begin{block}{Example Scenario}
        A facial recognition system trained on lighter-skinned individuals shows higher accuracy for that group but misidentifies individuals with darker skin.
    \end{block}

    \begin{block}{Potential Consequences}
        \begin{itemize}
            \item Legal repercussions (e.g., discrimination lawsuits)
            \item Loss of public trust
            \item Ethical obligations to uphold societal values
        \end{itemize}
    \end{block}
\end{frame}

\begin{frame}[fragile]
    \frametitle{Conclusion - Recap of Key Points}
    \begin{enumerate}
        \item \textbf{Definition of Supervised Learning:}
        \begin{itemize}
            \item Training a model on labeled data with input features paired to target outcomes.
            \item Models predict outputs from unseen inputs.
        \end{itemize}
        
        \item \textbf{Classification Overview:}
        \begin{itemize}
            \item Focus on predicting categorical labels.
            \item Algorithms discussed: 
                \begin{itemize}
                    \item Logistic Regression
                    \item Decision Trees
                    \item Support Vector Machines (SVM)
                    \item k-Nearest Neighbors (k-NN)
                \end{itemize}
        \end{itemize}
    \end{enumerate}
\end{frame}

\begin{frame}[fragile]
    \frametitle{Conclusion - Evaluation Metrics}
    \begin{block}{Evaluation Metrics:}
        It is crucial to evaluate a classification model to gauge performance.
        \begin{itemize}
            \item \textbf{Accuracy:} Correct predictions / Total instances.
            \item \textbf{Precision:} True positives / Positive predictions; important when false positives are costly.
            \item \textbf{Recall (Sensitivity):} True positives / Actual positives; critical when missing positives is severe.
            \item \textbf{F1 Score:} Harmonic mean of precision and recall.
            \item \textbf{ROC Curve and AUC:} Visualize trade-offs between true positive and false positive rates.
        \end{itemize}
    \end{block}
\end{frame}

\begin{frame}[fragile]
    \frametitle{Conclusion - Importance of Evaluation}
    \begin{block}{Importance of Model Evaluation:}
        \begin{itemize}
            \item Ensures generalization to new data, preventing overfitting.
            \item Aids in tuning parameters and selecting performing algorithms.
            \item Highlights strengths and limitations, encouraging ethical evaluation.
        \end{itemize}
    \end{block}

    \begin{block}{Key Points to Emphasize:}
        \begin{itemize}
            \item \textbf{Model Selection:} Choose algorithms based on specific problem and evaluation metrics.
            \item \textbf{Bias and Fairness:} Identifying biases during evaluation fosters fairness in predictions.
            \item \textbf{Continuous Improvement:} Evaluation is ongoing; models must be refined with new data.
        \end{itemize}
    \end{block}
    
    \textbf{Example:} Consider using a confusion matrix for a clear view of model performance.
\end{frame}

\begin{frame}
    \frametitle{Q\&A Session}
    \begin{block}{Description}
        Open floor for questions regarding classification algorithms and their implementations.
    \end{block}
\end{frame}

\begin{frame}
    \frametitle{Learning Objectives Recap}
    Before we dive into the Q\&A, let’s briefly summarize what we've learned about classification:
    \begin{enumerate}
        \item \textbf{Supervised Learning}: A type of machine learning where the model is trained on labeled data to predict outcomes.
        \item \textbf{Classification Algorithms}: A variety of algorithms are used, including:
        \begin{itemize}
            \item \textbf{Logistic Regression}: Used for binary outcomes.
            \item \textbf{Decision Trees}: Easy to interpret and visualize.
            \item \textbf{Support Vector Machines}: Effective in high-dimensional spaces.
            \item \textbf{Random Forests}: An ensemble method that reduces overfitting.
            \item \textbf{Neural Networks}: Complex models that can capture non-linear relationships.
        \end{itemize}
        \item \textbf{Model Evaluation}: Importance of metrics such as accuracy, precision, recall, and F1-score to evaluate the performance of classification models.
    \end{enumerate}
\end{frame}

\begin{frame}[fragile]
    \frametitle{Key Concepts for Discussion}
    \begin{itemize}
        \item \textbf{Decision Boundary}: The hypothetical line that separates different classes in a dataset. Understanding how models define this boundary is foundational.
        
        \item \textbf{Overfitting vs. Underfitting}:
        \begin{itemize}
            \item \textit{Overfitting}: Model is too complex and captures noise. Evaluated by high training accuracy but low validation accuracy. 
            \item \textit{Underfitting}: Model is too simple, failing to capture the underlying trend.
        \end{itemize}
        \item \textbf{Hyperparameter Tuning}: Improving model performance through methods like Grid Search or Random Search, adjusting parameters like tree depth or learning rate.
    \end{itemize}
\end{frame}

\begin{frame}[fragile]
    \frametitle{Heuristic Questions to Prompt Participation}
    \begin{itemize}
        \item \textbf{Which classification algorithm do you find most intuitive, and why?}
        \item \textbf{Can anyone describe a real-world application of classification?}
        \item \textbf{What challenges have you faced while implementing a classification algorithm?}
    \end{itemize}
\end{frame}

\begin{frame}[fragile]
    \frametitle{Illustrative Code Snippet: Logistic Regression}
    \begin{lstlisting}[language=Python]
from sklearn.model_selection import train_test_split
from sklearn.linear_model import LogisticRegression
from sklearn.metrics import accuracy_score

# Sample Dataset
X = [[0, 0], [1, 1], [1, 0], [0, 1]]  # Feature set
y = [0, 1, 1, 0]  # Labels (Binary)

# Split the dataset into training and test sets
X_train, X_test, y_train, y_test = train_test_split(X, y, test_size=0.5)

# Create a model
model = LogisticRegression()

# Fit the model
model.fit(X_train, y_train)

# Predict results
predictions = model.predict(X_test)

# Evaluate the accuracy
print("Accuracy:", accuracy_score(y_test, predictions))
    \end{lstlisting}
\end{frame}

\begin{frame}
    \frametitle{Key Points to Emphasize}
    \begin{itemize}
        \item Understanding the \textbf{theoretical basis} of algorithms helps explain their implementation.
        \item Real-world application is critical; consider business problems where accurate classification is vital.
        \item The \textbf{choice of evaluation metric} should align with specific problem objectives (e.g., precision in fraud detection).
    \end{itemize}
\end{frame}

\begin{frame}
    \frametitle{Conclusion}
    This Q\&A session aims to clarify any lingering questions while reinforcing the importance of selecting the right classification algorithm for your data, continuously evaluating performance, and adapting your approach based on feedback and results.

    Feel free to ask any questions based on these concepts, examples, or your own experiences with classification algorithms!
\end{frame}


\end{document}