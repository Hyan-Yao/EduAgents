\documentclass[aspectratio=169]{beamer}

% Theme and Color Setup
\usetheme{Madrid}
\usecolortheme{whale}
\useinnertheme{rectangles}
\useoutertheme{miniframes}

% Additional Packages
\usepackage[utf8]{inputenc}
\usepackage[T1]{fontenc}
\usepackage{graphicx}
\usepackage{booktabs}
\usepackage{listings}
\usepackage{amsmath}
\usepackage{amssymb}
\usepackage{xcolor}
\usepackage{tikz}
\usepackage{pgfplots}
\pgfplotsset{compat=1.18}
\usetikzlibrary{positioning}
\usepackage{hyperref}

% Custom Colors
\definecolor{myblue}{RGB}{31, 73, 125}
\definecolor{mygray}{RGB}{100, 100, 100}
\definecolor{mygreen}{RGB}{0, 128, 0}
\definecolor{myorange}{RGB}{230, 126, 34}
\definecolor{mycodebackground}{RGB}{245, 245, 245}

% Set Theme Colors
\setbeamercolor{structure}{fg=myblue}
\setbeamercolor{frametitle}{fg=white, bg=myblue}
\setbeamercolor{title}{fg=myblue}
\setbeamercolor{section in toc}{fg=myblue}
\setbeamercolor{item projected}{fg=white, bg=myblue}
\setbeamercolor{block title}{bg=myblue!20, fg=myblue}
\setbeamercolor{block body}{bg=myblue!10}
\setbeamercolor{alerted text}{fg=myorange}

% Set Fonts
\setbeamerfont{title}{size=\Large, series=\bfseries}
\setbeamerfont{frametitle}{size=\large, series=\bfseries}
\setbeamerfont{caption}{size=\small}
\setbeamerfont{footnote}{size=\tiny}

% Document Start
\begin{document}

\frame{\titlepage}

\begin{frame}
    \title{Introduction to Data Manipulation in Python}
    \author{Your Name}
    \date{\today}
    \maketitle
\end{frame}

\begin{frame}
    \frametitle{Overview of Data Manipulation in Data Science}

    Data manipulation involves various processes to adjust, transform, and manage data for analysis. It's a foundational aspect of data science, enabling practitioners to derive meaningful insights from raw data.

    \begin{block}{Significance of Data Manipulation}
        \begin{itemize}
            \item \textbf{Data Cleaning:} Removes inaccuracies, null values, and duplicates to ensure data quality.
            \item \textbf{Data Transformation:} Converts data into desired formats (e.g., normalization, scaling).
            \item \textbf{Data Aggregation:} Combines data sets to summarize and extract useful patterns.
        \end{itemize}
    \end{block}
\end{frame}

\begin{frame}
    \frametitle{Examples of Data Manipulation in Practice}

    \begin{itemize}
        \item In a \textbf{sales analysis}, you might manipulate data to focus on specific sales regions or summarize revenue by product category.
        \item In \textbf{healthcare}, data manipulation can help identify patient trends over time by aggregating visit data to monitor health outcomes.
    \end{itemize}
\end{frame}

\begin{frame}
    \frametitle{Introduction to Pandas}

    Pandas is a powerful library in Python, widely used for data manipulation and analysis. It provides easy-to-use data structures and data analysis tools making it accessible for users.

    \begin{block}{Key Features of Pandas}
        \begin{itemize}
            \item \textbf{DataFrames:} Two-dimensional labeled data structures similar to tables in a database or worksheets in Excel.
            \item Operations include:
                \begin{itemize}
                    \item Indexing and selecting data.
                    \item Aggregating data for statistical operations.
                    \item Merging and joining datasets.
                \end{itemize}
        \end{itemize}
    \end{block}
\end{frame}

\begin{frame}[fragile]
    \frametitle{Example Code Snippet}

    Here's a simple example of how to create a DataFrame using Pandas:
    
    \begin{lstlisting}[language=Python]
import pandas as pd

# Create a DataFrame
data = {
    'Name': ['Alice', 'Bob', 'Charlie'],
    'Age': [25, 30, 35],
    'City': ['New York', 'Los Angeles', 'Chicago']
}

df = pd.DataFrame(data)

# Display the DataFrame
print(df)
    \end{lstlisting}
\end{frame}

\begin{frame}
    \frametitle{Example Output}

    \textbf{Output:}
    \begin{verbatim}
          Name  Age         City
    0    Alice   25     New York
    1      Bob   30  Los Angeles
    2  Charlie   35      Chicago
    \end{verbatim}
\end{frame}

\begin{frame}
    \frametitle{Key Points to Emphasize}

    \begin{itemize}
        \item Data manipulation is crucial for obtaining accurate insights from data.
        \item Pandas is the preferred library for data manipulation due to its simplicity and versatility.
        \item Understanding how to use DataFrames is essential for effective data analysis.
    \end{itemize}
\end{frame}

\begin{frame}
    \frametitle{Next Steps}

    In the forthcoming slides, we will delve deeper into the Pandas library, discussing its functionalities and how to leverage it for various data manipulation tasks.
\end{frame}

\begin{frame}[fragile]
    \frametitle{What is Pandas?}
    \begin{block}{Introduction}
        Pandas is an open-source Python library designed for data manipulation and analysis. It provides essential tools for handling structured data, allowing users to easily manage and analyze large datasets effectively.
    \end{block}
\end{frame}

\begin{frame}[fragile]
    \frametitle{Role in Data Manipulation}
    \begin{itemize}
        \item \textbf{Data Structures:}
        \begin{itemize}
            \item \textbf{Series:} A one-dimensional labeled array capable of holding any data type.
            \item \textbf{DataFrame:} A two-dimensional labeled data structure with columns of different types, similar to a spreadsheet or SQL table.
        \end{itemize}
        
        \item \textbf{Data Importing and Exporting:} 
        \begin{itemize}
            \item Read and write data in various formats (CSV, Excel, SQL, JSON) with simple syntax.
        \end{itemize}
    \end{itemize}
\end{frame}

\begin{frame}[fragile]
    \frametitle{Common Use Cases in Data Analysis}
    \begin{itemize}
        \item \textbf{Data Cleaning:} Remove or fill missing values, filter out unwanted data, and correct inconsistencies.
        \item \textbf{Data Transformation:} Apply functions, group data for aggregation, and use pivot tables.
        \item \textbf{Data Visualization:} Integrates well with libraries like Matplotlib for visual representation.
    \end{itemize}
    
    \begin{block}{Example Code}
    \begin{lstlisting}[language=Python]
import pandas as pd

# Creating a DataFrame
data = {
    'Name': ['Alice', 'Bob', 'Charlie'],
    'Age': [24, 30, 22],
    'City': ['New York', 'Los Angeles', 'Chicago']
}
df = pd.DataFrame(data)

# Displaying the DataFrame
print(df)

# Simple data manipulation: Selecting rows where Age > 23
filtered_df = df[df['Age'] > 23]
print(filtered_df)
    \end{lstlisting}
    \end{block}
\end{frame}

\begin{frame}
    \frametitle{Key Features of Pandas}
    \begin{block}{Introduction to Pandas}
        Pandas is a powerful and flexible open-source data analysis tool built on Python, ideal for structured data.
    \end{block}
\end{frame}

\begin{frame}[fragile]
    \frametitle{Data Structures}
    \begin{itemize}
        \item \textbf{Series}:
            \begin{itemize}
                \item A one-dimensional labeled array for any data type.
                \item \textbf{Example}:
                \begin{lstlisting}[language=Python]
import pandas as pd
data = pd.Series([1, 2, 3, 4], index=['a', 'b', 'c', 'd'])
                \end{lstlisting}
                \item \textbf{Key Point}: Think of it as a table column with index labels for easy access.
            \end{itemize}
        
        \item \textbf{DataFrame}:
            \begin{itemize}
                \item A two-dimensional labeled data structure.
                \item \textbf{Example}:
                \begin{lstlisting}[language=Python]
data = {
    'Name': ['Alice', 'Bob', 'Charlie'],
    'Age': [25, 30, 35],
    'Salary': [50000, 60000, 70000]
}
df = pd.DataFrame(data)
                \end{lstlisting}
                \item \textbf{Key Point}: A DataFrame is akin to a complete table or spreadsheet.
            \end{itemize}
    \end{itemize}
\end{frame}

\begin{frame}[fragile]
    \frametitle{Ease of Use}
    \begin{itemize}
        \item \textbf{Intuitive Syntax}:
            \begin{itemize}
                \item User-friendly, concise code.
                \item \textbf{Example}: Filtering rows.
                \begin{lstlisting}[language=Python]
df[df['Age'] > 28]
                \end{lstlisting}
            \end{itemize}
        
        \item \textbf{Data Manipulation}:
            \begin{itemize}
                \item Easily integrates with NumPy for mathematical operations.
                \item \textbf{Example}: Calculate the mean salary.
                \begin{lstlisting}[language=Python]
df['Salary'].mean()
                \end{lstlisting}
            \end{itemize}
    \end{itemize}
\end{frame}

\begin{frame}
    \frametitle{Performance}
    \begin{itemize}
        \item \textbf{Speed}:
            \begin{itemize}
                \item Optimized for efficient memory management and vectorized operations.
            \end{itemize}
        
        \item \textbf{Large Data Handling}:
            \begin{itemize}
                \item Manages datasets larger than memory using chunking and lazy loading.
            \end{itemize}
    \end{itemize}
\end{frame}

\begin{frame}
    \frametitle{Summary and Conclusion}
    \begin{itemize}
        \item Pandas offers flexible and efficient data structures (Series and DataFrames).
        \item Provides an easy-to-use interface and integrates well with other Python libraries.
        \item Designed for performance, enabling handling of large datasets efficiently.
    \end{itemize}
    
    \begin{block}{Conclusion}
        Pandas is essential for data analysis in Python, facilitating efficient data manipulation and analysis.
    \end{block}
\end{frame}

\begin{frame}[fragile]
    \frametitle{Understanding DataFrames}
    \begin{block}{What is a DataFrame?}
        A DataFrame is a powerful two-dimensional data structure provided by the Pandas library in Python, similar to a spreadsheet or SQL table. It consists of rows and columns where each column can hold different types of data (e.g., integers, floats, strings).
    \end{block}
\end{frame}

\begin{frame}[fragile]
    \frametitle{Key Characteristics of a DataFrame}
    \begin{itemize}
        \item \textbf{Labeled Axes:} Each row and column has labels for easy data retrieval.
        \item \textbf{Heterogeneous Data:} Different columns can hold different data types.
        \item \textbf{Size-Mutable:} You can add or drop columns/rows dynamically.
        \item \textbf{Powerful Functions:} Equipped with many built-in functions for analysis and manipulation.
    \end{itemize}
\end{frame}

\begin{frame}[fragile]
    \frametitle{Structure of a DataFrame}
    A DataFrame can be visualized as a table with:
    \begin{itemize}
        \item \textbf{Rows:} Individual records.
        \item \textbf{Columns:} Attributes of the records.
    \end{itemize}
    
    \begin{block}{Example of a DataFrame:}
        \begin{tabular}{|c|c|c|}
            \hline
            \textbf{Name} & \textbf{Age} & \textbf{City} \\
            \hline
            Alice & 30 & New York \\
            \hline
            Bob & 22 & Los Angeles \\
            \hline
            Charlie & 25 & Chicago \\
            \hline
        \end{tabular}
    \end{block}
\end{frame}

\begin{frame}[fragile]
    \frametitle{How DataFrames Differ from Other Data Structures}
    \begin{enumerate}
        \item \textbf{Versus Lists:}
            \begin{itemize}
                \item A list is one-dimensional; a DataFrame is two-dimensional with labeled axes.
                \item Example: A list can store ages: \texttt{[30, 22, 25]}.
            \end{itemize}
        \item \textbf{Versus Dictionaries:}
            \begin{itemize}
                \item A dictionary lacks inherent structure for rows and columns.
                \item Example: \texttt{{"Alice": 30, "Bob": 22}} results in less data manipulation capability.
            \end{itemize}
        \item \textbf{Versus NumPy Arrays:}
            \begin{itemize}
                \item NumPy arrays are homogeneous; DataFrames can hold mixed data types.
                \item Example: A NumPy array \texttt{np.array([[30, 'Alice'], [22, 'Bob']])} is less readable.
            \end{itemize}
    \end{enumerate}
\end{frame}

\begin{frame}[fragile]
    \frametitle{Creating a Simple DataFrame}
    \begin{block}{Code Snippet:}
    \begin{lstlisting}[language=Python]
import pandas as pd

# Sample data
data = {
    "Name": ["Alice", "Bob", "Charlie"],
    "Age": [30, 22, 25],
    "City": ["New York", "Los Angeles", "Chicago"]
}

# Creating a DataFrame
df = pd.DataFrame(data)

# Displaying the DataFrame
print(df)
    \end{lstlisting}
    \end{block}
\end{frame}

\begin{frame}[fragile]
    \frametitle{Conclusion}
    \begin{block}{Key Points}
        \begin{itemize}
            \item DataFrames are essential for data analysis in Python, offering both simplicity and flexibility.
            \item They allow for complex data manipulation, making tasks straightforward.
            \item Understanding DataFrames is crucial for mastering data analysis in Python.
        \end{itemize}
    \end{block}
\end{frame}

\begin{frame}[fragile]
    \frametitle{Creating DataFrames - Introduction}
    \begin{block}{Introduction to DataFrames}
        A DataFrame is a two-dimensional, size-mutable, potentially heterogeneous tabular data structure with labeled axes (rows and columns). 
        It is a core data structure in pandas, making data manipulation in Python efficient and intuitive.
    \end{block}
\end{frame}

\begin{frame}[fragile]
    \frametitle{Creating DataFrames - Methods}
    \begin{enumerate}
        \item \textbf{From a List:}
            \begin{itemize}
                \item DataFrames can be created from a list of lists or a list of dictionaries.
                \item Each inner list represents a row.
            \end{itemize}
            \begin{lstlisting}[language=Python]
import pandas as pd

# List of lists
data = [[1, 'Alice'], [2, 'Bob'], [3, 'Charlie']]
df = pd.DataFrame(data, columns=['ID', 'Name'])
print(df)
            \end{lstlisting}

        \item \textbf{From a Dictionary:}
            \begin{itemize}
                \item Keys represent column labels, values can be lists or NumPy arrays.
            \end{itemize}
            \begin{lstlisting}[language=Python]
data = {
    'ID': [1, 2, 3],
    'Name': ['Alice', 'Bob', 'Charlie']
}
df = pd.DataFrame(data)
print(df)
            \end{lstlisting}
    \end{enumerate}
\end{frame}

\begin{frame}[fragile]
    \frametitle{Creating DataFrames - More Methods}
    \begin{enumerate}
        \setcounter{enumi}{2}
        \item \textbf{From a CSV File:}
            \begin{itemize}
                \item Use \texttt{read\_csv()} to create a DataFrame from a CSV file.
            \end{itemize}
            \begin{lstlisting}[language=Python]
df = pd.read_csv('data.csv')  # Ensure 'data.csv' is in your working directory
print(df.head())  # Display the first 5 rows
            \end{lstlisting}

        \item \textbf{From a Numpy Array:}
            \begin{itemize}
                \item Useful for creating DataFrames from numerical data.
            \end{itemize}
            \begin{lstlisting}[language=Python]
import numpy as np
data = np.array([[1, 2, 3], [4, 5, 6]])
df = pd.DataFrame(data, columns=['A', 'B', 'C'])
print(df)
            \end{lstlisting}

        \item \textbf{From JSON:}
            \begin{itemize}
                \item Construct DataFrames from JSON data using \texttt{read\_json()}.
            \end{itemize}
            \begin{lstlisting}[language=Python]
df = pd.read_json('data.json')  # Load data from a JSON file
print(df)
            \end{lstlisting}
    \end{enumerate}
\end{frame}

\begin{frame}[fragile]
    \frametitle{Key Points and Conclusion}
    \begin{block}{Key Points}
        \begin{itemize}
            \item DataFrames are versatile and can be created from multiple data formats (lists, dictionaries, CSVs, JSON).
            \item The structure of the input data determines how the DataFrame is constructed.
            \item They are essential for data manipulation and analysis in data science projects.
        \end{itemize}
    \end{block}

    \begin{block}{Conclusion}
        Understanding how to create DataFrames from various data sources is the first step towards efficient data manipulation in Python. 
        Mastery of these techniques will facilitate advanced data analysis methods to extract meaningful insights from data.
    \end{block}
    
    \begin{block}{Suggested Reading}
        Explore pandas documentation for more details on DataFrame creation: 
        \url{https://pandas.pydata.org/pandas-docs/stable/reference/api/pandas.DataFrame.html}
    \end{block}
\end{frame}

\begin{frame}[fragile]
    \frametitle{Data Inspection - Understanding DataFrames}
    \begin{block}{Overview}
        After creating DataFrames from various data sources, the next crucial step is to \textbf{inspect} them. 
        This process helps us understand the structure, types, and basic statistics of the data, which is vital for effective data manipulation and analysis.
    \end{block}
\end{frame}

\begin{frame}[fragile]
    \frametitle{Data Inspection - Key Techniques}
    Python's Pandas library provides several methods to inspect DataFrames:
    \begin{enumerate}
        \item \textbf{head()} Method
        \item \textbf{tail()} Method
        \item \textbf{info()} Method
    \end{enumerate}
\end{frame}

\begin{frame}[fragile]
    \frametitle{Data Inspection - head() Method}
    \begin{block}{Purpose}
        Displays the first 5 rows of a DataFrame by default.
    \end{block}
    \begin{block}{Usage}
    \begin{lstlisting}[language=python]
    import pandas as pd

    # Sample DataFrame
    df = pd.DataFrame({
        'Name': ['Alice', 'Bob', 'Charlie'],
        'Age': [24, 30, 22],
        'City': ['New York', 'Los Angeles', 'Chicago']
    })

    # Inspecting the first 5 rows
    print(df.head())
    \end{lstlisting}
    \end{block}
    \begin{block}{Output}
    \begin{verbatim}
        Name  Age         City
    0    Alice   24     New York
    1      Bob   30  Los Angeles
    2  Charlie   22      Chicago
    \end{verbatim}
    \end{block}
\end{frame}

\begin{frame}[fragile]
    \frametitle{Data Inspection - tail() Method}
    \begin{block}{Purpose}
        Displays the last 5 rows of the DataFrame by default. Useful for checking the end of the data, especially in large datasets.
    \end{block}
    \begin{block}{Usage}
    \begin{lstlisting}[language=python]
    # Inspecting the last 5 rows
    print(df.tail())
    \end{lstlisting}
    \end{block}
    \begin{block}{Output}
    The output will show the last 5 rows, similar to the head() method.
    \end{block}
\end{frame}

\begin{frame}[fragile]
    \frametitle{Data Inspection - info() Method}
    \begin{block}{Purpose}
        Provides a concise summary of the DataFrame, including the number of entries, column names, data types, and memory usage.
    \end{block}
    \begin{block}{Usage}
    \begin{lstlisting}[language=python]
    # Displaying DataFrame summary
    df.info()
    \end{lstlisting}
    \end{block}
    \begin{block}{Output}
    \begin{verbatim}
    <class 'pandas.core.frame.DataFrame'>
    RangeIndex: 3 entries, 0 to 2
    Data columns (total 3 columns):
     #   Column   Non-Null Count  Dtype 
    ---  ------   --------------  ----- 
     0   Name     3 non-null      object
     1   Age      3 non-null      int64 
     2   City     3 non-null      object
    dtypes: int64(1), object(2)
    memory usage: 136.0 bytes
    \end{verbatim}
    \end{block}
\end{frame}

\begin{frame}[fragile]
    \frametitle{Data Inspection - Key Points}
    \begin{itemize}
        \item \textbf{head()} is useful for quick previews of data.
        \item \textbf{tail()} helps check the end of large datasets.
        \item \textbf{info()} provides a metadata overview crucial before any data cleaning or analysis.
        \item Combining these methods offers a solid initial understanding of the DataFrame’s structure and content.
    \end{itemize}
\end{frame}

\begin{frame}[fragile]
    \frametitle{Data Inspection - Example Use Case}
    In practice, when dealing with a CSV file containing customer data:
    \begin{itemize}
        \item Use \textbf{head()} to quickly identify column names (like 'Customer ID', 'Purchase Amount').
        \item Use \textbf{tail()} to check for any unexpected entries at the end of the dataset.
        \item Use \textbf{info()} to detect data types and check for any missing values before proceeding with more complex analyses or data cleaning.
    \end{itemize}
\end{frame}

\begin{frame}[fragile]
    \frametitle{Data Selection and Filtering - Overview}
    In data analysis, especially using \textbf{Pandas}, selecting specific rows and columns and filtering data is crucial for focused analysis. 
    This slide will explore how to efficiently manipulate \textbf{DataFrames} to extract relevant data based on certain conditions.
\end{frame}

\begin{frame}[fragile]
    \frametitle{Data Selection and Filtering - Selecting Rows and Columns}
    \begin{block}{Selecting Columns}
        \begin{itemize}
            \item Single column selection:
            \begin{lstlisting}
            df['column_name']
            \end{lstlisting}
            \item Multiple columns selection:
            \begin{lstlisting}
            df[['column1', 'column2']]
            \end{lstlisting}
        \end{itemize}
    \end{block}

    \begin{block}{Selecting Rows}
        \begin{itemize}
            \item Label-based selection with \texttt{loc}:
            \begin{lstlisting}
            df.loc[row_label]
            \end{lstlisting}
            \item Position-based selection with \texttt{iloc}:
            \begin{lstlisting}
            df.iloc[row_index]
            \end{lstlisting}
            \item Selecting multiple rows using slice:
            \begin{lstlisting}
            df[10:20]  % select rows 10 to 19
            \end{lstlisting}
        \end{itemize}
    \end{block}
\end{frame}

\begin{frame}[fragile]
    \frametitle{Data Selection and Filtering - Filtering Data Based on Conditions}
    \begin{block}{Filtering Data}
        Filtering allows you to extract data that meets certain criteria:
        \begin{enumerate}
            \item Basic Filtering:
            \begin{lstlisting}
            filtered_data = df[df['column_name'] > value]
            \end{lstlisting}

            \item Multiple Conditions:
            \begin{lstlisting}
            filtered_data = df[(df['column1'] > value1) & (df['column2'] < value2)]
            \end{lstlisting}

            \item Using \texttt{.query()} for readability:
            \begin{lstlisting}
            filtered_data = df.query('column1 > value1 and column2 < value2')
            \end{lstlisting}
        \end{enumerate}
    \end{block}
\end{frame}

\begin{frame}[fragile]
    \frametitle{Data Selection and Filtering - Practical Example}
    Consider a DataFrame \texttt{df} with the following structure:
    
    \begin{center}
    \begin{tabular}{|c|c|c|c|}
    \hline
    Name & Age & Gender & Salary \\
    \hline
    John & 28 & M & 50000 \\
    Alice & 34 & F & 60000 \\
    Bob & 29 & M & 50000 \\
    Carol & 25 & F & 48000 \\
    \hline
    \end{tabular}
    \end{center}

    \begin{block}{Examples}
        \begin{itemize}
            \item Selecting the 'Name' and 'Salary' columns:
            \begin{lstlisting}
            selected_columns = df[['Name', 'Salary']]
            \end{lstlisting}

            \item Filtering employees with Salary greater than 50000:
            \begin{lstlisting}
            high_salary = df[df['Salary'] > 50000]
            \end{lstlisting}

            \item Filtering adults (Age > 30) who are female:
            \begin{lstlisting}
            female_adults = df[(df['Age'] > 30) & (df['Gender'] == 'F')]
            \end{lstlisting}
        \end{itemize}
    \end{block}
\end{frame}

\begin{frame}
  \frametitle{Data Cleaning Techniques}
  \begin{block}{Introduction to Data Cleaning}
    Data cleaning is the process of identifying and correcting inaccuracies or inconsistencies in data to improve its quality. In this slide, we will discuss three common data cleaning techniques:
    \begin{enumerate}
      \item Handling Missing Values
      \item Removing Duplicates
      \item Data Type Conversions
    \end{enumerate}
  \end{block}
\end{frame}

\begin{frame}[fragile]
  \frametitle{Handling Missing Values}
  \begin{block}{Explanation}
    Missing values can lead to misleading analysis and results. Common strategies to handle them include:
    \begin{itemize}
      \item \textbf{Removal}: Simply remove rows with missing values if they are not significant.
      \item \textbf{Imputation}: Replace missing values with substitutes such as:
        \begin{itemize}
          \item Mean/median (for numerical data)
          \item Mode (for categorical data)
          \item A specific constant (e.g., 0, or “unknown”)
        \end{itemize}
    \end{itemize}
  \end{block}

  \begin{block}{Example in Pandas}
    \begin{lstlisting}[language=Python]
import pandas as pd

# Sample DataFrame
data = {'name': ['Alice', 'Bob', None],
        'age': [25, None, 30]}

df = pd.DataFrame(data)

# Remove rows with missing values
df_cleaned = df.dropna()

# Impute missing age with mean
df['age'].fillna(df['age'].mean(), inplace=True)
    \end{lstlisting}
  \end{block}
\end{frame}

\begin{frame}[fragile]
  \frametitle{Removing Duplicates and Data Type Conversions}
  
  \begin{block}{Removing Duplicates}
    \begin{block}{Explanation}
      Duplicate entries can skew data analysis, making it critical to identify and remove them.
    \end{block}
    
    \begin{block}{Example in Pandas}
      \begin{lstlisting}[language=Python]
# Sample DataFrame with duplicates
data = {'name': ['Alice', 'Bob', 'Alice'],
        'age': [25, 30, 25]}

df = pd.DataFrame(data)

# Remove duplicate rows
df_unique = df.drop_duplicates()
      \end{lstlisting}
    \end{block}
  \end{block}

  \begin{block}{Data Type Conversions}
    \begin{block}{Explanation}
      Data might be in an incorrect format; for instance, a numerical value might be stored as a string. Converting data types can prevent analysis errors.
    \end{block}
    
    \begin{block}{Example in Pandas}
      \begin{lstlisting}[language=Python]
# Sample DataFrame with incorrect data types
data = {'name': ['Alice', 'Bob'],
        'age': ['25', '30']}  # Age as strings

df = pd.DataFrame(data)

# Convert age to integer
df['age'] = df['age'].astype(int)
      \end{lstlisting}
    \end{block}
  \end{block}
\end{frame}

\begin{frame}
  \frametitle{Key Points and Conclusion}
  \begin{block}{Key Points to Emphasize}
    \begin{itemize}
      \item Clean data is critical for reliable analysis and insights.
      \item Common techniques include handling missing values, removing duplicates, and converting data types.
      \item Use Pandas functions like \texttt{dropna()}, \texttt{drop_duplicates()}, and \texttt{astype()} for effective data cleaning.
    \end{itemize}
  \end{block}

  \begin{block}{Conclusion}
    Mastering data cleaning techniques is essential for any data manipulation task. These foundational skills will enable you to prepare datasets for deeper analysis and insight extraction successfully.
  \end{block}
  
  \begin{block}{Questions}
    Feel free to ask questions on any of the techniques or methods discussed today!
  \end{block}
\end{frame}

\begin{frame}
    \frametitle{Data Manipulation Operations}
    \begin{block}{Overview}
        Data manipulation is a crucial step in data analysis, allowing us to organize, transform, and summarize data effectively. This presentation covers key operations in Python's Pandas library, focusing on sorting, grouping, and aggregating data.
    \end{block}
\end{frame}

\begin{frame}[fragile]
    \frametitle{Key Operations in Pandas - Sorting Data}
    \begin{itemize}
        \item \textbf{Sorting Data:}
        \begin{itemize}
            \item \textbf{Definition:} Organizes data in a specific order (ascending or descending) based on one or more columns.
            \item \textbf{Code Snippet:}
        \end{itemize}
    \end{itemize}
    \begin{lstlisting}[language=Python]
import pandas as pd

data = {'Name': ['Alice', 'Bob', 'Charlie'],
        'Score': [85, 95, 70]}
df = pd.DataFrame(data)

# Sort by 'Score' in descending order
sorted_df = df.sort_values(by='Score', ascending=False)
print(sorted_df)
    \end{lstlisting}
    \begin{block}{Output}
         \begin{verbatim}
             Name  Score
         1    Bob     95
         0  Alice     85
         2 Charlie     70
         \end{verbatim}
    \end{block}
\end{frame}

\begin{frame}[fragile]
    \frametitle{Key Operations in Pandas - Grouping and Aggregating}
    \begin{itemize}
        \item \textbf{Grouping Data:}
        \begin{itemize}
            \item \textbf{Definition:} Splits data into subsets based on unique values of one or more columns.
            \item \textbf{Code Snippet:}
        \end{itemize}
    \end{itemize}
    \begin{lstlisting}[language=Python]
data = {'Category': ['A', 'B', 'A', 'B', 'A'],
        'Score': [85, 95, 75, 80, 90]}
df = pd.DataFrame(data)

# Group by 'Category' and calculate the mean of 'Score'
grouped_df = df.groupby('Category').mean()
print(grouped_df)
    \end{lstlisting}
    \begin{block}{Output}
         \begin{verbatim}
                  Score
         Category       
         A          83.33
         B          87.50
         \end{verbatim}
    \end{block}
\end{frame}

\begin{frame}[fragile]
    \frametitle{Key Operations in Pandas - Aggregating Data}
    \begin{itemize}
        \item \textbf{Aggregating Data:}
        \begin{itemize}
            \item \textbf{Definition:} Summarizes data by applying functions like sum, mean, and count to grouped data.
            \item \textbf{Code Snippet:}
        \end{itemize}
    \end{itemize}
    \begin{lstlisting}[language=Python]
data = {'Category': ['A', 'B', 'A', 'B', 'A'],
        'Score': [85, 95, 75, 80, 90]}
df = pd.DataFrame(data)

# Group by 'Category' and aggregate with multiple functions
agg_df = df.groupby('Category').agg({'Score': ['mean', 'sum', 'count']})
print(agg_df)
    \end{lstlisting}
    \begin{block}{Output}
         \begin{verbatim}
                  Score         
                   mean  sum count
         Category                  
         A          83.33  250    3
         B          87.50  175    2
         \end{verbatim}
    \end{block}
\end{frame}

\begin{frame}[fragile]
    \frametitle{Merging and Joining DataFrames - Introduction}
    \begin{block}{Introduction}
        In data manipulation, combining multiple DataFrames is essential for data analysis.
        In Python's Pandas library, you can:
        \begin{itemize}
            \item Merge DataFrames
            \item Join DataFrames
            \item Concatenate DataFrames
        \end{itemize}
    \end{block}
\end{frame}

\begin{frame}[fragile]
    \frametitle{Merging DataFrames}
    \begin{block}{Merge}
        The \texttt{merge()} function allows combining two DataFrames based on keys (common columns).
        \begin{itemize}
            \item \textbf{Syntax}:
            \begin{lstlisting}
pd.merge(left, right, how='type', on='key')
            \end{lstlisting}
            \item \textbf{Parameters}:
            \begin{itemize}
                \item \texttt{left} and \texttt{right}: DataFrames to merge.
                \item \texttt{how}: Type of merge (e.g. 'inner', 'outer', 'left', 'right').
                \item \texttt{on}: Column names to join on.
            \end{itemize}
            \item \textbf{Example}:
            \begin{lstlisting}
import pandas as pd

df1 = pd.DataFrame({'A': ['A0', 'A1'], 'B': ['B0', 'B1']})
df2 = pd.DataFrame({'A': ['A0', 'A1'], 'C': ['C0', 'C1']})

result = pd.merge(df1, df2, on='A', how='inner')
            \end{lstlisting}
        \end{itemize}
        \textbf{Result}:
        \begin{lstlisting}
   A   B   C
0 A0  B0  C0
1 A1  B1  C1
        \end{lstlisting}
    \end{block}
\end{frame}

\begin{frame}[fragile]
    \frametitle{Joining DataFrames}
    \begin{block}{Join}
        The \texttt{join()} method primarily joins DataFrames on their index.
        \begin{itemize}
            \item \textbf{Syntax}:
            \begin{lstlisting}
df1.join(df2, how='type')
            \end{lstlisting}
            \item \textbf{Example}:
            \begin{lstlisting}
df1 = pd.DataFrame({'A': ['A0', 'A1'], 'B': ['B0', 'B1']}).set_index('A')
df2 = pd.DataFrame({'C': ['C0', 'C1']}, index=['A0', 'A1'])

result = df1.join(df2)
            \end{lstlisting}
        \end{itemize}
        \textbf{Result}:
        \begin{lstlisting}
    B   C
A       
A0  B0  C0
A1  B1  C1
        \end{lstlisting}
    \end{block}
\end{frame}

\begin{frame}[fragile]
    \frametitle{Concatenating DataFrames}
    \begin{block}{Concatenate}
        The \texttt{concat()} function combines DataFrames along a specified axis.
        \begin{itemize}
            \item \textbf{Syntax}:
            \begin{lstlisting}
pd.concat([df1, df2], axis=0 or 1)
            \end{lstlisting}
            \item \textbf{Parameters}:
            \begin{itemize}
                \item \texttt{axis}: 0 for stacking rows, 1 for side-by-side columns.
            \end{itemize}
            \item \textbf{Example}:
            \begin{lstlisting}
df1 = pd.DataFrame({'A': ['A0', 'A1']})
df2 = pd.DataFrame({'B': ['B0', 'B1']})

result = pd.concat([df1, df2], axis=1)
            \end{lstlisting}
        \end{itemize}
        \textbf{Result}:
        \begin{lstlisting}
   A   B
0 A0  B0
1 A1  B1
        \end{lstlisting}
    \end{block}
\end{frame}

\begin{frame}[fragile]
    \frametitle{Key Points and Summary}
    \begin{block}{Key Points}
        \begin{itemize}
            \item \textbf{Inner Join}: Only includes matching keys from both DataFrames.
            \item \textbf{Outer Join}: Includes all keys while filling missing with NaN.
            \item \textbf{Left/Right Joins}: Includes all keys from one DataFrame and matches from another.
            \item Use \texttt{ignore\_index=True} with \texttt{concat()} to reset the index.
        \end{itemize}
    \end{block}

    \begin{block}{Summary}
        Understanding how to merge, join, and concatenate DataFrames in Pandas is crucial for effective data manipulation and analysis.
        Leverage these methods for efficient dataset combinations to gain deeper insights.
    \end{block}
\end{frame}

\begin{frame}
    \frametitle{Real-World Applications of Pandas}
    \begin{block}{Overview}
        Pandas is a powerful data manipulation library for Python that is heavily utilized in various industries for data analysis, cleaning, and transformation.
        Let's explore some real-world applications where Pandas showcases its capabilities.
    \end{block}
\end{frame}

\begin{frame}[fragile]
    \frametitle{Financial Data Analysis}
    \begin{itemize}
        \item \textbf{Example:} Stock Price Analysis
        \item \textbf{Description:} Analysts use Pandas to analyze historical stock prices, calculate moving averages, visualize trends, and perform risk assessments.
    \end{itemize}
    
    \begin{lstlisting}[language=Python]
import pandas as pd
# Load stock price data
data = pd.read_csv('stock_prices.csv')
# Calculate moving average
data['MA50'] = data['Close'].rolling(window=50).mean()
    \end{lstlisting}
    
    \begin{itemize}
        \item \textbf{Key Points:}
        \begin{itemize}
            \item Moving averages help in identifying price trends.
            \item Time series analysis can uncover seasonal patterns.
        \end{itemize}
    \end{itemize}
\end{frame}

\begin{frame}[fragile]
    \frametitle{Data Cleaning and Preprocessing}
    \begin{itemize}
        \item \textbf{Example:} Preparing Datasets for Machine Learning
        \item \textbf{Description:} Before feeding data into ML models, it's crucial to clean and preprocess datasets, including removing duplicates and handling missing values.
    \end{itemize}
    
    \begin{lstlisting}[language=Python]
# Removing duplicates
data = data.drop_duplicates()
# Filling missing values
data['Age'] = data['Age'].fillna(data['Age'].median())
    \end{lstlisting}
    
    \begin{itemize}
        \item \textbf{Key Points:}
        \begin{itemize}
            \item Clean data ensures better model performance.
            \item Pandas provides methods for efficient data preprocessing.
        \end{itemize}
    \end{itemize}
\end{frame}

\begin{frame}[fragile]
    \frametitle{Sales and Marketing Analytics}
    \begin{itemize}
        \item \textbf{Example:} Customer Segmentation
        \item \textbf{Description:} Marketers use sales data with Pandas to segment customers based on behavior, demographics, and preferences.
    \end{itemize}
    
    \begin{lstlisting}[language=Python]
# Group by customer segment
segmented_data = data.groupby('Customer_Segment').agg({'Sales': 'sum'})
    \end{lstlisting}

    \begin{itemize}
        \item \textbf{Key Points:}
        \begin{itemize}
            \item Segmentation helps tailor marketing strategies.
            \item Aggregative functions like \texttt{sum}, \texttt{mean}, and \texttt{count} are essential for insights.
        \end{itemize}
    \end{itemize}
\end{frame}

\begin{frame}[fragile]
    \frametitle{Health Data Monitoring}
    \begin{itemize}
        \item \textbf{Example:} Analyzing Patient Records
        \item \textbf{Description:} Healthcare professionals use Pandas to monitor patient data, track treatment efficacy, and analyze health trends over time.
    \end{itemize}

    \begin{lstlisting}[language=Python]
# Analyze treatment outcomes
outcomes = data.groupby('Treatment_Type')['Success_Rate'].mean()
    \end{lstlisting}

    \begin{itemize}
        \item \textbf{Key Points:}
        \begin{itemize}
            \item Data analysis can improve healthcare outcomes through evidence-based practices.
            \item Pandas simplifies complex data manipulation tasks.
        \end{itemize}
    \end{itemize}
\end{frame}

\begin{frame}[fragile]
    \frametitle{Web Scraping and Data Aggregation}
    \begin{itemize}
        \item \textbf{Example:} Collecting and Analyzing Web Data
        \item \textbf{Description:} Data scientists use Pandas with web scraping libraries (like BeautifulSoup) to aggregate and analyze online data.
    \end{itemize}

    \begin{lstlisting}[language=Python]
# Scraping example and creating DataFrame
import requests
from bs4 import BeautifulSoup

response = requests.get('https://example.com/data')
soup = BeautifulSoup(response.text, 'html.parser')
data = pd.DataFrame(soup.find_all('table'))  # Simplified
    \end{lstlisting}

    \begin{itemize}
        \item \textbf{Key Points:}
        \begin{itemize}
            \item Pandas easily manages large datasets collected from various sources.
            \item Integration with other libraries enhances functionality.
        \end{itemize}
    \end{itemize}
\end{frame}

\begin{frame}
    \frametitle{Conclusion}
    \begin{itemize}
        \item Pandas transforms raw data into actionable insights across multiple domains.
        \item Mastering Pandas enhances your ability to handle and analyze data efficiently!
    \end{itemize}
\end{frame}

\begin{frame}[fragile]
  \frametitle{Best Practices in Data Manipulation - Overview}
  
  \begin{block}{Overview}
  Data manipulation is a crucial step in data analysis, especially when using libraries like Pandas in Python. Following best practices enhances efficiency and ensures data integrity. This slide outlines key guidelines for effective data manipulation while maintaining high data quality.
  \end{block}
  
  \begin{itemize}
      \item Understand Data Structure
      \item Data Cleaning
      \item Use Vectorized Operations
      \item Check for Duplicates
      \item Data Transformation
      \item Documentation and Version Control
  \end{itemize}
  
\end{frame}

\begin{frame}[fragile]
  \frametitle{Best Practices in Data Manipulation - Understanding Your Data}
  
  \begin{block}{1. Understand Your Data}
      \begin{itemize}
          \item Explore data structure using:
          \begin{lstlisting}
import pandas as pd

df = pd.read_csv('data.csv')
print(df.info())
print(df.describe())
print(df.head())
          \end{lstlisting}
          \item \textbf{Key Takeaway:} Knowing the data structure helps identify cleaning methods.
      \end{itemize}
  \end{block}
  
\end{frame}

\begin{frame}[fragile]
  \frametitle{Best Practices in Data Manipulation - Data Cleaning}
  
  \begin{block}{2. Data Cleaning}
      \begin{itemize}
          \item Handle Missing Values:
          \begin{lstlisting}
df.fillna(0, inplace=True)  # Fill missing values
df.dropna(inplace=True)     # Drop rows with missing values
          \end{lstlisting}
          
          \item Standardize Data Formats:
          \begin{lstlisting}
df['date'] = pd.to_datetime(df['date'])
          \end{lstlisting}
      \end{itemize}
  \end{block}
  
\end{frame}

\begin{frame}[fragile]
  \frametitle{Best Practices in Data Manipulation - Usage and Documentation}
  
  \begin{block}{3. Use Vectorized Operations}
      \begin{itemize}
          \item Efficiency with Pandas:
          \begin{lstlisting}
df['total'] = df['quantity'] * df['price']  # Vectorized operation
          \end{lstlisting}
      \end{itemize}
  \end{block}
  
  \begin{block}{4. Check for Duplicates}
      \begin{itemize}
          \item Identify and remove duplicates:
          \begin{lstlisting}
df.drop_duplicates(inplace=True)
          \end{lstlisting}
      \end{itemize}
  \end{block}
  
\end{frame}

\begin{frame}[fragile]
  \frametitle{Best Practices in Data Manipulation - Documentation and Key Points}
  
  \begin{block}{5. Data Transformation}
      \begin{itemize}
          \item Apply Functions Using \texttt{apply()}:
          \begin{lstlisting}
df['new_column'] = df['old_column'].apply(lambda x: x + 10)
          \end{lstlisting}
      \end{itemize}
  \end{block}
  
  \begin{block}{6. Documentation and Version Control}
      \begin{itemize}
          \item Comment your code for clarity and future reference.
          \item Utilize version control systems (e.g., Git) for tracking changes.
      \end{itemize}
  \end{block}
  
  \begin{block}{Key Points to Emphasize}
      \begin{itemize}
          \item Understanding your data is critical.
          \item Clean data leads to reliable analyses.
          \item Leverage Pandas features for efficiency.
          \item Documentation is essential for collaboration.
      \end{itemize}
  \end{block}
  
\end{frame}

\begin{frame}[fragile]
    \frametitle{Summary and Key Takeaways - Overview}
    \begin{block}{Overview of Data Manipulation}
        Data manipulation involves transforming, reorganizing, and analyzing data to make it more informative for decision-making processes. Mastery of data manipulation is crucial in data science as it lays the groundwork for effective data analysis, modeling, and visualization.
    \end{block}
\end{frame}

\begin{frame}[fragile]
    \frametitle{Summary and Key Takeaways - Key Concepts Covered}
    \begin{enumerate}
        \item \textbf{Pandas Library}
        \begin{itemize}
            \item The primary library for data manipulation in Python.
            \item Provides flexible data structures: Series and DataFrames.
        \end{itemize}
        
        \item \textbf{Basic Operations}
        \begin{itemize}
            \item Reading Data: Use \texttt{pd.read\_csv()} to load datasets.
            \item Inspecting Data: Functions like \texttt{.head()}, \texttt{.info()}, and \texttt{.describe()} provide insights.
        \end{itemize}
        
        \item \textbf{Data Cleaning}
        \begin{itemize}
            \item Handling missing values with \texttt{.fillna()} and \texttt{.dropna()}.
            \item Removing duplicates using \texttt{.drop\_duplicates()}.
        \end{itemize}
    \end{enumerate}
\end{frame}

\begin{frame}[fragile]
    \frametitle{Summary and Key Takeaways - Importance and Example}
    \begin{block}{Importance of Proficiency in Data Manipulation}
        \begin{itemize}
            \item Enhances Data Quality.
            \item Facilitates Advanced Analysis.
            \item Supports Real-World Applications.
        \end{itemize}
    \end{block}

    \begin{block}{Example Code Snippet}
    \begin{lstlisting}[language=Python]
    import pandas as pd

    # Load dataset
    df = pd.read_csv('data.csv')

    # Clean data: drop missing values
    df.dropna(inplace=True)

    # Transform data: create a new column
    df['total'] = df['price'] * df['quantity']

    # Group by a category and calculate the average
    average_sales = df.groupby('category')['total'].mean()

    print(average_sales)
    \end{lstlisting}
    \end{block}
\end{frame}

\begin{frame}[fragile]
  \frametitle{Q\&A Session - Overview}
  This Q\&A session invites you to engage with the concepts of data manipulation explored in Python throughout this chapter. 
  \begin{itemize}
      \item Bring forward your questions and clarifications.
      \item Engage in discussions about real-world applications of data manipulation techniques.
  \end{itemize}
\end{frame}

\begin{frame}[fragile]
  \frametitle{Q\&A Session - Key Questions}
  \begin{enumerate}
      \item \textbf{Understanding Data Manipulation}:
          \begin{itemize}
              \item Core techniques used in data manipulation.
              \item Differences between data manipulation and data analysis.
          \end{itemize}
      \item \textbf{Pandas Basics}:
          \begin{itemize}
              \item Efficient usage of Pandas for data manipulation.
              \item Example of cleaning and transforming data using Pandas.
          \end{itemize}
      \item \textbf{Common Functions}:
          \begin{itemize}
              \item Commonly used Pandas functions such as \texttt{drop()}, \texttt{fillna()}, and \texttt{groupby()}.
          \end{itemize}
      \item \textbf{Real-world Applications}:
          \begin{itemize}
              \item Share examples from fields like finance, healthcare, or social media analysis.
          \end{itemize}
  \end{enumerate}
\end{frame}

\begin{frame}[fragile]
  \frametitle{Q\&A Session - Examples}
  \begin{block}{Data Cleaning Example}
      Here’s a simple example using Pandas to demonstrate data cleaning:
      \begin{lstlisting}[language=Python]
import pandas as pd

# Creating a sample DataFrame
data = {
    'Name': ['Alice', 'Bob', None, 'David'],
    'Age': [25, 30, 22, None],
    'Score': [88.5, 95.5, 79.0, 91.0]
}
df = pd.DataFrame(data)

# Drop rows with missing values
cleaned_df = df.dropna()
print(cleaned_df)
      \end{lstlisting}
  \end{block}
  
  \begin{block}{Data Transformation Example}
      To convert a column, you can use:
      \begin{lstlisting}[language=Python]
# Converting Age to integer
df['Age'] = df['Age'].fillna(0).astype(int)
print(df)
      \end{lstlisting}
  \end{block}
\end{frame}


\end{document}