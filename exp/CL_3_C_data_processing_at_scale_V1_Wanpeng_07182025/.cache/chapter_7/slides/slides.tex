\documentclass[aspectratio=169]{beamer}

% Theme and Color Setup
\usetheme{Madrid}
\usecolortheme{whale}
\useinnertheme{rectangles}
\useoutertheme{miniframes}

% Additional Packages
\usepackage[utf8]{inputenc}
\usepackage[T1]{fontenc}
\usepackage{graphicx}
\usepackage{booktabs}
\usepackage{listings}
\usepackage{amsmath}
\usepackage{amssymb}
\usepackage{xcolor}
\usepackage{tikz}
\usepackage{pgfplots}
\pgfplotsset{compat=1.18}
\usetikzlibrary{positioning}
\usepackage{hyperref}

% Custom Colors
\definecolor{myblue}{RGB}{31, 73, 125}
\definecolor{mygray}{RGB}{100, 100, 100}
\definecolor{mygreen}{RGB}{0, 128, 0}
\definecolor{myorange}{RGB}{230, 126, 34}
\definecolor{mycodebackground}{RGB}{245, 245, 245}

% Set Theme Colors
\setbeamercolor{structure}{fg=myblue}
\setbeamercolor{frametitle}{fg=white, bg=myblue}
\setbeamercolor{title}{fg=myblue}
\setbeamercolor{section in toc}{fg=myblue}
\setbeamercolor{item projected}{fg=white, bg=myblue}
\setbeamercolor{block title}{bg=myblue!20, fg=myblue}
\setbeamercolor{block body}{bg=myblue!10}
\setbeamercolor{alerted text}{fg=myorange}

% Set Fonts
\setbeamerfont{title}{size=\Large, series=\bfseries}
\setbeamerfont{frametitle}{size=\large, series=\bfseries}
\setbeamerfont{caption}{size=\small}
\setbeamerfont{footnote}{size=\tiny}

% Document Start
\begin{document}

\frame{\titlepage}

\begin{frame}[fragile]
    \frametitle{Introduction to Data Ethics and Governance - Overview}
    
    \begin{block}{Overview of Data Ethics}
        Data ethics refers to the moral implications and responsibilities associated with data collection, processing, and dissemination. 
        It addresses questions of fairness, privacy, accountability, and transparency when handling data—especially large sets.
    \end{block}
    
    \begin{block}{Significance in Large-Scale Data Processing}
        \begin{itemize}
            \item \textbf{Protection of Individuals:} Risks of misuse, discrimination, and loss of privacy.
            \item \textbf{Trust and Reputation:} Upholding ethical standards fosters trust among stakeholders.
            \item \textbf{Compliance and Risk Management:} Crucial to avoid legal penalties (e.g., GDPR, CCPA).
        \end{itemize}
    \end{block}
\end{frame}

\begin{frame}[fragile]
    \frametitle{Introduction to Data Ethics and Governance - Key Considerations and Frameworks}
    
    \begin{block}{Key Considerations in Data Ethics}
        \begin{itemize}
            \item \textbf{Consent:} Explicit and informed consent must be obtained without coercion.
            \item \textbf{Transparency:} Clarity about data practices through user-friendly privacy policies.
            \item \textbf{Accountability:} Data stewards must own their data practices and face consequences for misuse.
        \end{itemize}
    \end{block}
    
    \begin{block}{Frameworks to be Discussed}
        \begin{itemize}
            \item \textbf{General Data Protection Regulation (GDPR):} User rights and responsibilities in data processing.
            \item \textbf{Health Insurance Portability and Accountability Act (HIPAA):} Ethics in health information and patient privacy.
            \item \textbf{Fair Housing Act and Algorithmic Accountability:} Addressing systemic biases in data usage.
        \end{itemize}
    \end{block}
\end{frame}

\begin{frame}[fragile]
    \frametitle{Introduction to Data Ethics and Governance - Conclusion}
    
    \begin{block}{Conclusion}
        Data ethics is critical for navigating contemporary data environments, ensuring that practices are legally compliant and ethically sound. 
        Next, we will explore GDPR, outlining its principles and significant implications for data governance.
    \end{block}
    
    \begin{block}{Key Points to Remember}
        \begin{enumerate}
            \item Definition of Data Ethics: Focuses on moral considerations in data handling.
            \item Importance of Ethical Data Use: Protects individuals, builds trust, and ensures compliance.
            \item Frameworks: GDPR and HIPAA guide ethical practices in specific contexts.
        \end{enumerate}
    \end{block}
\end{frame}

\begin{frame}[fragile]{Understanding GDPR - Introduction}
    \begin{block}{What is GDPR?}
        The General Data Protection Regulation (GDPR) is a landmark regulation in the EU that governs how organizations handle personal data. It came into force on May 25, 2018, with the aim of protecting the privacy and rights of individuals within the European Union (EU) and the European Economic Area (EEA).
    \end{block}
\end{frame}

\begin{frame}[fragile]{Understanding GDPR - Key Concepts}
    \begin{itemize}
        \item \textbf{Personal Data}: Any information that relates to an identified or identifiable person (e.g., names, emails, location data).
        \item \textbf{Data Processing}: Any operation performed on personal data, including collection, storage, modification, retrieval, and deletion.
        \item \textbf{Data Subject}: The individual whose personal data is being processed and who holds specific rights under GDPR.
    \end{itemize}
\end{frame}

\begin{frame}[fragile]{Understanding GDPR - Key Principles}
    \begin{itemize}
        \item \textbf{Lawfulness, Fairness, and Transparency}: Processing must be lawful, fair, and transparent to the data subject.
        \item \textbf{Purpose Limitation}: Data must be collected for specified, legitimate purposes.
        \item \textbf{Data Minimization}: Only data necessary for the purpose should be collected.
        \item \textbf{Accuracy}: Must be accurate and kept up to date; steps taken to rectify inaccuracies.
        \item \textbf{Storage Limitation}: Retain personal data only as long as necessary.
        \item \textbf{Integrity and Confidentiality}: Data must be processed securely to protect against unauthorized access.
    \end{itemize}
\end{frame}

\begin{frame}[fragile]{Understanding GDPR - Implications}
    \begin{itemize}
        \item \textbf{Data Subject Rights}: Rights include access, rectification, erasure, data portability, and objection to processing.
        \item \textbf{Accountability and Compliance}: Organizations must demonstrate compliance through records, Designated Data Protection Officers (DPOs), and Data Protection Impact Assessments (DPIAs).
        \item \textbf{Penalties for Non-Compliance}: Significant fines up to 4\% of annual global turnover or €20 million, whichever is higher.
    \end{itemize}
\end{frame}

\begin{frame}[fragile]{Understanding GDPR - Example Scenario}
    \begin{block}{Example Scenario}
        A company that collects customer data for marketing purposes must ensure:
        \begin{itemize}
            \item Individuals are informed about the purpose of data collection (transparency).
            \item Only relevant data is collected (data minimization).
            \item Easy requests for data deletion if individuals opt-out (right to erasure).
        \end{itemize}
    \end{block}
\end{frame}

\begin{frame}[fragile]{Understanding GDPR - Summary and Further Learning}
    \begin{itemize}
        \item GDPR is essential for protecting personal data and privacy rights in the digital age.
        \item Organizations must adopt rigorous protocols for data governance to remain compliant.
    \end{itemize}
    \begin{block}{References for Further Learning}
        \begin{itemize}
            \item Official GDPR Documentation: \texttt{https://ec.europa.eu/info/law/law/2016_679}
            \item Articles on Data Subject Rights and Best Practices for Compliance.
        \end{itemize}
    \end{block}
\end{frame}

\begin{frame}[fragile]
    \frametitle{Key Principles of GDPR - Overview}
    The General Data Protection Regulation (GDPR) is an essential framework for protecting personal data of individuals within the European Union (EU). 
    This presentation covers the key principles, including:
    \begin{itemize}
        \item Data Subject Rights
        \item Lawful Data Processing
        \item Accountability Measures
    \end{itemize}
\end{frame}

\begin{frame}[fragile]
    \frametitle{Key Principles of GDPR - Data Subject Rights}
    Under GDPR, individuals (data subjects) are empowered with specific rights regarding their personal data:
    \begin{enumerate}
        \item \textbf{Right to Access}: Individuals can request access to their personal data held by organizations.
        \item \textbf{Right to Rectification}: Individuals can request corrections to inaccurate or incomplete personal data.
        \item \textbf{Right to Erasure (Right to be Forgotten)}: Data subjects can request deletion of their data when it is no longer necessary.
        \item \textbf{Right to Data Portability}: Individuals can obtain their data in a structured, commonly used format for transfer.
        \item \textbf{Right to Object}: Individuals can object to the processing of their personal data based on legitimate interests.
    \end{enumerate}
\end{frame}

\begin{frame}[fragile]
    \frametitle{Key Principles of GDPR - Lawful Data Processing}
    GDPR establishes lawful bases for processing personal data:
    \begin{itemize}
        \item \textbf{Consent}: Explicit permission from the data subject for processing their data.
        \item \textbf{Contractual Necessity}: Necessary processing for fulfilling a contract with the data subject.
        \item \textbf{Legal Obligation}: Processing required to comply with a legal duty.
        \item \textbf{Legitimate Interests}: Processing that serves a legitimate interest, not overriding the rights of the data subject.
    \end{itemize}
\end{frame}

\begin{frame}[fragile]
    \frametitle{Key Principles of GDPR - Accountability Measures}
    Accountability is a key principle that requires organizations to take responsibility for compliance:
    \begin{itemize}
        \item \textbf{Documented Procedures}: Maintaining records of processing activities for transparency.
        \item \textbf{Data Protection Impact Assessments (DPIAs)}: Necessary for high-risk data processing activities.
        \item \textbf{Data Protection Officer (DPO)}: Certain organizations must appoint a DPO for compliance oversight.
        \item \textbf{Breach Notification}: Organizations must notify affected individuals and authorities within 72 hours of a data breach.
    \end{itemize}
\end{frame}

\begin{frame}[fragile]
    \frametitle{Key Principles of GDPR - Key Takeaways}
    \begin{itemize}
        \item GDPR empowers individuals with rights over their personal data.
        \item Organizations must establish legitimate reasons for processing personal data.
        \item Accountability rests with organizations to demonstrate compliance efforts.
    \end{itemize}
    \textbf{Engagement Tip:} Reflect on how these principles apply in real-world scenarios and the implementation considerations for organizations.
\end{frame}

\begin{frame}[fragile]
    \frametitle{Understanding HIPAA - Introduction}
    \begin{block}{Introduction to HIPAA}
        The Health Insurance Portability and Accountability Act (HIPAA) was enacted in 1996 to protect the privacy and security of health information. 
        Its primary goal is to ensure that:
        \begin{itemize}
            \item Patients' medical records and health information are protected.
            \item The flow of health information is maintained to provide high-quality healthcare.
        \end{itemize}
    \end{block}
\end{frame}

\begin{frame}[fragile]
    \frametitle{Understanding HIPAA - Relevance to Data Management}
    \begin{block}{Relevance to Data Management in Healthcare}
        HIPAA significantly impacts healthcare data management through:
        \begin{enumerate}
            \item \textbf{Patient Privacy:} Establishes standards to keep health information confidential.
            \item \textbf{Data Security:} Requires safeguards for electronic health information (ePHI).
            \item \textbf{Data Portability:} Facilitates transfer of health coverage between employers.
        \end{enumerate}
    \end{block}
\end{frame}

\begin{frame}[fragile]
    \frametitle{Understanding HIPAA - Key Concepts}
    \begin{block}{Key Concepts Related to HIPAA}
        \begin{itemize}
            \item \textbf{Protected Health Information (PHI):} Any health information that identifies an individual.
            \item \textbf{Covered Entities:} Health care providers, health plans, and healthcare clearinghouses subject to HIPAA.
            \item \textbf{Business Associates:} Entities performing functions on behalf of a covered entity that involve PHI.
        \end{itemize}
    \end{block}

    \begin{block}{Examples of HIPAA Compliance}
        Examples include:
        \begin{itemize}
            \item \textbf{Encryption of ePHI:} Required to protect against data breaches.
            \item \textbf{Access Controls:} Role-based access to protect sensitive data.
        \end{itemize}
    \end{block}
\end{frame}

\begin{frame}[fragile]
    \frametitle{Understanding HIPAA - Key Points and Conclusion}
    \begin{block}{Key Points to Emphasize}
        \begin{itemize}
            \item \textbf{Patient Rights:} Patients can access their health records and request corrections.
            \item \textbf{Breach Notification:} Patients and HHS must be notified of breaches in a timely manner.
            \item \textbf{Non-compliance Penalties:} Fines range from \$100 to \$50,000 per violation, with a maximum annual limit of \$1.5 million.
        \end{itemize}
    \end{block}

    \begin{block}{Conclusion}
        HIPAA is crucial for protecting sensitive health information while ensuring the necessary flow of data to maintain quality healthcare. Professionals in data management must familiarize themselves with HIPAA provisions for compliance and patient privacy protection.
    \end{block}
\end{frame}

\begin{frame}[fragile]
    \frametitle{Key Provisions of HIPAA - Introduction}
    \begin{block}{Introduction to HIPAA}
        The Health Insurance Portability and Accountability Act (HIPAA) is a pivotal regulation in the healthcare sector that ensures the protection of patients' private health information. This legislation is designed to create a standard for handling sensitive patient data, emphasizing patient confidentiality and data security.
    \end{block}
\end{frame}

\begin{frame}[fragile]
    \frametitle{Key Provisions of HIPAA - Core Provisions}
    \begin{block}{Core Provisions of HIPAA}
        \begin{enumerate}
            \item \textbf{Patient Confidentiality}:
                \begin{itemize}
                    \item Patients must be informed when their health information is used or disclosed.
                    \item Patients have the right to access their own medical records.
                    \item \textit{Example}: Authentication required for lab result requests.
                \end{itemize}
            \item \textbf{Data Security Requirements}:
                \begin{itemize}
                    \item \textit{Administrative Safeguards}: Policies and procedures for managing data access.
                    \item \textit{Physical Safeguards}: Protections for physical access to electronic health information.
                    \item \textit{Technical Safeguards}: Technologies to protect electronically stored data.
                \end{itemize}
            \item \textbf{Penalties for Non-Compliance}:
                \begin{itemize}
                    \item \textit{Tier 1}: Unintentional violations ($100 - \$50,000$).
                    \item \textit{Tier 2}: Reasonable cause for violations ($1,000 - \$50,000$).
                    \item \textit{Tier 3}: Willful neglect corrected ($10,000 - \$50,000$).
                    \item \textit{Tier 4}: Willful neglect uncorrected (up to \$1.5 million/year).
                \end{itemize}
        \end{enumerate}
    \end{block}
\end{frame}

\begin{frame}[fragile]
    \frametitle{Key Provisions of HIPAA - Conclusion}
    \begin{block}{Key Points to Emphasize}
        \begin{itemize}
            \item \textbf{Patient Rights}: Patients have control over their health information, promoting transparency.
            \item \textbf{Security Measures}: Proper safeguards are crucial for compliance.
            \item \textbf{Consequences of Non-Compliance}: Understanding penalties reinforces the importance of adhering to the regulations.
        \end{itemize}
    \end{block}
    \begin{block}{Conclusion}
        Understanding HIPAA's provisions is essential for professionals in the healthcare industry. Adhering to these standards preserves patient confidentiality and protects organizations from severe financial penalties.
    \end{block}
\end{frame}

\begin{frame}[fragile]
    \frametitle{Ethical Frameworks for Data Use - Introduction}
    \begin{block}{Understanding Ethical Frameworks}
        In the realm of data governance, ethical frameworks serve as foundational principles guiding the responsible use of data. 
        Understanding these frameworks helps organizations and individuals make judicious decisions regarding data collection, processing, and dissemination.
    \end{block}
\end{frame}

\begin{frame}[fragile]
    \frametitle{Ethical Frameworks for Data Use - Major Frameworks}
    \begin{enumerate}
        \item Utilitarianism
        \item Deontology
        \item Virtue Ethics
    \end{enumerate}
\end{frame}

\begin{frame}[fragile]
    \frametitle{Utilitarianism}
    \begin{block}{Definition}
        Utilitarianism advocates for actions that promote the greatest good for the greatest number, emphasizing overall benefits while minimizing harm.
    \end{block}
    \begin{itemize}
        \item \textbf{Key Point:} Decisions are evaluated based on outcomes; the right action produces the most positive results.
        \item \textbf{Example:} A healthcare company may use patient data to develop new treatments, arguing the societal benefits outweigh privacy concerns.
    \end{itemize}
\end{frame}

\begin{frame}[fragile]
    \frametitle{Deontology}
    \begin{block}{Definition}
        Deontology focuses on duties and rules, asserting that some actions are inherently right or wrong regardless of consequences.
    \end{block}
    \begin{itemize}
        \item \textbf{Key Point:} This framework emphasizes adherence to moral rules and obligations.
        \item \textbf{Example:} A company may refuse to sell user data to third parties, as it is committed to respecting user privacy.
    \end{itemize}
\end{frame}

\begin{frame}[fragile]
    \frametitle{Virtue Ethics}
    \begin{block}{Definition}
        Virtue ethics centers on the character of the moral agent, emphasizing the importance of developing virtuous traits that inform ethical decisions.
    \end{block}
    \begin{itemize}
        \item \textbf{Key Point:} Decisions are guided by what a virtuous person would deem appropriate based on context and integrity.
        \item \textbf{Example:} An organization prioritizing ethical data practices as a reflection of their values and commitment to transparency.
    \end{itemize}
\end{frame}

\begin{frame}[fragile]
    \frametitle{Key Takeaways}
    \begin{itemize}
        \item Understanding these frameworks aids in navigating complex ethical dilemmas in data governance.
        \item Utilitarianism focuses on overall benefit, Deontology emphasizes rules and duties, and Virtue Ethics concerns moral character.
        \item Ethical data use is not solely about compliance but fostering trust and societal good.
    \end{itemize}
\end{frame}

\begin{frame}[fragile]
    \frametitle{Conclusion}
    Integrating ethical frameworks into data governance ensures that data practices are responsible, respectful, and aligned with legal standards and societal values. 
    Understanding these frameworks equips individuals and organizations to address data ethics challenges meaningfully.
\end{frame}

\begin{frame}[fragile]
    \frametitle{Importance of Ethical Data Use - Overview}
    \begin{block}{Ethical Data Use Defined}
        Ethical data use refers to the responsible collection, processing, storage, and sharing of data, while respecting individual rights and adhering to legal standards.
    \end{block}
    
    \begin{block}{Why Ethical Considerations Matter}
        \begin{itemize}
            \item \textbf{Trust Building:} Establishes confidence between organizations and the public.
            \item \textbf{Risk Management:} Helps avoid legal repercussions and mitigates data misuse risks.
            \item \textbf{Fairness:} Promotes equitable treatment, supporting a fairer society.
        \end{itemize}
    \end{block}
\end{frame}

\begin{frame}[fragile]
    \frametitle{Importance of Ethical Data Use - Case Studies}
    \begin{block}{Case Studies Highlighting Ethical Dilemmas}
        \begin{enumerate}
            \item \textbf{Cambridge Analytica Scandal}
                \begin{itemize}
                    \item In 2016, data from millions of Facebook users was harvested without consent.
                    \item Raises questions about informed consent, privacy rights, and data ownership.
                \end{itemize}
                
            \item \textbf{Facial Recognition Technology}
                \begin{itemize}
                    \item Increased use by law enforcement without comprehensive guidelines.
                    \item Issues of bias and surveillance prompt debate between privacy vs. security.
                \end{itemize}

            \item \textbf{Google’s Project Maven}
                \begin{itemize}
                    \item Google partnered with the U.S. Department of Defense for analyzing drone footage.
                    \item Employees objected to technology use in warfare, conflicting with company values.
                \end{itemize}
        \end{enumerate}
    \end{block}
\end{frame}

\begin{frame}[fragile]
    \frametitle{Importance of Ethical Data Use - Key Takeaways}
    \begin{block}{Key Points to Emphasize}
        \begin{itemize}
            \item Ethical data use enhances societal trust and relationships with stakeholders.
            \item Organizations should implement clear ethical guidelines to navigate complex dilemmas.
            \item Continuous education on data ethics is crucial for informed decision-making.
        \end{itemize}
    \end{block}
    
    \begin{block}{Conclusion}
        Emphasizing ethical data use is essential for data-driven organizations to align with societal values and legal frameworks.
    \end{block}
\end{frame}

\begin{frame}[fragile]
    \frametitle{Comparative Analysis of GDPR and HIPAA - Overview}
    \begin{block}{Key Frameworks}
        The General Data Protection Regulation (GDPR) and the Health Insurance Portability and Accountability Act (HIPAA) are pivotal for data protection and privacy rights.
    \end{block}
    \begin{itemize}
        \item \textbf{GDPR:} Governs personal data protection across all sectors within the EU.
        \item \textbf{HIPAA:} Specifically governs health information in the United States.
    \end{itemize}
\end{frame}

\begin{frame}[fragile]
    \frametitle{Comparative Analysis of GDPR and HIPAA - Key Points}
    \begin{table}[h]
        \centering
        \begin{tabular}{|c|c|c|}
            \hline
            \textbf{Aspect}                  & \textbf{GDPR}                                                & \textbf{HIPAA}                                          \\
            \hline
            \textbf{Scope}                  & All personal data of EU citizens, regardless of sector.    & Healthcare data (Protected Health Information - PHI). \\
            \hline
            \textbf{Data Subject Rights}    & Extensive rights: access, rectification, deletion, portability. & Limited rights: access and amendments to PHI.          \\
            \hline
            \textbf{Consent}                & Explicit consent required; can be withdrawn anytime.         & Implied for treatment; explicit for certain disclosures. \\
            \hline
            \textbf{Penalties}              & Fines up to €20 million or 4\% of annual turnover.          & Civil penalties from \$100 to \$50,000, capped at \$1.5 million. \\
            \hline
        \end{tabular}
    \end{table}
\end{frame}

\begin{frame}[fragile]
    \frametitle{Comparative Analysis of GDPR and HIPAA - Implications for Organizations}
    \begin{enumerate}
        \item \textbf{Regulatory Compliance:} Understand specific requirements relevant to the industry.
        \item \textbf{Resource Allocation:} GDPR may require more extensive resources than HIPAA.
        \item \textbf{Policy Development:} Create policies to comply with the legal framework and ethical standards.
        \item \textbf{Training Programs:} Ongoing training for employees on data protection is essential.
    \end{enumerate}
\end{frame}

\begin{frame}[fragile]
    \frametitle{Comparative Analysis of GDPR and HIPAA - Example Applications}
    \begin{itemize}
        \item \textbf{Healthcare Startup in the EU:} Must comply with GDPR ensuring explicit consent, data access and deletion rights, or face steep fines.
        \item \textbf{US-Based Healthcare Provider:} Must comply with HIPAA for patient information confidentiality but without the stringent consent requirements of GDPR unless dealing with EU data.
    \end{itemize}
\end{frame}

\begin{frame}[fragile]
    \frametitle{Comparative Analysis of GDPR and HIPAA - Conclusion}
    Understanding the nuances between GDPR and HIPAA is crucial:
    \begin{itemize}
        \item Compliance helps avoid penalties.
        \item It fosters trust and protects data subject rights.
        \item Best practices for ethical data management will be explored in the next chapter.
    \end{itemize}
\end{frame}

\begin{frame}[fragile]
    \frametitle{Best Practices for Ethical Data Management - Introduction}
    \begin{itemize}
        \item Organizations must align data management practices with legal frameworks such as GDPR and HIPAA.
        \item Ethical standards also play a critical role in data handling.
        \item This slide outlines recommendations to ensure compliance and uphold ethical integrity.
    \end{itemize}
\end{frame}

\begin{frame}[fragile]
    \frametitle{Key Concepts}
    \begin{itemize}
        \item \textbf{Data Ethics}: Principles guiding the responsible use of data to protect individual rights and societal interests.
        \item \textbf{GDPR (General Data Protection Regulation)}: Regulation in EU law affecting how organizations handle personal data.
        \item \textbf{HIPAA (Health Insurance Portability and Accountability Act)}: U.S. law providing data privacy and security for medical information.
    \end{itemize}
\end{frame}

\begin{frame}[fragile]
    \frametitle{Best Practices for Ethical Data Management}
    \begin{enumerate}
        \item \textbf{Data Minimization}:
            \begin{itemize}
                \item Collect only necessary data.
                \item Example: Do not collect social security numbers if they are not required.
            \end{itemize}
        \item \textbf{Transparency}:
            \begin{itemize}
                \item Inform individuals about data collection, purpose, and usage.
                \item Example: Provide a clear privacy notice on your website.
            \end{itemize}
        \item \textbf{Informed Consent}:
            \begin{itemize}
                \item Obtain explicit permission before using data.
                \item Example: Utilize opt-in forms for consent.
            \end{itemize}
    \end{enumerate}
\end{frame}

\begin{frame}[fragile]
    \frametitle{Continued Best Practices}
    \begin{enumerate}
        \setcounter{enumi}{3} % Start from the next number
        \item \textbf{Anonymization}:
            \begin{itemize}
                \item Remove personal identifiers from data sets.
                \item Example: Use codes instead of names in patient data.
            \end{itemize}
        \item \textbf{Data Security}:
            \begin{itemize}
                \item Implement robust security measures.
                \item Example: Use encryption for sensitive information.
            \end{itemize}
        \item \textbf{Audit and Accountability}:
            \begin{itemize}
                \item Regular assessments of data management practices.
                \item Example: Conduct compliance audits with GDPR and HIPAA.
            \end{itemize}
        \item \textbf{Continuous Training}:
            \begin{itemize}
                \item Train employees on data ethics and privacy laws.
                \item Example: Conduct regular workshops on data handling.
            \end{itemize}
    \end{enumerate}
\end{frame}

\begin{frame}[fragile]
    \frametitle{Conclusion and Key Points}
    \begin{itemize}
        \item Compliance with GDPR and HIPAA fosters trust with stakeholders.
        \item Ethical data management is essential in today's data-driven world.
        \item Key points to emphasize:
            \begin{itemize}
                \item Compliance and ethical standards go hand-in-hand.
                \item Transparency and informed consent are vital for trust-building.
                \item Continuous improvement through audits and training ensures ongoing adherence.
            \end{itemize}
    \end{itemize}
\end{frame}

\begin{frame}[fragile]
    \frametitle{Ethical Data Management Model}
    \begin{block}{Text-Based Diagram}
        \begin{center}
        +---------------------+ \\
        |   Data Minimization  | \\
        +---------------------+ \\
                  | \\
                  v \\
        +---------------------+ \\
        |       Transparency   | \\
        +----------+----------+ \\
                   | \\
                   v \\
        +---------------------+ \\
        |    Informed Consent  | \\
        +---------------------+
        \end{center}
    \end{block}
\end{frame}

\begin{frame}[fragile]
    \frametitle{Conclusion and Discussion Points - Key Takeaways}
    \begin{enumerate}
        \item \textbf{Understanding Data Ethics}
        \begin{itemize}
            \item Evaluating the moral implications of data collection, use, and sharing.
            \item Key Principle: \textit{Respect for individuals} and their data rights is paramount.
        \end{itemize}
        
        \item \textbf{Importance of Data Governance}
        \begin{itemize}
            \item Refers to policies and standards for data management and protection.
            \item Key Principle: Ensure accountability and compliance with regulations like GDPR and HIPAA.
        \end{itemize}

        \item \textbf{Best Practices in Ethical Data Management}
        \begin{itemize}
            \item Conduct regular data audits.
            \item Implement robust data protection measures.
            \item Train staff on ethical data handling.
        \end{itemize}

        \item \textbf{Implications of Non-Compliance}
        \begin{itemize}
            \item Risks include penalties, loss of public trust, and reputational damage.
            \item Example: GDPR violations can incur fines up to 4\% of annual revenue.
        \end{itemize}
    \end{enumerate}
\end{frame}

\begin{frame}[fragile]
    \frametitle{Conclusion and Discussion Points - Discussion Prompts}
    \begin{enumerate}
        \item \textbf{Real-Life Scenarios}
        \begin{itemize}
            \item Analyze a news story involving data breaches. 
            \item Discuss ethical principles violated and preventive measures.
        \end{itemize}

        \item \textbf{Personal Experience}
        \begin{itemize}
            \item Reflect on an instance of perceived data privacy compromise.
            \item Consider the organization's response and potential actions to improve.
        \end{itemize}

        \item \textbf{Future Considerations}
        \begin{itemize}
            \item Discuss ethical considerations in advancing technologies (e.g., AI, IoT).
            \item Explore challenges for current governance structures.
        \end{itemize}

        \item \textbf{Role of Stakeholders}
        \begin{itemize}
            \item Consider how consumers, organizations, and regulators influence data ethics.
            \item Identify roles each should play in promoting ethical practices.
        \end{itemize}
    \end{enumerate}
\end{frame}

\begin{frame}[fragile]
    \frametitle{Conclusion and Discussion Points - Key Emphases}
    \begin{itemize}
        \item Data ethics and governance are crucial for maintaining trust in digital interactions.
        \item Clear communication of data practices enhances customer loyalty and compliance.
        \item Understanding these concepts is essential for responsible data stewardship as future professionals.
    \end{itemize}
\end{frame}


\end{document}