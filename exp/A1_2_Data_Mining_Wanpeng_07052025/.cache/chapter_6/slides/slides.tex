\documentclass[aspectratio=169]{beamer}

% Theme and Color Setup
\usetheme{Madrid}
\usecolortheme{whale}
\useinnertheme{rectangles}
\useoutertheme{miniframes}

% Additional Packages
\usepackage[utf8]{inputenc}
\usepackage[T1]{fontenc}
\usepackage{graphicx}
\usepackage{booktabs}
\usepackage{listings}
\usepackage{amsmath}
\usepackage{amssymb}
\usepackage{xcolor}
\usepackage{tikz}
\usepackage{pgfplots}
\pgfplotsset{compat=1.18}
\usetikzlibrary{positioning}
\usepackage{hyperref}

% Custom Colors
\definecolor{myblue}{RGB}{31, 73, 125}
\definecolor{mygray}{RGB}{100, 100, 100}
\definecolor{mygreen}{RGB}{0, 128, 0}
\definecolor{myorange}{RGB}{230, 126, 34}
\definecolor{mycodebackground}{RGB}{245, 245, 245}

% Set Theme Colors
\setbeamercolor{structure}{fg=myblue}
\setbeamercolor{frametitle}{fg=white, bg=myblue}
\setbeamercolor{title}{fg=myblue}
\setbeamercolor{section in toc}{fg=myblue}
\setbeamercolor{item projected}{fg=white, bg=myblue}
\setbeamercolor{block title}{bg=myblue!20, fg=myblue}
\setbeamercolor{block body}{bg=myblue!10}
\setbeamercolor{alerted text}{fg=myorange}

% Set Fonts
\setbeamerfont{title}{size=\Large, series=\bfseries}
\setbeamerfont{frametitle}{size=\large, series=\bfseries}
\setbeamerfont{caption}{size=\small}
\setbeamerfont{footnote}{size=\tiny}

% Code Listing Style
\lstdefinestyle{customcode}{
  backgroundcolor=\color{mycodebackground},
  basicstyle=\footnotesize\ttfamily,
  breakatwhitespace=false,
  breaklines=true,
  commentstyle=\color{mygreen}\itshape,
  keywordstyle=\color{blue}\bfseries,
  stringstyle=\color{myorange},
  numbers=left,
  numbersep=8pt,
  numberstyle=\tiny\color{mygray},
  frame=single,
  framesep=5pt,
  rulecolor=\color{mygray},
  showspaces=false,
  showstringspaces=false,
  showtabs=false,
  tabsize=2,
  captionpos=b
}
\lstset{style=customcode}

% Custom Commands
\newcommand{\hilight}[1]{\colorbox{myorange!30}{#1}}
\newcommand{\source}[1]{\vspace{0.2cm}\hfill{\tiny\textcolor{mygray}{Source: #1}}}
\newcommand{\concept}[1]{\textcolor{myblue}{\textbf{#1}}}
\newcommand{\separator}{\begin{center}\rule{0.5\linewidth}{0.5pt}\end{center}}

% Footer and Navigation Setup
\setbeamertemplate{footline}{
  \leavevmode%
  \hbox{%
  \begin{beamercolorbox}[wd=.3\paperwidth,ht=2.25ex,dp=1ex,center]{author in head/foot}%
    \usebeamerfont{author in head/foot}\insertshortauthor
  \end{beamercolorbox}%
  \begin{beamercolorbox}[wd=.5\paperwidth,ht=2.25ex,dp=1ex,center]{title in head/foot}%
    \usebeamerfont{title in head/foot}\insertshorttitle
  \end{beamercolorbox}%
  \begin{beamercolorbox}[wd=.2\paperwidth,ht=2.25ex,dp=1ex,center]{date in head/foot}%
    \usebeamerfont{date in head/foot}
    \insertframenumber{} / \inserttotalframenumber
  \end{beamercolorbox}}%
  \vskip0pt%
}

% Turn off navigation symbols
\setbeamertemplate{navigation symbols}{}

% Title Page Information
\title[Academic Presentation on Random Forests]{Chapter 6: Supervised Learning Techniques - Random Forest}
\subtitle{Understanding Random Forest Algorithm}
\author[J. Smith]{John Smith, Ph.D.}
\institute[University Name]{
  Department of Computer Science\\
  University Name\\
  \vspace{0.3cm}
  Email: email@university.edu\\
  Website: www.university.edu
}
\date{\today}

% Document Start
\begin{document}

\frame{\titlepage}

\begin{frame}[fragile]
    \frametitle{Introduction to Random Forests - Overview}
    \begin{block}{What is Random Forest?}
        Random Forest is an ensemble learning method primarily used for classification and regression tasks. It operates by constructing multiple decision trees during training and outputting the mode (classification) or mean prediction (regression) of the individual trees.
    \end{block}
    
    \begin{block}{Key Concepts}
        \begin{itemize}
            \item \textbf{Ensemble Learning:} Combines multiple models to improve accuracy and mitigate overfitting.
            \item \textbf{Decision Trees:} Each tree is built using a random subset of the training data and a random subset of features at each split.
        \end{itemize}
    \end{block}
\end{frame}

\begin{frame}[fragile]
    \frametitle{Introduction to Random Forests - Significance}
    \begin{block}{Significance in Supervised Learning}
        \begin{itemize}
            \item \textbf{Versatility:} Can handle large datasets with higher dimensionality.
            \item \textbf{Robustness:} Less prone to overfitting than individual decision trees, thanks to its averaging approach.
            \item \textbf{Feature Importance:} Provides insights into the importance of different features used in predictions.
        \end{itemize}
    \end{block}
\end{frame}

\begin{frame}[fragile]
    \frametitle{Introduction to Random Forests - Application Example}
    \begin{block}{Example of Random Forest Application}
        Consider a scenario where we are trying to predict whether a patient has a disease based on several health indicators such as age, blood pressure, and cholesterol levels:
        \begin{enumerate}
            \item The Random Forest algorithm creates multiple decision trees using different samples of the health indicators.
            \item Each tree predicts independently whether the patient has the disease.
            \item The final decision is based on the majority vote of all trees in the forest.
        \end{enumerate}
    \end{block}
\end{frame}

\begin{frame}[fragile]
    \frametitle{Introduction to Random Forests - Key Points and Prediction Formulas}
    \begin{block}{Key Points to Emphasize}
        \begin{itemize}
            \item \textbf{Randomness:} Random selection for data points and features improves accuracy and prevents overfitting.
            \item \textbf{Hyperparameters:} Important parameters include the number of trees (n\_estimators) and the maximum depth of each tree (max\_depth). Tuning these can optimize performance.
        \end{itemize}
    \end{block}
    
    \begin{block}{Formulas for Random Forest Prediction}
        For classification:
        \[
        \hat{y}_{\text{final}} = \text{mode}(\hat{y}_1, \hat{y}_2, \dots, \hat{y}_N)
        \]
        For regression:
        \[
        \hat{y}_{\text{final}} = \frac{1}{N} \sum_{i=1}^{N} \hat{y}_i
        \]
    \end{block}
\end{frame}

\begin{frame}[fragile]{What is Random Forest?}
    \begin{block}{Definition}
        Random Forest is an \textbf{ensemble learning method} primarily used for classification and regression tasks. 
        It operates by constructing multiple decision trees during training and outputs either the mode (for classification) or mean prediction (for regression) of the individual trees' outputs. 
        This approach aims to improve prediction accuracy and control overfitting compared to a single decision tree.
    \end{block}
\end{frame}

\begin{frame}[fragile]{What is Random Forest? - Basic Explanation}
    \begin{itemize}
        \item \textbf{Ensemble Learning:} Techniques that create a model by combining multiple base learners to improve overall performance. 
        Random Forest is effective by combining many decision trees to produce more reliable results.
        
        \item \textbf{Decision Trees:} The foundational units of Random Forest that model relationships between input features and the target variable by splitting data based on feature values.
        
        \item \textbf{Bootstrapping:} A technique that Random Forest employs to build each tree on a random subset of the training data, enhancing model diversity.
        
        \item \textbf{Feature Randomness:} Each tree is constructed using a random subset of features at each split, promoting diversity among the trees.
    \end{itemize}
\end{frame}

\begin{frame}[fragile]{What is Random Forest? - Example}
    \begin{block}{Example}
        Imagine predicting whether a patient has a specific disease based on several medical tests:
        \begin{enumerate}
            \item \textbf{Training Data:} Gather patient records with features like age, blood pressure, and test results.
            \item \textbf{Building Trees:} Create numerous decision trees using different samples of the training set and randomly selecting features.
            \item \textbf{Classifying New Patient:} Each tree makes a prediction. The final prediction is made by majority voting across all trees.
        \end{enumerate}
    \end{block}
\end{frame}

\begin{frame}[fragile]{What is Random Forest? - Key Points}
    \begin{itemize}
        \item \textbf{Robustness:} Random Forest is robust against overfitting, especially with larger datasets, by synthesizing predictions from many trees.
        
        \item \textbf{Feature Importance:} It can assess the importance of different features, helping to identify the most significant variables impacting predictions.
        
        \item \textbf{Versatility:} Applicable for both classification and regression problems, making it a versatile tool in machine learning.
    \end{itemize}
\end{frame}

\begin{frame}[fragile]{What is Random Forest? - Conclusion}
    \begin{block}{Conclusion}
        Random Forest is a powerful technique in supervised learning that combines the strengths of multiple decision trees, 
        leading to high accuracy and robustness in predictions. 
        Understanding this ensemble method lays a strong foundation for utilizing more complex models and enhances predictive performance in various applications.
    \end{block}
\end{frame}

\begin{frame}[fragile]{What is Random Forest? - Example Code Snippet}
    \begin{lstlisting}[language=Python]
from sklearn.ensemble import RandomForestClassifier

# Creating a random forest model
model = RandomForestClassifier(n_estimators=100, random_state=42)

# Fitting the model to training data
model.fit(X_train, y_train)

# Making predictions
predictions = model.predict(X_test)
    \end{lstlisting}
\end{frame}

\begin{frame}[fragile]
    \frametitle{Foundation of Random Forest - Understanding Decision Trees}
    \begin{block}{What is a Decision Tree?}
        A Decision Tree is a flowchart-like structure used to make decisions based on a series of questions. 
        \begin{itemize}
            \item Each internal node represents a feature (or attribute).
            \item Each branch represents a decision rule.
            \item Each leaf node represents an outcome or label (in classification tasks).
        \end{itemize}
    \end{block}

    \begin{block}{Key Characteristics}
        \begin{itemize}
            \item \textbf{Easy to Interpret:} Resembles human decision-making.
            \item \textbf{Non-Linear Relationships:} No requirement for linear relationships between features and target variable.
            \item \textbf{Handling Various Data Types:} Capable of processing both numerical and categorical data.
        \end{itemize}
    \end{block}
\end{frame}

\begin{frame}[fragile]
    \frametitle{Foundation of Random Forest - Decision Tree Example}
    \begin{block}{Example of a Decision Tree Structure}
        Consider a decision tree for classifying whether an individual plays tennis based on weather conditions:
        \begin{verbatim}
                                [Outlook]
                             /      |      \
                        Sunny   Overcast   Rain
                       /  |  \                 
                  [Humidity]                  (Play Tennis = Yes)
                 /    \
             High     Normal  
             / \       / \
         (No) (Yes) (Yes) (Yes)
        \end{verbatim}
    \end{block}

    \begin{block}{Decision Path}
        - The tree starts by examining the "Outlook" attribute.
        - Each path down the tree leads to a decision: "Yes" (Play Tennis) or "No" (Do Not Play Tennis).
    \end{block}
\end{frame}

\begin{frame}[fragile]
    \frametitle{Foundation of Random Forest - Role of Decision Trees}
    \begin{block}{Role of Decision Trees in Random Forest}
        Random Forest uses an ensemble learning approach based on decision trees:
        \begin{itemize}
            \item \textbf{Multiple Trees:} Creates a collection of decision trees to improve accuracy.
            \item \textbf{Random Sampling:} Employs bootstrap aggregating (bagging) for training each tree.
            \item \textbf{Feature Randomness:} Randomly selects a subset of features during training.
        \end{itemize}
    \end{block}

    \begin{block}{Benefits of Random Forest}
        \begin{itemize}
            \item \textbf{Reduction of Overfitting:} Averages predictions from many trees.
            \item \textbf{Improved Accuracy:} Ensemble method outperforms single trees on complex datasets.
        \end{itemize}
    \end{block}
    
    \begin{block}{Conclusion}
        The foundation of Random Forest is intrinsically linked to decision trees, which enhance predictive power while reducing variance.
    \end{block}
\end{frame}

\begin{frame}[fragile]
    \frametitle{How Random Forest Works - Overview}
    \begin{itemize}
        \item Random Forest is an ensemble learning method for classification and regression tasks.
        \item It constructs multiple decision trees during training.
        \item Outputs the mode of classes (classification) or mean prediction (regression).
        \item Enhances accuracy and controls overfitting.
    \end{itemize}
\end{frame}

\begin{frame}[fragile]
    \frametitle{Key Concepts of Random Forest}
    \begin{enumerate}
        \item \textbf{Ensemble Learning:}
            \begin{itemize}
                \item Combines multiple models for improved performance.
                \item Reduces variance and bias, leading to robust outcomes.
            \end{itemize}
            
        \item \textbf{Decision Trees:}
            \begin{itemize}
                \item Base learners that split data based on feature values.
                \item Prone to overfitting, especially on noisy datasets.
            \end{itemize}
            
        \item \textbf{Bootstrap Aggregating (Bagging):}
            \begin{itemize}
                \item Trains multiple models on different data subsets.
                \item Samples 'n' observations with replacement for training.
                \item Enhances diversity and reduces overfitting.
            \end{itemize}
    \end{enumerate}
\end{frame}

\begin{frame}[fragile]
    \frametitle{How Random Forest Works - Step-by-Step}
    \begin{enumerate}
        \item \textbf{Data Sampling:}
            \begin{itemize}
                \item Select 70 samples from 100 available (with replacement).
                \item Create multiple subsets for different trees.
            \end{itemize}
            
        \item \textbf{Tree Building:}
            \begin{itemize}
                \item Each tree is built independently.
                \item Random features considered at each node for splitting.
            \end{itemize}

        \item \textbf{Voting/Averaging:}
            \begin{itemize}
                \item Classification: Each tree votes; the majority class wins.
                \item Regression: Average predictions from all trees.
            \end{itemize}
    \end{enumerate}
\end{frame}

\begin{frame}
    \frametitle{Building a Random Forest Model}
    \textbf{Overview:}
    
    \begin{itemize}
        \item Constructing a Random Forest model involves several key steps.
        \item Versatile algorithm used for both classification and regression tasks.
    \end{itemize}
\end{frame}

\begin{frame}[fragile]
    \frametitle{1. Data Preparation}
    \begin{itemize}
        \item \textbf{Collect Data:} Gather relevant datasets from public databases, company records, etc.
        
        \item \textbf{Exploratory Data Analysis (EDA):} Visualize and summarize data.
        \begin{block}{Example}
            Analyzing features for predicting house prices: size, location, number of rooms.
        \end{block}
        
        \item \textbf{Data Cleaning:} Handle missing values, duplicates, and outliers.
        \begin{itemize}
            \item \textbf{Handling Missing Values:} Strategies include:
            \begin{itemize}
                \item Elimination
                \item Imputation (mean/mode/median)
                \item Predictive modeling
            \end{itemize}
        \end{itemize}
        
        \item \textbf{Feature Selection/Engineering:} Choose relevant features or create new ones.
        \begin{block}{Example}
            Extracting 'month' and 'year' from 'date sold' to capture seasonal trends.
        \end{block}
    \end{itemize}
\end{frame}

\begin{frame}[fragile]
    \frametitle{2. Model Building and Evaluation}
    \begin{enumerate}
        \item \textbf{Splitting the Dataset:}
        \begin{itemize}
            \item Train-Test Split (commonly 70\%-30\%).
            \end{itemize}
            \begin{lstlisting}[language=Python]
from sklearn.model_selection import train_test_split

X_train, X_test, y_train, y_test = train_test_split(X, y, test_size=0.3, random_state=42)
            \end{lstlisting}
        
        \item \textbf{Model Building:}
        \begin{itemize}
            \item Import Random Forest Module.
            \begin{lstlisting}[language=Python]
from sklearn.ensemble import RandomForestRegressor  # For regression
from sklearn.ensemble import RandomForestClassifier   # For classification
            \end{lstlisting}
            \item Initialize the Model.
            \begin{lstlisting}[language=Python]
model = RandomForestRegressor(n_estimators=100, random_state=42)  # Example for regression
            \end{lstlisting}
            \item Fit the Model.
            \begin{lstlisting}[language=Python]
model.fit(X_train, y_train)
            \end{lstlisting}
        \end{itemize}
        
        \item \textbf{Model Evaluation:}
        \begin{itemize}
            \item Predict values using the test dataset.
            \begin{lstlisting}[language=Python]
predictions = model.predict(X_test)
            \end{lstlisting}
            \item Evaluate performance using metrics:
            \begin{lstlisting}[language=Python]
from sklearn.metrics import mean_squared_error

mse = mean_squared_error(y_test, predictions)
rmse = mse**0.5
            \end{lstlisting}
        \end{itemize}
    \end{enumerate}
\end{frame}

\begin{frame}[fragile]
    \frametitle{4. Feature Importance}
    \begin{itemize}
        \item Analyze feature contributions to predictions.
        \begin{lstlisting}[language=Python]
importances = model.feature_importances_
        \end{lstlisting}
    \end{itemize}

    \textbf{Key Points to Emphasize:}
    \begin{itemize}
        \item Random Forest combines many decision trees for enhanced accuracy and to prevent overfitting.
        \item Proper data preparation is critical—garbage in, garbage out.
        \item Assessing model performance using unseen (test) data is vital for real-world applicability.
    \end{itemize}

    \textbf{Visuals \& Diagrams:}
    \begin{itemize}
        \item Include a workflow diagram illustrating data collection, preparation, model training, and evaluation.
    \end{itemize}
\end{frame}

\begin{frame}[fragile]
    \frametitle{Hyperparameters in Random Forest}
    \begin{block}{Overview of Hyperparameters}
        Hyperparameters are critical in constructing a Random Forest model as they influence model performance, stability, and interpretability. We will explore key hyperparameters and their effects.
    \end{block}
\end{frame}

\begin{frame}[fragile]
    \frametitle{Key Hyperparameters - Part 1}
    \begin{enumerate}
        \item \textbf{Number of Trees} (\texttt{n\_estimators}):
        \begin{itemize}
            \item \textbf{Definition}: Number of decision trees in the forest.
            \item \textbf{Impact}:
            \begin{itemize}
                \item More trees generally improve model accuracy due to the ensemble effect.
                \item Excessively high values can lead to diminishing returns and increased computational cost.
            \end{itemize}
            \item \textbf{Example}: 
            \begin{itemize}
                \item \texttt{n\_estimators=100} results in 100 trees averaging their predictions.
                \item \textbf{Suggested Range}: 50 to 500.
            \end{itemize}
        \end{itemize}
        
    \end{enumerate}
\end{frame}

\begin{frame}[fragile]
    \frametitle{Key Hyperparameters - Part 2}
    \begin{enumerate}
        \setcounter{enumi}{1}
        \item \textbf{Maximum Depth} (\texttt{max\_depth}):
        \begin{itemize}
            \item \textbf{Definition}: Maximum depth of each decision tree.
            \item \textbf{Impact}:
            \begin{itemize}
                \item Controls overfitting; deeper trees may overfit while shallower may underfit.
            \end{itemize}
            \item \textbf{Example}: 
            \begin{itemize}
                \item \texttt{max\_depth=5} allows at most 5 levels.
                \item \textbf{Suggested Range}: 1 to 30.
            \end{itemize}
        \end{itemize}
        
        \item \textbf{Minimum Samples Split} (\texttt{min\_samples\_split}):
        \begin{itemize}
            \item \textbf{Definition}: Minimum number of samples required to split an internal node.
            \item \textbf{Impact}: 
            \begin{itemize}
                \item Higher values can prevent overly complex trees and overfitting.
            \end{itemize}
            \item \textbf{Example}: 
            \begin{itemize}
                \item \texttt{min\_samples\_split=10} requires at least 10 samples to split a node.
            \end{itemize}
        \end{itemize}
    \end{enumerate}
\end{frame}

\begin{frame}[fragile]
    \frametitle{Conclusion and Key Points}
    \begin{itemize}
        \item Proper tuning of hyperparameters significantly affects model performance.
        \item Techniques such as Cross-Validation or Grid Search can help identify optimal hyperparameters.
        \item Balancing model accuracy, interpretability, and computational efficiency is vital.
    \end{itemize}
    \begin{block}{Visual Aid Suggestions}
        Use diagrams to illustrate:
        \begin{itemize}
            \item Decision Tree Structure showing depth influence.
            \item Performance Graph on how the number of trees affects accuracy.
        \end{itemize}
    \end{block}
\end{frame}

\begin{frame}[fragile]
    \frametitle{Advantages of Random Forest - Overview}
    Random Forest (RF) is an ensemble learning method that combines multiple decision trees to enhance prediction accuracy and control overfitting. Here, we will explore some key advantages of Random Forest, making it a popular choice for various machine learning tasks.
\end{frame}

\begin{frame}[fragile]
    \frametitle{Advantages of Random Forest - High Accuracy}
    \begin{block}{High Accuracy}
        \begin{itemize}
            \item Random Forests typically provide high accuracy due to their ensemble nature. 
            \item By averaging predictions from multiple trees, they can reduce variance and improve model generalization.
        \end{itemize}
    \end{block}
    \begin{block}{Example}
        In a classification task predicting customer churn based on features like age, income, and usage, a Random Forest model can outperform a single decision tree by effectively capturing complex patterns in the data.
    \end{block}
\end{frame}

\begin{frame}[fragile]
    \frametitle{Advantages of Random Forest - Robustness to Overfitting}
    \begin{block}{Robustness to Overfitting}
        \begin{itemize}
            \item Individual decision trees often suffer from overfitting, where they model the noise in the training data.
            \item Random Forest mitigates this by averaging the results of several trees trained on random subsets of the data.
            \item The "bootstrapping" method involves sampling the training data with replacement to create diverse trees, minimizing overfitting.
        \end{itemize}
    \end{block}
    \begin{block}{Illustration}
        A single decision tree might perfectly fit a dataset on house prices (overfitting), but a Random Forest averages many trees' outputs, resulting in a more generalized model.
    \end{block}
\end{frame}

\begin{frame}[fragile]
    \frametitle{Advantages of Random Forest - Handling Missing Values}
    \begin{block}{Handling Missing Values}
        \begin{itemize}
            \item Random Forest can handle missing values without the need for imputation.
            \item It uses surrogate splits to maintain predictive accuracy even when certain features are absent.
        \end{itemize}
    \end{block}
    \begin{block}{Example}
        In a medical diagnosis model, if a patient's blood pressure data is missing, a Random Forest can still make predictions using other available features such as age and cholesterol levels.
    \end{block}
\end{frame}

\begin{frame}[fragile]
    \frametitle{Advantages of Random Forest - Feature Importance and Versatility}
    \begin{block}{Feature Importance Measurement}
        \begin{itemize}
            \item Random Forest provides insights into the importance of different features used in the model.
            \item This is useful for feature selection and helps in interpreting the model, guiding decision-making or investigations into significant factors.
        \end{itemize}
    \end{block}
    \begin{block}{Versatility}
        \begin{itemize}
            \item Random Forest can be used for both classification and regression tasks, making it a versatile tool.
            \item It can predict both customer churn (classification) and house prices (regression) with little modification.
        \end{itemize}
    \end{block}
\end{frame}

\begin{frame}[fragile]
    \frametitle{Advantages of Random Forest - Conclusion and Key Takeaways}
    \begin{block}{Conclusion}
        Understanding these advantages illustrates why Random Forest is a favored technique in supervised learning. Its ability to achieve high accuracy and maintain robustness against overfitting makes it suitable for various applications.
    \end{block}
    \begin{block}{Key Takeaways}
        \begin{itemize}
            \item High accuracy through ensemble learning.
            \item Robustness against overfitting with multiple decision trees.
            \item Ability to handle missing values directly.
            \item Insights into feature importance.
            \item Versatility in both classification and regression tasks.
        \end{itemize}
    \end{block}
\end{frame}

\begin{frame}[fragile]
    \frametitle{Code Snippet - Random Forest Example}
    \begin{lstlisting}[language=Python]
from sklearn.ensemble import RandomForestClassifier

# Example of creating a Random Forest model
rf_model = RandomForestClassifier(n_estimators=100, max_depth=10)
rf_model.fit(X_train, y_train)  # Fit the model on training data
predictions = rf_model.predict(X_test)  # Make predictions
    \end{lstlisting}
\end{frame}

\begin{frame}[fragile]
    \frametitle{Limitations of Random Forest}
    \begin{block}{Overview}
        Examination of the limitations and potential pitfalls of using Random Forest models.
    \end{block}
\end{frame}

\begin{frame}[fragile]
    \frametitle{Limitations - Interpretability and Resources}
    \begin{enumerate}
        \item \textbf{Interpretability Challenges}
            \begin{itemize}
                \item \textbf{Complexity:} Composed of many decision trees, making it less interpretable than simpler models (e.g., linear regression).
                \item \textbf{Feature Importance:} Calculating feature importances is possible, but explaining their contributions can be non-intuitive.
            \end{itemize}
            \textit{Example:} In healthcare, it's challenging to convey which patient characteristics are most influential.

        \item \textbf{Computational Resources}
            \begin{itemize}
                \item \textbf{High Memory and Processing Demand:} Training multiple trees can require significant resources, especially with large datasets.
                \item \textbf{Longer Training Times:} The number of trees affects training time, which can be a limitation for real-time applications.
            \end{itemize}
            \textit{Key Point:} Always consider computational capabilities and dataset size when opting for Random Forest.
    \end{enumerate}
\end{frame}

\begin{frame}[fragile]
    \frametitle{Limitations - Overfitting, Extrapolation, and Unstructured Data}
    \begin{enumerate}
        \setcounter{enumi}{2}
        \item \textbf{Overfitting with Noisy Data}
            \begin{itemize}
                \item \textbf{Sensitivity to Noisy Features:} Can fit to noise rather than underlying patterns, especially with deep trees.
                \item \textbf{Balanced Data Importance:} May prioritize the majority class in imbalanced datasets, leading to biased predictions.
            \end{itemize}
            \textit{Illustration:} Complex trees may focus on irrelevant features rather than key attributes.

        \item \textbf{Limitations in Extrapolation}
            \begin{itemize}
                \item \textbf{Less Effective for Extrapolation:} Poor at predicting values outside the range of training data.
            \end{itemize}
            \textit{Example:} A model trained on local house prices may struggle with predictions for properties outside that neighborhood.

        \item \textbf{Difficulty with Unstructured Data}
            \begin{itemize}
                \item \textbf{Less Effective on Images/Texts:} Not ideal for unstructured data compared to deep learning methods (e.g., CNNs for images).
            \end{itemize}
    \end{enumerate}
\end{frame}

\begin{frame}[fragile]
    \frametitle{Conclusion and Key Takeaway}
    \begin{block}{Conclusion}
        While Random Forests are robust models suitable for various applications, it is crucial to evaluate their limitations actively when implementing them.
    \end{block}
    
    \begin{block}{Key Takeaway}
        Always weigh the advantages of Random Forest against its limitations. The best model depends on the specifics of the data and the goals of the analysis.
    \end{block}
\end{frame}

\begin{frame}[fragile]
    \frametitle{Performance Metrics - Introduction}
    Evaluating the performance of a Random Forest model is crucial for understanding its effectiveness on unseen data. 
    In this section, we will cover key performance metrics:
    \begin{itemize}
        \item Accuracy
        \item Precision
        \item Recall
        \item F1-Score
    \end{itemize}
\end{frame}

\begin{frame}[fragile]
    \frametitle{Performance Metrics - Accuracy}
    \begin{block}{Accuracy}
        \textbf{Definition:} Accuracy measures the ratio of correctly predicted instances to the total instances in the dataset.
        \begin{equation}
            \text{Accuracy} = \frac{\text{True Positives} + \text{True Negatives}}{\text{Total Instances}}
        \end{equation}

        \textbf{Example:} If a Random Forest model classifies 90 out of 100 test instances correctly, the accuracy is:
        \begin{equation}
            \text{Accuracy} = \frac{90}{100} = 0.90 \quad \text{or } 90\%
        \end{equation}
    \end{block}
\end{frame}

\begin{frame}[fragile]
    \frametitle{Performance Metrics - Precision and Recall}
    \begin{block}{Precision}
        \textbf{Definition:} Precision indicates the ratio of true positive predictions to the total predicted positives. This metric is vital when the costs of false positives are high.
        \begin{equation}
            \text{Precision} = \frac{\text{True Positives}}{\text{True Positives} + \text{False Positives}}
        \end{equation}

        \textbf{Example:} If out of 50 predicted positive instances, 40 are true positives:
        \begin{equation}
            \text{Precision} = \frac{40}{50} = 0.80 \quad \text{or } 80\%
        \end{equation}
    \end{block}

    \begin{block}{Recall (Sensitivity)}
        \textbf{Definition:} Recall measures the ratio of true positive predictions to the actual positives. It’s essential when the cost of false negatives is high.
        \begin{equation}
            \text{Recall} = \frac{\text{True Positives}}{\text{True Positives} + \text{False Negatives}}
        \end{equation}

        \textbf{Example:} If there are 60 actual positive instances and the model correctly identifies 50 of them:
        \begin{equation}
            \text{Recall} = \frac{50}{60} \approx 0.83 \quad \text{or } 83\%
        \end{equation}
    \end{block}
\end{frame}

\begin{frame}[fragile]
    \frametitle{Performance Metrics - F1-Score}
    \begin{block}{F1-Score}
        \textbf{Definition:} The F1-score is the harmonic mean of Precision and Recall, balancing both metrics. It’s particularly useful in imbalanced datasets.
        \begin{equation}
            F1 = 2 \times \frac{\text{Precision} \times \text{Recall}}{\text{Precision} + \text{Recall}}
        \end{equation}
        
        \textbf{Example:} If Precision is 80\% and Recall is 83\%:
        \begin{equation}
            F1 \approx 0.815 \quad \text{or } 81.5\%
        \end{equation}
    \end{block}

    \textbf{Key Points to Emphasize:}
    \begin{itemize}
        \item Consider specific problem requirements when choosing a metric.
        \item In imbalanced cases, rely more on Precision, Recall, and F1-Score.
        \item Understand trade-offs: optimizing one can decrease the other.
    \end{itemize}
\end{frame}

\begin{frame}[fragile]
    \frametitle{Performance Metrics - Summary and Closing}
    These performance metrics are foundational in assessing the efficacy of Random Forest models, helping answer critical questions about model trustworthiness.

    \textbf{Closing Note:} 
    In the next slide, we will explore how Random Forest determines feature importance and the insights it provides for model interpretation.
\end{frame}

\begin{frame}[fragile]
    \frametitle{Feature Importance in Random Forest}
    \begin{block}{Understanding Feature Importance}
        Feature importance helps determine the relevance of features in predicting the target variable using a Random Forest model. It provides insights into which attributes influence the model's predictive performance significantly.
    \end{block}
\end{frame}

\begin{frame}[fragile]
    \frametitle{How Random Forest Measures Feature Importance}
    \begin{itemize}
        \item Random Forest evaluates feature importance based on:
            \begin{itemize}
                \item How frequently each feature is used for splits in the trees.
                \item The improvement in model accuracy due to those splits.
            \end{itemize}
        \item Two popular methods:
            \begin{enumerate}
                \item \textbf{Mean Decrease Impurity (Gini Importance):}
                \begin{itemize}
                    \item Measures total decrease in impurity from a feature, averaged across all trees.
                    \item Formula for Gini Index (impurity):
                    \begin{equation}
                        Gini(p) = 1 - \sum_{i=1}^{n} (p_i)^2
                    \end{equation}
                    (where \(p_i\) is the probability of each class)
                \end{itemize}
                \item \textbf{Mean Decrease Accuracy (Permutation Importance):}
                \begin{itemize}
                    \item Measures model accuracy decrease when feature values are permuted.
                \end{itemize}
            \end{enumerate}
    \end{itemize}
\end{frame}

\begin{frame}[fragile]
    \frametitle{Implications for Model Interpretation}
    \begin{itemize}
        \item \textbf{Feature Selection:} 
            \begin{itemize}
                \item Identifying important features leads to reducing complexity by excluding insignificant variables.
            \end{itemize}
        \item \textbf{Model Visualization:} 
            \begin{itemize}
                \item Feature importance values can be visualized, aiding in the understanding of model decisions.
            \end{itemize}
        \item \textbf{Domain Relevance:} 
            \begin{itemize}
                \item Confirms model alignment with domain knowledge, enhancing confidence in predictions.
            \end{itemize}
    \end{itemize}
\end{frame}

\begin{frame}[fragile]
    \frametitle{Example}
    Consider a Random Forest model predicting customer purchases based on features:
    \begin{itemize}
        \item Age: 0.45 (most important)
        \item Income: 0.30
        \item Previous Purchase History: 0.25
    \end{itemize}
    This indicates age is the most influential variable, guiding targeted marketing strategies.
\end{frame}

\begin{frame}[fragile]
    \frametitle{Key Points}
    \begin{itemize}
        \item Feature importance provides critical insights for model interpretation and feature engineering.
        \item Random Forest uses both Gini importance and permutation importance to assess feature impacts.
        \item Understanding impactful features enhances model explainability and trustworthiness.
    \end{itemize}
    This understanding strengthens the analysis of Random Forest models and guides practical feature selection and deployment decisions.
\end{frame}

\begin{frame}[fragile]
    \frametitle{Implementation in Python - Overview}
    \begin{block}{Overview of Random Forest}
        Random Forest is an ensemble learning technique used for both classification and regression tasks. 
        It constructs multiple decision trees during training and outputs the mode (classification) or mean prediction (regression) of the individual trees. This process improves model accuracy and reduces overfitting.
    \end{block}
\end{frame}

\begin{frame}[fragile]
    \frametitle{Implementation in Python - Key Libraries}
    \begin{block}{Key Python Libraries for Implementation}
        \begin{itemize}
            \item \textbf{Scikit-learn}: A popular machine learning library for data mining and analysis.
            \item \textbf{NumPy}: Library for numerical computations essential for handling arrays and matrices.
            \item \textbf{Pandas}: A data manipulation library that allows for easy handling of structured data.
        \end{itemize}
    \end{block}
\end{frame}

\begin{frame}[fragile]
    \frametitle{Implementation in Python - Basic Steps}
    \begin{block}{Basic Steps for Implementing Random Forest}
        \begin{enumerate}
            \item \textbf{Import Required Libraries}
            \begin{lstlisting}[language=Python]
import pandas as pd
import numpy as np
from sklearn.model_selection import train_test_split
from sklearn.ensemble import RandomForestClassifier
from sklearn.metrics import accuracy_score, classification_report
            \end{lstlisting}

            \item \textbf{Load Your Data}
            \begin{lstlisting}[language=Python]
data = pd.read_csv('data.csv')
            \end{lstlisting}

            \item \textbf{Preprocess the Data}
            \begin{itemize}
                \item Handle missing values, encode categorical variables, and normalize numerical features.
            \end{itemize}

            \item \textbf{Split the Data into Features and Target}
            \begin{lstlisting}[language=Python]
X = data.drop('target', axis=1)  # Feature set
y = data['target']  # Target variable
            \end{lstlisting}

            \item \textbf{Train-Test Split}
            \begin{lstlisting}[language=Python]
X_train, X_test, y_train, y_test = train_test_split(X, y, test_size=0.2, random_state=42)
            \end{lstlisting}

            \item \textbf{Initialize and Train the Random Forest Model}
            \begin{lstlisting}[language=Python]
model = RandomForestClassifier(n_estimators=100, random_state=42)
model.fit(X_train, y_train)
            \end{lstlisting}

            \item \textbf{Make Predictions}
            \begin{lstlisting}[language=Python]
y_pred = model.predict(X_test)
            \end{lstlisting}

            \item \textbf{Evaluate Model Performance}
            \begin{lstlisting}[language=Python]
accuracy = accuracy_score(y_test, y_pred)
print(f'Accuracy: {accuracy:.2f}')
print(classification_report(y_test, y_pred))
            \end{lstlisting}
        \end{enumerate}
    \end{block}
\end{frame}

\begin{frame}[fragile]
    \frametitle{Implementation in Python - Use Case}
    \begin{block}{Example Use Case}
        Using Random Forest to predict customer churn based on features like age, purchase history, and engagement metrics. Visualizing feature importance can help assess which factors significantly influence customer decisions.
    \end{block}

    \begin{block}{Key Points to Emphasize}
        \begin{itemize}
            \item Random Forest reduces overfitting compared to individual decision trees.
            \item The ensemble nature allows the model to generalize better on new data.
            \item Always evaluate your model using metrics such as accuracy, precision, recall, and F1-score.
        \end{itemize}
    \end{block}
\end{frame}

\begin{frame}
    \frametitle{Case Study: Random Forest in Use}
    \begin{block}{Real-world Application of Random Forests}
        Random Forest is an ensemble learning technique widely used for classification and regression tasks. This section explores its practical application in the healthcare industry, specifically in predicting patient outcomes based on various health parameters.
    \end{block}
\end{frame}

\begin{frame}
    \frametitle{Context}
    \begin{itemize}
        \item \textbf{Dataset:} Medical records from patients diagnosed with diabetes.
        \begin{itemize}
            \item Features include Age, Gender, Body Mass Index (BMI), Blood Pressure, Blood Glucose Levels, Cholesterol Levels, and Family History of Diabetes.
        \end{itemize}
        \item \textbf{Objective:} Predict whether a patient will develop diabetes within the next 5 years based on their medical history and health metrics.
    \end{itemize}
\end{frame}

\begin{frame}[fragile]
    \frametitle{Steps Involved in Using Random Forest for Diabetes Prediction}
    \begin{enumerate}
        \item \textbf{Data Preparation:}
            \begin{itemize}
                \item Clean the dataset by handling missing values and outliers.
                \item Encode categorical variables such as gender using techniques like one-hot encoding.
            \end{itemize}
        \item \textbf{Feature Selection:} Use Random Forest's built-in feature importance scores to identify relevant features.
        \item \textbf{Model Training:}
            \begin{itemize}
                \item Split the data (80\% training, 20\% testing).
                \item Train the model using \texttt{scikit-learn}.
            \end{itemize}
    \end{enumerate}
\end{frame}

\begin{frame}[fragile]
    \frametitle{Model Training Code Snippet}
    \begin{lstlisting}[language=Python]
from sklearn.model_selection import train_test_split
from sklearn.ensemble import RandomForestClassifier
from sklearn.metrics import accuracy_score

# Load the dataset
# Assuming 'df' is a DataFrame containing the dataset
X = df.drop('DiabetesOutcome', axis=1)
y = df['DiabetesOutcome']

# Split the data
X_train, X_test, y_train, y_test = train_test_split(X, y, test_size=0.2, random_state=42)

# Create and train the model
model = RandomForestClassifier(n_estimators=100, random_state=42)
model.fit(X_train, y_train)

# Make predictions
predictions = model.predict(X_test)

# Evaluate the model
accuracy = accuracy_score(y_test, predictions)
print(f'Model Accuracy: {accuracy:.2f}')
    \end{lstlisting}
\end{frame}

\begin{frame}
    \frametitle{Model Evaluation and Deployment}
    \begin{itemize}
        \item \textbf{Model Evaluation:} Use accuracy, precision, recall, F1-score, and confusion matrix to evaluate performance.
        \item \textbf{Deployment:} Implement the model in a clinical decision support system for informed patient management.
    \end{itemize}
\end{frame}

\begin{frame}
    \frametitle{Key Points to Emphasize}
    \begin{itemize}
        \item \textbf{Robustness:} Random Forest effectively handles continuous and categorical data and is resistant to overfitting.
        \item \textbf{Interpretability:} Provides feature importance scores for better understanding of influential factors.
        \item \textbf{Real-World Impact:} Early diabetes prediction leads to timely interventions, reducing the risk of complications.
    \end{itemize}
\end{frame}

\begin{frame}
    \frametitle{Conclusion}
    Random Forest excels in healthcare applications for informed decision-making and patient management strategies. Its versatility and ease of use in Python make it a preferred method for predictive analytics across various industries.
\end{frame}

\begin{frame}[fragile]
    \frametitle{Comparative Analysis - Overview}
    In this section, we’ll compare Random Forest, a powerful ensemble learning technique, with other commonly used supervised learning methods: Decision Trees and Logistic Regression. 
    \begin{itemize}
        \item Understanding their strengths and weaknesses will guide the selection of the best approach for a given problem.
    \end{itemize}
\end{frame}

\begin{frame}[fragile]
    \frametitle{Comparative Analysis - Random Forest vs. Decision Trees}
    \begin{block}{Random Forest}
        \begin{itemize}
            \item \textbf{Definition}: An ensemble of multiple decision trees, predicting via majority voting or averaging.
            \item \textbf{Advantages}:
                \begin{itemize}
                    \item Improved Accuracy: Reduces overfitting.
                    \item Robustness: Resilient to outliers and noise.
                    \item Feature Importance: Measures contribution of features.
                \end{itemize}
            \item \textbf{Complexity}: Higher computational cost due to averaging multiple trees.
        \end{itemize}
    \end{block}

    \begin{block}{Decision Trees}
        \begin{itemize}
            \item \textbf{Definition}: A model that splits datasets via binary decision rules.
            \item \textbf{Advantages}:
                \begin{itemize}
                    \item Interpretability: Simple visualization.
                    \item Fast Training: Quick training time.
                \end{itemize}
            \item \textbf{Disadvantages}:
                \begin{itemize}
                    \item Overfitting: Risk of overly complex models.
                \end{itemize}
        \end{itemize}
    \end{block}
\end{frame}

\begin{frame}[fragile]
    \frametitle{Comparative Analysis - Random Forest vs. Logistic Regression}
    \begin{block}{Random Forest}
        \begin{itemize}
            \item \textbf{Application}: Suitable for both classification and regression tasks, excels in complex, non-linear relationships.
            \item \textbf{Non-Linearity}: Handles feature interactions without transformation.
        \end{itemize}
    \end{block}

    \begin{block}{Logistic Regression}
        \begin{itemize}
            \item \textbf{Definition}: A method for binary classification modeling independent variables' relationship to a binary outcome via a logistic function.
            \item \textbf{Advantages}:
                \begin{itemize}
                    \item Simplicity: Easy to implement and understand for linear relationships.
                    \item Probabilistic Output: Useful for risk assessments.
                \end{itemize}
            \item \textbf{Disadvantages}:
                \begin{itemize}
                    \item Linearity Assumption: Poor performance on complex datasets.
                \end{itemize}
            \item \textbf{Key Formula}:
                \begin{equation}
                    P(Y = 1|X) = \frac{1}{1 + e^{-(\beta_0 + \beta_1 X_1 + \beta_2 X_2 + ... + \beta_n X_n)}}
                \end{equation}
        \end{itemize}
    \end{block}
\end{frame}

\begin{frame}
    \frametitle{Best Practices for Using and Tuning Random Forest Models}
    \begin{itemize}
        \item Data Preparation
        \item Feature Selection
        \item Tuning Hyperparameters
        \item Cross-Validation
        \item Addressing Overfitting
        \item Ensemble Techniques
        \item Evaluate Performance
        \item Key Points to Remember
        \item Conclusion
    \end{itemize}
\end{frame}

\begin{frame}
    \frametitle{1. Data Preparation}
    \begin{block}{Quality Over Quantity}
        Ensure to clean and preprocess your data.
        Handle missing values, outliers, and categorical variables effectively.
    \end{block}
    \begin{itemize}
        \item Example: In customer churn datasets, missing data should be imputed or removed to avoid bias.
    \end{itemize}
    
    \frametitle{2. Feature Selection}
    \begin{block}{Use Feature Importances}
        Random Forest provides feature importance scores, helping identify the most predictive variables.
    \end{block}
    \begin{lstlisting}[language=Python]
from sklearn.ensemble import RandomForestClassifier
model = RandomForestClassifier()
model.fit(X_train, y_train)
importances = model.feature_importances_
    \end{lstlisting}
\end{frame}

\begin{frame}
    \frametitle{3. Tuning Hyperparameters}
    \begin{itemize}
        \item \textbf{n\_estimators:} Start with 100 trees and increase if necessary.
        \item \textbf{max\_features:} Consider "sqrt" or "log2".
        \item \textbf{max\_depth:} Limit depth to reduce overfitting.
        \item \textbf{Tip:} Use validation sets or cross-validation to find optimal parameters.
    \end{itemize}

    \frametitle{4. Cross-Validation and Performance}
    \begin{itemize}
        \item Validate with K-Fold for robust performance assessments.
        \item Address Overfitting: Use learning curves to compare training and validation performances.
        \item Check overfitting; prune trees with max depth or min samples per leaf.
    \end{itemize}

    \begin{block}{Evaluate Performance}
        Always validate predictions using confusion matrices, F1 scores, and ROC-AUC scores.
    \end{block}
\end{frame}

\begin{frame}
    \frametitle{Key Points and Conclusion}
    \begin{itemize}
        \item Random Forests are effective for noisy datasets.
        \item Use appropriate hyperparameters based on dataset characteristics.
        \item Always validate models using robust methods like cross-validation.
    \end{itemize}
    
    \begin{block}{Conclusion}
        Implementing these best practices can enhance model performance and reliability, making Random Forest a powerful supervised learning tool.
    \end{block}
\end{frame}

\begin{frame}[fragile]
    \frametitle{Future Trends in Random Forest and Ensemble Learning Techniques - Overview}
    
    \begin{block}{Introduction}
        As machine learning evolves, Random Forest and ensemble techniques are advancing. Key trends shaping their future will be discussed.
    \end{block}
\end{frame}

\begin{frame}[fragile]
    \frametitle{Future Trends in Random Forest - Integration with Deep Learning}
    
    \begin{itemize}
        \item \textbf{Hybrid Models}: Combining Random Forests with deep learning.
        \item \textbf{Example}: Use of Random Forests for feature selection in CNN-based image classification.
    \end{itemize}
\end{frame}

\begin{frame}[fragile]
    \frametitle{Future Trends in Random Forest - Interpretability Enhancements}
    
    \begin{itemize}
        \item \textbf{Importance}: Understanding decision-making in complex models is critical.
        \item \textbf{Emerging Tools}:
            \begin{itemize}
                \item SHAP (SHapley Additive exPlanations)
                \item LIME (Local Interpretable Model-Agnostic Explanations)
            \end{itemize}
        \item \textbf{Accountability}: Increased transparency is vital for sectors like healthcare and finance.
    \end{itemize}
\end{frame}

\begin{frame}[fragile]
    \frametitle{Future Trends in Random Forest - Automated Machine Learning}
    
    \begin{itemize}
        \item \textbf{AutoML Overview}: Automates model selection and hyperparameter tuning.
        \item \textbf{Advancements}: More efficient tuning of Random Forest parameters.
        \item \textbf{Benefit}: Enhances accessibility for non-experts while improving models.
    \end{itemize}
\end{frame}

\begin{frame}[fragile]
    \frametitle{Future Trends in Random Forest - Scalability and Custom Techniques}
    
    \begin{enumerate}
        \item \textbf{Scalability}:
            \begin{itemize}
                \item Use of distributed computing (e.g., Apache Spark).
                \item GPU acceleration for faster training.
            \end{itemize}
        \item \textbf{Custom Ensemble Techniques}:
            \begin{itemize}
                \item Innovations beyond traditional methods like stacking.
                \item Using Random Forests as base learners to improve accuracy.
            \end{itemize}
    \end{enumerate}
\end{frame}

\begin{frame}[fragile]
    \frametitle{Future Trends in Random Forest - Handling Imbalanced Datasets}
    
    \begin{itemize}
        \item \textbf{Focus}: Improving Random Forest's ability to manage imbalanced datasets.
        \item \textbf{Techniques}:
            \begin{itemize}
                \item Adaptive sampling methods.
                \item Cost-sensitive learning for optimizing minority class predictions.
            \end{itemize}
    \end{itemize}
\end{frame}

\begin{frame}[fragile]
    \frametitle{Future Trends in Random Forest - Key Points & Conclusion}
    
    \begin{itemize}
        \item The evolution of Random Forests coincides with machine learning advancements.
        \item Integrating interpretability with complexity is essential.
        \item Future tools will focus on usability and performance.
    \end{itemize}
    
    \begin{block}{Conclusion}
        Random Forests will become more powerful and user-friendly, effectively addressing diverse machine learning challenges.
    \end{block}
\end{frame}

\begin{frame}[fragile]
    \frametitle{Conclusion - Overview}
    \begin{block}{Summary of Key Points}
        \begin{itemize}
            \item \textbf{Definition of Random Forest:}
              An ensemble method for classification and regression using multiple decision trees.
            \item \textbf{Mechanics of Operation:}
              \begin{itemize}
                  \item Bootstrap sampling for variance reduction.
                  \item Random subsets of features considered at each split.
              \end{itemize}
            \item \textbf{Advantages:}
              \begin{itemize}
                  \item Robustness to noise.
                  \item Effective handling of missing values.
                  \item Built-in feature importance.
              \end{itemize}
            \item \textbf{Disadvantages:}
              \begin{itemize}
                  \item Limited interpretability (black-box model).
                  \item Computationally intensive on large datasets.
              \end{itemize}
        \end{itemize}
    \end{block}
\end{frame}

\begin{frame}[fragile]
    \frametitle{Conclusion - Importance}
    \begin{block}{Importance in Supervised Learning}
        \begin{itemize}
            \item \textbf{Real-world Applications:}
              Used in finance, healthcare, and marketing for various analyses.
            \item \textbf{Competition with Other Models:}
              Performs competitively with complex models requiring less tuning.
            \item \textbf{Foundation for Advanced Learning:}
              Principles of Random Forest contribute to advanced techniques like boosting and stacking.
        \end{itemize}
    \end{block}
\end{frame}

\begin{frame}[fragile]
    \frametitle{Conclusion - Final Takeaways}
    \begin{block}{Key Points to Emphasize}
        \begin{itemize}
            \item Random Forest is versatile, blending accuracy with user-friendliness.
            \item Its strengths in dealing with unbalanced datasets are critical in practice.
        \end{itemize}
    \end{block}
    \begin{block}{Example Code Snippet}
        \begin{lstlisting}[language=Python]
from sklearn.ensemble import RandomForestClassifier
from sklearn.model_selection import train_test_split
from sklearn.datasets import load_iris

# Load dataset
data = load_iris()
X = data.data
y = data.target

# Split the dataset
X_train, X_test, y_train, y_test = train_test_split(X, y, test_size=0.2, random_state=42)

# Create a Random Forest Classifier
rf = RandomForestClassifier(n_estimators=100, random_state=42)

# Fit the model
rf.fit(X_train, y_train)

# Make predictions
predictions = rf.predict(X_test)
        \end{lstlisting}
    \end{block}
\end{frame}


\end{document}