\documentclass[aspectratio=169]{beamer}

% Theme and Color Setup
\usetheme{Madrid}
\usecolortheme{whale}
\useinnertheme{rectangles}
\useoutertheme{miniframes}

% Additional Packages
\usepackage[utf8]{inputenc}
\usepackage[T1]{fontenc}
\usepackage{graphicx}
\usepackage{booktabs}
\usepackage{listings}
\usepackage{amsmath}
\usepackage{amssymb}
\usepackage{xcolor}
\usepackage{tikz}
\usepackage{pgfplots}
\pgfplotsset{compat=1.18}
\usetikzlibrary{positioning}
\usepackage{hyperref}

% Custom Colors
\definecolor{myblue}{RGB}{31, 73, 125}
\definecolor{mygray}{RGB}{100, 100, 100}
\definecolor{mygreen}{RGB}{0, 128, 0}
\definecolor{myorange}{RGB}{230, 126, 34}
\definecolor{mycodebackground}{RGB}{245, 245, 245}

% Set Theme Colors
\setbeamercolor{structure}{fg=myblue}
\setbeamercolor{frametitle}{fg=white, bg=myblue}
\setbeamercolor{title}{fg=myblue}
\setbeamercolor{section in toc}{fg=myblue}
\setbeamercolor{item projected}{fg=white, bg=myblue}
\setbeamercolor{block title}{bg=myblue!20, fg=myblue}
\setbeamercolor{block body}{bg=myblue!10}
\setbeamercolor{alerted text}{fg=myorange}

% Set Fonts
\setbeamerfont{title}{size=\Large, series=\bfseries}
\setbeamerfont{frametitle}{size=\large, series=\bfseries}
\setbeamerfont{caption}{size=\small}
\setbeamerfont{footnote}{size=\tiny}

% Custom Commands
\newcommand{\hilight}[1]{\colorbox{myorange!30}{#1}}
\newcommand{\concept}[1]{\textcolor{myblue}{\textbf{#1}}}
\newcommand{\separator}{\begin{center}\rule{0.5\linewidth}{0.5pt}\end{center}}

% Document Start
\begin{document}

\frame{\titlepage}

\begin{frame}
    \titlepage
\end{frame}

\begin{frame}[fragile]
    \frametitle{Welcome to Week 5!}
    \begin{block}{Overview}
        This week, we dive into the world of AI tools that are essential for developing machine learning models and advanced data analysis. 
        Our focus will primarily be on industry-standard libraries: \textbf{TensorFlow} and \textbf{Scikit-learn}. 
        By the end of this week, you will be equipped with foundational skills to utilize these tools effectively in your projects.
    \end{block}
\end{frame}

\begin{frame}[fragile]
    \frametitle{Key AI Tools: TensorFlow}
    \begin{itemize}
        \item \textbf{What is it?} A powerful open-source library developed by Google for numerical computation and machine learning.
        \item \textbf{Key Features:}
        \begin{itemize}
            \item Extensive support for deep learning models.
            \item Flexibility and scalability for various applications—from research prototypes to production systems.
        \end{itemize}
        \item \textbf{Example:} Constructing a neural network to classify images or predicting time series data.
    \end{itemize}
    \begin{block}{Basic Code Snippet}
    \begin{lstlisting}[language=Python]
    import tensorflow as tf
    model = tf.keras.Sequential([
        tf.keras.layers.Dense(128, activation='relu', input_shape=(784,)),
        tf.keras.layers.Dense(10, activation='softmax')
    ])
    model.compile(optimizer='adam', loss='sparse_categorical_crossentropy', metrics=['accuracy'])
    \end{lstlisting}
    \end{block}
\end{frame}

\begin{frame}[fragile]
    \frametitle{Key AI Tools: Scikit-learn}
    \begin{itemize}
        \item \textbf{What is it?} A user-friendly open-source library for machine learning in Python, primarily used for traditional machine learning algorithms.
        \item \textbf{Key Features:}
        \begin{itemize}
            \item Simplified implementation of algorithms like classification, regression, clustering, and dimensionality reduction.
            \item Preprocessing capabilities, such as scaling and encoding.
        \end{itemize}
        \item \textbf{Example:} Utilizing decision trees to classify emails as spam or not spam.
    \end{itemize}
    \begin{block}{Basic Code Snippet}
    \begin{lstlisting}[language=Python]
    from sklearn.tree import DecisionTreeClassifier
    clf = DecisionTreeClassifier()
    clf.fit(X_train, y_train)
    predictions = clf.predict(X_test)
    \end{lstlisting}
    \end{block}
\end{frame}

\begin{frame}
    \frametitle{Learning Objectives for the Week}
    \begin{itemize}
        \item Understand the functionalities and applications of TensorFlow and Scikit-learn.
        \item Learn how to set up these libraries and preprocess data for analysis.
        \item Develop hands-on skills in building and evaluating models using both tools.
        \item Discover ethical considerations in AI tool deployment.
    \end{itemize}
    \begin{block}{Conclusion}
        By engaging with these industry-standard tools, you will not only grasp the theoretical foundations but also apply them in practical scenarios. 
        Stay tuned for the next slide, where we will outline the learning objectives in detail!
    \end{block}
\end{frame}

\begin{frame}
    \frametitle{Learning Objectives - Overview}
    In this week's exploration of AI tools, we will achieve several key learning objectives designed to enhance your understanding and practical skills. By the end of this session, you should be able to effectively utilize industry-standard AI tools and critically evaluate their capabilities and applicability.
\end{frame}

\begin{frame}
    \frametitle{Learning Objectives - Tool Utilization}
    \begin{block}{1. Tool Utilization}
        \begin{itemize}
            \item \textbf{Definition}: Understanding how to effectively use AI tools such as TensorFlow and Scikit-learn to implement machine learning models.
            \item \textbf{Illustration}:
                \begin{itemize}
                    \item \textbf{Example}: Using TensorFlow to build a simple neural network for image classification tasks.
                \end{itemize}
        \end{itemize}
    \end{block}
\end{frame}

\begin{frame}[fragile]
    \frametitle{Learning Objectives - Tool Utilization (Code Snippet)}
    \begin{block}{Code Snippet for TensorFlow}
        \begin{lstlisting}[language=Python]
import tensorflow as tf
from tensorflow import keras

# Load and preprocess the dataset
mnist = keras.datasets.mnist
(x_train, y_train), (x_test, y_test) = mnist.load_data()
x_train, x_test = x_train / 255.0, x_test / 255.0

# Build the model
model = keras.Sequential([
    keras.layers.Flatten(input_shape=(28, 28)),
    keras.layers.Dense(128, activation='relu'),
    keras.layers.Dropout(0.2),
    keras.layers.Dense(10, activation='softmax')
])

model.compile(optimizer='adam',
              loss='sparse_categorical_crossentropy',
              metrics=['accuracy'])

# Train the model
model.fit(x_train, y_train, epochs=5)
        \end{lstlisting}
    \end{block}
\end{frame}

\begin{frame}
    \frametitle{Learning Objectives - Evaluation Skills}
    \begin{block}{2. Evaluation Skills}
        \begin{itemize}
            \item \textbf{Definition}: Developing the ability to assess the performance of AI models critically.
            \item \textbf{Illustration}:
                \begin{itemize}
                    \item \textbf{Example}: Utilizing evaluation metrics such as accuracy, precision, recall, and F1 score to determine model performance.
                \end{itemize}
        \end{itemize}
    \end{block}
\end{frame}

\begin{frame}
    \frametitle{Learning Objectives - Evaluation Skills (Formula)}
    \begin{block}{Evaluation Metrics}
        \textbf{Accuracy}:
        \begin{equation}
            Accuracy = \frac{TP + TN}{TP + TN + FP + FN}
        \end{equation}
        Where:
        \begin{itemize}
            \item TP = True Positives
            \item TN = True Negatives
            \item FP = False Positives
            \item FN = False Negatives
        \end{itemize}
    \end{block}
\end{frame}

\begin{frame}
    \frametitle{Conclusion}
    By focusing on these objectives, you will gain hands-on experience using cutting-edge tools and develop the analytical skills needed to evaluate AI solutions effectively. This not only prepares you for practical application but also lays a foundational understanding of ethical considerations when deploying AI technologies.
    Keep these objectives in mind as you explore the tools and their implications in various industries.
\end{frame}

\begin{frame}[fragile]
    \frametitle{What is TensorFlow? - Introduction}
    \begin{block}{Introduction to TensorFlow}
        TensorFlow is an open-source machine learning framework developed by Google. It offers a comprehensive ecosystem for building and deploying machine learning models, simplifying workflows for both simple applications and complex neural networks.
    \end{block}
\end{frame}

\begin{frame}[fragile]
    \frametitle{What is TensorFlow? - Key Features}
    \begin{itemize}
        \item \textbf{Flexible Architecture}: Supports various platforms such as cloud, mobile, and edge devices.
        \item \textbf{High-level APIs}: User-friendly APIs like Keras simplify the building, training, and evaluation of deep learning models.
        \item \textbf{Computation Graphs}: Data flow graphs represent computations, enabling efficient execution across multiple CPUs, GPUs, and TPUs.
        \item \textbf{Ecosystem and Community}: A vast support community with a rich set of libraries and tools fosters collaboration and learning.
    \end{itemize}
\end{frame}

\begin{frame}[fragile]
    \frametitle{What is TensorFlow? - Applications and Code Example}
    \begin{block}{Applications in AI and Machine Learning}
        \begin{itemize}
            \item \textbf{Natural Language Processing (NLP)}: Develops applications such as chatbots and sentiment analysis.
            \item \textbf{Computer Vision}: Aids in image recognition and object detection tasks.
            \item \textbf{Reinforcement Learning}: Suitable for algorithms learning optimal actions through trial and error.
        \end{itemize}
    \end{block}
    
    \begin{block}{Example of TensorFlow Code}
        \begin{lstlisting}[language=Python]
import tensorflow as tf
from tensorflow import keras

# Load a dataset (e.g., MNIST)
(train_images, train_labels), (test_images, test_labels) = keras.datasets.mnist.load_data()

# Normalize the data
train_images = train_images / 255.0
test_images = test_images / 255.0

# Build the model
model = keras.Sequential([
    keras.layers.Flatten(input_shape=(28, 28)),
    keras.layers.Dense(128, activation='relu'),
    keras.layers.Dropout(0.2),
    keras.layers.Dense(10, activation='softmax')
])

# Compile the model
model.compile(optimizer='adam',
              loss='sparse_categorical_crossentropy',
              metrics=['accuracy'])

# Train the model
model.fit(train_images, train_labels, epochs=5)

# Evaluate the model
test_loss, test_accuracy = model.evaluate(test_images, test_labels)
print('\nTest accuracy:', test_accuracy)
        \end{lstlisting}
    \end{block}
\end{frame}

\begin{frame}[fragile]
    \frametitle{Hands-on Session: TensorFlow}
    \begin{block}{Introduction}
        In this practical session, we will dive into TensorFlow, a powerful library for building machine learning models. This hands-on experience will guide you through the fundamental process of creating a basic AI model, aligning with our learning objectives of understanding AI tools and their applications.
    \end{block}
\end{frame}

\begin{frame}[fragile]
    \frametitle{Objectives}
    \begin{itemize}
        \item \textbf{Understand the TensorFlow Framework:} Recognize core concepts and API usage.
        \item \textbf{Build a Basic Model:} Implement a simple AI model using TensorFlow.
        \item \textbf{Learn Through Practice:} Provide practical coding experience for AI.
    \end{itemize}
\end{frame}

\begin{frame}[fragile]
    \frametitle{Key Concepts}
    \begin{enumerate}
        \item \textbf{TensorFlow Basics}
        \begin{itemize}
            \item A flexible framework developed by Google for deep learning models.
            \item Utilizes data flow graphs for executing operations on tensors (multi-dimensional arrays).
        \end{itemize}

        \item \textbf{Neural Networks}
        \begin{itemize}
            \item Composed of layers: input, hidden, and output layers.
            \item Uses weights for connections and applies activation functions to introduce non-linearity.
        \end{itemize}
    \end{enumerate}
\end{frame}

\begin{frame}[fragile]
    \frametitle{Example Model: Creating a Simple Neural Network}
    In this session, we will create a simple neural network to classify the MNIST dataset of handwritten digits.

    \begin{block}{Setting Up the Environment}
    \begin{lstlisting}[language=Python]
    # Install TensorFlow if not already installed
    !pip install tensorflow
    \end{lstlisting}
    \end{block}

    \begin{block}{Import Required Libraries}
    \begin{lstlisting}[language=Python]
    import tensorflow as tf
    from tensorflow.keras import layers, models
    from tensorflow.keras.datasets import mnist
    \end{lstlisting}
    \end{block}
\end{frame}

\begin{frame}[fragile]
    \frametitle{Building and Training the Model}
    \begin{block}{Load and Preprocess Data}
    \begin{lstlisting}[language=Python]
    (x_train, y_train), (x_test, y_test) = mnist.load_data()
    x_train, x_test = x_train / 255.0, x_test / 255.0  # Normalize the pixel values
    \end{lstlisting}
    \end{block}

    \begin{block}{Build and Compile the Model}
    \begin{lstlisting}[language=Python]
    model = models.Sequential([
        layers.Flatten(input_shape=(28, 28)),
        layers.Dense(128, activation='relu'),
        layers.Dense(10, activation='softmax')
    ])

    model.compile(optimizer='adam',
                  loss='sparse_categorical_crossentropy',
                  metrics=['accuracy'])
    \end{lstlisting}
    \end{block}
\end{frame}

\begin{frame}[fragile]
    \frametitle{Training and Evaluating the Model}
    \begin{block}{Train the Model}
    \begin{lstlisting}[language=Python]
    model.fit(x_train, y_train, epochs=5)
    \end{lstlisting}
    \end{block}

    \begin{block}{Evaluate the Model}
    \begin{lstlisting}[language=Python]
    test_loss, test_acc = model.evaluate(x_test, y_test)
    print('\nTest accuracy:', test_acc)
    \end{lstlisting}
    \end{block}
\end{frame}

\begin{frame}[fragile]
    \frametitle{Key Points to Emphasize}
    \begin{itemize}
        \item \textbf{Normalization:} Normalizing input data significantly improves model performance.
        \item \textbf{Activation Functions:} Understand roles of functions like ReLU and softmax for learning complex patterns.
        \item \textbf{Model Evaluation:} Always evaluate performance on a separate test set to ensure generalization.
    \end{itemize}
\end{frame}

\begin{frame}[fragile]
    \frametitle{Summary}
    This hands-on session emphasizes practical aspects of building an AI model with TensorFlow. By following outlined steps, you will gain foundational skills necessary for exploring more complex models and applications in the future. Engage fully, as hands-on experience is crucial for applying theoretical knowledge to practical scenarios.
\end{frame}

\begin{frame}
    \frametitle{What is Scikit-learn?}
    \begin{block}{Introduction}
        Scikit-learn is a powerful and widely-used open-source library in Python for machine learning and data analysis, providing efficient tools for data mining.
    \end{block}
\end{frame}

\begin{frame}
    \frametitle{Key Features of Scikit-learn}
    \begin{enumerate}
        \item \textbf{User-Friendly API:} Straightforward and consistent interface for beginners.
        \item \textbf{Wide Range of Algorithms:} Includes algorithms for classification, regression, clustering, and dimensionality reduction.
        \item \textbf{Model Evaluation \& Selection:} Tools for cross-validation and performance metrics for effective model assessment.
        \item \textbf{Data Preprocessing:} Simplifies preprocessing tasks like scaling, encoding, and handling missing values.
        \item \textbf{Pipeline Support:} Create workflows that streamline data transformation and model building.
    \end{enumerate}
\end{frame}

\begin{frame}[fragile]
    \frametitle{Example: A Simple Classification Task}
    \begin{block}{Python Code}
        \begin{lstlisting}[language=Python]
from sklearn.datasets import load_iris
from sklearn.model_selection import train_test_split
from sklearn.ensemble import RandomForestClassifier
from sklearn.metrics import accuracy_score

# Load dataset
iris = load_iris()
X, y = iris.data, iris.target

# Split the data
X_train, X_test, y_train, y_test = train_test_split(X, y, test_size=0.2, random_state=42)

# Initialize and train the model
model = RandomForestClassifier()
model.fit(X_train, y_train)

# Make predictions
predictions = model.predict(X_test)

# Evaluate the model
accuracy = accuracy_score(y_test, predictions)
print(f'Accuracy: {accuracy:.2f}')
        \end{lstlisting}
    \end{block}
\end{frame}

\begin{frame}
    \frametitle{Conclusion}
    \begin{itemize}
        \item Scikit-learn is essential for efficient machine learning work.
        \item Its simplicity and robust features facilitate quick experimentation.
        \item Understanding Scikit-learn enhances data science and machine learning skills significantly.
    \end{itemize}
\end{frame}

\begin{frame}
    \frametitle{Hands-on Session: Scikit-learn}
    \begin{block}{Overview}
        In this hands-on session, we will use Scikit-learn, a prominent machine learning library in Python, to build and evaluate a machine learning model. The key objectives include:
        \begin{itemize}
            \item Gaining practical experience.
            \item Understanding the modeling process.
            \item Applying theoretical knowledge in a real-world scenario.
        \end{itemize}
    \end{block}
\end{frame}

\begin{frame}
    \frametitle{Key Concepts to Cover}
    \begin{itemize}
        \item \textbf{Understanding Scikit-learn's Architecture:}
        \begin{itemize}
            \item Provides efficient tools for data mining and analysis, built on NumPy, SciPy, and Matplotlib.
            \item Five main components essential for machine learning:
            \begin{itemize}
                \item \textbf{Preprocessing:} Data cleaning and transformation (e.g., \texttt{StandardScaler}, \texttt{OneHotEncoder}).
                \item \textbf{Model Selection:} Choosing the right model (e.g., linear regression, decision trees).
                \item \textbf{Model Evaluation:} Cross-validation and performance metrics.
                \item \textbf{Hyperparameter Tuning:} Optimizing model settings (e.g., \texttt{GridSearchCV}).
                \item \textbf{Pipeline:} Streamlining tasks to prevent data leakage.
            \end{itemize}
        \end{itemize}
    \end{itemize}
\end{frame}

\begin{frame}[fragile]
    \frametitle{Practical Exercise: Building a Simple Model}
    \begin{block}{Building a Classification Model}
        We will build a model to predict Titanic passenger survival based on available features. Here are the steps:
    \end{block}
    \begin{enumerate}
        \item \textbf{Import Necessary Libraries:}
        \begin{lstlisting}[language=Python]
import pandas as pd
from sklearn.model_selection import train_test_split
from sklearn.ensemble import RandomForestClassifier
from sklearn.metrics import accuracy_score, confusion_matrix
from sklearn.preprocessing import StandardScaler
        \end{lstlisting}
        
        \item \textbf{Load Dataset:}
        \begin{lstlisting}[language=Python]
data = pd.read_csv('titanic.csv')
        \end{lstlisting}
        
        \item \textbf{Preprocess the Data:}
        \begin{lstlisting}[language=Python]
data.fillna(method='ffill', inplace=True)
data = pd.get_dummies(data, columns=['Sex', 'Embarked'])
        \end{lstlisting}
    \end{enumerate}
\end{frame}

\begin{frame}[fragile]
    \frametitle{Continuing Practical Exercise}
    \begin{enumerate}
        \setcounter{enumi}{3}
        \item \textbf{Define Features and Labels:}
        \begin{lstlisting}[language=Python]
X = data[['Pclass', 'Age', 'SibSp', 'Parch', 'Fare', 'Sex_male', 'Embarked_Q', 'Embarked_S']]
y = data['Survived']
        \end{lstlisting}

        \item \textbf{Train-Test Split:}
        \begin{lstlisting}[language=Python]
X_train, X_test, y_train, y_test = train_test_split(X, y, test_size=0.2, random_state=42)
        \end{lstlisting}
        
        \item \textbf{Build and Train the Model:}
        \begin{lstlisting}[language=Python]
model = RandomForestClassifier()
model.fit(X_train, y_train)
        \end{lstlisting}
        
        \item \textbf{Make Predictions:}
        \begin{lstlisting}[language=Python]
predictions = model.predict(X_test)
        \end{lstlisting}
        
        \item \textbf{Evaluate the Model:}
        \begin{lstlisting}[language=Python]
accuracy = accuracy_score(y_test, predictions)
cm = confusion_matrix(y_test, predictions)
print(f'Accuracy: {accuracy}')
print('Confusion Matrix:\n', cm)
        \end{lstlisting}
    \end{enumerate}
\end{frame}

\begin{frame}
    \frametitle{Key Points to Emphasize}
    \begin{itemize}
        \item \textbf{Model Evaluation:} 
        Understanding accuracy and interpreting the confusion matrix are crucial for model performance.
        
        \item \textbf{Training vs. Testing:}
        The train-test split helps in preventing overfitting and ensures the model generalizes well to unseen data.
        
        \item \textbf{Feature Engineering:}
        Proper data preparation and transformation significantly impact accuracy.
    \end{itemize}
\end{frame}

\begin{frame}
    \frametitle{Concluding Activity}
    \begin{itemize}
        \item Reflect on the model performance and discuss potential improvements or alternative models.
        \item Consider ethical implications and biases in the dataset. How might these affect model outcomes?
    \end{itemize}
    \begin{block}{Goal}
        This session aims to solidify your understanding of Scikit-learn and empower you with skills to implement machine learning solutions. Let’s dive into the code and start building!
    \end{block}
\end{frame}

\begin{frame}[fragile]
    \frametitle{Overview}
    \begin{block}{Purpose}
        In this slide, we will conduct a comparative analysis between TensorFlow and Scikit-learn, two of the most widely used AI frameworks. Understanding their strengths and weaknesses will enable you to choose the right tool for your machine learning tasks.
    \end{block}
\end{frame}

\begin{frame}[fragile]
    \frametitle{Definitions}
    \begin{itemize}
        \item \textbf{TensorFlow}: An open-source library developed by Google for numerical computation and large-scale machine learning. Well-suited for deep learning applications.
        \item \textbf{Scikit-learn}: A Python library providing simple and efficient tools for data mining and data analysis. Built on NumPy, SciPy, and Matplotlib, it is ideal for classical machine learning algorithms.
    \end{itemize}
\end{frame}

\begin{frame}[fragile]
    \frametitle{Strengths and Weaknesses}
    \begin{table}[htbp]
        \centering
        \begin{tabular}{|l|l|l|}
            \hline
            \textbf{Feature} & \textbf{TensorFlow} & \textbf{Scikit-learn} \\
            \hline
            \textbf{Strengths} & 
            \begin{itemize}
                \item Supports deep learning and neural networks
                \item Scalable across multiple CPUs and GPUs
                \item Extensive community support and documentation
                \item Good for complex models (e.g., convolutional networks)
            \end{itemize} & 
            \begin{itemize}
                \item User-friendly API, easy for beginners
                \item Large collection of classic algorithms (e.g., regression, clustering)
                \item Excellent for small to medium-sized datasets
                \item Simple to deploy models with minimal configuration
            \end{itemize} \\
            \hline
            \textbf{Weaknesses} & 
            \begin{itemize}
                \item Steeper learning curve
                \item Overhead for simple tasks
                \item Requires more resources for setup and model training
            \end{itemize} & 
            \begin{itemize}
                \item Not designed for deep learning tasks
                \item Limited scalability for large datasets
                \item Can struggle with high-dimensional data
            \end{itemize} \\
            \hline
        \end{tabular}
    \end{table}
\end{frame}

\begin{frame}[fragile]
    \frametitle{Use Cases and Code Snippets}
    \begin{block}{Use Cases}
        \begin{itemize}
            \item \textbf{TensorFlow}: Best for tasks involving large datasets or complex neural architectures, such as image recognition and natural language processing.
            \item \textbf{Scikit-learn}: Ideal for traditional machine learning models in data preprocessing, exploratory data analysis, and smaller datasets, such as customer segmentation and predictive modeling.
        \end{itemize}
    \end{block}
\end{frame}

\begin{frame}[fragile]
    \frametitle{Code Examples}
    \begin{block}{Example of Linear Regression in Scikit-learn}
        \begin{lstlisting}[language=Python]
from sklearn.linear_model import LinearRegression
model = LinearRegression()
model.fit(X_train, y_train)
predictions = model.predict(X_test)
        \end{lstlisting}
    \end{block}
    \begin{block}{Example of Neural Network in TensorFlow}
        \begin{lstlisting}[language=Python]
import tensorflow as tf
model = tf.keras.Sequential([
    tf.keras.layers.Dense(10, activation='relu', input_shape=(input_shape,)),
    tf.keras.layers.Dense(1)
])
model.compile(optimizer='adam', loss='mean_squared_error')
model.fit(X_train, y_train, epochs=10)
        \end{lstlisting}
    \end{block}
\end{frame}

\begin{frame}[fragile]
    \frametitle{Key Points and Conclusion}
    \begin{itemize}
        \item \textbf{Selection Criteria}: Consider your project type, dataset size, and complexity when choosing between TensorFlow and Scikit-learn.
        \item \textbf{Learning Path}: Beginners may start with Scikit-learn due to ease of use, then transition to TensorFlow for more advanced projects involving deep learning.
    \end{itemize}
    \begin{block}{Conclusion}
        Understanding the nuances between TensorFlow and Scikit-learn is crucial for effective model development. Each tool has its unique features and best-use scenarios, guiding you in making informed choices for your machine learning tasks.
    \end{block}
\end{frame}

\begin{frame}
    \frametitle{Evaluation of AI Models}
    \begin{block}{Overview of Performance Metrics}
        Evaluating AI models is crucial to understanding their performance in real-world scenarios. Three of the most important metrics are:
        \begin{itemize}
            \item \textbf{Accuracy}
            \item \textbf{Precision}
            \item \textbf{Recall}
        \end{itemize}
        These metrics measure different aspects of a model's performance.
    \end{block}
\end{frame}

\begin{frame}
    \frametitle{Key Metrics Explained}
    \begin{enumerate}
        \item \textbf{Accuracy}
            \begin{itemize}
                \item \textbf{Definition}: Ratio of correctly predicted observations to total observations.
                \item \textbf{Formula}:  
                \[
                \text{Accuracy} = \frac{\text{TP + TN}}{\text{TP + TN + FP + FN}}
                \]
                \item \textbf{Example}: If a model predicts 80 out of 100 samples correctly, accuracy is 80\%.
            \end{itemize}
    
        \item \textbf{Precision}
            \begin{itemize}
                \item \textbf{Definition}: Proportion of true positive predictions among all positive predictions.
                \item \textbf{Formula}:  
                \[
                \text{Precision} = \frac{\text{TP}}{\text{TP + FP}}
                \]
                \item \textbf{Example}: If a model predicts 30 instances as positive, but only 20 are correct, precision is \( \frac{20}{30} = 0.67 \) or 67\%.
            \end{itemize}
    
        \item \textbf{Recall}
            \begin{itemize}
                \item \textbf{Definition}: Ratio of true positive predictions to actual positives (also called sensitivity).
                \item \textbf{Formula}:  
                \[
                \text{Recall} = \frac{\text{TP}}{\text{TP + FN}}
                \]
                \item \textbf{Example}: If there are 50 actual positives and the model identifies 30, recall is \( \frac{30}{50} = 0.6 \) or 60\%.
            \end{itemize}
    \end{enumerate}
\end{frame}

\begin{frame}[fragile]
    \frametitle{Key Points and Practical Application}
    \begin{block}{Key Points to Emphasize}
        \begin{itemize}
            \item \textbf{Trade-offs}: High precision can lower recall and vice versa. Balance is crucial based on application needs.
            \item \textbf{Use Cases}: Accuracy is useful when classes are balanced; precision and recall give a nuanced view in imbalanced scenarios.
            \item \textbf{F1 Score}: Consider the F1 Score for tasks with imbalanced classes as it balances precision and recall.
        \end{itemize}
    \end{block}

    \begin{block}{Practical Application}
        \begin{lstlisting}[language=Python, frame=single]
from sklearn.metrics import accuracy_score, precision_score, recall_score

y_true = [0, 1, 1, 0, 1]  # Actual labels
y_pred = [0, 1, 0, 0, 1]  # Predicted labels

accuracy = accuracy_score(y_true, y_pred)
precision = precision_score(y_true, y_pred)
recall = recall_score(y_true, y_pred)

print(f'Accuracy: {accuracy}')
print(f'Precision: {precision}')
print(f'Recall: {recall}')
        \end{lstlisting}
    \end{block}
\end{frame}

\begin{frame}
    \frametitle{Group Activity: Tool Utilization}
    \begin{block}{Objective}
        In this group activity, you will collaborate to apply practical knowledge of AI tools, specifically TensorFlow and Scikit-learn, to build a machine learning project. This hands-on experience aims to reinforce theoretical concepts discussed in earlier sessions, including model evaluation metrics (accuracy, precision, recall).
    \end{block}
\end{frame}

\begin{frame}[fragile]
    \frametitle{Overview of Tools}
    \begin{itemize}
        \item \textbf{TensorFlow}
        \begin{itemize}
            \item \textbf{Purpose}: An open-source library developed by Google for building and training deep learning models.
            \item \textbf{Key Features}:
            \begin{itemize}
                \item \textbf{Neural Networks}: Simplified mechanisms for modeling complex data relationships.
                \item \textbf{TensorFlow Serving}: Allows for easy deployment of trained models.
            \end{itemize}
            \item \textbf{Example Use Case}: Image classification using convolutional neural networks (CNNs).
        \end{itemize}
        
        \begin{lstlisting}[language=Python]
import tensorflow as tf
from tensorflow.keras import layers, models

model = models.Sequential([
    layers.Conv2D(32, (3, 3), activation='relu', input_shape=(28, 28, 1)),
    layers.MaxPooling2D((2, 2)),
    layers.Flatten(),
    layers.Dense(64, activation='relu'),
    layers.Dense(10, activation='softmax')
])
        \end{lstlisting}

        \item \textbf{Scikit-learn}
        \begin{itemize}
            \item \textbf{Purpose}: A robust Python library for traditional machine learning algorithms.
            \item \textbf{Key Features}:
            \begin{itemize}
                \item \textbf{Simple and Efficient Tools}: For data mining and data analysis.
                \item \textbf{Supports Model Evaluation}: Prebuilt functions for splitting data and evaluating performance.
            \end{itemize}
            \item \textbf{Example Use Case}: Predicting house prices using regression.
        \end{itemize}
        
        \begin{lstlisting}[language=Python]
from sklearn.linear_model import LinearRegression
from sklearn.model_selection import train_test_split
from sklearn.metrics import mean_squared_error

X_train, X_test, y_train, y_test = train_test_split(X, y, test_size=0.2)
model = LinearRegression().fit(X_train, y_train)
predictions = model.predict(X_test)
error = mean_squared_error(y_test, predictions)
        \end{lstlisting}
    \end{itemize}
\end{frame}

\begin{frame}
    \frametitle{Activity Instructions and Key Points}
    \begin{block}{Activity Instructions}
        \begin{enumerate}
            \item \textbf{Form Groups}: Divide into small groups of 4-5 participants.
            \item \textbf{Choose a Project Topic}: Options include, but are not limited to:
                \begin{itemize}
                    \item Image classification
                    \item Text sentiment analysis
                    \item Predictive modeling with tabular data
                \end{itemize}
            \item \textbf{Define Project Scope}: Identify datasets and allocate roles (data preprocessing, model building, evaluation).
            \item \textbf{Implement the Project}: Use TensorFlow for deep learning tasks, and Scikit-learn for statistical analysis.
            \item \textbf{Prepare a Brief Presentation}: Present findings, model usage, evaluation metrics, and challenges faced.
        \end{enumerate}
    \end{block}

    \begin{block}{Key Points to Emphasize}
        \begin{itemize}
            \item Collaboration is Essential: Diverse ideas and problem-solving approaches arise from group work.
            \item Practical Application: Bridging theory with practice enhances understanding and retention.
            \item Performance Metrics Matter: Evaluating model performance critically impacts project success.
        \end{itemize}
    \end{block}
\end{frame}

\begin{frame}[fragile]
    \frametitle{Ethical Considerations in AI - Introduction}
    As AI technologies become more integrated into various sectors, it’s crucial to address the ethical implications of their deployment. Ethical considerations encompass how AI impacts society, ensuring fairness, accountability, and transparency. This slide discusses the two critical aspects of ethics in AI:
    \begin{enumerate}
        \item \textbf{Bias}
        \item \textbf{Transparency}
    \end{enumerate}
\end{frame}

\begin{frame}[fragile]
    \frametitle{Ethical Considerations in AI - Part 1: Bias in AI}
    \begin{block}{Definition}
    Bias in AI occurs when algorithms produce unfair outcomes due to prejudiced data or systematic error.
    \end{block}
    
    This can take various forms, such as:
    \begin{itemize}
        \item \textbf{Data Bias}: Training data reflecting societal inequalities can lead to skewed results. For example, if a facial recognition system is trained predominantly on images of light-skinned individuals, it may perform poorly on individuals with darker skin tones.
        \item \textbf{Algorithmic Bias}: Even if data is balanced, algorithm design might unintentionally favor one group over another.
    \end{itemize}
    
    \begin{block}{Example}
    In a hiring tool, if historical data shows a preference for male candidates, the AI may learn to prefer male applicants, perpetuating gender inequality.
    \end{block}
    
    \begin{block}{Key Point}
    To mitigate bias, it’s essential to regularly audit AI systems and implement diverse and representative datasets during the training phase.
    \end{block}
\end{frame}

\begin{frame}[fragile]
    \frametitle{Ethical Considerations in AI - Part 2: Transparency in AI}
    \begin{block}{Definition}
    Transparency refers to the clarity regarding how AI systems make decisions. A lack of transparency can lead to a lack of trust and accountability in AI applications.
    \end{block}
    
    \begin{itemize}
        \item \textbf{Black Box Models}: Many AI systems, especially deep learning models, operate as "black boxes," where the decision-making process is not apparent.
        \item \textbf{Explainable AI (XAI)}: Researchers are developing XAI methods that illustrate how models arrive at decisions, enhancing understanding and trust.
    \end{itemize}
    
    \begin{block}{Example}
    If a loan approval algorithm denies an application, transparency allows the applicant to understand the specific factors influencing the decision, enabling potential corrections or appeals.
    \end{block}
    
    \begin{block}{Key Point}
    Striving for greater transparency not only builds trust with users but also enhances accountability and the ability to troubleshoot or improve AI systems.
    \end{block}
\end{frame}

\begin{frame}[fragile]
    \frametitle{Ethical Considerations in AI - Conclusion and Call to Action}
    Engaging with ethical considerations in AI is not optional; it is a fundamental component of responsible AI development and deployment. 

    By addressing bias and striving for transparency, we can foster technologies that better serve all individuals in society.

    \textbf{Call to Action:}
    Encourage students to explore the ethical implications of AI tools they encounter in projects and daily life. Consider questions like:
    \begin{itemize}
        \item How can we ensure our AI solutions are fair?
        \item What measures can we implement to enhance transparency?
    \end{itemize}
    
    \textbf{References:}
    \begin{itemize}
        \item Barocas, S., Hardt, M., \& Narayanan, A. (2019). Fairness and Machine Learning: Limitations and Opportunities.
        \item Lipton, Z. C. (2018). The Mythos of Model Interpretability.
    \end{itemize}
\end{frame}

\begin{frame}[fragile]
    \frametitle{Wrap-Up and Reflection - Overview of Learning Outcomes}
    \begin{itemize}
        \item This week, we explored various AI tools and their practical applications:
        \item \textbf{Understanding AI Tools}:
        \begin{itemize}
            \item Identified categories such as chatbots, language models, and image processing tools.
            \item Discussed the functions, operations, and uses of these tools.
        \end{itemize}
        \item \textbf{Ethical Considerations in AI}:
        \begin{itemize}
            \item Examined biases in algorithms and the importance of transparency.
            \item Analyzed real-world cases of ethical lapses, brainstorming mitigation strategies.
        \end{itemize}
        \item \textbf{Hands-On Experience}:
        \begin{itemize}
            \item Students gained firsthand experience with selected AI tools through practical exercises.
            \item Examples included using ChatGPT for text generation and image creation platforms.
        \end{itemize}
    \end{itemize}
\end{frame}

\begin{frame}[fragile]
    \frametitle{Wrap-Up and Reflection - Key Points for Reflection}
    \begin{itemize}
        \item \textbf{What have you learned?} 
        \begin{itemize}
            \item Consider both technical skills and ethical frameworks discussed. 
            \item Reflect on how these insights affect your view of AI tools and their societal impacts.
        \end{itemize}
        \item \textbf{Personal Experiences}:
        \begin{itemize}
            \item Reflect on your interactions with AI tools. 
            \item Which was the most engaging tool, and what challenged or surprised you?
        \end{itemize}
        \item \textbf{Future Applications}:
        \begin{itemize}
            \item Consider how these AI tools can be utilized in future projects or careers.
            \item Identify relevant ethical considerations for deploying these technologies.
        \end{itemize}
    \end{itemize}
\end{frame}

\begin{frame}[fragile]
    \frametitle{Wrap-Up and Reflection - Next Steps}
    \begin{itemize}
        \item \textbf{Continuous Learning}:
        \begin{itemize}
            \item Remember, mastering AI tools is an ongoing journey.
            \item Use this week’s insights to build a strong foundation for future topics.
        \end{itemize}
        \item \textbf{Prepare for Next Week}:
        \begin{itemize}
            \item Bring your reflections and questions about AI tools and their ethical use.
            \item We will explore advanced concepts and peer projects.
        \end{itemize}
    \end{itemize}
    \begin{block}{Final Thoughts}
        Engage with peers and share thoughts as we continue our exploration of AI together!
    \end{block}
\end{frame}

\begin{frame}[fragile]
    \frametitle{Next Steps - Overview}
    \begin{block}{Overview of Upcoming Activities and Topics}
        As we move into the next week, our focus will be on deepening your understanding of artificial intelligence (AI) tools and their practical applications.
        This slide outlines key activities and concepts to foster a continuous learning environment.
    \end{block}
\end{frame}

\begin{frame}[fragile]
    \frametitle{Next Steps - Key Activities}
    \begin{enumerate}
        \item \textbf{Hands-On Project: AI Tool Application}
            \begin{itemize}
                \item \textbf{Objective:} Apply your knowledge of AI tools in a practical project.
                \item \textbf{Details:} Select an AI tool (e.g., ChatGPT, stable diffusion, or GPT-4) and demonstrate its capabilities through a project documenting your process and findings.
            \end{itemize}
        
        \item \textbf{Interactive Discussion: Ethical Use of AI}
            \begin{itemize}
                \item \textbf{Objective:} Understand the ethical implications of AI tools.
                \item \textbf{Format:} Small group discussions on case studies showcasing positive and negative consequences of AI applications.
            \end{itemize}

        \item \textbf{Guest Lecture: Trends in AI Development}
            \begin{itemize}
                \item \textbf{Objective:} Gain insights from industry experts.
                \item \textbf{Details:} A guest speaker from the AI research community will discuss the latest trends, including advancements in models like ChatGPT-4 and Phi.
            \end{itemize}
    \end{enumerate}
\end{frame}

\begin{frame}[fragile]
    \frametitle{Next Steps - Topics to Explore}
    \begin{enumerate}
        \item \textbf{Latest AI Models:}
            \begin{itemize}
                \item Investigate functionalities and innovations of recent models (ChatGPT-4, etc.) and their applications across various fields (healthcare, education, business).
            \end{itemize}
        
        \item \textbf{Project-Based Learning:}
            \begin{itemize}
                \item Discuss how to integrate learning objectives and ethical implications into your practical applications for the upcoming project.
            \end{itemize}

        \item \textbf{Self-Directed Learning:}
            \begin{itemize}
                \item Explore resources for independent growth, such as online courses, AI podcasts, and research papers. Set personal learning goals regarding AI tools.
            \end{itemize}
    \end{enumerate}
\end{frame}


\end{document}