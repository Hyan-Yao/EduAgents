\documentclass[aspectratio=169]{beamer}

% Theme and Color Setup
\usetheme{Madrid}
\usecolortheme{whale}
\useinnertheme{rectangles}
\useoutertheme{miniframes}

% Additional Packages
\usepackage[utf8]{inputenc}
\usepackage[T1]{fontenc}
\usepackage{graphicx}
\usepackage{booktabs}
\usepackage{listings}
\usepackage{amsmath}
\usepackage{amssymb}
\usepackage{xcolor}
\usepackage{tikz}
\usepackage{pgfplots}
\pgfplotsset{compat=1.18}
\usetikzlibrary{positioning}
\usepackage{hyperref}

% Custom Colors
\definecolor{myblue}{RGB}{31, 73, 125}
\definecolor{mygray}{RGB}{100, 100, 100}
\definecolor{mygreen}{RGB}{0, 128, 0}
\definecolor{myorange}{RGB}{230, 126, 34}
\definecolor{mycodebackground}{RGB}{245, 245, 245}

% Set Theme Colors
\setbeamercolor{structure}{fg=myblue}
\setbeamercolor{frametitle}{fg=white, bg=myblue}
\setbeamercolor{title}{fg=myblue}
\setbeamercolor{section in toc}{fg=myblue}
\setbeamercolor{item projected}{fg=white, bg=myblue}
\setbeamercolor{block title}{bg=myblue!20, fg=myblue}
\setbeamercolor{block body}{bg=myblue!10}
\setbeamercolor{alerted text}{fg=myorange}

% Set Fonts
\setbeamerfont{title}{size=\Large, series=\bfseries}
\setbeamerfont{frametitle}{size=\large, series=\bfseries}
\setbeamerfont{caption}{size=\small}
\setbeamerfont{footnote}{size=\tiny}

% Code Listing Style
\lstdefinestyle{customcode}{
  backgroundcolor=\color{mycodebackground},
  basicstyle=\footnotesize\ttfamily,
  breakatwhitespace=false,
  breaklines=true,
  commentstyle=\color{mygreen}\itshape,
  keywordstyle=\color{blue}\bfseries,
  stringstyle=\color{myorange},
  numbers=left,
  numbersep=8pt,
  numberstyle=\tiny\color{mygray},
  frame=single,
  framesep=5pt,
  rulecolor=\color{mygray},
  showspaces=false,
  showstringspaces=false,
  showtabs=false,
  tabsize=2,
  captionpos=b
}
\lstset{style=customcode}

% Custom Commands
\newcommand{\hilight}[1]{\colorbox{myorange!30}{#1}}
\newcommand{\source}[1]{\vspace{0.2cm}\hfill{\tiny\textcolor{mygray}{Source: #1}}}
\newcommand{\concept}[1]{\textcolor{myblue}{\textbf{#1}}}
\newcommand{\separator}{\begin{center}\rule{0.5\linewidth}{0.5pt}\end{center}}

% Footer and Navigation Setup
\setbeamertemplate{footline}{
  \leavevmode%
  \hbox{%
  \begin{beamercolorbox}[wd=.3\paperwidth,ht=2.25ex,dp=1ex,center]{author in head/foot}%
    \usebeamerfont{author in head/foot}\insertshortauthor
  \end{beamercolorbox}%
  \begin{beamercolorbox}[wd=.5\paperwidth,ht=2.25ex,dp=1ex,center]{title in head/foot}%
    \usebeamerfont{title in head/foot}\insertshorttitle
  \end{beamercolorbox}%
  \begin{beamercolorbox}[wd=.2\paperwidth,ht=2.25ex,dp=1ex,center]{date in head/foot}%
    \usebeamerfont{date in head/foot}
    \insertframenumber{} / \inserttotalframenumber
  \end{beamercolorbox}}%
  \vskip0pt%
}

% Turn off navigation symbols
\setbeamertemplate{navigation symbols}{}

% Title Page Information
\title[Unsupervised Learning Techniques]{Weeks 10-12: Unsupervised Learning Techniques}
\author[J. Smith]{John Smith, Ph.D.}
\institute[University Name]{
  Department of Computer Science\\
  University Name\\
  \vspace{0.3cm}
  Email: email@university.edu\\
  Website: www.university.edu
}
\date{\today}

% Document Start
\begin{document}

\frame{\titlepage}

\begin{frame}[fragile]
    \frametitle{Introduction to Unsupervised Learning - Overview}
    \begin{block}{What is Unsupervised Learning?}
        Unsupervised learning is a type of machine learning that deals with datasets without labeled responses. The model learns patterns and structures from input data without explicit guidance on outputs. The primary goal is to find hidden structures or intrinsic relationships in the data.
    \end{block}
    
    \begin{block}{Key Characteristics}
        \begin{itemize}
            \item \textbf{No Labeled Data}: The algorithm identifies patterns based solely on the input data.
            \item \textbf{Pattern Recognition}: Focus on learning the underlying structure, clusters, and distributions in the data.
            \item \textbf{Data Exploration}: Often used for exploratory data analysis to gain insights about the data.
        \end{itemize}
    \end{block}
\end{frame}

\begin{frame}[fragile]
    \frametitle{Unsupervised Learning - Significance and Techniques}
    \begin{block}{Why is Unsupervised Learning Significant?}
        \begin{itemize}
            \item \textbf{Real-World Applications}: Used in customer segmentation and image analysis in various fields.
            \item \textbf{Feature Learning}: Helps discover significant features that can serve as inputs for supervised learning tasks.
            \item \textbf{Dimensionality Reduction}: Techniques like PCA simplify models while retaining essential information.
        \end{itemize}
    \end{block}
    
    \begin{block}{Examples of Techniques}
        \begin{itemize}
            \item \textbf{Clustering Algorithms}: Group similar data points (e.g., K-means, Hierarchical Clustering, DBSCAN).
            \item \textbf{Dimensionality Reduction}: Reduce features while preserving information (e.g., PCA, t-SNE).
            \item \textbf{Anomaly Detection}: Identify unusual data points (fraud detection, network security).
        \end{itemize}
    \end{block}
\end{frame}

\begin{frame}[fragile]
    \frametitle{Example Code: K-means Clustering in Python}
    Here’s a simple implementation of K-means clustering using Python and the Scikit-learn library:
    
    \begin{lstlisting}[language=Python]
from sklearn.cluster import KMeans
import numpy as np

# Sample data: 2D points
data = np.array([[1, 2], [1, 4], [1, 0],
                 [4, 2], [4, 4], [4, 0]])

# Applying K-means with 2 clusters
kmeans = KMeans(n_clusters=2, random_state=0).fit(data)

# Output the labels and cluster centers
print("Cluster Labels:", kmeans.labels_)
print("Cluster Centers:", kmeans.cluster_centers_)
    \end{lstlisting}
    
    \begin{block}{Key Points to Emphasize}
        \begin{itemize}
            \item Unsupervised learning is crucial for analyzing datasets without labels.
            \item Offers powerful tools for pattern recognition and data exploration.
            \item Understanding these techniques lays groundwork for advanced machine learning models.
        \end{itemize}
    \end{block}
\end{frame}

\begin{frame}[fragile]{What is Unsupervised Learning? - Definition}
    \begin{block}{Definition}
        Unsupervised learning is a type of machine learning that deals with input data without labeled responses. 
        The main objective is to find hidden patterns or intrinsic structures in the input data.
    \end{block}
\end{frame}

\begin{frame}[fragile]{What is Unsupervised Learning? - Key Characteristics}
    \begin{enumerate}
        \item \textbf{No Labeled Data}:
            \begin{itemize}
                \item Unlike supervised learning, unsupervised learning does not use labels. 
                \item Models learn directly from the data itself.
                \item \textit{Example}: A dataset of images without any categories or tags.
            \end{itemize}

        \item \textbf{Finding Patterns}:
            \begin{itemize}
                \item The primary goal is to discover patterns, groups, or structures in data.
                \item Common tasks include clustering, dimensionality reduction, and anomaly detection.
                \item \textit{Illustration}: Imagine organizing books on a shelf based on dimensions and colors rather than genres.
            \end{itemize}
    \end{enumerate}
\end{frame}

\begin{frame}[fragile]{What is Unsupervised Learning? - Algorithms and Applications}
    \begin{enumerate}
        \setcounter{enumi}{2}
        \item \textbf{Types of Algorithms}:
            \begin{itemize}
                \item \textbf{Clustering}: Groups data points based on similarity (e.g., K-Means, Hierarchical Clustering).
                \item \textbf{Dimensionality Reduction}: Reduces the number of features while retaining meaningful information (e.g., PCA, t-SNE).
                \item \textbf{Anomaly Detection}: Identifies outliers or rare occurrences within the data.
            \end{itemize}

        \item \textbf{Exploratory Data Analysis (EDA)}:
            \begin{itemize}
                \item Often used in EDA to summarize main characteristics of data.
                \item Useful for identifying trends and correlations which guide further analysis or model selection.
            \end{itemize}

        \item \textbf{Applications}:
            \begin{itemize}
                \item Used in various fields such as:
                \begin{itemize}
                    \item Customer segmentation in marketing
                    \item Image compression
                    \item Recommendation systems
                    \item Organizing large datasets
                \end{itemize}
            \end{itemize}
    \end{enumerate}
\end{frame}

\begin{frame}[fragile]{What is Unsupervised Learning? - Examples and Key Points}
    \begin{enumerate}
        \setcounter{enumi}{5}
        \item \textbf{Examples}:
            \begin{itemize}
                \item \textbf{Market Basket Analysis}: A retail store analyzes customer purchases to identify product associations (e.g., people who buy bread also tend to buy butter).
                \item \textbf{Image Segmentation}: Grouping similar pixels in an image to recognize objects (e.g., dividing an image into sections of sky, water, and land).
            \end{itemize}

        \item \textbf{Key Points to Emphasize}:
            \begin{itemize}
                \item Unsupervised learning is crucial for exploratory analysis when labels aren't available.
                \item Identifying structures in data can lead to insights that drive forward business decisions and improve systems.
            \end{itemize}
    \end{enumerate}
\end{frame}

\begin{frame}[fragile]
    \frametitle{Applications of Unsupervised Learning}
    \begin{block}{Overview}
        Unsupervised learning is a powerful subset of machine learning that identifies patterns in data without labeled outcomes. 
        It is widely applicable across various domains, enabling innovative solutions to complex problems. 
        Here, we explore key domains where unsupervised learning techniques are employed.
    \end{block}
\end{frame}

\begin{frame}[fragile]
    \frametitle{Key Applications - Healthcare and Finance}
    \begin{enumerate}
        \item \textbf{Healthcare}
        \begin{itemize}
            \item \textbf{Genomic Data Analysis}: Clustering techniques analyze gene expression data to identify groups with similar functions.
            \item \textbf{Patient Segmentation}: Hospitals utilize unsupervised learning to segment patients for personalized treatment plans.
        \end{itemize}
        \pause
        \item \textbf{Finance}
        \begin{itemize}
            \item \textbf{Fraud Detection}: Algorithms can highlight unusual patterns in transaction data that may indicate fraudulent behavior.
            \item \textbf{Customer Segmentation}: Groups customers based on purchasing behavior to enable targeted marketing.
        \end{itemize}
    \end{enumerate}
\end{frame}

\begin{frame}[fragile]
    \frametitle{Key Applications - Social Media and Key Points}
    \begin{enumerate}
        \setcounter{enumi}{2}
        \item \textbf{Social Media}
        \begin{itemize}
            \item \textbf{Content Recommendation}: Platforms analyze user behavior for refined content suggestions.
            \item \textbf{Sentiment Analysis}: Clustering techniques are used to understand user emotions from unstructured data.
        \end{itemize}
        \item \textbf{Key Points to Emphasize}
        \begin{itemize}
            \item \textbf{Pattern Recognition}: Identifies hidden structures and relationships in data.
            \item \textbf{Real-World Impact}: Techniques lead to advancements in various industries.
            \item \textbf{Dynamic Requirements}: The choice of techniques depends on specific goals and data nature.
        \end{itemize}
    \end{enumerate}
\end{frame}

\begin{frame}[fragile]
    \frametitle{Conclusion and Relevant Techniques}
    \begin{block}{Conclusion}
        Unsupervised learning uncovers underlying patterns in data, demonstrating transformative potential across fields like healthcare, finance, and social media.
    \end{block}

    \begin{block}{Relevant Techniques}
        \begin{itemize}
            \item \textbf{Clustering Algorithms}: K-means, Hierarchical Clustering
            \item \textbf{Dimensionality Reduction}: Principal Component Analysis (PCA), t-SNE
        \end{itemize}
    \end{block}
\end{frame}

\begin{frame}[fragile]
    \frametitle{Clustering Overview - Introduction}
    \begin{block}{Introduction to Clustering}
        Clustering is a fundamental technique in unsupervised learning, aimed at grouping a set of objects in such a way that objects in the same group (or cluster) are more similar to each other than to those in other groups.
    \end{block}
    \begin{itemize}
        \item Derives insights from unlabeled data
        \item Identifies intrinsic structures within data
    \end{itemize}
\end{frame}

\begin{frame}[fragile]
    \frametitle{Clustering Overview - Objectives and Key Concepts}
    \begin{block}{Objectives of Clustering}
        \begin{enumerate}
            \item Discover Patterns: Unveil hidden structures in data.
            \item Segmentation: Segregate data into meaningful groups.
            \item Data Reduction: Simplify data by summarizing with groups.
            \item Anomaly Detection: Identify outliers that may indicate issues.
        \end{enumerate}
    \end{block}
    \begin{block}{Key Concepts}
        \begin{itemize}
            \item Distance Metric: Measure of similarity (e.g., Euclidean, Manhattan).
            \item Centroid: Average point of all points in a cluster.
            \item Cluster Assignment: Assigning data points to the nearest cluster.
        \end{itemize}
    \end{block}
\end{frame}

\begin{frame}[fragile]
    \frametitle{Clustering Overview - Algorithms and Example}
    \begin{block}{Common Clustering Algorithms}
        \begin{itemize}
            \item K-means
            \item Hierarchical Clustering
            \item DBSCAN (Density-Based Spatial Clustering of Applications with Noise)
        \end{itemize}
    \end{block}
    \begin{block}{Example}
        Grouping customers based on purchasing patterns:
        \begin{itemize}
            \item High-Value Customers: Frequent, significant purchases.
            \item Bargain Hunters: Customers mainly purchasing discounted items.
            \item Occasional Shoppers: Infrequent buyers needing targeted promotions.
        \end{itemize}
    \end{block}
\end{frame}

\begin{frame}[fragile]
    \frametitle{K-means Clustering - Introduction}
    \begin{block}{What is K-means Clustering?}
        K-means clustering is an unsupervised learning algorithm that partitions a dataset into \textbf{K distinct clusters} based on feature similarity, aiming to minimize the distance between data points and their cluster centroids.
    \end{block}
\end{frame}

\begin{frame}[fragile]
    \frametitle{K-means Clustering - Working Principle}
    \begin{enumerate}
        \item \textbf{Initialization:}
        \begin{itemize}
            \item Choose K, the number of clusters.
            \item Randomly select K initial centroids from the data points or use strategies like K-means++.
        \end{itemize}
        
        \item \textbf{Assignment Step:}
        \begin{itemize}
            \item Assign each data point to the nearest centroid using the \textbf{Euclidean distance}:
            \begin{equation}
            Distance = \sqrt{\sum (x_i - c_j)^2}
            \end{equation}
            \item Where \(x_i\) is a data point and \(c_j\) is a centroid.
        \end{itemize}
        
        \item \textbf{Update Step:}
        \begin{itemize}
            \item Calculate new centroids \(c_j\) by taking the mean of all data points assigned to each cluster:
            \begin{equation}
            c_j = \frac{1}{N_j} \sum_{i=1}^{N_j} x_i
            \end{equation}
            \item \(N_j\) is the number of data points in cluster \(j\).
        \end{itemize}
        
        \item \textbf{Convergence:}
        \begin{itemize}
            \item Repeat the assignment and update steps until the centroids stabilize or a maximum number of iterations is reached.
        \end{itemize}
    \end{enumerate}
\end{frame}

\begin{frame}[fragile]
    \frametitle{K-means Clustering - Advantages and Limitations}
    \begin{block}{Advantages}
        \begin{itemize}
            \item \textbf{Simplicity}: Easy to implement and understand.
            \item \textbf{Speed}: Efficient for large datasets (linear time complexity).
            \item \textbf{Scalability}: Quickly handles many variables/features.
        \end{itemize}
    \end{block}

    \begin{block}{Limitations}
        \begin{itemize}
            \item \textbf{Choosing K}: Selecting the optimal number of clusters can be challenging; methods like the Elbow Method may help.
            \item \textbf{Sensitivity to Initialization}: Poor initial centroids can lead to local minima.
            \item \textbf{Assumes Spherical Clusters}: Ineffective for irregular shapes or varying densities.
        \end{itemize}
    \end{block}
\end{frame}

\begin{frame}[fragile]
    \frametitle{Hierarchical Clustering - Overview}
    \begin{block}{Overview}
        Hierarchical clustering is an unsupervised learning method that builds a hierarchy of clusters. It can be categorized into two primary approaches: 
    \end{block}
    \begin{itemize}
        \item Agglomerative Clustering
        \item Divisive Clustering
    \end{itemize}
\end{frame}

\begin{frame}[fragile]
    \frametitle{Hierarchical Clustering - Agglomerative Clustering}
    \begin{block}{Agglomerative Clustering}
        \begin{itemize}
            \item \textbf{Description}: A "bottom-up" approach where each data point starts as its own cluster.
            \item \textbf{Process}:
            \begin{enumerate}
                \item Start with each data point as a separate cluster.
                \item Compute the distance between every pair of clusters.
                \item Merge the two closest clusters.
                \item Repeat until one cluster remains or desired clusters are formed.
            \end{enumerate}
            \item \textbf{Distance Metrics}:
            \begin{itemize}
                \item Euclidean distance
                \item Manhattan distance
                \item Cosine similarity
            \end{itemize}
        \end{itemize}
    \end{block}
    \begin{example}
        \textbf{Example:} Imagine clusters of animals (cats, dogs, rabbits) merging based on similarities.
    \end{example}
\end{frame}

\begin{frame}[fragile]
    \frametitle{Hierarchical Clustering - Divisive Clustering}
    \begin{block}{Divisive Clustering}
        \begin{itemize}
            \item \textbf{Description}: A "top-down" approach where all data points start in a single cluster and are split recursively.
            \item \textbf{Process}:
            \begin{enumerate}
                \item Start with all data points in one cluster.
                \item Identify the most dissimilar cluster.
                \item Split this cluster into smaller clusters.
                \item Repeat until each point is isolated or desired clusters are achieved.
            \end{enumerate}
        \end{itemize}
    \end{block}
    \begin{example}
        \textbf{Example:} Start with all animals and first split by mammals/non-mammals, then cats/dogs.
    \end{example}
\end{frame}

\begin{frame}[fragile]
    \frametitle{Comparison with K-means Clustering}
    \begin{block}{Comparison Table}
        \begin{tabular}{|l|l|l|}
            \hline
            \textbf{Feature} & \textbf{Hierarchical Clustering} & \textbf{K-means Clustering} \\
            \hline
            Approach & Hierarchical (Agglomerative / Divisive) & Partition-based \\
            Number of Clusters & Determined post-algorithm & Specified beforehand \\
            Cluster Shape & Captures complex shapes & Assumes spherical clusters \\
            Scalability & Less scalable; computationally expensive & More scalable for large datasets \\
            Initialization & No need for random initialization & Sensitive to initialization \\
            Interpretability & Produces a dendrogram & Provides cluster centroids \\
            \hline
        \end{tabular}
    \end{block}
\end{frame}

\begin{frame}[fragile]
    \frametitle{Key Points and Additional Material}
    \begin{block}{Key Points}
        \begin{itemize}
            \item Flexibility in exploring different levels of granularity in data.
            \item Dendrograms visualize cluster arrangements.
            \item Choice of clustering depends on dataset characteristics.
        \end{itemize}
    \end{block}
    \begin{block}{Distance Measurement \& Linkage Criteria}
        \begin{itemize}
            \item Distance can be calculated using metrics like Euclidean or Manhattan.
            \item Linkage criteria include:
            \begin{itemize}
                \item Single Linkage: Minimum distance between points.
                \item Complete Linkage: Maximum distance between points.
                \item Average Linkage: Average distance between all points.
            \end{itemize}
        \end{itemize}
    \end{block}
\end{frame}

\begin{frame}[fragile]
    \frametitle{Comparison of Clustering Methods}
    
    \begin{block}{Overview of K-means and Hierarchical Clustering}
        In the realm of unsupervised learning, clustering serves as a powerful method for grouping data points into distinct categories based on their features. We will compare two popular clustering techniques: \textbf{K-means} and \textbf{Hierarchical Clustering}.
    \end{block}
\end{frame}

\begin{frame}[fragile]
    \frametitle{Comparison of Clustering Methods - Criteria Table}

    \begin{table}
        \centering
        \begin{tabular}{|l|l|l|}
            \hline
            \textbf{Criteria} & \textbf{K-means Clustering} & \textbf{Hierarchical Clustering} \\
            \hline
            \textbf{Definition} & Partitions the dataset into K distinct clusters by minimizing variance. & Builds a tree (dendrogram) of clusters either by agglomerative or divisive methods. \\
            \hline
            \textbf{Advantages} & 
            \begin{itemize}
                \item Fast and efficient for large datasets (O(n*k) complexity)
                \item Easy to implement and understand
                \item Works well with spherical clusters
            \end{itemize} & 
            \begin{itemize}
                \item Provides a visual representation of data through a dendrogram.
                \item Does not require the number of clusters to be specified in advance.
                \item Can capture complex relationships between clusters.
            \end{itemize} \\
            \hline
            \textbf{Disadvantages} & 
            \begin{itemize}
                \item Requires predefining the number of clusters (K)
                \item Sensitive to initial centroid placement and outliers
                \item Assumes clusters are convex and isotropic (spherical)
            \end{itemize} & 
            \begin{itemize}
                \item Computationally expensive for large datasets (O(n^3) complexity).
                \item The choice of linkage method can affect results (e.g., single, complete).
                \item May lead to overfitting with small datasets due to various configurations.
            \end{itemize} \\
            \hline
        \end{tabular}
    \end{table}
\end{frame}

\begin{frame}[fragile]
    \frametitle{Key Concepts in Clustering}

    \begin{block}{K-means Clustering}
        \begin{itemize}
            \item \textbf{How it Works}: The algorithm initializes K centroids, assigns points to the nearest centroid, and iteratively updates the centroids until convergence is achieved.
            \item \textbf{Formula}:
            \begin{equation}
                C_k = \frac{1}{N_k} \sum_{i=1}^{N_k} x_i
            \end{equation}
            where \( N_k \) is the number of points assigned to cluster \( k \), and \( x_i \) are the points in that cluster.
        \end{itemize}
    \end{block}
    
    \begin{block}{Hierarchical Clustering}
        \begin{itemize}
            \item \textbf{How it Works}: Creates a hierarchy starting from individual points and merges them into clusters based on a distance metric (e.g., Euclidean distance).
            \item \textbf{Dendrogram}: A visual representation that shows the arrangement of the clusters formed with respect to the distance at which clusters are merged.
        \end{itemize}
    \end{block}
\end{frame}

\begin{frame}[fragile]
    \frametitle{Practical Applications}

    \begin{block}{Applications of Clustering Methods}
        \begin{itemize}
            \item \textbf{K-means}: Commonly used in market segmentation, image compression, and document clustering.
            \item \textbf{Hierarchical Clustering}: Useful in situations where the data structure is unclear or when a visual representation of clustering is necessary, such as genetics or social science data.
        \end{itemize}
    \end{block}

    \begin{block}{Key Points to Remember}
        \begin{itemize}
            \item Selecting the right clustering technique depends on the specific data characteristics and the problem at hand.
            \item For large datasets with well-defined clusters, K-means is often preferred.
            \item For exploratory analysis and understanding the data structure, Hierarchical clustering may provide more insights despite its computational costs.
        \end{itemize}
    \end{block}
\end{frame}

\begin{frame}[fragile]
    \frametitle{Dimensionality Reduction Overview}
    \begin{block}{What is Dimensionality Reduction?}
        Dimensionality reduction is a technique that simplifies large datasets by reducing the number of input variables while preserving meaningful information.
    \end{block}
    
    \begin{block}{Importance of Dimensionality Reduction}
        \begin{itemize}
            \item Simplification and Visualization
            \item Improving Algorithm Performance
            \item Noise Reduction
            \item Storage and Processing Efficiency
        \end{itemize}
    \end{block}
\end{frame}

\begin{frame}[fragile]
    \frametitle{Importance of Dimensionality Reduction - Details}
    \begin{enumerate}
        \item \textbf{Simplification and Visualization}
            \begin{itemize}
                \item Easier to visualize datasets, allowing better understanding.
            \end{itemize}
    
        \item \textbf{Improving Algorithm Performance}
            \begin{itemize}
                \item Reduces the "curse of dimensionality," leading to faster computations and improved accuracy.
            \end{itemize}

        \item \textbf{Noise Reduction}
            \begin{itemize}
                \item Filters out irrelevant features, making models less prone to overfitting.
            \end{itemize}

        \item \textbf{Storage and Processing Efficiency}
            \begin{itemize}
                \item Smaller datasets require less storage and quicker processing.
            \end{itemize}
    \end{enumerate}
\end{frame}

\begin{frame}[fragile]
    \frametitle{Common Techniques of Dimensionality Reduction}
    \begin{itemize}
        \item \textbf{Principal Component Analysis (PCA)}
        \begin{itemize}
            \item Transforms data into a new coordinate system maximizing variance.
            \item \textit{Use Case:} Visualizing high-dimensional biological data.
        \end{itemize}

        \item \textbf{t-Distributed Stochastic Neighbor Embedding (t-SNE)}
        \begin{itemize}
            \item Reduces dimensions while keeping similar instances together.
            \item \textit{Use Case:} Visualizing high-dimensional image datasets.
        \end{itemize}

        \item \textbf{Linear Discriminant Analysis (LDA)}
        \begin{itemize}
            \item Projects features to maximize class separability.
            \item \textit{Use Case:} Supervised learning and classification tasks.
        \end{itemize}
    \end{itemize}
\end{frame}

\begin{frame}[fragile]
    \frametitle{Dimensionality Reduction Example}
    \begin{block}{Example}
        Consider a dataset with 50 features measuring different aspects of flowers. Using PCA, you could reduce these features to 2 principal components that capture most of the variance. The transformed dataset can be easily graphed, allowing for visualization of flower clusters based on similarities.
    \end{block}
    
    \begin{block}{Conclusion}
        Dimensionality reduction enhances interpretability and performance in data science. It is essential for effective data analysis and visualization. 
    \end{block}
\end{frame}

\begin{frame}[fragile]
    \frametitle{Principal Component Analysis (PCA) - Overview}
    \begin{block}{What is PCA?}
        Principal Component Analysis (PCA) is a powerful statistical technique used for dimensionality reduction. It simplifies high-dimensional data into a lower-dimensional form while preserving as much variability as possible.
    \end{block}
    
    \begin{itemize}
        \item \textbf{Dimensionality Reduction}: Reduces the number of features in a dataset while retaining essential patterns.
        \item \textbf{Variance Maximization}: Finds new axes (principal components) along which the data varies the most.
    \end{itemize}
\end{frame}

\begin{frame}[fragile]
    \frametitle{The PCA Process}
    \begin{enumerate}
        \item \textbf{Standardization}:
            \begin{itemize}
                \item Scale the data to ensure each feature contributes equally.
                \item Formula:
                \begin{equation}
                    Z = \frac{X - \mu}{\sigma}
                \end{equation}
            \end{itemize}

        \item \textbf{Covariance Matrix Calculation}:
            \begin{itemize}
                \item Compute the covariance matrix to understand feature relationships.
                \item If \( X \) is a data matrix with dimensions \( n \times p \):
                \begin{equation}
                    C = \frac{1}{n - 1} (X^T X)
                \end{equation}
            \end{itemize}

        \item \textbf{Eigenvalue and Eigenvector Computation}:
            \begin{itemize}
                \item Calculate eigenvalues and eigenvectors of the covariance matrix.
            \end{itemize}
    \end{enumerate}
\end{frame}

\begin{frame}[fragile]
    \frametitle{The PCA Process (Continued)}
    \begin{enumerate}
        \setcounter{enumi}{3}
        \item \textbf{Selection of Principal Components}:
            \begin{itemize}
                \item Sort eigenvalues in descending order and select the top \( k \) eigenvectors.
            \end{itemize}

        \item \textbf{Transformation}:
            \begin{itemize}
                \item Project the original data onto the new feature space:
                \begin{equation}
                    Y = XW
                \end{equation}
                \item Where \( Y \) is the transformed data, \( W \) is the matrix of eigenvectors.
            \end{itemize}
    \end{enumerate}

    \begin{block}{Applications of PCA}
        \begin{itemize}
            \item Data Visualization
            \item Preprocessing for ML Models
            \item Noise Reduction
            \item Image Compression
        \end{itemize}
    \end{block}
\end{frame}

\begin{frame}[fragile]
    \frametitle{When to Use PCA}
    \begin{itemize}
        \item \textbf{High-Dimensional Data}: Ideal for datasets with many features.
        \item \textbf{Multicollinearity Issues}: Effective when features are highly correlated.
        \item \textbf{Data Exploration}: Useful during exploratory data analysis.
    \end{itemize}

    \begin{block}{Example}
        Consider a dataset with features (height, weight, age):
        \begin{itemize}
            \item Standardization reveals covariance relationships.
            \item Eigenvectors enhance variance representation.
            \item First principal component might represent physical size.
        \end{itemize}
    \end{block}
\end{frame}

\begin{frame}[fragile]
    \frametitle{t-Distributed Stochastic Neighbor Embedding (t-SNE)}
    \begin{block}{Introduction to t-SNE}
        t-Distributed Stochastic Neighbor Embedding (t-SNE) is a powerful technique for visualizing high-dimensional data in a lower-dimensional space (commonly 2D or 3D).
        It reveals structure in complex datasets, making it popular in fields such as machine learning, bioinformatics, and image processing.
    \end{block}
\end{frame}

\begin{frame}[fragile]
    \frametitle{How t-SNE Works - Part 1}
    \begin{enumerate}
        \item \textbf{Probabilistic Approach}:
        t-SNE transforms high-dimensional data into a probabilistic distribution, calculating the probability that one point would pick another as a neighbor.
        \begin{equation}
        p_{j|i} = \frac{\exp\left(- \frac{||x_i - x_j||^2}{2\sigma_i^2}\right)}{\sum_{k \neq i} \exp\left(- \frac{||x_i - x_k||^2}{2\sigma_i^2}\right)}
        \end{equation}
        \item \textbf{Low-Dimensional Map}:
        In the lower-dimensional space, it maintains these relationships using:
        \begin{equation}
        q_{j|i} = \frac{(1 + ||y_i - y_j||^2)^{-1}}{\sum_{k \neq i} (1 + ||y_i - y_k||^2)^{-1}}
        \end{equation}
    \end{enumerate}
\end{frame}

\begin{frame}[fragile]
    \frametitle{How t-SNE Works - Part 2}
    \begin{enumerate}
        \setcounter{enumi}{2}
        \item \textbf{Minimizing Divergence}:
        The algorithm minimizes the difference between the two probability distributions \( P \) and \( Q \) using Kullback-Leibler (KL) divergence:
        \begin{equation}
        \text{KL}(P || Q) = \sum_{i} \sum_{j} p_{j|i} \log\left(\frac{p_{j|i}}{q_{j|i}}\right)
        \end{equation}
        This optimization is usually implemented via gradient descent.
    \end{enumerate}
\end{frame}

\begin{frame}[fragile]
    \frametitle{Key Features and Applications of t-SNE}
    \begin{itemize}
        \item \textbf{Preserves Local Structure}: Excellent for identifying clusters of similar items.
        \item \textbf{Non-Linear Mapping}: Captures non-linear relationships, contrasting linear methods like PCA.
        \item \textbf{Responsive to Clusters}: Visualizes clusters as separate in lower dimensions.
    \end{itemize}

    \begin{block}{Example}
        Visualizing thousands of handwritten digits (0-9) in 2D space. Each point represents an image; similar digits cluster together.
    \end{block}

    \begin{block}{Applications}
        \begin{itemize}
            \item Image and Text Data: Useful for visualizing deep learning feature embeddings.
            \item Biological Data: Analyzing gene expressions and clustering cellular profiles.
        \end{itemize}
    \end{block}
\end{frame}

\begin{frame}[fragile]
    \frametitle{Conclusion}
    t-SNE is an essential tool for data scientists to explore high-dimensional data interactively. It reveals patterns and structures that may not be visible in higher dimensions and is effective for cluster visualization.
\end{frame}

\begin{frame}[fragile]
    \frametitle{Comparison of Dimensionality Reduction Techniques - Overview}
    Dimensionality reduction techniques are essential for simplifying datasets while preserving their meaningful structures. This slide compares two popular techniques: Principal Component Analysis (PCA) and t-Distributed Stochastic Neighbor Embedding (t-SNE). 
\end{frame}

\begin{frame}[fragile]
    \frametitle{Comparison: PCA vs t-SNE}
    \begin{block}{Criteria}
        \begin{tabular}{|l|l|l|}
            \hline
            **Criteria** & **Principal Component Analysis (PCA)** & **t-Distributed Stochastic Neighbor Embedding (t-SNE)** \\
            \hline
            **Methodology** & 
            \begin{itemize}
                \item Linear technique that transforms data into orthogonal components.
                \item Aimed at maximizing variance while minimizing dimensionality.
                \item Uses eigenvalues and eigenvectors of the covariance matrix.
            \end{itemize} 
            & 
            \begin{itemize}
                \item Non-linear technique primarily used for visualization.
                \item Converts high-dimensional space into lower-dimensional space.
                \item Employs probability distributions to represent neighborhoods of data points.
            \end{itemize} \\
            \hline
            **Use Cases** & 
            \begin{itemize}
                \item Suitable for noise reduction and feature extraction.
                \item Often used as a preprocessing step for supervised learning models.
                \item Commonly applied in exploratory data analysis and visualization.
            \end{itemize} 
            & 
            \begin{itemize}
                \item Effective for visualizing complex relationships, especially in clusters.
                \item Used in single-cell RNA sequencing data analysis.
                \item Ideal for revealing local structures rather than global representations.
            \end{itemize} \\
            \hline
            **Performance** & 
            \begin{itemize}
                \item Computationally efficient for high-dimensional datasets.
                \item Can struggle with non-linear correlations.
                \item Performance scales well but interpretability may decline with too many components.
            \end{itemize} 
            & 
            \begin{itemize}
                \item Computationally intensive and slow for large datasets.
                \item Maintains local structures effectively, often better cluster representation.
                \item Performance sensitive to hyperparameters, especially perplexity.
            \end{itemize} \\
            \hline
        \end{tabular}
    \end{block}
\end{frame}

\begin{frame}[fragile]
    \frametitle{Key Points and Examples}
    \begin{block}{Key Points}
        \begin{itemize}
            \item PCA is best for linear relationships and noise reduction.
            \item t-SNE excels in preserving local structures within data.
            \item Selection depends on specific analysis goals: variance maximization (PCA) vs. intricate pattern visualization (t-SNE).
        \end{itemize}
    \end{block}

    \begin{block}{Example Applications}
        \begin{itemize}
            \item \textbf{PCA Example:} 
            In image processing, PCA might reduce pixel numbers while retaining essence for facial recognition.
            \item \textbf{t-SNE Example:} 
            In genomics, t-SNE visualizes high-dimensional gene expression data, making it easier to observe distinct cellular clusters.
        \end{itemize}
    \end{block}
\end{frame}

\begin{frame}[fragile]
    \frametitle{Summary}
    Choosing the right dimensionality reduction technique is crucial when working with complex datasets. 
    Understanding the strengths and weaknesses of PCA and t-SNE enhances data analysis capabilities and leads to better interpretations of findings.
\end{frame}

\begin{frame}[fragile]
    \frametitle{Model Evaluation for Unsupervised Learning - Introduction}
    \begin{itemize}
        \item Unsupervised learning operates without labeled data.
        \item Evaluation is more challenging than in supervised learning.
        \item Key metrics are used to assess the quality of clusters.
        \item Focus on:
        \begin{itemize}
            \item Silhouette Score
            \item Cluster Validity Indices
        \end{itemize}
    \end{itemize}
\end{frame}

\begin{frame}[fragile]
    \frametitle{Model Evaluation for Unsupervised Learning - Silhouette Score}
    \begin{block}{Definition}
        Quantifies how similar an object is to its own cluster versus other clusters. Ranges from -1 to +1:
    \end{block}
    \begin{itemize}
        \item **+1**: Far from the nearest neighboring cluster.
        \item **0**: Near the decision boundary between clusters.
        \item **-1**: Likely assigned to the wrong cluster.
    \end{itemize}
    
    \begin{equation}
        \text{Silhouette Score} = \frac{b - a}{\max(a, b)}
    \end{equation}
    Where:
    \begin{itemize}
        \item \( a \): Average distance to points in the same cluster.
        \item \( b \): Average distance to points in the nearest cluster.
    \end{itemize}
\end{frame}

\begin{frame}[fragile]
    \frametitle{Model Evaluation for Unsupervised Learning - Cluster Validity Indices}
    \begin{block}{Definition}
        Assess the quality of clusters from algorithms, aiding in optimal cluster selection.
    \end{block}
    \begin{itemize}
        \item **Davies-Bouldin Index (DBI)**: Lower values indicate better clustering quality.
        \item **Calinski-Harabasz Index**: Higher values signify better-defined clusters.
    \end{itemize}

    \begin{block}{Example Application}
        Using K-means on customer data, the Calinski-Harabasz index can help determine the better representation between 3 or 5 clusters.
    \end{block}
    
    \begin{block}{Key Points to Emphasize}
        \begin{itemize}
            \item Evaluation is critical due to lack of target values.
            \item Technique selection may vary based on algorithm and data characteristics.
            \item Analyze results contextually; high scores don’t always imply practical usefulness.
        \end{itemize}
    \end{block}
\end{frame}

\begin{frame}[fragile]
    \frametitle{Model Evaluation for Unsupervised Learning - Conclusion}
    Evaluating unsupervised learning models is essential for deriving insights from unstructured data. Understanding techniques like:
    \begin{itemize}
        \item Silhouette Score
        \item Cluster Validity Indices
    \end{itemize}
    empowers data scientists to make informed decisions based on clustering results.
\end{frame}

\begin{frame}
    \frametitle{Ethical Considerations in Unsupervised Learning}
    \begin{block}{Understanding the Ethical Challenges}
        Unsupervised learning algorithms, which draw patterns from unlabeled data without human intervention, raise several ethical concerns, especially around bias and transparency.
    \end{block}
\end{frame}

\begin{frame}
    \frametitle{Ethical Issues: Bias and Transparency}
    \begin{enumerate}
        \item \textbf{Bias in Data:}
            \begin{itemize}
                \item \textbf{Definition:} Occurs when training data is not representative of the broader population, leading to skewed patterns.
                \item \textbf{Example:} In customer segmentation, a dataset from affluent neighborhoods may result in unfair targeting or exclusion of other demographics.
            \end{itemize}

        \item \textbf{Algorithmic Transparency:}
            \begin{itemize}
                \item \textbf{Definition:} Clarity on how algorithms operate and make decisions. 
                \item \textbf{Importance:} Lack of transparency can lead to distrust, especially in healthcare or criminal justice.
                \item \textbf{Example:} If an unsupervised learning model identifies potential fraud without clear insights, stakeholders may struggle to understand its rationale.
            \end{itemize}
    \end{enumerate}
\end{frame}

\begin{frame}[fragile]
    \frametitle{Strategies to Address Ethical Issues}
    \begin{block}{Key Points to Emphasize}
        \begin{itemize}
            \item \textbf{Bias Mitigation:} Ensure data diversity and representativeness through techniques like resampling.
            \item \textbf{Promoting Transparency:} Use techniques like SHAP (SHapley Additive exPlanations) or LIME (Local Interpretable Model-agnostic Explanations).
            \item \textbf{Regulatory Considerations:} Adhere to evolving ethical standards like GDPR to emphasize accountability.
        \end{itemize}
    \end{block}
\end{frame}

\begin{frame}[fragile]
    \frametitle{Responsible Data Practices}
    \begin{enumerate}
        \item \textbf{Data Collection:} Scrutinize datasets for inherent biases before training.
        \item \textbf{Model Monitoring:} Regularly evaluate model outcomes to identify potential biases.
        \item \textbf{Stakeholder Engagement:} Involve diverse groups in design and evaluation to broaden perspectives.
    \end{enumerate}
\end{frame}

\begin{frame}[fragile]
    \frametitle{Code Snippet for Bias Detection}
    \begin{lstlisting}[language=Python]
import pandas as pd

# Load dataset
data = pd.read_csv('customer_data.csv')

# Calculate demographic distributions
demographic_distribution = data['demographic'].value_counts(normalize=True)

# Display demographic representation
print(demographic_distribution)
    \end{lstlisting}
\end{frame}

\begin{frame}
    \frametitle{Conclusion}
    \begin{block}{Summary}
        Ethical considerations in unsupervised learning are paramount. By addressing bias and promoting transparency, we foster trust and fairness in algorithm deployment, ensuring equitable benefits for all stakeholders.
    \end{block}
\end{frame}

\begin{frame}[fragile]
    \frametitle{Practical Applications and Case Studies - Overview}
    \begin{block}{Understanding Unsupervised Learning}
        Unsupervised learning refers to a type of machine learning where models are trained on data without labeled responses. The goal is to identify patterns, groupings, or structures within the data, allowing for insights and conclusions that were not explicitly taught.
    \end{block}
\end{frame}

\begin{frame}[fragile]
    \frametitle{Key Applications of Unsupervised Learning}
    \begin{enumerate}
        \item \textbf{Customer Segmentation}
            \begin{itemize}
                \item \textbf{Overview:} Identification of distinct customer segments based on purchasing behaviors.
                \item \textbf{Example:} Retailers use K-means clustering to segment customers.
            \end{itemize}
            
        \item \textbf{Anomaly Detection}
            \begin{itemize}
                \item \textbf{Overview:} Identifying unusual data points, which may indicate fraud or equipment failures.
                \item \textbf{Example:} Financial institutions flag atypical transactions to prevent fraud.
            \end{itemize}
        
        \item \textbf{Recommendation Systems}
            \begin{itemize}
                \item \textbf{Overview:} Generating recommendations by finding patterns in user behavior.
                \item \textbf{Example:} E-commerce platforms employ collaborative filtering.
            \end{itemize}
    \end{enumerate}
\end{frame}

\begin{frame}[fragile]
    \frametitle{More Applications and Case Studies}
    \begin{enumerate}
        \setcounter{enumi}{3}
        \item \textbf{Image and Video Analysis}
            \begin{itemize}
                \item \textbf{Overview:} Techniques used in analyzing and categorizing media content.
                \item \textbf{Example:} Google uses clustering for image recognition tasks.
            \end{itemize}
        
        \item \textbf{Market Basket Analysis}
            \begin{itemize}
                \item \textbf{Overview:} Examines items frequently purchased together.
                \item \textbf{Example:} Grocery stores use the Apriori algorithm for product placements.
            \end{itemize}
        
        \item \textbf{Case Studies}
            \begin{enumerate}
                \item \textbf{Spotify's Music Recommendation:} Analyzes listening habits for personalized playlists.
                \item \textbf{Netflix's Viewing Patterns:} Uses clustering to improve content recommendations.
                \item \textbf{Google News Clustering:} Groups similar articles for streamlined user experience.
            \end{enumerate}
    \end{enumerate}
\end{frame}

\begin{frame}[fragile]
    \frametitle{Key Points and Conclusion}
    \begin{block}{Key Points to Emphasize}
        \begin{itemize}
            \item \textbf{Data-Driven Insights:} Uncovers hidden structures leading to actionable insights.
            \item \textbf{Flexibility Across Domains:} Applicable in finance, marketing, healthcare, and tech.
            \item \textbf{Foundation for Future Learning:} Serves as stepping stones for supervised models.
        \end{itemize}
    \end{block}

    \begin{block}{Conclusion}
        Unsupervised learning techniques enhance analytical capabilities in data-rich environments, enabling organizations to make informed decisions and drive growth through innovative solutions.
    \end{block}
\end{frame}

\begin{frame}[fragile]
    \frametitle{Future Trends in Unsupervised Learning}
    \begin{block}{Introduction}
        Unsupervised learning is a critical segment of machine learning that focuses on identifying patterns in data without labeled outputs. As technology evolves, several trends are shaping the future of unsupervised learning.
    \end{block}
\end{frame}

\begin{frame}[fragile]
    \frametitle{Key Trends and Advancements - Part 1}
    \begin{enumerate}
        \item \textbf{Integration with Deep Learning}
            \begin{itemize}
                \item \textbf{Concept}: Deep learning techniques are being integrated into unsupervised learning to enhance feature extraction and pattern recognition.
                \item \textbf{Example}: Generative Adversarial Networks (GANs).
            \end{itemize}
        
        \item \textbf{Graph-Based Learning}
            \begin{itemize}
                \item \textbf{Concept}: Leveraging graph structures to uncover hidden patterns.
                \item \textbf{Example}: Applications in social network analysis.
            \end{itemize}
    \end{enumerate}
\end{frame}

\begin{frame}[fragile]
    \frametitle{Key Trends and Advancements - Part 2}
    \begin{enumerate}
        \setcounter{enumi}{2}
        \item \textbf{Hybrid Models}
            \begin{itemize}
                \item \textbf{Concept}: Combining supervised and unsupervised learning for better outcomes.
                \item \textbf{Example}: Semi-supervised learning techniques.
            \end{itemize}

        \item \textbf{Self-Supervised Learning}
            \begin{itemize}
                \item \textbf{Concept}: The system generates its own supervisory signals.
                \item \textbf{Example}: Contrastive learning techniques.
            \end{itemize}

        \item \textbf{Explainability and Interpretability}
            \begin{itemize}
                \item \textbf{Concept}: The need for explainability in complex models.
                \item \textbf{Example}: Tools to interpret clustering results.
            \end{itemize}
    \end{enumerate}
\end{frame}

\begin{frame}[fragile]
    \frametitle{Implications for Practice}
    \begin{itemize}
        \item \textbf{Personalization and Recommendation Systems}: Enhancing user experiences through behavior pattern analysis.
        \item \textbf{Anomaly Detection}: Applications in finance and cybersecurity to identify unusual behavior.
        \item \textbf{Data Privacy and Ethics}: Balancing data utilization while protecting individual privacy rights.
    \end{itemize}
\end{frame}

\begin{frame}[fragile]
    \frametitle{Conclusion}
    The future of unsupervised learning is bright, with continuous advancements enhancing its applicability across various domains. Educators, practitioners, and researchers must keep abreast of these trends to effectively leverage unsupervised techniques in solving real-world problems.
\end{frame}

\begin{frame}[fragile]
    \frametitle{Conclusion and Key Takeaways - Conclusion}
    \begin{block}{Overview}
        In this chapter, we explored unsupervised learning techniques, their significance, and applications across various fields. This method extracts patterns from unlabeled data, facilitating insights without predetermined categories.
    \end{block}
    
    \begin{block}{Key Points}
        \begin{enumerate}
            \item Unsupervised learning identifies hidden structures in data without predefined labels.
            \item Key techniques such as clustering and dimensionality reduction were discussed.
            \item Evaluating unsupervised models presents unique challenges due to the absence of labeled data.
        \end{enumerate}
    \end{block}
\end{frame}

\begin{frame}[fragile]
    \frametitle{Conclusion and Key Takeaways - Key Concepts}
    \begin{block}{Techniques Covered}
        \begin{itemize}
            \item \textbf{Clustering}:
            \begin{itemize}
                \item K-means Clustering
                \item Hierarchical Clustering
                \item DBSCAN
            \end{itemize}
            \item \textbf{Dimensionality Reduction}:
            \begin{itemize}
                \item Principal Component Analysis (PCA)
                \item t-Distributed Stochastic Neighbor Embedding (t-SNE)
            \end{itemize}
        \end{itemize}
    \end{block}
    
    \begin{block}{Model Evaluation}
        Evaluating models is challenging due to the lack of labeled data. Important metrics include:
        \begin{itemize}
            \item Silhouette Score
            \item Inertia (specifically in K-means)
        \end{itemize}
    \end{block}
\end{frame}

\begin{frame}[fragile]
    \frametitle{Conclusion and Key Takeaways - Importance}
    \begin{block}{Significance of Mastering Techniques}
        \begin{itemize}
            \item Facilitates data exploration and feature selection.
            \item Real-world applications such as:
            \begin{itemize}
                \item Customer Segmentation
                \item Anomaly Detection
            \end{itemize}
            \item Foundation for advancing to semi-supervised and reinforcement learning techniques.
        \end{itemize}
    \end{block}

    \begin{block}{Closing Note}
        Mastering unsupervised learning equips professionals to address various data-driven challenges, making it essential in today's data-centric landscape. 
    \end{block}
\end{frame}


\end{document}