\documentclass[aspectratio=169]{beamer}

% Theme and Color Setup
\usetheme{Madrid}
\usecolortheme{whale}
\useinnertheme{rectangles}
\useoutertheme{miniframes}

% Additional Packages
\usepackage[utf8]{inputenc}
\usepackage[T1]{fontenc}
\usepackage{graphicx}
\usepackage{booktabs}
\usepackage{listings}
\usepackage{amsmath}
\usepackage{amssymb}
\usepackage{xcolor}
\usepackage{tikz}
\usepackage{pgfplots}
\pgfplotsset{compat=1.18}
\usetikzlibrary{positioning}
\usepackage{hyperref}

% Custom Colors
\definecolor{myblue}{RGB}{31, 73, 125}
\definecolor{mygray}{RGB}{100, 100, 100}
\definecolor{mygreen}{RGB}{0, 128, 0}
\definecolor{myorange}{RGB}{230, 126, 34}
\definecolor{mycodebackground}{RGB}{245, 245, 245}

% Set Theme Colors
\setbeamercolor{structure}{fg=myblue}
\setbeamercolor{frametitle}{fg=white, bg=myblue}
\setbeamercolor{title}{fg=myblue}
\setbeamercolor{section in toc}{fg=myblue}
\setbeamercolor{item projected}{fg=white, bg=myblue}
\setbeamercolor{block title}{bg=myblue!20, fg=myblue}
\setbeamercolor{block body}{bg=myblue!10}
\setbeamercolor{alerted text}{fg=myorange}

% Set Fonts
\setbeamerfont{title}{size=\Large, series=\bfseries}
\setbeamerfont{frametitle}{size=\large, series=\bfseries}
\setbeamerfont{caption}{size=\small}
\setbeamerfont{footnote}{size=\tiny}

% Code Listing Style
\lstdefinestyle{customcode}{
  backgroundcolor=\color{mycodebackground},
  basicstyle=\footnotesize\ttfamily,
  breakatwhitespace=false,
  breaklines=true,
  commentstyle=\color{mygreen}\itshape,
  keywordstyle=\color{blue}\bfseries,
  stringstyle=\color{myorange},
  numbers=left,
  numbersep=8pt,
  numberstyle=\tiny\color{mygray},
  frame=single,
  framesep=5pt,
  rulecolor=\color{mygray},
  showspaces=false,
  showstringspaces=false,
  showtabs=false,
  tabsize=2,
  captionpos=b
}
\lstset{style=customcode}

% Custom Commands
\newcommand{\hilight}[1]{\colorbox{myorange!30}{#1}}
\newcommand{\source}[1]{\vspace{0.2cm}\hfill{\tiny\textcolor{mygray}{Source: #1}}}
\newcommand{\concept}[1]{\textcolor{myblue}{\textbf{#1}}}
\newcommand{\separator}{\begin{center}\rule{0.5\linewidth}{0.5pt}\end{center}}

% Footer and Navigation Setup
\setbeamertemplate{footline}{
  \leavevmode%
  \hbox{%
  \begin{beamercolorbox}[wd=.3\paperwidth,ht=2.25ex,dp=1ex,center]{author in head/foot}%
    \usebeamerfont{author in head/foot}\insertshortauthor
  \end{beamercolorbox}%
  \begin{beamercolorbox}[wd=.5\paperwidth,ht=2.25ex,dp=1ex,center]{title in head/foot}%
    \usebeamerfont{title in head/foot}\insertshorttitle
  \end{beamercolorbox}%
  \begin{beamercolorbox}[wd=.2\paperwidth,ht=2.25ex,dp=1ex,center]{date in head/foot}%
    \usebeamerfont{date in head/foot}
    \insertframenumber{} / \inserttotalframenumber
  \end{beamercolorbox}}%
  \vskip0pt%
}

% Turn off navigation symbols
\setbeamertemplate{navigation symbols}{}

% Title Page Information
\title[Unsupervised Learning]{Weeks 10-12: Unsupervised Learning Techniques}
\author[J. Smith]{John Smith, Ph.D.}
\institute[University Name]{
  Department of Computer Science\\
  University Name\\
  \vspace{0.3cm}
  Email: email@university.edu\\
  Website: www.university.edu
}
\date{\today}

% Document Start
\begin{document}

\frame{\titlepage}

\begin{frame}[fragile]
    \frametitle{Introduction to Unsupervised Learning}
    \begin{block}{Overview}
        Unsupervised learning is a branch of machine learning focused on finding hidden patterns in data without labeled outputs.
    \end{block}
\end{frame}

\begin{frame}[fragile]
    \frametitle{Key Concepts}
    \begin{enumerate}
        \item \textbf{Definition}:
        \begin{itemize}
            \item Unsupervised Learning: Algorithms work with data that contains no explicit labels, aiming to model the underlying structure.
        \end{itemize}
        \item \textbf{Significance}:
        \begin{itemize}
            \item Data Exploration: Understand data distribution and structure.
            \item Feature Extraction: Automatically identify relevant features.
            \item Dimensionality Reduction: Improve algorithm performance by reducing variables.
            \item Anomaly Detection: Identify significant deviations from normative data.
        \end{itemize}
    \end{enumerate}
\end{frame}

\begin{frame}[fragile]
    \frametitle{Common Techniques}
    \begin{itemize}
        \item \textbf{Clustering}: Groups similar data points, e.g., customers based on shopping behavior.
        \item \textbf{Association Rule Learning}: Discovers relationships in data, e.g., market basket analysis.
        \item \textbf{Dimensionality Reduction Techniques}:
        \begin{itemize}
            \item Principal Component Analysis (PCA)
            \item t-Distributed Stochastic Neighbor Embedding (t-SNE)
        \end{itemize}
    \end{itemize}
\end{frame}

\begin{frame}[fragile]
    \frametitle{Example Illustration}
    \begin{block}{Image Dataset}
        Consider a dataset of unlabeled animal images:
        \begin{itemize}
            \item Clusters similar images (e.g., all cats together).
            \item Identifies features (color, size, texture) without manual input.
        \end{itemize}
    \end{block}
\end{frame}

\begin{frame}[fragile]
    \frametitle{Key Points to Emphasize}
    \begin{itemize}
        \item Unsupervised learning requires no labeled data, making it practical in many real-world scenarios.
        \item Serves as a critical step in the data preprocessing pipeline, yielding valuable insights.
        \item Result interpretation depends significantly on domain expertise.
    \end{itemize}
\end{frame}

\begin{frame}[fragile]
    \frametitle{Conclusion}
    \begin{block}{Summary}
        Unsupervised learning plays a key role in machine learning, allowing the exploration of complex datasets without explicit labels. It uncovers patterns and enables data-driven decisions.
    \end{block}
\end{frame}

\begin{frame}[fragile]
    \frametitle{Next Steps}
    In the following slide, we will investigate what constitutes unsupervised learning and compare it to supervised learning to highlight the fundamental differences.
\end{frame}

\begin{frame}[fragile]{What is Unsupervised Learning? - Definition}
    \begin{block}{Definition of Unsupervised Learning}
        Unsupervised learning is a type of machine learning where algorithms identify patterns within datasets 
        without labeled outcomes. It differs from supervised learning, which relies on labeled input-output pairs.
    \end{block}

    \begin{itemize}
        \item **Nature of Data**: Data is unlabelled; consists of feature vectors without target values.
        \item **Objective**: Discover the inherent structure of the data (e.g., grouping data points).
        \item **Exploratory Focus**: Used for exploratory analysis to reveal hidden patterns.
    \end{itemize}
\end{frame}

\begin{frame}[fragile]{What is Unsupervised Learning? - Differences from Supervised Learning}
    \begin{block}{Differences Between Supervised and Unsupervised Learning}
        \begin{tabular}{|c|c|c|}
            \hline
            Aspect & Supervised Learning & Unsupervised Learning \\
            \hline
            \textbf{Data Type} & Labeled (input-output pairs) & Unlabeled (only input features) \\
            \hline
            \textbf{Goals} & Prediction of outcomes from known labels & Identification of patterns and structures \\
            \hline
            \textbf{Algorithms} & Regression, Classification & Clustering, Association \\
            \hline
            \textbf{Examples} & Spam detection, Image classification & Market basket analysis, Customer segmentation \\
            \hline
            \textbf{Performance Evaluation} & Assessed using accuracy, precision, recall & Evaluated qualitatively or using scores \\
            \hline
        \end{tabular}
    \end{block}
\end{frame}

\begin{frame}[fragile]{What is Unsupervised Learning? - Examples and Key Points}
    \begin{block}{Examples of Unsupervised Learning}
        \begin{enumerate}
            \item \textbf{Clustering}:
                \begin{itemize}
                    \item Grouping customers based on purchasing behavior without labeled data.
                    \item \textbf{K-means Formula}: 
                    \begin{equation}
                        J = \sum_{i=1}^{k} \sum_{j=1}^{n} ||x_j^{(i)} - \mu_i||^2
                    \end{equation}
                    \item Here, $J$ is the cost function, representing the sum of squared distances to cluster centers $\mu_i$.
                \end{itemize}
            \item \textbf{Dimensionality Reduction}:
                \begin{itemize}
                    \item Techniques like PCA reduce features while preserving essential information.
                    \item Easier visualization of high-dimensional data with 2D representation helps identify patterns.
                \end{itemize}
        \end{enumerate}
    \end{block}

    \begin{block}{Key Points to Emphasize}
        \begin{itemize}
            \item Unsupervised learning reveals hidden structures in data, aiding decision-making.
            \item Critical in applications from market analysis to network security anomaly detection.
        \end{itemize}
    \end{block}
\end{frame}

\begin{frame}[fragile]
  \frametitle{Applications of Unsupervised Learning}
  \begin{block}{Introduction to Unsupervised Learning}
    Unsupervised Learning is a type of machine learning where algorithms are used to find patterns or structures in data without supervision. This means the model is not given labeled outcomes to guide its learning. Instead, it identifies inherent structures within the input data. 
  \end{block}
\end{frame}

\begin{frame}[fragile]
  \frametitle{Key Applications of Unsupervised Learning - Overview}
  \begin{enumerate}
    \item Customer Segmentation
    \item Anomaly Detection
    \item Market Basket Analysis
    \item Dimensionality Reduction
    \item Image Compression and Processing
  \end{enumerate}
\end{frame}

\begin{frame}[fragile]
  \frametitle{Customer Segmentation}
  \begin{itemize}
    \item \textbf{Definition}: Identifying distinct groups of customers based on purchasing behavior and demographic data.
    \item \textbf{Example}: A retail company uses clustering algorithms like K-Means to segment customers into categories such as “frequent buyers,” “seasonal shoppers,” and “bargain hunters.”
    \item \textbf{Benefit}: Enables targeted marketing strategies tailored to each segment.
  \end{itemize}
\end{frame}

\begin{frame}[fragile]
  \frametitle{Anomaly Detection}
  \begin{itemize}
    \item \textbf{Definition}: Finding rare items, events, or observations that differ significantly from the majority of the data.
    \item \textbf{Example}: In fraud detection, identifying unusual patterns in credit card transactions using unsupervised learning.
    \item \textbf{Benefit}: Helps in proactively safeguarding assets against fraud.
  \end{itemize}
\end{frame}

\begin{frame}[fragile]
  \frametitle{Market Basket Analysis}
  \begin{itemize}
    \item \textbf{Definition}: Discovering associations between products purchased together.
    \item \textbf{Example}: An online retailer employs Association Rule Learning (e.g., Apriori algorithm) to find rules such as “customers who bought bread also bought butter.”
    \item \textbf{Benefit}: Improves product placement and targeted recommendations to increase sales.
  \end{itemize}
\end{frame}

\begin{frame}[fragile]
  \frametitle{Dimensionality Reduction}
  \begin{itemize}
    \item \textbf{Definition}: Simplifying datasets by reducing the number of features while retaining essential information.
    \item \textbf{Example}: Using Principal Component Analysis (PCA) to reduce a dataset with hundreds of variables into a few principal components.
    \item \textbf{Benefit}: Enhances the efficiency of data processing and can improve the performance of supervised learning algorithms.
  \end{itemize}
\end{frame}

\begin{frame}[fragile]
  \frametitle{Image Compression and Processing}
  \begin{itemize}
    \item \textbf{Definition}: Using algorithms to reduce the size of image files while maintaining essential details.
    \item \textbf{Example}: Techniques like K-Means clustering can group similar pixels in an image for effective compression.
    \item \textbf{Benefit}: Saves storage space and speeds up image rendering.
  \end{itemize}
\end{frame}

\begin{frame}[fragile]
  \frametitle{Conclusion}
  Unsupervised learning techniques play a crucial role in various industries by enabling data-driven decision-making and revealing insights that would otherwise be hidden. The flexibility to identify patterns in unlabeled data makes these techniques invaluable across different applications.
\end{frame}

\begin{frame}[fragile]
  \frametitle{Key Points to Remember}
  \begin{itemize}
    \item Unsupervised Learning identifies hidden patterns without labeled data.
    \item Applications span diverse fields including marketing, finance, and image processing.
    \item Techniques like clustering, anomaly detection, and dimensionality reduction enhance data analysis capabilities.
  \end{itemize}
\end{frame}

\begin{frame}[fragile]
    \frametitle{Clustering Overview}
    Clustering is a critical unsupervised learning technique that groups a set of objects so that items in the same group (or cluster) are more similar to one another than to those in other groups. It is widely used in data analysis to uncover patterns, extract insights, and segment data.
\end{frame}

\begin{frame}[fragile]
    \frametitle{Importance of Clustering in Data Analysis}
    \begin{itemize}
        \item \textbf{Data Simplification}: Reduces complexity in large datasets.
        \item \textbf{Pattern Recognition}: Helps identify underlying structures in data.
        \item \textbf{Segmentation}: Facilitates targeted marketing and personalization.
        \item \textbf{Anomaly Detection}: Aids in recognizing outliers, indicating potential fraud or defects.
    \end{itemize}
\end{frame}

\begin{frame}[fragile]
    \frametitle{Common Clustering Methods}
    \begin{enumerate}
        \item \textbf{K-Means Clustering}
            \begin{itemize}
                \item Minimizes variance within clusters.
                \item Steps:
                \begin{enumerate}
                    \item Choose number of clusters (K).
                    \item Initialize K centroids randomly.
                    \item Assign points to nearest centroid.
                    \item Update centroids based on mean of assigned points.
                    \item Repeat until convergence.
                \end{enumerate}
                \item \textbf{Use Cases}: Market segmentation, image compression, document clustering.
            \end{itemize}
        \item \textbf{Hierarchical Clustering}
            \begin{itemize}
                \item Builds a tree of clusters (merging or splitting).
                \item Types: Agglomerative and Divisive.
                \item \textbf{Use Cases}: Gene expression analysis, social network analysis.
            \end{itemize}
        \item \textbf{DBSCAN}
            \begin{itemize}
                \item Groups densely packed points and identifies outliers.
                \item Key Concepts:
                \begin{itemize}
                    \item $\epsilon$: Maximum distance for neighborhood.
                    \item MinPts: Minimum points to form a dense region.
                \end{itemize}
                \item \textbf{Use Cases}: Spatial data analysis, geographical clustering.
            \end{itemize}
    \end{enumerate}
\end{frame}

\begin{frame}[fragile]
    \frametitle{K-Means Clustering}
    \begin{block}{What is K-Means Clustering?}
        K-Means Clustering is an unsupervised learning algorithm that partitions a dataset into K distinct clusters based on feature similarity. It aims to group similar data points together while ensuring that each group (or cluster) is as different from others as possible.
    \end{block}
\end{frame}

\begin{frame}[fragile]
    \frametitle{Steps of the K-Means Algorithm}
    \begin{enumerate}
        \item \textbf{Initialization}:
        \begin{itemize}
            \item Choose the number of clusters, K.
            \item Randomly select K data points as initial centroids.
        \end{itemize}
        
        \item \textbf{Assignment Step}:
        \begin{itemize}
            \item Assign each data point to the nearest centroid based on Euclidean distance.
            \item \textbf{Distance Formula}: 
            \begin{equation}
                d(x_i, c_j) = \sqrt{\sum_{k=1}^{n} (x_{ik} - c_{jk})^2}
            \end{equation}
        \end{itemize}
        
        \item \textbf{Update Step}:
        \begin{itemize}
            \item Recalculate the centroids by averaging the data points assigned to that cluster.
            \item New Centroid Calculation:
            \begin{equation}
                c_j = \frac{1}{m} \sum_{i=1}^{m} x_i
            \end{equation}
        \end{itemize}
        
        \item \textbf{Convergence Check}:
        \begin{itemize}
            \item Repeat Assignment and Update steps until centroids stabilize.
        \end{itemize}
    \end{enumerate}
\end{frame}

\begin{frame}[fragile]
    \frametitle{Example of K-Means Clustering}
    \begin{block}{Imagine clustering a set of animals based on features like weight and height:}
        \begin{itemize}
            \item \textbf{Data Points:}
            \begin{itemize}
                \item Cat (Weight: 4kg, Height: 25cm)
                \item Dog (Weight: 10kg, Height: 40cm)
                \item Rabbit (Weight: 2kg, Height: 15cm)
            \end{itemize}
            \item Choosing K = 2: Initialize centroids randomly.
            \item Assign animals to the nearest centroid based on weight and height.
            \item Update centroids by averaging the assigned animals’ features.
            \item Repeat until convergence, resulting in two clusters: smaller animals (cats and rabbits) and larger animals (dogs).
        \end{itemize}
    \end{block}
\end{frame}

\begin{frame}[fragile]
    \frametitle{Use Cases of K-Means Clustering}
    \begin{itemize}
        \item \textbf{Market Segmentation}: Identifying different customer segments based on purchasing behavior.
        \item \textbf{Image Compression}: Reducing the number of colors in an image by grouping similar colors.
        \item \textbf{Anomaly Detection}: Identifying unusual data points that do not fit well into any cluster.
    \end{itemize}
    
    \begin{block}{Key Points to Emphasize}
        \begin{itemize}
            \item Sensitive to the initial placement of centroids; different initial conditions can lead to different results.
            \item Requires the number of clusters (K) to be specified beforehand.
            \item Can be computationally efficient for large datasets but struggles with noise and outliers.
        \end{itemize}
    \end{block}
\end{frame}

\begin{frame}
  \frametitle{Hierarchical Clustering}
  \begin{block}{Overview}
    Hierarchical clustering is an unsupervised learning technique used to group similar data points into clusters by creating a hierarchy of clusters. This method does not require a predefined number of clusters, making it particularly useful for exploratory data analysis.
  \end{block}
\end{frame}

\begin{frame}
  \frametitle{Types of Hierarchical Clustering}
  \begin{enumerate}
    \item \textbf{Agglomerative Hierarchical Clustering (Bottom-Up Approach):}
    \begin{itemize}
      \item Begins with each data point as its own cluster.
      \item Iteratively merges the closest pairs of clusters.
      \item Common linkage criteria include:
      \begin{itemize}
        \item \textbf{Single Linkage}: Shortest distance between data points.
        \item \textbf{Complete Linkage}: Maximum distance between points.
        \item \textbf{Average Linkage}: Average distance of all points.
      \end{itemize}
    \end{itemize}

    \item \textbf{Divisive Hierarchical Clustering (Top-Down Approach):}
    \begin{itemize}
      \item Begins with one all-inclusive cluster and recursively splits it.
      \item Less common and more computationally intensive.
    \end{itemize}
  \end{enumerate}
\end{frame}

\begin{frame}
  \frametitle{Dendrogram Construction}
  \begin{block}{Dendrogram}
    A dendrogram is a tree-like diagram that illustrates the arrangement of the clusters formed by hierarchical clustering.
    \begin{itemize}
      \item Each leaf node represents an individual data point.
      \item Each branch represents a merge or split of clusters.
      \item Height of merges indicates the distance between clusters; higher merges imply less similarity.
    \end{itemize}
  \end{block}

  \begin{block}{Example}
    Consider distances between points A, B, C, D:
    \begin{itemize}
      \item Distance(A, B) = 1
      \item Distance(A, C) = 3
      \item Distance(B, C) = 2
      \item Distance(D, A) = 4
      \item Distance(D, B) = 5
      \item Distance(D, C) = 6
    \end{itemize}
    The merging process will be:
    \begin{enumerate}
      \item Merge A and B (distance = 1).
      \item Merge {A, B} with C (distance = 2).
      \item Merge {A, B, C} with D (distance = 4).
    \end{enumerate}
  \end{block}
\end{frame}

\begin{frame}[fragile]
  \frametitle{Code Example}
  \begin{block}{Python Code for Agglomerative Clustering}
    \begin{lstlisting}[language=Python]
import numpy as np
import matplotlib.pyplot as plt
from scipy.cluster.hierarchy import dendrogram, linkage

# Sample data
data = np.array([[1, 2], [2, 3], [3, 6], [7, 8], [8, 9]])

# Perform hierarchical clustering
linked = linkage(data, 'single')

# Create dendrogram
plt.figure(figsize=(10, 5))
dendrogram(linked)
plt.xlabel('Data Points')
plt.ylabel('Distance')
plt.title('Hierarchical Clustering Dendrogram')
plt.show()
    \end{lstlisting}
  \end{block}
\end{frame}

\begin{frame}
  \frametitle{Conclusion}
  \begin{block}{Key Points}
    \begin{itemize}
      \item \textbf{No Predefined Cluster Count:} Allows for in-depth exploration of data.
      \item \textbf{Visual Representation:} Dendrograms provide a clear overview of data structure.
      \item \textbf{Scalability:} Effective on smaller datasets; computationally expensive on larger ones.
    \end{itemize}
  \end{block}

  Hierarchical clustering is a powerful tool for discovering relationships within data through a structured approach. Its visual output in the form of a dendrogram makes it highly interpretive and usable for further analysis.
\end{frame}

\begin{frame}[fragile]
    \frametitle{Evaluating Clustering Methods - Overview}
    \begin{block}{Overview}
        Evaluating clustering methods is essential to understand how well our algorithms perform in grouping similar data points. Unlike supervised learning, where we have ground-truth labels, clustering often relies on intrinsic metrics or external benchmarks to assess quality.
    \end{block}
\end{frame}

\begin{frame}[fragile]
    \frametitle{Evaluating Clustering Methods - Key Metrics}
    
    \begin{block}{Key Metrics for Evaluating Clusters}
        \begin{enumerate}
            \item \textbf{Internal Evaluation Metrics}:
                \begin{itemize}
                    \item \textbf{Silhouette Score}:
                        \begin{equation}
                        s(i) = \frac{b(i) - a(i)}{\max(a(i), b(i))}
                        \end{equation}
                        where:
                        \begin{itemize}
                            \item \(a(i)\) = Average distance from point \(i\) to points in its own cluster.
                            \item \(b(i)\) = Average distance from point \(i\) to points in the nearest cluster.
                        \end{itemize}
                    \item \textbf{Davies-Bouldin Index}: Lower values indicate better clustering.
                    \item \textbf{Dunn Index}: Higher values suggest better clustering.
                \end{itemize}
            \item \textbf{External Evaluation Metrics}:
                \begin{itemize}
                    \item \textbf{Adjusted Rand Index (ARI)}: Ranges from -1 (poor clustering) to 1 (perfect clustering).
                    \item \textbf{Normalized Mutual Information (NMI)}: Ranges from 0 to 1, with 1 indicating perfect correlation.
                \end{itemize}
        \end{enumerate}
    \end{block}
\end{frame}

\begin{frame}[fragile]
    \frametitle{Evaluating Clustering Methods - Example and Conclusion}
    
    \begin{block}{Example}
        Consider a clustering algorithm applied to a dataset of animal species. 
        By examining the Silhouette Score, we can discern if an animal (like a lion) is more similar to its cluster (carnivores) than to an adjacent cluster (herbivores). A high Silhouette Score would suggest that the algorithm has grouped animals appropriately.
    \end{block}
    
    \begin{block}{Conclusion}
        Evaluating clustering methods effectively requires a blend of internal and external metrics. Understanding the metrics in the context of the data being analyzed will lead to clearer insights and guide improvement. In the next chapter, we will explore how dimensionality reduction techniques can enhance our clustering efforts.
    \end{block}
\end{frame}

\begin{frame}[fragile]
    \frametitle{Evaluating Clustering Methods - Practical Application}
    
    \begin{block}{Practical Application: Code Snippet}
    Here’s a brief example using Python’s \texttt{sklearn} library to compute the Silhouette Score:
    \begin{lstlisting}[language=Python]
from sklearn.metrics import silhouette_score
from sklearn.cluster import KMeans
import numpy as np

# Example data
data = np.random.rand(100, 2)  # 100 points in 2D
kmeans = KMeans(n_clusters=3)
clusters = kmeans.fit_predict(data)

# Calculate Silhouette Score
score = silhouette_score(data, clusters)
print(f'Silhouette Score: {score}')
    \end{lstlisting}
    \end{block}
\end{frame}

\begin{frame}[fragile]
    \frametitle{Dimensionality Reduction Techniques - Introduction}
    \begin{block}{What is Dimensionality Reduction?}
        Dimensionality Reduction is a process used in data preprocessing that aims to reduce the number of features in a dataset while preserving its essential characteristics, thereby simplifying models and improving efficiency.
    \end{block}
\end{frame}

\begin{frame}[fragile]
    \frametitle{Dimensionality Reduction Techniques - Significance in Machine Learning}
    \begin{itemize}
        \item \textbf{Mitigation of the Curse of Dimensionality:}
        \begin{itemize}
            \item High-dimensional datasets can lead to models that overfit the training data.
            \item Reducing dimensions helps retain only the most informative features.
        \end{itemize}
        
        \item \textbf{Improved Visualization:}
        \begin{itemize}
            \item Makes it easier to identify patterns or clusters.
            \item Projects high-dimensional data into 2D or 3D for better visualization.
        \end{itemize}
        
        \item \textbf{Enhanced Computation Efficiency:}
        \begin{itemize}
            \item Fewer features lead to faster training times and less memory usage.
        \end{itemize}
        
        \item \textbf{Noise Reduction:}
        \begin{itemize}
            \item Removes irrelevant features and enhances model performance.
        \end{itemize}
    \end{itemize}
\end{frame}

\begin{frame}[fragile]
    \frametitle{Dimensionality Reduction Techniques - Common Techniques}
    \begin{enumerate}
        \item \textbf{Principal Component Analysis (PCA):}
        \begin{itemize}
            \item Transform dataset into a new coordinate system maximizing variance.
            \item Example: Reducing 100 features to 2 principal components.
            \begin{equation}
                \text{maximize} \quad \sum_{i=1}^{N} (x_i - \mu)^2
            \end{equation}
        \end{itemize}
        
        \item \textbf{t-Distributed Stochastic Neighbor Embedding (t-SNE):}
        \begin{itemize}
            \item Effective for visualizing high-dimensional datasets by preserving local structures.
        \end{itemize}
        
        \item \textbf{Linear Discriminant Analysis (LDA):}
        \begin{itemize}
            \item Finds a linear combination of features to separate classes effectively.
        \end{itemize}
    \end{enumerate}
\end{frame}

\begin{frame}
    \titlepage
\end{frame}

\begin{frame}[fragile]
    \frametitle{What is PCA?}
    \begin{itemize}
        \item Principal Component Analysis (PCA) is a statistical technique for dimensionality reduction.
        \item Preserves as much variance as possible in high-dimensional datasets.
        \item Transforms the data into a new coordinate system where the axes (principal components) capture the directions of maximum variance.
    \end{itemize}
\end{frame}

\begin{frame}[fragile]
    \frametitle{Mathematical Foundations of PCA}
    \begin{enumerate}
        \item \textbf{Data Centering:}
        \begin{equation}
            X_i' = X_i - \bar{X}
        \end{equation}
        
        \item \textbf{Covariance Matrix Calculation:}
        \begin{equation}
            C = \frac{1}{n-1} X^T X
        \end{equation}
        
        \item \textbf{Eigenvalue Decomposition:}
        \begin{equation}
            Cv = \lambda v
        \end{equation}
        
        \item \textbf{Selecting Principal Components:}
        \begin{itemize}
            \item Sort eigenvalues and eigenvectors in descending order.
            \item Choose the top \(k\) eigenvectors.
        \end{itemize}

        \item \textbf{Projecting the Data:}
        \begin{equation}
            Y = X W
        \end{equation}
    \end{enumerate}
\end{frame}

\begin{frame}[fragile]
    \frametitle{Applications of PCA}
    \begin{itemize}
        \item \textbf{Data Visualization:} 
        \begin{itemize}
            \item Reduces dimensions for better plotting of high-dimensional data.
        \end{itemize}
        
        \item \textbf{Noise Reduction:}
        \begin{itemize}
            \item Enhances model performance and interpretability by removing less significant features. 
        \end{itemize}
        
        \item \textbf{Preprocessing:}
        \begin{itemize}
            \item Used before supervised learning to reduce overfitting and computation time.
        \end{itemize}
    \end{itemize}
\end{frame}

\begin{frame}[fragile]
    \frametitle{Example and Key Points}
    \begin{itemize}
        \item \textbf{Example:} 
        \begin{itemize}
            \item PCA can reduce features for flowers (e.g., sepal length, sepal width) from 4 dimensions to 2 for visualization.
        \end{itemize}
        
        \item \textbf{Key Points:} 
        \begin{itemize}
            \item PCA does not classify; it transforms features.
            \item Strength lies in summarizing complex datasets and revealing patterns.
        \end{itemize}
    \end{itemize}
\end{frame}

\begin{frame}[fragile]
    \frametitle{Summary}
    \begin{block}{}
        PCA is an essential unsupervised learning technique for dimensionality reduction by identifying axes that capture maximum variance in data. 
        Understanding PCA helps in handling high-dimensional spaces effectively, simplifying data while maintaining critical information.
    \end{block}
\end{frame}

\begin{frame}[fragile]
    \frametitle{t-Distributed Stochastic Neighbor Embedding (t-SNE)}
    \begin{block}{Overview of t-SNE}
        t-Distributed Stochastic Neighbor Embedding (t-SNE) is a nonlinear dimensionality reduction technique used for visualizing high-dimensional data. It excels at retaining the relationships between data points in lower-dimensional space, making it effective for exploratory data analysis and visualization.
    \end{block}
\end{frame}

\begin{frame}[fragile]
    \frametitle{How t-SNE Works - Part 1}
    \begin{enumerate}
        \item \textbf{Pairwise Similarities in High Dimensions:}
        \begin{itemize}
            \item t-SNE models similarities between data points, converting distances into probabilities.
            \item For points \( x_i \) and \( x_j \):
            \begin{equation}
                p_{j|i} = \frac{\exp\left(-\frac{\|x_i - x_j\|^2}{2\sigma^2}\right)}{\sum_{k \neq i} \exp\left(-\frac{\|x_i - x_k\|^2}{2\sigma^2}\right)}
            \end{equation}
        \end{itemize}
        
        \item \textbf{Symmetrization:}
        \begin{itemize}
            \item The similarities are symmetrized to create joint probability distribution \( p_{ij} \):
            \begin{equation}
                p_{ij} = \frac{p_{j|i} + p_{i|j}}{2N}
            \end{equation}
        \end{itemize}
    \end{enumerate}
\end{frame}

\begin{frame}[fragile]
    \frametitle{How t-SNE Works - Part 2}
    \begin{enumerate}
        \setcounter{enumi}{2}
        \item \textbf{Low-Dimensional Representation:}
        \begin{itemize}
            \item t-SNE finds a representation \( y_i \) based on a t-distribution:
            \begin{equation}
                q_{ij} = \frac{(1 + \|y_i - y_j\|^2)^{-1}}{\sum_{k \neq l} (1 + \|y_i - y_k\|^2)^{-1}}
            \end{equation}
        \end{itemize}
        
        \item \textbf{Cost Function:}
        \begin{itemize}
            \item t-SNE minimizes the Kullback-Leibler divergence:
            \begin{equation}
                C(Y) = KL(P || Q) = \sum_{i}\sum_{j} p_{ij} \log\frac{p_{ij}}{q_{ij}}
            \end{equation}
        \end{itemize}
    \end{enumerate}
\end{frame}

\begin{frame}[fragile]
    \frametitle{Applications and Key Points}
    \begin{block}{Applications of t-SNE}
        \begin{itemize}
            \item \textbf{Data Visualization:} Useful in visualizing clusters in datasets across various fields.
            \item \textbf{Image and Text Data:} Helps understand embeddings from neural networks.
        \end{itemize}
    \end{block}
    
    \begin{block}{Key Points}
        \begin{itemize}
            \item Nonlinear technique capturing complex relationships.
            \item Maintains local structure of data.
            \item Parameter sensitivity (e.g., perplexity).
        \end{itemize}
    \end{block}
\end{frame}

\begin{frame}[fragile]
    \frametitle{Example Code Snippet}
    \begin{lstlisting}[language=Python]
from sklearn.manifold import TSNE
import matplotlib.pyplot as plt

# Assuming X is your high-dimensional dataset
tsne = TSNE(n_components=2, perplexity=30)
X_embedded = tsne.fit_transform(X)

# Plotting the results
plt.scatter(X_embedded[:, 0], X_embedded[:, 1])
plt.title('t-SNE Visualization of High-Dimensional Data')
plt.xlabel('Component 1')
plt.ylabel('Component 2')
plt.show()
    \end{lstlisting}
\end{frame}

\begin{frame}[fragile]
    \frametitle{Applications of Dimensionality Reduction}
    \begin{block}{Introduction to Dimensionality Reduction}
        Dimensionality reduction is a technique used in machine learning and data analysis to simplify models, decrease computational cost, and enhance interpretability by focusing on the most relevant features.
    \end{block}
\end{frame}

\begin{frame}[fragile]
    \frametitle{Key Applications (Part 1)}
    \begin{enumerate}
        \item \textbf{Data Visualization} 
            \begin{itemize}
                \item Example: Using PCA or t-SNE to visualize high-dimensional datasets (e.g., gene expression data).
            \end{itemize}
        
        \item \textbf{Noise Reduction}
            \begin{itemize}
                \item Example: Autoencoders filter out noise from image data, retaining essential components and improving classification quality.
            \end{itemize}
    \end{enumerate}
\end{frame}

\begin{frame}[fragile]
    \frametitle{Key Applications (Part 2)}
    \begin{enumerate}
        \setcounter{enumi}{3} % Continues from where we left off
        \item \textbf{Improving Model Performance}
            \begin{itemize}
                \item Example: Reducing features from thousands to hundreds enhances training speed and accuracy with algorithms like SVM.
            \end{itemize}

        \item \textbf{Facilitating Clustering}
            \begin{itemize}
                \item Example: Applying UMAP before K-means helps to reveal underlying data structures, leading to meaningful customer segments.
            \end{itemize}
    \end{enumerate}
\end{frame}

\begin{frame}[fragile]
    \frametitle{Summary & Additional Notes}
    \begin{block}{Summary & Key Points}
        \begin{itemize}
            \item Enhances Interpretability: Simplifies data for better understanding.
            \item Optimal Trade-off: Balances dimensionality reduction against potential information loss.
            \item Applicable Across Domains: Vital in various fields like NLP and image recognition.
        \end{itemize}
    \end{block}
    
    \begin{block}{Conclusion}
        Dimensionality reduction is essential for effective processing of high-dimensional data, from visualization to model efficiency.
    \end{block}

    \begin{lstlisting}[language=Python]
from sklearn.decomposition import PCA
from sklearn.manifold import TSNE

# PCA Example
pca = PCA(n_components=2)
pca_result = pca.fit_transform(data)

# t-SNE Example
tsne = TSNE(n_components=2)
tsne_result = tsne.fit_transform(data)
    \end{lstlisting}
\end{frame}

\begin{frame}[fragile]
    \frametitle{Integration of Clustering and Dimensionality Reduction}
    \begin{block}{Overview}
        Clustering and dimensionality reduction are complementary techniques in unsupervised learning that help extract meaningful insights from high-dimensional data. By combining these methods, we simplify complex datasets and identify patterns.
    \end{block}
\end{frame}

\begin{frame}[fragile]
    \frametitle{Key Concepts}
    \begin{itemize}
        \item \textbf{Dimensionality Reduction}
            \begin{itemize}
                \item \textbf{Definition}: Processes that reduce the number of random variables under consideration.
                \item \textbf{Techniques}:
                    \begin{itemize}
                        \item \textit{Principal Component Analysis (PCA)}: Projects data onto lower-dimensional space while preserving variance.
                        \item \textit{t-Distributed Stochastic Neighbor Embedding (t-SNE)}: Focuses on preserving local similarities nonlinearly.
                    \end{itemize}
            \end{itemize}
        \item \textbf{Clustering}
            \begin{itemize}
                \item \textbf{Definition}: Grouping objects such that those in the same group are more similar to one another.
                \item \textbf{Techniques}:
                    \begin{itemize}
                        \item \textit{K-Means Clustering}: Partitions data into K distinct clusters based on distance to the centroid.
                        \item \textit{DBSCAN}: Groups together points that are closely packed, marking others as outliers.
                    \end{itemize}
            \end{itemize}
    \end{itemize}
\end{frame}

\begin{frame}[fragile]
    \frametitle{How They Work Together}
    \begin{enumerate}
        \item \textbf{Dimensionality Reduction Prior to Clustering}
            \begin{itemize}
                \item \textit{Purpose}: Simplifying the data reduces computational complexity and noise.
                \item \textit{Example}: PCA can reduce high-dimensional datasets (e.g., images) to manageable scales.
            \end{itemize}
        \item \textbf{Clustering After Dimensionality Reduction}
            \begin{itemize}
                \item \textit{Purpose}: Reveals patterns in reduced dimensions, which are less obvious in high dimensions.
                \item \textit{Example}: K-Means can identify consumer segments after PCA application.
            \end{itemize}
    \end{enumerate}
\end{frame}

\begin{frame}[fragile]
    \frametitle{Practical Workflow}
    \begin{enumerate}
        \item \textbf{Start with High-Dimensional Data}:
            \begin{itemize}
                \item Collect data with numerous features (e.g., customer data with 50+ attributes).
            \end{itemize}
        \item \textbf{Apply Dimensionality Reduction}:
            \begin{itemize}
                \item Use PCA/t-SNE to reduce to 2 or 3 dimensions.
                \item Retain a significant amount of variance (e.g., 90\%).
            \end{itemize}
        \item \textbf{Apply Clustering Algorithm}:
            \begin{itemize}
                \item Implement K-Means or DBSCAN on reduced data and analyze clusters.
            \end{itemize}
    \end{enumerate}
\end{frame}

\begin{frame}[fragile]
    \frametitle{Practical Example in Code}
    \begin{lstlisting}[language=Python]
from sklearn.decomposition import PCA
from sklearn.cluster import KMeans
import matplotlib.pyplot as plt

# Sample data: X is a high-dimensional dataset
X = ...

# Step 1: Dimensionality Reduction
pca = PCA(n_components=2)
X_reduced = pca.fit_transform(X)

# Step 2: Clustering
kmeans = KMeans(n_clusters=3)
clusters = kmeans.fit_predict(X_reduced)

# Visualization
plt.scatter(X_reduced[:, 0], X_reduced[:, 1], c=clusters)
plt.title('K-Means Clustering on PCA-reduced Data')
plt.xlabel('First Principal Component')
plt.ylabel('Second Principal Component')
plt.show()
    \end{lstlisting}
\end{frame}

\begin{frame}[fragile]
    \frametitle{Conclusion}
    \begin{block}{Key Points}
        \begin{itemize}
            \item \textbf{Complementary Nature}: Each technique enhances the other; dimensionality reduction makes clustering more efficient.
            \item \textbf{Visual Insights}: Reduced dimensions allow easier visualization of clusters, aiding decision-making.
            \item \textbf{Data Quality \& Noise}: Effective dimensionality reduction improves clustering results by focusing on relevant features.
        \end{itemize}
    \end{block}
\end{frame}

\begin{frame}[fragile]
    \frametitle{Challenges in Unsupervised Learning - Introduction}
    \begin{block}{What is Unsupervised Learning?}
        Unsupervised learning is a subset of machine learning where the model learns patterns from unlabeled data. 
        Unlike supervised learning, where we have clear labels for training, unsupervised learning techniques aim to discover inherent structures without prior knowledge about the data.
    \end{block}
\end{frame}

\begin{frame}[fragile]
    \frametitle{Challenges in Unsupervised Learning - Key Challenges}
    \begin{enumerate}
        \item \textbf{Lack of Ground Truth}
            \begin{itemize}
                \item Without labeled data, validating model performance becomes challenging.
                \item \textit{Example:} In customer segmentation, clustering customers without labels lacks a benchmark to confirm accuracy.
            \end{itemize}
            
        \item \textbf{Choice of Algorithm}
            \begin{itemize}
                \item Different algorithms yield different structures based on assumptions.
                \item \textit{Example:} K-means assumes spherical clusters, which may not fit actual data shapes.
            \end{itemize}
            
        \item \textbf{Parameter Sensitivity}
            \begin{itemize}
                \item Algorithms often require parameter tuning for optimal results.
                \item \textit{Example:} Too few clusters may merge dissimilar groups; too many can lead to noise.
            \end{itemize}
    \end{enumerate}
\end{frame}

\begin{frame}[fragile]
    \frametitle{Challenges in Unsupervised Learning - Continued}
    \begin{enumerate}[resume]
        \item \textbf{High Dimensionality}
            \begin{itemize}
                \item As feature count increases, the meaning of distances diminishes.
                \item \textit{Example:} PCA may overlook important features, affecting representation quality.
            \end{itemize}
            
        \item \textbf{Interpretability}
            \begin{itemize}
                \item Outcomes can be complex, hindering actionable insights.
                \item \textit{Example:} Clusters may represent a mix of attributes that don’t translate into clear decisions.
            \end{itemize}
            
        \item \textbf{Dependency on Data Quality}
            \begin{itemize}
                \item Poor quality or noisy data can impair techniques, necessitating preprocessing.
                \item \textit{Example:} Outliers can distort clustering, causing incorrect groupings.
            \end{itemize}
    \end{enumerate}
\end{frame}

\begin{frame}[fragile]
    \frametitle{Challenges in Unsupervised Learning - Summary and Conclusion}
    \begin{block}{Summary of Key Points}
        \begin{itemize}
            \item Unsupervised learning is challenging due to lack of labeled data.
            \item Sensitivity to algorithm choice and parameter tuning is significant.
            \item Data dimensionality and quality heavily influence effectiveness.
            \item Results are often difficult to interpret, affecting practical applications.
        \end{itemize}
    \end{block}
    \begin{block}{Conclusion}
        Understanding these challenges is crucial for effectively applying unsupervised learning in real-world scenarios. By addressing these issues, we can enhance the use of unsupervised algorithms for valuable insights from data.
    \end{block}
\end{frame}

\begin{frame}[fragile]
    \frametitle{Case Studies in Unsupervised Learning}
    \begin{block}{Overview}
        Unsupervised learning is a type of machine learning that identifies patterns in datasets without labeled outcomes. 
        This technique is beneficial for exploring data and gaining insights that may not be immediately apparent. 
        We will analyze several real-world case studies showcasing the effectiveness of unsupervised learning.
    \end{block}
\end{frame}

\begin{frame}[fragile]
    \frametitle{Case Study 1: Customer Segmentation}
    \begin{itemize}
        \item \textbf{Context}: A retail company aims to better understand its customer base for improved marketing strategies.
        \item \textbf{Technique Used}: K-Means Clustering.
        \item \textbf{Process}:
        \begin{enumerate}
            \item Data Collection: Gather customer data including purchase history, demographics, and online behavior.
            \item Feature Selection: Identify key features such as frequency of purchase, average order value, and customer age.
            \item Clustering: Apply K-Means to categorize customers into distinct segments based on similarity.
        \end{enumerate}
        \item \textbf{Outcome}: The company identified high-value customer segments and tailored personalized marketing campaigns, resulting in a 25\% increase in customer engagement.
    \end{itemize}
\end{frame}

\begin{frame}[fragile]
    \frametitle{Case Study 2: Anomaly Detection in Network Security}
    \begin{itemize}
        \item \textbf{Context}: A cybersecurity firm seeks to detect unusual patterns indicative of security breaches.
        \item \textbf{Technique Used}: DBSCAN (Density-Based Spatial Clustering of Applications with Noise).
        \item \textbf{Process}:
        \begin{enumerate}
            \item Data Stream: Continuously analyze network traffic data.
            \item Model Application: Use DBSCAN to model normal usage patterns. Anomalies are flagged outside these clusters.
        \end{enumerate}
        \item \textbf{Outcome}: Several potential threats were identified before they materialized, protecting client data and reducing breach incidents.
    \end{itemize}
\end{frame}

\begin{frame}[fragile]
    \frametitle{Case Study 3: Market Basket Analysis}
    \begin{itemize}
        \item \textbf{Context}: A grocery store aims to determine frequently purchased products together.
        \item \textbf{Technique Used}: Association Rule Learning using the Apriori Algorithm.
        \item \textbf{Process}:
        \begin{enumerate}
            \item Transaction Data: Analyze historical sales data to find patterns in product purchases.
            \item Rule Generation: Generate association rules (e.g., if customers buy bread, they are likely to buy butter).
        \end{enumerate}
        \item \textbf{Outcome}: Enhanced product placement strategies and promotional offers, leading to a 15\% increase in bundled product sales.
    \end{itemize}
\end{frame}

\begin{frame}[fragile]
    \frametitle{Key Insights from Unsupervised Learning}
    \begin{itemize}
        \item \textbf{Versatility}: Applicable in various domains including marketing, security, and retail.
        \item \textbf{Pattern Discovery}: Excels in revealing hidden patterns without predefined labels.
        \item \textbf{Actionable Insights}: Insights from unsupervised learning significantly impact business strategies and decision-making.
    \end{itemize}
\end{frame}

\begin{frame}[fragile]
    \frametitle{Conclusion: Power of Unsupervised Learning}
    \begin{block}{Summary}
        Unsupervised learning provides powerful tools for analyzing data and uncovering insights. 
        The case studies illustrate its application across different fields and how organizations leverage these techniques for a competitive edge.
    \end{block}
\end{frame}

\begin{frame}[fragile]
    \frametitle{Implementation Example}
    \begin{block}{Example Code for K-Means Clustering}
    \begin{lstlisting}[language=Python]
from sklearn.cluster import KMeans
import pandas as pd

# Load customer data
data = pd.read_csv('customer_data.csv')

# K-Means Clustering 
kmeans = KMeans(n_clusters=5)
data['Cluster'] = kmeans.fit_predict(data[['Age', 'Annual Income']])
    \end{lstlisting}
    \end{block}
\end{frame}

\begin{frame}[fragile]
  \frametitle{Ethical Considerations in Unsupervised Learning}
  \begin{block}{Overview}
    Unsupervised learning is a powerful tool for uncovering hidden patterns in data without labeled outcomes. 
    However, it raises significant ethical challenges that practitioners must address for responsible use of technology.
  \end{block}
\end{frame}

\begin{frame}[fragile]
  \frametitle{Key Ethical Considerations - Part 1}
  \begin{enumerate}
    \item \textbf{Bias and Fairness}
      \begin{itemize}
        \item Unsupervised learning algorithms can amplify existing biases in data.
        \item \textbf{Example}: Clustering demographic data may lead to biased segmentations.
        \item \textbf{Impact}: Can perpetuate discrimination in recruitment, policing, and healthcare.
      \end{itemize}
    
    \item \textbf{Transparency}
      \begin{itemize}
        \item The complexity of models can lead to a lack of transparency in decisions.
        \item \textbf{Example}: PCA can obscure the causes of pattern detection.
        \item \textbf{Impact}: Users may distrust automated systems due to lack of understanding.
      \end{itemize}
  \end{enumerate}
\end{frame}

\begin{frame}[fragile]
  \frametitle{Key Ethical Considerations - Part 2}
  \begin{enumerate}
    \setcounter{enumi}{2} % start from the third item
    \item \textbf{Data Privacy}
      \begin{itemize}
        \item Using personal data raises privacy violation concerns.
        \item \textbf{Example}: Clustering without anonymizing personal data could expose individuals.
        \item \textbf{Impact}: Can lead to security breaches and loss of trust.
      \end{itemize}

    \item \textbf{Informed Consent}
      \begin{itemize}
        \item Organizations must ensure individuals consent to their data usage.
        \item \textbf{Example}: Analyzing web behavior without notifying users is ethically dubious.
        \item \textbf{Impact}: Lack of consent can lead to ethical and legal issues.
      \end{itemize}

    \item \textbf{Misinterpretation of Results}
      \begin{itemize}
        \item Misleading conclusions can arise without appropriate context.
        \item \textbf{Example}: Misleading clustering can lead stakeholders to incorrect strategic decisions.
        \item \textbf{Impact}: Poor decisions can harm organizations and individuals.
      \end{itemize}
  \end{enumerate}
\end{frame}

\begin{frame}[fragile]
  \frametitle{Ethical Guidelines for Practice}
  \begin{itemize}
    \item Evaluate Data Sources: Ensure diverse and representative data sets to minimize bias.
    \item Enhance Transparency: Strive for explainable models clarifying how insights are derived.
    \item Prioritize Privacy: Implement strong data protection measures and anonymize personal data.
    \item Involve Stakeholders: Encourage dialogue with data subjects about data use.
    \item Regular Audit and Review: Continuously monitor ethical implications of models and outputs.
  \end{itemize}
\end{frame}

\begin{frame}[fragile]
  \frametitle{Conclusion}
  Being vigilant about these ethical considerations protects individuals and strengthens the integrity of data science. 
  Responsible application of unsupervised learning fosters innovation while upholding ethical standards.
\end{frame}

\begin{frame}[fragile]
    \frametitle{Conclusion and Future Directions - Key Takeaways}
    \begin{enumerate}
        \item \textbf{Understanding Unsupervised Learning}:
        \begin{itemize}
            \item Refers to machine learning tasks where the model learns from unlabelled data.
        \end{itemize}
        
        \item \textbf{Core Techniques}:
        \begin{itemize}
            \item \textit{Clustering}: Grouping similar data points (e.g., K-Means).
            \item \textit{Dimensionality Reduction}: Reducing variables (e.g., PCA).
            \item \textit{Anomaly Detection}: Identifying outliers.
        \end{itemize}
        
        \item \textbf{Applications}:
        \begin{itemize}
            \item Market segmentation and customer grouping.
            \item Social network community detection.
            \item Image compression and feature extraction.
        \end{itemize}

        \item \textbf{Ethical Implications}:
        \begin{itemize}
            \item Consideration of data privacy and potential bias.
        \end{itemize}
    \end{enumerate}
\end{frame}

\begin{frame}[fragile]
    \frametitle{Conclusion and Future Directions - Future Trends}
    \begin{enumerate}
        \setcounter{enumi}{5} % Continue from the previous frame
        \item \textbf{Integration with Supervised Learning}:
        \begin{itemize}
            \item Combining unsupervised methods with supervised techniques.
        \end{itemize}

        \item \textbf{Advancements in Neural Networks}:
        \begin{itemize}
            \item Use of deep learning techniques (e.g., GANs, VAEs).
        \end{itemize}

        \item \textbf{Improved Interpretability}:
        \begin{itemize}
            \item Techniques to gain insights into model outputs.
        \end{itemize}

        \item \textbf{Transfer Learning}:
        \begin{itemize}
            \item Applying knowledge from one domain to related domains.
        \end{itemize}

        \item \textbf{Real-time and Streaming Data Analysis}:
        \begin{itemize}
            \item Algorithms for real-time clustering and anomaly detection.
        \end{itemize}

        \item \textbf{Incorporating Domain Knowledge}:
        \begin{itemize}
            \item Integrating prior knowledge to enhance results.
        \end{itemize}
    \end{enumerate}
\end{frame}

\begin{frame}[fragile]
    \frametitle{Conclusion and Future Directions - Key Points and Example}
    \begin{itemize}
        \item Unsupervised learning is powerful for data analysis without labeled datasets.
        \item Applications span various sectors, highlighting versatility.
        \item Ethical considerations must be prioritized.
        \item Future advancements promise enhanced insights.
    \end{itemize}
    
    \textbf{Illustrative Example:} \\
    Consider a retail company analyzing customer purchasing behavior. Clustering enables grouping customers by purchase patterns without prior labels, which can be integrated with supervised models for improved targeted marketing strategies.
\end{frame}


\end{document}