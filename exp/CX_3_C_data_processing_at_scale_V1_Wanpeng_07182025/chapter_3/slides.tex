\documentclass[aspectratio=169]{beamer}

% Theme and Color Setup
\usetheme{Madrid}
\usecolortheme{whale}
\useinnertheme{rectangles}
\useoutertheme{miniframes}

% Additional Packages
\usepackage[utf8]{inputenc}
\usepackage[T1]{fontenc}
\usepackage{graphicx}
\usepackage{booktabs}
\usepackage{listings}
\usepackage{amsmath}
\usepackage{amssymb}
\usepackage{xcolor}
\usepackage{tikz}
\usepackage{pgfplots}
\pgfplotsset{compat=1.18}
\usetikzlibrary{positioning}
\usepackage{hyperref}

% Custom Colors
\definecolor{myblue}{RGB}{31, 73, 125}
\definecolor{mygray}{RGB}{100, 100, 100}
\definecolor{mygreen}{RGB}{0, 128, 0}
\definecolor{myorange}{RGB}{230, 126, 34}
\definecolor{mycodebackground}{RGB}{245, 245, 245}

% Set Theme Colors
\setbeamercolor{structure}{fg=myblue}
\setbeamercolor{frametitle}{fg=white, bg=myblue}
\setbeamercolor{title}{fg=myblue}
\setbeamercolor{section in toc}{fg=myblue}
\setbeamercolor{item projected}{fg=white, bg=myblue}
\setbeamercolor{block title}{bg=myblue!20, fg=myblue}
\setbeamercolor{block body}{bg=myblue!10}
\setbeamercolor{alerted text}{fg=myorange}

% Set Fonts
\setbeamerfont{title}{size=\Large, series=\bfseries}
\setbeamerfont{frametitle}{size=\large, series=\bfseries}
\setbeamerfont{caption}{size=\small}
\setbeamerfont{footnote}{size=\tiny}

% Footer and Navigation Setup
\setbeamertemplate{footline}{
  \leavevmode%
  \hbox{%
  \begin{beamercolorbox}[wd=.3\paperwidth,ht=2.25ex,dp=1ex,center]{author in head/foot}%
    \usebeamerfont{author in head/foot}\insertshortauthor
  \end{beamercolorbox}%
  \begin{beamercolorbox}[wd=.5\paperwidth,ht=2.25ex,dp=1ex,center]{title in head/foot}%
    \usebeamerfont{title in head/foot}\insertshorttitle
  \end{beamercolorbox}%
  \begin{beamercolorbox}[wd=.2\paperwidth,ht=2.25ex,dp=1ex,center]{date in head/foot}%
    \usebeamerfont{date in head/foot}
    \insertframenumber{} / \inserttotalframenumber
  \end{beamercolorbox}}%
  \vskip0pt%
}

% Turn off navigation symbols
\setbeamertemplate{navigation symbols}{}

% Title Page Information
\title[Hadoop Overview]{Week 3: Overview of Hadoop}
\author[J. Smith]{John Smith, Ph.D.}
\institute[University Name]{
  Department of Computer Science\\
  University Name\\
  \vspace{0.3cm}
  Email: email@university.edu\\
  Website: www.university.edu
}
\date{\today}

% Document Start
\begin{document}

\frame{\titlepage}

\begin{frame}[fragile]
    \frametitle{Introduction to Hadoop Ecosystem}
    \begin{block}{What is Hadoop?}
        Hadoop is an open-source framework for storing and processing large data sets in a distributed computing environment. It supports parallel data processing, making it essential for big data analytics.
    \end{block}
    
    \begin{itemize}
        \item \textbf{Scalability}: Easily scales by adding more machines.
        \item \textbf{Fault Tolerance}: Replicates data across nodes for integrity.
        \item \textbf{Cost-Effective}: Efficient use of commodity hardware.
    \end{itemize}
\end{frame}

\begin{frame}[fragile]
    \frametitle{Importance of Hadoop in Data Processing at Scale}
    
    Organizations generate massive amounts of data, challenging traditional systems. Hadoop addresses the "3 Vs" of big data:
    
    \begin{itemize}
        \item \textbf{Massive Storage}: Stores petabytes of data.
        \item \textbf{Data Processing Power}: Quick processing through parallel computing.
        \item \textbf{Support for Various Data Types}: Handles structured, semi-structured, and unstructured data.
    \end{itemize}
    
    \begin{block}{Example}
        A social media platform capturing millions of interactions can utilize Hadoop to process data in real-time for insights on user behavior.
    \end{block}
\end{frame}

\begin{frame}[fragile]
    \frametitle{Key Components of the Hadoop Ecosystem}

    Hadoop consists of several integral components for big data management:
    
    \begin{enumerate}
        \item \textbf{Hadoop Distributed File System (HDFS)}:
        \begin{itemize}
            \item Stores large files across multiple machines.
            \item \textbf{Important Terms:}
            \begin{itemize}
                \item \textbf{Blocks}: Data chunks for storage.
                \item \textbf{Nodes}: DataNode stores data; NameNode manages metadata.
            \end{itemize}
        \end{itemize}

        \item \textbf{MapReduce}:
        \begin{itemize}
            \item A programming model for processing large data sets.
            \item \textbf{Key Phases:}
            \begin{itemize}
                \item \textbf{Map}: Processes data into key-value pairs.
                \item \textbf{Reduce}: Aggregates key-value pairs for output.
            \end{itemize}
        \end{itemize}

        \item \textbf{Hadoop Common}: Libraries and utilities for other Hadoop modules.
        
        \item \textbf{Additional Tools}:
        \begin{itemize}
            \item \textbf{Apache Hive}: Data warehousing software for querying HDFS data.
            \item \textbf{Apache Pig}: Scripting language for data processing.
            \item \textbf{Apache HBase}: NoSQL database on HDFS.
            \item \textbf{Apache Spark}: Fast cluster-computing system.
        \end{itemize}
    \end{enumerate}
\end{frame}

\begin{frame}[fragile]
    \frametitle{Core Components of Hadoop - Overview}
    \begin{itemize}
        \item Hadoop is an open-source framework for distributed processing of large data sets.
        \item Main components:
        \begin{enumerate}
            \item HDFS (Hadoop Distributed File System)
            \item MapReduce
        \end{enumerate}
        \item Together, HDFS and MapReduce enable scalable and efficient data processing.
    \end{itemize}
\end{frame}

\begin{frame}[fragile]
    \frametitle{Core Components of Hadoop - HDFS}
    \begin{block}{HDFS (Hadoop Distributed File System)}
        \begin{itemize}
            \item \textbf{Distributed Storage}: Stores large files across multiple machines.
            \item \textbf{Block Storage}: Files are divided into blocks (default size: 128 MB or 256 MB).
            \item \textbf{Fault Tolerance}: Data is replicated across multiple nodes (default replication factor: 3).
        \end{itemize}
    \end{block}
    \pause
    \textbf{Example:} A 1 GB file would be split into 8 blocks of 128 MB each. If one block's node fails, data can be retrieved from replicas.
\end{frame}

\begin{frame}[fragile]
    \frametitle{Core Components of Hadoop - MapReduce}
    \begin{block}{MapReduce}
        \begin{itemize}
            \item \textbf{Processing Engine}: A programming model for processing large data sets using parallel algorithms.
            \item \textbf{Two Phases}:
            \begin{enumerate}
                \item \textbf{Map Phase}: Divides input data into smaller sub-problems processed independently.
                \item \textbf{Reduce Phase}: Aggregates the results from the map phase for final output.
            \end{enumerate}
        \end{itemize}
    \end{block}
    \pause
    \textbf{Example: Word Count Program}
    \begin{itemize}
        \item \textbf{Map}: Emits key-value pairs (e.g., "word": 1).
        \item \textbf{Reduce}: Aggregates values by keys for the final count.
    \end{itemize}
\end{frame}

\begin{frame}[fragile]
    \frametitle{Core Components of Hadoop - Code Snippet}
    \begin{lstlisting}[language=java, basicstyle=\small]
    // Mapper
    public class WordCountMapper extends Mapper<LongWritable, Text, Text, IntWritable> {
        public void map(LongWritable key, Text value, Context context) {
            String[] words = value.toString().split(" ");
            for (String word : words) {
                context.write(new Text(word), new IntWritable(1));
            }
        }
    }

    // Reducer
    public class WordCountReducer extends Reducer<Text, IntWritable, Text, IntWritable> {
        public void reduce(Text key, Iterable<IntWritable> values, Context context) {
            int sum = 0;
            for (IntWritable val : values) {
                sum += val.get();
            }
            context.write(key, new IntWritable(sum));
        }
    }
    \end{lstlisting}
\end{frame}

\begin{frame}[fragile]
    \frametitle{Core Components of Hadoop - Key Points}
    \begin{itemize}
        \item \textbf{HDFS} enables scalable storage and ensures data durability through replication.
        \item \textbf{MapReduce} provides a systematic approach to parallel processing, improving speed for large data sets.
        \item Together, they form the backbone of the Hadoop ecosystem, facilitating efficient big data solutions.
    \end{itemize}
\end{frame}

\begin{frame}[fragile]
    \frametitle{Core Components of Hadoop - Conclusion}
    \begin{itemize}
        \item Understanding HDFS and MapReduce is essential for effective use of Hadoop.
        \item They complement each other: HDFS offers a robust storage solution, while MapReduce provides a powerful processing framework.
    \end{itemize}
\end{frame}

\begin{frame}[fragile]
    \frametitle{Hadoop Distributed File System (HDFS) - Overview}
    \begin{block}{Overview of HDFS}
        The Hadoop Distributed File System (HDFS) is the primary storage system used by Hadoop applications. It is designed to handle large amounts of data across multiple machines, ensuring reliability, efficiency, and scalability.
    \end{block}
\end{frame}

\begin{frame}[fragile]
    \frametitle{Hadoop Distributed File System (HDFS) - Key Features}
    \begin{enumerate}
        \item \textbf{Architecture:}
        \begin{itemize}
            \item Master-Slave Architecture
            \begin{itemize}
                \item \textbf{NameNode:} Manages metadata and regulates access to files.
                \item \textbf{DataNode:} Stores the actual data.
            \end{itemize}
            \item File System Hierarchy similar to traditional file systems.
            \item Data Flow: Clients interact with NameNode for file locations.
        \end{itemize}
        
        \item \textbf{Block Storage:}
        \begin{itemize}
            \item Data is divided into blocks (default size 128MB or 256MB).
            \item Each block is stored on different DataNodes for efficiency.
            \item Blocks are replicated (default factor is 3) for durability and availability.
        \end{itemize}
        
        \item \textbf{Fault Tolerance:}
        \begin{itemize}
            \item Automatic recovery of data following DataNode failures.
            \item Heartbeat Mechanism to monitor DataNode status.
            \item Data integrity checked using checksums.
        \end{itemize}
    \end{enumerate}
\end{frame}

\begin{frame}[fragile]
    \frametitle{Hadoop Distributed File System (HDFS) - Example Usage}
    Consider a company storing petabytes of data:
    \begin{enumerate}
        \item \textbf{Step 1:} Split data into blocks (~128MB each).
        \item \textbf{Step 2:} Replicate each block on multiple DataNodes.
        \item \textbf{Step 3:} NameNode maintains metadata and assists clients.
    \end{enumerate}
    If one DataNode fails, data remains accessible via replicas on other DataNodes, enhancing reliability.
\end{frame}

\begin{frame}[fragile]
    \frametitle{Hadoop Distributed File System (HDFS) - Key Points}
    \begin{itemize}
        \item HDFS is tailored for large-scale data processing and designed for high throughput.
        \item Fault tolerance features ensure minimal data loss and downtime.
        \item Block storage mechanism allows for efficient data access and handling of large files.
    \end{itemize}
    This structured approach helps visualize key concepts of HDFS, paving the way for learning about the next topic: the MapReduce Framework.
\end{frame}

\begin{frame}[fragile]
    \frametitle{Overview of MapReduce}
    \begin{block}{Definition}
        \textbf{MapReduce} is a programming model and processing framework utilized for handling large datasets across distributed systems. It simplifies data processing tasks by dividing them into two primary functions: \textbf{Map} and \textbf{Reduce}.
    \end{block}
\end{frame}

\begin{frame}[fragile]
    \frametitle{Key Concepts}
    \begin{enumerate}
        \item \textbf{Map Function:}
        \begin{itemize}
            \item Processes input data and generates key-value pair output. 
            \item Transforms the dataset into a structured format.
            \item \textit{Example:} Counting frequency of each word in a text document.
        \end{itemize}
        \item \textbf{Reduce Function:}
        \begin{itemize}
            \item Aggregates key-value output from the Map phase.
            \item Combines intermediate key-value pairs to produce a smaller dataset.
            \item \textit{Example:} Summing counts of each word to get total occurrences.
        \end{itemize}
    \end{enumerate}
\end{frame}

\begin{frame}[fragile]
    \frametitle{Workflow of MapReduce}
    \begin{enumerate}
        \item \textbf{Input Data Splitting:} 
            \begin{itemize} 
                \item Input data split into fixed-size blocks (default is 128 MB).
                \item Each block processed by a separate Mapper.
            \end{itemize} 
        \item \textbf{Mapping:} 
            \begin{itemize} 
                \item Each Mapper processes its block and emits intermediate key-value pairs.
            \end{itemize} 
        \item \textbf{Shuffling and Sorting:} 
            \begin{itemize} 
                \item Output of the Mappers is sorted by key and grouped by value.
                \item Prepares data for aggregation.
            \end{itemize} 
        \item \textbf{Reducing:} 
            \begin{itemize} 
                \item Reducers receive sorted key-value pairs and reduce them to a smaller set.
            \end{itemize} 
        \item \textbf{Output:} 
            \begin{itemize}
                \item Final results are written back into the distributed file system (HDFS).
            \end{itemize}
    \end{enumerate}
\end{frame}

\begin{frame}[fragile]
    \frametitle{Example Code Snippet}
    Here is a simple illustration of a MapReduce job to count word frequency in Python using the Hadoop Streaming API.
    
    \begin{block}{Mapper Code (mapper.py)}
        \begin{lstlisting}[language=Python]
import sys

for line in sys.stdin:
    words = line.strip().split()
    for word in words:
        # Output each word with a count of 1
        print(f"{word}\t1")
        \end{lstlisting}
    \end{block}

    \begin{block}{Reducer Code (reducer.py)}
        \begin{lstlisting}[language=Python]
import sys

current_word = None
current_count = 0
word = None

for line in sys.stdin:
    word, count = line.strip().split('\t')
    count = int(count)
    
    if current_word == word:
        current_count += count
    else:
        if current_word:
            # Output the previous word's count
            print(f"{current_word}\t{current_count}")
        current_count = count
        current_word = word

if current_word == word:
    # Output the final word's count
    print(f"{current_word}\t{current_count}")
        \end{lstlisting}
    \end{block}
\end{frame}

\begin{frame}[fragile]
    \frametitle{Key Points to Remember}
    \begin{itemize}
        \item \textbf{Scalability:} Efficiently handles increasing volumes of data through distributed processing.
        \item \textbf{Fault Tolerance:} Ensures robustness by reassigning tasks in case of node failure.
        \item \textbf{Suitable for Batch Processing:} Ideal for tasks that can be handled in large chunks, as opposed to real-time processing.
    \end{itemize}
    
    \begin{block}{Conclusion}
        The MapReduce framework is a powerful method for processing large datasets efficiently, particularly within the Hadoop ecosystem.
    \end{block}
\end{frame}

\begin{frame}
    \frametitle{Hadoop Ecosystem: Tools and Applications}
    \begin{block}{Overview}
        The Hadoop ecosystem consists of a variety of tools and applications that complement the core functionality of Hadoop, focusing on data storage, processing, and analysis.
    \end{block}
    \begin{itemize}
        \item Apache Hive
        \item Apache Pig
        \item Apache HBase
    \end{itemize}
\end{frame}

\begin{frame}[fragile]
    \frametitle{1. Apache Hive}
    \begin{itemize}
        \item \textbf{Description:} Hive is a data warehousing and SQL-like query language tool for Hadoop, enabling users to write queries in HiveQL.
        \item \textbf{Key Features:}
        \begin{itemize}
            \item Supports read/write operations on large datasets.
            \item Provides abstraction over MapReduce, simplifying complex data operations.
            \item Integrates well with existing Hadoop infrastructures.
        \end{itemize}
    \end{itemize}
    
    \begin{block}{Example Query}
        \begin{lstlisting}[language=SQL]
SELECT user_id, COUNT(activity) 
FROM user_logs 
WHERE activity_date BETWEEN '2023-01-01' AND '2023-01-31' 
GROUP BY user_id;
        \end{lstlisting}
    \end{block}
\end{frame}

\begin{frame}[fragile]
    \frametitle{2. Apache Pig}
    \begin{itemize}
        \item \textbf{Description:} Pig is a high-level platform for creating programs that run on Hadoop using Pig Latin.
        \item \textbf{Key Features:}
        \begin{itemize}
            \item Optimized for data flows, enabling complex processing pipelines.
            \item Abstracts the complexity of Java MapReduce coding.
            \item Facilitates iterative development with a simpler syntax.
        \end{itemize}
    \end{itemize}
    
    \begin{block}{Example Script}
        \begin{lstlisting}[language=Pig]
user_logs = LOAD 'hdfs://path/to/user_logs' USING PigStorage(',') 
    AS (user_id:chararray, activity:chararray, activity_date:chararray);
filtered_logs = FILTER user_logs BY activity_date >= '2023-01-01' 
    AND activity_date <= '2023-01-31';
grouped_logs = GROUP filtered_logs BY user_id;
count_logs = FOREACH grouped_logs GENERATE group, COUNT(filtered_logs);
DUMP count_logs;
        \end{lstlisting}
    \end{block}
\end{frame}

\begin{frame}
    \frametitle{3. Apache HBase}
    \begin{itemize}
        \item \textbf{Description:} HBase is a distributed, scalable, NoSQL database that runs on top of HDFS, allowing for random, real-time read/write access to large datasets.
        \item \textbf{Key Features:}
        \begin{itemize}
            \item Schema-less data model for flexible data representation.
            \item Efficient for random access queries.
            \item Close integration with Hadoop, offering high throughput for data processing.
        \end{itemize}
        \item \textbf{Use Case:} Ideal for applications needing real-time analytics, such as social media tracking user interactions with posts and updates.
    \end{itemize}
\end{frame}

\begin{frame}
    \frametitle{Key Points to Remember}
    \begin{itemize}
        \item \textbf{Integration:} All these tools integrate seamlessly within the Hadoop ecosystem, providing a robust architecture for big data management.
        \item \textbf{User Accessibility:} Allows non-programmers or analysts to manipulate and analyze large datasets without deep programming knowledge.
        \item \textbf{Enhanced Functionality:} These tools enhance the usability of Hadoop, making it easier to work with complex data sets.
    \end{itemize}
\end{frame}

\begin{frame}[fragile]
    \frametitle{Advantages of Using Hadoop - Overview}
    Hadoop is a powerful framework for processing and storing large datasets in a distributed computing environment. Its architecture offers various advantages for organizations dealing with big data. Here, we will discuss three primary benefits: 
    \begin{itemize}
        \item Scalability
        \item Cost-effectiveness
        \item Flexibility
    \end{itemize}
\end{frame}

\begin{frame}[fragile]
    \frametitle{Advantages of Using Hadoop - Scalability}
    \begin{block}{Explanation}
        Hadoop's architecture allows for horizontal scaling, enabling capacity increases by simply adding more nodes to the existing framework, especially beneficial as data volume grows.
    \end{block}
    
    \begin{itemize}
        \item \textbf{Linear Scalability:} Adding more nodes allows for proportional data processing increases without performance drops.
        \item \textbf{Distributed Storage:} Data stored across multiple nodes enables parallel processing of large volumes.
    \end{itemize}
    
    \begin{block}{Example}
        Consider a retail company with a surge in customer data over holidays; they can add nodes to handle the increase instead of overhauling their entire system.
    \end{block}
\end{frame}

\begin{frame}[fragile]
    \frametitle{Advantages of Using Hadoop - Cost-Effectiveness}
    \begin{block}{Explanation}
        Hadoop runs on commodity hardware, allowing organizations to avoid investments in high-end servers and instead use off-the-shelf machines.
    \end{block}
    
    \begin{itemize}
        \item \textbf{Lower Investment:} Reduces upfront costs typically associated with specialized hardware.
        \item \textbf{Open Source:} No licensing fees, making it a fiscally attractive option for many companies.
    \end{itemize}
    
    \begin{block}{Example}
        A startup can deploy Hadoop on standard servers without significant funds, enabling growth without financial burden.
    \end{block}
\end{frame}

\begin{frame}[fragile]
    \frametitle{Advantages of Using Hadoop - Flexibility}
    \begin{block}{Explanation}
        Hadoop supports various data types and formats, from structured data in databases to unstructured data like text and images, enabling diverse data source utilization.
    \end{block}
    
    \begin{itemize}
        \item \textbf{Varied Data Processing:} Analyze data from multiple sources without the need for strict formatting.
        \item \textbf{Iterative Processing:} Allows running multiple analyses without predefined schemas, suitable for various analytical tasks.
    \end{itemize}
    
    \begin{block}{Example}
        A healthcare organization can analyze data from electronic health records, genetic data, and wearables to enhance patient care and research.
    \end{block}
\end{frame}

\begin{frame}[fragile]
    \frametitle{Conclusion}
    In conclusion, Hadoop’s advantages make it a compelling choice for organizations aiming to leverage big data. Its scalability, cost-effectiveness, and flexibility enable efficient analysis of vast data amounts. With the growing demand for modern data processing frameworks, understanding these advantages is critical for informed technological decisions.
\end{frame}

\begin{frame}[fragile]
    \frametitle{Challenges and Considerations - Overview}
    \begin{block}{Introduction}
        While Hadoop offers numerous advantages, various challenges must be addressed for a successful implementation. 
        Understanding these challenges helps organizations prepare adequately for transitions and integrations.
    \end{block}
\end{frame}

\begin{frame}[fragile]
    \frametitle{Challenges and Considerations - Data Security}
    \begin{block}{Data Security}
        \begin{itemize}
            \item **Overview**: Data security is a paramount concern due to the volume and sensitivity of data processed.
            \item **Considerations**:
                \begin{itemize}
                    \item \textbf{Access Control}: Implement strong authentication and authorization mechanisms (e.g., Kerberos).
                    \item \textbf{Data Encryption}: Utilize encryption at rest (e.g., Apache Ranger) and in transit (e.g., SSL/TLS). 
                \end{itemize}
            \item **Example**: A healthcare provider must ensure encrypted storage of patient records to comply with regulations like HIPAA.
        \end{itemize}
    \end{block}
\end{frame}

\begin{frame}[fragile]
    \frametitle{Challenges and Considerations - Integration and Performance}
    \begin{block}{Integration with Existing Systems}
        \begin{itemize}
            \item **Overview**: Legacy systems must work alongside Hadoop.
            \item **Considerations**:
                \begin{itemize}
                    \item \textbf{Compatibility}: Ensure existing data architectures integrate seamlessly with Hadoop.
                    \item \textbf{Data Migration}: Develop strategies to maintain data quality during transfers.
                \end{itemize}
            \item **Example**: Enterprises may need to regularly pull data from SQL databases into Hadoop using tools like Apache Sqoop.
        \end{itemize}
    \end{block}

    \begin{block}{Performance Tuning}
        \begin{itemize}
            \item **Overview**: Performance varies based on job design and cluster configuration.
            \item **Considerations**:
                \begin{itemize}
                    \item \textbf{Resource Management}: Use Hadoop YARN for efficient resource allocation.
                    \item \textbf{Job Optimization}: Optimize MapReduce jobs to improve processing time.
                \end{itemize}
            \item **Example**: Optimizing the number of mappers and reducers can significantly decrease job run time.
        \end{itemize}
    \end{block}
\end{frame}

\begin{frame}[fragile]
    \frametitle{Challenges and Considerations - Workforce and Conclusion}
    \begin{block}{Skilled Workforce}
        \begin{itemize}
            \item **Overview**: Expertise in Hadoop is vital for implementation and maintenance.
            \item **Considerations**:
                \begin{itemize}
                    \item \textbf{Training Programs}: Invest in training employees on Hadoop technologies.
                    \item \textbf{Hiring Experts}: Bring in experienced professionals to guide the transition.
                \end{itemize}
            \item **Example**: Companies may opt for external training to bridge skill gaps.
        \end{itemize}
    \end{block}

    \begin{block}{Conclusion}
        While implementing Hadoop has challenges, careful planning around data security, system integration, performance tuning, and workforce skill development can mitigate these issues.
    \end{block}
\end{frame}

\begin{frame}[fragile]
    \frametitle{Summary and Key Takeaways - Recap of Key Concepts Covered}
    
    \begin{enumerate}
        \item \textbf{Overview of Hadoop ecosystem}
        \begin{itemize}
            \item \textbf{HDFS} (Hadoop Distributed File System): 
            Allows storage of large datasets across clusters of machines, ensuring data redundancy and fault-tolerance.
            
            \item \textbf{YARN} (Yet Another Resource Negotiator): 
            Manages resources in the cluster, allowing multiple applications to run concurrently.
            
            \item \textbf{MapReduce}: 
            A programming model for processing large data sets using a distributed algorithm on a cluster.
        \end{itemize}
    \end{enumerate}
    
\end{frame}

\begin{frame}[fragile]
    \frametitle{Summary and Key Takeaways - Scalability and Data Management}

    \begin{enumerate}
        \setcounter{enumi}{1}
        \item \textbf{Scalability of Data Processing}
        \begin{itemize}
            \item Hadoop's architecture is designed to scale horizontally, allowing the addition of more machines to handle increased workloads.
            
            \item \textit{Example}: If your data grows from 1 TB to 10 TB, you can simply add more nodes to your cluster.
        \end{itemize}
        
        \item \textbf{Data Storage and Management}
        \begin{itemize}
            \item HDFS efficiently stores both structured and unstructured data, essential for future analysis and reporting.
            \item Importance of managing the data lifecycle: ingestion, processing, storage, and eventual archiving or deletion.
        \end{itemize}
    \end{enumerate}

\end{frame}

\begin{frame}[fragile]
    \frametitle{Summary and Key Takeaways - Challenges and Key Points}

    \begin{enumerate}
        \setcounter{enumi}{3}
        \item \textbf{Challenges \& Considerations}
        \begin{itemize}
            \item Security, data governance, integration with legacy systems.
            \item \textit{Example}: Implementing permissions and authentication in Hadoop to secure sensitive data.
        \end{itemize}
        
        \item \textbf{Key Points to Emphasize}
        \begin{itemize}
            \item \textbf{Real-World Relevance}: Companies like Facebook and LinkedIn use Hadoop for data analysis to improve user experience.
            \item \textbf{Versatility}: Supports various programming languages (Java, Python, R) and integrates with tools (e.g., Apache Spark, Hive).
            \item \textbf{Community \& Support}: Strong open-source community ensures continual improvements.
        \end{itemize}
        
        \item \textbf{Conclusion}
        \begin{itemize}
            \item Understanding Hadoop and its components is essential for leveraging big data for insights and decision-making.
        \end{itemize}
    \end{enumerate}

\end{frame}

\begin{frame}[fragile]
    \frametitle{Q\&A Session - Introduction}
    \begin{itemize}
        \item This session provides an opportunity to clarify doubts and deepen the understanding of the Hadoop ecosystem.
        \item Focus can range from specific questions about concepts, tools, or functionalities within Hadoop.
    \end{itemize}
\end{frame}

\begin{frame}[fragile]
    \frametitle{Q\&A Session - Key Discussion Topics}
    \begin{enumerate}
        \item \textbf{Hadoop Architecture:}
            \begin{itemize}
                \item \textbf{HDFS:} Storage for large data sets across clusters.
                \item \textbf{YARN:} Manages resources and scheduling.
                \item \textbf{MapReduce:} Programming model for processing.
            \end{itemize}
        \item \textbf{Common Use Cases:}
            \begin{itemize}
                \item Big data analytics.
                \item Data warehousing and ETL.
                \item Log processing and analysis.
            \end{itemize}
        \item \textbf{Ecosystem Components:}
            \begin{itemize}
                \item \textbf{Pig:} High-level data flow programs.
                \item \textbf{Hive:} Data warehousing software.
                \item \textbf{HBase:} NoSQL database for large datasets.
            \end{itemize}
    \end{enumerate}
\end{frame}

\begin{frame}[fragile]
    \frametitle{Q\&A Session - FAQs and Engaging Points}
    \begin{itemize}
        \item \textbf{FAQs:}
            \begin{itemize}
                \item What types of processors can be used with Hadoop?
                \item How does Hadoop handle fault tolerance?
                \item Can Hadoop run in a cloud environment?
            \end{itemize}
        \item \textbf{Key Discussion Points:}
            \begin{itemize}
                \item Scalability through horizontal expansion.
                \item Cost-effectiveness with commodity hardware.
                \item Strong community support for open-source development.
            \end{itemize}
    \end{itemize}
\end{frame}

\begin{frame}[fragile]
    \frametitle{Q\&A Session - Code Snippet Example}
    Here’s a simple example of a MapReduce job in Hadoop:
    \begin{lstlisting}[language=java]
public class WordCount {
    public static class TokenizerMapper extends Mapper<Object, Text, Text, IntWritable> {
        // Mapper code here
    }
    
    public static class IntSumReducer extends Reducer<Text, IntWritable, Text, IntWritable> {
        // Reducer code here
    }
    
    public static void main(String[] args) throws Exception {
        // Job configuration and execution code here
    }
}
    \end{lstlisting}
\end{frame}

\begin{frame}[fragile]
    \frametitle{Q\&A Session - Conclusion}
    \begin{itemize}
        \item This Q\&A session aims to solidify your understanding and clarify any complexities regarding Hadoop.
        \item Feel free to share experiences, challenges, or insights related to Hadoop’s functionalities!
    \end{itemize}
\end{frame}


\end{document}