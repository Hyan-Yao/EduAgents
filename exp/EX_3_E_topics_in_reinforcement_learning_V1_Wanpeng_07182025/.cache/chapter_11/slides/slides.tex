\documentclass[aspectratio=169]{beamer}

% Theme and Color Setup
\usetheme{Madrid}
\usecolortheme{whale}
\useinnertheme{rectangles}
\useoutertheme{miniframes}

% Additional Packages
\usepackage[utf8]{inputenc}
\usepackage[T1]{fontenc}
\usepackage{graphicx}
\usepackage{booktabs}
\usepackage{listings}
\usepackage{amsmath}
\usepackage{amssymb}
\usepackage{xcolor}
\usepackage{tikz}
\usepackage{pgfplots}
\pgfplotsset{compat=1.18}
\usetikzlibrary{positioning}
\usepackage{hyperref}

% Custom Colors
\definecolor{myblue}{RGB}{31, 73, 125}
\definecolor{mygray}{RGB}{100, 100, 100}
\definecolor{mygreen}{RGB}{0, 128, 0}
\definecolor{myorange}{RGB}{230, 126, 34}
\definecolor{mycodebackground}{RGB}{245, 245, 245}

% Set Theme Colors
\setbeamercolor{structure}{fg=myblue}
\setbeamercolor{frametitle}{fg=white, bg=myblue}
\setbeamercolor{title}{fg=myblue}
\setbeamercolor{section in toc}{fg=myblue}
\setbeamercolor{item projected}{fg=white, bg=myblue}
\setbeamercolor{block title}{bg=myblue!20, fg=myblue}
\setbeamercolor{block body}{bg=myblue!10}
\setbeamercolor{alerted text}{fg=myorange}

% Set Fonts
\setbeamerfont{title}{size=\Large, series=\bfseries}
\setbeamerfont{frametitle}{size=\large, series=\bfseries}
\setbeamerfont{caption}{size=\small}
\setbeamerfont{footnote}{size=\tiny}

% Document Start
\begin{document}

\frame{\titlepage}

\begin{frame}[fragile]
    \frametitle{Introduction to Literature Review Presentation}
    \begin{block}{Overview}
        An overview of the objectives and importance of presenting research findings in reinforcement learning.
    \end{block}
\end{frame}

\begin{frame}[fragile]
    \frametitle{Objectives of the Literature Review}
    \begin{itemize}
        \item \textbf{Summarization of Existing Research:}
        \begin{itemize}
            \item Present a cohesive summary of past research.
            \item Identify trends, gaps, and areas of consensus within reinforcement learning (RL).
        \end{itemize}
        
        \item \textbf{Contextualizing Your Own Research:}
        \begin{itemize}
            \item Position findings within the established knowledge base to enhance understanding of research contributions.
        \end{itemize}
        
        \item \textbf{Identifying Research Gaps:}
        \begin{itemize}
            \item Highlight missing elements in existing literature, guiding future research directions.
        \end{itemize}
    \end{itemize}
\end{frame}

\begin{frame}[fragile]
    \frametitle{Importance of Presenting Research Findings in RL}
    \begin{itemize}
        \item \textbf{Advancing the Field:}
        \begin{itemize}
            \item Enables collaboration and innovation within the RL community.
            \item Fosters improvement and application of RL algorithms.
        \end{itemize}
        
        \item \textbf{Establishing Credibility:}
        \begin{itemize}
            \item Demonstrates familiarity with the field, establishing credibility among peers and stakeholders.
        \end{itemize}
        
        \item \textbf{Educating Stakeholders:}
        \begin{itemize}
            \item Makes complex RL concepts accessible to industry partners, policymakers, and educators.
        \end{itemize}
    \end{itemize}
\end{frame}

\begin{frame}[fragile]
    \frametitle{Key Points to Emphasize}
    \begin{itemize}
        \item \textbf{Clarity and Organization Matter:}
        \begin{itemize}
            \item Structure your presentation logically with a clear flow from introduction to results.
        \end{itemize}
        
        \item \textbf{Engagement through Visuals:}
        \begin{itemize}
            \item Incorporate graphs, charts, or diagrams to illustrate trends and comparisons.
        \end{itemize}
        
        \item \textbf{Critical Analysis over Simple Summary:}
        \begin{itemize}
            \item Critically analyze existing studies to foster deeper learning and discussion.
        \end{itemize}
    \end{itemize}
\end{frame}

\begin{frame}[fragile]
    \frametitle{Example of Literature Review Structure}
    \begin{enumerate}
        \item \textbf{Introduction:}
            \begin{itemize}
                \item Define the research question and scope.
            \end{itemize}
        \item \textbf{Methodology:}
            \begin{itemize}
                \item Discuss how the literature was gathered and selected.
            \end{itemize}
        \item \textbf{Key Findings:}
            \begin{itemize}
                \item Summarize the main findings and insights.
            \end{itemize}
        \item \textbf{Discussion:}
            \begin{itemize}
                \item Analyze trends, debates, and gaps identified.
            \end{itemize}
        \item \textbf{Conclusion:}
            \begin{itemize}
                \item Recap the findings and suggest future research directions.
            \end{itemize}
    \end{enumerate}
\end{frame}

\begin{frame}[fragile]
    \frametitle{Conclusion}
    \begin{block}{Significance of Literature Review}
        A comprehensive literature review is crucial for advancing knowledge in reinforcement learning. As you prepare for your presentation, consider the broader impact of your findings within the field and beyond.
    \end{block}
\end{frame}

\begin{frame}[fragile]
    \frametitle{Course Objectives Recap - Overview}
    \begin{itemize}
        \item Review key learning objectives: 
            \begin{itemize}
                \item Knowledge Acquisition
                \item Algorithm Implementation
                \item Literature Review
            \end{itemize}
    \end{itemize}
\end{frame}

\begin{frame}[fragile]
    \frametitle{Course Objectives Recap - Knowledge Acquisition}
    \begin{block}{Key Learning Objective 1: Knowledge Acquisition}
        \begin{itemize}
            \item \textbf{Definition:} Understanding the fundamental concepts and theories underlying reinforcement learning (RL).
            \item \textbf{Importance:} Gaining a solid foundational knowledge is crucial for analyzing and interpreting research findings.
            \item \textbf{Example:} Familiarity with key RL components such as agents, environments, states, actions, and rewards. This knowledge allows you to grasp how agents learn from the environment and adapt their strategies.
        \end{itemize}
    \end{block}
\end{frame}

\begin{frame}[fragile]
    \frametitle{Course Objectives Recap - Algorithm Implementation}
    \begin{block}{Key Learning Objective 2: Algorithm Implementation}
        \begin{itemize}
            \item \textbf{Definition:} The practical application of RL algorithms to solve problems or tasks.
            \item \textbf{Importance:} Implementing algorithms helps reinforce theoretical knowledge and provides hands-on experience with coding and debugging.
            \item \textbf{Example:} 
            \begin{lstlisting}[language=Python, basicstyle=\ttfamily]
    Initialize Q-table with zeros
    for each episode:
        Initialize state
        while not terminal:
            Choose action (e.g., epsilon-greedy)
            Take action, observe reward and new state
            Update Q-value:
            Q(state, action) ← Q(state, action) + α * 
               [reward + γ * max_a Q(new_state, action) - Q(state, action)]
            Update state to new state
            \end{lstlisting}
        \end{itemize}
    \end{block}
\end{frame}

\begin{frame}[fragile]
    \frametitle{Course Objectives Recap - Literature Review}
    \begin{block}{Key Learning Objective 3: Literature Review}
        \begin{itemize}
            \item \textbf{Definition:} Critically analyzing and synthesizing existing research on RL to identify trends, gaps, and future directions.
            \item \textbf{Importance:} A thorough literature review equips you with the knowledge to position your own research within the broader context of the discipline.
            \item \textbf{Example:} Summarizing key findings from at least three significant studies in RL that demonstrate different approaches, like deep reinforcement learning, policy gradients, and model-based RL.
        \end{itemize}
    \end{block}
\end{frame}

\begin{frame}[fragile]
    \frametitle{Key Points to Emphasize}
    \begin{itemize}
        \item \textbf{Integration of Learning:} All three objectives are interrelated; knowledge acquisition feeds into algorithm implementation, which is enhanced by literature reviews.
        \item \textbf{Practical Applications:} These objectives prepare you for academic research and tackling real-world problems using RL.
        \item \textbf{Continuous Learning:} Staying updated with new developments is essential in the rapidly evolving field of RL.
    \end{itemize}
\end{frame}

\begin{frame}[fragile]
    \frametitle{Research Topics in Reinforcement Learning - Introduction}
    \begin{block}{What is Reinforcement Learning (RL)?}
        Reinforcement Learning (RL) is a subset of machine learning where:
        \begin{itemize}
            \item An agent learns to make decisions by interacting with an environment.
            \item The goal is to maximize cumulative rewards guided by actions, observations, and feedback.
        \end{itemize}
    \end{block}
\end{frame}

\begin{frame}[fragile]
    \frametitle{Research Topics in Reinforcement Learning - Key Subfields}
    \begin{enumerate}
        \item Markov Decision Processes (MDPs)
        \item Value-Based Methods
        \item Policy Gradient Methods
        \item Exploration vs. Exploitation Dilemma
        \item Temporal Difference Learning (TD Learning)
        \item Multi-Agent Reinforcement Learning
    \end{enumerate}
\end{frame}

\begin{frame}[fragile]
    \frametitle{Markov Decision Processes (MDPs)}
    \begin{block}{Concept}
        MDPs provide a mathematical framework for modeling decision-making where outcomes are:
        \begin{itemize}
            \item Partly random and partly under the control of a decision-maker.
        \end{itemize}
    \end{block}
    \begin{block}{Key Components}
        \begin{itemize}
            \item States ($S$)
            \item Actions ($A$)
            \item Transition probabilities ($P$)
            \item Rewards ($R$)
            \item Discount factor ($\gamma$)
        \end{itemize}
    \end{block}
    \begin{block}{Example}
        A robot navigating a maze can be modeled as an MDP with:
        \begin{itemize}
            \item States: Positions
            \item Actions: Movements
            \item Rewards: Reaching the exit
        \end{itemize}
    \end{block}
\end{frame}

\begin{frame}[fragile]
    \frametitle{Value-Based Methods}
    \begin{block}{Concept}
        Focus on estimating the value of being in a state or taking an action in that state.
    \end{block}
    \begin{block}{Popular Algorithms}
        \begin{itemize}
            \item Q-Learning: A model-free algorithm that learns the value of actions in states.
            \begin{equation}
                Q(s, a) \leftarrow Q(s, a) + \alpha \left[ r + \gamma \max_{a'} Q(s', a') - Q(s, a) \right]
            \end{equation}
            \item Deep Q-Networks (DQN): Utilizes deep learning to approximate Q-values for high-dimensional state spaces.
        \end{itemize}
    \end{block}
    \begin{block}{Example}
        Learning to play a game by maximizing scores through Q-values.
    \end{block}
\end{frame}

\begin{frame}[fragile]
    \frametitle{Policy Gradient Methods and Exploration}
    \begin{block}{Policy Gradient Methods}
        \begin{itemize}
            \item Directly optimize the policy function rather than focusing on value functions.
            \item Key Algorithm: REINFORCE—a Monte Carlo method updating policies based on returns.
        \end{itemize}
    \end{block}
    \begin{block}{Exploration vs. Exploitation Dilemma}
        The balance between:
        \begin{itemize}
            \item Exploring new actions to discover their rewards.
            \item Exploiting known actions that yield high rewards.
        \end{itemize}
        \begin{itemize}
            \item Strategies: 
            \begin{itemize}
                \item $\epsilon$-greedy: Choose a random action to explore with probability $\epsilon$.
                \item Softmax selection: Actions chosen based on a probability distribution weighted by their estimated value.
            \end{itemize}
        \end{itemize}
    \end{block}
\end{frame}

\begin{frame}[fragile]
    \frametitle{Temporal Difference Learning and Multi-Agent RL}
    \begin{block}{Temporal Difference Learning (TD Learning)}
        A blend of Monte Carlo methods and dynamic programming; updates estimates based on learned estimates.
        \begin{itemize}
            \item Example: Updates the value of a state based on values of subsequent states.
        \end{itemize}
    \end{block}
    \begin{block}{Multi-Agent Reinforcement Learning}
        \begin{itemize}
            \item Involves multiple agents learning simultaneously, leading to complex interactions.
            \item Example: Autonomous vehicles coordinating traffic flow in a smart city.
        \end{itemize}
    \end{block}
\end{frame}

\begin{frame}[fragile]
    \frametitle{Conclusion}
    \begin{itemize}
        \item Reinforcement learning encompasses various subfields and techniques.
        \item Understanding MDPs is critical as they form the backbone of many RL algorithms.
        \item Both exploration and exploitation strategies are essential for developing effective RL agents.
    \end{itemize}
\end{frame}

\begin{frame}[fragile]
    \frametitle{Conducting a Literature Review - Introduction}
    A literature review is a comprehensive survey of existing research related to a specific topic, concept, or research question.
    
    It serves several crucial purposes in the research process:
    
    \begin{itemize}
        \item \textbf{Contextualizing Research:} Situates your study within existing knowledge to identify gaps, trends, and debates.
        \item \textbf{Building a Foundation:} Provides a theoretical framework and empirical background to promote robust methodology and hypothesis formation.
        \item \textbf{Identifying Methodologies:} Showcases various research methods from previous studies to inform your own design.
    \end{itemize}
\end{frame}

\begin{frame}[fragile]
    \frametitle{Conducting a Literature Review - Step-by-Step Process}
    \begin{enumerate}
        \item \textbf{Define Your Research Question or Topic}
            \begin{itemize}
                \item Clearly articulate the focus of your research.
                \item Example: ``What are the effects of reinforcement learning in game AI development?''
            \end{itemize}
        
        \item \textbf{Conduct a Comprehensive Search}
            \begin{itemize}
                \item Use academic databases and library resources to find relevant articles and papers.
                \item Refine your search with keywords, synonyms, and Boolean operators.
            \end{itemize}
        
        \item \textbf{Select Relevant Literature}
            \begin{itemize}
                \item Choose pertinent studies and consider the credibility of sources.
            \end{itemize}
        
        \item \textbf{Analyze and Synthesize the Literature}
            \begin{itemize}
                \item Organize findings and create an annotated bibliography to outline key points.
            \end{itemize}
        
        \item \textbf{Write the Review}
            \begin{itemize}
                \item Follow a clear structure including introduction, methodology, findings, and conclusion.
            \end{itemize}
    \end{enumerate}
\end{frame}

\begin{frame}[fragile]
    \frametitle{Conducting a Literature Review - Key Points and Conclusion}
    \begin{block}{Key Points to Emphasize}
        \begin{itemize}
            \item \textbf{Significance:} Critical for demonstrating the need for your research.
            \item \textbf{Critical Engagement:} Evaluate strengths and weaknesses of findings.
            \item \textbf{Iteration:} The process requires multiple rounds of reading and analysis.
        \end{itemize}
    \end{block}

    \vspace{0.5cm}
    \textbf{Conclusion:} Conducting a literature review is a foundational step that strengthens your study's validity and enhances your credibility as a researcher.
\end{frame}

\begin{frame}[fragile]
    \frametitle{Analyzing Literature Findings}
    \begin{block}{Introduction to Analysis}
        Analyzing findings in literature is crucial for understanding the breadth of research in a given field. It involves critically examining and synthesizing results from various studies, allowing researchers to draw conclusions and identify trends or gaps in the current knowledge.
    \end{block}
\end{frame}

\begin{frame}[fragile]
    \frametitle{Methods for Analyzing Literature Findings - Part 1}
    \begin{enumerate}
        \item \textbf{Critical Appraisal}
            \begin{itemize}
                \item Assess the quality, relevance, and credibility of each study.
                \item Consider factors such as:
                    \begin{itemize}
                        \item Research design (e.g., qualitative vs. quantitative)
                        \item Sample size and selection
                        \item Methodology used
                        \item Bias and limitations
                    \end{itemize}
                \item \textbf{Example:} A randomized controlled trial may be given more weight than a case study due to its rigorous design.
            \end{itemize}
        
        \item \textbf{Thematic Analysis}
            \begin{itemize}
                \item Identify recurring themes and patterns across studies.
                \item Group findings into categories that highlight similarities and differences.
                \item \textbf{Example:} In a review of studies on learning methods, themes may include "peer learning," "visual aids," and "technology integration."
            \end{itemize}
    \end{enumerate}
\end{frame}

\begin{frame}[fragile]
    \frametitle{Methods for Analyzing Literature Findings - Part 2}
    \begin{enumerate}
        \setcounter{enumi}{2} % Start enumeration from 3
        \item \textbf{Meta-Analysis}
            \begin{itemize}
                \item Statistical technique to combine results from multiple studies to arrive at a consolidated conclusion.
                \item Provides a stronger evidence base by increasing sample size and statistical power.
                \item \textbf{Formula for Effect Size:}
                \begin{equation}
                ES = \frac{M_1 - M_2}{SD_p}
                \end{equation}
                Where \( M_1 \) and \( M_2 \) are the means of groups being compared, and \( SD_p \) is the pooled standard deviation.
            \end{itemize}

        \item \textbf{Synthesis of Findings}
            \begin{itemize}
                \item Create a narrative that integrates findings, discussing not only what studies agree on but also where discrepancies exist.
                \item Highlight implications for practice or further research.
                \item \textbf{Example:} If multiple studies indicate that technology enhances learning, but one study finds no improvement, explore factors such as context or population differences.
            \end{itemize}

        \item \textbf{Using Frameworks}
            \begin{itemize}
                \item Employ theoretical frameworks or models that help structure the analysis.
                \item \textbf{Example:} The PICO (Population, Intervention, Comparison, and Outcome) framework can help focus the analysis on relevant studies.
            \end{itemize}
    \end{enumerate}
\end{frame}

\begin{frame}[fragile]
    \frametitle{Key Points to Emphasize}
    \begin{itemize}
        \item \textbf{Critical Thinking:} Analyze beyond surface-level findings; question methodologies and interpretations.
        \item \textbf{Contextual Understanding:} Always consider the context of each study, including settings, participants, and cultural factors.
        \item \textbf{Diverse Perspectives:} Include varied types of studies to provide a holistic view of the topic.
        \item \textbf{Documentation:} Keep thorough notes on each article's strengths, weaknesses, and contributions for your analysis.
    \end{itemize}
    
    \begin{block}{Conclusion}
        Analyzing literature findings allows researchers to create a comprehensive understanding of a topic, laying the groundwork for future research proposals that fill in identified gaps. This critical process ensures informed decision-making and scholarship advancement in your field. By following these methods, you will enhance your literature review by not only presenting findings but also critically engaging with the research landscape.
    \end{block}
\end{frame}

\begin{frame}[fragile]
    \frametitle{Formulating Research Proposals}
    A research proposal outlines a planned study, justifying the need for the research and establishing its relevance.  
    Formulating a coherent research proposal involves:
    \begin{itemize}
        \item Integrating findings from the literature.
        \item Identifying gaps in existing knowledge.
        \item Outlining future research directions.
    \end{itemize}
\end{frame}

\begin{frame}[fragile]
    \frametitle{Steps to Develop a Coherent Research Proposal}
    \begin{enumerate}
        \item \textbf{Identify Research Problems or Gaps}
            \begin{itemize}
                \item A research gap is an area not fully explored in existing literature.
                \item Example: Extensive research on climate change impacts on agriculture, but limited studies on its effects on mental health.
            \end{itemize}

        \item \textbf{Conduct a Comprehensive Literature Review}
            \begin{itemize}
                \item Analyze previous studies to understand research trends and methodologies.
                \item Use systematic frameworks like PRISMA.
            \end{itemize}
    \end{enumerate}
\end{frame}

\begin{frame}[fragile]
    \frametitle{Continued: Steps to Develop a Coherent Research Proposal}
    \begin{enumerate}
        \setcounter{enumi}{2} % Continue numbering from the previous slide
        \item \textbf{Define Your Research Questions}
            \begin{itemize}
                \item Articulate specific and measurable research questions.
                \item Example: "How does climate change affect the mental health of individuals in rural areas?"
            \end{itemize}

        \item \textbf{Establish Objectives and Hypotheses}
            \begin{itemize}
                \item State objectives and formulate testable hypotheses.
                \item Example: "Increased climate variability correlates with higher anxiety levels among rural populations."
            \end{itemize}

        \item \textbf{Select Appropriate Methodologies}
            \begin{itemize}
                \item Choose methodologies that align with your research questions.
                \item Example: Surveys for qualitative insights, statistical analysis for quantitative data.
            \end{itemize}
    \end{enumerate}
\end{frame}

\begin{frame}[fragile]
    \frametitle{Presentation Skills - Introduction}
    Presenting research findings is a critical skill for articulating complex information clearly and engagingly. Effective presentation skills can significantly enhance the understanding of your research. This slide outlines best practices for:
    \begin{itemize}
        \item Clarity
        \item Engagement
        \item Handling audience questions
    \end{itemize}
\end{frame}

\begin{frame}[fragile]
    \frametitle{Presentation Skills - Clarity}
    \begin{block}{1. Clarity in Presentation}
        \begin{itemize}
            \item \textbf{A. Structure Your Content}
            \begin{itemize}
                \item \textbf{Introduction:} Clearly state your research question or hypothesis.
                \item \textbf{Methods:} Briefly explain the methodology.
                \item \textbf{Findings:} Present key results with clear visuals (charts, graphs).
                \item \textbf{Conclusion:} Summarize the findings and their implications.
            \end{itemize}

            \item \textbf{B. Use Plain Language}
            \begin{itemize}
                \item Avoid jargon unless necessary; when used, define terms.
                \item Use visuals, such as diagrams, to illustrate complex ideas.
            \end{itemize}
            \textbf{Key Point:} Aim for a fifth-grade reading level in text on slides to ensure accessibility.
        \end{itemize}
    \end{block}
\end{frame}

\begin{frame}[fragile]
    \frametitle{Presentation Skills - Engagement Strategies}
    \begin{block}{2. Engagement Strategies}
        \begin{itemize}
            \item \textbf{A. Know Your Audience}
            \begin{itemize}
                \item Tailor your content based on the audience’s expertise and interests.
                \item Use relatable examples that resonate with their experiences.
                \item \textbf{Example:} Relate complex algorithms to everyday decision-making.
            \end{itemize}

            \item \textbf{B. Use Visual Aids Wisely}
            \begin{itemize}
                \item Incorporate graphs, charts, and images to support your narrative.
                \item Ensure slides are visually engaging but not cluttered.
            \end{itemize}
            \textbf{Key Point:} Follow the 10-20-30 rule.
        \end{itemize}
    \end{block}
\end{frame}

\begin{frame}[fragile]
    \frametitle{Presentation Skills - Handling Questions and Conclusion}
    \begin{block}{3. Handling Audience Questions}
        \begin{itemize}
            \item \textbf{A. Anticipate Questions}
            \begin{itemize}
                \item Prepare for potential questions based on your research.
                \item Consider creating a FAQ slide at the end.
            \end{itemize}

            \item \textbf{B. Encourage Questions}
            \begin{itemize}
                \item Invite questions throughout the presentation.
                \item Set clear time limits to maintain focus.
            \end{itemize}

            \item \textbf{C. Respond Thoughtfully}
            \begin{itemize}
                \item Take a moment to think before responding.
                \item Thank the audience for their questions.
            \end{itemize}
        \end{itemize}
    \end{block}
    
    \begin{block}{Conclusion}
        Effective presentation skills involve clarity, engagement, and the ability to handle questions gracefully. Regular practice and feedback are essential for improvement.
    \end{block}
\end{frame}

\begin{frame}[fragile]
    \frametitle{Ethical Considerations in Reinforcement Learning}
    \begin{block}{Introduction}
        Reinforcement Learning (RL) is a powerful subset of machine learning where agents learn to make decisions by taking actions in an environment to maximize cumulative rewards. 
        As its applications grow, particularly in sensitive areas such as healthcare, finance, and autonomous systems, ethical considerations become critical.
    \end{block}
\end{frame}

\begin{frame}[fragile]
    \frametitle{Key Ethical Considerations - Part 1}

    \begin{enumerate}
        \item \textbf{Bias in Training Data}
            \begin{itemize}
                \item RL systems learn from historical data. If the data reflects societal biases, the agent may perpetuate or amplify these biases.
                \item \textit{Example}: An RL agent trained on biased hiring data may favor certain demographics, disadvantaging other candidates.
            \end{itemize}
        
        \item \textbf{Transparency and Accountability}
            \begin{itemize}
                \item The decision-making process in RL can be opaque ("black box").
                \item \textit{Example}: In finance, if an RL model makes a poor investment decision, accountability may be unclear: is it the developers, the data, or the system?
            \end{itemize}
    \end{enumerate}
\end{frame}

\begin{frame}[fragile]
    \frametitle{Key Ethical Considerations - Part 2}

    \begin{enumerate}
        \setcounter{enumi}{2} % Continue from previous enumeration

        \item \textbf{Autonomy and Decision-Making}
            \begin{itemize}
                \item Increasingly autonomous RL agents challenge human control.
                \item \textit{Example}: An autonomous vehicle making split-second decisions raises ethical questions about the value of human life.
            \end{itemize}

        \item \textbf{Long-Term Societal Impact}
            \begin{itemize}
                \item Decisions by RL agents can impact societal structures and human behavior.
                \item \textit{Example}: RL in law enforcement could lead to over-policing communities based on learned data.
            \end{itemize}
    \end{enumerate}
\end{frame}

\begin{frame}[fragile]
    \frametitle{Key Points and Conclusion}

    \begin{block}{Key Points to Emphasize}
        \begin{itemize}
            \item \textbf{Ethical Governance}: Involve ethicists, sociologists, and community stakeholders in the design of RL systems.
            \item \textbf{Diversity in Development Teams}: Diversity can help recognize biases in datasets and improve model integrity.
            \item \textbf{Continuous Monitoring and Adjustment}: Ongoing evaluation is crucial to identify and correct biased patterns.
        \end{itemize}
    \end{block}

    \begin{block}{Conclusion}
        Navigating the ethical landscape of RL involves recognizing the consequences of automated decision-making. Researchers and developers must prioritize ethical considerations for beneficial societal outcomes.
    \end{block}
\end{frame}

\begin{frame}[fragile]
    \frametitle{Student Presentations - Overview}
    The Student Presentations during our literature review session are an opportunity for you to share your research insights and engage with your peers. 
    \begin{itemize}
        \item Consolidate understanding of the literature
        \item Enhance presentation skills
        \item Structure and expectations outlined below
    \end{itemize}
\end{frame}

\begin{frame}[fragile]
    \frametitle{Student Presentations - Structure}
    \begin{enumerate}
        \item \textbf{Introduction (1-2 minutes)}
        \begin{itemize}
            \item Provide an overarching view of your topic.
            \item State your research question or hypothesis clearly.
            \item Define key terms relevant to your literature review.
        \end{itemize}
        
        \item \textbf{Literature Overview (3 minutes)}
        \begin{itemize}
            \item Summarize key studies related to your topic.
            \item Highlight major findings and methodologies.
            \item Discuss gaps in the literature.
            \item \textit{Example:} "Ethical Implications of AI" with specific studies.
        \end{itemize}
    \end{enumerate}
\end{frame}

\begin{frame}[fragile]
    \frametitle{Student Presentations - Continuing Structure}
    \begin{enumerate}
        \setcounter{enumii}{2} % Continue enumerating from 2
        \item \textbf{Methods (2 minutes)}
        \begin{itemize}
            \item Outline research methods used in reviewed studies.
            \item Discuss appropriateness of methods.
        \end{itemize}

        \item \textbf{Findings (3 minutes)}
        \begin{itemize}
            \item Analyze and discuss literature findings.
            \item Engage the audience with thought-provoking questions.
        \end{itemize}

        \item \textbf{Conclusion (2 minutes)}
        \begin{itemize}
            \item Summarize main takeaways and implications.
            \item Discuss unexplored areas.
        \end{itemize}

        \item \textbf{Q\&A Session (2-3 minutes)}
        \begin{itemize}
            \item Encourage audience questions.
            \item Be prepared for critiques and to expand on arguments.
        \end{itemize}
    \end{enumerate}
\end{frame}

\begin{frame}[fragile]
    \frametitle{Key Presentation Tips and Expectations}
    \begin{block}{Key Presentation Tips}
        \begin{itemize}
            \item Engage your audience with stimulating questions.
            \item Use visual aids effectively, minimizing text.
            \item Practice to stay within 15 minutes.
            \item Be prepared for questions and critique.
        \end{itemize}
    \end{block}

    \begin{block}{Expectations}
        \begin{itemize}
            \item Clear language and meet the 15-20 minute limit.
            \item Properly cite sources to maintain academic integrity.
            \item Actively listen and respect diverse views during discussions.
        \end{itemize}
    \end{block}
\end{frame}

\begin{frame}[fragile]
    \frametitle{Conclusion}
    Your literature review presentations are crucial for your academic journey. 
    By articulating findings and engaging with peers, you enhance your understanding and improve communication skills. 
    Remember to practice, utilize your slides effectively, and enjoy sharing your research!
\end{frame}

\begin{frame}[fragile]
    \frametitle{Conclusion and Reflection - Key Takeaways}
    
    \begin{enumerate}
        \item \textbf{Diverse Perspectives}:
        \begin{itemize}
            \item Each presentation highlighted unique angles on the same topic, showcasing the multifaceted nature of research.
        \end{itemize}
        
        \item \textbf{Research Methodologies}:
        \begin{itemize}
            \item Various methodologies emphasized the importance of selecting appropriate research designs—quantitative and qualitative.
        \end{itemize}
        
        \item \textbf{Critical Engagement}:
        \begin{itemize}
            \item Students critiqued literature, identifying gaps and suggesting future research directions.
        \end{itemize}
        
        \item \textbf{Interdisciplinary Connections}:
        \begin{itemize}
            \item Illustrations of connections between fields revealed how interdisciplinary collaboration enhances understanding.
        \end{itemize}
        
        \item \textbf{Presentation Skills}:
        \begin{itemize}
            \item Highlighted the importance of effective communication in conveying complex ideas and enhancing audience comprehension.
        \end{itemize}
    \end{enumerate}
\end{frame}

\begin{frame}[fragile]
    \frametitle{Conclusion and Reflection - Learning Experience}
    
    \begin{itemize}
        \item \textbf{Self-Assessment}:
        \begin{itemize}
            \item Reflect on your learning from presenting and listening—how did peer interactions evolve your understanding?
        \end{itemize}
        
        \item \textbf{Peer Feedback}:
        \begin{itemize}
            \item Analyze feedback received during presentations—how can constructive criticism guide future research?
        \end{itemize}
        
        \item \textbf{Future Applications}:
        \begin{itemize}
            \item Consider how the skills developed (research, synthesis, presentation, critique) apply to your academic and professional endeavors.
        \end{itemize}
    \end{itemize}
\end{frame}

\begin{frame}[fragile]
    \frametitle{Conclusion and Reflection - Encouragement for Continuous Reflection}
    
    \begin{block}{Questions for Reflection}
        \begin{itemize}
            \item What was the most surprising aspect of your peers’ research?
            \item How did your perspective on the topic shift as a result of this experience?
            \item What strategies will you adopt in your future literature reviews or presentations?
        \end{itemize}
    \end{block}

    \begin{block}{Next Steps}
        \begin{itemize}
            \item Set personal goals for future research projects based on insights gained—deepening understanding of methodologies or enhancing critique skills.
        \end{itemize}
    \end{block}

    \begin{block}{Conclusion}
        By synthesizing insights from presentations and fostering a habit of reflection, enhance your academic journey and engage with dialogue in your field.
    \end{block}
\end{frame}


\end{document}