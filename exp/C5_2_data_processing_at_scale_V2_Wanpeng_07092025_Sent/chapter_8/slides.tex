\documentclass[aspectratio=169]{beamer}

% Theme and Color Setup
\usetheme{Madrid}
\usecolortheme{whale}
\useinnertheme{rectangles}
\useoutertheme{miniframes}

% Additional Packages
\usepackage[utf8]{inputenc}
\usepackage[T1]{fontenc}
\usepackage{graphicx}
\usepackage{booktabs}
\usepackage{listings}
\usepackage{amsmath}
\usepackage{amssymb}
\usepackage{xcolor}
\usepackage{tikz}
\usepackage{pgfplots}
\pgfplotsset{compat=1.18}
\usetikzlibrary{positioning}
\usepackage{hyperref}

% Custom Colors
\definecolor{myblue}{RGB}{31, 73, 125}
\definecolor{mygray}{RGB}{100, 100, 100}
\definecolor{mygreen}{RGB}{0, 128, 0}
\definecolor{myorange}{RGB}{230, 126, 34}
\definecolor{mycodebackground}{RGB}{245, 245, 245}

% Set Theme Colors
\setbeamercolor{structure}{fg=myblue}
\setbeamercolor{frametitle}{fg=white, bg=myblue}
\setbeamercolor{title}{fg=myblue}
\setbeamercolor{section in toc}{fg=myblue}
\setbeamercolor{item projected}{fg=white, bg=myblue}
\setbeamercolor{block title}{bg=myblue!20, fg=myblue}
\setbeamercolor{block body}{bg=myblue!10}
\setbeamercolor{alerted text}{fg=myorange}

% Set Fonts
\setbeamerfont{title}{size=\Large, series=\bfseries}
\setbeamerfont{frametitle}{size=\large, series=\bfseries}
\setbeamerfont{caption}{size=\small}
\setbeamerfont{footnote}{size=\tiny}

% Code Listing Style
\lstdefinestyle{customcode}{
  backgroundcolor=\color{mycodebackground},
  basicstyle=\footnotesize\ttfamily,
  breakatwhitespace=false,
  breaklines=true,
  commentstyle=\color{mygreen}\itshape,
  keywordstyle=\color{blue}\bfseries,
  stringstyle=\color{myorange},
  numbers=left,
  numbersep=8pt,
  numberstyle=\tiny\color{mygray},
  frame=single,
  framesep=5pt,
  rulecolor=\color{mygray},
  showspaces=false,
  showstringspaces=false,
  showtabs=false,
  tabsize=2,
  captionpos=b
}
\lstset{style=customcode}

% Custom Commands
\newcommand{\hilight}[1]{\colorbox{myorange!30}{#1}}
\newcommand{\source}[1]{\vspace{0.2cm}\hfill{\tiny\textcolor{mygray}{Source: #1}}}
\newcommand{\concept}[1]{\textcolor{myblue}{\textbf{#1}}}
\newcommand{\separator}{\begin{center}\rule{0.5\linewidth}{0.5pt}\end{center}}

% Footer and Navigation Setup
\setbeamertemplate{footline}{
  \leavevmode%
  \hbox{%
  \begin{beamercolorbox}[wd=.3\paperwidth,ht=2.25ex,dp=1ex,center]{author in head/foot}%
    \usebeamerfont{author in head/foot}\insertshortauthor
  \end{beamercolorbox}%
  \begin{beamercolorbox}[wd=.5\paperwidth,ht=2.25ex,dp=1ex,center]{title in head/foot}%
    \usebeamerfont{title in head/foot}\insertshorttitle
  \end{beamercolorbox}%
  \begin{beamercolorbox}[wd=.2\paperwidth,ht=2.25ex,dp=1ex,center]{date in head/foot}%
    \usebeamerfont{date in head/foot}
    \insertframenumber{} / \inserttotalframenumber
  \end{beamercolorbox}}%
  \vskip0pt%
}

% Turn off navigation symbols
\setbeamertemplate{navigation symbols}{}

% Title Page Information
\title[Hands-on with NoSQL]{Chapter 8: Hands-on with NoSQL: MongoDB \& Cassandra}
\author[J. Smith]{John Smith, Ph.D.}
\institute[University Name]{
  Department of Computer Science\\
  University Name\\
  \vspace{0.3cm}
  Email: email@university.edu\\
  Website: www.university.edu
}
\date{\today}

% Document Start
\begin{document}

\frame{\titlepage}

\begin{frame}[fragile]
    \frametitle{Introduction to NoSQL - Overview}
    \begin{block}{What is NoSQL?}
        NoSQL, short for ``Not Only SQL,'', refers to a class of databases designed to handle large volumes of unstructured and semi-structured data.
        Unlike traditional relational databases, NoSQL databases use varied data models, making them versatile for modern applications that demand scalability and flexibility.
    \end{block}
\end{frame}

\begin{frame}[fragile]
    \frametitle{Introduction to NoSQL - Key Characteristics}
    \begin{enumerate}
        \item \textbf{Schema-less Design}: NoSQL databases do not require a predefined schema, allowing for dynamic management of data and easier updates.
        
        \item \textbf{Scalability}: Designed for horizontal scaling, NoSQL databases can expand across distributed systems, making them suitable for handling enormous datasets.
        
        \item \textbf{Variety of Data Models}:
        \begin{itemize}
            \item Document Stores (e.g., MongoDB)
            \item Key-Value Stores (e.g., Redis)
            \item Column-Family Stores (e.g., Cassandra)
            \item Graph Databases (e.g., Neo4j)
        \end{itemize}
        
        \item \textbf{High Performance}: Optimized for high-speed data access and quick read/write operations, catering to high-traffic applications.
    \end{enumerate}
\end{frame}

\begin{frame}[fragile]
    \frametitle{Introduction to NoSQL - Significance and Examples}
    \begin{block}{Significance in Modern Data Processing}
        \begin{itemize}
            \item \textbf{Handling Big Data}: Capable of managing large volumes of information efficiently.
            \item \textbf{Flexibility with Data Types}: Supports various data types like JSON and XML.
            \item \textbf{Real-time Applications}: Enables real-time data processing for applications such as social media, recommendation engines, and IoT.
        \end{itemize}
    \end{block}

    \begin{block}{Examples}
        \begin{itemize}
            \item \textbf{MongoDB}: A document-oriented database ideal for fast querying and dynamic schemas.
            \item \textbf{Cassandra}: Known for high availability and fault tolerance, used by companies like Netflix.
        \end{itemize}
    \end{block}
    
    \begin{block}{Example of a MongoDB Document}
        \begin{lstlisting}
        {
          "name": "John Doe",
          "email": "john@example.com",
          "age": 30,
          "interests": ["Photography", "Travel"]
        }
        \end{lstlisting}
    \end{block}
\end{frame}

\begin{frame}[fragile]{Data Models: Relational vs. NoSQL - Introduction}
    \begin{block}{Overview}
        In today's data-driven world, it's crucial to understand the differences between relational databases and NoSQL databases, as each has unique strengths and weaknesses. This comparison helps you choose the right database model for your application's specific needs.
    \end{block}
\end{frame}

\begin{frame}[fragile]{Data Models: Relational vs. NoSQL - Relational Databases}
    \frametitle{Relational Databases}
    \begin{itemize}
        \item \textbf{Definition:} Relational databases, like MySQL and PostgreSQL, store data in structured tables with predefined schemas.
        \item \textbf{Key Characteristics:}
            \begin{itemize}
                \item \textbf{Schema-Based:} Require a fixed schema.
                \item \textbf{ACID Compliance:} Ensure strong consistency and reliability.
                \item \textbf{JOIN Operations:} Allow complex queries involving multiple tables.
            \end{itemize}
        \item \textbf{Use Cases:}
            \begin{itemize}
                \item Applications with structured data (e.g., ERP systems).
                \item Scenarios needing complex transactions (e.g., Banking).
            \end{itemize}
        \item \textbf{Limitations:}
            \begin{itemize}
                \item Scalability Issues: Vertical scaling can be expensive.
                \item Performance Degradation: Slow performance with large datasets.
            \end{itemize}
    \end{itemize}
\end{frame}

\begin{frame}[fragile]{Data Models: Relational vs. NoSQL - NoSQL Databases}
    \frametitle{NoSQL Databases}
    \begin{itemize}
        \item \textbf{Definition:} NoSQL databases, such as MongoDB and Cassandra, provide flexible data models designed to scale horizontally.
        \item \textbf{Key Characteristics:}
            \begin{itemize}
                \item \textbf{Schema-less:} Dynamic data structures without predefined schemas.
                \item \textbf{Eventual Consistency:} Emphasize availability over strong consistency.
                \item \textbf{Diverse Data Models:} Include key-value stores, document databases, etc.
            \end{itemize}
        \item \textbf{Use Cases:}
            \begin{itemize}
                \item Real-Time Big Data Processing (e.g., Social Media Analytics).
                \item Applications requiring scalability (e.g., IoT).
            \end{itemize}
        \item \textbf{Limitations:}
            \begin{itemize}
                \item Weaker Consistency Guarantees: May lead to conflicts with concurrent writes.
                \item Lack of Standardization: Different query languages and approaches.
            \end{itemize}
    \end{itemize}
\end{frame}

\begin{frame}[fragile]{Data Models: Relational vs. NoSQL - Conclusion}
    \begin{block}{Conclusion}
        Choosing between relational and NoSQL databases depends on your application's requirements, including data structure and scale. Understanding these differences empowers informed database selection.
    \end{block}
    \begin{itemize}
        \item \textbf{Key Points to Emphasize:}
            \begin{itemize}
                \item Relational: strong consistency, structured data.
                \item NoSQL: flexible, scalable, but potential consistency trade-offs.
            \end{itemize}
    \end{itemize}
\end{frame}

\begin{frame}[fragile]{Data Models: Relational vs. NoSQL - Example Diagrams}
    \frametitle{Example Diagrams}
    \textbf{Relational Database Diagram (Table Structure):}

    \begin{lstlisting}[basicstyle=\ttfamily]
    Customers Table:
    +----+------+---------+
    | ID | Name | Email   |
    +----+------+---------+
    | 1  | John | john@example.com |
    +----+------+---------+
    \end{lstlisting}

    \textbf{NoSQL Data (JSON Document):}
    \begin{lstlisting}[language=json]
    {
      "customer": {
         "id": 1,
         "name": "John",
         "email": "john@example.com",
         "orders": [
            {"orderId": 101, "amount": 299.99},
            {"orderId": 102, "amount": 149.99}
         ]
      }
    }
    \end{lstlisting}
\end{frame}

\begin{frame}[fragile]
    \frametitle{Understanding MongoDB}
    % Overview of MongoDB as a NoSQL database with its key advantages.
    MongoDB is a popular NoSQL database known for its flexibility and performance. It stores data in a BSON format, allowing for dynamic schemas ideal for modern applications.
\end{frame}

\begin{frame}[fragile]
    \frametitle{Key Features of MongoDB}
    \begin{enumerate}
        \item \textbf{Schema Flexibility}: Documents in the same collection can have different fields.
        \item \textbf{Scalability}: Supports horizontal scaling through sharding.
        \item \textbf{High Performance}: Delivers high read/write throughput.
        \item \textbf{Document-Based Storage}: Stores data in nested BSON documents.
        \item \textbf{Powerful Query Language}: Provides a rich language for complex queries.
    \end{enumerate}
\end{frame}

\begin{frame}[fragile]
    \frametitle{Architecture of MongoDB}
    \begin{itemize}
        \item \textbf{Database}: A container for collections.
        \item \textbf{Collection}: A grouping of MongoDB documents.
        \item \textbf{Document}: The basic data unit in BSON format.
        \item \textbf{Replica Sets}: A group maintaining the same dataset for redundancy.
        \item \textbf{Sharding}: Distributing data across multiple servers for scalability.
    \end{itemize}
    \begin{block}{Basic Structure of MongoDB}
        \begin{verbatim}
+------------------+
|     Database     |
| +--------------+ |
| |  Collection  | |  
| | +----------+ | | 
| | | Document | | | 
| | +----------+ | | 
| +--------------+ |
+------------------+
        \end{verbatim}
    \end{block}
\end{frame}

\begin{frame}[fragile]
    \frametitle{Use Cases of MongoDB}
    \begin{itemize}
        \item \textbf{Real-time Analytics}: Suited for quick data processing.
        \item \textbf{Content Management Systems}: Accommodates diverse data types.
        \item \textbf{IoT Applications}: Handles varying data from sensors.
        \item \textbf{Mobile Applications}: Fast and scalable database for growing user bases.
    \end{itemize}
\end{frame}

\begin{frame}[fragile]
    \frametitle{Example Code Snippet: Inserting a Document}
    \begin{lstlisting}[language=JavaScript]
db.users.insertOne({
    name: "John Doe",
    age: 30,
    interests: ["coding", "music", "sports"],
    address: {
        street: "123 Main St",
        city: "Anytown"
    }
});
    \end{lstlisting}
\end{frame}

\begin{frame}[fragile]
    \frametitle{Conclusion}
    MongoDB represents a shift towards flexible, scalable, and high-performance database design, making it ideal for modern applications. Its adaptability and powerful features justify selecting a NoSQL solution over traditional relational systems.
\end{frame}

\begin{frame}
    \frametitle{MongoDB Hands-on Project}
    \begin{block}{Real-world Scenario}
        Implementing a data model in MongoDB.
    \end{block}
\end{frame}

\begin{frame}[fragile]
    \frametitle{Introduction to MongoDB Data Modeling}
    \begin{itemize}
        \item Data modeling in MongoDB is crucial for optimizing data storage and retrieval.
        \item Uses a flexible, document-oriented data model, unlike traditional relational databases.
        \item Allows a more organized and efficient storage structure based on application needs.
    \end{itemize}
\end{frame}

\begin{frame}[fragile]
    \frametitle{Key Concepts in MongoDB}
    \begin{enumerate}
        \item \textbf{Document-Based Storage:}
            \begin{itemize}
                \item Data stored in documents (JSON-like format) grouped into collections.
                \item Each document can have a unique structure, providing schema flexibility.
            \end{itemize}
        \item \textbf{Collections:}
            \begin{itemize}
                \item Group of MongoDB documents, similar to a relational database table.
            \end{itemize}
        \item \textbf{Embedded vs. Referenced Data:}
            \begin{itemize}
                \item \textbf{Embedded Documents:} Store related data within a single document.
                \item \textbf{Referenced Documents:} Store related data in separate documents, linked by ObjectIDs.
            \end{itemize}
    \end{enumerate}
\end{frame}

\begin{frame}[fragile]
    \frametitle{Example Scenario: Building a Simple Blog Application}
    \begin{itemize}
        \item Developing a blog application to handle posts and comments.
        \item \textbf{Post Document:}
        \begin{lstlisting}[language=json]
        {
          "_id": "post_id_1",
          "title": "First Blog Post",
          "content": "This is the content of the first blog post.",
          "author": "author_id_1",
          "tags": ["mongodb", "nosql", "database"],
          "comments": [
            {
              "commentId": "comment_id_1",
              "text": "Great post!",
              "createdAt": "2023-10-20T14:48:00.000Z"
            }
          ],
          "createdAt": "2023-10-20T12:00:00.000Z"
        }
        \end{lstlisting}
        \item \textbf{Comment Document (if stored separately):}
        \begin{lstlisting}[language=json]
        {
          "_id": "comment_id_1",
          "postId": "post_id_1",
          "text": "Great post!",
          "author": "user_id_1",
          "createdAt": "2023-10-20T14:48:00.000Z"
        }
        \end{lstlisting}
    \end{itemize}
\end{frame}

\begin{frame}[fragile]
    \frametitle{Key Considerations}
    \begin{itemize}
        \item If comments are small and frequently accessed with the post, embedding is ideal.
        \item If comments grow significantly or require independent querying, separate collection is better.
    \end{itemize}
\end{frame}

\begin{frame}[fragile]
    \frametitle{Key Points to Emphasize}
    \begin{itemize}
        \item \textbf{Flexibility:} Schema-less design allows easy adaptation of data structures.
        \item \textbf{Performance:} Properly understanding data embedding vs. referencing can enhance query performance.
        \item \textbf{CRUD Operations:} Familiarize with MongoDB CRUD operations to effectively interact with data.
    \end{itemize}
\end{frame}

\begin{frame}[fragile]
    \frametitle{Code Snippet: Inserting a New Post}
    Here’s how to insert a new post into the blog collection:
    \begin{lstlisting}[language=javascript]
    db.posts.insertOne({
        title: "Second Blog Post",
        content: "Exploring data modeling with MongoDB.",
        author: "author_id_1",
        tags: ["mongodb", "data-modeling"],
        comments: [],
        createdAt: new Date()
    });
    \end{lstlisting}
\end{frame}

\begin{frame}
    \frametitle{Conclusion}
    By practicing these concepts through a hands-on project, you can reinforce your understanding of MongoDB data modeling in real-world applications, preparing you for developing robust database solutions.
\end{frame}

\begin{frame}[fragile]
    \frametitle{Understanding Cassandra - Introduction}
    \begin{block}{Introduction to Apache Cassandra}
        \begin{itemize}
            \item \textbf{Definition}: Apache Cassandra is a highly scalable, distributed NoSQL database designed to handle large amounts of data across many commodity servers, providing high availability with no single point of failure.
        \end{itemize}
    \end{block}
\end{frame}

\begin{frame}[fragile]
    \frametitle{Understanding Cassandra - Key Architectural Features}
    \begin{enumerate}
        \item \textbf{Distributed Architecture}:
            \begin{itemize}
                \item \textbf{Peer-to-Peer}: All nodes in the cluster are equal, allowing any node to handle requests, ensuring no downtime.
                \item \textbf{Data Replication}: Data is automatically replicated across multiple nodes for fault tolerance and availability.
            \end{itemize}
        
        \item \textbf{SSTable Files}:
            \begin{itemize}
                \item Data is stored in sorted string tables (SSTables), which are created from memory tables (memtables) flushed to disk.
            \end{itemize}
        
        \item \textbf{Partitioning}:
            \begin{itemize}
                \item Data is distributed using a partition key, hashed to determine its storage location for balanced loads.
            \end{itemize}
        
        \item \textbf{Tunable Consistency}:
            \begin{itemize}
                \item Developers can choose consistency levels (e.g., One, Quorum, All) for read/write operations.
            \end{itemize}
    \end{enumerate}
\end{frame}

\begin{frame}[fragile]
    \frametitle{Understanding Cassandra - Key Features and Use Cases}
    \begin{block}{Key Features}
        \begin{itemize}
            \item \textbf{Scalability}: Easily add new nodes without downtime; performance improves with more nodes.
            \item \textbf{High Availability}: Automatic data replication ensures data is available even if nodes fail.
            \item \textbf{Flexible Data Model}: Supports a variety of data types and structures, with variable column numbers.
            \item \textbf{CQL (Cassandra Query Language)}: Simplifies interactions similar to SQL.
        \end{itemize}
    \end{block}

    \begin{block}{Summary}
        \begin{itemize}
            \item Ideal for applications with scalable needs, high write speeds, and large data volume:
                \begin{itemize}
                    \item Social media applications
                    \item IoT data storage
                    \item Online transaction processing systems
                \end{itemize}
        \end{itemize}
    \end{block}
    
    \begin{lstlisting}[language=SQL]
    // Example of creating a table in Cassandra
    CREATE TABLE users (
        user_id UUID PRIMARY KEY,
        name TEXT,
        email TEXT,
        created_at TIMESTAMP
    );
    \end{lstlisting}
\end{frame}

\begin{frame}[fragile]
    \frametitle{Understanding Cassandra - Architecture Diagram}
    \begin{block}{Diagram Representation}
    \begin{verbatim}
           +---------+          +---------+
           | NODE 1  |          | NODE 2  |
           +---------+          +---------+
                  \                /
                     \          /
                      +--------+
                      |  Load  |
                      | Balancer|
                      +--------+
                         /      \
                    +---------+ +---------+
                    | NODE 3  | | NODE 4  |
                    +---------+ +---------+
    \end{verbatim}
    \end{block}
    
    \begin{block}{Conclusion}
        Understanding Cassandra's architecture and features equips developers to leverage its strengths effectively for modern applications requiring scalability and high availability.
    \end{block}
\end{frame}

\begin{frame}[fragile]
    \frametitle{Cassandra Hands-on Project}
    \begin{block}{Real-World Scenario}
        Setting up a Cassandra database and performing CRUD operations.
    \end{block}
    \begin{block}{Objective}
        By the end of this session, you will be able to set up a Cassandra database and understand how to perform Create, Read, Update, and Delete (CRUD) operations effectively.
    \end{block}
\end{frame}

\begin{frame}[fragile]
    \frametitle{Setting up the Cassandra Database}
    \begin{enumerate}
        \item \textbf{Download Cassandra:}
        \begin{itemize}
            \item Visit the Apache Cassandra \texttt{[official download page]}(http://cassandra.apache.org/download/) and choose the appropriate version for your operating system.
        \end{itemize}
        
        \item \textbf{Install Dependencies:}
        \begin{itemize}
            \item Ensure you have Java (JDK 8 or 11) installed.
            \item Verify the installation with:
            \begin{lstlisting}[language=bash]
            java -version
            \end{lstlisting}
        \end{itemize}
        
        \item \textbf{Run Cassandra:}
        \begin{itemize}
            \item Extract the downloaded package.
            \item Navigate to the Cassandra directory and start it:
            \begin{lstlisting}[language=bash]
            bin/cassandra -f
            \end{lstlisting}
        \end{itemize}
        
        \item \textbf{Cassandra Shell (CQLSH):}
        \begin{itemize}
            \item Open another terminal and run:
            \begin{lstlisting}[language=bash]
            bin/cqlsh
            \end{lstlisting}
        \end{itemize}
    \end{enumerate}
\end{frame}

\begin{frame}[fragile]
    \frametitle{Performing CRUD Operations}
    \begin{block}{CRUD Overview}
        Learn how to perform basic operations on the school.students table:
    \end{block}
    
    \begin{enumerate}
        \item \textbf{Create:}
        \begin{lstlisting}[language=sql]
        INSERT INTO school.students (id, name, age, major) 
        VALUES (uuid(), 'Alice', 22, 'Computer Science');
        \end{lstlisting}
        
        \item \textbf{Read:}
        \begin{lstlisting}[language=sql]
        SELECT * FROM school.students WHERE name = 'Alice';
        \end{lstlisting}
        
        \item \textbf{Update:}
        \begin{lstlisting}[language=sql]
        UPDATE school.students SET age = 23 WHERE name = 'Alice';
        \end{lstlisting}
        
        \item \textbf{Delete:}
        \begin{lstlisting}[language=sql]
        DELETE FROM school.students WHERE name = 'Alice';
        \end{lstlisting}
    \end{enumerate}
\end{frame}

\begin{frame}
    \frametitle{Querying NoSQL Databases}
    \begin{block}{Introduction}
        In this section, we explore how to query NoSQL databases, focusing on MongoDB and Cassandra.
        Understanding effective querying is crucial for data retrieval and manipulation.
    \end{block}
    
    \begin{itemize}
        \item \textbf{NoSQL Databases:} Non-tabular storage with flexible data models and horizontal scaling.
        \item \textbf{Querying:} The process of retrieving data based on certain criteria.
    \end{itemize}
\end{frame}

\begin{frame}[fragile]
    \frametitle{Querying MongoDB}
    \begin{block}{Overview}
        MongoDB is a document-oriented NoSQL database that stores data in BSON format (JSON-like documents), which supports rich data structures.
    \end{block}
    
    \begin{block}{Basic Syntax}
        The primary method for querying in MongoDB is the \texttt{.find()} operation.
    \end{block}
    
    \begin{lstlisting}[language=JavaScript]
db.collectionName.find({ "field": "value" })
    \end{lstlisting}

    \begin{block}{Example Explained}
        \begin{itemize}
            \item \texttt{collectionName}: The name of the MongoDB collection.
            \item \texttt{field}: The document's attribute.
            \item \texttt{value}: The value you want to search for.
        \end{itemize}
    \end{block}
    
    \begin{block}{Example Query}
        \begin{lstlisting}[language=JavaScript]
db.users.find({ "age": { "$gt": 18 } })
        \end{lstlisting}
        This query retrieves all users older than 18.
    \end{block}
\end{frame}

\begin{frame}[fragile]
    \frametitle{Querying Cassandra}
    \begin{block}{Overview}
        Cassandra is a wide-column store NoSQL database designed for high availability and scalability, with a flexible schema.
    \end{block}
    
    \begin{block}{Basic Syntax}
        The primary method for querying data in Cassandra is through the \texttt{SELECT} statement.
    \end{block}
    
    \begin{lstlisting}[language=SQL]
SELECT * FROM keyspaceName.tableName WHERE condition;
    \end{lstlisting}

    \begin{block}{Example Explained}
        \begin{itemize}
            \item \texttt{keyspaceName}: Namespace containing tables.
            \item \texttt{tableName}: The table from which to retrieve data.
            \item \texttt{condition}: Criteria for selecting records.
        \end{itemize}
    \end{block}
    
    \begin{block}{Example Query}
        \begin{lstlisting}[language=SQL]
SELECT * FROM users WHERE age > 20;
        \end{lstlisting}
        This retrieves all user records where the age is greater than 20.
    \end{block}
\end{frame}

\begin{frame}
    \frametitle{Summary and Next Steps}
    \begin{block}{Summary}
        \begin{itemize}
            \item \textbf{MongoDB:} Supports rich querying with a flexible document structure.
            \item \textbf{Cassandra:} Offers powerful querying capabilities focused on performance and scalability using CQL.
        \end{itemize}
    \end{block}
    
    \begin{block}{Next Steps}
        In the upcoming slide, we will discuss how these querying methodologies impact 
        \textit{Data Scalability and Performance} in NoSQL databases.
    \end{block}

    \begin{block}{Conclusion}
        Mastering queries in MongoDB and Cassandra enhances your ability to effectively retrieve and manipulate data in various applications.
        Understanding these foundational concepts sets the stage for advanced database operations and optimization strategies.
    \end{block}
    
    \begin{alertblock}{Reminder}
        Practice querying both databases with different data sets to solidify these concepts!
    \end{alertblock}
\end{frame}

\begin{frame}[fragile]
  \frametitle{Data Scalability and Performance - Introduction}
  \begin{block}{Overview}
    Data scalability refers to the ability of a database to handle increasing volumes of data and user load without sacrificing performance.
    NoSQL databases like MongoDB and Cassandra are designed to handle large datasets across distributed systems.
  \end{block}
\end{frame}

\begin{frame}[fragile]
  \frametitle{Data Scalability - Key Concepts}
  \begin{enumerate}
    \item \textbf{Horizontal Scaling}:
      \begin{itemize}
        \item Adding more machines or servers to distribute the load.
        \item Example: Handling 10,000 users per server; adding 2 servers allows handling 30,000 users.
      \end{itemize}
      
    \item \textbf{Vertical Scaling}:
      \begin{itemize}
        \item Increasing capacities of existing machines (RAM, CPUs).
        \item Example: Upgrading server RAM from 16GB to 64GB for data-heavy applications.
      \end{itemize}
  \end{enumerate}
\end{frame}

\begin{frame}[fragile]
  \frametitle{Performance Optimization Strategies}
  To optimize performance in NoSQL systems, consider the following:
  \begin{enumerate}
    \item \textbf{Data Modeling}:
      \begin{itemize}
        \item Structure data for access patterns; use embedded documents in MongoDB.
        \item Example: Embed comments within the blog post document.
      \end{itemize}
      
    \item \textbf{Indexing}:
      \begin{itemize}
        \item Create indexes on frequently queried fields.
        \item Example:
        \begin{lstlisting}[language=javascript]
db.collection.createIndex({ "fieldName": 1 })
        \end{lstlisting}
      \end{itemize}
      
    \item \textbf{Caching}:
      \begin{itemize}
        \item Use Redis or Memcached for frequently accessed data to reduce read times.
      \end{itemize}
      
    \item \textbf{Sharding}:
      \begin{itemize}
        \item Distribute data across nodes to manage larger datasets.
        \item Example: Horizontally partitioning data based on user ID ranges.
      \end{itemize}
  \end{enumerate}
\end{frame}

\begin{frame}[fragile]
  \frametitle{Load Balancing and Conclusion}
  \begin{itemize}
    \item \textbf{Load Balancing}:
      \begin{itemize}
        \item Distribute incoming traffic evenly to prevent server bottlenecks.
      \end{itemize}
      
    \item \textbf{Key Takeaways}:
      \begin{itemize}
        \item Understand scalability needs of applications.
        \item Tailor optimization strategies based on specific use cases.
        \item Regularly monitor performance and adjust strategies as needed.
      \end{itemize}
      
      By leveraging NoSQL capabilities, applications can scale and perform effectively in a growing landscape.
  \end{itemize}
\end{frame}

\begin{frame}[fragile]
  \frametitle{Case Study: NoSQL in Practice}
  \begin{block}{Introduction to NoSQL Case Studies}
    NoSQL databases transform how businesses manage and utilize data, especially when scalability, speed, and flexibility are crucial. This slide examines real-world examples showcasing how organizations leverage NoSQL technologies such as MongoDB and Cassandra to solve specific business challenges.
  \end{block}
\end{frame}

\begin{frame}[fragile]
  \frametitle{Case Study 1: MongoDB at Lyft}
  \begin{itemize}
    \item \textbf{Challenge}: 
      \begin{itemize}
        \item Managing exponential growth of ride-sharing data (user info, ride details, location data).
      \end{itemize}
    \item \textbf{Solution}: 
      \begin{itemize}
        \item Implemented MongoDB for its document-oriented structure, enabling rapid querying and complex data model storage.
      \end{itemize}
    \item \textbf{Benefits}:
      \begin{itemize}
        \item \textbf{Scalability}: Handles millions of rides per day using MongoDB's sharding capabilities.
        \item \textbf{Flexible Schema}: Adapts database schema with minimal downtime.
      \end{itemize}
  \end{itemize}
\end{frame}

\begin{frame}[fragile]
  \frametitle{Case Study 2: Cassandra at Netflix}
  \begin{itemize}
    \item \textbf{Challenge}: 
      \begin{itemize}
        \item Issues related to high availability and performance during peak usage times with millions of users.
      \end{itemize}
    \item \textbf{Solution}: 
      \begin{itemize}
        \item Adopted Apache Cassandra to manage huge structured data with a distributed architecture.
      \end{itemize}
    \item \textbf{Benefits}:
      \begin{itemize}
        \item \textbf{High Availability}: No service interruptions, even during system updates.
        \item \textbf{Real-time Data Access}: Personalized recommendations and insights provided in real-time.
      \end{itemize}
  \end{itemize}
\end{frame}

\begin{frame}[fragile]
  \frametitle{Key Takeaways and Conclusion}
  \begin{itemize}
    \item \textbf{Flexibility and Scalability}: Horizontal scaling accommodates growth without sacrificing performance.
    \item \textbf{Adaptability}: Schema-less structures promote rapid development and deployment.
    \item \textbf{High Availability}: Distributed systems ensure uninterrupted service, maintaining customer trust.
  \end{itemize}
  
  \begin{block}{Conclusion}
    These case studies highlight the diverse applications of NoSQL databases, emphasizing informed integration into data strategies.
  \end{block}
\end{frame}

\begin{frame}
    \frametitle{Integration with Cloud Technologies}
    \begin{block}{Introduction}
        Cloud Computing platforms provide scalable, flexible, and cost-effective environments for deploying NoSQL databases such as MongoDB and Cassandra. This integration allows businesses to leverage the capabilities of NoSQL while enjoying the benefits of cloud infrastructure.
    \end{block}
\end{frame}

\begin{frame}
    \frametitle{Key Enhancements Offered by Cloud Technologies}
    \begin{enumerate}
        \item \textbf{Scalability}: Resources can be easily adjusted to handle varying workloads.
            \begin{itemize}
                \item \textit{Example}: A retail application during the holiday season can automatically scale its database resources to handle a spike in online transactions.
            \end{itemize}
        \item \textbf{Flexibility}: Deployment models (IaaS, PaaS, SaaS) offer various levels of control for NoSQL databases.
            \begin{itemize}
                \item \textit{Illustration}:
                \begin{itemize}
                    \item \textbf{IaaS}: Manage the database on virtual machines.
                    \item \textbf{PaaS}: Services like MongoDB Atlas provide managed databases.
                \end{itemize}
            \end{itemize}
        \item \textbf{Cost Efficiency}: Users pay for what they use, reducing the need for upfront investments in hardware.
            \begin{itemize}
                \item \textit{Example}: Startups can begin with low-cost cloud-based NoSQL options.
            \end{itemize}
    \end{enumerate}
\end{frame}

\begin{frame}
    \frametitle{Key Enhancements Continued}
    \begin{enumerate}
        \setcounter{enumi}{3} % Continue numbering from the previous frame
        \item \textbf{Automatic Backups and Disaster Recovery}: Enhances data safety with built-in solutions.
            \begin{itemize}
                \item \textit{Illustration}: Managed cloud services can schedule regular backups of a MongoDB database.
            \end{itemize}
        \item \textbf{Performance Optimization}: Features such as caching and load balancing enhance database performance.
            \begin{itemize}
                \item \textit{Example}: Using Amazon DynamoDB with automatic partitioning for high-traffic queries efficiently.
            \end{itemize}
    \end{enumerate}
\end{frame}

\begin{frame}
    \frametitle{Important Considerations}
    \begin{itemize}
        \item \textbf{Vendor Lock-In}: Evaluate risks associated with dependency on a specific vendor.
        \item \textbf{Data Security}: Understand security protocols and implement encryption and access controls to safeguard sensitive data.
    \end{itemize}
\end{frame}

\begin{frame}
    \frametitle{Conclusion}
    Integrating NoSQL databases with cloud technologies fundamentally transforms how businesses manage their data. It enables scalable, flexible, and cost-effective solutions tailored to changing needs while ensuring high performance and reliability.
\end{frame}

\begin{frame}[fragile]
    \frametitle{Code Snippet - MongoDB Atlas Connection Example}
    \begin{lstlisting}[language=JavaScript]
const { MongoClient } = require('mongodb');

async function connectToDatabase() {
    const uri = "<Your MongoDB Atlas Connection String>";
    const client = new MongoClient(uri, { useNewUrlParser: true, useUnifiedTopology: true });
    try {
        await client.connect();
        console.log("Connected successfully to MongoDB Atlas!");
    } finally {
        await client.close();
    }
}

connectToDatabase();
    \end{lstlisting}
\end{frame}

\begin{frame}[fragile]
    \frametitle{Challenges of NoSQL Implementation}
    \begin{block}{Introduction}
        Adopting NoSQL technologies offers advantages in scalability, performance, and flexibility.
        However, several challenges may impede successful implementation. 
        Understanding these challenges can help mitigate risks and ensure a smoother transition.
    \end{block}
\end{frame}

\begin{frame}[fragile]
    \frametitle{Key Challenges - Part 1}
    \begin{enumerate}
        \item \textbf{Data Modeling Complexity}
        \begin{itemize}
            \item NoSQL databases require a different approach to data modeling.
            \item Flexible schema complicates relationships between data entities.
            \item \textit{Example:} In MongoDB, related data may be stored together or separately, leading to redundancy or complexity in querying.
        \end{itemize}
        
        \item \textbf{Consistency and Transaction Management}
        \begin{itemize}
            \item NoSQL databases may prioritize availability, leading to eventual consistency.
            \item \textit{Example:} Cassandra may show immediate write success, but data may not be consistent across nodes.
            \item Developers need to assess trade-offs between consistency and performance.
        \end{itemize}
    \end{enumerate}
\end{frame}

\begin{frame}[fragile]
    \frametitle{Key Challenges - Part 2}
    \begin{enumerate}
        \setcounter{enumi}{2} % Continue numbering from the last frame
        \item \textbf{Skill Gap and Knowledge Base}
        \begin{itemize}
            \item Transitioning to NoSQL requires understanding of distributed systems.
            \item Teams familiar with relational databases may struggle with NoSQL concepts.
            \item \textit{Example:} Sharding and CAP theorem understanding may require training and workshops.
        \end{itemize}
        
        \item \textbf{Integration with Existing Systems}
        \begin{itemize}
            \item Integrating NoSQL with legacy systems can be complex due to differing data models.
            \item \textit{Example:} Migration from relational to NoSQL systems needs careful planning for data integrity.
        \end{itemize}
        
        \item \textbf{Monitoring and Maintenance}
        \begin{itemize}
            \item Mature relational databases have established monitoring tools, while NoSQL may need custom solutions.
            \item Regular maintenance requires additional resources for performance tracking.
        \end{itemize}
    \end{enumerate}
\end{frame}

\begin{frame}[fragile]
    \frametitle{Key Challenges - Part 3}
    \begin{enumerate}
        \setcounter{enumi}{5} % Continue numbering from the last frame
        \item \textbf{Vendor Lock-in}
        \begin{itemize}
            \item Some NoSQL solutions are proprietary, leading to vendor lock-in.
            \item Always consider data portability and API requirements before committing to a vendor.
        \end{itemize}
    \end{enumerate}
    
    \begin{block}{Conclusion}
        While NoSQL presents opportunities, organizations must consider these challenges to successfully leverage benefits. 
        Careful planning, training, and clear understanding can guide effective NoSQL adoption.
    \end{block}
    
    \begin{block}{Additional Tips}
        \begin{itemize}
            \item Foster a growth mindset within the team.
            \item Start with pilot projects to identify potential issues early on.
        \end{itemize}
    \end{block}
\end{frame}

\begin{frame}[fragile]
    \frametitle{Future Trends in NoSQL}
    \begin{block}{Introduction to Future Trends}
        As the landscape of data management continues to evolve, NoSQL databases are being increasingly adopted in various sectors due to their flexibility, scalability, and performance. Understanding future trends can help organizations make informed decisions on data architecture.
    \end{block}
\end{frame}

\begin{frame}[fragile]
    \frametitle{Key Trends in NoSQL - Part 1}
    \begin{enumerate}
        \item \textbf{Hybrid Database Models}
        \begin{itemize}
            \item \textbf{Description}: The convergence of SQL and NoSQL capabilities into hybrid models allows developers to leverage the strengths of both paradigms.
            \item \textbf{Example}: A cloud-native database that combines relational data management for structured queries with document storage for unstructured data.
        \end{itemize}
        
        \item \textbf{Multi-Model Databases}
        \begin{itemize}
            \item \textbf{Description}: Increasing popularity of databases that support multiple data models (e.g., document, graph, key-value) within a single platform to serve diverse application needs.
            \item \textbf{Example}: Databases like ArangoDB or OrientDB that allow users to combine the power of different data structures.
        \end{itemize}
    \end{enumerate}
\end{frame}

\begin{frame}[fragile]
    \frametitle{Key Trends in NoSQL - Part 2}
    \begin{enumerate}
        \setcounter{enumi}{2} % Continue numbering from previous frame
        \item \textbf{Serverless Architectures}
        \begin{itemize}
            \item \textbf{Description}: Serverless computing will enable auto-scaling, high availability, and reduced management overhead for NoSQL databases.
            \item \textbf{Example}: AWS DynamoDB operates on a serverless model, allowing businesses to pay for what they use without needing to manage infrastructure.
        \end{itemize}
        
        \item \textbf{Increased Adoption of Graph Databases}
        \begin{itemize}
            \item \textbf{Description}: With the rise of connected data, graph databases will become more prominent, particularly in social networks and recommendation systems.
            \item \textbf{Example}: Neo4j is widely used for analyzing relationships and connections among large datasets.
        \end{itemize}
        
        \item \textbf{AI and Machine Learning Integration}
        \begin{itemize}
            \item \textbf{Description}: NoSQL databases will increasingly incorporate AI/ML capabilities to enhance data analytics and predictive insights.
            \item \textbf{Example}: MongoDB's integration with TensorFlow for machine learning applications allows data scientists to use real-time data for model training.
        \end{itemize}
    \end{enumerate}
\end{frame}

\begin{frame}[fragile]
    \frametitle{Key Trends in NoSQL - Part 3}
    \begin{enumerate}
        \setcounter{enumi}{5} % Continue numbering from previous frame
        \item \textbf{Focus on Data Privacy and Security}
        \begin{itemize}
            \item \textbf{Description}: As regulations around data privacy strengthen (like GDPR), NoSQL databases will evolve to incorporate robust security measures.
            \item \textbf{Example}: Implementing field-level encryption in databases like Couchbase to protect sensitive information while stored.
        \end{itemize}
    \end{enumerate}
    
    \begin{block}{Summary of Key Points}
        \begin{itemize}
            \item Hybrid and multi-model databases are shaping how data is structured.
            \item Serverless architectures are streamlining operations and cost efficiency.
            \item Graph databases are key for understanding complex relationships in data.
            \item AI/ML integration is essential for driving advanced analytics.
            \item Data privacy remains a top priority, influencing database design.
        \end{itemize}
    \end{block}
\end{frame}

\begin{frame}[fragile]
    \frametitle{Conclusion and Call to Action}
    \begin{block}{Conclusion}
        Staying ahead of these trends helps organizations effectively manage their data and leverage new opportunities in a rapidly changing digital landscape.
    \end{block}
    
    \begin{block}{Call to Action}
        Encouraged to explore these trends further and consider how they can be applied to enhance projects and solutions within your organization. Incorporate these insights to prepare for discussions on existing projects and potential future implementations of NoSQL technologies.
    \end{block}
\end{frame}

\begin{frame}[fragile]
    \frametitle{Collaborative Project Overview - Objectives}
    % Objectives of Collaborative Projects with NoSQL
    As part of your learning experience, engage in projects using two NoSQL databases: 
    \textbf{MongoDB} and \textbf{Cassandra}. The aim is to apply theoretical concepts to real-world scenarios, fostering teamwork and practical skills in data management.
\end{frame}

\begin{frame}[fragile]
    \frametitle{Collaborative Project Overview - Guidelines}
    % Project Guidelines
    \begin{enumerate}
        \item \textbf{Team Composition:}
        \begin{itemize}
            \item Teams of 3-5 members for effective collaboration.
            \item Roles based on strengths: database architect, data analyst, programmer, project manager.
        \end{itemize}
        
        \item \textbf{Choosing a Project Idea:}
        \begin{itemize}
            \item Select problem domains beneficial for NoSQL, e.g., e-commerce, healthcare.
            \item Example: Platform for managing users and product catalogs.
        \end{itemize}
        
        \item \textbf{Technology Stack:}
        \begin{itemize}
            \item \textbf{MongoDB:} Flexible schemas, document stores, rich queries.
            \item \textbf{Cassandra:} High availability, scalability, and write-heavy workloads.
        \end{itemize}
    \end{enumerate}
\end{frame}

\begin{frame}[fragile]
    \frametitle{Collaborative Project Overview - Development Phases}
    % Key Phases of Project Development
    \begin{enumerate}
        \item \textbf{Research:} Understand project requirements and NoSQL support.
        
        \item \textbf{Design:} Create a data model. \\
        Example MongoDB user collection:
        \begin{lstlisting}
        {
          "userId": "123",
          "name": "Alice",
          "email": "alice@example.com",
          "purchases": [
            {"productId": "p1", "date": "2023-01-15"},
            {"productId": "p2", "date": "2023-01-20"}
          ]
        }
        \end{lstlisting}
        
        \item \textbf{Implementation:} Set up the database and interact with the data model.

        \item \textbf{Testing:} Assess data integrity and performance, e.g.,
        \begin{lstlisting}
        db.users.find({"name": "Alice"}) // retrieves Alice's data
        \end{lstlisting}
    \end{enumerate}
\end{frame}

\begin{frame}
    \frametitle{Assessment Methods}
    \begin{block}{Overview of Evaluation Criteria for Hands-On Projects}
        As you embark on your hands-on projects using MongoDB and Cassandra, 
        it is essential to understand how you will be evaluated. This assessment 
        not only tests your technical skills but also your ability to collaborate 
        effectively and apply theoretical knowledge in practical scenarios.
    \end{block}
\end{frame}

\begin{frame}
    \frametitle{Evaluation Categories}
    \begin{enumerate}
        \item \textbf{Project Design (30\%)}
            \begin{itemize}
                \item \textbf{Criteria}: Clarity of the project's objectives, user requirements, and overall architecture design.
                \item \textbf{Example}: A well-documented schema design for MongoDB or table design for Cassandra that addresses scalability and query efficiency.
            \end{itemize}
        
        \item \textbf{Implementation (40\%)}
            \begin{itemize}
                \item \textbf{Criteria}: Quality and functionality of your code and data management practices.
                \item \textbf{Key Point}: Code should follow best practices such as proper indexing, error handling, and the use of appropriate data models.
            \end{itemize}
    \end{enumerate}
\end{frame}

\begin{frame}[fragile]
    \frametitle{Code Snippet Example}
    \begin{lstlisting}[language=JavaScript]
    // MongoDB: Inserting data into a collection
    db.students.insertOne({
        name: "Alice",
        age: 24,
        major: "Computer Science"
    });
    \end{lstlisting}
\end{frame}

\begin{frame}
    \frametitle{Collaboration and Presentation}
    \begin{enumerate}
        \setcounter{enumi}{2}
        \item \textbf{Collaboration and Teamwork (20\%)}
            \begin{itemize}
                \item \textbf{Criteria}: Effective communication, role distribution, and contribution to the team's overall progress.
                \item \textbf{Example}: Regular use of project management tools to track tasks and updates (e.g., Trello or Jira).
            \end{itemize}
        
        \item \textbf{Presentation and Documentation (10\%)}
            \begin{itemize}
                \item \textbf{Criteria}: Clarity and professionalism in your final presentation, including visual aids and technical documentation.
                \item \textbf{Key Point}: Use visuals to convey your project’s workflow, and keep documentation concise yet comprehensive.
            \end{itemize}
    \end{enumerate}
\end{frame}

\begin{frame}
    \frametitle{Additional Considerations}
    \begin{itemize}
        \item \textbf{Peer Reviews}: During the assessment phase, you may also participate in peer reviews, providing feedback on your teammate's contributions.
        \item \textbf{Iterative Feedback}: Throughout the project, you will receive formative feedback. Utilize this to make necessary adjustments before final submission.
    \end{itemize}
\end{frame}

\begin{frame}
    \frametitle{Key Points to Remember}
    \begin{itemize}
        \item \textbf{Understand the Criteria}: Familiarize yourself with each evaluation aspect and focus on improving areas where you may be weaker.
        \item \textbf{Communicate}: Keep an ongoing dialog within your team to ensure everyone is on the same page and contributing effectively.
        \item \textbf{Embrace Feedback}: Use the input you receive to refine both your project and your learning process.
    \end{itemize}
    By keeping these assessment methods in mind, you’ll enhance your technical skills and gain valuable experience in a team setting. Good luck with your projects!
\end{frame}

\begin{frame}[fragile]
  \frametitle{Conclusion - Recap of Lessons Learned}
  In this section, we summarize key learnings and insights gained from practical experiences with two popular NoSQL databases: 
  \textbf{MongoDB} and \textbf{Cassandra}.
\end{frame}

\begin{frame}[fragile]
  \frametitle{Key Insights from MongoDB}
  \begin{enumerate}
    \item \textbf{Understanding NoSQL Databases}:
      \begin{itemize}
        \item NoSQL databases use various data models (document-oriented, key-value, etc.) for flexibility and scalability.
      \end{itemize}
      
    \item \textbf{Hands-on with MongoDB}:
      \begin{itemize}
        \item \textit{Document Data Model}: MongoDB stores data as JSON-like documents (BSON format).
        \item CRUD operations allow creating, reading, updating, and deleting data.
        
        \begin{block}{Example}
        \begin{lstlisting}[language=json]
        {
          "name": "John Doe",
          "email": "john@example.com",
          "age": 30,
          "interests": ["music", "sports", "travel"]
        }
        \end{lstlisting}
        \end{block}
        
        \item \textit{Powerful Querying}: Rich query syntax and indexing enable fast data retrieval.
      \end{itemize}
  \end{enumerate}
\end{frame}

\begin{frame}[fragile]
  \frametitle{Key Insights from Cassandra}
  \begin{enumerate}
    \item \textbf{Hands-on with Cassandra}:
      \begin{itemize}
        \item \textit{Column-Family Data Model}: Optimizes large-scale data writes.
        \begin{block}{Example}
        \begin{lstlisting}[language=sql]
        CREATE TABLE users (
          username TEXT PRIMARY KEY,
          email TEXT,
          age INT,
          interests LIST<TEXT>
        );
        \end{lstlisting}
        \end{block}
        
        \item \textit{Scalability and Partitioning}: Excels in horizontal scaling and handling massive datasets.
        
        \item \textit{Eventual Consistency Model}: Trade-offs of consistency levels optimize for high availability.
      \end{itemize}
  \end{enumerate}

  \begin{block}{Key Points to Emphasize}
    - Choose the right database based on application needs.
    - Design an appropriate data structure for performance.
    - Engage in hands-on experience for informed decision-making.
  \end{block}
\end{frame}

\begin{frame}[fragile]
    \frametitle{Q\&A Session - Introduction}
    Welcome to the Q\&A Session! This is an opportunity for you to clarify any doubts or misunderstandings regarding:
    \begin{itemize}
        \item NoSQL databases, specifically MongoDB and Cassandra
        \item Our hands-on projects
    \end{itemize}
\end{frame}

\begin{frame}[fragile]
    \frametitle{Q\&A Session - Key Concepts}
    \begin{block}{Key Concepts to Address}
        \begin{itemize}
            \item \textbf{NoSQL Overview}
                \begin{itemize}
                    \item Differences from relational databases: schema flexibility, scalability.
                    \item Types of NoSQL databases: key-value, document, column-family, graph.
                \end{itemize}
            \item \textbf{MongoDB}
                \begin{itemize}
                    \item Data Model: documents and collections.
                    \item Query Language: key operations like \texttt{find()}, \texttt{insert()}, \texttt{update()}.
                    \item Use Cases: effective scenarios like content management and real-time analytics.
                \end{itemize}
            \item \textbf{Cassandra}
                \begin{itemize}
                    \item Data Model: tables, rows, columns, and partitioning.
                    \item Query Language: CQL (Cassandra Query Language) with commands like \texttt{SELECT}, \texttt{INSERT}.
                    \item Use Cases: applications needing high availability and scalability.
                \end{itemize}
        \end{itemize}
    \end{block}
\end{frame}

\begin{frame}[fragile]
    \frametitle{Q\&A Session - Discussion Points}
    \begin{block}{Example Questions to Encourage Discussion}
        \begin{itemize}
            \item What are the main differences in data modeling between MongoDB and Cassandra?
            \item How do we ensure data consistency in NoSQL databases?
            \item Can you provide a specific use case where MongoDB is more advantageous than Cassandra, or vice versa?
            \item What was the most challenging aspect of our hands-on projects with NoSQL databases?
            \item How can we optimize query performance in MongoDB and Cassandra?
        \end{itemize}
    \end{block}

    \begin{block}{Important Points to Remember}
        \begin{itemize}
            \item Schema design is crucial for performance and scalability.
            \item Understand strategies like replication and sharding for high availability.
            \item Familiarize yourself with indexing strategies for performance optimization.
            \item Access resources such as documentation, forums, and community support.
        \end{itemize}
    \end{block}
\end{frame}

\begin{frame}[fragile]
    \frametitle{Q\&A Session - Conclusion}
    Feel free to ask questions, share insights, or provide feedback based on your hands-on experience. Your contributions can deepen our collective understanding and enhance our practical knowledge of NoSQL databases!
\end{frame}


\end{document}