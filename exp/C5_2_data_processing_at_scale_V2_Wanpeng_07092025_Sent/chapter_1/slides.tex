\documentclass[aspectratio=169]{beamer}

% Theme and Color Setup
\usetheme{Madrid}
\usecolortheme{whale}
\useinnertheme{rectangles}
\useoutertheme{miniframes}

% Additional Packages
\usepackage[utf8]{inputenc}
\usepackage[T1]{fontenc}
\usepackage{graphicx}
\usepackage{booktabs}
\usepackage{listings}
\usepackage{amsmath}
\usepackage{amssymb}
\usepackage{xcolor}
\usepackage{tikz}
\usepackage{pgfplots}
\pgfplotsset{compat=1.18}
\usetikzlibrary{positioning}
\usepackage{hyperref}

% Custom Colors
\definecolor{myblue}{RGB}{31, 73, 125}
\definecolor{mygray}{RGB}{100, 100, 100}
\definecolor{mygreen}{RGB}{0, 128, 0}
\definecolor{myorange}{RGB}{230, 126, 34}
\definecolor{mycodebackground}{RGB}{245, 245, 245}

% Set Theme Colors
\setbeamercolor{structure}{fg=myblue}
\setbeamercolor{frametitle}{fg=white, bg=myblue}
\setbeamercolor{title}{fg=myblue}
\setbeamercolor{section in toc}{fg=myblue}
\setbeamercolor{item projected}{fg=white, bg=myblue}
\setbeamercolor{block title}{bg=myblue!20, fg=myblue}
\setbeamercolor{block body}{bg=myblue!10}
\setbeamercolor{alerted text}{fg=myorange}

% Set Fonts
\setbeamerfont{title}{size=\Large, series=\bfseries}
\setbeamerfont{frametitle}{size=\large, series=\bfseries}
\setbeamerfont{caption}{size=\small}
\setbeamerfont{footnote}{size=\tiny}

% Code Listing Style
\lstdefinestyle{customcode}{
  backgroundcolor=\color{mycodebackground},
  basicstyle=\footnotesize\ttfamily,
  breakatwhitespace=false,
  breaklines=true,
  commentstyle=\color{mygreen}\itshape,
  keywordstyle=\color{blue}\bfseries,
  stringstyle=\color{myorange},
  numbers=left,
  numbersep=8pt,
  numberstyle=\tiny\color{mygray},
  frame=single,
  framesep=5pt,
  rulecolor=\color{mygray},
  showspaces=false,
  showstringspaces=false,
  showtabs=false,
  tabsize=2,
  captionpos=b
}
\lstset{style=customcode}

% Custom Commands
\newcommand{\hilight}[1]{\colorbox{myorange!30}{#1}}
\newcommand{\source}[1]{\vspace{0.2cm}\hfill{\tiny\textcolor{mygray}{Source: #1}}}
\newcommand{\concept}[1]{\textcolor{myblue}{\textbf{#1}}}
\newcommand{\separator}{\begin{center}\rule{0.5\linewidth}{0.5pt}\end{center}}

% Footer and Navigation Setup
\setbeamertemplate{footline}{
  \leavevmode%
  \hbox{%
  \begin{beamercolorbox}[wd=.3\paperwidth,ht=2.25ex,dp=1ex,center]{author in head/foot}%
    \usebeamerfont{author in head/foot}\insertshortauthor
  \end{beamercolorbox}%
  \begin{beamercolorbox}[wd=.5\paperwidth,ht=2.25ex,dp=1ex,center]{title in head/foot}%
    \usebeamerfont{title in head/foot}\insertshorttitle
  \end{beamercolorbox}%
  \begin{beamercolorbox}[wd=.2\paperwidth,ht=2.25ex,dp=1ex,center]{date in head/foot}%
    \usebeamerfont{date in head/foot}
    \insertframenumber{} / \inserttotalframenumber
  \end{beamercolorbox}}%
  \vskip0pt%
}

% Turn off navigation symbols
\setbeamertemplate{navigation symbols}{}

% Title Page Information
\title[Introduction to Data Models]{Chapter 1: Introduction to Data Models}
\author[J. Smith]{John Smith, Ph.D.}
\institute[University Name]{
  Department of Computer Science\\
  University Name\\
  \vspace{0.3cm}
  Email: email@university.edu\\
  Website: www.university.edu
}
\date{\today}

% Document Start
\begin{document}

\frame{\titlepage}

\begin{frame}[fragile]
    \frametitle{Introduction to Data Models}
    \begin{block}{Overview of Data Models}
        Data models are foundational frameworks that define how data is structured, stored, and manipulated within a database or software application. They serve as blueprints for organizing data in ways that enhance accessibility, manageability, and utility across various applications.
    \end{block}
\end{frame}

\begin{frame}[fragile]
    \frametitle{Importance of Data Models in Modern Applications}
    \begin{enumerate}
        \item \textbf{Organization}
            \begin{itemize}
                \item Systematically organizes data for easier retrieval and manipulation.
                \item \textit{Example:} In a library database, entities like 'Books', 'Authors', and 'Members' are clearly defined.
            \end{itemize}
        
        \item \textbf{Communication}
            \begin{itemize}
                \item Provides a common language for stakeholders to discuss and understand data.
                \item \textit{Illustration:} Visual data models (like Entity-Relationship Diagrams) simplify complex concepts.
            \end{itemize}
        
        \item \textbf{Efficiency}
            \begin{itemize}
                \item Enables efficient querying and report generation, improving application performance.
                \item \textit{Key Point:} Well-designed models lead to faster access and reduced data redundancy.
            \end{itemize}
    \end{enumerate}
\end{frame}

\begin{frame}[fragile]
    \frametitle{Continuing the Importance of Data Models}
    \begin{enumerate}[resume]
        \item \textbf{Scalability}
            \begin{itemize}
                \item Facilitates integration of new data sources without disruption.
                \item \textit{Example:} A retail company expanding from sales to a comprehensive model including inventory and analytics.
            \end{itemize}

        \item \textbf{Data Integrity}
            \begin{itemize}
                \item Enforces rules that ensure data accuracy and consistency.
                \item \textit{Key Point:} Primary keys and validation rules are crucial for reliable data maintenance.
            \end{itemize}

        \item \textbf{Supports Different Use Cases}
            \begin{itemize}
                \item Tailors to various applications like transactional systems or operational dashboards.
                \item \textit{Key Point:} Different models (relational, NoSQL, object-oriented) align with specific needs.
            \end{itemize}
    \end{enumerate}
\end{frame}

\begin{frame}[fragile]
    \frametitle{Conclusion and Key Takeaways}
    Data models are essential in the digital age, enabling modern applications across industries. Understanding their role equips individuals with the ability to manage and utilize data effectively.

    \begin{block}{Key Takeaways}
        \begin{itemize}
            \item Data models serve as blueprints for data organization and access.
            \item They facilitate communication, efficiency, and scalability.
            \item Appropriate modeling ensures data integrity and supports specific application needs.
        \end{itemize}
    \end{block}

    By recognizing the significance of data models, learners can approach subsequent topics in data management and application development with greater insight.
\end{frame}

\begin{frame}[fragile]{What is a Data Model? - Definition}
    \begin{block}{Definition of a Data Model}
        A \textbf{data model} is an abstract representation that defines how data is structured and organized within a database or information system. It serves as the blueprint for storing, managing, and manipulating data efficiently. By providing a systematic framework, data models help to ensure data integrity, accuracy, and ease of access.
    \end{block}
\end{frame}

\begin{frame}[fragile]{What is a Data Model? - Role in Organizing Data}
    \begin{block}{Role of Data Models in Organizing and Structuring Data}
        \begin{enumerate}
            \item \textbf{Organization of Data:} Data models specify how data is grouped, categorized, and related. 
            \begin{itemize}
                \item Example: A customer database may include tables for \textbf{Customers}, \textbf{Orders}, and \textbf{Products}.
            \end{itemize}
            
            \item \textbf{Data Integrity:} They enforce rules about data storage and relationships to maintain accuracy and consistency.
            \begin{itemize}
                \item Example: In a relational database, constraints may ensure an order exists only if it corresponds to a valid customer ID.
            \end{itemize}
            
            \item \textbf{Facilitate Communication:} Data models provide a common language for developers and stakeholders.
            \begin{itemize}
                \item Example: An Entity-Relationship (ER) diagram visually presents how entities relate.
            \end{itemize}
            
            \item \textbf{Base for Implementation:} They guide the implementation of databases, including table structures and indexes.
            \begin{itemize}
                \item Example: A data model might specify that a customer ID field in the Customers table should be of integer type and indexed.
            \end{itemize}
        \end{enumerate}
    \end{block}
\end{frame}

\begin{frame}[fragile]{What is a Data Model? - Key Points and Conclusion}
    \begin{block}{Key Points to Emphasize}
        \begin{itemize}
            \item Data models are essential for organizing complex data meaningfully.
            \item They provide the foundation for data integrity and structured management.
            \item Understanding data models is crucial for efficient database design.
        \end{itemize}
    \end{block}
    
    \begin{block}{Conclusion}
        Data models are foundational components in database design. They help organize data and relationships clearly and consistently, facilitating effective data management and supporting business intelligence efforts. Understanding the fundamentals of data models is critical for anyone involved in data management.
    \end{block}
\end{frame}

\begin{frame}[fragile]{What is a Data Model? - Diagram}
    \begin{block}{Example: Simple Representation of a Data Model}
        Below is a basic Entity-Relationship Diagram (ERD) illustrating the relationship between entities:
        
        \begin{center}
            \begin{verbatim}
+----------------+       +----------------+
|   Customers    |<----- |     Orders     |
+----------------+       +----------------+
| CustomerID (PK)|       | OrderID (PK)   |
| Name           |       | OrderDate      |
| Email          |       | CustomerID (FK)|
+----------------+       +----------------+
            \end{verbatim}
        \end{center}
        
        \textbf{Legend:} 
        \begin{itemize}
            \item PK: Primary Key
            \item FK: Foreign Key
        \end{itemize}
        
        In this diagram, \textbf{Customers} is linked to \textbf{Orders} through CustomerID.
    \end{block}
\end{frame}

\begin{frame}[fragile]
    \frametitle{Types of Data Models - Overview}
    \begin{block}{Overview of Different Types of Data Models}
        Data models are essential frameworks that determine how data is stored, organized, and manipulated. Understanding the types of data models helps us choose the right architecture for specific application needs.
    \end{block}
\end{frame}

\begin{frame}[fragile]
    \frametitle{Types of Data Models - Relational Data Models}
    \begin{enumerate}
        \item \textbf{Relational Data Models}
        \begin{itemize}
            \item \textbf{Definition}: Organizes data into tables (relations) with rows and columns & relates through foreign keys.
            \item \textbf{Key Concepts}:
            \begin{itemize}
                \item Tables: Collections of related data entries (e.g., \texttt{Customers}, \texttt{Orders}).
                \item Schema: The structure defining how data is organized, including tables, fields, and data types.
                \item Primary Key: A unique identifier for each record in a table.
            \end{itemize}
            \item \textbf{Example}:
            \begin{center}
                \begin{tabular}{|c|c|c|c|}
                    \hline
                    CustomerID & FirstName & LastName & Email \\
                    \hline
                    1 & John & Doe & john@example.com \\
                    2 & Jane & Smith & jane@example.com \\
                    \hline
                \end{tabular}
            \end{center}
            \item \textbf{Emphasis}: Optimal for structured data and support complex queries using SQL.
        \end{itemize}
    \end{enumerate}
\end{frame}

\begin{frame}[fragile]
    \frametitle{Types of Data Models - NoSQL and Graph Data Models}
    \begin{enumerate}
        \setcounter{enumi}{1}
        \item \textbf{NoSQL Data Models}
        \begin{itemize}
            \item \textbf{Definition}: Designed to handle large volumes of unstructured or semi-structured data with flexibility, scalability, and high performance.
            \item \textbf{Key Concepts}:
            \begin{itemize}
                \item Document Stores: Use documents (often JSON-like structures) to store data. Example: MongoDB.
                \item Key-Value Stores: Store data as key-value pairs. Example: Redis.
                \item Column-Family Stores: Organize data into columns rather than rows. Example: Cassandra.
            \end{itemize}
            \item \textbf{Example (Document Store)}:
            \begin{lstlisting}[language=json]
            {
              "ProductID": "12345",
              "Name": "Laptop",
              "Specs": {
                "CPU": "Intel i7",
                "RAM": "16GB",
                "Storage": "512GB SSD"
              }
            }
            \end{lstlisting}
            \item \textbf{Emphasis}: Excels in handling diverse data structures for big data and real-time applications.
        \end{itemize}
        
        \item \textbf{Graph Data Models}
        \begin{itemize}
            \item \textbf{Definition}: Use graph structures with nodes, edges, and properties to represent and store data, illustrating relationships effectively.
            \item \textbf{Key Concepts}:
            \begin{itemize}
                \item Nodes: Entities such as people, products, or locations.
                \item Edges: Connections or relationships between nodes (e.g., "friend", "purchased").
                \item Properties: Attributes related to nodes and edges.
            \end{itemize}
            \item \textbf{Example}:
            \begin{center}
                (Alice) --fr:friend--> (Bob) \\ 
                (Alice) --lp:likes--> (Laptop)
            \end{center}
            \item \textbf{Emphasis}: Highly efficient for traversing relationships and modeling complex networks, ideal for social networks and fraud detection.
        \end{itemize}
    \end{enumerate}
\end{frame}

\begin{frame}[fragile]
    \frametitle{Types of Data Models - Key Points}
    \begin{block}{Key Points to Remember}
        \begin{itemize}
            \item \textbf{Relational Models}: Best for structured data with complex queries (SQL).
            \item \textbf{NoSQL Models}: Offer flexibility and scalability for various data types.
            \item \textbf{Graph Models}: Ideal for representing interconnected data and relationships.
        \end{itemize}
    \end{block}
    
    \begin{block}{Conclusion}
        This overview provides a foundation to appreciate the diversity in data modeling approaches and their respective applications in modern data management systems. Transitioning to the next segment, we will delve deeper into relational databases, unpacking their core components!
    \end{block}
\end{frame}

\begin{frame}[fragile]
    \frametitle{Relational Databases - Introduction}
    \begin{itemize}
        \item Relational databases store data in a structured format.
        \item Central concepts include:
        \begin{itemize}
            \item Tables
            \item Schema
            \item Queries
        \end{itemize}
        \item Understanding these concepts is essential for effective data management.
    \end{itemize}
\end{frame}

\begin{frame}[fragile]
    \frametitle{Relational Databases - Key Concepts}
    \begin{enumerate}
        \item \textbf{Tables}
        \begin{itemize}
            \item A collection of related data organized in rows and columns.
            \item \textbf{Rows}: Each row is a record.
            \item \textbf{Columns}: Each column is a specific attribute.
        \end{itemize}
        \item \textbf{Schema}
        \begin{itemize}
            \item Defines the structure of the database.
            \item Components include:
            \begin{itemize}
                \item Tables
                \item Data Types
            \end{itemize}
        \end{itemize}
        \item \textbf{Queries}
        \begin{itemize}
            \item Requests to retrieve or manipulate data, usually written in SQL.
        \end{itemize}
    \end{enumerate}
\end{frame}

\begin{frame}[fragile]
    \frametitle{Relational Databases - Examples}
    \begin{block}{Example Table: Customers}
    \begin{tabular}{|c|c|c|c|}
        \hline
        CustomerID & FirstName & LastName & Email \\
        \hline
        1 & John & Doe & john.doe@email.com \\
        2 & Jane & Smith & jane.smith@email.com \\
        \hline
    \end{tabular}
    \end{block}

    \begin{block}{Sample SQL Queries}
    \begin{lstlisting}[language=SQL]
-- Selecting all customer records
SELECT * FROM Customers;

-- Inserting a new customer
INSERT INTO Customers (CustomerID, FirstName, LastName, Email)
VALUES (3, 'Alice', 'Johnson', 'alice.johnson@email.com');
    \end{lstlisting}
    \end{block}
\end{frame}

\begin{frame}[fragile]
    \frametitle{Use Cases for Relational Databases - Overview}
    \begin{block}{Summary}
        Relational databases excel in applications requiring:
        \begin{itemize}
            \item High data integrity
            \item Complex querying capabilities
            \item Structured relationships among data entities
        \end{itemize}
    \end{block}
    \begin{block}{Key Points}
        \begin{itemize}
            \item ACID properties ensure data accuracy and consistency.
            \item Structured data management enhances efficiency.
            \item SQL enables powerful data analysis.
        \end{itemize}
    \end{block}
\end{frame}

\begin{frame}[fragile]
    \frametitle{Use Cases for Relational Databases - Part 1}
    \begin{enumerate}
        \item \textbf{Transactional Systems}
            \begin{itemize}
                \item \textbf{Description:} Well-suited for applications needing strong data integrity (e.g., banking, e-commerce).
                \item \textbf{Example:} E-commerce services require multiple steps in transactions, ensuring adherence to ACID properties.
            \end{itemize}
        \item \textbf{Customer Relationship Management (CRM)}
            \begin{itemize}
                \item \textbf{Description:} Handles structured data efficiently for customer profiles and interactions.
                \item \textbf{Example:} CRM systems store customer and sales records in interconnected tables for efficient data retrieval.
            \end{itemize}
    \end{enumerate}
\end{frame}

\begin{frame}[fragile]
    \frametitle{Use Cases for Relational Databases - Part 2}
    \begin{enumerate}
        \setcounter{enumi}{2}
        \item \textbf{Data Analysis \& Reporting}
            \begin{itemize}
                \item \textbf{Description:} Complex queries for generating reports and analytics.
                \item \textbf{Example:} Sales departments analyze data with SQL queries to identify trends.
            \end{itemize}
        \item \textbf{Content Management Systems (CMS)}
            \begin{itemize}
                \item \textbf{Description:} Manages content metadata and user information efficiently.
                \item \textbf{Example:} WordPress uses MySQL to manage users, posts, and settings.
            \end{itemize}
        \item \textbf{Inventory Management Systems}
            \begin{itemize}
                \item \textbf{Description:} Keeps track of stock levels and product data.
                \item \textbf{Example:} Retail stores monitor product stock levels and supplier information.
            \end{itemize}
    \end{enumerate}
\end{frame}

\begin{frame}[fragile]
    \frametitle{Limitations of Relational Databases - Introduction}
    \begin{block}{Overview}
        Relational databases have been pivotal in data management due to their structured approach and powerful querying capabilities. However, several limitations can pose challenges, especially in modern environments.
    \end{block}
\end{frame}

\begin{frame}[fragile]
    \frametitle{Limitations of Relational Databases - Scalability & Complexity}
    \begin{enumerate}
        \item \textbf{Scalability Issues:}
        \begin{itemize}
            \item \textbf{Vertical Scaling:} Often requires upgrading single server resources (e.g., CPU/RAM), which can be costly.
            \item \textbf{Management of Large Data Volumes:} Difficult and expensive to manage as data grows exponentially.
        \end{itemize}
        
        \item \textbf{Complexity of Managing Relationships:}
        \begin{itemize}
            \item \textbf{Joins and Performance:} Complex queries with multiple tables can lead to slow performance.
        \end{itemize}
    \end{enumerate}
\end{frame}

\begin{frame}[fragile]
    \frametitle{Limitations of Relational Databases - Schema & Flexibility}
    \begin{enumerate}
        \setcounter{enumi}{2}
        \item \textbf{Fixed Schema:}
        \begin{itemize}
            \item \textbf{Rigidity:} Changing database structure requires careful planning, often resulting in downtime.
        \end{itemize}

        \item \textbf{Limited Flexibility with Unstructured Data:}
        \begin{itemize}
            \item \textbf{Handling Varied Data Types:} Less effective for unstructured or semi-structured data.
        \end{itemize}
            
        \item \textbf{Transaction Overhead:}
        \begin{itemize}
            \item \textbf{ACID Compliance:} Maintaining transactional properties can introduce performance overhead, especially under load.
        \end{itemize}
    \end{enumerate}
\end{frame}

\begin{frame}[fragile]
    \frametitle{Limitations of Relational Databases - Key Points and Conclusion}
    \begin{block}{Key Points to Emphasize}
        - Despite their reliability, relational databases have limitations that must be considered in environments needing high scalability and support for diverse data.
        - Understanding these challenges is essential for informed decisions about data architecture.
    \end{block}

    \begin{block}{Conclusion}
        - While powerful, relational databases may not be suitable for all modern applications. Awareness of their limitations can facilitate smoother transitions to other models like NoSQL.
    \end{block}

    \begin{block}{Next Up}
        In the following slide, we will explore NoSQL databases as a response to some challenges faced by traditional relational databases.
    \end{block}
\end{frame}

\begin{frame}[fragile]
    \frametitle{Introduction to NoSQL Databases}
    \begin{block}{Overview of NoSQL Databases}
        NoSQL (Not Only SQL) databases are designed to handle large amounts of structured, semi-structured, and unstructured data. They serve as an alternative to traditional relational databases to meet the needs of modern applications.
    \end{block}
\end{frame}

\begin{frame}[fragile]
    \frametitle{Key Characteristics of NoSQL Databases}
    \begin{enumerate}
        \item \textbf{Flexibility in Data Models}
            \begin{itemize}
                \item Can handle various data types without a predefined structure.
                \item \textit{Example}: MongoDB stores JSON-like documents with varying fields.
            \end{itemize}

        \item \textbf{Scalability}
            \begin{itemize}
                \item Designed to scale horizontally by adding more servers.
                \item \textit{Example}: Cassandra distributes data across many servers for quick scalability.
            \end{itemize}

        \item \textbf{High Availability}
            \begin{itemize}
                \item Built with replication and fault tolerance.
                \item \textit{Example}: DynamoDB provides consistent performance with automatic data replication.
            \end{itemize}
    \end{enumerate}
\end{frame}

\begin{frame}[fragile]
    \frametitle{Continued Characteristics of NoSQL Databases}
    \begin{enumerate}
        \setcounter{enumi}{3} % Continue from previous frame
        \item \textbf{High Performance}
            \begin{itemize}
                \item Optimized for specific access patterns, yielding faster data retrieval.
                \item \textit{Example}: Redis is used for caching to speed up response times.
            \end{itemize}

        \item \textbf{Eventual Consistency}
            \begin{itemize}
                \item Prioritizes availability and partition tolerance, adopting an eventual consistency model.
                \item \textit{Example}: In social media apps, user posts may appear with a delay for higher availability.
            \end{itemize}
    \end{enumerate}
    
    \begin{block}{Comparison to Relational Databases}
        \begin{itemize}
            \item Schema-less vs. fixed schema
            \item Different handling of data relationships
            \item Use cases: NoSQL for big data vs. Relational for structured data
        \end{itemize}
    \end{block}
\end{frame}

\begin{frame}[fragile]
    \frametitle{Key Points and Conclusion}
    \begin{itemize}
        \item NoSQL is not a replacement for SQL but provides alternatives for specific needs.
        \item Careful consideration of data requirements is essential for choosing between NoSQL and relational databases.
        \item NoSQL represents a shift in data management approaches, crucial for scalable and flexible applications.
    \end{itemize}

    \textbf{Transition to Next Slide:} In the next slide, we will explore different types of NoSQL databases and their unique features.
\end{frame}

\begin{frame}
    \frametitle{Types of NoSQL Databases}
    \begin{block}{Introduction to NoSQL Database Types}
        NoSQL databases offer flexible schemas and scalability, catering to various data storage needs. 
        In this section, we will explore four primary types of NoSQL databases:
        \begin{itemize}
            \item Document
            \item Key-Value
            \item Column-Family
            \item Graph
        \end{itemize}
    \end{block}
\end{frame}

\begin{frame}[fragile]
    \frametitle{Document Databases}
    \begin{block}{Definition}
        Store data in documents similar to JSON or XML. Each document can contain nested structures and varying fields.
    \end{block}
    
    \begin{block}{Example}
        MongoDB, Couchbase
    \end{block}
    
    \begin{block}{Use Case}
        Ideal for applications needing unstructured data representation, like content management systems or user profiles.
    \end{block}
    
    \begin{lstlisting}[language=json]
    {
        "user_id": "12345",
        "name": "John Doe",
        "email": "john@example.com",
        "preferences": {
            "newsletter": true,
            "notifications": ["email", "SMS"]
        }
    }
    \end{lstlisting}
\end{frame}

\begin{frame}
    \frametitle{Key-Value Stores}
    \begin{block}{Definition}
        Data is stored as a collection of key-value pairs. Values can be any type of data, including objects or binary data.
    \end{block}
    
    \begin{block}{Example}
        Redis, DynamoDB
    \end{block}
    
    \begin{block}{Use Case}
        Best for high-speed lookups of simple data structures, such as session storage or caching systems.
    \end{block}
    
    \begin{itemize}
        \item Key: "user:12345"
        \item Value: "John Doe"
    \end{itemize}
\end{frame}

\begin{frame}
    \frametitle{Column-Family Databases}
    \begin{block}{Definition}
        Store data in columns rather than rows, allowing for efficient storage of large data sets. Each column family can contain many rows, and data can be grouped based on column families.
    \end{block}
    
    \begin{block}{Example}
        Cassandra, HBase
    \end{block}
    
    \begin{block}{Use Case}
        Suited for analytical data processing and time-series data where read and write throughput is critical.
    \end{block}
    
    \begin{tabular}{|c|c|c|}
        \hline
        User ID & Name        & Email                \\
        \hline
        12345   & John Doe   & john@example.com     \\
        \hline
        67890   & Jane Smith & jane@example.com     \\
        \hline
    \end{tabular}
\end{frame}

\begin{frame}
    \frametitle{Graph Databases}
    \begin{block}{Definition}
        Designed to store data in graph structures with nodes, edges, and properties, optimizing relationships and connections between data points.
    \end{block}
    
    \begin{block}{Example}
        Neo4j, ArangoDB
    \end{block}
    
    \begin{block}{Use Case}
        Best for applications that analyze relationships, such as social networks or recommendation systems.
    \end{block}
    
    \begin{itemize}
        \item Nodes: Represent entities (e.g., people, places)
        \item Edges: Represent relationships (e.g., friendships)
    \end{itemize}
\end{frame}

\begin{frame}
    \frametitle{Summary of Key Points}
    \begin{itemize}
        \item Each NoSQL database type is tailored for specific requirements, balancing speed, scalability, and flexibility.
        \item Understanding the nature of the data and the specific use case will guide the choice of the most appropriate NoSQL database.
    \end{itemize}
    
    \begin{block}{Closing Thought}
        By recognizing these diverse database types, developers can select the optimal NoSQL solution aligning with data organization, access patterns, and scalability needs.
    \end{block}
\end{frame}

\begin{frame}[fragile]
    \frametitle{Use Cases for NoSQL Databases - Introduction}
    \begin{block}{Overview}
        NoSQL databases have gained immense popularity due to their flexibility, scalability, and performance in specific scenarios. 
        Unlike traditional relational databases, NoSQL systems handle unstructured and semi-structured data, making them ideal for various applications.
    \end{block}
\end{frame}

\begin{frame}[fragile]
    \frametitle{Use Cases for NoSQL Databases - Key Use Cases}
    \begin{enumerate}
        \item \textbf{Big Data Applications}
            \begin{itemize}
                \item NoSQL databases excel in environments with vast amounts of data.
                \item \textit{Example}: Companies like Netflix and LinkedIn manage extensive user data using NoSQL.
            \end{itemize}

        \item \textbf{Real-Time Data Processing}
            \begin{itemize}
                \item Applications requiring immediate processing benefit from NoSQL’s quick data access.
                \item \textit{Example}: Online gaming applications track player actions in real-time efficiently.
            \end{itemize}

        \item \textbf{Content Management Systems (CMS)}
            \begin{itemize}
                \item Supports varying data formats without a predefined schema.
                \item \textit{Example}: Websites like Pinterest use document databases to manage user-generated content.
            \end{itemize}
    \end{enumerate}
\end{frame}

\begin{frame}[fragile]
    \frametitle{Use Cases for NoSQL Databases - Continued}
    \begin{enumerate}
        \setcounter{enumi}{3} % Resume from the previous frame count
        \item \textbf{Social Networks}
            \begin{itemize}
                \item Require flexible data models that evolve as interactions grow.
                \item \textit{Example}: Facebook uses graph databases for complex user interactions. 
            \end{itemize}

        \item \textbf{IoT and Mobile Applications}
            \begin{itemize}
                \item Handle enormous streams of data from devices efficiently.
                \item \textit{Example}: Smart home devices process regular updates effectively with NoSQL technologies.
            \end{itemize}

        \item \textbf{Flexible Schema Requirements}
            \begin{itemize}
                \item NoSQL allows effective schema evolution for variable data structures.
                \item \textit{Example}: E-commerce platforms can easily adapt their schemas for new product attributes.
            \end{itemize}
    \end{enumerate}
\end{frame}

\begin{frame}[fragile]
    \frametitle{Use Cases for NoSQL Databases - Conclusion}
    \begin{block}{Conclusion}
        NoSQL databases thrive in scenarios demanding scalability, flexibility, and fast processing of diverse data types. 
        Understanding these use cases helps businesses choose the right database solution tailored to specific needs.
    \end{block}
    
    \begin{block}{Key Points to Emphasize}
        \begin{itemize}
            \item Effective in handling vast data volumes characteristic of big data.
            \item Low latency benefits real-time applications.
            \item Flexible schema supports dynamic data needs.
        \end{itemize}
    \end{block}
\end{frame}

\begin{frame}[fragile]
    \frametitle{Limitations of NoSQL Databases}
    \begin{block}{Introduction}
        While NoSQL databases offer numerous advantages, they also have limitations. Understanding these drawbacks is crucial for making informed decisions about data storage and management.
    \end{block}
\end{frame}

\begin{frame}[fragile]
    \frametitle{Common Limitations of NoSQL Databases}
    \begin{enumerate}
        \item \textbf{Lack of Standardization}
            \begin{itemize}
                \item Differences in data models and query languages compared to SQL databases.
                \item Example: MongoDB uses a document-based model; Cassandra uses a column-family structure.
                \item \textbf{Key Point:} Leads to steep learning curves and increased complexity.
            \end{itemize}
        \item \textbf{Limited Query Capabilities}
            \begin{itemize}
                \item Many NoSQL databases lack support for complex queries and joins.
                \item Example: Document stores may require application-side processing for joins.
                \item \textbf{Key Point:} Affects performance for applications with intricate data relationships.
            \end{itemize}
    \end{enumerate}
\end{frame}

\begin{frame}[fragile]
    \frametitle{Additional Limitations}
    \begin{enumerate}[resume]
        \item \textbf{Eventual Consistency}
            \begin{itemize}
                \item Updates may not be immediately visible to all users.
                \item Example: In Amazon DynamoDB, a user may access stale data after an update.
                \item \textbf{Key Point:} Critical for applications needing real-time data accuracy.
            \end{itemize}
        \item \textbf{Complexity in Transactions}
            \begin{itemize}
                \item ACID compliance is often relaxed in NoSQL systems.
                \item Example: Partial updates across distributed nodes can cause integrity issues.
                \item \textbf{Key Point:} Strong transactional support is a challenge in these databases.
            \end{itemize}
        \item \textbf{Scalability Challenges}
            \begin{itemize}
                \item Managing large clusters can become complex despite horizontal scaling.
                \item Example: Operational overhead increases with more nodes.
                \item \textbf{Key Point:} Scaling out requires robust management strategies.
            \end{itemize}
    \end{enumerate}
\end{frame}

\begin{frame}[fragile]
    \frametitle{Final Limitations and Conclusion}
    \begin{enumerate}[resume]
        \item \textbf{Support and Community}
            \begin{itemize}
                \item Popular systems have good support, but emerging ones might lack resources.
                \item Example: Less popular databases may have minimal documentation compared to SQL databases like MySQL.
                \item \textbf{Key Point:} Businesses face hurdles in finding expert support.
            \end{itemize}
    \end{enumerate}
    
    \begin{block}{Conclusion}
        NoSQL databases provide flexibility and performance benefits but come with limitations that must be critically evaluated. Weigh the benefits against drawbacks to find the best fit for your application needs.
    \end{block}
    
    \begin{block}{Final Thought}
        When selecting a database solution, consider the trade-offs of NoSQL within the context of your specific use case.
    \end{block}
\end{frame}

\begin{frame}[fragile]
    \frametitle{Introduction to Graph Databases}
    Graph databases are a type of NoSQL database designed to handle and represent data in a graph format. 
    They utilize graph structures composed of nodes, edges, and properties to model and query complex relationships between data, 
    making them particularly powerful for applications involving interconnected data.
\end{frame}

\begin{frame}[fragile]
    \frametitle{Key Components of Graph Databases}
    \begin{enumerate}
        \item \textbf{Nodes}: Represents entities (e.g., persons, products, organizations).
        \begin{itemize}
            \item Example: In a social network, each user is a node.
        \end{itemize}
        
        \item \textbf{Edges}: Represents relationships between nodes (e.g., friendships, transactions).
        \begin{itemize}
            \item Example: A friendship between two users can be an edge connecting user nodes.
        \end{itemize}
        
        \item \textbf{Properties}: Attributes associated with nodes and edges (e.g., a user's name, age).
        \begin{itemize}
            \item Example: A node for a user might have properties like ``name: Alice, age: 30''.
        \end{itemize}
    \end{enumerate}
\end{frame}

\begin{frame}[fragile]
    \frametitle{Structure and Data Representation}
    Graph databases visualize relationships as connections between entities, which can be traversed efficiently. 
    The interconnected nature mimics real-world scenarios, allowing for intuitive data modeling.
    
    \begin{block}{Graph Model Representation}
      % Visual representation
      \begin{verbatim}
      [User A] --[Friend]--> [User B]
                   |   
                   +--[Follows]--> [User C]
      \end{verbatim}
      \end{block}
    
    This representation shows that User A is friends with User B and follows User C.
    
    \begin{itemize}
        \item \textbf{Benefits of Graph Databases:}
        \begin{itemize}
            \item Flexible schema adaptable to changes in data relationships.
            \item Efficient traversal for fast querying of complex relationships.
            \item Natural representation that models real-world scenarios intuitively.
        \end{itemize}
    \end{itemize}
\end{frame}

\begin{frame}[fragile]
    \frametitle{Use Cases and Key Points}
    \begin{itemize}
        \item \textbf{Use Cases in Real Life:}
        \begin{itemize}
            \item Social Networks: Understanding connections and facilitating interactions.
            \item Recommendation Engines: Suggesting products based on user behavior.
            \item Fraud Detection: Identifying unusual patterns in transactions.
        \end{itemize}
        
        \item \textbf{Key Points to Emphasize:}
        \begin{itemize}
            \item Graph databases excel in scenarios with complex, interconnected data.
            \item They provide flexibility and speed in data querying compared to relational databases.
            \item Understanding relational patterns is essential for leveraging graph database potential.
        \end{itemize}
    \end{itemize}
\end{frame}

\begin{frame}[fragile]
    \frametitle{Conclusion}
    Graph databases represent a powerful tool in data modeling, offering an efficient way to manage and explore relationships between diverse data entities. 
    As we progress to the next slide, we will uncover specific use cases where graph databases shine, demonstrating their practical applications in various industries.
\end{frame}

\begin{frame}[fragile]
    \frametitle{Use Cases for Graph Databases - Overview}
    \begin{block}{Overview}
        Graph databases excel in managing and analyzing complex relationships between data points. Unlike traditional relational databases, they are designed to handle data that is interconnected, allowing for efficient querying and visualization of relationships.
    \end{block}
\end{frame}

\begin{frame}[fragile]
    \frametitle{Use Cases for Graph Databases - Key Advantages}
    \begin{itemize}
        \item \textbf{Relationship-Centric Queries}:
        \begin{itemize}
            \item Graph databases optimize queries involving traversing connections between entities. For example, "finding friends of friends" in social networks is more efficient than in relational databases.
        \end{itemize}
        
        \item \textbf{Dynamic Schema}:
        \begin{itemize}
            \item They support flexible schemas, allowing for new node types and relationships to be added without disrupting existing structures. This is useful in evolving domains like social networks or recommendation engines.
        \end{itemize}
        
        \item \textbf{Performance with Complex Queries}:
        \begin{itemize}
            \item Performance degrades in relational databases as relationships grow complex. Graph databases maintain performance by indexing relationships inherently, speeding up tasks like pathfinding and clustering.
        \end{itemize}
    \end{itemize}
\end{frame}

\begin{frame}[fragile]
    \frametitle{Use Cases for Graph Databases - Scenarios}
    \begin{enumerate}
        \item \textbf{Social Networks}:
        \begin{itemize}
            \item Example: Connecting users, posts, and likes can be modeled as a graph, facilitating real-time recommendations and user behavior analysis.
            \item Advantage: Fast retrieval of connections (friends, followers) and personalized recommendations.
        \end{itemize}
        
        \item \textbf{Recommendation Engines}:
        \begin{itemize}
            \item Example: E-commerce platforms use graph databases to recommend products based on purchase history and product similarities.
            \item Advantage: Generates highly relevant and personalized recommendations.
        \end{itemize}

        \item \textbf{Fraud Detection}:
        \begin{itemize}
            \item Example: In finance, graph databases identify fraudulent patterns and connections between users, transactions, and locations.
            \item Advantage: Analyzing transaction webs quickly surfaces anomalies.
        \end{itemize}

        \item \textbf{Network and IT Operations}:
        \begin{itemize}
            \item Example: Complex network infrastructures can be visualized and analyzed as graphs, showing device connections.
            \item Advantage: Simplifies understanding of network topologies.
        \end{itemize}
        
        \item \textbf{Knowledge Graphs}:
        \begin{itemize}
            \item Example: Search engines use knowledge graphs to enhance search results with contextual entity information.
            \item Advantage: Provides enhanced insights and contextual relevance in searches.
        \end{itemize}
    \end{enumerate}
\end{frame}

\begin{frame}[fragile]
    \frametitle{Use Cases for Graph Databases - Visual Representation}
    \begin{block}{Conceptual Model of a Social Network}
    Consider the following graph:
    \begin{lstlisting}
        [User A] --[FRIENDS_WITH]-- [User B]
            |                |      
            --[LIKES]-- [Post X]  
    \end{lstlisting}
    \end{block}
    \begin{itemize}
        \item Nodes represent entities (Users, Posts).
        \item Edges represent relationships (Friends, Likes).
        \item This structure allows for effective queries, such as finding "friends of User A who liked Post X."
    \end{itemize}
\end{frame}

\begin{frame}[fragile]
    \frametitle{Use Cases for Graph Databases - Conclusion}
    \begin{block}{Conclusion}
        Graph databases are particularly advantageous in scenarios requiring flexibility, complex relationships, and performance in querying interconnected datasets. As businesses increasingly rely on data-driven insights, understanding the power of graph databases will be essential.
    \end{block}
\end{frame}

\begin{frame}[fragile]
    \frametitle{Limitations of Graph Databases - Introduction}
    \begin{block}{Introduction}
        While graph databases are powerful for managing and querying complex relationships, they do come with limitations and challenges. Understanding these constraints is essential for evaluating whether a graph database is the right choice for a specific application.
    \end{block}
\end{frame}

\begin{frame}[fragile]
    \frametitle{Limitations of Graph Databases - Key Limitations}
    \begin{enumerate}
        \item \textbf{Complexity in Implementation and Optimization}
        \begin{itemize}
            \item Requires deep understanding of graph theory and data modeling.
            \item \textit{Example}: Designing schema can be more complicated than for relational databases, especially for cyclical relationships.
        \end{itemize}

        \item \textbf{Scalability Challenges}
        \begin{itemize}
            \item Can lead to performance issues with millions of nodes and edges.
            \item \textit{Illustration}: High write throughput (e.g., social networks) can create bottlenecks.
        \end{itemize}

        \item \textbf{Limited Query Language Maturity}
        \begin{itemize}
            \item Primary query languages (e.g., Cypher for Neo4j) may lack maturity compared to SQL.
            \item \textit{Key Point}: Steeper learning curve for developers.
        \end{itemize}
    \end{enumerate}
\end{frame}

\begin{frame}[fragile]
    \frametitle{Limitations of Graph Databases - Additional Limitations}
    \begin{enumerate}
        \setcounter{enumi}{3}
        \item \textbf{Transaction Management}
        \begin{itemize}
            \item May not support ACID transactions as robustly as relational databases.
            \item \textit{Example}: Risks in financial applications due to strict transaction guarantees.
        \end{itemize}

        \item \textbf{Integration with Existing Systems}
        \begin{itemize}
            \item Challenges integrating with legacy systems designed for relational databases.
            \item \textit{Key Point}: Data migration can be resource-intensive.
        \end{itemize}

        \item \textbf{Cost and Resource Allocation}
        \begin{itemize}
            \item Can be more expensive to implement and maintain than conventional databases.
            \item \textit{Example}: High licensing costs for enterprise solutions.
        \end{itemize}
    \end{enumerate}
\end{frame}

\begin{frame}[fragile]
    \frametitle{Comparative Analysis of Data Models}
    \begin{block}{Overview of Data Models}
        Data models are essential frameworks that govern how data is stored, organized, and retrieved. 
        This analysis will compare three primary data models: Relational, NoSQL, and Graph databases.
    \end{block}
\end{frame}

\begin{frame}[fragile]
    \frametitle{1. Relational Databases}
    \begin{itemize}
        \item \textbf{Definition}: Stores data in structured tables with predefined schemas, utilizing rows and columns.
        \item \textbf{Key Features}:
        \begin{itemize}
            \item Schema-based: Enforces strict data types and structures.
            \item ACID properties: Ensures reliable transaction processing.
        \end{itemize}
        \item \textbf{Example}: MySQL, PostgreSQL
        \item \textbf{Use Case}: Ideal for structured data and complex queries, e.g., financial systems.
    \end{itemize}
    
    \begin{block}{Pros and Cons}
        \textbf{Pros}:
        \begin{itemize}
            \item Strong consistency and integrity
            \item Powerful query capabilities (SQL)
        \end{itemize}
        
        \textbf{Cons}:
        \begin{itemize}
            \item Rigid schema may be cumbersome for evolving data needs
        \end{itemize}
    \end{block}
\end{frame}

\begin{frame}[fragile]
    \frametitle{2. NoSQL Databases}
    \begin{itemize}
        \item \textbf{Definition}: Designed for unstructured, semi-structured, and structured data; uses various data models (document, key-value, etc.).
        \item \textbf{Key Features}:
        \begin{itemize}
            \item Flexible schema: Allows dynamic and varied data structures.
            \item Scalability: Distributes data across multiple servers.
        \end{itemize}
        \item \textbf{Example}: MongoDB (Document), Cassandra (Column-family)
        \item \textbf{Use Case}: Best for big data applications, real-time web applications.
    \end{itemize}
    
    \begin{block}{Pros and Cons}
        \textbf{Pros}:
        \begin{itemize}
            \item High scalability and flexibility
            \item Faster write and query times for certain data types
        \end{itemize}
        
        \textbf{Cons}:
        \begin{itemize}
            \item May sacrifice consistency for availability (CAP theorem)
        \end{itemize}
    \end{block}
\end{frame}

\begin{frame}[fragile]
    \frametitle{3. Graph Databases}
    \begin{itemize}
        \item \textbf{Definition}: Use graph structures (nodes and edges) to represent and store data, focusing on relationships.
        \item \textbf{Key Features}:
        \begin{itemize}
            \item Relationship-centric: Naturally represents interconnected data.
            \item Schema-less: Allows for evolving data relationships.
        \end{itemize}
        \item \textbf{Example}: Neo4j, Amazon Neptune
        \item \textbf{Use Case}: Ideal for social networks, recommendation engines.
    \end{itemize}
    
    \begin{block}{Pros and Cons}
        \textbf{Pros}:
        \begin{itemize}
            \item Efficiently handles complex queries involving relationships
            \item Intuitive data modeling for relational data
        \end{itemize}
        
        \textbf{Cons}:
        \begin{itemize}
            \item Performance can be hampered by large-scale data without optimization
        \end{itemize}
    \end{block}
\end{frame}

\begin{frame}[fragile]
    \frametitle{Summary of Comparative Analysis}
    \begin{table}[ht]
        \centering
        \begin{tabular}{|l|l|l|l|}
            \hline
            \textbf{Feature} & \textbf{Relational Databases} & \textbf{NoSQL Databases} & \textbf{Graph Databases} \\
            \hline
            Data Structure & Structured Tables & Flexible Data Models & Nodes and Edges \\
            \hline
            Schema & Fixed & Dynamic & Schema-less \\
            \hline
            Consistency & Strong (ACID) & Eventual or Strong & Often Eventual \\
            \hline
            Scalability & Vertical & Horizontal & Varies per implementation \\
            \hline
            Optimized For & Complex Queries & Large Volumes of Data & Relationship Queries \\
            \hline
        \end{tabular}
    \end{table}
\end{frame}

\begin{frame}[fragile]
    \frametitle{Choosing the Right Data Model - Overview}
    \begin{block}{Guidelines for Selecting Data Models}
        When choosing a data model, consider:
        \begin{itemize}
            \item Nature of Data
            \item Data Relationships
            \item Access Patterns
            \item Scale and Performance Requirements
            \item Data Consistency Needs
            \item Specific Use Case Scenarios
        \end{itemize}
    \end{block}
\end{frame}

\begin{frame}[fragile]
    \frametitle{Choosing the Right Data Model - Data Needs}
    \begin{enumerate}
        \item \textbf{Understand Your Data Needs}
        \begin{itemize}
            \item \textbf{Nature of Data}: Is it structured, semi-structured, or unstructured?
                \begin{itemize}
                    \item Example: Relational databases (SQL) for structured data; NoSQL for unstructured.
                \end{itemize}
            \item \textbf{Data Relationships}: Complexity of relationships among data entities.
                \begin{itemize}
                    \item Example: For interconnected data, use graph databases (e.g., Neo4j).
                \end{itemize}
        \end{itemize}
    \end{enumerate}
\end{frame}

\begin{frame}[fragile]
    \frametitle{Choosing the Right Data Model - Access Patterns & Scalability}
    \begin{enumerate}
        \setcounter{enumi}{2}
        \item \textbf{Evaluate Access Patterns}
        \begin{itemize}
            \item \textbf{Read vs. Write Operations}: Balance between the two.
                \begin{itemize}
                    \item Example: Denormalized models (key-value stores) for more read ops.
                \end{itemize}
            \item \textbf{Query Types}: Types of queries needed.
                \begin{itemize}
                    \item Example: Complex joins favor relational databases; large dataset analysis favors NoSQL.
                \end{itemize}
        \end{itemize}

        \item \textbf{Scale and Performance Requirements}
        \begin{itemize}
            \item \textbf{Scalability}: Horizontal vs. vertical scaling.
                \begin{itemize}
                    \item Example: MongoDB for horizontal scaling; traditional SQL might struggle.
                \end{itemize}
            \item \textbf{Performance Metrics}: Define goals such as latency and throughput.
                \begin{itemize}
                    \item Example: In-memory databases (e.g., Redis) for low latency.
                \end{itemize}
        \end{itemize}
    \end{enumerate}
\end{frame}

\begin{frame}[fragile]
    \frametitle{Choosing the Right Data Model - Consistency Needs & Conclusion}
    \begin{enumerate}
        \setcounter{enumi}{4}
        \item \textbf{Data Consistency Needs}
        \begin{itemize}
            \item \textbf{ACID vs. BASE Properties}: Understand consistency requirements.
                \begin{itemize}
                    \item Example: Relational models for strict consistency; NoSQL for eventual consistency.
                \end{itemize}
        \end{itemize}

        \item \textbf{Use Case Scenarios}
        \begin{itemize}
            \item \textbf{Relational Databases}: Strong integrity, financial systems (e.g., MySQL).
            \item \textbf{NoSQL Databases}: Real-time apps, big data (e.g., MongoDB).
            \item \textbf{Graph Databases}: Social networks, recommendation engines (e.g., Neo4j).
        \end{itemize}
    \end{enumerate}

    \begin{block}{Key Points}
        \begin{itemize}
            \item Choosing the right data model is critical for performance.
            \item Analyze specific application needs before selection.
            \item Keep updated with new database technologies.
        \end{itemize}
    \end{block}

    \begin{block}{Conclusion}
        Selecting the right data model involves understanding data structures, access patterns, and application needs to ensure successful data management. 
    \end{block}
\end{frame}

\begin{frame}[fragile]
    \frametitle{Conclusion and Key Takeaways - Summary of Key Points}
    \begin{enumerate}
        \item \textbf{Understanding Data Models}:
        \begin{itemize}
            \item A data model describes how data is structured and utilized.
            \item Facilitates organization, management, and retrieval of data.
        \end{itemize}
        
        \item \textbf{Types of Data Models}:
        \begin{itemize}
            \item Hierarchical Model: Tree-like structure (e.g., organizational chart).
            \item Network Model: Multiple relationships (e.g., transportation network).
            \item Relational Model: Uses tables for records and attributes (e.g., student database).
            \item Object-oriented Model: Combines data with behavior (e.g., video game characters).
        \end{itemize}
    \end{enumerate}
\end{frame}

\begin{frame}[fragile]
    \frametitle{Conclusion and Key Takeaways - Choosing the Right Data Model}
    \begin{enumerate}
        \setcounter{enumi}{3}
        \item \textbf{Choosing the Right Data Model}:
        \begin{itemize}
            \item Depends on application needs: complexity, relationships, processing.
            \item Consider factors: scalability, ease of use, performance.
        \end{itemize}

        \item \textbf{Importance of Data Models in Data Processing}:
        \begin{itemize}
            \item Facilitates communication among stakeholders.
            \item Enhances data integrity and security by enforcing rules.
            \item Optimizes performance with efficient query handling.
        \end{itemize}
    \end{enumerate}
\end{frame}

\begin{frame}[fragile]
    \frametitle{Conclusion and Key Takeaways - Key Messages}
    \begin{itemize}
        \item Data models are foundational for effective data management.
        \item Understanding different data models aids in creating effective data strategies.
        \item The appropriate model influences data processing efficiency.
    \end{itemize}

    \begin{block}{Closing Thought}
        Data models are essential tools in system design, implementation, and decision-making, significantly impacting data management success.
    \end{block}
\end{frame}


\end{document}