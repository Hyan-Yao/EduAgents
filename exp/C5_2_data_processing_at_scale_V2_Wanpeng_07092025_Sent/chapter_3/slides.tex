\documentclass[aspectratio=169]{beamer}

% Theme and Color Setup
\usetheme{Madrid}
\usecolortheme{whale}
\useinnertheme{rectangles}
\useoutertheme{miniframes}

% Additional Packages
\usepackage[utf8]{inputenc}
\usepackage[T1]{fontenc}
\usepackage{graphicx}
\usepackage{booktabs}
\usepackage{listings}
\usepackage{amsmath}
\usepackage{amssymb}
\usepackage{xcolor}
\usepackage{tikz}
\usepackage{pgfplots}
\pgfplotsset{compat=1.18}
\usetikzlibrary{positioning}
\usepackage{hyperref}

% Custom Colors
\definecolor{myblue}{RGB}{31, 73, 125}
\definecolor{mygray}{RGB}{100, 100, 100}
\definecolor{mygreen}{RGB}{0, 128, 0}
\definecolor{myorange}{RGB}{230, 126, 34}
\definecolor{mycodebackground}{RGB}{245, 245, 245}

% Set Theme Colors
\setbeamercolor{structure}{fg=myblue}
\setbeamercolor{frametitle}{fg=white, bg=myblue}
\setbeamercolor{title}{fg=myblue}
\setbeamercolor{section in toc}{fg=myblue}
\setbeamercolor{item projected}{fg=white, bg=myblue}
\setbeamercolor{block title}{bg=myblue!20, fg=myblue}
\setbeamercolor{block body}{bg=myblue!10}
\setbeamercolor{alerted text}{fg=myorange}

% Set Fonts
\setbeamerfont{title}{size=\Large, series=\bfseries}
\setbeamerfont{frametitle}{size=\large, series=\bfseries}
\setbeamerfont{caption}{size=\small}
\setbeamerfont{footnote}{size=\tiny}

% Code Listing Style
\lstdefinestyle{customcode}{
  backgroundcolor=\color{mycodebackground},
  basicstyle=\footnotesize\ttfamily,
  breakatwhitespace=false,
  breaklines=true,
  commentstyle=\color{mygreen}\itshape,
  keywordstyle=\color{blue}\bfseries,
  stringstyle=\color{myorange},
  numbers=left,
  numbersep=8pt,
  numberstyle=\tiny\color{mygray},
  frame=single,
  framesep=5pt,
  rulecolor=\color{mygray},
  showspaces=false,
  showstringspaces=false,
  showtabs=false,
  tabsize=2,
  captionpos=b
}
\lstset{style=customcode}

% Custom Commands
\newcommand{\hilight}[1]{\colorbox{myorange!30}{#1}}
\newcommand{\source}[1]{\vspace{0.2cm}\hfill{\tiny\textcolor{mygray}{Source: #1}}}
\newcommand{\concept}[1]{\textcolor{myblue}{\textbf{#1}}}
\newcommand{\separator}{\begin{center}\rule{0.5\linewidth}{0.5pt}\end{center}}

% Footer and Navigation Setup
\setbeamertemplate{footline}{
  \leavevmode%
  \hbox{%
  \begin{beamercolorbox}[wd=.3\paperwidth,ht=2.25ex,dp=1ex,center]{author in head/foot}%
    \usebeamerfont{author in head/foot}\insertshortauthor
  \end{beamercolorbox}%
  \begin{beamercolorbox}[wd=.5\paperwidth,ht=2.25ex,dp=1ex,center]{title in head/foot}%
    \usebeamerfont{title in head/foot}\insertshorttitle
  \end{beamercolorbox}%
  \begin{beamercolorbox}[wd=.2\paperwidth,ht=2.25ex,dp=1ex,center]{date in head/foot}%
    \usebeamerfont{date in head/foot}
    \insertframenumber{} / \inserttotalframenumber
  \end{beamercolorbox}}%
  \vskip0pt%
}

% Turn off navigation symbols
\setbeamertemplate{navigation symbols}{}

% Title Page Information
\title[Distributed Systems]{Chapter 3: Introduction to Distributed Systems}
\author[J. Smith]{John Smith, Ph.D.}
\institute[University Name]{
  Department of Computer Science\\
  University Name\\
  \vspace{0.3cm}
  Email: email@university.edu\\
  Website: www.university.edu
}
\date{\today}

% Document Start
\begin{document}

\frame{\titlepage}

\begin{frame}[fragile]
    \frametitle{Introduction to Distributed Systems}
    \begin{block}{Overview}
        A distributed system is a collection of independent computers that appear to its users as a single coherent system. These computers are connected through a network and work together to achieve a common goal, often providing increased functionality, reliability, and performance.
    \end{block}
\end{frame}

\begin{frame}[fragile]
    \frametitle{Significance in Modern Computing - Part 1}
    \begin{itemize}
        \item \textbf{Scalability:}
        \begin{itemize}
            \item Distributed systems can handle growing amounts of work by adding more machines to the network.
            \item \textbf{Example:} Cloud services like Amazon Web Services (AWS) allow businesses to scale their operations seamlessly without overhauling their infrastructure.
        \end{itemize}
        
        \item \textbf{Fault Tolerance:}
        \begin{itemize}
            \item By distributing tasks and data across multiple nodes, systems can continue functioning even when individual nodes fail.
            \item \textbf{Illustration:} In a cloud service, if one server goes down, others in the system can take over its responsibilities, ensuring service continuity.
        \end{itemize}
    \end{itemize}
\end{frame}

\begin{frame}[fragile]
    \frametitle{Significance in Modern Computing - Part 2}
    \begin{itemize}
        \item \textbf{Resource Sharing:}
        \begin{itemize}
            \item Resources (such as storage and processing power) can be shared across the network, optimizing efficiency and reducing costs.
            \item \textbf{Example:} Distributed databases allow various applications to access and modify a shared dataset in real time.
        \end{itemize}
        
        \item \textbf{Parallel Processing:}
        \begin{itemize}
            \item Tasks can be divided and processed simultaneously across multiple machines, drastically reducing the overall processing time.
            \item \textbf{Example:} Large-scale simulations or data processing tasks (like weather forecasting) utilize distributed systems for faster results.
        \end{itemize}
        
        \item \textbf{Geographical Distribution:}
        \begin{itemize}
            \item Components can be located in different physical locations, allowing for services to be provided globally, enhancing accessibility and latency.
            \item \textbf{Example:} Content Delivery Networks (CDNs) distribute copies of content across multiple locations to serve users from the nearest server.
        \end{itemize}
    \end{itemize}
\end{frame}

\begin{frame}[fragile]
    \frametitle{Key Points to Emphasize}
    \begin{itemize}
        \item Distributed systems provide a powerful framework for building robust, efficient, and scalable applications that are foundational to modern computing.
        \item Understanding distributed systems is essential for harnessing technologies such as cloud computing, IoT, and big data analytics.
        \item The shift from centralized to distributed systems represents a major evolution in tech infrastructure, impacting everything from business operations to internet services.
    \end{itemize}
\end{frame}

\begin{frame}[fragile]
    \frametitle{Next Slide Preview}
    In the upcoming slide, we will define distributed systems more comprehensively and explore their core characteristics, setting the stage for a deeper understanding of their architecture and functioning.
\end{frame}

\begin{frame}[fragile]
    \frametitle{What are Distributed Systems?}
    \begin{block}{Definition}
        A \textbf{Distributed System} is a network of independent computers that appears to its users as a single coherent system. These systems coordinate their activities through message passing and share overall goals while maintaining their individual autonomy.
    \end{block}
\end{frame}

\begin{frame}[fragile]
    \frametitle{Characteristics of Distributed Systems}
    \begin{enumerate}
        \item \textbf{Transparency}
            \begin{itemize}
                \item Users perceive the system as a whole, without needing to know the details of distribution.
                \item \textbf{Example}: Cloud storage services like Google Drive allow seamless file uploads and access without managing data storage specifics.
            \end{itemize}
            
        \item \textbf{Scalability}
            \begin{itemize}
                \item The ability to grow and manage an increasing amount of workload effectively by adding resources, such as nodes.
                \item \textbf{Example}: Social media platforms like Facebook scale to handle billions of users by adding additional servers.
            \end{itemize}
    \end{enumerate}
\end{frame}

\begin{frame}[fragile]
    \frametitle{Characteristics of Distributed Systems (continued)}
    \begin{enumerate}
        \setcounter{enumi}{2}
        \item \textbf{Fault Tolerance}
            \begin{itemize}
                \item Capability of a system to continue functioning even when some components fail, often through redundancy.
                \item \textbf{Example}: In peer-to-peer networks like BitTorrent, if one user goes offline, others still provide the same files.
            \end{itemize}
            
        \item \textbf{Concurrency}
            \begin{itemize}
                \item Multiple processes run simultaneously, allowing efficient resource use and quicker task completion.
                \item \textbf{Example}: Online banking systems can handle transactions from multiple users at the same time without errors.
            \end{itemize}
            
        \item \textbf{Heterogeneity}
            \begin{itemize}
                \item System can be composed of different types of hardware and software.
                \item \textbf{Example}: Networked systems often include diverse operating systems and devices working together seamlessly.
            \end{itemize}
    \end{enumerate}
\end{frame}

\begin{frame}[fragile]
    \frametitle{Key Points and Summary}
    \begin{block}{Key Points}
        \begin{itemize}
            \item Distributed systems enable resources and data access on a global scale.
            \item Underlie essential services such as cloud computing, web services, and communication platforms.
            \item Understanding characteristics is crucial for designing and troubleshooting distributed applications.
        \end{itemize}
    \end{block}

    \begin{block}{Summary}
        Distributed systems are integral to modern computing, providing scalable, efficient, and resilient solutions that allow users to leverage resources and collaborate across geographical boundaries.
        Understanding their definition and characteristics enables better design and management of systems that support continuous and reliable operations.
    \end{block}
\end{frame}

\begin{frame}[fragile]
    \frametitle{Types of Distributed Systems}
    Overview of various types of distributed systems: client-server, peer-to-peer, and hybrid.

    \begin{block}{Key Types}
        Distributed systems can be categorized into three main types:
        \begin{itemize}
            \item Client-Server Systems
            \item Peer-to-Peer (P2P) Systems
            \item Hybrid Systems
        \end{itemize}
    \end{block}
\end{frame}

\begin{frame}[fragile]
    \frametitle{Client-Server Systems}
    \begin{block}{Definition}
        A centralized server provides services or resources to multiple clients. Clients initiate requests, while the server processes these requests and returns the appropriate responses.
    \end{block}

    \begin{itemize}
        \item \textbf{Characteristics:}
        \begin{itemize}
            \item Centralization: The server holds the main resources.
            \item Scalability: Can handle multiple client requests; however, there's a limit based on server capacity.
        \end{itemize}
    
        \item \textbf{Examples:}
        \begin{itemize}
            \item Web Applications: Browsers act as clients that request web pages from a web server.
            \item Email Services: Email clients retrieve email from a dedicated mail server.
        \end{itemize}
    \end{itemize}
\end{frame}

\begin{frame}[fragile]
    \frametitle{Peer-to-Peer (P2P) Systems}
    \begin{block}{Definition}
        Each node (peer) acts as both a client and a server, allowing for direct sharing of resources and data without relying on a central server.
    \end{block}

    \begin{itemize}
        \item \textbf{Characteristics:}
        \begin{itemize}
            \item Decentralization: Each peer can contribute to the network.
            \item Scalability: More scalable and resilient, as adding more peers increases resources.
        \end{itemize}
    
        \item \textbf{Examples:}
        \begin{itemize}
            \item File Sharing: Applications like BitTorrent allow direct file sharing.
            \item Cryptocurrencies: Bitcoin operates on a P2P network.
        \end{itemize}
    \end{itemize}
\end{frame}

\begin{frame}[fragile]
    \frametitle{Hybrid Systems}
    \begin{block}{Definition}
        Hybrid systems blend client-server and P2P architectures, using centralized servers for certain functionalities while allowing peer-to-peer interactions for resource sharing.
    \end{block}

    \begin{itemize}
        \item \textbf{Characteristics:}
        \begin{itemize}
            \item Flexibility: Combines benefits from both models, enhancing efficiency and resource utilization.
            \item Optimized Performance: Centralized servers manage control, while peers handle distributed tasks.
        \end{itemize}
    
        \item \textbf{Examples:}
        \begin{itemize}
            \item Cloud Computing: Platforms like Dropbox use servers for storage management.
            \item Social Media Networks: Users connect to a central platform but also interact directly.
        \end{itemize}
    \end{itemize}
\end{frame}

\begin{frame}[fragile]
    \frametitle{Key Points to Emphasize}
    \begin{itemize}
        \item Client-Server Systems: Best for structured environments with a clear hierarchy.
        \item Peer-to-Peer Systems: Ideal for decentralized applications emphasizing resource sharing.
        \item Hybrid Systems: Provide a flexible approach, taking advantage of both centralized management and decentralized power.
    \end{itemize}
\end{frame}

\begin{frame}[fragile]
    \frametitle{Advantages of Distributed Systems - Overview}
    \begin{block}{Definition and Overview}
        Distributed systems consist of multiple independent components that communicate and coordinate with each other over a network. 
        These systems provide significant benefits compared to centralized architectures.
    \end{block}
\end{frame}

\begin{frame}[fragile]
    \frametitle{Advantages of Distributed Systems - Key Advantages}
    \begin{enumerate}
        \item \textbf{Scalability}
        \begin{itemize}
            \item \textbf{Explanation:} Ability to grow and manage increased load without major disruptions.
            \item \textbf{Example:} A web application can scale horizontally by adding more servers to handle additional requests.
        \end{itemize}

        \item \textbf{Fault Tolerance}
        \begin{itemize}
            \item \textbf{Explanation:} Capability to continue operating effectively despite component failures through redundancy.
            \item \textbf{Example:} In a distributed database, if one node fails, replicas on other nodes ensure continued access.
        \end{itemize}

        \item \textbf{Resource Sharing}
        \begin{itemize}
            \item \textbf{Explanation:} Sharing of resources across locations, enhancing efficiency and flexibility.
            \item \textbf{Example:} Cloud services allow users to access distributed storage across geographic locations.
        \end{itemize}
    \end{enumerate}
\end{frame}

\begin{frame}[fragile]
    \frametitle{Advantages of Distributed Systems - Key Points}
    \begin{itemize}
        \item \textbf{Efficiency:} Improves resource utilization by enabling concurrent tasks across multiple systems.
        
        \item \textbf{Flexibility:} Adapts to changing workloads and demands with dynamic resource reconfiguration.
        
        \item \textbf{Improved Performance:} Parallel processing significantly reduces computation time.
    \end{itemize}
\end{frame}

\begin{frame}[fragile]
    \frametitle{Advantages of Distributed Systems - Conclusion}
    \begin{block}{Conclusion}
        Distributed systems offer crucial advantages suitable for various applications, from cloud services to large-scale web applications. Their scalability, fault tolerance, and resource-sharing capabilities are fundamental principles that enable organizations to deliver reliable and efficient computing solutions.
    \end{block}
\end{frame}

\begin{frame}[fragile]
  \frametitle{Challenges in Distributed Computing}
  Distributed systems are networks of computers that communicate and coordinate their actions by passing messages. While they offer numerous advantages, they also present unique challenges that must be addressed to ensure efficient and reliable operation. 
  \begin{itemize}
      \item Data Consistency
      \item Reliability
      \item Security
  \end{itemize}
\end{frame}

\begin{frame}[fragile]
  \frametitle{1. Data Consistency}
  \begin{block}{Definition}
      Data consistency refers to the principle that all nodes in a distributed system have a unified view of the data at any point in time.
  \end{block}
  
  \begin{itemize}
      \item Complexity in maintaining synchronization due to multiple copies of data.
      \item Example: In a banking application, simultaneous transactions may result in discrepancies if two transactions update the same account balance.
  \end{itemize}

  \begin{block}{Key Concepts}
      \begin{itemize}
          \item \textbf{Consistency Models:}
          \begin{itemize}
              \item Strong Consistency: Guarantees all reads return the most recent write.
              \item Eventual Consistency: If no new updates are made, eventually all accesses will return the last updated value.
          \end{itemize}
      \end{itemize}
  \end{block}
\end{frame}

\begin{frame}[fragile]
  \frametitle{2. Reliability and 3. Security}
  \textbf{Reliability}
  \begin{block}{Definition}
      Reliability refers to the ability of a system to continue functioning correctly in the face of failures.
  \end{block}
  \begin{itemize}
      \item Network failures, server crashes, and data corruption can lead to downtime or data loss.
      \item Example: In a cloud service, if a server fails, the system must reroute traffic to minimize downtime.
  \end{itemize}

  \begin{block}{Key Concepts}
      \begin{itemize}
          \item Redundancy: Implementing replicas of data and services to allow for failover.
          \item Failure Detection Algorithms: Protocols such as Paxos or Raft to manage system failures.
      \end{itemize}
  \end{block}
  
  \textbf{Security}
  \begin{block}{Definition}
      Security encompasses measures to protect data from unauthorized access and ensure confidentiality, integrity, and availability.
  \end{block}
  \begin{itemize}
      \item Data interception during transmission is a significant risk.
      \item Must defend against attacks like spoofing and eavesdropping.
  \end{itemize}

  \begin{block}{Key Concepts}
      \begin{itemize}
          \item Encryption: Protects data in transit; e.g., using TLS for secure communications.
          \item Access Control: Mechanisms to authorize users who can access or modify data.
      \end{itemize}
  \end{block}
\end{frame}

\begin{frame}[fragile]
    \frametitle{Data Models in Distributed Systems - Introduction}
    \begin{block}{Introduction to Data Models}
        In distributed systems, selecting an appropriate data model is crucial for efficient data management, retrieval, and system performance. This slide focuses on three major data models:
    \end{block}
    \begin{itemize}
        \item \textbf{Relational Databases}
        \item \textbf{NoSQL Databases}
        \item \textbf{Graph Databases}
    \end{itemize}
\end{frame}

\begin{frame}[fragile]
    \frametitle{Data Models in Distributed Systems - Relational Databases}
    \begin{block}{1. Relational Databases}
        \begin{itemize}
            \item \textbf{Definition:} Use a structured schema (tables) with defined relationships through foreign keys.
            \item \textbf{Key Characteristics:}
                \begin{itemize}
                    \item Adheres to ACID properties (Atomicity, Consistency, Isolation, Durability).
                    \item Strong data integrity and support for complex queries using SQL.
                \end{itemize}
            \item \textbf{Example:} MySQL
                \begin{table}[]
                    \centering
                    \begin{tabular}{|c|c|c|}
                        \hline
                        UserID & Name  & Email            \\ \hline
                        1      & Alice & alice@email.com   \\ \hline
                        2      & Bob   & bob@email.com     \\ \hline
                    \end{tabular}
                \end{table}
            \item \textbf{Use Cases:} Suitable for applications requiring complex queries such as banking systems.
        \end{itemize}
    \end{block}
\end{frame}

\begin{frame}[fragile]
    \frametitle{Data Models in Distributed Systems - NoSQL Databases and Graph Databases}
    \begin{block}{2. NoSQL Databases}
        \begin{itemize}
            \item \textbf{Definition:} Provide various storage formats (document, key-value, etc.) with flexibility and scalability.
            \item \textbf{Key Characteristics:}
                \begin{itemize}
                    \item Schema-less structure.
                    \item Optimized for horizontal scaling and high availability.
                \end{itemize}
            \item \textbf{Example:} MongoDB
                \begin{lstlisting}
                {
                  "UserID": 1,
                  "Name": "Alice",
                  "Email": "alice@email.com"
                }
                \end{lstlisting}
            \item \textbf{Use Cases:} Ideal for big data applications like social networks.
        \end{itemize}
    \end{block}
    
    \begin{block}{3. Graph Databases}
        \begin{itemize}
            \item \textbf{Definition:} Store data as nodes, edges, and properties, excelling in managing interconnected data.
            \item \textbf{Key Characteristics:}
                \begin{itemize}
                    \item Flexible schema with relationships as first-class citizens.
                    \item Fast query traversal of relationships.
                \end{itemize}
            \item \textbf{Example:} Neo4j
                \begin{itemize}
                    \item Nodes: (Users: Alice, Bob)
                    \item Edges: (Friendship between Alice and Bob)
                \end{itemize}
            \item \textbf{Use Cases:} Suitable for applications like recommendation engines and fraud detection.
        \end{itemize}
    \end{block}
\end{frame}

\begin{frame}[fragile]
    \frametitle{Data Models in Distributed Systems - Summary and Key Points}
    \begin{block}{Choosing the Right Model}
        The choice of data model should align with application requirements such as data relationships, transaction processes, and scalability needs.
    \end{block}
    
    \begin{block}{Key Points}
        \begin{itemize}
            \item Different data models serve different application types.
            \item Understanding use-case scenarios is vital for effective choices.
        \end{itemize}
    \end{block}
    
    \begin{block}{Summary}
        Understanding the distinctions between data models is fundamental. Relational databases maintain integrity, NoSQL databases offer flexibility, and graph databases focus on relationships.
    \end{block}
\end{frame}

\begin{frame}[fragile]
    \frametitle{Scalable Query Processing - Importance}
    \begin{block}{Definition}
        Scalable query processing refers to the ability of a system to handle increasing amounts of data efficiently without a significant drop in performance.
    \end{block}
    
    \begin{itemize}
        \item \textbf{Need for Scalability}: 
              \begin{itemize}
                  \item Traditional databases may struggle as data grows, making timely complex queries critical for maintaining responsive interactions.
              \end{itemize}
        \item \textbf{Key Considerations}:
              \begin{itemize}
                  \item \textbf{Volume}: Managing large datasets (Big Data).
                  \item \textbf{Velocity}: Processing high-speed data streams.
                  \item \textbf{Variety}: Handling diverse data types from various sources.
              \end{itemize}
    \end{itemize}
\end{frame}

\begin{frame}[fragile]
    \frametitle{Scalable Query Processing - Examples}
    \begin{block}{Scenario}
        A company collects user interaction data every second from millions of devices. If relying on traditional SQL databases, querying total interactions might take hours.
    \end{block}
    
    \begin{block}{Illustration}
        \begin{itemize}
            \item Imagine a funnel where millions of interactions enter as raw data and queries filter this down to meaningful insights quickly.
        \end{itemize}
    \end{block}
\end{frame}

\begin{frame}[fragile]
    \frametitle{Introduction to Frameworks - Hadoop}
    \begin{block}{Framework Overview}
        Hadoop is a distributed processing framework designed to handle large datasets using clusters of commodity hardware.
    \end{block}
    
    \begin{itemize}
        \item \textbf{Core Components}:
              \begin{itemize}
                  \item \textbf{HDFS}: Stores vast amounts of data across multiple machines.
                  \item \textbf{MapReduce}: A programming model for processing large data sets with a parallel, distributed algorithm.
              \end{itemize}
        \item \textbf{Example Use Case}: 
              An online retailer analyzes customer behavior patterns to optimize inventory using Hadoop.
    \end{itemize}
\end{frame}

\begin{frame}[fragile]
    \frametitle{Introduction to Frameworks - Spark}
    \begin{block}{Framework Overview}
        Apache Spark is a fast, in-memory data processing engine with elegant and expressive development APIs.
    \end{block}
    
    \begin{itemize}
        \item \textbf{Features}:
              \begin{itemize}
                  \item \textbf{Speed}: Processes data in memory, reducing processing times compared to Hadoop (disk-based).
                  \item \textbf{Ease of Use}: Supports languages like Python, Scala, and R.
              \end{itemize}
        \item \textbf{Example Use Case}:
              A financial institution performing real-time fraud detection across transactions using Spark Streaming.
    \end{itemize}
\end{frame}

\begin{frame}[fragile]
    \frametitle{Key Points and Closing Thoughts}
    \begin{itemize}
        \item Scalability is crucial for businesses to derive insights from large volumes of data quickly.
        \item \textbf{Hadoop} is best for batch processing of large datasets, while \textbf{Spark} excels in real-time processing and machine learning tasks.
        \item Understanding both frameworks is key for working in data-intensive environments.
    \end{itemize}
    
    \begin{block}{Closing Thoughts}
        Choose the right tool based on specific data processing needs. Both Hadoop and Spark offer unique strengths for unlocking the value of data effectively.
    \end{block}
\end{frame}

\begin{frame}[fragile]
    \frametitle{Code Snippet - MapReduce Example}
    \begin{lstlisting}[language=Java]
public class WordCount {
    public static class TokenizerMapper extends Mapper<Object, Text, Text, IntWritable> {
        private final static IntWritable one = new IntWritable(1);
        private Text word = new Text();

        public void map(Object key, Text value, Context context) throws IOException, InterruptedException {
            StringTokenizer itr = new StringTokenizer(value.toString());
            while (itr.hasMoreTokens()) {
                word.set(itr.nextToken());
                context.write(word, one);
            }
        }
    }
}
    \end{lstlisting}
\end{frame}

\begin{frame}
    \frametitle{Design Considerations for Distributed Databases}
    \begin{block}{Key Factors}
        Distributed databases are crucial for modern applications requiring scalability, reliability, and performance. Key design considerations include:
    \end{block}
\end{frame}

\begin{frame}[fragile]
    \frametitle{1. Data Fragmentation}
    \begin{itemize}
        \item \textbf{Explanation:} Split data into smaller, manageable pieces (fragments).
        \item \textbf{Types:}
            \begin{itemize}
                \item \textit{Horizontal Fragmentation:} Rows divided into subsets.
                \item \textit{Vertical Fragmentation:} Columns of a table split.
            \end{itemize}
        \item \textbf{Example:} An e-commerce platform might horizontally fragment customer data based on geographical regions to improve access speed for local users.
    \end{itemize}
\end{frame}

\begin{frame}[fragile]
    \frametitle{2. Data Replication}
    \begin{itemize}
        \item \textbf{Explanation:} Creating copies of data in multiple locations for reliability and availability.
        \item \textbf{Replication Strategies:}
            \begin{itemize}
                \item \textit{Full Replication:} Every site has a complete copy.
                \item \textit{Partial Replication:} Only essential data is duplicated.
            \end{itemize}
        \item \textbf{Example:} Critical user information can be fully replicated across all servers, while less critical logs might be partially replicated.
    \end{itemize}
\end{frame}

\begin{frame}[fragile]
    \frametitle{3. Consistency Models and Network Considerations}
    \begin{itemize}
        \item \textbf{Consistency Models:}
            \begin{itemize}
                \item \textit{Strong Consistency:} All users see the same data at the same time.
                \item \textit{Eventual Consistency:} Updates will propagate eventually, allowing for temporary discrepancies.
            \end{itemize}
        \item \textbf{Example:} Amazon's shopping cart may use eventual consistency.
        \item \textbf{Network Latency and Bandwidth:}
            \begin{itemize}
                \item Minimize latency through geographical data distribution.
                \item Optimize network traffic to reduce congestion.
            \end{itemize}
    \end{itemize}
\end{frame}

\begin{frame}[fragile]
    \frametitle{4. Failover and Recovery Mechanisms}
    \begin{itemize}
        \item \textbf{Explanation:} Strategies to maintain data integrity and availability during failures.
        \item \textbf{Elements:}
            \begin{itemize}
                \item \textit{Automatic Failover:} System detects failures and redirects requests to a backup.
                \item \textit{Data Backup and Restoration:} Regular data snapshots for restoration.
            \end{itemize}
        \item \textbf{Example:} A financial application may implement automatic failover for continuous transaction processing.
    \end{itemize}
\end{frame}

\begin{frame}[fragile]
    \frametitle{5. Load Balancing and Security}
    \begin{itemize}
        \item \textbf{Load Balancing:}
            \begin{itemize}
                \item \textit{Static Load Balancing:} Fixed distribution based on predefined criteria.
                \item \textit{Dynamic Load Balancing:} Adjusts distribution in real-time based on demand.
            \end{itemize}
        \item \textbf{Example:} A video streaming service may use dynamic load balancing based on peak viewing times.
        \item \textbf{Security Considerations:}
            \begin{itemize}
                \item \textit{Encryption:} Secure data both at rest and in transit.
                \item \textit{Access Controls:} Implement strict policies on who can access data.
            \end{itemize}
    \end{itemize}
\end{frame}

\begin{frame}[fragile]
    \frametitle{Key Takeaways and Example}
    \begin{itemize}
        \item Distributed databases require consideration of fragmentation, replication, and consistency models.
        \item Performance is influenced by network, load balancing, and security measures.
        \item Effective failover and load balancing are essential for reliability.
    \end{itemize}
   
    \begin{block}{Illustrative Example of Fragmentation and Replication}
        \begin{lstlisting}
-- Example: Creating a Fragmentation Strategy for Customer Data
CREATE TABLE Customers (
    CustomerID INT PRIMARY KEY,
    Name VARCHAR(100),
    Region VARCHAR(50)
);

-- Horizontal Fragmentation: Customers in the US
SELECT * FROM Customers WHERE Region = 'US';

-- Replication: Copying important customer data to a backup server
INSERT INTO Backup_Customers SELECT * FROM Customers WHERE IMPORTANT_FLAG = 1;
        \end{lstlisting}
    \end{block}
\end{frame}

\begin{frame}[fragile]
  \frametitle{Data Infrastructure Management}
  \begin{block}{Overview}
    Data Infrastructure Management involves overseeing data pipelines, storage solutions, and retrieval systems, particularly in cloud environments. 
    Its focus is on ensuring seamless data flow from sources to endpoints while maintaining performance, reliability, and security.
  \end{block}
\end{frame}

\begin{frame}[fragile]
  \frametitle{Key Concepts - Part 1}
  \begin{enumerate}
    \item \textbf{Data Pipelines:}
    \begin{itemize}
      \item \textit{Definition:} A data pipeline automates the flow of data from a source to a destination, involving data ingestion, transformation, and delivery.
      \item \textit{Example:} 
      \begin{lstlisting}
      Extract user data from a web application, transform it, and load it into a data warehouse (ETL Process).
      \end{lstlisting}
    \end{itemize}
    
    \item \textbf{Data Storage:}
    \begin{itemize}
      \item \textit{Types of Storage Solutions:}
      \begin{itemize}
        \item Structured Data: Relational databases (e.g., PostgreSQL)
        \item Unstructured Data: NoSQL databases (e.g., MongoDB)
        \item Data Lakes: Repositories for raw, unstructured data
      \end{itemize}
    \end{itemize}
  \end{enumerate}
\end{frame}

\begin{frame}[fragile]
  \frametitle{Key Concepts - Part 2}
  \begin{itemize}
    \item \textbf{Data Retrieval:}
    \begin{itemize}
      \item \textit{Importance of Efficient Retrieval:} Fast retrieval is crucial for real-time applications.
      \item \textit{Indexes:} Indexing frequently accessed fields improves speed but can slow down writes.
      \item \textit{Example of Retrieval Query:}
      \begin{lstlisting}
      SELECT * FROM users WHERE user_id = 12345;
      \end{lstlisting}
    \end{itemize}
  \end{itemize}
\end{frame}

\begin{frame}[fragile]
  \frametitle{Managing Cloud Environments}
  \begin{enumerate}
    \item \textbf{Scalability:} 
    \begin{itemize}
      \item Infrastructure must scale horizontally (more machines) and vertically (more resources).
    \end{itemize}
    
    \item \textbf{Availability:} 
    \begin{itemize}
      \item Cloud services ensure high availability through redundancy.
    \end{itemize}
    
    \item \textbf{Cost Management:} 
    \begin{itemize}
      \item Monitor and optimize cloud usage to control costs (e.g., use spot instances).
    \end{itemize}
  \end{enumerate}
\end{frame}

\begin{frame}[fragile]
  \frametitle{Key Points and Conclusion}
  \begin{enumerate}
    \item \textbf{Automation is Key:} Increases efficiency and reduces human error.
    \item \textbf{Monitoring and Analytics:} Use tools like CloudWatch or Prometheus for performance analytics.
    \item \textbf{Security Best Practices:} 
    \begin{itemize}
      \item Ensure data is encrypted in transit and at rest; implement access controls.
    \end{itemize}
  \end{enumerate}
  
  \begin{block}{Conclusion}
    Effective Data Infrastructure Management is essential in distributed systems and sets the foundation for leveraging advanced industry tools discussed later.
  \end{block}
\end{frame}

\begin{frame}[fragile]
  \frametitle{Industry Tools for Distributed Data Processing - Introduction}
  In today's data-driven world, distributed data processing has become essential for managing vast amounts of information efficiently. This slide discusses several industry-standard tools that facilitate distributed data processing, including AWS, Kubernetes, PostgreSQL, and NoSQL databases.
\end{frame}

\begin{frame}[fragile]
  \frametitle{Key Tools for Distributed Data Processing}
  \begin{enumerate}
    \item \textbf{Amazon Web Services (AWS)}
      \begin{itemize}
        \item \textbf{Overview}: A comprehensive cloud platform offering scalable services for computing, storage, and databases.
        \item \textbf{Key Features}:
          \begin{itemize}
            \item Elastic Compute Cloud (EC2) for virtual servers.
            \item Simple Storage Service (S3) for scalable storage.
            \item Managed database services like Amazon RDS and DynamoDB.
          \end{itemize}
        \item \textbf{Example}: A company might use AWS EC2 to run a web application while storing user data in Amazon RDS for relational database management.
      \end{itemize}
  \end{enumerate}
\end{frame}

\begin{frame}[fragile]
  \frametitle{Key Tools for Distributed Data Processing (cont.)}
  \begin{enumerate}
    \setcounter{enumi}{1} % Continue enumeration from previous frame
    \item \textbf{Kubernetes}
      \begin{itemize}
        \item \textbf{Overview}: An open-source platform for automating deployment, scaling, and operation of application containers.
        \item \textbf{Key Features}:
          \begin{itemize}
            \item Efficient resource management through orchestration of containerized applications.
            \item Built-in load balancing and scaling capabilities.
            \item Self-healing features to restart failed containers.
          \end{itemize}
        \item \textbf{Example}: A microservices architecture can utilize Kubernetes to manage different service containers, ensuring each service communicates seamlessly while remaining scalable.
      \end{itemize}

    \setcounter{enumi}{2} % Continue enumeration
    \item \textbf{PostgreSQL}
      \begin{itemize}
        \item \textbf{Overview}: A powerful, open-source relational database system emphasizing extensibility and SQL compliance.
        \item \textbf{Key Features}:
          \begin{itemize}
            \item Supports advanced data types and indexing for efficient data retrieval.
            \item ACID compliance ensures reliable transactions.
            \item Supports JSON data types to facilitate storage of both relational and non-relational data.
          \end{itemize}
        \item \textbf{Example}: A fintech application may use PostgreSQL to handle transactional data while employing JSON features to store user preferences.
      \end{itemize}
  \end{enumerate}
\end{frame}

\begin{frame}[fragile]
  \frametitle{Key Tools for Distributed Data Processing (cont.)}
  \begin{enumerate}
    \setcounter{enumi}{3} % Continue enumeration
    \item \textbf{NoSQL Databases (e.g., MongoDB, Cassandra)}
      \begin{itemize}
        \item \textbf{Overview}: NoSQL databases provide flexible schemas and horizontal scalability for unstructured and semi-structured data.
        \item \textbf{Key Features}:
          \begin{itemize}
            \item Schema-less data models for rapid development.
            \item High availability and partitioned data storage capabilities.
            \item Support for various data formats including key-value pairs, documents, wide-column stores, etc.
          \end{itemize}
        \item \textbf{Example}: An e-commerce platform might utilize MongoDB to store product catalogs, allowing rapid queries and changes without the rigid structure of traditional relational databases.
      \end{itemize}
  \end{enumerate}
\end{frame}

\begin{frame}[fragile]
  \frametitle{Key Points to Emphasize}
  \begin{itemize}
    \item \textbf{Scalability}: Tools like AWS and Kubernetes allow scalable solutions that adapt to changing workloads.
    \item \textbf{Flexibility}: NoSQL databases offer flexible schema designs, enabling developers to innovate without being constrained by traditional database structures.
    \item \textbf{Integration}: All mentioned tools can be integrated to create a powerful and efficient data processing architecture that supports various application needs.
  \end{itemize}
\end{frame}

\begin{frame}[fragile]
    \frametitle{Collaboration in Distributed Projects}
    \begin{block}{Importance of Teamwork}
        In a distributed computing environment, effective teamwork is crucial for productivity, problem-solving, and innovation.
    \end{block}
    
    \begin{itemize}
        \item Diverse Skill Sets: Varied expertise leads to comprehensive solutions.
        \item Improved Problem-Solving: Encourages brainstorming for innovative solutions.
        \item Enhanced Communication: Frequent interactions help align project goals.
    \end{itemize}
\end{frame}

\begin{frame}[fragile]
    \frametitle{Strategies for Effective Collaboration}
    \begin{enumerate}
        \item \textbf{Utilize Collaboration Tools}
        \begin{itemize}
            \item Tools like Slack, JIRA, and GitHub facilitate seamless teamwork.
            \item \textit{Example:} Using GitHub enables simultaneous feature development.
        \end{itemize}
        
        \item \textbf{Define Roles and Responsibilities}
        \begin{itemize}
            \item Clear roles help prevent overlap and confusion.
            \item \textit{Example:} Assign project manager, lead developer, and database administrator roles.
        \end{itemize}
        
        \item \textbf{Regular Meetings and Updates}
        \begin{itemize}
            \item Schedule frequent check-ins to motivate and inform.
            \item \textit{Example:} Weekly video conference for progress reporting.
        \end{itemize}
    \end{enumerate}
\end{frame}

\begin{frame}[fragile]
    \frametitle{Further Strategies for Collaboration}
    \begin{enumerate}[resume]
        \item \textbf{Establish Clear Communication Protocols}
        \begin{itemize}
            \item Set guidelines for communication channels and response times.
            \item \textit{Example:} Document major updates in a shared Google Doc.
        \end{itemize}
        
        \item \textbf{Encourage a Culture of Feedback}
        \begin{itemize}
            \item Promote an open environment for sharing suggestions.
            \item \textit{Example:} Bi-weekly sessions for sharing insights on improvements.
        \end{itemize}
    \end{enumerate}

    \begin{block}{Key Points to Emphasize}
        \begin{itemize}
            \item Collaboration is essential for overcoming complexities in distributed systems.
            \item Effective strategies streamline processes and foster positive environments.
            \item Adaptability and learning can enhance project outcomes.
        \end{itemize}
    \end{block}
\end{frame}

\begin{frame}[fragile]
  \frametitle{Case Study Analysis}
  \begin{block}{Evaluating Past Solutions and Extracting Best Practices in Distributed Systems}
    \begin{itemize}
      \item Definition: In-depth examination of past projects in distributed systems to identify successes, failures, and lessons learned.
      \item Purpose: Inform future design and implementation decisions to improve performance, reliability, and scalability.
    \end{itemize}
  \end{block}
\end{frame}

\begin{frame}[fragile]
  \frametitle{Importance of Analyzing Past Cases}
  \begin{itemize}
    \item \textbf{Learn from Experience}: Understanding successes and failures helps avoid repeating mistakes.
    \item \textbf{Best Practices}: Generalizable insights from specific cases allow teams to adopt proven strategies.
  \end{itemize}
\end{frame}

\begin{frame}[fragile]
  \frametitle{Key Areas to Evaluate}
  \begin{enumerate}
    \item \textbf{Architecture Choices}:
      \begin{itemize}
        \item Compare microservices vs. monolithic architectures.
      \end{itemize}
    \item \textbf{Data Management}:
      \begin{itemize}
        \item Analyze SQL vs. NoSQL impact on performance.
      \end{itemize}
    \item \textbf{Fault Tolerance}:
      \begin{itemize}
        \item Study Google's Spanner for transaction management.
      \end{itemize}
    \item \textbf{Communication Protocols}:
      \begin{itemize}
        \item Evaluate gRPC vs. REST performance in microservices.
      \end{itemize}
    \item \textbf{Deployment Strategies}:
      \begin{itemize}
        \item Continuous Deployment vs. Blue/Green deployments.
      \end{itemize}
  \end{enumerate}
\end{frame}

\begin{frame}[fragile]
  \frametitle{Real-World Case Study Example: Netflix}
  \begin{itemize}
    \item \textbf{Challenge}: Deliver streaming services globally with high availability and low latency.
    \item \textbf{Solution}:
      \begin{itemize}
        \item Adopted a microservices architecture for independent service deployment.
        \item Implemented chaos engineering to enhance stability testing.
      \end{itemize}
    \item \textbf{Best Practices Extracted}:
      \begin{itemize}
        \item Use containerization (Docker) for portability.
        \item Leverage cloud services (AWS) for scalable resource management.
      \end{itemize}
  \end{itemize}
\end{frame}

\begin{frame}[fragile]
  \frametitle{Key Points to Emphasize}
  \begin{itemize}
    \item \textbf{Iterative Learning}: Analyze multiple case studies for a robust collection of best practices.
    \item \textbf{Adaptation and Flexibility}: Best practices should be adapted to the specific context of new projects.
    \item \textbf{Collaboration}: Engage multiple stakeholders for diverse perspectives.
  \end{itemize}
\end{frame}

\begin{frame}[fragile]
  \frametitle{Conclusion and Call to Action}
  \begin{block}{Conclusion}
    Utilizing case study analysis in distributed systems is essential to incorporate historical knowledge and optimize project outcomes.
  \end{block}
  \begin{block}{Call to Action}
    Encourage students to conduct their own case study analyses of notable distributed systems projects for practical understanding.
  \end{block}
\end{frame}

\begin{frame}[fragile]
    \frametitle{Real-World Applications of Distributed Systems}
    \begin{block}{Introduction to Distributed Systems}
        Distributed systems are collections of independent computers that appear as a single coherent system to users. They enable:
        \begin{itemize}
            \item Resource sharing
            \item Scalability
            \item Fault tolerance
        \end{itemize}
    \end{block}
\end{frame}

\begin{frame}[fragile]
    \frametitle{Key Advantages of Distributed Systems}
    \begin{itemize}
        \item \textbf{Scalability:} Systems can grow in size by adding more nodes.
        \item \textbf{Redundancy and Fault Tolerance:} If one node fails, others continue operating.
        \item \textbf{Resource Sharing:} Encourages collaborative processing and resource utilization.
    \end{itemize}
\end{frame}

\begin{frame}[fragile]
    \frametitle{Examples of Successful Implementations}
    \begin{enumerate}
        \item \textbf{Cloud Computing:}
            \begin{itemize}
                \item Services like AWS, Microsoft Azure, and Google Cloud offer scalable distributed computing resources.
                \item Companies deploy applications without managing physical servers, enabling quick updates.
            \end{itemize}
            
        \item \textbf{E-Commerce:}
            \begin{itemize}
                \item Amazon employs distributed systems for inventory management and real-time order processing.
                \item This enhances responsiveness and reduces time to market for products.
            \end{itemize}
            
        \item \textbf{Healthcare:}
            \begin{itemize}
                \item Distributed systems maintain patient records across multiple locations, improving patient care.
            \end{itemize}
    \end{enumerate}
\end{frame}

\begin{frame}[fragile]
    \frametitle{Further Examples and Key Points}
    \begin{enumerate}
        \item \textbf{Social Media Platforms:}
            \begin{itemize}
                \item Facebook uses a distributed architecture to manage user data and interactions.
                \item This ensures fast data retrieval and a seamless user experience.
            \end{itemize}
            
        \item \textbf{Research Computing:}
            \begin{itemize}
                \item SETI@home utilizes volunteers' computers to analyze radio signals.
                \item Shows effective use of distributed computing for scientific research.
            \end{itemize}
    \end{enumerate}

    \begin{block}{Key Points to Emphasize}
        \begin{itemize}
            \item Flexibility: Adapts to different needs and workloads.
            \item User Experience: Enhances application performance and availability.
            \item Collaboration: Harnesses collective computational power for innovations.
        \end{itemize}
    \end{block}
\end{frame}

\begin{frame}[fragile]
    \frametitle{Conclusion}
    The successful implementation of distributed systems showcases their transformative impact across various sectors. By leveraging these systems, organizations can achieve:
    \begin{itemize}
        \item Operational efficiency
        \item Enhanced service delivery
        \item Better decision-making
    \end{itemize}
    Recognizing their potential is crucial in our technology-driven future.
\end{frame}

\begin{frame}[fragile]
  \frametitle{Future Trends in Distributed Computing}
  
  \begin{block}{Introduction}
    As technology continues to evolve, distributed systems are also progressing rapidly. Understanding future trends in distributed computing is essential for anyone looking to specialize in IT, cloud computing, or networking.
  \end{block}

  \begin{block}{Key Trends Shaping the Future}
    \begin{enumerate}
      \item Edge Computing
      \item Serverless Architectures
      \item Microservices and Containerization
      \item Distributed Ledger Technologies (DLT)
      \item Artificial Intelligence and Machine Learning
      \item Quantum Computing
    \end{enumerate}
  \end{block}
\end{frame}

\begin{frame}[fragile]
  \frametitle{Key Trends in Detail}

  \begin{itemize}
    \item \textbf{Edge Computing}
      \begin{itemize}
        \item Brings computation closer to data sources.
        \item Example: Autonomous vehicles process data on-site for quicker decisions.
      \end{itemize}
    
    \item \textbf{Serverless Architectures}
      \begin{itemize}
        \item Developers run applications without managing infrastructure.
        \item Example: AWS Lambda executes functions on demand.
      \end{itemize}

    \item \textbf{Microservices and Containerization}
      \begin{itemize}
        \item Applications are broken into smaller independent services.
        \item Example: A retail website manages different services (authentication, catalog, payment).
      \end{itemize}
    
    \item \textbf{Distributed Ledger Technologies (DLT)}
      \begin{itemize}
        \item Provides decentralized record-keeping.
        \item Example: Bitcoin uses blockchain for transaction security.
      \end{itemize}
    
    \item \textbf{Artificial Intelligence and Machine Learning}
      \begin{itemize}
        \item Integrated into distributed systems for better data analysis.
        \item Example: TensorFlow allows distributed training across machines.
      \end{itemize}
    
    \item \textbf{Quantum Computing}
      \begin{itemize}
        \item Promises exponential speedup for certain computations.
        \item Example: IBM is exploring quantum systems for optimization.
      \end{itemize}
  \end{itemize}
\end{frame}

\begin{frame}[fragile]
    \frametitle{Course Overview}
    In this course, we will explore the fascinating realm of Distributed Systems – a field that enables resource sharing, improved reliability, and scalability through interconnected computing nodes. Distributed systems are crucial for modern applications, from cloud computing to blockchain technology.
    
    Key objectives of this course include:
    \begin{itemize}
        \item \textbf{Understanding the Principles}: Learn foundational concepts including decentralization, fault tolerance, and scalability.
        \item \textbf{Designing Distributed Architectures}: Examine architectural styles such as client-server, peer-to-peer, and microservices.
        \item \textbf{Exploring Distributed Algorithms}: Study key algorithms for resource management, coordination, and consensus mechanisms.
    \end{itemize}
\end{frame}

\begin{frame}[fragile]
    \frametitle{Learning Outcomes - Part 1}
    By the end of this course, you will be able to:
    \begin{enumerate}
        \item \textbf{Explain the Key Characteristics of Distributed Systems}
            \begin{itemize}
                \item \textbf{Example}: Identify how redundancy and resilience are built into systems like Google Cloud or Amazon AWS.
            \end{itemize}
        
        \item \textbf{Identify Various Architectural Models}
            \begin{itemize}
                \item \textbf{Illustration}: Differentiate between various models:
                \begin{itemize}
                    \item \textbf{Client-Server}: A model where a centralized server provides resources/services to users.
                    \item \textbf{Peer-to-Peer (P2P)}: A decentralized model where each node can act as both client and server.
                \end{itemize}
            \end{itemize}
            
        \item \textbf{Analyze Common Challenges and Solutions}
            \begin{itemize}
                \item Understand and address issues such as latency, data consistency, and network partitioning.
                \item \textbf{Key Point}: Use concepts like the CAP Theorem to explain the trade-offs in achieving consistency, availability, and partition tolerance.
            \end{itemize}
    \end{enumerate}
\end{frame}

\begin{frame}[fragile]
    \frametitle{Learning Outcomes - Part 2}
    Continuing from our learning outcomes:
    \begin{enumerate}[resume]
        \item \textbf{Implement Distributed Systems Techniques}
            \begin{itemize}
                \item \textbf{Code Snippet}: A simple example of handling concurrency with mutex locks in a distributed environment:
            \end{itemize}
            \begin{lstlisting}[language=Python]
from threading import Lock

lock = Lock()

def safe_increment(counter):
    with lock:
        counter += 1
        return counter
            \end{lstlisting}

        \item \textbf{Experiment with Real-World Technologies}
            \begin{itemize}
                \item \textbf{Example}: Engage in hands-on projects that utilize technologies like Kubernetes, Docker, and Apache Kafka to build and deploy distributed applications.
            \end{itemize}
    \end{enumerate}
\end{frame}

\begin{frame}[fragile]
    \frametitle{Conclusion}
    By completing this course, you will gain the skills to design, implement, and evaluate distributed systems effectively, equipping you for various challenges in technology environments.
    
    Make sure to leverage the upcoming sessions to deepen your understanding and apply these principles to real-world situations!
\end{frame}

\begin{frame}[fragile]
  \frametitle{Conclusion and Questions - Key Points Recap}
  As we conclude our discussion on Distributed Systems, let’s crystallize the core concepts we’ve covered:

  \begin{enumerate}
    \item \textbf{Definition of Distributed Systems:}
    \begin{itemize}
      \item Multiple interconnected computers working as a single coherent system.
      \item Enables resource sharing, fault tolerance, and scalability.
    \end{itemize}

    \item \textbf{Key Characteristics:}
    \begin{itemize}
      \item \textbf{Scalability:} Handling increasing loads and seamless expansion.
      \item \textbf{Fault Tolerance:} Continuous operation despite individual component failures.
      \item \textbf{Concurrency:} Multiple simultaneous processes sharing resources.
    \end{itemize}
  
    \item \textbf{Types of Distributed Systems:}
    \begin{itemize}
      \item \textbf{Client-Server:} Server provides resources; clients request them.
      \item \textbf{Peer-to-Peer (P2P):} Nodes act as both clients and servers.
      \item \textbf{Cloud Computing:} Distributed resources accessed on-demand over the internet.
    \end{itemize}
  \end{enumerate}
\end{frame}

\begin{frame}[fragile]
  \frametitle{Conclusion and Questions - Challenges and Use Cases}
  \begin{enumerate}
    \setcounter{enumi}{3}
    \item \textbf{Challenges in Distributed Systems:}
    \begin{itemize}
      \item \textbf{Synchronization:} Coordinating operations across nodes.
      \item \textbf{Data Consistency:} Maintaining uniform data amid concurrent modifications.
      \item \textbf{Network Partitioning:} Managing scenarios with isolated system parts.
    \end{itemize}

    \item \textbf{Use Cases:}
    \begin{itemize}
      \item Examples: Online banking, social media platforms, global content delivery networks (CDNs).
    \end{itemize}
  \end{enumerate}
\end{frame}

\begin{frame}[fragile]
  \frametitle{Conclusion and Questions - Real-World Example and Discussion}
  \textbf{Example in Real Life:}

  Consider a global e-commerce platform like Amazon:
  \begin{itemize}
    \item Multiple servers manage user requests in various locations.
    \item Fault tolerance through data replication across servers.
    \item Scalable during high traffic events like Black Friday by adding resources.
  \end{itemize}

  \textbf{Open Floor for Questions:}
  \begin{itemize}
    \item What aspects of distributed systems do you find most challenging?
    \item Are there specific examples or case studies you’d like to explore?
    \item What areas in distributed systems would you like to cover in future chapters?
  \end{itemize}
\end{frame}


\end{document}