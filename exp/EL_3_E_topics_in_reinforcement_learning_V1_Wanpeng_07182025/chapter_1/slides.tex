\documentclass[aspectratio=169]{beamer}

% Theme and Color Setup
\usetheme{Madrid}
\usecolortheme{whale}
\useinnertheme{rectangles}
\useoutertheme{miniframes}

% Additional Packages
\usepackage[utf8]{inputenc}
\usepackage[T1]{fontenc}
\usepackage{graphicx}
\usepackage{booktabs}
\usepackage{listings}
\usepackage{amsmath}
\usepackage{amssymb}
\usepackage{xcolor}
\usepackage{tikz}
\usepackage{pgfplots}
\pgfplotsset{compat=1.18}
\usetikzlibrary{positioning}
\usepackage{hyperref}

% Custom Colors
\definecolor{myblue}{RGB}{31, 73, 125}
\definecolor{mygray}{RGB}{100, 100, 100}
\definecolor{mygreen}{RGB}{0, 128, 0}
\definecolor{myorange}{RGB}{230, 126, 34}
\definecolor{mycodebackground}{RGB}{245, 245, 245}

% Set Theme Colors
\setbeamercolor{structure}{fg=myblue}
\setbeamercolor{frametitle}{fg=white, bg=myblue}
\setbeamercolor{title}{fg=myblue}
\setbeamercolor{section in toc}{fg=myblue}
\setbeamercolor{item projected}{fg=white, bg=myblue}
\setbeamercolor{block title}{bg=myblue!20, fg=myblue}
\setbeamercolor{block body}{bg=myblue!10}
\setbeamercolor{alerted text}{fg=myorange}

% Set Fonts
\setbeamerfont{title}{size=\Large, series=\bfseries}
\setbeamerfont{frametitle}{size=\large, series=\bfseries}
\setbeamerfont{caption}{size=\small}
\setbeamerfont{footnote}{size=\tiny}

% Code Listing Style
\lstdefinestyle{customcode}{
  backgroundcolor=\color{mycodebackground},
  basicstyle=\footnotesize\ttfamily,
  breakatwhitespace=false,
  breaklines=true,
  commentstyle=\color{mygreen}\itshape,
  keywordstyle=\color{blue}\bfseries,
  stringstyle=\color{myorange},
  numbers=left,
  numbersep=8pt,
  numberstyle=\tiny\color{mygray},
  frame=single,
  framesep=5pt,
  rulecolor=\color{mygray},
  showspaces=false,
  showstringspaces=false,
  showtabs=false,
  tabsize=2,
  captionpos=b
}
\lstset{style=customcode}

% Custom Commands
\newcommand{\hilight}[1]{\colorbox{myorange!30}{#1}}
\newcommand{\source}[1]{\vspace{0.2cm}\hfill{\tiny\textcolor{mygray}{Source: #1}}}
\newcommand{\concept}[1]{\textcolor{myblue}{\textbf{#1}}}
\newcommand{\separator}{\begin{center}\rule{0.5\linewidth}{0.5pt}\end{center}}

% Footer and Navigation Setup
\setbeamertemplate{footline}{
  \leavevmode%
  \hbox{%
  \begin{beamercolorbox}[wd=.3\paperwidth,ht=2.25ex,dp=1ex,center]{author in head/foot}%
    \usebeamerfont{author in head/foot}\insertshortauthor
  \end{beamercolorbox}%
  \begin{beamercolorbox}[wd=.5\paperwidth,ht=2.25ex,dp=1ex,center]{title in head/foot}%
    \usebeamerfont{title in head/foot}\insertshorttitle
  \end{beamercolorbox}%
  \begin{beamercolorbox}[wd=.2\paperwidth,ht=2.25ex,dp=1ex,center]{date in head/foot}%
    \usebeamerfont{date in head/foot}
    \insertframenumber{} / \inserttotalframenumber
  \end{beamercolorbox}}%
  \vskip0pt%
}

% Turn off navigation symbols
\setbeamertemplate{navigation symbols}{}

% Title Page Information
\title[Introduction to Reinforcement Learning]{Week 1: Introduction to Reinforcement Learning}
\author[J. Smith]{John Smith, Ph.D.}
\institute[University Name]{
  Department of Computer Science\\
  University Name\\
  \vspace{0.3cm}
  Email: email@university.edu\\
  Website: www.university.edu
}
\date{\today}

% Document Start
\begin{document}

\frame{\titlepage}

\begin{frame}[fragile]
    \frametitle{Introduction to Reinforcement Learning}
    \begin{block}{What is Reinforcement Learning?}
        Reinforcement Learning (RL) is a subfield of artificial intelligence (AI) that focuses on how agents should take actions in an environment to maximize cumulative rewards.
    \end{block}
    \begin{block}{Key Characteristics}
        - Learning through trial and error \\
        - Interaction with the environment \\
        - Feedback in the form of rewards or penalties
    \end{block}
\end{frame}

\begin{frame}[fragile]
    \frametitle{Key Concepts in Reinforcement Learning}
    \begin{itemize}
        \item \textbf{Agent}: The learner or decision maker (e.g., a robot or a software program).
        \item \textbf{Environment}: Everything the agent interacts with, including the rules governing these interactions.
        \item \textbf{Action}: The choices made by the agent (e.g., moving left, jumping, etc.).
        \item \textbf{State}: A concrete representation of the environment at a given time (e.g., the position of the agent).
        \item \textbf{Reward}: A feedback signal received after an action which reinforces learning, guiding the agent towards favorable actions.
    \end{itemize}
\end{frame}

\begin{frame}[fragile]
    \frametitle{How Reinforcement Learning Works}
    \begin{enumerate}
        \item \textbf{Exploration vs. Exploitation}: The agent must balance exploring new actions and utilizing known actions that yield high rewards.
        \item \textbf{Feedback Loop}:
            \begin{itemize}
                \item The agent takes an action in a specific state.
                \item The environment responds by providing a new state and a reward.
                \item The agent updates its knowledge based on this feedback.
            \end{itemize}
    \end{enumerate}
\end{frame}

\begin{frame}[fragile]
    \frametitle{Importance of Reinforcement Learning}
    \begin{itemize}
        \item \textbf{Real-World Applications}:
            \begin{itemize}
                \item Robotics (for navigation)
                \item Gaming (e.g., AlphaGo)
                \item Finance (for trading strategies)
                \item Healthcare (for personalized treatment planning)
            \end{itemize}
        \item \textbf{Autonomous Decision-Making}:
            \begin{itemize}
                \item Enables machines to make decisions autonomously in complex environments.
                \item Advances AI systems that operate in uncertain and dynamic settings.
            \end{itemize}
    \end{itemize}
\end{frame}

\begin{frame}[fragile]
    \frametitle{Example Scenario: Training a Game-Playing Agent}
    \begin{itemize}
        \item \textbf{Environment}: A game board where various actions affect the outcome.
        \item \textbf{States}: Different configurations of pieces on the board.
        \item \textbf{Actions}: Possible moves the player can make (e.g., moving a piece, attacking).
        \item \textbf{Rewards}: Points gained for winning a round, or penalties for losing pieces.
    \end{itemize}
    The agent learns from numerous game iterations, refining its strategy based on outcomes over time.
\end{frame}

\begin{frame}[fragile]
    \frametitle{Summary}
    Reinforcement Learning is a powerful framework that mimics how humans and animals learn behaviors through interaction with their environment. 
    Understanding its fundamentals is essential for grasping the concepts that underlie many modern AI applications. 
    As we advance in this course, we'll explore the history, algorithms, and various applications of RL to deepen our understanding.
\end{frame}

\begin{frame}[fragile]
    \frametitle{History of Reinforcement Learning - Overview}
    \begin{itemize}
        \item Reinforcement Learning (RL) has evolved significantly over the last decades.
        \item History provides valuable context for current applications and advancements.
    \end{itemize}
\end{frame}

\begin{frame}[fragile]
    \frametitle{History of Reinforcement Learning - Timeline of Key Milestones}
    \begin{enumerate}
        \item \textbf{1950s - Foundational Concepts}
        \begin{itemize}
            \item 1950: Alan Turing proposes the concept of a "learning machine."
            \item 1956: John McCarthy coins "artificial intelligence" at the Dartmouth Conference.
        \end{itemize}
        
        \item \textbf{1970s - The Birth of RL}
        \begin{itemize}
            \item 1979: Richard Sutton introduces "Temporal Difference" learning.
        \end{itemize}

        \item \textbf{1980s - Early Algorithms}
        \begin{itemize}
            \item 1983: Q-learning formulated by Christopher Watkins.
            \item 1989: Sutton and Barto publish “Reinforcement Learning: An Introduction.”
        \end{itemize}

        \item \textbf{1990s - Theoretical Advancements}
        \begin{itemize}
            \item 1996: Research on policy gradient methods emerges.
            \item 1999: Barto, Sutton, and Anderson propose "intrinsic motivation."
        \end{itemize}
    \end{enumerate}
\end{frame}

\begin{frame}[fragile]
    \frametitle{History of Reinforcement Learning - Continued}
    \begin{enumerate}
        \setcounter{enumi}{4}
        \item \textbf{2000s - Integration with Neural Networks}
        \begin{itemize}
            \item 2006: "Deep Reinforcement Learning" term gains traction.
            \item 2008: RL achieves success in games (Go and Chess).
        \end{itemize}

        \item \textbf{2010s - Advanced Applications}
        \begin{itemize}
            \item 2015: Google DeepMind introduces "Deep Q-Networks" (DQN).
            \item 2016: AlphaGo defeats world champion Go player.
        \end{itemize}

        \item \textbf{2020s - Current Trends and Future Directions}
        \begin{itemize}
            \item Focus on multi-agent systems, safety, and ethical considerations.
            \item Ongoing advancements in stability, efficiency, and applications in various fields.
        \end{itemize}
    \end{enumerate}
\end{frame}

\begin{frame}[fragile]
    \frametitle{Examples and Conclusion}
    \begin{block}{Illustrative Example}
        To illustrate the concept, consider an RL agent trained to play chess:
        \begin{itemize}
            \item \textbf{State}: The current board configuration.
            \item \textbf{Action}: A possible move (e.g., moving a knight).
            \item \textbf{Reward}: Winning or losing informs learning.
        \end{itemize}
    \end{block}

    \begin{block}{Conclusion}
        The history of reinforcement learning showcases its dynamic development from theory to practical applications, emphasizing its role in AI today. Understanding milestones provides insights into contemporary RL methodologies.
    \end{block}
\end{frame}

\begin{frame}[fragile]
    \frametitle{Core Concepts of Reinforcement Learning - Introduction}
    Reinforcement Learning (RL) is a contemporary machine learning paradigm that emphasizes learning through interaction with an environment. The key concepts that underlie RL are:
    \begin{itemize}
        \item Markov Decision Processes (MDPs)
        \item Q-learning
        \item Deep Q-Networks (DQNs)
    \end{itemize}
\end{frame}

\begin{frame}[fragile]
    \frametitle{Core Concepts of Reinforcement Learning - Markov Decision Processes (MDPs)}
    \begin{block}{Definition}
        A mathematical framework for modeling decision-making in situations where the outcomes are partly random and partly under the control of a decision-maker.
    \end{block}
    \begin{itemize}
        \item \textbf{Components}:
        \begin{itemize}
            \item \textbf{States (S)}: All possible situations in which an agent can find itself.
            \item \textbf{Actions (A)}: Choices available to the agent in each state.
            \item \textbf{Transition Function (P)}: Probability distribution of reaching the next state given the current state and action, $P(s'|s, a)$.
            \item \textbf{Reward Function (R)}: Immediate return received after transitioning from state $s$ to state $s'$ via action $a$, $R(s, a, s')$.
            \item \textbf{Discount Factor ($\gamma$)}: A factor (0 ≤ $\gamma$ < 1) that weighs immediate rewards higher than future rewards.
        \end{itemize}
        \item \textbf{Example}: A simple MDP could be a grid world where an agent must navigate to a goal while avoiding obstacles.
    \end{itemize}
\end{frame}

\begin{frame}[fragile]
    \frametitle{Core Concepts of Reinforcement Learning - Q-Learning}
    \begin{block}{Definition}
        A model-free RL algorithm that enables an agent to learn the optimal action-selection policy by interacting with the environment and using Q-values.
    \end{block}
    \begin{itemize}
        \item \textbf{Key Formula}:
        \begin{equation}
            Q(s, a) \leftarrow Q(s, a) + \alpha \left( R + \gamma \max_{a'} Q(s', a') - Q(s, a) \right)
        \end{equation}
        Where:
        \begin{itemize}
            \item $s$ = current state
            \item $a$ = current action
            \item $s'$ = next state
            \item $\alpha$ = learning rate
        \end{itemize}
        \item \textbf{Example}: In a video game, Q-learning could help an agent learn strategies for defeating opponents based on the states (health, weapon status) and actions (attack, defend, flee).
    \end{itemize}
\end{frame}

\begin{frame}[fragile]
    \frametitle{Core Concepts of Reinforcement Learning - Deep Q-Networks (DQNs)}
    \begin{block}{Definition}
        An extension of Q-learning that utilizes deep neural networks to approximate Q-values.
    \end{block}
    \begin{itemize}
        \item \textbf{Architecture}: DQNs typically use a convolutional neural network (CNN) to process input states and output Q-values.
        \item \textbf{Key Innovations}:
        \begin{itemize}
            \item \textbf{Experience Replay}: Stores experiences to improve stability.
            \item \textbf{Target Network}: A separate network to stabilize updates.
        \end{itemize}
        \item \textbf{Example}: DQNs were famously applied in playing Atari games, where the agent learns directly from pixels.
    \end{itemize}
\end{frame}

\begin{frame}[fragile]
    \frametitle{Core Concepts of Reinforcement Learning - Key Points}
    \begin{itemize}
        \item \textbf{Integration}: Understanding MDPs is fundamental as they form the backbone of both Q-learning and DQNs.
        \item \textbf{Practical Application}: Q-learning is a stepping stone to more complex algorithms like DQNs for real-world applications.
        \item \textbf{Learning Process}: RL involves balancing exploration vs. exploitation, where agents try new actions while utilizing known rewarding ones.
    \end{itemize}
    With these core concepts, we build a foundation to explore details in the following slides.
\end{frame}

\begin{frame}[fragile]
    \frametitle{Q-Learning - Introduction}
    \begin{block}{Overview}
        Q-Learning is a model-free reinforcement learning algorithm that focuses on learning the value of actions in states to maximize expected future rewards. It achieves this through iterative updates to the Q-value function.
    \end{block}
\end{frame}

\begin{frame}[fragile]
    \frametitle{Q-Learning - Key Components}
    \begin{enumerate}
        \item \textbf{Q-Value Function (Q)}: Represents expected utility of taking an action in a state.
        \item \textbf{States (s)}: Various situations the agent can occupy.
        \item \textbf{Actions (a)}: Possible choices for the agent in each state.
        \item \textbf{Reward (R)}: Feedback signal from the environment regarding the action taken.
        \item \textbf{Policy ($\pi$)}: Strategy mapping states to actions.
    \end{enumerate}
\end{frame}

\begin{frame}[fragile]
    \frametitle{Q-Learning - Algorithm Steps}
    \begin{enumerate}
        \item Initialize the Q-value table arbitrarily (often set to zero).
        \item Observe the current state $s$.
        \item Choose an action using an exploration strategy (e.g., epsilon-greedy).
        \item Take action $a$, observe reward $R$ and the next state $s'$.
        \item Update Q-value:
            \begin{equation}
            Q(s, a) \leftarrow Q(s, a) + \alpha \left( R + \gamma \max_{a'} Q(s', a') - Q(s, a) \right)
            \end{equation}
        \item Repeat until convergence or for a specified number of episodes.
    \end{enumerate}
\end{frame}

\begin{frame}[fragile]
    \frametitle{Q-Learning - Example in a Grid Environment}
    \begin{block}{Scenario}
        Imagine a 3x3 grid where the agent starts at (0, 0) seeking to reach (2, 2).\\
        \textbf{States:} Cells in the grid.\\
        \textbf{Actions:} Up, Down, Left, Right.\\
        \textbf{Rewards:} +10 for reaching (2, 2), -1 for each move.
    \end{block}
\end{frame}

\begin{frame}[fragile]
    \frametitle{Q-Learning - Real-World Applications}
    \begin{itemize}
        \item \textbf{Game Playing:} Training AI agents for games such as Chess or Go.
        \item \textbf{Robotics:} Navigation and learning control strategies for robots.
        \item \textbf{Finance:} Optimizing investment decisions and algorithmic trading.
        \item \textbf{Recommendation Systems:} Personalizing content by learning user preferences.
    \end{itemize}
\end{frame}

\begin{frame}[fragile]
    \frametitle{Q-Learning - Key Points to Emphasize}
    \begin{itemize}
        \item Q-Learning is model-free and adapts dynamically to changes.
        \item Importance of balancing exploration and exploitation.
        \item Can be extended using function approximation (e.g., Deep Q-Networks) for larger state-action spaces.
    \end{itemize}
\end{frame}

\begin{frame}[fragile]
    \frametitle{Deep Q-Networks - Introduction}
    \begin{itemize}
        \item Deep Q-Networks (DQNs) are an advancement over traditional Q-learning methods.
        \item They utilize deep learning to enable agents to make decisions in complex environments.
        \item DQNs approximate the Q-value function using neural networks.
    \end{itemize}
\end{frame}

\begin{frame}[fragile]
    \frametitle{Deep Q-Networks - Core Concepts}
    \begin{enumerate}
        \item \textbf{Q-Learning Review:}
        \begin{itemize}
            \item Q-learning is a model-free reinforcement learning algorithm.
            \item Estimates the value of future rewards based on current state and action.
            \item Q-value update using the Bellman equation:
            \begin{equation}
                Q(s, a) \leftarrow Q(s, a) + \alpha \left[ r + \gamma \max_{a'} Q(s', a') - Q(s, a) \right]
            \end{equation}
            where:
            \begin{itemize}
                \item \( s \): current state
                \item \( a \): current action
                \item \( r \): reward received
                \item \( s' \): next state
                \item \( \alpha \): learning rate
                \item \( \gamma \): discount factor
            \end{itemize}
        \end{itemize}
        
        \item \textbf{Limitations of Traditional Q-Learning:}
        \begin{itemize}
            \item Struggles with high-dimensional state spaces.
            \item Requires a well-defined Q-value table, impractical for complex problems.
        \end{itemize}
    \end{enumerate}
\end{frame}

\begin{frame}[fragile]
    \frametitle{Deep Q-Networks - Architecture}
    \begin{itemize}
        \item \textbf{Neural Network as Function Approximator:}
        \begin{itemize}
            \item A neural network approximates the Q-value function \( Q(s, a; \theta) \).
            \item Network parameters \( \theta \) are updated based on observed rewards.
        \end{itemize}
        
        \item \textbf{Key Components of DQN:}
        \begin{itemize}
            \item \textbf{Experience Replay:} 
            \begin{itemize}
                \item Stores experiences in a replay buffer.
                \item Samples mini-batches to break correlation between experiences.
            \end{itemize}

            \item \textbf{Target Network:}
            \begin{itemize}
                \item A separate network for stable Q-value targets.
                \item Weights updated less frequently for training stability.
            \end{itemize}
        \end{itemize}
    \end{itemize}
\end{frame}

\begin{frame}[fragile]
    \frametitle{Markov Decision Processes}
    \begin{block}{Understanding Markov Decision Processes}
        A Markov Decision Process (MDP) is a mathematical framework used to describe an environment in reinforcement learning. It models decision-making with outcomes that are partly random and partly under the control of a decision-maker.
    \end{block}
\end{frame}

\begin{frame}[fragile]
    \frametitle{Key Components of MDP}
    \begin{enumerate}
        \item \textbf{States (S)}:
        \begin{itemize}
            \item Definition: Represents possible situations the agent can be in.
            \item Example: In chess, each possible board configuration is a unique state.
        \end{itemize}
        
        \item \textbf{Actions (A)}:
        \begin{itemize}
            \item Definition: Choices available to the agent in each state.
            \item Example: Moving a pawn or a knight in chess.
        \end{itemize}
        
        \item \textbf{Rewards (R)}:
        \begin{itemize}
            \item Definition: Immediate benefits received after taking an action.
            \item Example: Capturing a piece may yield +10; losing a piece results in -10.
        \end{itemize}
        
        \item \textbf{Policies (𝜋)}:
        \begin{itemize}
            \item Definition: Strategy defining actions taken in each state.
            \item Example: “Always capture pieces when possible” could be a policy.
        \end{itemize}
    \end{enumerate}
\end{frame}

\begin{frame}[fragile]
    \frametitle{Transition Dynamics and MDP Framework}
    \begin{block}{Transition Probability (P)}
        Represents the probability of moving from one state to another given an action. Notation: 
        \[
        P(s'|s, a) = P(\text{transition to } s' \text{ after action } a \text{ in state } s)
        \]
    \end{block}

    \begin{block}{MDP Definition}
        An MDP is defined as a tuple:
        \[
        (S, A, R, P, \gamma)
        \]
        where:
        \begin{itemize}
            \item \( S \): Set of states
            \item \( A \): Set of actions
            \item \( R \): Reward function \( R: S \times A \rightarrow \mathbb{R} \)
            \item \( P \): State transition probabilities \( P: S \times A \times S \rightarrow [0,1] \)
            \item \( \gamma \): Discount factor (0 < \( \gamma \) < 1) for balancing immediate and future rewards.
        \end{itemize}
    \end{block}
\end{frame}

\begin{frame}[fragile]
    \frametitle{Importance of MDPs in Reinforcement Learning}
    \begin{itemize}
        \item \textbf{Decision Making}: Provides a structure for decision-making over time based on probabilistic outcomes.
        \item \textbf{Optimal Policies}: Enables finding the best policy \( \pi^* \) that maximizes expected rewards.
        \item \textbf{Applications}: Foundational in fields like robotics and finance where sequential decision-making is crucial.
    \end{itemize}
\end{frame}

\begin{frame}[fragile]
    \frametitle{Example Illustration}
    \begin{block}{Robot Navigation in a Grid}
        \begin{itemize}
            \item \textbf{States (S)}: Each cell in the grid.
            \item \textbf{Actions (A)}: Move up, down, left, right.
            \item \textbf{Rewards (R)}: +10 for reaching the goal, -1 for hitting a wall.
            \item \textbf{Policy (π)}: Decides the direction to move based on the current state.
        \end{itemize}
    \end{block}
    This example visualizes the application of MDP concepts in practice, highlighting the importance of states, actions, rewards, and policies.
\end{frame}

\begin{frame}[fragile]
    \frametitle{Probability Theory in RL - Overview}
    \begin{block}{Overview}
        Understanding essential probability theory concepts is crucial for grasping reinforcement learning algorithms. 
        Probability underpins how agents make decisions based on experiences and influences learning from environments.
    \end{block}
\end{frame}

\begin{frame}[fragile]
    \frametitle{Probability Theory in RL - Key Concepts}
    \begin{enumerate}
        \item \textbf{Random Variables:}
            \begin{itemize}
                \item Definition: A variable whose possible values are outcomes of a random phenomenon.
                \item Example: In a game, the score can be a random variable influenced by the strategies employed by an agent.
            \end{itemize}
        \item \textbf{Probability Distributions:}
            \begin{itemize}
                \item Definition: A function that describes the likelihood of obtaining the possible values that a random variable can take.
                \item Common Types:
                    \begin{itemize}
                        \item Discrete: E.g., Bernoulli, Binomial distributions (used for reward signals).
                        \item Continuous: E.g., Normal distribution (used for approximating state values).
                    \end{itemize}
                \item Example: An agent's possible rewards can be modeled as a discrete probability distribution.
            \end{itemize}
    \end{enumerate}
\end{frame}

\begin{frame}[fragile]
    \frametitle{Probability Theory in RL - Key Concepts Cont.}
    \begin{enumerate}
        \setcounter{enumi}{2} % Continue from previous enumeration
        \item \textbf{Expected Value:}
            \begin{itemize}
                \item Definition: The average outcome considering all possible outcomes weighted by their probabilities.
                \item Formula: 
                    \[
                    E[X] = \sum (x_i \cdot P(x_i))
                    \]
                \item Example: A 70\% chance of earning \$10 and a 30\% chance of earning \$0 results in:
                    \[
                    E[X] = (10 \cdot 0.7) + (0 \cdot 0.3) = 7
                    \]
            \end{itemize}
        \item \textbf{Markov Property:}
            \begin{itemize}
                \item Definition: The future states depend only on the current state and action.
                \item Importance: Supports Markov Decision Processes (MDPs) central to RL algorithms.
                \item Example: A robot's next position is determined by its current position and action.
            \end{itemize}
    \end{enumerate}
\end{frame}

\begin{frame}[fragile]
    \frametitle{Probability Theory in RL - Key Concepts Cont.}
    \begin{enumerate}
        \setcounter{enumi}{4} % Continue the enumeration
        \item \textbf{Bayes' Theorem:}
            \begin{itemize}
                \item Definition: Updates the probability of a hypothesis based on new evidence.
                \item Formula:
                    \[
                    P(A|B) = \frac{P(B|A) \cdot P(A)}{P(B)}
                    \]
                \item Application in RL: Used in decision-making where prior knowledge is updated.
            \end{itemize}
        \item \textbf{Exploration vs. Exploitation:}
            \begin{itemize}
                \item Concept: Balancing the need to try new actions (exploration) vs. leveraging known rewarding actions (exploitation).
                \item Example: An agent must decide whether to explore a new strategy or exploit a previously successful one.
            \end{itemize}
    \end{enumerate}
\end{frame}

\begin{frame}[fragile]
    \frametitle{Conclusion}
    \begin{block}{Key Points to Emphasize}
        \begin{itemize}
            \item Probabilistic Models: Many RL algorithms rely on probabilistic models of environments.
            \item Decision Making: Understanding probabilities helps agents make informed decisions about future states and rewards.
        \end{itemize}
    \end{block}
    \begin{block}{Final Thoughts}
        Mastery of probability theory is essential for understanding the foundations of reinforcement learning algorithms 
        and how agents interact with complex environments, improving strategies over time.
    \end{block}
\end{frame}

\begin{frame}[fragile]
    \frametitle{Linear Algebra in Reinforcement Learning}
    \begin{block}{Introduction to Key Linear Algebra Concepts}
        Linear algebra is essential for grasping the mechanics of reinforcement learning (RL) as it deals with vectors, matrices, and linear transformations. These concepts are vital for representing high-dimensional data, states, actions, and value functions in RL.
    \end{block}
\end{frame}

\begin{frame}[fragile]
    \frametitle{Core Concepts - Part 1}
    \begin{enumerate}
        \item \textbf{Vectors:}
            \begin{itemize}
                \item A vector is a one-dimensional array of numbers.
                \item Example: A state in a grid world represented as \(s = [x, y]\).
            \end{itemize}
        
        \item \textbf{Matrices:}
            \begin{itemize}
                \item A matrix is a two-dimensional array of numbers.
                \item Example: Transition probabilities in RL can be represented as a matrix \(T(i, j)\).
            \end{itemize}
    \end{enumerate}
\end{frame}

\begin{frame}[fragile]
    \frametitle{Core Concepts - Part 2}
    \begin{enumerate}[resume]
        \item \textbf{Dot Product:}
            \begin{itemize}
                \item Quantifies similarity between two vectors.
                \item Example: For vectors \(a = [a_1, a_2]\) and \(b = [b_1, b_2]\),
                \[
                a \cdot b = a_1 \times b_1 + a_2 \times b_2.
                \]
            \end{itemize}

        \item \textbf{Linear Combinations:}
            \begin{itemize}
                \item Involves scaling vectors and adding them together.
                \item Fundamental for forming policies in RL.
            \end{itemize}

        \item \textbf{Matrix Multiplication:}
            \begin{itemize}
                \item Essential for transforming vectors and combining matrices.
                \item Example: \(A \cdot w\) represents outcomes of applying weights to features.
            \end{itemize}
    \end{enumerate}
\end{frame}

\begin{frame}[fragile]
    \frametitle{Key Points and Example Scenario}
    \begin{block}{Key Points to Emphasize}
        \begin{itemize}
            \item RL often involves high-dimensional states and actions, making linear algebra crucial.
            \item Linear mappings are essential for modeling changes and rewards.
            \item Many RL algorithms rely on matrices for storing and updating value functions.
        \end{itemize}
    \end{block}

    \begin{block}{Example Scenario: Q-Learning}
        The action-value function \(Q(s, a)\) can be represented using a matrix \(Q\). A basic update rule is:
        \[
        Q(s, a) \leftarrow Q(s, a) + \alpha \left( r + \gamma \max_{a'} Q(s', a') - Q(s, a) \right),
        \]
        highlighting the role of linear algebra in RL.
    \end{block}
\end{frame}

\begin{frame}[fragile]
    \frametitle{Conclusion}
    Mastering linear algebra is crucial for understanding the mathematical foundation of RL. These concepts will enhance your ability to design, analyze, and optimize complex systems effectively.
\end{frame}

\begin{frame}[fragile]
    \frametitle{Case Study: Real-World Applications}
    \begin{block}{Introduction to Reinforcement Learning (RL)}
        Reinforcement Learning is a subset of machine learning where an agent learns to make decisions by taking actions in an environment. The agent receives feedback in the form of rewards or penalties, enabling it to learn optimal behaviors over time.
    \end{block}
\end{frame}

\begin{frame}[fragile]
    \frametitle{Key Real-World Applications of Reinforcement Learning}
    \begin{enumerate}
        \item \textbf{Robotics}
            \begin{itemize}
                \item \textbf{Example: Robotic Manipulation}
                \begin{itemize}
                    \item RL is utilized in robotic arms to learn how to grasp and manipulate objects.
                    \item A robotic system trained with RL can adapt its actions to improve object manipulation through trial and error.
                \end{itemize}
            \end{itemize}
        \item \textbf{Gaming}
            \begin{itemize}
                \item \textbf{Example: AlphaGo}
                \begin{itemize}
                    \item Developed by DeepMind, AlphaGo utilized RL to master the game of Go, defeating world champions.
                    \item It learned complex strategies by playing against itself millions of times.
                \end{itemize}
            \end{itemize}
        \item \textbf{Autonomous Vehicles}
            \begin{itemize}
                \item \textbf{Example: Self-Driving Cars}
                \begin{itemize}
                    \item RL algorithms are critical in driving policies, allowing vehicles to navigate, avoid obstacles, and follow traffic rules.
                \end{itemize}
            \end{itemize}
    \end{enumerate}
\end{frame}

\begin{frame}[fragile]
    \frametitle{Continued Applications of Reinforcement Learning}
    \begin{enumerate}
        \setcounter{enumi}{3} % Start enumeration from 4
        \item \textbf{Healthcare}
            \begin{itemize}
                \item \textbf{Example: Personalized Treatment Plans}
                \begin{itemize}
                    \item RL can optimize treatment protocols by tailoring medication dosages based on patient feedback over time.
                \end{itemize}
            \end{itemize}
        \item \textbf{Finance}
            \begin{itemize}
                \item \textbf{Example: Algorithmic Trading}
                \begin{itemize}
                    \item Financial institutions implement RL to develop trading strategies that adapt to market changes, maximizing returns by learning from past trades.
                \end{itemize}
            \end{itemize}
    \end{enumerate}
\end{frame}

\begin{frame}[fragile]
    \frametitle{Conclusion and References}
    \begin{block}{Conclusion}
        Reinforcement Learning is transforming various industries by providing adaptive and efficient solutions. Understanding these applications showcases the power of RL and inspires innovative approaches to problem-solving in real-world scenarios.
    \end{block}

    \begin{block}{References}
        \begin{itemize}
            \item Sutton, R. S., \& Barto, A. G. (2018). \textit{Reinforcement Learning: An Introduction}. MIT Press.
            \item Silver, D. et al. (2016). \textit{Mastering the game of Go with deep neural networks and tree search}. Nature, 529, 484-489.
        \end{itemize}
    \end{block}
\end{frame}

\begin{frame}[fragile]
    \frametitle{Ethical Considerations in RL - Introduction}
    \begin{block}{Introduction to Ethical Implications in Reinforcement Learning (RL)}
    Reinforcement Learning (RL) possesses transformative potential in sectors like healthcare, finance, and autonomous systems. 
    While these applications can enhance productivity and effectiveness, significant ethical considerations also arise. 
    Understanding these implications is crucial for responsible deployment.
    \end{block}
\end{frame}

\begin{frame}[fragile]
    \frametitle{Ethical Considerations in RL - Key Points}
    \begin{block}{Key Ethical Considerations}
        \begin{enumerate}
            \item \textbf{Bias and Fairness}
                \begin{itemize}
                    \item RL algorithms learn from historical data, which may contain biases.
                    \item Example: An RL model trained on biased hiring data may favor certain demographics.
                    \item Implication: Implement fairness-aware learning techniques to mitigate biases.
                \end{itemize}
          
            \item \textbf{Safety and Control}
                \begin{itemize}
                    \item RL agents must operate safely in complex and dynamic environments.
                    \item Example: An RL agent optimizing for speed might perform risky maneuvers in self-driving cars.
                    \item Implication: Develop robust safety protocols and ensure human oversight.
                \end{itemize}
        \end{enumerate}
    \end{block}
\end{frame}

\begin{frame}[fragile]
    \frametitle{Ethical Considerations in RL - Continued}
    \begin{block}{Key Ethical Considerations (Continued)}
        \begin{enumerate}
            \setcounter{enumi}{2} % continue enumeration from previous frame
            \item \textbf{Transparency and Explainability}
                \begin{itemize}
                    \item RL models can act as "black boxes," complicating understanding of their decisions.
                    \item Example: An RL system in healthcare must explain its treatment recommendations to gain trust.
                    \item Implication: Strive for explainable AI to improve trust and facilitate interactions.
                \end{itemize}

            \item \textbf{Accountability}
                \begin{itemize}
                    \item The complexity of assigning accountability when RL makes autonomous decisions.
                    \item Example: If a delivery drone causes damage, accountability can be unclear.
                    \item Implication: Establish clear accountability frameworks for ethical governance.
                \end{itemize}

            \item \textbf{Long-term Impact on Society}
                \begin{itemize}
                    \item RL-driven decision-making can lead to societal changes requiring management.
                    \item Example: Job automation through RL may improve efficiency but lead to job displacement.
                    \item Implication: Consider societal implications and workforce transitions.
                \end{itemize}
        \end{enumerate}
    \end{block}

\end{frame}

\begin{frame}[fragile]
    \frametitle{Ethical Considerations in RL - Conclusion}
    \begin{block}{Conclusion}
    Responsible deployment of RL technologies depends on addressing ethical concerns. 
    Awareness and incorporation of these considerations into design and application can leverage RL's power while ensuring fairness, safety, and societal well-being.
    \end{block}

    \begin{block}{Key Points to Emphasize}
        \begin{itemize}
            \item Understanding and mitigating bias is crucial for fair outcomes in RL.
            \item Safety protocols must prevent harmful actions by RL agents.
            \item Transparency and explainability improve user trust and accountability.
            \item Proactively managing RL's ethical implications avoids adverse societal impacts.
        \end{itemize}
    \end{block}
\end{frame}

\begin{frame}[fragile]
    \frametitle{Engagement with Current Research}
    % Overview of recent research trends in reinforcement learning and their implications.
    Understanding recent trends in Reinforcement Learning (RL) is crucial for positioning future research effectively.
\end{frame}

\begin{frame}[fragile]
    \frametitle{Overview of Recent Trends in Reinforcement Learning}
    \begin{itemize}
        \item Model-Free vs. Model-Based Learning
        \begin{itemize}
            \item Model-Free Methods: Q-learning, Policy Gradient
            \item Model-Based Methods: Learning models of the environment (e.g., AlphaGo, Dreamer)
            \item Implications: Integrating both paradigms for sample-efficient learning.
        \end{itemize}

        \item Hierarchical Reinforcement Learning (HRL)
        \begin{itemize}
            \item Breaks down tasks into sub-tasks for complex behavior learning.
            \item Example: Robot navigating a maze - learn to reach a doorway first.
            \item Future Directions: Training hierarchical policies for faster learning.
        \end{itemize}
    \end{itemize}
\end{frame}

\begin{frame}[fragile]
    \frametitle{Key Trends and Takeaways in RL}
    \begin{itemize}
        \item Multi-Agent Reinforcement Learning (MARL)
        \begin{itemize}
            \item Involves multiple agents; dynamics of cooperation/competition.
            \item Example: StarCraft II games.
            \item Future Implications: Research on communication strategies among agents.
        \end{itemize}

        \item Exploration Strategies
        \begin{itemize}
            \item Curiosity-driven and goal-based exploration.
            \item Example: Exploration bonuses for novel states.
            \item Implications: Better algorithms for real-world exploration.
        \end{itemize}

        \item Transfer Learning in RL
        \begin{itemize}
            \item Transfers knowledge across tasks to reduce training time.
            \item Example: Learning simple tasks facilitates tackling complex versions.
        \end{itemize}
        
        \item Key Takeaway: Reinforcement Learning is evolving, focusing on integrating methodologies and improving applications.
    \end{itemize}
\end{frame}

\begin{frame}[fragile]
    \frametitle{Conclusion and Future Directions - Part 1}
    
    \begin{block}{Key Learnings from Week 1}
        \begin{enumerate}
            \item \textbf{Definition of Reinforcement Learning (RL)}:
            \begin{itemize}
                \item RL is a subset of machine learning where an agent learns to make decisions by performing actions in an environment to maximize cumulative rewards.
                \item \textbf{Components of RL}:
                    \begin{itemize}
                        \item \textbf{Agent}: The learner or decision maker.
                        \item \textbf{Environment}: Everything the agent interacts with.
                        \item \textbf{Actions}: Choices the agent can make.
                        \item \textbf{Rewards}: Feedback from the environment based on the agent's actions.
                    \end{itemize}
            \end{itemize}
            
            \item \textbf{Core Concepts}:
            \begin{itemize}
                \item \textbf{Exploration vs. Exploitation}: Balancing new actions (exploration) against known high-reward actions (exploitation).
                \item \textbf{Value Functions}: Estimate expected return from each state or action.
                \item \textbf{Policies}: Strategies defining the agent's actions based on the current state.
            \end{itemize}
        \end{enumerate}
    \end{block}
\end{frame}

\begin{frame}[fragile]
    \frametitle{Conclusion and Future Directions - Part 2}

    \begin{block}{Learning Mechanisms}
        \begin{itemize}
            \item \textbf{Q-Learning}: A model-free RL algorithm learning the value of taking a certain action in a given state.
            \item \textbf{Temporal-Difference Learning}: Blends Monte Carlo ideas and dynamic programming, crucial for learning from incomplete episodes.
        \end{itemize}
    \end{block}

    \begin{block}{Future Directions for Research}
        \begin{enumerate}
            \item \textbf{Scalability and Efficiency}
            \item \textbf{Integration with Other Learning Paradigms}
            \item \textbf{Ethical Considerations and Safety}
            \item \textbf{Application-Specific Developments}
            \item \textbf{Theoretical Advances}
        \end{enumerate}
    \end{block}
\end{frame}

\begin{frame}[fragile]
    \frametitle{Conclusion and Future Directions - Part 3}
    
    \begin{block}{Key Points to Emphasize}
        \begin{itemize}
            \item RL is a powerful paradigm with vast application potential but faces challenges in scalability, efficiency, and ethical implications.
            \item Collaborative approaches between different learning methodologies could pave the way for more robust RL systems.
            \item Ongoing RL research has the potential to reshape many sectors, guided by ethical standards and safety protocols.
        \end{itemize}
    \end{block}

    \begin{block}{Summary}
        As we wrap up Week 1, it is essential to recognize foundational concepts of reinforcement learning while remaining aware of numerous future research directions. The synergy of ongoing advancements and ethical considerations will be crucial in harnessing the full potential of RL technologies moving forward.
    \end{block}
\end{frame}


\end{document}