\documentclass[aspectratio=169]{beamer}

% Theme and Color Setup
\usetheme{Madrid}
\usecolortheme{whale}
\useinnertheme{rectangles}
\useoutertheme{miniframes}

% Additional Packages
\usepackage[utf8]{inputenc}
\usepackage[T1]{fontenc}
\usepackage{graphicx}
\usepackage{booktabs}
\usepackage{listings}
\usepackage{amsmath}
\usepackage{amssymb}
\usepackage{xcolor}
\usepackage{tikz}
\usepackage{pgfplots}
\pgfplotsset{compat=1.18}
\usetikzlibrary{positioning}
\usepackage{hyperref}

% Custom Colors
\definecolor{myblue}{RGB}{31, 73, 125}
\definecolor{mygray}{RGB}{100, 100, 100}
\definecolor{mygreen}{RGB}{0, 128, 0}
\definecolor{myorange}{RGB}{230, 126, 34}
\definecolor{mycodebackground}{RGB}{245, 245, 245}

% Set Theme Colors
\setbeamercolor{structure}{fg=myblue}
\setbeamercolor{frametitle}{fg=white, bg=myblue}
\setbeamercolor{title}{fg=myblue}
\setbeamercolor{section in toc}{fg=myblue}
\setbeamercolor{item projected}{fg=white, bg=myblue}
\setbeamercolor{block title}{bg=myblue!20, fg=myblue}
\setbeamercolor{block body}{bg=myblue!10}
\setbeamercolor{alerted text}{fg=myorange}

% Set Fonts
\setbeamerfont{title}{size=\Large, series=\bfseries}
\setbeamerfont{frametitle}{size=\large, series=\bfseries}
\setbeamerfont{caption}{size=\small}
\setbeamerfont{footnote}{size=\tiny}

% Title Page Information
\title[Final Project Presentations]{Week 14: Final Project Presentations}
\author[J. Smith]{John Smith, Ph.D.}
\institute[University Name]{
  Department of Computer Science\\
  University Name\\
  \vspace{0.3cm}
  Email: email@university.edu\\
  Website: www.university.edu
}
\date{\today}

% Document Start
\begin{document}

\frame{\titlepage}

\begin{frame}[fragile]
    \frametitle{Introduction to Final Project Presentations}
    \begin{block}{Overview of Final Project}
        As we approach the final week of our course, we focus on the culmination of our learning: the final project presentations. This project is a critical assessment of your ability to implement and refine algorithms, particularly in the context of machine learning and Reinforcement Learning (RL).
    \end{block}
\end{frame}

\begin{frame}[fragile]
    \frametitle{Key Concepts}
    \begin{enumerate}
        \item \textbf{Algorithm Implementation}:
        \begin{itemize}
            \item Turning theoretical understanding of algorithms into practical code.
            \item Key tasks: selecting the right algorithm, coding, and integration within chosen platforms (e.g., Python, TensorFlow).
            \item \textbf{Example:} If using a Q-learning algorithm, prepare a code snippet for the action-value function:
            \end{itemize}
            \begin{lstlisting}[language=Python]
# Sample Python Code for Q-learning Update
Q[state, action] += alpha * (reward + gamma * max(Q[new_state, a]) - Q[state, action])
            \end{lstlisting}
    \end{enumerate}
\end{frame}

\begin{frame}[fragile]
    \frametitle{Key Concepts (Continued)}
    \begin{enumerate}
        \setcounter{enumi}{1} % Start from the second item
        \item \textbf{Algorithm Refinement}:
        \begin{itemize}
            \item Enhancing performance post-implementation by tweaking hyperparameters (e.g., learning rates).
            \item \textbf{Illustration:} To improve the learning speed of an RL agent, you might increase the learning rate for faster convergence.
        \end{itemize}
    \end{enumerate}

    \begin{block}{Objectives for Students During Presentations}
        \begin{itemize}
            \item \textbf{Demonstrate Understanding}: Articulate algorithm choice, rationale, and performance.
            \item \textbf{Showcase Results}: Present empirical evidence like accuracy and convergence rates.
            \item \textbf{Discuss Challenges}: Reflect on obstacles and refinement impacts.
        \end{itemize}
    \end{block}
\end{frame}

\begin{frame}[fragile]
    \frametitle{Key Points to Emphasize}
    \begin{itemize}
        \item \textbf{Clarity and Organization}: Use clear headings, bullet points, and logical flow to guide your audience.
        \item \textbf{Engagement}: Foster a collaborative atmosphere by inviting questions and discussions.
        \item \textbf{Practice}: Rehearse to ensure smooth delivery within the time limit.
    \end{itemize}

    \begin{block}{Conclusion}
        Focus on algorithm implementation and refinement to showcase your technical skills and apply knowledge in practical scenarios. Communicate effectively and creatively in your final presentations!
    \end{block}
\end{frame}

\begin{frame}[fragile]
    \frametitle{Project Objectives - Introduction}
    The final project is a culmination of your learning journey throughout the course. By engaging with the project, you will accomplish several key objectives that are crucial to mastering Reinforcement Learning (RL) concepts, evaluating model performance, and understanding the ethical dimensions of your work.
\end{frame}

\begin{frame}[fragile]
    \frametitle{Project Objectives - Mastering RL Concepts}
    \begin{block}{Key Objectives}
        \begin{enumerate}
            \item \textbf{Mastering Reinforcement Learning Concepts}
            \begin{itemize}
                \item \textbf{Understanding RL Framework:}
                \begin{itemize}
                    \item \textbf{Agent:} The learner or decision-maker.
                    \item \textbf{Environment:} The context within which the agent operates.
                    \item \textbf{Actions:} The set of all possible moves the agent can make.
                    \item \textbf{States:} Different situations the agent can encounter.
                    \item \textbf{Rewards:} Feedback from the environment used to gauge the success of actions.
                \end{itemize}
                \item \textbf{Application of Algorithms:} Implement and refine key algorithms like Q-learning, Policy Gradient, and DQN.
                \begin{itemize}
                    \item \textbf{Example:} Implementing a Q-learning algorithm to train an agent to play Tic-Tac-Toe.
                \end{itemize}
            \end{itemize}
        \end{enumerate}
    \end{block}
\end{frame}

\begin{frame}[fragile]
    \frametitle{Project Objectives - Performance Metrics and Ethics}
    \begin{block}{Key Objectives (Continued)}
        \begin{enumerate}
            \setcounter{enumi}{1}
            \item \textbf{Performance Metrics}
            \begin{itemize}
                \item \textbf{Evaluating Success:} Understanding performance metrics essential for assessing RL algorithms.
                \begin{itemize}
                    \item \textbf{Cumulative Reward:} Total reward received over a period.
                    \item \textbf{Convergence Time:} How quickly the agent learns.
                    \item \textbf{Stability:} Consistency of the policy over time.
                \end{itemize}
                \item \textbf{Illustration:} A graph showing cumulative rewards over episodes.
            \end{itemize}
            
            \item \textbf{Ethical Considerations}
            \begin{itemize}
                \item \textbf{Responsible AI:} Implications of RL solutions.
                \item \textbf{Key Points:}
                \begin{itemize}
                    \item \textbf{Bias and Fairness:} Training data can introduce biases.
                    \item \textbf{Transparency:} Importance of clear decision-making processes.
                    \item \textbf{Impact Assessment:} Evaluating societal impacts before deployment.
                \end{itemize}
            \end{itemize}
        \end{enumerate}
    \end{block}
\end{frame}

\begin{frame}[fragile]
    \frametitle{Project Objectives - Conclusion}
    Through this final project, you will:
    \begin{itemize}
        \item Apply theoretical knowledge.
        \item Develop critical thinking around performance evaluation and ethical considerations.
    \end{itemize}
    By achieving these objectives, you will be well-equipped to tackle real-world problems using Reinforcement Learning effectively and responsibly.

    \begin{block}{Summary of Objectives}
        \begin{itemize}
            \item Master core RL concepts through practical application.
            \item Evaluate model performance with key metrics.
            \item Understand and address ethical implications of RL technologies.
        \end{itemize}
    \end{block}
\end{frame}

\begin{frame}[fragile]
  \frametitle{Outline of the Final Project - Overview}
  The final project serves as a culmination of the concepts learned throughout the course. It includes the following core components:
  \begin{enumerate}
    \item Project Proposal
    \item Progress Report
    \item Final Deliverable
  \end{enumerate}
\end{frame}

\begin{frame}[fragile]
  \frametitle{Outline of the Final Project - Project Proposal}
  \begin{block}{Definition}
    A document that outlines your project idea, objectives, and methodology. It serves as the foundation for your project, guiding your research and development process.
  \end{block}

  \begin{block}{Key Elements to Include}
    \begin{itemize}
      \item Project Title: A concise and descriptive title.
      \item Introduction: Describe the problem your project addresses.
      \item Objectives: Outline what you aim to achieve.
      \item Methodology: Detail the approach or algorithms you will use.
      \item Expected Outcomes: What do you hope to learn or prove?
      \item Ethical Considerations: Address any potential ethical implications.
    \end{itemize}
  \end{block}

  \begin{block}{Example}
    "Optimizing Warehouse Logistics Using Reinforcement Learning," where you outline the use of Q-learning to reduce retrieval times.
  \end{block}
\end{frame}

\begin{frame}[fragile]
  \frametitle{Outline of the Final Project - Progress Report}
  \begin{block}{Definition}
    A checkpoint that summarizes your project's current status, challenges encountered, and any changes to the original plan.
  \end{block}

  \begin{block}{Key Elements to Include}
    \begin{itemize}
      \item Summary of Work Completed: What tasks have you finished?
      \item Challenges Faced: Unforeseen issues encountered.
      \item Revised Timeline: Adjust deadlines as necessary.
      \item Next Steps: Upcoming tasks to keep the project on track.
    \end{itemize}
  \end{block}
  
  \begin{block}{Example}
    Indicate that data collection took longer than expected, but preliminary models show promising results.
  \end{block}
\end{frame}

\begin{frame}[fragile]
  \frametitle{Outline of the Final Project - Final Deliverable}
  \begin{block}{Definition}
    The comprehensive output of your project, reflecting all research and development.
  \end{block}

  \begin{block}{Key Elements}
    \begin{itemize}
      \item Final Report: Detailed documentation including:
      \begin{itemize}
        \item Introduction
        \item Methodology
        \item Results and Discussion
        \item Conclusions and Future Work
      \end{itemize}
      \item Presentation: A summary to be delivered to the class.
      \item Demonstration: Show your model in action or provide a code repository.
    \end{itemize}
  \end{block}

  \begin{block}{Example}
    A report documenting how your reinforced model outperformed traditional methods in logistics, accompanied by a PowerPoint presentation.
  \end{block}
\end{frame}

\begin{frame}[fragile]
  \frametitle{Key Points and Conclusion}
  \begin{itemize}
    \item Clarity and Structure: Each component should flow logically and support your project's goals.
    \item Regular Updates: Communicate progress for transparency and feedback.
    \item Attention to Ethics: Enhances credibility and responsibility.
  \end{itemize}
  
  \begin{block}{Conclusion}
    The outline of your final project sets expectations and provides a roadmap for your journey ahead.
  \end{block}
\end{frame}

\begin{frame}[fragile]
    \frametitle{Project Proposal Milestone}
    \begin{block}{Overview of the Project Proposal}
        The project proposal serves as the foundation for your final project, outlining your goals, methodology, and considerations.
        Key components include:
        \begin{itemize}
            \item Algorithm Description
            \item Expected Outcomes
            \item Ethical Implications
        \end{itemize}
    \end{block}
\end{frame}

\begin{frame}[fragile]
    \frametitle{Algorithm Description}
    \begin{block}{Definition}
        An algorithm is a step-by-step procedure for solving a problem. It outlines how data will be processed.
    \end{block}

    \begin{block}{Requirements}
        \begin{itemize}
            \item \textbf{Detailed Steps}: Break down your algorithm into clear, actionable steps.
            \item \textbf{Flowchart/Diagram} (Optional): Visual representations can help clarify the flow.
            \item \textbf{Example}: Describe a specific algorithm relevant to your project.
        \end{itemize}
    \end{block}
    
    \begin{block}{Example of Steps}
        \begin{enumerate}
            \item Select a Pivot.
            \item Partition the Array based on the pivot.
            \item Recursively apply the same method to sub-arrays.
        \end{enumerate}
    \end{block}
\end{frame}

\begin{frame}[fragile]
    \frametitle{Expected Outcomes and Ethical Implications}
    
    \begin{block}{Expected Outcomes}
        \begin{itemize}
            \item \textbf{Definition}: Anticipated results that demonstrate project success.
            \item \textbf{Requirements}:
            \begin{itemize}
                \item Quantitative Metrics: Define measurable metrics (e.g., accuracy).
                \item Qualitative Outcomes: Discuss improvements in UX or stakeholder value.
            \end{itemize}
            \item \textbf{Example}: Expect an increase in user engagement by 20% and improved accuracy to 90\%.
        \end{itemize}
    \end{block}
    
    \begin{block}{Ethical Implications}
        \begin{itemize}
            \item \textbf{Definition}: Considerations regarding data use, privacy, and social impact.
            \item \textbf{Requirements}:
            \begin{itemize}
                \item Data Privacy: Ensure confidentiality and data security.
                \item Bias in Algorithms: Mitigate potential biases.
                \item Social Impact: Reflect on broader implications for society.
            \end{itemize}
        \end{itemize}
    \end{block}
\end{frame}

\begin{frame}[fragile]
    \frametitle{Conclusion}
    \begin{block}{Final Thoughts}
        Your project proposal is crucial as it guides your project's direction and helps anticipate challenges.
        By clearly defining your algorithm, expected outcomes, and considering ethical implications, 
        you lay a strong foundation for a successful and impactful project.
    \end{block}
\end{frame}

\begin{frame}[fragile]
    \frametitle{Progress Report Milestone}
    \begin{block}{Objectives of the Progress Report}
        The Progress Report serves as an essential communication tool in your project timeline. Its main goals are to:
    \end{block}
    \begin{itemize}
        \item \textbf{Document Current Status}: Provide a snapshot of your project's progress against the proposed timeline.
        \item \textbf{Identify Challenges}: Clearly outline obstacles faced during implementation and explore their potential effects on the overall project.
        \item \textbf{Reflect Progress}: Evaluate whether the project is on track to meet goals set in the Project Proposal Milestone.
    \end{itemize}
\end{frame}

\begin{frame}[fragile]
    \frametitle{Structure of the Progress Report}
    \begin{enumerate}
        \item \textbf{Introduction}  
            \begin{itemize}
                \item Briefly revisit the project objectives outlined in your proposal.
                \item Articulate the purpose of this report.
            \end{itemize}
        
        \item \textbf{Implementation Status}
            \begin{itemize}
                \item \textbf{Current Progress}: Describe what has been accomplished to date.
                    \begin{itemize}
                        \item Algorithm development \% completed.
                        \item Data collection milestones achieved.
                        \item Testing phases initiated.
                    \end{itemize}
                \item \textbf{Key Milestones Achieved}: Listing significant milestones such as completion of a prototype or successful initial testing.
            \end{itemize}
    \end{enumerate}
\end{frame}

\begin{frame}[fragile]
    \frametitle{Challenges and Next Steps}
    \begin{enumerate}
        \setcounter{enumi}{2} % To continue numbering from previous frame
        
        \item \textbf{Challenges Encountered}
            \begin{itemize}
                \item \textbf{Technical Challenges}: Describe unforeseen issues with software, algorithms, or tools.
                    \begin{itemize}
                        \item Example: "We faced difficulties in integrating the machine learning algorithm with the existing software framework."
                    \end{itemize}
                \item \textbf{Resource Challenges}: Discuss limitations in resources, such as time constraints or team availability.
                \item \textbf{Mitigation Strategies}: Explain how you plan to address or have already addressed these challenges.
            \end{itemize}

        \item \textbf{Next Steps}
            \begin{itemize}
                \item Outline remaining tasks and expected timelines.
                    \begin{itemize}
                        \item "Finalize the user interface by [date]."
                        \item "Conduct user testing by [date]."
                    \end{itemize}
                \item Indicate adjustments to your initial timeline if necessary.
            \end{itemize}
    \end{enumerate}
\end{frame}

\begin{frame}[fragile]
    \frametitle{Key Points and Example Template}
    \begin{block}{Key Points to Emphasize}
        \begin{itemize}
            \item \textbf{Clarity and Detail}: Be specific in your descriptions to provide a clear picture of your project’s trajectory.
            \item \textbf{Honesty in Reporting}: Be candid about challenges and setbacks while also showcasing problem-solving skills.
            \item \textbf{Alignment with Objectives}: Ensure that your report reflects the goals set out in the project proposal and the learned experiences along the way.
        \end{itemize}
    \end{block}
    
    \begin{block}{Example Template (for your report)}
        \begin{verbatim}
**Project Title: [Your Project Title]**  
**Team Members: [Names]**  
**Report Date: [Date]**

1. **Introduction**  
   - Brief summary of goals and purposes.

2. **Implementation Status**  
   - Current Progress: [Detail]
   - Key Milestones Achieved: 
     - [Milestone 1]
     - [Milestone 2]

3. **Challenges Encountered**  
   - Technical Challenges: [Detail]
   - Resource Challenges: [Detail]
   - Mitigation Strategies: [Detail]

4. **Next Steps**  
   - Remaining Tasks: [List tasks with deadlines]
        \end{verbatim}
    \end{block}
\end{frame}

\begin{frame}
    \frametitle{Final Deliverable Overview}
    \begin{block}{Overview of Submission Requirements}
        The final submission consists of three primary components:
        \begin{itemize}
            \item \textbf{Documentation}
            \item \textbf{Code Deliverables}
            \item \textbf{Performance Metrics}
        \end{itemize}
    \end{block}
\end{frame}

\begin{frame}[fragile]
    \frametitle{Documentation Requirements}
    \begin{block}{Purpose}
        Documentation serves as a guide to your project's structure and functionalities.
    \end{block}
    
    \begin{block}{Requirements}
        \begin{itemize}
            \item \textbf{Project Introduction:} Objectives, scope, and significance.
            \item \textbf{System Architecture:} Diagrams illustrating the framework (e.g., block diagrams).
            \item \textbf{User Manual:} Instructions for installation and usage.
            \item \textbf{Technical Documentation:} Algorithms, data structures, and external libraries details.
        \end{itemize}
    \end{block}
    
    \begin{block}{Example}
        A flowchart showcasing data flow and processing layers.
    \end{block}
\end{frame}

\begin{frame}[fragile]
    \frametitle{Code Deliverable Requirements}
    \begin{block}{Purpose}
        The code represents the heart of the project, bringing design and concepts to life.
    \end{block}
    
    \begin{block}{Requirements}
        \begin{itemize}
            \item \textbf{Clean and Well-Commented Code:} Explain complex logic.
            \item \textbf{Code Repository:} Use a version control system with a README file.
            \item \textbf{Versioning:} Include history to track changes and milestones.
        \end{itemize}
    \end{block}
    
    \begin{block}{Example}
        \begin{lstlisting}[language=Python]
# Function to calculate factorial
def factorial(n):
    """Returns the factorial of n."""
    if n < 0:
        raise ValueError("Negative numbers do not have a factorial.")
    elif n == 0:
        return 1
    else:
        return n * factorial(n - 1)
        \end{lstlisting}
    \end{block}
\end{frame}

\begin{frame}
    \frametitle{Performance Metrics}
    \begin{block}{Purpose}
        Performance metrics evaluate the efficiency and accuracy of your solution.
    \end{block}
    
    \begin{block}{Requirements}
        \begin{itemize}
            \item \textbf{Benchmarking Results:} Compare performance against criteria (e.g., speed).
            \item \textbf{Graphical Representations:} Use charts to visualize data.
            \item \textbf{Analysis:} Discuss effectiveness and areas for improvement.
        \end{itemize}
    \end{block}
    
    \begin{block}{Example Metrics}
        \begin{itemize}
            \item \textbf{Execution Time:} Measure time taken for functions/features.
            \item \textbf{Accuracy Rate:} Report percentage of correct predictions in ML projects.
        \end{itemize}
    \end{block}
    
    \begin{block}{Key Points}
        \begin{itemize}
            \item Thorough documentation enhances usability.
            \item Clean and modular code encourages collaboration.
            \item Performance metrics provide proof of project capabilities.
        \end{itemize}
    \end{block}
\end{frame}

\begin{frame}
    \frametitle{Final Notes}
    By following these guidelines, you'll ensure a professional final project submission. Review the assessment criteria on the next slide to align your deliverable with expectations.
\end{frame}

\begin{frame}[fragile]
    \frametitle{Assessment Criteria - Overview}
    In this section, we will break down the assessment criteria used to evaluate your project proposal, progress report, and final project. Understanding these criteria will help you focus on the key elements that matter most for your successful project execution.
\end{frame}

\begin{frame}[fragile]
    \frametitle{Assessment Criteria - Project Proposal}
    \textbf{Project Proposal (20\% of Total Grade)}

    \begin{block}{Objective}
        Your proposal is the foundation of your project. It sets the direction and outlines your research question, objectives, and planned methodology.
    \end{block}

    \textbf{Key Criteria:}
    \begin{itemize}
        \item \textbf{Clarity of Objectives} (5\%)
        \begin{itemize}
            \item Are your research questions and goals clearly articulated?
        \end{itemize}
        \item \textbf{Feasibility} (5\%)
        \begin{itemize}
            \item Is the proposed project realistic and achievable within the given timeframe?
        \end{itemize}
        \item \textbf{Relevance to Course Material} (5\%)
        \begin{itemize}
            \item Does your proposal connect to principles taught in the course?
        \end{itemize}
        \item \textbf{Literature Review} (5\%)
        \begin{itemize}
            \item Is there a brief overview of existing research that supports your project?
        \end{itemize}
    \end{itemize}
\end{frame}

\begin{frame}[fragile]
    \frametitle{Assessment Criteria - Progress Report}
    \textbf{Progress Report (20\% of Total Grade)}

    \begin{block}{Objective}
        The progress report should reflect your current standing in the project and provide insight into your developments, challenges, and next steps.
    \end{block}

    \textbf{Key Criteria:}
    \begin{itemize}
        \item \textbf{Current Status} (5\%)
        \begin{itemize}
            \item Have you detailed what has been completed so far?
        \end{itemize}
        \item \textbf{Integration of Feedback} (5\%)
        \begin{itemize}
            \item Have you addressed previous feedback from your proposal?
        \end{itemize}
        \item \textbf{Challenges and Solutions} (5\%)
        \begin{itemize}
            \item Are you identifying obstacles and proposing strategies to overcome them?
        \end{itemize}
        \item \textbf{Updated Timeline} (5\%)
        \begin{itemize}
            \item Is there a revised schedule that reflects your current progress?
        \end{itemize}
    \end{itemize}
\end{frame}

\begin{frame}[fragile]
    \frametitle{Assessment Criteria - Final Project}
    \textbf{Final Project (60\% of Total Grade)}

    \begin{block}{Objective}
        The final project encompasses your complete work and should demonstrate your understanding and application of course concepts.
    \end{block}

    \textbf{Key Criteria:}
    \begin{itemize}
        \item \textbf{Technical Implementation} (20\%)
        \begin{itemize}
            \item Is your code functional and does it fulfill the project's objectives?
        \end{itemize}
        \item \textbf{Documentation} (15\%)
        \begin{itemize}
            \item Is your code well-documented with comments explaining why specific choices were made?
        \end{itemize}
        \item \textbf{Analysis and Results} (15\%)
        \begin{itemize}
            \item Are your results analyzed correctly, and do they connect back to your initial objectives?
        \end{itemize}
        \item \textbf{Presentation} (10\%)
        \begin{itemize}
            \item Is your final presentation clear, engaging, and well-structured?
        \end{itemize}
    \end{itemize}
\end{frame}

\begin{frame}[fragile]
    \frametitle{Assessment Criteria - Important Points}
    \begin{itemize}
        \item Focus on clarity in both the written and coding components of your project.
        \item Keep all communication professional and structured, especially during presentations.
        \item Remember, integration and application of course concepts are crucial!
    \end{itemize}
\end{frame}

\begin{frame}[fragile]
    \frametitle{Assessment Criteria - Example Code}
    \begin{block}{Formula for Q-learning}
        If discussing Q-learning, include this relevant formula to explain the learning process in your project:
        \begin{equation}
            Q(s, a) \leftarrow Q(s, a) + \alpha \left[ r + \gamma \max_{a'} Q(s', a') - Q(s, a) \right]
        \end{equation}
    \end{block}
\end{frame}

\begin{frame}[fragile]
    \frametitle{Connecting Theory to Practice - Overview}
    In this slide, we will explore how the theoretical foundations of Reinforcement Learning (RL) translate to practical implementations in your final projects. The goal is to illuminate the synergy between what you’ve learned and your application of those concepts in real-world scenarios.
\end{frame}

\begin{frame}[fragile]
    \frametitle{Key Concepts in Reinforcement Learning}
    \begin{itemize}
        \item \textbf{Agent, Environment, and Actions}:
        \begin{itemize}
            \item \textbf{Agent}: The learner or decision-maker.
            \item \textbf{Environment}: The space where the agent operates.
            \item \textbf{Actions}: Choices made by the agent that affect its state.
        \end{itemize}
        \begin{block}{Example}
            In a game of chess, the agent is the player, the environment is the chessboard, and the actions are the possible moves.
        \end{block}
        
        \item \textbf{Rewards}:
        \begin{itemize}
            \item Feedback from the environment based on actions taken.
            \item Reinforcement signals guide the agent towards optimal behavior.
        \end{itemize}
        \begin{block}{Example}
            In a self-driving car, receiving a reward for successfully avoiding obstacles can prompt the agent to reproduce that behavior in future scenarios.
        \end{block}
        
        \item \textbf{Policies}:
        \begin{itemize}
            \item Strategy that defines the agent's behavior.
            \item Can be deterministic or stochastic.
        \end{itemize}

        \item \textbf{Value Functions}:
        \begin{itemize}
            \item Estimate how good it is for an agent to be in a given state, informing the expected return.
        \end{itemize}
        \begin{block}{Example}
            Using a value function to predict future rewards in a stock trading algorithm can help formulate better investment strategies.
        \end{block}
    \end{itemize}
\end{frame}

\begin{frame}[fragile]
    \frametitle{Project Integration Example}
    \begin{itemize}
        \item \textbf{Theoretical Basis}: You learned about Q-Learning, a model-free RL algorithm used to find optimal actions.
        \item \textbf{Practical Application}: Implement Q-Learning to teach an agent to navigate a maze.
    \end{itemize}
    \textbf{Steps}:
    \begin{enumerate}
        \item \textbf{Define the Environment}: Specify walls, paths, and the goal.
        \item \textbf{Implement the Q-Learning Algorithm}:
        \begin{lstlisting}[language=Python]
import numpy as np

# Initialize Q-table
Q = np.zeros((state_space_size, action_space_size))

# Q-Learning parameters
learning_rate = 0.1
discount_factor = 0.9
exploration_rate = 1.0  # Start fully exploring
        \end{lstlisting}
        \item \textbf{Train the Agent}: Use episodes, updating Q-values based on rewards to ensure learning.
    \end{enumerate}
\end{frame}

\begin{frame}[fragile]
    \frametitle{Ethics in Reinforcement Learning - Overview}
    \begin{block}{Overview}
        Ethics in reinforcement learning (RL) is essential for ensuring that models align with societal values and legal standards. 
        Addressing ethical concerns is critical as RL systems become more pervasive.
    \end{block}
\end{frame}

\begin{frame}[fragile]
    \frametitle{Ethics in Reinforcement Learning - Key Ethical Considerations}
    \begin{enumerate}
        \item \textbf{Fairness}:
            \begin{itemize}
                \item RL models may replicate biases from training data.
                \item \textit{Example}: Loan approval algorithms may disadvantage certain demographic groups.
            \end{itemize}

        \item \textbf{Transparency}:
            \begin{itemize}
                \item Understanding decision-making processes is crucial in high-stakes applications.
                \item \textit{Example}: Stakeholders must understand factors in treatment recommendations.
            \end{itemize}

        \item \textbf{Accountability}:
            \begin{itemize}
                \item Defining responsibility for RL system decisions is essential.
                \item \textit{Example}: Determining fault in self-driving car accidents.
            \end{itemize}

        \item \textbf{Safety and Reliability}:
            \begin{itemize}
                \item Agents must navigate complex environments without harmful actions.
                \item \textit{Example}: Energy distribution must operate safely to avoid crises.
            \end{itemize}
    \end{enumerate}
\end{frame}

\begin{frame}[fragile]
    \frametitle{Ethics in Reinforcement Learning - Frameworks for Evaluation}
    \begin{block}{Frameworks for Evaluation}
        \begin{enumerate}
            \item \textbf{Fairness-aware Training}:
                \begin{itemize}
                    \item Implement measures to mitigate biases with formulas such as:
                    \[
                    \text{Fairness}_i = \frac{P(\text{Decision} | \text{Group}_i)}{P(\text{Decision} | \text{Overall})}
                    \]
                \end{itemize}

            \item \textbf{Transparency Metrics}:
                \begin{itemize}
                    \item Utilize tools (e.g., SHAP values, LIME) to enhance model interpretability.
                \end{itemize}
            
            \item \textbf{Accountability Protocols}:
                \begin{itemize}
                    \item Create guidelines to define accountability and establish audit trails.
                \end{itemize}

            \item \textbf{Safety Constraints}:
                \begin{itemize}
                    \item Design systems with safety mechanisms, such as:
                    \begin{lstlisting}
def safe_reward(observation):
    if leads_to_harm(observation):
        return -100  # Penalize harmful actions
    return default_reward(observation)
                    \end{lstlisting}
                \end{itemize}
        \end{enumerate}
    \end{block}
\end{frame}

\begin{frame}[fragile]
    \frametitle{Ethics in Reinforcement Learning - Key Messages}
    \begin{block}{Key Messages}
        \begin{itemize}
            \item Ethical considerations are fundamental to responsible AI development and deployment.
            \item Utilizing frameworks ensures ethical concerns are prioritized throughout the project lifecycle.
        \end{itemize}
    \end{block}
    \begin{block}{Conclusion}
        By integrating ethical considerations, we contribute to reinforcement learning systems that align with ethical standards and societal values.
    \end{block}
\end{frame}

\begin{frame}[fragile]
    \frametitle{Q\&A Session - Introduction}
    \begin{itemize}
        \item As we approach the culmination of our semester, this Q\&A session serves as a vital opportunity for you to clarify any concepts surrounding your final projects.
        \item Engaging in open discussions not only enhances your understanding but also refines your project outcomes.
    \end{itemize}
\end{frame}

\begin{frame}[fragile]
    \frametitle{Q\&A Session - Objectives}
    \begin{enumerate}
        \item \textbf{Clarification of Project Components}:
            \begin{itemize}
                \item Address uncertainties regarding organization, methodology, or ethical considerations.
            \end{itemize}
        \item \textbf{Encouraging Peer Interaction}:
            \begin{itemize}
                \item Sharing questions and insights can lead to collaborative problem-solving.
            \end{itemize}
        \item \textbf{Preparation for Presentations}:
            \begin{itemize}
                \item Prepare for upcoming presentations and equip yourself with comprehensive responses.
            \end{itemize}
    \end{enumerate}
\end{frame}

\begin{frame}[fragile]
    \frametitle{Q\&A Session - Key Concepts}
    \begin{itemize}
        \item \textbf{Project Components}:
            \begin{itemize}
                \item \textbf{Research Phase}: Discuss sources and their contributions to credibility.
                \item \textbf{Methodology}: Outline your choice of methods and alternatives considered.
                \item \textbf{Ethical Considerations}: Discuss integrated ethical frameworks (e.g., fairness, transparency).
            \end{itemize}
        \item \textbf{Outcomes}: Anticipated results and how they address the problem statement.
        \item \textbf{Challenges Faced}: Obstacles encountered and methods of overcoming them.
    \end{itemize}
\end{frame}

\begin{frame}[fragile]
    \frametitle{Q\&A Session - Discussion Questions}
    \begin{enumerate}
        \item What challenges did you face during your literature review, and how did you navigate them?
        \item Can you explain your choice of a specific algorithm or model? What alternatives were considered?
        \item How did your project align with the ethical considerations previously discussed?
    \end{enumerate}
\end{frame}

\begin{frame}[fragile]
    \frametitle{Q\&A Session - Tips for Engagement}
    \begin{itemize}
        \item \textbf{Listen Actively}: Pay attention to both your questions and your peers’ inquiries.
        \item \textbf{Respect Each Contribution}: Encourage a respectful atmosphere for sharing questions and thoughts.
        \item \textbf{Utilize Visual Aids}: Reference slides or diagrams to clarify points.
    \end{itemize}
\end{frame}

\begin{frame}[fragile]
    \frametitle{Q\&A Session - Conclusion}
    \begin{itemize}
        \item Effective engagement during this Q\&A session can significantly enhance your project development.
        \item Feel free to ask questions and share insights for a productive discussion.
    \end{itemize}
\end{frame}

\begin{frame}[fragile]
    \frametitle{Project Timeline - Overview}
    A project timeline is essential in keeping your final project on track and ensuring that all students meet expectations. 
    \begin{itemize}
        \item Clear expectations for each milestone
        \item Importance of deadlines for project management
    \end{itemize}
\end{frame}

\begin{frame}[fragile]
    \frametitle{Key Milestones}
    \begin{enumerate}
        \item \textbf{Project Proposal Submission}
        \begin{itemize}
            \item \textit{Due Date:} [Insert Date]
            \item Submit a detailed outline of your project, including objectives and expected outcomes.
        \end{itemize}
        
        \item \textbf{Literature Review Completion}
        \begin{itemize}
            \item \textit{Due Date:} [Insert Date]
            \item Conduct a thorough review of existing research related to your project topic.
        \end{itemize}

        \item \textbf{Draft Submission}
        \begin{itemize}
            \item \textit{Due Date:} [Insert Date]
            \item Submit a full draft of your project for feedback.
        \end{itemize}
        
        \item \textbf{Peer Review Feedback}
        \begin{itemize}
            \item \textit{Due Date:} [Insert Date]
            \item Provide and receive constructive feedback among peers.
        \end{itemize}

        \item \textbf{Final Project Submission}
        \begin{itemize}
            \item \textit{Due Date:} [Insert Date]
            \item Submit the polished final version, incorporating all feedback.
        \end{itemize}
    \end{enumerate}
\end{frame}

\begin{frame}[fragile]
    \frametitle{Tips for Success}
    \begin{itemize}
        \item \textbf{Stay Organized:} Use a project management tool to track progress.
        \item \textbf{Set Personal Goals:} Break milestones into manageable tasks.
        \item \textbf{Regular Check-Ins:} Schedule regular meetings with instructors or mentors.
    \end{itemize}
\end{frame}

\begin{frame}[fragile]
    \frametitle{Visualizing Your Timeline}
    Consider creating a Gantt Chart as a visual representation of your timeline:
    \begin{block}{Example Gantt Chart Structure}
    \begin{lstlisting}
| Milestone                | Due Date   | Status       |
|-------------------------|------------|--------------|
| Project Proposal        | [Insert]   | Not Started   |
| Literature Review       | [Insert]   | Not Started   |
| Draft Submission        | [Insert]   | Not Started   |
| Peer Review Feedback     | [Insert]   | Not Started   |
| Final Project Submission | [Insert]   | Not Started   |
    \end{lstlisting}
    \end{block}
\end{frame}

\begin{frame}[fragile]
    \frametitle{Conclusion}
    Understanding and adhering to the project timeline is key to ensuring your final project meets all required standards.
    \begin{itemize}
        \item Prioritize tasks
        \item Stay organized
        \item Seek support when needed
    \end{itemize}
\end{frame}


\end{document}