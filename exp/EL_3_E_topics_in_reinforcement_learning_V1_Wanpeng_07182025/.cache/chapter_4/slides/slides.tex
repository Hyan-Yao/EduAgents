\documentclass[aspectratio=169]{beamer}

% Theme and Color Setup
\usetheme{Madrid}
\usecolortheme{whale}
\useinnertheme{rectangles}
\useoutertheme{miniframes}

% Additional Packages
\usepackage[utf8]{inputenc}
\usepackage[T1]{fontenc}
\usepackage{graphicx}
\usepackage{booktabs}
\usepackage{listings}
\usepackage{amsmath}
\usepackage{amssymb}
\usepackage{xcolor}
\usepackage{tikz}
\usepackage{pgfplots}
\pgfplotsset{compat=1.18}
\usetikzlibrary{positioning}
\usepackage{hyperref}

% Custom Colors
\definecolor{myblue}{RGB}{31, 73, 125}
\definecolor{mygray}{RGB}{100, 100, 100}
\definecolor{mygreen}{RGB}{0, 128, 0}
\definecolor{myorange}{RGB}{230, 126, 34}
\definecolor{mycodebackground}{RGB}{245, 245, 245}

% Set Theme Colors
\setbeamercolor{structure}{fg=myblue}
\setbeamercolor{frametitle}{fg=white, bg=myblue}
\setbeamercolor{title}{fg=myblue}
\setbeamercolor{section in toc}{fg=myblue}
\setbeamercolor{item projected}{fg=white, bg=myblue}
\setbeamercolor{block title}{bg=myblue!20, fg=myblue}
\setbeamercolor{block body}{bg=myblue!10}
\setbeamercolor{alerted text}{fg=myorange}

% Set Fonts
\setbeamerfont{title}{size=\Large, series=\bfseries}
\setbeamerfont{frametitle}{size=\large, series=\bfseries}
\setbeamerfont{caption}{size=\small}
\setbeamerfont{footnote}{size=\tiny}

% Document Start
\begin{document}

\frame{\titlepage}

\begin{frame}[fragile]
    \frametitle{Introduction to Monte Carlo Methods - Overview}
    \begin{block}{Overview of Monte Carlo Methods}
        Monte Carlo methods are computational algorithms that use random sampling to obtain numerical results. 
        In reinforcement learning (RL), they are essential for evaluating and improving policies based on observed outcomes.
    \end{block}
\end{frame}

\begin{frame}[fragile]
    \frametitle{Introduction to Monte Carlo Methods - Significance in RL}
    \begin{block}{Significance in Reinforcement Learning}
        \begin{enumerate}
            \item \textbf{Policy Evaluation:} 
            \begin{itemize}
                \item Policies define the behavior of an agent in RL.
                \item Monte Carlo methods enable policy evaluation by averaging returns for state-action pairs across multiple episodes.
                \item They facilitate direct estimation of the value function for a given policy without requiring a model of the environment.
            \end{itemize}

            \item \textbf{Exploratory Nature:}
            \begin{itemize}
                \item Leverage randomness, allowing agents to explore various outcomes and learn from the environment.
            \end{itemize}
        \end{enumerate}
    \end{block}
\end{frame}

\begin{frame}[fragile]
    \frametitle{Introduction to Monte Carlo Methods - Working Mechanism}
    \begin{block}{How Monte Carlo Methods Work}
        An agent interacts with the environment through the following steps:
        \begin{enumerate}
            \item \textbf{Simulate Episodes:} 
            \begin{itemize}
                \item Generate multiple episodes using the current policy.
            \end{itemize}
            \item \textbf{Calculate Returns:} 
            \begin{equation}
                G_t = R_{t+1} + \gamma R_{t+2} + \gamma^2 R_{t+3} + \ldots
            \end{equation}
            where $\gamma$ (discount factor) weighs the importance of future rewards.
            \item \textbf{Update Value Estimates:}
            \begin{itemize}
                \item Average returns from all visits to state-action pairs to refine value estimates.
            \end{itemize}
        \end{enumerate}
    \end{block}
\end{frame}

\begin{frame}[fragile]
    \frametitle{Understanding Monte Carlo Methods - Overview}
    \begin{itemize}
        \item Definition of Monte Carlo methods
        \item Importance in various fields
        \item Use in reinforcement learning for policy evaluation
    \end{itemize}
\end{frame}

\begin{frame}[fragile]
    \frametitle{What are Monte Carlo Methods?}
    \begin{block}{Definition}
        Monte Carlo methods are a class of computational algorithms that rely on repeated random sampling to obtain numerical results.
    \end{block}
    \begin{itemize}
        \item Used in statistics, physics, engineering, and reinforcement learning.
        \item Crucial for evaluating and improving policies.
    \end{itemize}
\end{frame}

\begin{frame}[fragile]
    \frametitle{How Monte Carlo Methods Work}
    \begin{enumerate}
        \item \textbf{Random Sampling:} Simulate various outcomes through random samples.
        \item \textbf{Policy Evaluation:}
            \begin{itemize}
                \item Define a policy $\pi$ and generate episodes.
                \item Collect rewards and calculate cumulative return $G_t$:
                \[
                G_t = R_{t+1} + \gamma R_{t+2} + \gamma^2 R_{t+3} + \ldots
                \]
            \end{itemize}
        \item \textbf{Estimating Value Function:}
            \[
            V(s) \approx \frac{1}{N} \sum_{i=1}^{N} G_t^i
            \]
        \item \textbf{Convergence:} Estimates converge to true values as episodes increase.
    \end{enumerate}
\end{frame}

\begin{frame}[fragile]
    \frametitle{Key Points to Emphasize}
    \begin{itemize}
        \item \textbf{Exploration:} Random sampling allows exploration of the environment.
        \item \textbf{Simplicity:} No need for a model of the environment.
        \item \textbf{Sample Size:} More samples lead to improved evaluations.
    \end{itemize}
\end{frame}

\begin{frame}[fragile]
    \frametitle{Example Illustration}
    \begin{block}{Board Game Scenario}
        Imagine a board game where a player rolls a die (random sampling). The aim is to reach the goal (win the game). By simulating many plays:
        \begin{itemize}
            \item Evaluate strategies (policies).
            \item Understand which moves contribute to success.
        \end{itemize}
    \end{block}
\end{frame}

\begin{frame}[fragile]
    \frametitle{Conclusion}
    \begin{itemize}
        \item Monte Carlo methods are powerful for evaluating policies in reinforcement learning.
        \item They leverage randomness and sampling for robust insights.
        \item Further exploration will cover advantages and limitations.
    \end{itemize}
\end{frame}

\begin{frame}[fragile]
    \frametitle{References}
    \begin{itemize}
        \item Sutton, R. S., \& Barto, A. G. (2018). \textit{Reinforcement Learning: An Introduction} (2nd ed.). MIT Press.
        \item Russel, S. J., \& Norvig, P. (2016). \textit{Artificial Intelligence: A Modern Approach} (3rd ed.). Pearson.
    \end{itemize}
\end{frame}

\begin{frame}[fragile]
    \frametitle{Applications in Policy Evaluation - Introduction}
    Monte Carlo methods serve as powerful tools in evaluating reinforcement learning (RL) policies. The key aspects include:
    \begin{itemize}
        \item A policy defines the agent's behavior in an environment.
        \item Policy evaluation aims to determine the expected performance of these policies.
        \item The process is achieved through sampling to estimate expected returns.
    \end{itemize}
\end{frame}

\begin{frame}[fragile]
    \frametitle{Key Concepts in Monte Carlo Methods}
    \begin{enumerate}
        \item \textbf{Random Sampling:} Generate episodes through random sampling across the state space to evaluate policies.
        \item \textbf{Return Calculation:} The return \( G_i \) from episode \( i \) starting from state \( s \) is computed as:
        \begin{equation}
            V(s) \approx \frac{1}{N} \sum_{i=1}^{N} G_i
        \end{equation}
        where \( N \) is the number of episodes.
    \end{enumerate}
\end{frame}

\begin{frame}[fragile]
    \frametitle{Advantages and Limitations of Monte Carlo Methods}
    \begin{block}{Advantages}
        \begin{itemize}
            \item \textbf{Simplicity:} Straightforward to implement, relying on sampling and averaging.
            \item \textbf{Model-Free:} Applicable in environments with unknown dynamics.
            \item \textbf{Effective in Finite Horizons:} Works well in finite-horizon problems.
        \end{itemize}
    \end{block}
    \begin{block}{Limitations}
        \begin{itemize}
            \item \textbf{High Variance:} Estimates can vary greatly, leading to unreliable evaluations.
            \item \textbf{Sample Inefficiency:} Computationally costly in sparse reward situations.
            \item \textbf{Delay in Learning:} Requires complete episodes for updates, which slows learning.
        \end{itemize}
    \end{block}
\end{frame}

\begin{frame}[fragile]
    \frametitle{Example of Policy Evaluation}
    In a gridworld scenario for evaluating a policy:
    \begin{enumerate}
        \item Run multiple episodes where the agent follows the defined policy, collecting rewards.
        \item After sufficient episodes, calculate the average return for visited states.
        \item Assess the policy's effectiveness using these average returns to guide refinements.
    \end{enumerate}
\end{frame}

\begin{frame}[fragile]
    \frametitle{Conclusion and Key Takeaways}
    \begin{itemize}
        \item Monte Carlo methods quantitatively evaluate policies using random sampling.
        \item They offer simplicity and model-free applicability but suffer from high variance and sample inefficiency.
        \item Practical evaluation involves multiple episode runs and averaging returns to inform policy improvement.
    \end{itemize}
\end{frame}

\begin{frame}[fragile]
    \frametitle{The Monte Carlo Process - Introduction}
    \begin{block}{Introduction to the Monte Carlo Process}
        Monte Carlo methods are widely used in reinforcement learning to evaluate the performance of policies. The fundamental idea is to use random sampling to approximate the value of different actions based on experience obtained through episodes in the environment.
    \end{block}
    
    \begin{itemize}
        \item Learn from actions and their impact on outcomes.
        \item Useful in environments where transitions are stochastic and complex.
    \end{itemize}
\end{frame}

\begin{frame}[fragile]
    \frametitle{The Monte Carlo Process - Steps Overview}
    \begin{enumerate}
        \item Define Policy
        \item Episode Generation
        \item Return Calculation
        \item Estimate Value Function
    \end{enumerate}
\end{frame}

\begin{frame}[fragile]
    \frametitle{The Monte Carlo Process - Steps Detailed}
    \begin{enumerate}
        \item \textbf{Define Policy}:
        \begin{itemize}
            \item Denoted as \( \pi(a|s) \) for action \( a \) based on state \( s \).
            \item The strategy we wish to evaluate.
        \end{itemize}

        \item \textbf{Episode Generation}:
        \begin{itemize}
            \item Generate multiple episodes by following policy \( \pi \).
            \item Each episode structured as \( (s_1, a_1, r_1, \ldots) \).
            \item Example: Epsilon-greedy policy in a grid world.
        \end{itemize}
    \end{enumerate}
\end{frame}

\begin{frame}[fragile]
    \frametitle{The Monte Carlo Process - Return Calculation}
    \begin{block}{Return Calculation}
        For each episode, at each time step \( t \):
        \begin{equation}
        G_t = R_t + \gamma R_{t+1} + \gamma^2 R_{t+2} + \ldots = \sum_{k=0}^{T-t} \gamma^k R_{t+k}
        \end{equation}
    \end{block}
    
    \begin{itemize}
        \item \( \gamma \) is the discount factor (0 ≤ \( \gamma \) < 1).
        \item Example: 
        \[
        G_1 = 1 + 0.9 \cdot 0 + 0.9^2 \cdot 2 = 1 + 0 + 0.81 = 1.81
        \]
    \end{itemize}
\end{frame}

\begin{frame}[fragile]
    \frametitle{The Monte Carlo Process - Estimate Value Function}
    \begin{block}{Estimate Value Function}
        \begin{itemize}
            \item After generating episodes, estimate the value function \( V(s) \):
            \begin{equation}
            V(s) \leftarrow \frac{1}{N(s)} \sum_{i=1}^{N(s)} G_t
            \end{equation}
            \item \( N(s) \) is the number of visits to state \( s \).
        \end{itemize}
    \end{block}
    
    \begin{itemize}
        \item Returns provide insight into long-term state and action values.
    \end{itemize}
\end{frame}

\begin{frame}[fragile]
    \frametitle{The Monte Carlo Process - Key Points and Conclusion}
    \begin{block}{Key Points to Emphasize}
        \begin{itemize}
            \item Episodic interactions allow effective evaluation.
            \item Balancing exploration and exploitation influences evaluation quality.
        \end{itemize}
    \end{block}
    
    \begin{block}{Conclusion}
        The Monte Carlo process enhances reinforcement learning by enabling action value estimation based on sampled experiences.
    \end{block}
\end{frame}

\begin{frame}[fragile]
    \frametitle{Exploration vs. Exploitation - Introduction}
    In the context of Monte Carlo methods, particularly in reinforcement learning and decision-making scenarios:
    \begin{itemize}
        \item **Exploration**: Taking actions with uncertain outcomes to gather new information about the environment.
        \item **Exploitation**: Using known information to maximize immediate rewards based on what has already been learned.
    \end{itemize}
    This balance is crucial for finding an optimal policy and enhancing learning effectiveness.
\end{frame}

\begin{frame}[fragile]
    \frametitle{Key Concepts}
    \begin{block}{Exploration}
        \begin{itemize}
            \item **Definition**: Taking uncertain actions to gather new information.
            \item **Purpose**: Discover better strategies or rewards.
            \item **Example**: Exploring various paths in a maze can reveal the shortest route to the exit.
        \end{itemize}
    \end{block}

    \begin{block}{Exploitation}
        \begin{itemize}
            \item **Definition**: Utilizing known information to secure higher returns.
            \item **Purpose**: Maximize immediate rewards based on existing knowledge.
            \item **Example**: Following a known rewarding path in a maze instead of trying new paths.
        \end{itemize}
    \end{block}
\end{frame}

\begin{frame}[fragile]
    \frametitle{Exploration-Exploitation Trade-off}
    \begin{itemize}
        \item A balance is essential in Monte Carlo methods to enhance learning:
        \begin{itemize}
            \item **Too Much Exploration**: Slow learning and performance due to excessive testing of new actions.
            \item **Too Much Exploitation**: Convergence to suboptimal solutions, leading to stagnation.
        \end{itemize}
        \item This balance can be modeled using the strategy parameter \( \epsilon \):
        \begin{equation}
            \text{Epsilon-greedy policy: } 
            \begin{cases}
                \text{With probability } \epsilon, \text{ choose a random action (explore).} \\
                \text{With probability } 1 - \epsilon, \text{ choose the best-known action (exploit).}
            \end{cases}
        \end{equation}
        where \( \epsilon \) is a small value (e.g., 0.1).
    \end{itemize}
\end{frame}

\begin{frame}[fragile]
    \frametitle{Final Thoughts & Further Reading}
    \begin{itemize}
        \item Finding the right balance between exploration and exploitation is vital for effective learning in Monte Carlo methods.
        \item Dynamic adjustment strategies like gradually decreasing \( \epsilon \) can be beneficial.
        \item This trade-off is applicable across various machine learning frameworks, including multi-armed bandit problems.
    \end{itemize}

    \textbf{Further Reading:}
    \begin{itemize}
        \item Explore dynamic adjustment methods for \( \epsilon \).
        \item Learn about Upper Confidence Bound (UCB) approaches.
        \item Dive deeper into Monte Carlo methods and their comparisons.
    \end{itemize}
\end{frame}

\begin{frame}{Types of Monte Carlo Approaches}
    \begin{itemize}
        \item Monte Carlo methods use random sampling to provide numerical results.
        \item Two primary categories:
        \begin{itemize}
            \item On-Policy
            \item Off-Policy
        \end{itemize}
    \end{itemize}
\end{frame}

\begin{frame}{Introduction to Monte Carlo Methods}
    \begin{block}{Definition}
        Monte Carlo methods are a class of algorithms that rely on random sampling to obtain numerical results.
    \end{block}
    \begin{itemize}
        \item Applications include:
        \begin{itemize}
            \item Statistical analysis
            \item Simulation of physical systems
            \item Optimization problems in reinforcement learning
        \end{itemize}
    \end{itemize}
\end{frame}

\begin{frame}{On-Policy Monte Carlo Methods}
    \frametitle{On-Policy Monte Carlo Methods}
    
    \begin{block}{Definition}
        On-policy methods evaluate and improve the policy currently in use.
    \end{block}

    \begin{itemize}
        \item Key Features:
        \begin{itemize}
            \item Uses only actions taken by the current policy.
            \item Learns the value of the current policy and modifies it simultaneously.
        \end{itemize}
        
        \item Example:
        \begin{itemize}
            \item Agent playing tic-tac-toe with an aggressive strategy.
            \item Outcome updates based solely on aggressive moves.
        \end{itemize}
    \end{itemize}
\end{frame}

\begin{frame}{Off-Policy Monte Carlo Methods}
    \frametitle{Off-Policy Monte Carlo Methods}
    
    \begin{block}{Definition}
        Off-policy methods evaluate or improve a policy different from the one used to generate actions.
    \end{block}

    \begin{itemize}
        \item Key Features:
        \begin{itemize}
            \item Uses data generated by another policy.
            \item Allows learning from a broader range of experiences for better convergence.
        \end{itemize}
        
        \item Example:
        \begin{itemize}
            \item Agent learning from less aggressive strategies.
            \item Update estimates for aggressive actions based on different policy outcomes.
        \end{itemize}
    \end{itemize}
\end{frame}

\begin{frame}{Summary of Comparisons}
    \begin{table}[ht]
        \centering
        \begin{tabular}{|c|c|c|}
            \hline
            Feature & On-Policy & Off-Policy \\
            \hline
            Policy Being Evaluated & Current policy & Different/alternative policy \\
            \hline
            Data Utilization & Actions from current policy only & Actions from any policy \\
            \hline
            Use Case & Policy improvement and evaluation & Learning from exploratory behavior \\
            \hline
        \end{tabular}
    \end{table}
\end{frame}

\begin{frame}{Conclusion and Learning Checkpoint}
    \begin{block}{Conclusion}
        Understanding the differences between on-policy and off-policy Monte Carlo methods is crucial for effective reinforcement learning algorithms.
    \end{block}
    
    \begin{itemize}
        \item On-policy: Rapid adjustments.
        \item Off-policy: Flexibility and broader experience utilization.
    \end{itemize}

    \begin{block}{Learning Checkpoint}
        \begin{itemize}
            \item What is the primary difference between on-policy and off-policy methods?
            \item Provide an example of when to prefer one method over the other.
        \end{itemize}
    \end{block}
\end{frame}

\begin{frame}[fragile]{Code Snippet Example}
    \frametitle{Pseudocode: On-Policy Monte Carlo Learning}
    \begin{lstlisting}[language=Python]
def on_policy_monte_carlo(env, num_episodes):
    returns = {}
    policy = initialize_policy(env)
    
    for episode in range(num_episodes):
        states, actions, rewards = play_episode(env, policy)
        G = sum(rewards)
        
        for state, action in zip(states, actions):
            if (state, action) not in returns:
                returns[(state, action)] = []
            returns[(state, action)].append(G)
            policy[state][action] = np.mean(returns[(state, action)])
    return policy
    \end{lstlisting}
\end{frame}

\begin{frame}[fragile]
    \frametitle{Implementing Monte Carlo Methods - Overview}
    \begin{block}{What are Monte Carlo Methods?}
        Monte Carlo methods use random sampling to estimate mathematical functions or probabilities. In reinforcement learning, they are used for:
        \begin{itemize}
            \item Policy evaluation
            \item Estimating the expected return of specific policies
        \end{itemize}
    \end{block}
\end{frame}

\begin{frame}[fragile]
    \frametitle{Implementing Monte Carlo Methods - Policy Evaluation}
    \begin{block}{Policy Evaluation}
        Policy evaluation estimates the value function \( V^{\pi}(s) \)
        \begin{itemize}
            \item Represents expected return starting from state \( s \) and following policy \( \pi \)
            \item Use sampled episodes to approximate this value
        \end{itemize}
        
        \begin{equation}
            V^{\pi}(s) \approx \frac{1}{N(s)} \sum_{i=1}^{N(s)} G_i
        \end{equation}
        
        Where:
        \begin{itemize}
            \item \( G_i \) = return from episode \( i \)
            \item \( N(s) \) = number of times state \( s \) has been visited
        \end{itemize}
    \end{block}
\end{frame}

\begin{frame}[fragile]
    \frametitle{Implementing Monte Carlo Methods - Code Snippets}
    \begin{block}{Python Implementation Steps}
        \textbf{Step 1: Import Libraries}
        \begin{lstlisting}[language=Python]
import numpy as np
import random
        \end{lstlisting}

        \textbf{Step 2: Define the Environment and Policy}
        \begin{lstlisting}[language=Python]
def policy(state):
    return np.random.choice([0, 1])  # Action 0 or 1
        \end{lstlisting}

        \textbf{Step 3: Generate Episodes}
        \begin{lstlisting}[language=Python]
def generate_episode(env, policy):
    state = env.reset()
    episode = []
    done = False
    
    while not done:
        action = policy(state)
        next_state, reward, done = env.step(action)
        episode.append((state, action, reward))
        state = next_state
        
    return episode
        \end{lstlisting}
    \end{block}
\end{frame}

\begin{frame}[fragile]
    \frametitle{Implementing Monte Carlo Methods - Code Continuation}
    \begin{block}{Perform Monte Carlo Evaluation}
        \textbf{Step 4: Monte Carlo Evaluation}
        \begin{lstlisting}[language=Python]
def monte_carlo_evaluation(env, policy, num_episodes):
    returns = {}
    N = {}
    V = {}

    for episode in range(num_episodes):
        episode_data = generate_episode(env, policy)
        G = sum([reward for _, _, reward in episode_data])  # Total return

        for state, _, _ in episode_data:
            if state not in returns:
                returns[state] = 0
                N[state] = 0
            
            returns[state] += G  # Add return to state
            N[state] += 1        # Increment visit count
            V[state] = returns[state] / N[state]  # Update value function

    return V
        \end{lstlisting}

        \textbf{Step 5: Execute the Evaluation}
        \begin{lstlisting}[language=Python]
env = YourEnvironment()  # Replace with your actual environment
num_episodes = 1000
value_function = monte_carlo_evaluation(env, policy, num_episodes)
print(value_function)
        \end{lstlisting}
    \end{block}
\end{frame}

\begin{frame}
    \frametitle{Challenges and Solutions in Monte Carlo Methods}
    \begin{block}{Introduction to Monte Carlo Methods}
        Monte Carlo methods rely on random sampling to estimate numerical outcomes and evaluate policies in various environments. While powerful, these methods present specific challenges that can impede accurate evaluations.
    \end{block}
\end{frame}

\begin{frame}[fragile]
    \frametitle{Common Challenges in Policy Evaluation Using Monte Carlo Methods}
    \begin{enumerate}
        \item \textbf{Variance in Estimates}:
            \begin{itemize}
                \item Explanation: High variance can lead to unreliable evaluations.
                \item Example: In a financial simulation, varying market conditions yield significantly different returns, affecting portfolio evaluation.
            \end{itemize}
        \item \textbf{Computational Load}:
            \begin{itemize}
                \item Explanation: Significant computational resources are required, especially for complex models.
                \item Example: Evaluating a policy in dynamic programming with thousands of states can lead to long run times.
            \end{itemize}
        \item \textbf{Convergence Issues}:
            \begin{itemize}
                \item Explanation: Ensuring estimates converge can be problematic based on sampling methods.
                \item Example: Insufficient sampling could miss rare but impactful events, leading to skewed evaluations.
            \end{itemize}
        \item \textbf{Exploration vs. Exploitation}:
            \begin{itemize}
                \item Explanation: Balancing exploration of new policies with exploitation of known information is challenging.
                \item Example: In reinforcement learning, too much exploration can delay discovering optimal strategies.
            \end{itemize}
    \end{enumerate}
\end{frame}

\begin{frame}
    \frametitle{Potential Solutions}
    \begin{enumerate}
        \item \textbf{Reduction of Variance}:
            \begin{itemize}
                \item Solution: Employ variance reduction techniques (e.g., control variates).
                \item Example: Control variates adjust estimates of a random sampling process.
            \end{itemize}
        \item \textbf{Parallelization}:
            \begin{itemize}
                \item Solution: Run simulations in parallel for efficiency.
                \item Example: Utilizing libraries like \texttt{multiprocessing} in Python can reduce computation time significantly.
            \end{itemize}
        \item \textbf{Adaptive Sampling}:
            \begin{itemize}
                \item Solution: Focus more on uncertain areas for improved convergence rates.
                \item Example: Gradual increase in sample size in areas of uncertainty enhances efficiency.
            \end{itemize}
        \item \textbf{Use of Efficient Policies}:
            \begin{itemize}
                \item Solution: Combine Monte Carlo methods with other techniques for balanced exploration and exploitation.
                \item Example: Epsilon-greedy strategies ensure a percentage of trials are exploratory.
            \end{itemize}
    \end{enumerate}
\end{frame}

\begin{frame}[fragile]
    \frametitle{Key Takeaways and Technical Details}
    \begin{block}{Key Points to Emphasize}
        \begin{itemize}
            \item The effectiveness of Monte Carlo methods hinges on managing variance, computational resources, and ensuring convergence.
            \item Variance reduction, parallelization, and adaptive strategies greatly enhance the reliability and efficiency of evaluations.
            \item Combining Monte Carlo methods with other techniques can create more robust evaluation frameworks.
        \end{itemize}
    \end{block}
    
    \begin{block}{Variance Reduction Example (Control Variates)}
        \begin{equation}
            \hat{A} = \hat{X} + \beta (C - \hat{Y})
        \end{equation}
        where:
        \begin{itemize}
            \item \( \hat{A} \): adjusted estimate
            \item \( \hat{X} \): initial Monte Carlo estimate
            \item \( C \): known mean of control variate
            \item \( \hat{Y} \): estimate of control variate
            \item \( \beta \): sensitivity factor based on covariance
        \end{itemize}
    \end{block}
\end{frame}

\begin{frame}[fragile]
    \frametitle{Code Snippet for Parallelization}
    \begin{lstlisting}[language=Python]
    import numpy as np
    from multiprocessing import Pool

    def simulate_policy(policy):
        # Simulate policy and return the outcome
        return np.random.rand()  # Placeholder for actual policy evaluation

    policies = [policy1, policy2, policy3]
    with Pool(processes=4) as pool:
        results = pool.map(simulate_policy, policies)
    \end{lstlisting}
\end{frame}

\begin{frame}[fragile]
    \frametitle{Real-world Examples of Monte Carlo Methods}
    Monte Carlo methods are computational algorithms that utilize repeated random sampling for numerical results. They are especially useful when exact calculations are complex or infeasible.
\end{frame}

\begin{frame}[fragile]
    \frametitle{Key Applications of Monte Carlo Methods - Part 1}
    \begin{enumerate}
        \item \textbf{Finance:}
            \begin{itemize}
                \item \textbf{Option Pricing:}
                    Monte Carlo simulations are used for pricing complex derivatives. The value \( V \) of an option can be computed as:
                    \begin{equation}
                    V = e^{-rT} \cdot \frac{1}{N} \sum_{i=1}^{N} \max(S_i - K, 0)
                    \end{equation}
                    where \( N \) is the number of simulated paths, \( S_i \) is the simulated stock price, \( K \) is the strike price, and \( r \) is the risk-free rate.
                    
                \item \textbf{Risk Assessment:}
                    Assess risks by simulating various economic scenarios to predict portfolio performance.
            \end{itemize}
    \end{enumerate}
\end{frame}

\begin{frame}[fragile]
    \frametitle{Key Applications of Monte Carlo Methods - Part 2}
    \begin{enumerate}
        \setcounter{enumi}{2} % continue numbering from previous frame
        \item \textbf{Game Playing:}
            \begin{itemize}
                \item \textbf{Monte Carlo Tree Search (MCTS):}
                    An algorithm used in games like chess or Go to simulate potential outcomes to determine the best move.
                \item \textbf{Poker AI:}
                    Helps evaluate hand strengths and actions through numerous simulated rounds.
            \end{itemize} 
        \item \textbf{Engineering:}
            \begin{itemize}
                \item \textbf{Reliability Analysis:}
                    Assess systems' reliability under uncertainty through modeling different failure modes.
            \end{itemize} 
        \item \textbf{Environmental Science:}
            \begin{itemize}
                \item \textbf{Climate Modeling:}
                    Used in predicting weather patterns based on various scenarios and conditions.
            \end{itemize}
    \end{enumerate}
\end{frame}

\begin{frame}[fragile]
    \frametitle{Final Remarks on Monte Carlo Methods}
    \begin{block}{Key Points to Emphasize}
        \begin{itemize}
            \item \textbf{Versatility:} Applied across diverse fields from finance to AI.
            \item \textbf{Random Sampling:} Core principle to approximate solutions for complex problems.
            \item \textbf{Simulation Over Analytical:} Provides a viable alternative when analytical solutions are infeasible.
        \end{itemize}
    \end{block}
    
    Monte Carlo methods leverage probability and simulation to address real-world issues, enhancing decision-making through predictive models.
\end{frame}

\begin{frame}[fragile]
    \frametitle{Conclusion - Summary of Key Points}
    
    \begin{enumerate}
        \item \textbf{Definition of Monte Carlo Methods:} 
        Statistical techniques that rely on random sampling for numerical results.

        \item \textbf{Importance in Reinforcement Learning (RL):} 
        - Critical for estimating policy values by simulating outcomes over time.
        - Helps in approximating the value function, essential for policy evaluation.

        \item \textbf{Applications:} 
        - Game Playing: Monte Carlo Tree Search (MCTS) in games like Chess and Go.
        - Finance: Pricing complex derivatives and assessing risks by simulating market conditions.
    \end{enumerate}
\end{frame}

\begin{frame}[fragile]
    \frametitle{Conclusion - Advantages and Limitations}

    \begin{block}{Advantages}
        \begin{itemize}
            \item \textbf{Simple to Implement:} Conceptually straightforward with no need for detailed system knowledge.
            \item \textbf{Flexibility:} Applicable across various fields and problems.
        \end{itemize}
    \end{block}

    \begin{block}{Limitations}
        \begin{itemize}
            \item High variance in estimates may lead to fluctuating results, requiring many samples.
            \item Computationally intensive, especially for accuracy in large simulations.
        \end{itemize}
    \end{block}
\end{frame}

\begin{frame}[fragile,plain]
    \frametitle{Conclusion - Illustrative Example and Key Formula}

    \textbf{Illustrative Example:} 
    Consider a robotic agent navigating a maze using Monte Carlo methods:
    \begin{itemize}
        \item Randomly samples paths through the maze.
        \item Each sample provides a reward for reaching the goal efficiently.
        \item Averaging rewards over simulations helps evaluate navigation policies.
    \end{itemize}

    \textbf{Key Formula:}
    \begin{equation}
        V^{\pi}(s) \approx \frac{1}{N} \sum_{i=1}^N G_t^i
    \end{equation}
    Where:
    \begin{itemize}
        \item \( V^{\pi}(s) \) = estimated value of state \( s \) under policy \( \pi \).
        \item \( N \) = number of episodes sampled.
        \item \( G_t^i \) = return (cumulative reward) from time \( t \) in the \( i^{th} \) episode.
    \end{itemize}
\end{frame}


\end{document}