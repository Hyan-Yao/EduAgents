\documentclass[aspectratio=169]{beamer}

% Theme and Color Setup
\usetheme{Madrid}
\usecolortheme{whale}
\useinnertheme{rectangles}
\useoutertheme{miniframes}

% Additional Packages
\usepackage[utf8]{inputenc}
\usepackage[T1]{fontenc}
\usepackage{graphicx}
\usepackage{booktabs}
\usepackage{listings}
\usepackage{amsmath}
\usepackage{amssymb}
\usepackage{xcolor}
\usepackage{tikz}
\usepackage{pgfplots}
\pgfplotsset{compat=1.18}
\usetikzlibrary{positioning}
\usepackage{hyperref}

% Custom Colors
\definecolor{myblue}{RGB}{31, 73, 125}
\definecolor{mygray}{RGB}{100, 100, 100}
\definecolor{mygreen}{RGB}{0, 128, 0}
\definecolor{myorange}{RGB}{230, 126, 34}
\definecolor{mycodebackground}{RGB}{245, 245, 245}

% Set Theme Colors
\setbeamercolor{structure}{fg=myblue}
\setbeamercolor{frametitle}{fg=white, bg=myblue}
\setbeamercolor{title}{fg=myblue}
\setbeamercolor{section in toc}{fg=myblue}
\setbeamercolor{item projected}{fg=white, bg=myblue}
\setbeamercolor{block title}{bg=myblue!20, fg=myblue}
\setbeamercolor{block body}{bg=myblue!10}
\setbeamercolor{alerted text}{fg=myorange}

% Set Fonts
\setbeamerfont{title}{size=\Large, series=\bfseries}
\setbeamerfont{frametitle}{size=\large, series=\bfseries}
\setbeamerfont{caption}{size=\small}
\setbeamerfont{footnote}{size=\tiny}

% Custom Commands
\newcommand{\concept}[1]{\textcolor{myblue}{\textbf{#1}}}
\newcommand{\hilight}[1]{\colorbox{myorange!30}{#1}}
\newcommand{\separator}{\begin{center}\rule{0.5\linewidth}{0.5pt}\end{center}}

% Title Page Information
\title[Data Processing Architectures]{Week 2: Data Processing Architectures}
\author[J. Smith]{John Smith, Ph.D.}
\institute[University Name]{
  Department of Computer Science\\
  University Name\\
  \vspace{0.3cm}
  Email: email@university.edu\\
  Website: www.university.edu
}
\date{\today}

% Document Start
\begin{document}

\frame{\titlepage}

\begin{frame}[fragile]
    \frametitle{Introduction to Data Processing Architectures}
    \begin{block}{Overview}
        Data processing architecture refers to the systematic framework that outlines how data is collected, processed, and analyzed. Understanding the architecture is crucial for determining the efficiency, scalability, and performance of data processing systems.
    \end{block}
\end{frame}

\begin{frame}[fragile]
    \frametitle{Key Concepts}
    \begin{enumerate}
        \item \textbf{Data Processing Architectures}:
        \begin{itemize}
            \item \textbf{Batch Processing}: Processes large volumes of data at once, typically on a scheduled basis.
            \item \textbf{Stream Processing}: Analyzes and processes continuous data streams in real-time.
        \end{itemize}
        \item \textbf{Importance of Data Processing Architectures}:
        \begin{itemize}
            \item \textbf{Performance}: How quickly data can be processed and insights generated.
            \item \textbf{Scalability}: The ability to handle increasing volumes of data over time.
            \item \textbf{Flexibility}: Adapting to different types of data and varied processing needs.
        \end{itemize}
    \end{enumerate}
\end{frame}

\begin{frame}[fragile]
    \frametitle{Batch Processing Overview}
    \begin{block}{Definition}
        Batch processing involves collecting and processing data in groups or 'batches'. It's commonly used in scenarios where immediate data output is not required.
    \end{block}
    
    \begin{columns}
        \column{0.5\textwidth}
            \textbf{Characteristics}:
            \begin{itemize}
                \item \textbf{Scheduled Execution}: Jobs run at specific intervals (e.g., nightly processing).
                \item \textbf{High Data Volume}: Efficient for processing large datasets.
                \item \textbf{Resource Utilization}: Optimizes resource use during specified times.
            \end{itemize}
        
        \column{0.5\textwidth}
            \textbf{Use Cases}:
            \begin{itemize}
                \item Monthly payroll processing
                \item Data warehouse updates
                \item Large-scale data transformations
            \end{itemize}
    \end{columns}
\end{frame}

\begin{frame}[fragile]
    \frametitle{Stream Processing Overview}
    \begin{block}{Definition}
        Stream processing deals with continuous data flow, enabling real-time analysis and response to data as it arrives.
    \end{block}
    
    \begin{columns}
        \column{0.5\textwidth}
            \textbf{Characteristics}:
            \begin{itemize}
                \item \textbf{Low Latency}: Provides immediate insights with minimal delay.
                \item \textbf{Event-Driven}: Processes data as events occur, such as clicks on a website.
                \item \textbf{Horizontal Scalability}: Can effortlessly scale out as data volume increases.
            \end{itemize}
        
        \column{0.5\textwidth}
            \textbf{Use Cases}:
            \begin{itemize}
                \item Monitoring social media feeds
                \item Real-time fraud detection
                \item Live analytics for online transactions
            \end{itemize}
    \end{columns}
\end{frame}

\begin{frame}[fragile]
    \frametitle{Key Points to Emphasize}
    \begin{itemize}
        \item Choosing between batch and stream processing depends on the specific needs of your application and the nature of the data.
        \item While batch processing is cost-effective for large data volumes, stream processing is essential for scenarios requiring instant insights.
    \end{itemize}
\end{frame}

\begin{frame}[fragile]
    \frametitle{Visual Representation}
    Consider the following conceptual diagram explaining the data flow in both architectures:
    
    \begin{center}
        \textbf{Data Flow Diagram}
        \begin{verbatim}
            +--------------------+         +---------------------+
            |   Incoming Data    | ------> |  Batch Processing   |
            +--------------------+         |   (Scheduled Jobs)  |
                    |                         +---------------------+
                    |                                  |
                    |                                  |
                    |                                  |
                    |                                  v
                    |                         +---------------------+
                    |                         | Stream Processing    |
                    +------------------------>| (Real-Time Data)    |
                                              +---------------------+
        \end{verbatim}
    \end{center}
\end{frame}

\begin{frame}[fragile]
    \frametitle{Batch Processing Overview - Definition}
    Batch processing refers to a method of executing a series of jobs or transactions in a grouped manner at a predetermined time. Unlike real-time processing, where data is processed immediately as it is received, batch processing accumulates data over a period and processes it all at once.

    \begin{block}{Key Characteristics}
        \begin{itemize}
            \item \textbf{Non-Real-Time:} Data is collected and processed later rather than immediately.
            \item \textbf{Large Data Volumes:} Capable of handling significant amounts of data at once.
            \item \textbf{Scheduled Execution:} Jobs are run according to a schedule, such as nightly or weekly.
            \item \textbf{Resource Efficiency:} Optimizes resource usage by avoiding idle times during data processing.
        \end{itemize}
    \end{block}
\end{frame}

\begin{frame}[fragile]
    \frametitle{Batch Processing Overview - Use Cases}
    Batch processing excels in scenarios such as:

    \begin{enumerate}
        \item \textbf{End-of-Day Reports:}
            \begin{itemize}
                \item Example: Banks aggregate transaction data at the end of the day for financial reports.
            \end{itemize}
        \item \textbf{Data Warehousing:}
            \begin{itemize}
                \item Example: Companies load large datasets into warehouses for analytical queries.
            \end{itemize}
        \item \textbf{Log Processing:}
            \begin{itemize}
                \item Example: Web servers compile logs into reports at scheduled intervals.
            \end{itemize}
        \item \textbf{Data Migration:}
            \begin{itemize}
                \item Example: Transferring data from legacy systems to new platforms in batches.
            \end{itemize}
    \end{enumerate}
\end{frame}

\begin{frame}[fragile]
    \frametitle{Batch Processing Overview - Summary Points}
    \begin{block}{Key Takeaways}
        \begin{itemize}
            \item Batch processing is ideal for tasks that do not require immediate results but need processing of large datasets.
            \item Common scenarios include reporting, data warehousing, log assessment, and maintenance tasks.
            \item The efficiency gained can enhance operational productivity and reduce costs in data-intensive environments.
        \end{itemize}
    \end{block}

    \begin{block}{Example Scenario}
        \textbf{Customer Data Processing:}
        \begin{itemize}
            \item A retail company collects customer data throughout the week.
            \item At week's end, a batch job analyzes trends and generates summary reports to inform marketing strategies.
        \end{itemize}
    \end{block}
\end{frame}

\begin{frame}[fragile]
    \frametitle{Stream Processing Overview}
    \begin{block}{Definition of Stream Processing}
        Stream Processing is a method of computing where data is continuously inputted and processed in real-time. It contrasts with batch processing, which works on fixed datasets. This approach allows for immediate insights and actions based on live data.
    \end{block}
\end{frame}

\begin{frame}[fragile]
    \frametitle{Characteristics of Stream Processing}
    \begin{enumerate}
        \item \textbf{Real-time Data Handling}: Analyzes data in real-time, offering immediate feedback and decision-making opportunities.
        \item \textbf{Event-driven Architecture}: Reacts to incoming data events rather than waiting for a complete dataset.
        \item \textbf{Scalability}: Can scale horizontally to handle increased data loads effectively by adding more processing units.
        \item \textbf{Low Latency}: Designed to minimize latency, which is crucial for applications requiring immediate responses.
        \item \textbf{Fault Tolerance}: Incorporates mechanisms to handle failures gracefully, ensuring data is not lost.
    \end{enumerate}
\end{frame}

\begin{frame}[fragile]
    \frametitle{Use Cases and Scenarios}
    \begin{itemize}
        \item \textbf{Financial Trading}: Allows traders to analyze market data in real-time for immediate trading decisions.
        \item \textbf{IoT Data Analysis}: Enhances operational efficiency by continuously monitoring and responding to data from sensors and smart devices.
        \item \textbf{Social Media Monitoring}: Enables brands to monitor feeds for trends and sentiments, responding quickly to customer interactions.
        \item \textbf{Fraud Detection}: Facilitates real-time detection of fraudulent transactions, allowing for quicker intervention.
    \end{itemize}
\end{frame}

\begin{frame}[fragile]
    \frametitle{Comparison of Batch and Stream Processing}
    \begin{block}{Overview}
        Batch and stream processing are two fundamental paradigms in data architectures. Understanding their differences is crucial for selecting the appropriate approach for specific applications.
    \end{block}
\end{frame}

\begin{frame}[fragile]
    \frametitle{Key Concepts: Batch Processing}
    \begin{itemize}
        \item \textbf{Definition:} Data is collected and processed together as a single unit (batch).
        \item \textbf{Characteristics:}
            \begin{itemize}
                \item Delayed execution: Jobs executed at specific intervals.
                \item Suitable for large datasets where real-time processing is not critical.
            \end{itemize}
    \end{itemize}
\end{frame}

\begin{frame}[fragile]
    \frametitle{Key Concepts: Stream Processing}
    \begin{itemize}
        \item \textbf{Definition:} Handles data continuously upon arrival for near-instantaneous processing.
        \item \textbf{Characteristics:}
            \begin{itemize}
                \item Real-time execution: Immediate insights or actions.
                \item Ideal for real-time analytics, such as social media monitoring.
            \end{itemize}
    \end{itemize}
\end{frame}

\begin{frame}[fragile]
    \frametitle{Key Differences: Performance and Latency}
    \begin{block}{Performance}
        \begin{itemize}
            \item \textbf{Batch Processing:} 
                \begin{itemize}
                    \item Processes large quantities at once; optimized for throughput.
                    \item Example: Processing monthly invoices yields comprehensive insights.
                \end{itemize}
            \item \textbf{Stream Processing:}
                \begin{itemize}
                    \item Handles continuous data; optimized for low-latency.
                    \item Example: Financial trader making decisions with real-time stock data.
                \end{itemize}
        \end{itemize}
    \end{block}

    \begin{block}{Latency}
        \begin{itemize}
            \item \textbf{Batch Processing:} 
                \begin{itemize}
                    \item Higher latency; results available only after batch processing is complete.
                    \item Average latency: minutes to hours or days.
                \end{itemize}
            \item \textbf{Stream Processing:}
                \begin{itemize}
                    \item Very low latency (milliseconds); data processed immediately.
                    \item Ideal for time-sensitive applications like fraud detection.
                \end{itemize}
        \end{itemize}
    \end{block}
\end{frame}

\begin{frame}[fragile]
    \frametitle{Key Differences: Resource Usage and Summary}
    \begin{block}{Resource Usage}
        \begin{itemize}
            \item \textbf{Batch Processing:}
                \begin{itemize}
                    \item More efficient resource usage; scheduled within specific intervals.
                    \item Requires sufficient resources at batch execution time.
                \end{itemize}
            \item \textbf{Stream Processing:}
                \begin{itemize}
                    \item Requires continuous resource allocation for real-time processing.
                    \item Variable resource consumption may occur.
                \end{itemize}
        \end{itemize}
    \end{block}

    \begin{block}{Summary of When to Use}
        \begin{itemize}
            \item \textbf{Batch Processing:} Large volumes of historical data, delayed tasks (e.g., monthly reports).
            \item \textbf{Stream Processing:} Real-time analytics and immediate event responses.
        \end{itemize}
    \end{block}
\end{frame}

\begin{frame}[fragile]
    \frametitle{Conclusion}
    \begin{itemize}
        \item Understanding the differences between batch and stream processing is vital for effective decision-making in data architectures.
        \item Align processing capabilities with business needs and objectives.
    \end{itemize}
    
    \begin{block}{Key Points to Remember}
        \begin{itemize}
            \item \textbf{Batch:} Best for volume; high latency; optimized for throughput.
            \item \textbf{Stream:} Best for speed; low latency; continuous processing.
        \end{itemize}
    \end{block}
\end{frame}

\begin{frame}[fragile]
    \frametitle{Use Cases for Batch Processing}
    \begin{block}{What is Batch Processing?}
        Batch processing involves executing a series of jobs on accumulated data at scheduled intervals. It handles data in groups, making it suitable for scenarios that do not require immediate results.
    \end{block}
\end{frame}

\begin{frame}[fragile]
    \frametitle{Ideal Use Cases for Batch Processing - Part 1}
    \begin{enumerate}
        \item \textbf{Data Warehousing}
        \begin{itemize}
            \item \textit{Description:} Consolidates large volumes of data from different sources into a central repository for analysis and reporting.
            \item \textit{Example:} A retail company processes accumulated sales data nightly for comprehensive reports on sales performance and inventory management.
        \end{itemize}

        \item \textbf{End-of-Day Reporting}
        \begin{itemize}
            \item \textit{Description:} Compiles daily business activities, finances, and operations reports.
            \item \textit{Example:} A bank updates transaction records after trading hours to manage balances and regulatory compliance.
        \end{itemize}
    \end{enumerate}
\end{frame}

\begin{frame}[fragile]
    \frametitle{Ideal Use Cases for Batch Processing - Part 2}
    \begin{enumerate}
        \setcounter{enumi}{2} % continue the enumeration
        \item \textbf{ETL Processes (Extract, Transform, Load)}
        \begin{itemize}
            \item \textit{Description:} Extracts data from various sources, transforms it, and loads into a data store.
            \item \textit{Example:} A healthcare provider standardizes patient records during non-peak hours for central management.
        \end{itemize}

        \item \textbf{Financial Data Processing}
        \begin{itemize}
            \item \textit{Description:} Processes batches of transactions for account reconciliation and financial statements.
            \item \textit{Example:} An insurance company processes claims monthly to analyze trends and prepare financial reports.
        \end{itemize}

        \item \textbf{Log Analysis}
        \begin{itemize}
            \item \textit{Description:} Analyzes server logs that are too large for real-time processing.
            \item \textit{Example:} A cloud provider reviews weekly logs for usage patterns and potential security breaches.
        \end{itemize}
    \end{enumerate}
\end{frame}

\begin{frame}[fragile]
    \frametitle{Key Points and Summary}
    \begin{itemize}
        \item \textbf{Efficiency:} Optimizes resources by consolidating operations, reducing processing overhead.
        \item \textbf{Scheduled Execution:} Jobs run at predefined intervals, allowing for managed system loads.
        \item \textbf{Historical Data Analysis:} Well-suited for applications relying on historical rather than immediate data insights.
    \end{itemize}
    
    \vspace{10pt}
    
    \begin{block}{Summary}
        Batch processing is essential in scenarios requiring non-immediate data processing, facilitating efficient data management and analytics across various industries.
    \end{block}
\end{frame}

\begin{frame}[fragile]
    \frametitle{Overview}
    Stream processing refers to the continuous ingestion, processing, and analysis of data in real-time. This approach allows organizations to derive immediate insights and take timely actions based on the data flowing in continuously.
\end{frame}

\begin{frame}[fragile]
    \frametitle{Key Concepts}
    \begin{enumerate}
        \item \textbf{Real-Time Data Processing}: Handles data as it arrives, essential for applications needing up-to-the-second information.
        \item \textbf{Event-driven Architecture}: Events trigger processing logic almost instantaneously.
    \end{enumerate}
\end{frame}

\begin{frame}[fragile]
    \frametitle{Use Cases - Real-Time Analytics}
    \begin{itemize}
        \item \textbf{Example}: Online Retail Analytics
        \item \textbf{Scenario}: E-commerce platform processes user interactions in real-time.
        \item \textbf{Benefit}: Allows adjustments to marketing strategies based on current user behavior.
    \end{itemize}
\end{frame}

\begin{frame}[fragile]
    \frametitle{Use Cases - IoT Sensor Data Monitoring}
    \begin{itemize}
        \item \textbf{Example}: Smart Home Devices
        \item \textbf{Scenario}: Smart thermostats track temperature and user behavior continuously.
        \item \textbf{Benefit}: Automates adjustments to optimize energy efficiency and comfort in real-time.
    \end{itemize}
\end{frame}

\begin{frame}[fragile]
    \frametitle{Use Cases - Fraud Detection}
    \begin{itemize}
        \item \textbf{Example}: Financial Transactions
        \item \textbf{Scenario}: Banks monitor transactions in real-time for unusual patterns indicating fraud.
        \item \textbf{Benefit}: Immediate alerts can prevent successful fraudulent transactions.
    \end{itemize}
\end{frame}

\begin{frame}[fragile]
    \frametitle{Use Cases - Social Media Monitoring}
    \begin{itemize}
        \item \textbf{Example}: Sentiment Analysis
        \item \textbf{Scenario}: Companies analyze social media feeds to gauge public sentiment during product launches.
        \item \textbf{Benefit}: Allows for prompt adaptations in messaging and strategies.
    \end{itemize}
\end{frame}

\begin{frame}[fragile]
    \frametitle{Use Cases - Network and Application Monitoring}
    \begin{itemize}
        \item \textbf{Example}: IT Infrastructure Management
        \item \textbf{Scenario}: Companies monitor network traffic and server performance in real-time.
        \item \textbf{Benefit}: Ensures system reliability and minimizes downtime with immediate corrective actions.
    \end{itemize}
\end{frame}

\begin{frame}[fragile]
    \frametitle{Key Points to Emphasize}
    \begin{itemize}
        \item \textbf{Latency}: Minimizes the time between data generation and insight generation.
        \item \textbf{Scalability}: Frameworks like Apache Kafka, Apache Flink, and Apache Spark Streaming handle large volumes of data.
        \item \textbf{Flexibility}: Integrates with various data sources and supports various data formats.
    \end{itemize}
\end{frame}

\begin{frame}[fragile]
    \frametitle{Conclusion}
    Stream processing enables businesses to respond intelligently and swiftly to events in real-time. Understanding where and how to apply this technology is crucial for leveraging its full potential in today's data-driven environment.
\end{frame}

\begin{frame}[fragile]
    \frametitle{Diagram Idea}
    \begin{block}{Conceptual Diagram of Stream Processing Lifecycle}
        - \textbf{Data Ingestion}: Events from various sources (sensors, user inputs). \\
        - \textbf{Processing Layer}: Event processing logic applying transformations and analyses. \\
        - \textbf{Output}: Aggregate results sent to dashboards or other systems for visualization.
    \end{block}
\end{frame}

\begin{frame}[fragile]
    \frametitle{Hybrid Architectures}
    \begin{block}{Introduction to Hybrid Architectures}
        Hybrid architectures combine \textbf{batch processing} and \textbf{stream processing}, leveraging the strengths of both to create a unified approach to data handling.
        This enables organizations to address various data workloads in a more efficient manner.
    \end{block}
\end{frame}

\begin{frame}[fragile]
    \frametitle{Understanding Processing Types}
    \begin{itemize}
        \item \textbf{Batch Processing}: 
            \begin{itemize}
                \item Involves running large volumes of data in chunks.
                \item Typically scheduled at specific intervals.
                \item Best for non-real-time tasks (e.g., monthly sales reports).
            \end{itemize}
        \item \textbf{Stream Processing}:
            \begin{itemize}
                \item Processes data in real-time as it arrives.
                \item Ideal for immediate insights (e.g., fraud detection in financial transactions).
            \end{itemize}
    \end{itemize}
\end{frame}

\begin{frame}[fragile]
    \frametitle{Advantages of Hybrid Architectures}
    \begin{enumerate}
        \item \textbf{Flexible Data Handling}
            \begin{itemize}
                \item Select processing methods as needed.
                \item Example: Real-time monitoring vs. historical data analysis.
            \end{itemize}
        
        \item \textbf{Efficiency and Cost-effectiveness}
            \begin{itemize}
                \item Reduces redundant processing.
                \item Saves on infrastructure and operational costs.
            \end{itemize}
        
        \item \textbf{Comprehensive Insights}
            \begin{itemize}
                \item Analyze both historical and real-time data simultaneously.
                \item Example: Monitoring inventory levels and analyzing past sales.
            \end{itemize}

        \item \textbf{Improved Scalability}
            \begin{itemize}
                \item Efficient workload distribution across systems.
            \end{itemize}

        \item \textbf{Enhanced Resilience and Fault Tolerance}
            \begin{itemize}
                \item If one system fails, the other can maintain operations.
            \end{itemize}
    \end{enumerate}
\end{frame}

\begin{frame}[fragile]
    \frametitle{Key Points}
    \begin{itemize}
        \item Hybrid architectures offer a flexible and efficient way to manage diverse data processing needs, catering to both immediate and strategic insights.
        \item Choosing the right method based on processing needs can lead to cost savings and better performance.
        \item Combining both batch and stream processing improves overall data strategy.
    \end{itemize}
\end{frame}

\begin{frame}[fragile]
    \frametitle{Example Architecture Diagram}
    \begin{center}
        \texttt{
        +----------------------------+\\
        |        Batch Processing    |\\
        | (Scheduled Jobs, ETL Tasks)|\\
        +----------------------------+\\
        \quad \quad \quad \quad \quad \quad \quad \quad \quad \quad \quad \quad\\
        +----------------------------+\\
        |      Stream Processing      |\\
        | (Real-time Analytics)      |\\
        +----------------------------+ \\
         \downarrow \\
        +-----------------------+\\
        |     Hybrid System     |\\
        | (Integrated Insights) |\\
        +-----------------------+
        }
    \end{center}
\end{frame}

\begin{frame}[fragile]
  \frametitle{Performance Considerations - Overview}
  \begin{itemize}
    \item Understanding performance considerations is crucial for efficient data processing systems.
    \item Key factors analyzed:
    \begin{itemize}
      \item Scalability
      \item Fault Tolerance
    \end{itemize}
    \item Implications on overall system performance are discussed.
  \end{itemize}
\end{frame}

\begin{frame}[fragile]
  \frametitle{Performance Considerations - Performance Factors}
  \begin{block}{Scalability}
    \begin{itemize}
      \item \textbf{Batch Processing:}
        \begin{itemize}
          \item Handles large datasets over fixed time.
          \item Scales by adding machines or increasing computing power.
          \item Example: Adding nodes to a Hadoop cluster as data volume doubles.
        \end{itemize}
      \item \textbf{Stream Processing:}
        \begin{itemize}
          \item Processes data in real-time with continuous input.
          \item Scalable by partitioning streams to distribute workloads.
          \item Example: Apache Kafka allows splitting topics across partitions for parallel processing.
        \end{itemize}
    \end{itemize}
  \end{block}
\end{frame}

\begin{frame}[fragile]
  \frametitle{Performance Considerations - Fault Tolerance}
  \begin{block}{Fault Tolerance}
    \begin{itemize}
      \item \textbf{Batch Processing:}
        \begin{itemize}
          \item Built-in data replication and error recovery (e.g., Hadoop).
          \item Tasks can be retried without data loss; tracks processed tasks.
          \item Example: Batch job restarts from the last successful checkpoint if a node fails.
        \end{itemize}
      \item \textbf{Stream Processing:}
        \begin{itemize}
          \item Requires reactive failure handling in real-time.
          \item Uses watermarking and checkpoints to ensure no event loss.
          \item Example: Apache Flink maintains state snapshots for recovery on failure.
        \end{itemize}
    \end{itemize}
  \end{block}
\end{frame}

\begin{frame}[fragile]
  \frametitle{Performance Considerations - Key Points}
  \begin{itemize}
    \item \textbf{Choosing the Right Architecture:}
      \begin{itemize}
        \item Decision based on use case's performance needs: latency and data volume.
      \end{itemize}
    \item \textbf{Balancing Scalability and Fault Tolerance:}
      \begin{itemize}
        \item High scalability should not compromise fault tolerance.
        \item Robust architectures balance both for uninterrupted processing.
      \end{itemize}
  \end{itemize}
\end{frame}

\begin{frame}[fragile]
  \frametitle{Performance Considerations - Design Considerations}
  \begin{itemize}
    \item \textbf{Performance Impact of Data Volume:}
      \begin{itemize}
        \item Increased volume slows batch processing; may need more computational resources.
        \item More events in stream processing could lead to latency unless designed elastically.
      \end{itemize}
    \item \textbf{Complexity of State Management:}
      \begin{itemize}
        \item State management in stream processing (e.g., windowing) can be complex.
        \item Critical to handle and recover state after failures properly.
      \end{itemize}
  \end{itemize}
\end{frame}

\begin{frame}[fragile]
  \frametitle{Performance Considerations - Conclusion}
  \begin{itemize}
    \item Performance considerations are central to efficient data processing system design.
    \item Understanding scalability and fault tolerance helps in creating better hybrid architectures.
    \item Leverage strengths of both batch and stream processing for optimal performance.
  \end{itemize}
\end{frame}

\begin{frame}[fragile]
    \frametitle{Conclusion - Recap of Key Points}
    In this segment, we summarize the distinctions between batch processing and stream processing, highlighting when to utilize each based on specific operational needs.
\end{frame}

\begin{frame}[fragile]
    \frametitle{Key Differences}
    \begin{block}{Batch Processing}
        \begin{itemize}
            \item \textbf{Definition}: Processes large volumes of data collected over a specific period at scheduled intervals.
            \item \textbf{Use Cases}: Ideal for applications where speed is less critical.
            \item \textbf{Examples}: Monthly financial reporting, end-of-day sales processing.
            \item \textbf{Advantages}:
            \begin{itemize}
                \item Efficient handling of large datasets.
                \item Suitable for complex calculations.
            \end{itemize}
            \item \textbf{Disadvantages}:
            \begin{itemize}
                \item Delayed insights due to processing time.
                \item Not suitable for immediate response needs.
            \end{itemize}
        \end{itemize}
    \end{block}

    \begin{block}{Stream Processing}
        \begin{itemize}
            \item \textbf{Definition}: Continuous input and processing of data for real-time analysis.
            \item \textbf{Use Cases}: Scenarios where timely insights are crucial.
            \item \textbf{Examples}: Real-time fraud detection, live data analytics.
            \item \textbf{Advantages}:
            \begin{itemize}
                \item Enables real-time decision-making.
                \item Reacts to events instantly.
            \end{itemize}
            \item \textbf{Disadvantages}:
            \begin{itemize}
                \item More complex implementation.
                \item Requires more computational resources.
            \end{itemize}
        \end{itemize}
    \end{block}
\end{frame}

\begin{frame}[fragile]
    \frametitle{When to Use Which?}
    \begin{block}{Choose Batch Processing When}
        \begin{itemize}
            \item Data analysis can tolerate latency.
            \item Complex transformations on large datasets are needed.
            \item The workload is massive for scheduled processing.
        \end{itemize}
    \end{block}

    \begin{block}{Choose Stream Processing When}
        \begin{itemize}
            \item Immediate insights and responses are required.
            \item Data is generated continuously and needs real-time analysis.
            \item Use cases involve real-time analytics or monitoring.
        \end{itemize}
    \end{block}

    \begin{block}{Summary}
        Understanding batch vs. stream processing allows organizations to align their data strategies with business objectives. Evaluating speed, volume, and complexity helps in making informed architecture choices.
    \end{block}
\end{frame}


\end{document}