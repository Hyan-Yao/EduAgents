\documentclass[aspectratio=169]{beamer}

% Theme and Color Setup
\usetheme{Madrid}
\usecolortheme{whale}
\useinnertheme{rectangles}
\useoutertheme{miniframes}

% Additional Packages
\usepackage[utf8]{inputenc}
\usepackage[T1]{fontenc}
\usepackage{graphicx}
\usepackage{booktabs}
\usepackage{listings}
\usepackage{amsmath}
\usepackage{amssymb}
\usepackage{xcolor}
\usepackage{tikz}
\usepackage{pgfplots}
\pgfplotsset{compat=1.18}
\usetikzlibrary{positioning}
\usepackage{hyperref}

% Set Theme Colors
\definecolor{myblue}{RGB}{31, 73, 125}
\definecolor{myorange}{RGB}{230, 126, 34}

\setbeamercolor{structure}{fg=myblue}
\setbeamercolor{frametitle}{fg=white, bg=myblue}
\setbeamercolor{alerted text}{fg=myorange}

% Title Page Information
\title[Chapter 8: Midterm Exam]{Chapter 8: Midterm Exam}
\author[J. Smith]{John Smith, Ph.D.}
\institute[University Name]{
  Department of Computer Science\\
  University Name\\
  \vspace{0.3cm}
  Email: email@university.edu\\
  Website: www.university.edu
}
\date{\today}

% Document Start
\begin{document}

\frame{\titlepage}

\begin{frame}[fragile]
    \frametitle{Introduction to Midterm Exam}
    \begin{block}{Overview}
        This presentation provides a comprehensive overview of the purpose and importance of the midterm exam, which covers topics from Weeks 1-7.
    \end{block}
\end{frame}

\begin{frame}[fragile]
    \frametitle{Purpose of the Midterm Exam}
    The midterm exam serves as a critical assessment tool designed to evaluate your understanding of the course material covered in the first seven weeks. Its purposes include:
    \begin{enumerate}
        \item \textbf{Measuring Progress}: Gauge understanding of key concepts and skills.
        \item \textbf{Identifying Strengths and Weaknesses}: Gain insights into areas of proficiency and those needing improvement.
        \item \textbf{Encouraging Review}: Promote revisiting material for better retention and understanding.
    \end{enumerate}
\end{frame}

\begin{frame}[fragile]
    \frametitle{Importance of the Midterm Exam}
    \begin{itemize}
        \item \textbf{Foundation for Future Topics}: Foundational knowledge for advanced concepts in later weeks.
        \item \textbf{Boosting Confidence}: Successfully completing the exam enhances motivation.
        \item \textbf{Feedback Mechanism}: Provides structured feedback for informed study strategies.
    \end{itemize}
\end{frame}

\begin{frame}[fragile]
    \frametitle{Key Topics Covered (Weeks 1-7)}
    \begin{itemize}
        \item \textbf{Conceptual Framework}: Core theories and models.
        \item \textbf{Practical Application}: Application of concepts through problems and case studies.
        \item \textbf{Techniques and Methodologies}: Key techniques used in practical scenarios.
    \end{itemize}
\end{frame}

\begin{frame}[fragile]
    \frametitle{Examples of Key Areas to Review}
    \begin{itemize}
        \item \textbf{Example 1}: Revise statistical methods; specifically calculation and interpretation of standard deviation and mean.
        \item \textbf{Example 2}: Review programming basics; focus on syntax and functions from Week 6 code snippets.
    \end{itemize}
\end{frame}

\begin{frame}[fragile]
    \frametitle{Tips for Success}
    \begin{itemize}
        \item \textbf{Study Groups}: Collaborate to discuss complex concepts and quiz each other.
        \item \textbf{Practice Tests}: Familiarize yourself with the exam format through practice exams.
        \item \textbf{Clarify Doubts}: Reach out to your instructor with any questions for clarification.
    \end{itemize}
\end{frame}

\begin{frame}[fragile]
    \frametitle{Conclusion}
    By consolidating your understanding of these topics and preparing strategically, you'll be well-equipped to succeed in your midterm exam and beyond!
\end{frame}

\begin{frame}[fragile]
    \frametitle{Exam Structure - Overview}
    The midterm exam is designed to assess your understanding and application of key concepts covered in Weeks 1-7. The exam will consist of various types of questions, allowing for a comprehensive evaluation of your knowledge. Below is a breakdown of the exam structure.
\end{frame}

\begin{frame}[fragile]
    \frametitle{Exam Structure - Types of Questions}
    \begin{enumerate}
        \item \textbf{Multiple Choice Questions (MCQs)}
        \begin{itemize}
            \item \textbf{Description}: Select the most appropriate answer from provided choices.
            \item \textbf{Example}:
            \begin{block}{Question}
                What is the primary function of a database?
                \begin{itemize}
                    \item A) Data processing
                    \item B) Data storage
                    \item C) Data visualization
                    \item D) Data analysis
                \end{itemize}
            \end{block}
            \textbf{Correct Answer}: B) Data storage
        \end{itemize}

        \item \textbf{Short Answer Questions}
        \begin{itemize}
            \item \textbf{Description}: Provide a brief written response explaining a concept.
            \item \textbf{Example}:
            \begin{block}{Question}
                Describe the differences between a stack and a queue in data structures.
            \end{block}
            \textbf{Sample Response}: A stack is a Last In, First Out (LIFO) structure where the last element added is the first to be removed. In contrast, a queue is a First In, First Out (FIFO) structure where the first element added is the first to be removed.
        \end{itemize}

        \item \textbf{Coding Problems}
        \begin{itemize}
            \item \textbf{Description}: Write code to solve specific problems demonstrating programming skills.
            \item \textbf{Example}:
            \begin{block}{Question}
                Write a function in Python that takes a list of numbers and returns the list sorted in ascending order.
            \end{block}
            \begin{lstlisting}[language=Python]
def sort_numbers(numbers):
    return sorted(numbers)
            \end{lstlisting}
        \end{itemize}
    \end{enumerate}
\end{frame}

\begin{frame}[fragile]
    \frametitle{Exam Structure - Key Points}
    \begin{itemize}
        \item \textbf{Diversity of Question Types}: MCQs, short answers, and coding problems test theoretical knowledge and practical skills.
        \item \textbf{Preparation Strategy}:
        \begin{itemize}
            \item Review key concepts from the chapters.
            \item Practice coding examples similar to the exam.
            \item Engage in group discussions to clarify doubts.
        \end{itemize}
        \item \textbf{Time Management}: Be aware of the time allocated for each section to ensure completion of all question types.
        \item \textbf{Summary}: Understanding the exam structure helps focus study efforts. Prepare both theoretically and practically for success.
    \end{itemize}
\end{frame}

\begin{frame}
    \frametitle{Learning Objectives Overview}
    \begin{block}{Description}
        This slide presents the key learning objectives from Weeks 1-7 that will be assessed in the upcoming midterm exam. Mastery of these concepts is vital for a comprehensive understanding of the course material and successful performance on the exam.
    \end{block}
\end{frame}

\begin{frame}
    \frametitle{Key Learning Objectives - Part 1}
    \begin{enumerate}
        \item \textbf{Fundamentals of Machine Learning (Week 1)}
        \begin{itemize}
            \item Understand the definition and significance of machine learning.
            \item \textbf{Key Concepts:}
            \begin{itemize}
                \item \textbf{Definition:} Machine Learning (ML) enables computers to learn from data and improve over time.
                \item \textbf{Types of Machine Learning:}
                \begin{itemize}
                    \item Supervised Learning: Learning from labeled data.
                    \item Unsupervised Learning: Finding patterns in unlabeled data.
                    \item Reinforcement Learning: Learning through rewards and punishments.
                \end{itemize}
            \end{itemize}
        \end{itemize}

        \item \textbf{Data Preprocessing Techniques (Week 2)}
        \begin{itemize}
            \item Learn about data cleaning, normalization, and transformation.
            \item \textbf{Example Techniques:}
            \begin{itemize}
                \item Handling missing values (imputation)
                \item Feature scaling (Min-Max Scaling, Standardization)
                \item \textbf{Code Snippet:}
                \begin{lstlisting}[language=Python]
from sklearn.preprocessing import StandardScaler
scaler = StandardScaler()
scaled_data = scaler.fit_transform(data)
                \end{lstlisting}
            \end{itemize}
        \end{itemize}
    \end{enumerate}
\end{frame}

\begin{frame}
    \frametitle{Key Learning Objectives - Part 2}
    \begin{enumerate}
        \setcounter{enumi}{2} % Continue from the last enumerated point
        
        \item \textbf{Exploratory Data Analysis (Week 3)}
        \begin{itemize}
            \item Ability to visualize data and extract meaningful insights.
            \item \textbf{Key Visualizations:}
            \begin{itemize}
                \item Histograms, Box plots, Scatter plots
                \item Understanding correlations (Pearson correlation coefficient)
            \end{itemize}
        \end{itemize}

        \item \textbf{Key Algorithms in ML (Weeks 4-5)}
        \begin{itemize}
            \item Familiarity with fundamental ML algorithms.
            \item \textbf{Example Algorithms:}
            \begin{itemize}
                \item Linear Regression for prediction.
                \item Decision Trees for classification.
                \item \textbf{Formula for Linear Regression:}
                \begin{equation}
                y = mx + b
                \end{equation}
            \end{itemize}
        \end{itemize}
    \end{enumerate}
\end{frame}

\begin{frame}
    \frametitle{Key Learning Objectives - Part 3}
    \begin{enumerate}
        \setcounter{enumi}{4} % Continue from the last enumerated point
        
        \item \textbf{Model Evaluation Techniques (Week 6)}
        \begin{itemize}
            \item Different metrics for evaluating models, such as accuracy, precision, recall, F1 score.
            \item \textbf{Emphasis on:}
            \begin{itemize}
                \item Cross-validation: Ensuring models generalize well to unseen data.
                \item \textbf{Formula for F1 Score:}
                \begin{equation}
                F1 = \frac{2 \cdot (\text{Precision} \cdot \text{Recall})}{\text{Precision} + \text{Recall}}
                \end{equation}
            \end{itemize}
        \end{itemize}

        \item \textbf{Ethics in Machine Learning (Week 7)}
        \begin{itemize}
            \item Understanding the ethical implications of ML models.
            \item \textbf{Key Points:}
            \begin{itemize}
                \item Bias in algorithms and data privacy concerns.
                \item The importance of transparency and accountability in ML applications.
            \end{itemize}
        \end{itemize}
    \end{enumerate}
\end{frame}

\begin{frame}[fragile]
    \frametitle{Week 1 Highlights: Introduction to Machine Learning}
    \begin{block}{Overview of Machine Learning}
        Machine Learning (ML) is a subset of artificial intelligence (AI) focused on algorithms and statistical models that enable computers to learn patterns from data for decision-making and predictions without explicit instructions.
    \end{block}
\end{frame}

\begin{frame}[fragile]
    \frametitle{Key Concepts of Machine Learning}
    \begin{enumerate}
        \item \textbf{Types of Machine Learning:}
            \begin{itemize}
                \item \textbf{Supervised Learning:} Trains on labeled datasets.
                    \begin{itemize}
                        \item \textit{Example:} Predicting house prices using features such as size and location.
                        \item \textit{Algorithms:} Linear Regression, Decision Trees.
                    \end{itemize}
                    
                \item \textbf{Unsupervised Learning:} Learns from data without labeled outcomes.
                    \begin{itemize}
                        \item \textit{Example:} Customer segmentation based on purchasing behavior.
                        \item \textit{Algorithms:} K-Means Clustering, Hierarchical Clustering.
                    \end{itemize}
                    
                \item \textbf{Reinforcement Learning:} Learning through actions and rewards.
                    \begin{itemize}
                        \item \textit{Example:} AI trained to play a game with rewards for winning.
                        \item \textit{Algorithm:} Q-Learning.
                    \end{itemize}
            \end{itemize}
    \end{enumerate}
\end{frame}

\begin{frame}[fragile]
    \frametitle{Applications of Machine Learning}
    \begin{itemize}
        \item \textbf{Healthcare:} Predicting patient outcomes and disease diagnosis.
        \item \textbf{Finance:} Fraud detection based on transaction patterns.
        \item \textbf{Marketing:} Personalized product/service recommendations.
        \item \textbf{Autonomous Vehicles:} Object detection and navigation decision-making.
    \end{itemize}
\end{frame}

\begin{frame}[fragile]
    \frametitle{Key Formulae and Concepts}
    \begin{block}{Basic Equation for Linear Regression}
        \begin{equation}
            y = mx + c
        \end{equation}
        Where:
        \begin{itemize}
            \item \(y\): predicted output
            \item \(m\): slope of the line
            \item \(x\): input feature
            \item \(c\): y-intercept
        \end{itemize}
    \end{block}

    \begin{block}{Distance Formula for Clustering}
        \begin{equation}
            d = \sqrt{\sum_{i=1}^n (x_i - y_i)^2}
        \end{equation}
        Where:
        \begin{itemize}
            \item \(d\): distance between points
            \item \(x\), \(y\): points in n-dimensional space
        \end{itemize}
    \end{block}
\end{frame}

\begin{frame}[fragile]
    \frametitle{Emphasis Points}
    \begin{itemize}
        \item Distinguishing between the three types of Machine Learning is crucial for effective problem-solving.
        \item Real-world applications highlight the broad impact of ML across various industries.
        \item Mastering foundational elements prepares you for deeper explorations in subsequent weeks.
    \end{itemize}
\end{frame}

\begin{frame}[fragile]
    \frametitle{Week 2 Highlights: Supervised vs. Unsupervised Learning}
    \begin{block}{Overview}
        Discussion of the differences and situations to apply supervised and unsupervised learning.
    \end{block}
\end{frame}

\begin{frame}[fragile]
    \frametitle{Understanding Supervised Learning}
    \begin{itemize}
        \item \textbf{Definition}: A type of machine learning where the model is trained on a labeled dataset.
        \item \textbf{Goal}: Learn a mapping from features to labels for predicting outputs.
        \item \textbf{Common Algorithms}:
            \begin{itemize}
                \item \textbf{Linear Regression}: Predicting continuous values (e.g., house prices).
                \item \textbf{Logistic Regression}: For binary classification (e.g., spam detection).
                \item \textbf{Support Vector Machines (SVM)}: Effective in high-dimensional spaces.
            \end{itemize}
        \item \textbf{Example}: Predicting house prices based on factors like size and location.
    \end{itemize}
\end{frame}

\begin{frame}[fragile]
    \frametitle{Understanding Unsupervised Learning}
    \begin{itemize}
        \item \textbf{Definition}: A type of machine learning focused on identifying patterns without labeled responses.
        \item \textbf{Goal}: Explore the data's structure to find hidden patterns and groupings.
        \item \textbf{Common Algorithms}:
            \begin{itemize}
                \item \textbf{K-Means Clustering}: Groups similar data points (e.g., customer segmentation).
                \item \textbf{Principal Component Analysis (PCA)}: Reduces dimensionality while preserving variance.
            \end{itemize}
        \item \textbf{Example}: Customer segmentation for marketing using demographic data.
    \end{itemize}
\end{frame}

\begin{frame}[fragile]
    \frametitle{Key Differences}
    \begin{tabular}{|l|l|l|}
        \hline
        \textbf{Feature} & \textbf{Supervised Learning} & \textbf{Unsupervised Learning} \\
        \hline
        \textbf{Data Type} & Labeled data & Unlabeled data \\
        \hline
        \textbf{Output} & Predict a specific outcome & Discover patterns or structures \\
        \hline
        \textbf{Use Cases} & Classification, regression & Clustering, association \\
        \hline
        \textbf{Feedback} & Based on known labels & Exploration-based learning \\
        \hline
    \end{tabular}
\end{frame}

\begin{frame}[fragile]
    \frametitle{When to Use}
    \begin{itemize}
        \item \textbf{Use Supervised Learning} when:
            \begin{itemize}
                \item The problem requires predictions based on existing data (e.g., fraud detection).
                \item Labeled data is available and accurate.
            \end{itemize}
        \item \textbf{Use Unsupervised Learning} when:
            \begin{itemize}
                \item You have no labeled data and need to find hidden patterns.
                \item Exploratory data analysis is needed to understand distribution.
            \end{itemize}
    \end{itemize}
\end{frame}

\begin{frame}[fragile]
    \frametitle{Conclusion}
    Understanding the core differences and applications of supervised and unsupervised learning is crucial for selecting the right approach. The technique you choose greatly impacts the success of your machine learning project.
\end{frame}

\begin{frame}[fragile]
    \frametitle{Week 3 Highlights: Data Preprocessing}
    \begin{block}{Overview}
        Overview of the importance of data cleaning, normalization, and transformation techniques.
    \end{block}
\end{frame}

\begin{frame}[fragile]
    \frametitle{Importance of Data Preprocessing}
    \begin{itemize}
        \item Data preprocessing transforms raw data into a clean and usable format.
        \item It's critical for the quality of results in data analysis.
        \item Key phases include:
        \begin{itemize}
            \item Data Cleaning
            \item Normalization
            \item Transformation
        \end{itemize}
    \end{itemize}
\end{frame}

\begin{frame}[fragile]
    \frametitle{1. Data Cleaning}
    Data cleaning involves:
    \begin{itemize}
        \item Identifying and correcting inaccuracies or inconsistencies.
        \item Handling missing values and removing duplicates.
    \end{itemize}

    \begin{block}{Examples}
        \begin{itemize}
            \item \textbf{Handling Missing Values:}
            \begin{lstlisting}[language=Python]
df.fillna(df.mean(), inplace=True)  # Mean imputation in Pandas
            \end{lstlisting}
            \item \textbf{Removing Duplicates:}
            \begin{lstlisting}[language=Python]
df.drop_duplicates(inplace=True)  # Remove duplicate rows
            \end{lstlisting}
        \end{itemize}
    \end{block}
\end{frame}

\begin{frame}[fragile]
    \frametitle{2. Normalization}
    Normalization scales data points for improved algorithm performance:
    \begin{itemize}
        \item Crucial for distance-based algorithms (e.g., K-means clustering).
        \item Common Techniques:
        \begin{itemize}
            \item \textbf{Min-Max Scaling:}
            \begin{equation}
                X' = \frac{X - X_{\min}}{X_{\max} - X_{\min}}
            \end{equation}
            \item \textbf{Z-score Normalization:}
            \begin{equation}
                Z = \frac{(X - \mu)}{\sigma}
            \end{equation}
        \end{itemize}
    \end{itemize}
\end{frame}

\begin{frame}[fragile]
    \frametitle{3. Transformation}
    Transformation modifies data to suit modeling needs:
    \begin{itemize}
        \item Enhances model performance and meets algorithm assumptions.
        \item Common Technique:
        \begin{itemize}
            \item \textbf{Log Transformation:} Reduces skewness in data.
        \end{itemize}
    \end{itemize}
\end{frame}

\begin{frame}[fragile]
    \frametitle{Key Points and Conclusion}
    \begin{itemize}
        \item \textbf{Data Quality Matters:} Well-preprocessed data results in more reliable models.
        \item \textbf{Iterative Process:} Multiple rounds of cleaning and transformation may be necessary.
        \item \textbf{Algorithm-Relevant:} Consider specific preprocessing requirements based on models.
    \end{itemize}
    \begin{block}{Conclusion}
        Data preprocessing is foundational for data analysis and model building. Investing time in this process maximizes accuracy and effectiveness in data science.
    \end{block}
\end{frame}

\begin{frame}[fragile]
    \frametitle{Quick Recap}
    \begin{itemize}
        \item Data Cleaning: Address inaccuracies in data records.
        \item Normalization: Scale data to common ranges.
        \item Transformation: Suit data to model requirements.
    \end{itemize}
\end{frame}

\begin{frame}[fragile]
    \frametitle{Week 4 Highlights: Linear Models and Regression Analysis}
    % Overview of key concepts in linear and logistic regression as well as evaluation metrics.
    \begin{itemize}
        \item Focus on Linear Regression and Logistic Regression.
        \item Discuss model evaluation techniques for both types of regression.
    \end{itemize}
\end{frame}

\begin{frame}[fragile]
    \frametitle{Linear Regression Explored}
    \begin{itemize}
        \item \textbf{Definition:} A statistical method modeling the relationship between a dependent variable (\(Y\)) and independent variable(s) (\(X\)).
        \item \textbf{Equation:}
        \begin{equation}
        Y = \beta_0 + \beta_1X + \epsilon
        \end{equation}
        \begin{itemize}
            \item \(Y\): predicted value
            \item \(\beta_0\): y-intercept
            \item \(\beta_1\): slope
            \item \(X\): independent variable
            \item \(\epsilon\): error term
        \end{itemize}
        \item \textbf{Example:} Predicting house prices based on size:
        \begin{equation}
        \text{Price} = 50,000 + 200 \times \text{Size}
        \end{equation}
        \item \textbf{Key Points:} Assumes linear relationship; can use multiple independent variables.
    \end{itemize}
\end{frame}

\begin{frame}[fragile]
    \frametitle{Logistic Regression Explained}
    \begin{itemize}
        \item \textbf{Definition:} Used for categorical dependent variables; predicts probability of an event.
        \item \textbf{Equation:}
        \begin{equation}
        P(Y=1) = \frac{1}{1 + e^{-(\beta_0 + \beta_1X)}}
        \end{equation}
        \begin{itemize}
            \item \(P(Y=1)\): probability of event
            \item \(e\): base of natural logarithm
        \end{itemize}
        \item \textbf{Example:} Predicting whether a student will pass:
        \begin{equation}
        P(\text{Pass}) = \frac{1}{1 + e^{-(0.5 + 0.1 \times \text{Hours})}}
        \end{equation}
        \item \textbf{Key Points:} Suitable for binary outcomes; outputs probabilities for classification.
    \end{itemize}
\end{frame}

\begin{frame}[fragile]
    \frametitle{Evaluating Regression Models}
    \begin{itemize}
        \item \textbf{Linear Regression Metrics:}
        \begin{itemize}
            \item \(\mathbf{R^2}\): 
            \begin{equation}
            R^2 = 1 - \frac{\text{SS}_{\text{res}}}{\text{SS}_{\text{tot}}}
            \end{equation}
                - Measures variance explained.
            \item Mean Absolute Error (MAE) and Mean Squared Error (MSE):
            \begin{equation}
            MAE = \frac{1}{n}\sum_{i=1}^n |Y_i - \hat{Y}_i|, \quad MSE = \frac{1}{n}\sum_{i=1}^n (Y_i - \hat{Y}_i)^2
            \end{equation}
        \end{itemize}
        
        \item \textbf{Logistic Regression Metrics:}
        \begin{itemize}
            \item Confusion Matrix: Compares predictions to actual outcomes.
            \item Accuracy, Precision, Recall: Metrics for effectiveness.
            \item ROC Curve: Graph of true positive rate vs. false positive rate.
        \end{itemize}
    \end{itemize}
\end{frame}

\begin{frame}[fragile]
    \frametitle{Conclusion}
    \begin{block}{}
        Understanding linear and logistic regression is crucial for effective statistical modeling and data analysis. By applying these models properly and evaluating their performance, we can extract valuable insights from data for informed decision-making.
    \end{block}
\end{frame}

\begin{frame}[fragile]
    \frametitle{Week 5 Highlights: Decision Trees and Ensemble Methods}
    \begin{block}{Summary}
        This week covers an overview of Decision Tree classifiers and Ensemble Learning techniques, including Random Forests and Boosting, focusing on improving model performance and accuracy.
    \end{block}
\end{frame}

\begin{frame}[fragile]
    \frametitle{Decision Trees}
    \begin{itemize}
        \item \textbf{Definition:} A flowchart-like structure with nodes representing tests on attributes, branches for outcomes, and leaf nodes for class labels.
        \item \textbf{How It Works:}
            \begin{itemize}
                \item \textbf{Splitting:} Selects the best feature to split data.
                \item \textbf{Stopping Criterion:} Continues until reaching max depth or no further gain.
            \end{itemize}
        \item \textbf{Example:}
            \begin{lstlisting}
            Is attendance > 75%?
                  /       \
               Yes         No
                |           |
            Is study hours > 5? 
                  /         \
               Yes           No
                |             |
            Pass            Fail
            \end{lstlisting}
    \end{itemize}
\end{frame}

\begin{frame}[fragile]
    \frametitle{Advantages and Disadvantages of Decision Trees}
    \begin{itemize}
        \item \textbf{Advantages:}
            \begin{itemize}
                \item Easy to understand and interpret.
                \item Handles both numerical and categorical data.
            \end{itemize}
        \item \textbf{Disadvantages:}
            \begin{itemize}
                \item Prone to overfitting, especially with deep trees.
                \item Sensitive to noise in the data.
            \end{itemize}
    \end{itemize}
\end{frame}

\begin{frame}[fragile]
    \frametitle{Ensemble Methods}
    \begin{itemize}
        \item \textbf{Definition:} Combines multiple models for better predictive performance.
        \item \textbf{Random Forests:}
            \begin{itemize}
                \item Ensemble of decision trees using bootstrap aggregating (bagging).
                \item Produces outputs via majority voting.
            \end{itemize}
        \item \textbf{Boosting:}
            \begin{itemize}
                \item Sequential technique correcting previous model errors.
                \item Assigns more weight to misclassified instances.
            \end{itemize}
    \end{itemize}
\end{frame}

\begin{frame}[fragile]
    \frametitle{Key Concepts in Ensemble Methods}
    \begin{itemize}
        \item \textbf{Key Points:}
            \begin{itemize}
                \item Ensemble methods enhance accuracy and robustness.
                \item Decision trees are foundational for both Random Forests and Boosting.
            \end{itemize}
        \item \textbf{Key Formulas:}
            \begin{equation}
                Gini = 1 - \sum p_i^2
            \end{equation}
            \begin{equation}
                Entropy = - \sum p_i \log_2(p_i)
            \end{equation}
            \begin{equation}
                \hat{y} = \text{argmax}_c \sum_{t=1}^T I(y_t = c)
            \end{equation}
    \end{itemize}
\end{frame}

\begin{frame}[fragile]
    \frametitle{Conclusion and Next Steps}
    \begin{itemize}
        \item \textbf{Conclusion:} Understanding Decision Trees and Ensemble Methods is crucial for enhancing model accuracy in machine learning.
        \item \textbf{Next Steps:} Prepare for Week 6 focusing on Clustering and Dimensionality Reduction techniques.
    \end{itemize}
\end{frame}

\begin{frame}[fragile]
    \frametitle{Week 6 Highlights: Clustering and Dimensionality Reduction}
    \begin{block}{Overview}
        This week, we focus on three key methods in data analysis:
        \begin{itemize}
            \item K-Means Clustering
            \item Hierarchical Clustering
            \item Principal Component Analysis (PCA)
        \end{itemize}
    \end{block}
\end{frame}

\begin{frame}[fragile]
    \frametitle{K-Means Clustering}
    \begin{block}{Concept}
        A partitioning method that clusters data by grouping data points into K distinct, non-overlapping subsets (clusters).
    \end{block}
    \begin{block}{Process}
        \begin{enumerate}
            \item \textbf{Initialization}: Randomly select K points as initial cluster centroids.
            \item \textbf{Assignment}: Assign each data point to the closest centroid.
            \item \textbf{Update}: Calculate new centroids as the mean of assigned points.
            \item \textbf{Repeat}: Continue until convergence.
        \end{enumerate}
    \end{block}
    \begin{block}{Objective Function}
        Minimize the sum of squared distances:
        \begin{equation}
        J = \sum_{i=1}^{K} \sum_{x \in C_i} \| x - \mu_i \|^2 
        \end{equation}
    \end{block}
    \begin{block}{Example}
        Clustering customer data based on purchasing behavior to identify distinct market segments.
    \end{block}
\end{frame}

\begin{frame}[fragile]
    \frametitle{Hierarchical Clustering}
    \begin{block}{Concept}
        A method that builds a hierarchy of clusters:
        \begin{itemize}
            \item \textbf{Agglomerative (Bottom-Up)}: Start with individual points and merge them.
            \item \textbf{Divisive (Top-Down)}: Start with one large cluster and recursively divide it.
        \end{itemize}
    \end{block}
    \begin{block}{Dendrogram}
        A tree-like diagram showing the arrangement of clusters. The x-axis represents data points, and the y-axis indicates distance or dissimilarity.
    \end{block}
    \begin{block}{Key Distance Metrics}
        \begin{itemize}
            \item \textbf{Single Linkage}: Distance between the closest points of two clusters.
            \item \textbf{Complete Linkage}: Distance between the farthest points of two clusters.
            \item \textbf{Average Linkage}: Average distance between all points in two clusters.
        \end{itemize}
    \end{block}
    \begin{block}{Example}
        Organizing documents by similarity to assist in content management.
    \end{block}
\end{frame}

\begin{frame}[fragile]
    \frametitle{Principal Component Analysis (PCA)}
    \begin{block}{Concept}
        A dimensionality reduction technique that transforms high-dimensional data into a lower-dimensional space while preserving variance.
    \end{block}
    \begin{block}{Process}
        \begin{enumerate}
            \item \textbf{Standardize Data}: Center and scale the data.
            \item \textbf{Covariance Matrix}: Construct to understand variable relationships.
            \item \textbf{Eigen Decomposition}: Compute eigenvalues and eigenvectors.
            \item \textbf{Select Components}: Rank eigenvectors by eigenvalues; select the top K.
            \item \textbf{Transform Data}: Project original data onto top K eigenvectors.
        \end{enumerate}
    \end{block}
    \begin{block}{Formula}
        Projected data:
        \begin{equation}
        Z = X W 
        \end{equation}
    \end{block}
    \begin{block}{Example}
        Reducing dimensionality of images for faster processing in machine learning applications.
    \end{block}
\end{frame}

\begin{frame}[fragile]
    \frametitle{Key Points to Emphasize}
    \begin{itemize}
        \item \textbf{K-Means}: Efficient but sensitive to initial placement and requires K.
        \item \textbf{Hierarchical Clustering}: No initial K needed but computationally intensive.
        \item \textbf{PCA}: Useful for visualization and noise reduction but may lose interpretability.
    \end{itemize}
    \begin{block}{Code Snippets}
        \begin{lstlisting}[language=Python]
# K-Means Example using Scikit-Learn
from sklearn.cluster import KMeans
kmeans = KMeans(n_clusters=3)
kmeans.fit(data)
labels = kmeans.labels_

# PCA Example
from sklearn.decomposition import PCA
pca = PCA(n_components=2)
reduced_data = pca.fit_transform(data)
        \end{lstlisting}
    \end{block}
\end{frame}

\begin{frame}[fragile]
    \frametitle{Week 7 Highlights: Model Evaluation Metrics}
    In this section, we will review important metrics used to evaluate the performance of classification models. Understanding these metrics is crucial for making informed decisions about model selection and optimization.
\end{frame}

\begin{frame}[fragile]
    \frametitle{Key Evaluation Metrics}
    \begin{enumerate}
        \item \textbf{Accuracy}
        \item \textbf{Precision}
        \item \textbf{Recall}
        \item \textbf{F1-Score}
    \end{enumerate}
\end{frame}

\begin{frame}[fragile]
    \frametitle{Accuracy}
    \begin{block}{Definition}
        Accuracy measures the proportion of correctly classified instances among the total instances.
    \end{block}
    \begin{block}{Formula}
        \begin{equation}
            \text{Accuracy} = \frac{\text{TP} + \text{TN}}{\text{TP} + \text{TN} + \text{FP} + \text{FN}}
        \end{equation}
    \end{block}
    \begin{itemize}
        \item TP: True Positives
        \item TN: True Negatives
        \item FP: False Positives
        \item FN: False Negatives
    \end{itemize}
    \begin{block}{Example}
        In a dataset of 100 patients, if a model correctly classifies 90 as healthy (TN) and 5 as sick (TP), 
        the accuracy is \( \frac{90 + 5}{100} = 0.95 \) or 95\%.
    \end{block}
\end{frame}

\begin{frame}[fragile]
    \frametitle{Precision}
    \begin{block}{Definition}
        Precision, also known as Positive Predictive Value, measures the accuracy of positive predictions.
    \end{block}
    \begin{block}{Formula}
        \begin{equation}
            \text{Precision} = \frac{\text{TP}}{\text{TP} + \text{FP}}
        \end{equation}
    \end{block}
    \begin{block}{Example}
        If a model predicts 10 patients as sick, of which 5 are actually sick, the precision is 
        \( \frac{5}{10} = 0.5 \) or 50\%.
    \end{block}
\end{frame}

\begin{frame}[fragile]
    \frametitle{Recall}
    \begin{block}{Definition}
        Recall, or Sensitivity, measures the ability of a model to find all the relevant cases (true positives).
    \end{block}
    \begin{block}{Formula}
        \begin{equation}
            \text{Recall} = \frac{\text{TP}}{\text{TP} + \text{FN}}
        \end{equation}
    \end{block}
    \begin{block}{Example}
        If out of 15 actual sick patients, the model correctly identifies 5, then the recall is 
        \( \frac{5}{15} = 0.33 \) or 33\%.
    \end{block}
\end{frame}

\begin{frame}[fragile]
    \frametitle{F1-Score}
    \begin{block}{Definition}
        The F1-score is the harmonic mean of precision and recall and is a good measure when we need a balance between both.
    \end{block}
    \begin{block}{Formula}
        \begin{equation}
            \text{F1} = 2 \times \frac{\text{Precision} \times \text{Recall}}{\text{Precision} + \text{Recall}}
        \end{equation}
    \end{block}
    \begin{block}{Example}
        If precision is 50\% (0.5) and recall is 33\% (0.33), the F1-score can be calculated as:
        \[
        \text{F1} = 2 \times \frac{0.5 \times 0.33}{0.5 + 0.33} = 0.4 \text{ (approximately)}
        \]
    \end{block}
\end{frame}

\begin{frame}[fragile]
    \frametitle{Key Points to Emphasize}
    \begin{itemize}
        \item \textbf{Application Context}: Accuracy can be misleading when data is imbalanced. It's essential to consider precision and recall in such cases.
        \item \textbf{Trade-offs}: Increasing precision can decrease recall and vice versa. The F1-score provides a balanced measure.
        \item \textbf{Multi-class Classification}: For models that classify into more than two classes, metrics can be extended by calculating them for each class and averaging (macro and micro averages).
    \end{itemize}
\end{frame}

\begin{frame}[fragile]
    \frametitle{Conclusion}
    Understanding these metrics helps in selecting the right model and improving its performance. The choice of metric largely depends on the specific context and business problem at hand. Remember to apply these concepts practically, especially as you prepare for the upcoming midterm exam!
\end{frame}

\begin{frame}[fragile]
    \frametitle{Prepare for the Exam}
    \begin{block}{Introduction}
        Effective exam preparation is crucial for success in your midterm. By employing the right strategies, you can enhance your understanding and retention of the material. This slide outlines key tips and strategies to help you prepare effectively.
    \end{block}
\end{frame}

\begin{frame}[fragile]
    \frametitle{Study Strategies - Part 1}
    \begin{enumerate}
        \item \textbf{Create a Study Schedule:}
            \begin{itemize}
                \item Break down your study material into manageable sections.
                \item Allocate specific time slots for each topic, ensuring you cover all areas before the exam.
                \item \textit{Example:} Allocate 2 hours to review model evaluation metrics and 1 hour for practice problems.
            \end{itemize}

        \item \textbf{Understand Key Concepts:}
            \begin{itemize}
                \item Focus on understanding rather than memorization. Grasp concepts like accuracy, precision, and F1-score.
                \item Use diagrams or charts to visualize relationships between these metrics.
            \end{itemize}
    \end{enumerate}
\end{frame}

\begin{frame}[fragile]
    \frametitle{Study Strategies - Part 2}
    \begin{enumerate}
        \setcounter{enumi}{2}
        \item \textbf{Practice with Sample Questions:}
            \begin{itemize}
                \item Use sample questions to familiarize yourself with exam formats.
                \item Engage in active problem-solving to apply your knowledge.
            \end{itemize}

        \item \textbf{Form Study Groups:}
            \begin{itemize}
                \item Collaborate with classmates to discuss complex topics.
                \item Teaching others can reinforce your understanding and reveal gaps in your knowledge.
            \end{itemize}

        \item \textbf{Utilize Online Resources:}
            \begin{itemize}
                \item Explore MOOCs or YouTube tutorials for additional explanations and examples.
                \item Practical applications of model evaluation metrics can be beneficial.
            \end{itemize}
    \end{enumerate}
\end{frame}

\begin{frame}[fragile]
    \frametitle{Key Points and Conclusion}
    \begin{enumerate}
        \item \textbf{Review Mistakes:}
            \begin{itemize}
                \item Analyze mistakes made during practice and understand why they were incorrect.
                \item Reflection on errors is key to improvement.
            \end{itemize}

        \item \textbf{Active Learning:}
            Engage actively with the material—don’t just read; solve problems and explain concepts out loud.

        \item \textbf{Consistent Review:}
            Regular, spaced repetition of concepts leads to better long-term retention compared to cramming.

        \item \textbf{Healthy Study Habits:}
            Ensure you take breaks, stay healthy, and get adequate sleep for mental clarity.

        \item \textbf{Formula Key:}
            \[
            \text{Precision} = \frac{TP}{TP + FP}
            \]
            where TP = True Positives, FP = False Positives.

    \end{enumerate}

    \begin{block}{Conclusion}
        By implementing these strategies, you can prepare effectively for the midterm exam and boost your confidence on exam day. Prepare well, and believe in your ability to succeed!
    \end{block}
\end{frame}

\begin{frame}[fragile]
    \frametitle{Sample Questions - Overview}
    \begin{block}{Understanding the Midterm Exam}
        In preparation for the midterm exam, it’s essential to familiarize yourself with the types of questions that may be presented. 
        This slide showcases sample questions that represent the content and complexity you might encounter, ensuring you are well-equipped for the exam.
    \end{block}
\end{frame}

\begin{frame}[fragile]
    \frametitle{Sample Questions - Conceptual and Application}
    \begin{enumerate}
        \item \textbf{Conceptual Understanding}
        \begin{itemize}
            \item \textbf{Question}: Explain the difference between supervised and unsupervised learning. Provide one real-world example for each.
            \item \textbf{Key Points}:
            \begin{itemize}
                \item \textbf{Supervised Learning}: Involves training a model on labeled data. Example: Email spam detection.
                \item \textbf{Unsupervised Learning}: Involves training a model on data without labeled responses. Example: Customer segmentation in marketing.
            \end{itemize}
        \end{itemize}
        
        \item \textbf{Application of Techniques}
        \begin{itemize}
            \item \textbf{Question}: Describe the steps involved in constructing a confusion matrix from a classification model output.
            \item \textbf{Key Points}:
            \begin{itemize}
                \item True Positives (TP): Correct predictions of positive class.
                \item True Negatives (TN): Correct predictions of negative class.
                \item False Positives (FP): Incorrect predictions of negative class.
                \item False Negatives (FN): Incorrect predictions of positive class.
            \end{itemize}
            \item \textbf{Formula}:
            \begin{equation}
            \text{Accuracy} = \frac{TP + TN}{TP + TN + FP + FN}
            \end{equation}
        \end{itemize}
    \end{enumerate}
\end{frame}

\begin{frame}[fragile]
    \frametitle{Sample Questions - Mathematics and Ethics}
    \begin{enumerate}
        \setcounter{enumi}{2} % continue numbering
        \item \textbf{Mathematical Concepts}
        \begin{itemize}
            \item \textbf{Question}: Given a dataset, how would you calculate the mean, median, and mode? Use the following dataset as an example: [5, 8, 10, 8, 7].
            \item \textbf{Key Points}:
            \begin{itemize}
                \item \textbf{Mean}: (5 + 8 + 10 + 8 + 7) / 5 = 7.6
                \item \textbf{Median}: Value in the middle when arranged in order (5, 7, 8, 8, 10) = 8
                \item \textbf{Mode}: The value that appears most frequently = 8
            \end{itemize}
        \end{itemize}

        \item \textbf{Ethical Consideration}
        \begin{itemize}
            \item \textbf{Question}: Discuss the importance of fairness in machine learning models and one example of algorithmic bias.
            \item \textbf{Key Points}:
            \begin{itemize}
                \item Fairness ensures that models do not discriminate against any groups.
                \item Example: A hiring algorithm may favor candidates from certain demographics if not properly trained or validated.
            \end{itemize}
        \end{itemize}
    \end{enumerate}
\end{frame}

\begin{frame}[fragile]
    \frametitle{Conclusion and Key Points}
    \begin{block}{Key Points to Emphasize}
        \begin{itemize}
            \item \textbf{Study Techniques}: Review these sample questions and practice articulating your answers to solidify your understanding.
            \item \textbf{Diverse Question Types}: The exam may include conceptual questions, technical applications, and ethical discussions.
            \item \textbf{Practice Regularly}: Regularly practice answering these types of questions to prepare efficiently.
        \end{itemize}
    \end{block}

    \begin{block}{Conclusion}
        Familiarity with the format and content of potential exam questions will enhance your confidence and performance. 
        Utilize these examples to guide your studies and identify areas where you may need further clarification or practice.
    \end{block}
\end{frame}

\begin{frame}[fragile]
    \frametitle{Ethical Considerations - Introduction}
    In the realm of machine learning (ML), ethical considerations are crucial in ensuring that technology serves humanity responsibly. 
    As ML systems become more integrated into decision-making processes, understanding the moral implications is essential for both developers and users.
\end{frame}

\begin{frame}[fragile]
    \frametitle{Ethical Considerations - Key Issues}
    \begin{enumerate}
        \item \textbf{Bias and Fairness}
        \begin{itemize}
            \item \textit{Definition:} Unfair advantages or disadvantages to groups based on biased training data.
            \item \textit{Example:} A hiring algorithm favoring certain demographics.
            \item \textit{Reinforcement:} Audit datasets for bias; use techniques like re-sampling.
        \end{itemize}

        \item \textbf{Transparency}
        \begin{itemize}
            \item \textit{Definition:} Understandability of ML algorithms to users and stakeholders.
            \item \textit{Example:} Credit scoring models should explain loan denial reasons.
            \item \textit{Reinforcement:} Use tools like SHAP values or LIME for model explainability.
        \end{itemize}
    \end{enumerate}
\end{frame}

\begin{frame}[fragile]
    \frametitle{Ethical Considerations - Key Issues (cont.)}
    \begin{enumerate}
        \setcounter{enumi}{2} % Continue enumerating from previous frame
        \item \textbf{Privacy}
        \begin{itemize}
            \item \textit{Definition:} Handling of personal data respecting individual rights.
            \item \textit{Example:} Facial recognition systems infringing on privacy without consent.
            \item \textit{Reinforcement:} Use data anonymization techniques; adhere to GDPR regulations.
        \end{itemize}

        \item \textbf{Accountability}
        \begin{itemize}
            \item \textit{Definition:} Responsibility for outcomes produced by AI systems.
            \item \textit{Example:} Liability in case of harm caused by autonomous vehicles.
            \item \textit{Reinforcement:} Establish guidelines for accountability through legal frameworks.
        \end{itemize}

        \item \textbf{Impact on Employment}
        \begin{itemize}
            \item \textit{Definition:} Job displacement due to automation.
            \item \textit{Example:} Customer service bots reducing demand for human agents.
            \item \textit{Reinforcement:} Initiatives for reskilling and creating new job opportunities.
        \end{itemize}
    \end{enumerate}
\end{frame}

\begin{frame}[fragile]
    \frametitle{Ethical Considerations - Conclusion}
    Effective deployment of machine learning necessitates thorough consideration of ethical issues. 
    Understanding these concepts will inform technical practices and enhance societal trust in ML applications.

    \textbf{Key Points to Remember:}
    \begin{itemize}
        \item Evaluate data for bias and apply fairness techniques.
        \item Advocate for transparency in model decision-making.
        \item Respect user privacy, ensuring compliance with data protection regulations.
        \item Clarify accountability measures in AI systems for responsible development.
        \item Be aware of socio-economic impacts of automation through ML.
    \end{itemize}

    \textbf{Suggested Actions:}
    \begin{itemize}
        \item Engage in discussions about ethics in technology.
        \item Follow current events and literature regarding ML ethics.
        \item Consider ethical frameworks in ML system development.
    \end{itemize}
\end{frame}

\begin{frame}[fragile]
    \frametitle{Final Reminders - Part 1}
    \begin{block}{Preparing for the Midterm Exam}
        As we approach the midterm exam, please pay attention to the following final reminders to ensure you are well-prepared and equipped for success.
    \end{block}
\end{frame}

\begin{frame}[fragile]
    \frametitle{Final Reminders - Exam Details}
    \begin{enumerate}
        \item \textbf{Exam Time and Date}
            \begin{itemize}
                \item \textbf{When:} [Insert date]
                \item \textbf{Time:} [Insert time]
                \item \textbf{Duration:} 2 hours
                \item \textbf{Note:} Arrive 15 minutes early to settle in.
            \end{itemize}
        
        \item \textbf{Exam Location}
            \begin{itemize}
                \item \textbf{Where:} [Insert location, e.g., Room 101, Science Building]
                \item \textbf{Tip:} Familiarize yourself with the location beforehand.
            \end{itemize}
    \end{enumerate}
\end{frame}

\begin{frame}[fragile]
    \frametitle{Final Reminders - Exam Preparation}
    \begin{enumerate}
        \setcounter{enumi}{2}
        \item \textbf{Materials to Bring}
            \begin{itemize}
                \item \textbf{Essential Items:}
                    \begin{itemize}
                        \item \textbf{Identification:} Student ID or any form of ID.
                        \item \textbf{Writing Tools:} Pens, pencils, and erasers.
                        \item \textbf{Calculator:} Ensure compliance with syllabus.
                        \item \textbf{Notes:} Concise notes if allowed.
                    \end{itemize}
                \item \textbf{Prohibited Items:} 
                    \begin{itemize}
                        \item No electronic devices unless specified.
                        \item No bags or outside materials unless permitted.
                    \end{itemize}
            \end{itemize}

        \item \textbf{Exam Format}
            \begin{itemize}
                \item \textbf{Types of Questions:}
                    \begin{itemize}
                        \item Multiple Choice
                        \item Short Answer
                        \item Problem-Solving Questions
                    \end{itemize}
            \end{itemize}
    \end{enumerate}
\end{frame}

\begin{frame}[fragile]
    \frametitle{Final Reminders - Study Tips and Mindset}
    \begin{enumerate}
        \setcounter{enumi}{4}
        \item \textbf{Study Tips}
            \begin{itemize}
                \item Review course materials thoroughly.
                \item Focus on ethical considerations discussed in class.
                \item Practice sample questions or previous exams.
            \end{itemize}
        
        \item \textbf{Stay Calm and Focused}
            \begin{itemize}
                \item Approach the exam with a positive attitude.
                \item Take deep breaths if needed to stay focused.
            \end{itemize}
        
        \item \textbf{Key Takeaway}
            \begin{itemize}
                \item Be well-prepared with necessary materials and information.
                \item Arrive early and approach each section methodically.
            \end{itemize}
    \end{enumerate}
    \begin{block}{Remember}
        The midterm exam is an opportunity to demonstrate your understanding and progress in the course.
    \end{block}
\end{frame}


\end{document}