\documentclass[aspectratio=169]{beamer}

% Theme and Color Setup
\usetheme{Madrid}
\usecolortheme{whale}
\useinnertheme{rectangles}
\useoutertheme{miniframes}

% Additional Packages
\usepackage[utf8]{inputenc}
\usepackage[T1]{fontenc}
\usepackage{graphicx}
\usepackage{booktabs}
\usepackage{listings}
\usepackage{amsmath}
\usepackage{amssymb}
\usepackage{xcolor}
\usepackage{tikz}
\usepackage{pgfplots}
\pgfplotsset{compat=1.18}
\usetikzlibrary{positioning}
\usepackage{hyperref}

% Custom Colors
\definecolor{myblue}{RGB}{31, 73, 125}
\definecolor{mygray}{RGB}{100, 100, 100}
\definecolor{mygreen}{RGB}{0, 128, 0}
\definecolor{myorange}{RGB}{230, 126, 34}
\definecolor{mycodebackground}{RGB}{245, 245, 245}

% Set Theme Colors
\setbeamercolor{structure}{fg=myblue}
\setbeamercolor{frametitle}{fg=white, bg=myblue}
\setbeamercolor{title}{fg=myblue}
\setbeamercolor{section in toc}{fg=myblue}
\setbeamercolor{item projected}{fg=white, bg=myblue}
\setbeamercolor{block title}{bg=myblue!20, fg=myblue}
\setbeamercolor{block body}{bg=myblue!10}
\setbeamercolor{alerted text}{fg=myorange}

% Set Fonts
\setbeamerfont{title}{size=\Large, series=\bfseries}
\setbeamerfont{frametitle}{size=\large, series=\bfseries}
\setbeamerfont{caption}{size=\small}
\setbeamerfont{footnote}{size=\tiny}

% Document Start
\begin{document}

\frame{\titlepage}

\begin{frame}[fragile]
    \titlepage
\end{frame}

\begin{frame}[fragile]
    \frametitle{Overview of Model Evaluation Metrics}
    \begin{block}{Significance}
        Model evaluation metrics are essential tools in machine learning that help assess the performance of algorithms. These metrics provide insights into how well a model performs, allowing data scientists to fine-tune and improve their algorithms. Understanding evaluation metrics is crucial for developing robust and accurate predictive models.
    \end{block}
\end{frame}

\begin{frame}[fragile]
    \frametitle{Importance of Model Evaluation Metrics}
    \begin{itemize}
        \item \textbf{Performance Assessment:} Quantifies accuracy, reliability, and generalizability, indicating effectiveness on unseen data.
        \item \textbf{Guiding Model Improvement:} Highlights weaknesses, facilitating targeted enhancements of the learning model.
        \item \textbf{Comparison of Models:} Enables the comparison of different models, assisting in selecting the best approach for a problem.
    \end{itemize}
\end{frame}

\begin{frame}[fragile]
    \frametitle{Common Model Evaluation Metrics}
    \begin{itemize}
        \item \textbf{Accuracy:}
        \begin{equation}
        \text{Accuracy} = \frac{\text{True Positives} + \text{True Negatives}}{\text{Total Predictions}}
        \end{equation}

        \item \textbf{Precision:}
        \begin{equation}
        \text{Precision} = \frac{\text{True Positives}}{\text{True Positives} + \text{False Positives}}
        \end{equation}

        \item \textbf{Recall (Sensitivity):}
        \begin{equation}
        \text{Recall} = \frac{\text{True Positives}}{\text{True Positives} + \text{False Negatives}}
        \end{equation}

        \item \textbf{F1 Score:}
        \begin{equation}
        \text{F1 Score} = 2 \times \frac{\text{Precision} \times \text{Recall}}{\text{Precision} + \text{Recall}}
        \end{equation}

        \item \textbf{ROC Curve \& AUC:} Illustrates the trade-off between true positive rate and false positive rate at various thresholds.
    \end{itemize}
\end{frame}

\begin{frame}[fragile]
    \frametitle{Example Scenario: Spam Detection}
    \begin{itemize}
        \item Consider a binary classification model for predicting spam emails:
        \begin{itemize}
            \item Correctly predicts 70 spam emails and 30 non-spam emails, misclassifying 10 actual spam emails and marking 20 non-spam emails as spam.
        \end{itemize}

        \item \textbf{Calculated Metrics:}
        \begin{itemize}
            \item \textbf{Accuracy:} \( \frac{70 + 30}{100} = 1 \) or 100\% (ideal, but not sufficient)
            \item \textbf{Precision:} \( \frac{70}{70 + 20} = 0.77 \) or 77\%
            \item \textbf{Recall:} \( \frac{70}{70 + 10} = 0.88 \) or 88\%
        \end{itemize}

        \item High accuracy, but room for improvement in precision, especially considering the cost of false positives.
    \end{itemize}
\end{frame}

\begin{frame}[fragile]
    \frametitle{Key Points to Emphasize}
    \begin{itemize}
        \item Choosing the right metric depends on the specific problem domain and the relative costs of different types of errors (e.g., false positives vs. false negatives).
        \item No single metric provides a comprehensive evaluation of model performance; using a combination enhances insight.
    \end{itemize}
\end{frame}

\begin{frame}[fragile]{Importance of Model Evaluation - Introduction}
    \begin{block}{Overview}
        Model evaluation metrics are crucial in the data science lifecycle, enabling quantitative assessment of machine learning model performance. They guide model selection and improvement, which is essential for informed decision-making.
    \end{block}
\end{frame}

\begin{frame}[fragile]{Importance of Model Evaluation - Key Concepts}
    \begin{enumerate}
        \item \textbf{Definition of Model Evaluation Metrics:}
            \begin{itemize}
                \item Quantitative measures that provide insights into model performance.
                \item Facilitate comparison between different models and baseline performance.
            \end{itemize}
        
        \item \textbf{Importance of Evaluation:}
            \begin{itemize}
                \item \textbf{Guidance for Improvement:} Metrics reveal areas of underperformance.
                \item \textbf{Model Selection:} Assist in choosing the best model for deployment.
                \item \textbf{Feedback Loop:} Continuous performance monitoring allows for iterative improvements.
            \end{itemize}
    \end{enumerate}
\end{frame}

\begin{frame}[fragile]{Importance of Model Evaluation - Common Metrics}
    \begin{block}{Common Evaluation Metrics}
        \begin{itemize}
            \item \textbf{Accuracy:} Proportion of correctly predicted instances. Can be misleading in imbalanced datasets.
            \item \textbf{Precision, Recall, and F1-Score:}
                \begin{itemize}
                    \item \textbf{Precision:} $\frac{TP}{TP + FP}$
                    \item \textbf{Recall:} $\frac{TP}{TP + FN}$
                    \item \textbf{F1-Score:} Harmonic mean of precision and recall.
                \end{itemize}
            \item \textbf{AUC-ROC Curve:} Illustrates trade-off between true positive rate and false positive rate. A higher AUC indicates better performance.
        \end{itemize}
    \end{block}
\end{frame}

\begin{frame}[fragile]{Importance of Model Evaluation - Key Points and Conclusion}
    \begin{block}{Key Points to Emphasize}
        \begin{itemize}
            \item \textbf{Context Matters:} Select metrics based on problem goals. E.g. reduce false negatives in medical diagnosis.
            \item \textbf{Complete Evaluation:} Use multiple metrics for a comprehensive performance view.
            \item \textbf{Continuous Monitoring:} Regular evaluation post-deployment ensures models perform well with new data.
        \end{itemize}
    \end{block}
    
    \begin{block}{Conclusion}
        Evaluation metrics are integral to ensuring the robustness of machine learning applications, influencing effective decision-making and model reliability.
    \end{block}
\end{frame}

\begin{frame}[fragile]
    \frametitle{Accuracy - Definition}
    \begin{block}{Definition of Accuracy}
        \begin{itemize}
            \item **Accuracy** is defined as the ratio of correctly predicted instances to the total instances.
            \item The formula for accuracy is:
            \[
            \text{Accuracy} = \frac{\text{Number of Correct Predictions}}{\text{Total Number of Predictions}}
            \]
            \item In a two-class problem, it can be represented as:
            \[
            \text{Accuracy} = \frac{TP + TN}{TP + TN + FP + FN}
            \]
            \item Where:
            \begin{itemize}
                \item \( TP \) = True Positives
                \item \( TN \) = True Negatives
                \item \( FP \) = False Positives
                \item \( FN \) = False Negatives
            \end{itemize}
        \end{itemize}
    \end{block}
\end{frame}

\begin{frame}[fragile]
    \frametitle{Accuracy - Limitations}
    \begin{block}{Limitations of Accuracy}
        \begin{itemize}
            \item **Imbalanced Classes**: High accuracy may occur in imbalanced datasets without true predictive power.
            \begin{itemize}
                \item \textbf{Example:} In a disease screening where 95 out of 100 patients are healthy, predicting all as healthy gives 95\% accuracy but misses the diseased patients.
            \end{itemize}
            \item **Performance Reflection**: Does not indicate the quantity or types of errors (e.g., false positives vs. false negatives).
            \item **Multi-class Issues**: High accuracy might be misleading when models perform poorly on minority classes.
        \end{itemize}
    \end{block}
\end{frame}

\begin{frame}[fragile]
    \frametitle{Accuracy - Key Points}
    \begin{block}{Key Points to Emphasize}
        \begin{itemize}
            \item **Use with Caution**: Apply accuracy when class distributions are balanced.
            \item **Complementary Metrics**: Use additional metrics like precision, recall, F1 score, and AUC-ROC for thorough evaluation.
            \item **Visual Representation**: Confusion matrix can clarify accuracy metrics alongside precision and recall for performance context.
        \end{itemize}
    \end{block}
\end{frame}

\begin{frame}[fragile]
    \frametitle{Precision - Definition}
    \begin{block}{Definition of Precision}
        Precision is a performance metric used to evaluate the accuracy of positive predictions in classification tasks. Mathematically, it is defined as:
        \begin{equation}
            \text{Precision} = \frac{\text{True Positives (TP)}}{\text{True Positives (TP)} + \text{False Positives (FP)}}
        \end{equation}
    \end{block}
    \begin{itemize}
        \item \textbf{True Positives (TP)}: Instances correctly predicted as positive.
        \item \textbf{False Positives (FP)}: Instances incorrectly predicted as positive.
    \end{itemize}
\end{frame}

\begin{frame}[fragile]
    \frametitle{Precision - Interpretation and Applications}
    \begin{block}{Interpretation}
        Precision provides insight into the quality of positive predictions. A high precision value indicates a high proportion of correct positive predictions.
    \end{block}
    
    \begin{block}{Applications in Classification Tasks}
        \begin{enumerate}
            \item \textbf{Binary Classification:} E.g., spam emails detection; measures trust in spam notifications.
            \item \textbf{Medical Diagnosis:} E.g., cancer detection; ensures high likelihood of true diagnoses.
            \item \textbf{E-commerce Recommendations:} Evaluates how well recommended products fit user preferences.
        \end{enumerate}
    \end{block}
\end{frame}

\begin{frame}[fragile]
    \frametitle{Precision - Key Points and Conclusion}
    \begin{itemize}
        \item Precision is critical when the cost of false positives is high.
        \item It is a core metric in fields like finance, healthcare, and security.
        \item Should be used with other metrics, like recall, for a holistic view of model performance.
    \end{itemize}
    
    \begin{block}{Example Calculation}
        Given:
        \begin{itemize}
            \item True Positives (TP) = 70
            \item False Positives (FP) = 30
        \end{itemize}
        Then, Precision is calculated as:
        \begin{equation}
            \text{Precision} = \frac{TP}{TP + FP} = \frac{70}{70 + 30} = 0.7 \text{ or } 70\%
        \end{equation}
    \end{block}

    \begin{block}{Conclusion}
        Understanding precision is vital for interpreting classification models, especially in high-reliability scenarios.
    \end{block}
\end{frame}

\begin{frame}[fragile]
    \frametitle{Recall - Definition}
    
    \begin{block}{Definition}
        Recall, also known as sensitivity or true positive rate, is a metric that assesses the ability of a classification model to identify positive instances. 
    \end{block}
    
    \begin{equation}
    \text{Recall} = \frac{\text{True Positives (TP)}}{\text{True Positives (TP)} + \text{False Negatives (FN)}}
    \end{equation}
    
    Where:
    \begin{itemize}
        \item **True Positives (TP)**: Correctly predicted positive instances.
        \item **False Negatives (FN)**: Actual positive instances that were incorrectly predicted as negative.
    \end{itemize}
\end{frame}

\begin{frame}[fragile]
    \frametitle{Recall - Significance}
    
    \begin{block}{Significance of Recall}
        Recall is particularly important in scenarios where missing a positive instance (false negative) can have severe consequences. For example:
    \end{block}
    
    \begin{itemize}
        \item \textbf{Medical Diagnosis}: In cancer screening, missing a diagnosis can be life-threatening.
        \item \textbf{Fraud Detection}: Missing fraudulent activities can result in significant financial losses.
    \end{itemize}
    
    \begin{block}{Key Points}
        \begin{itemize}
            \item High recall indicates most positive instances are correctly identified, but may sacrifice precision.
            \item It is crucial to balance recall with other metrics, especially in sensitive applications.
        \end{itemize}
    \end{block}
\end{frame}

\begin{frame}[fragile]
    \frametitle{Recall - Example Calculation}
    
    \begin{block}{Example Calculation}
        Consider a medical test with:
        \begin{itemize}
            \item True Positives (TP) = 80
            \item False Negatives (FN) = 20
        \end{itemize}
        
        Using the recall formula:
        \begin{equation}
        \text{Recall} = \frac{TP}{TP + FN} = \frac{80}{80 + 20} = \frac{80}{100} = 0.8
        \end{equation}
    \end{block}
    
    \begin{block}{Conclusion}
        A recall of 0.8 indicates that the test accurately identified 80\% of actual positive cases, vital for evaluating models when false negatives must be minimized.
    \end{block}
    
    \begin{block}{Transition}
        Next, we will explore the F1-Score which combines precision and recall for a comprehensive model evaluation.
    \end{block}
\end{frame}

\begin{frame}[fragile]
    \frametitle{F1-Score}
    \begin{block}{Understanding the F1-Score}
        The F1-score is a metric used to assess the performance of a binary classification model. It is calculated as the harmonic mean of two key metrics: \textbf{Precision} and \textbf{Recall}.
    \end{block}
\end{frame}

\begin{frame}[fragile]
    \frametitle{F1-Score - Formula}
    The F1-score is given by the formula:
    \begin{equation}
        F1 = 2 \times \left( \frac{\text{Precision} \times \text{Recall}}{\text{Precision} + \text{Recall}} \right)
    \end{equation}
    
    Where:
    \begin{itemize}
        \item \textbf{Precision:} 
        \begin{equation}
            \text{Precision} = \frac{\text{True Positives (TP)}}{\text{True Positives (TP)} + \text{False Positives (FP)}}
        \end{equation}
        \item \textbf{Recall (Sensitivity):} 
        \begin{equation}
            \text{Recall} = \frac{\text{True Positives (TP)}}{\text{True Positives (TP)} + \text{False Negatives (FN)}}
        \end{equation}
    \end{itemize}
\end{frame}

\begin{frame}[fragile]
    \frametitle{F1-Score - Importance and Interpretation}
    \begin{itemize}
        \item \textbf{Balancing Act:} The F1-score provides a balance between precision and recall, crucial for scenarios where one is prioritized over the other.
        \item \textbf{Use Cases:} 
        \begin{enumerate}
            \item Medical diagnosis (emphasizing recall to avoid missing positive cases).
            \item Spam detection (prioritizing precision to minimize false positives).
        \end{enumerate}
        \item \textbf{Interpretation:} The F1-score ranges from 0 to 1. 
        \begin{itemize}
            \item 1 indicates perfect precision and recall.
            \item 0 indicates no precision or recall.
        \end{itemize}
    \end{itemize}
\end{frame}

\begin{frame}[fragile]
    \frametitle{F1-Score - Example Calculation}
    Given:
    \begin{itemize}
        \item True Positives (TP) = 40  
        \item False Positives (FP) = 10  
        \item False Negatives (FN) = 5  
    \end{itemize}
    
    \textbf{Calculating Precision:}
    \begin{equation}
        \text{Precision} = \frac{40}{40 + 10} = 0.8
    \end{equation}

    \textbf{Calculating Recall:}
    \begin{equation}
        \text{Recall} = \frac{40}{40 + 5} \approx 0.888
    \end{equation}

    \textbf{Calculating F1-Score:}
    \begin{equation}
        F1 \approx 2 \times \left( \frac{0.8 \times 0.888}{0.8 + 0.888} \right) \approx 0.837
    \end{equation}
\end{frame}

\begin{frame}[fragile]
    \frametitle{F1-Score - Key Points}
    \begin{itemize}
        \item \textbf{Use of F1-score:} Particularly useful in imbalanced class situations (e.g., fraud detection).
        \item \textbf{Interpretation Limitations:} It is essential to analyze precision and recall individually for a comprehensive performance evaluation.
    \end{itemize}
\end{frame}

\begin{frame}[fragile]
    \frametitle{ROC Curve - Introduction}
    \begin{block}{Receiver Operating Characteristic (ROC) Curve}
        The ROC curve is a graphical representation to evaluate the performance of binary classification models.
        It illustrates the relationship between:
        \begin{itemize}
            \item True Positive Rate (TPR)
            \item False Positive Rate (FPR)
        \end{itemize}
    \end{block}
    
    \begin{block}{Key Definitions}
        \begin{itemize}
            \item \textbf{True Positive Rate (TPR)}:
            \[
            \text{TPR} = \frac{\text{True Positives}}{\text{True Positives} + \text{False Negatives}}
            \]
            \item \textbf{False Positive Rate (FPR)}:
            \[
            \text{FPR} = \frac{\text{False Positives}}{\text{False Positives} + \text{True Negatives}}
            \]
        \end{itemize}
    \end{block}
\end{frame}

\begin{frame}[fragile]
    \frametitle{ROC Curve - Trade-off and AUC}
    \begin{block}{Understanding the Trade-off}
        The ROC curve plots TPR against FPR for various threshold values:
        \begin{itemize}
            \item Lower threshold → higher TPR, higher FPR
            \item Higher threshold → lower TPR, lower FPR
        \end{itemize}
        This trade-off is crucial in scenarios with differing costs for false positives and false negatives (e.g., in medical diagnosis).
    \end{block}
    
    \begin{block}{Area Under the Curve (AUC)}
        The AUC quantifies the model’s ability to discriminate between classes:
        \begin{itemize}
            \item \textbf{AUC = 1}: Perfect model
            \item \textbf{AUC = 0.5}: Model with no discrimination ability
            \item \textbf{AUC < 0.5}: Model worse than random guessing
        \end{itemize}
    \end{block}
\end{frame}

\begin{frame}[fragile]
    \frametitle{ROC Curve - Example and Key Points}
    \begin{block}{Example Illustration}
        Consider an email classification model for spam detection:
        \begin{itemize}
            \item \textbf{Threshold = 0.1}: High TPR, high FPR (many emails classified as spam)
            \item \textbf{Threshold = 0.9}: Low TPR, low FPR (fewer emails classified as spam)
        \end{itemize}
        As the threshold changes, the ROC curve forms an S-shape.
    \end{block}
    
    \begin{block}{Key Points to Emphasize}
        \begin{itemize}
            \item ROC curves visualize model performance and allow comparisons.
            \item Understanding the trade-off between TPR and FPR aids in threshold selection.
            \item AUC summarizes model effectiveness with a single scalar value.
        \end{itemize}
    \end{block}
    
    \begin{block}{Practical Implications}
        When selecting a binary classification model, consider ROC curves and AUC alongside other metrics like F1-score for balanced evaluation.
    \end{block}
\end{frame}

\begin{frame}[fragile]
    \frametitle{Practical Examples of Model Evaluation Metrics}
    Model evaluation metrics are essential tools for assessing the performance of machine learning models. These metrics help determine how well a model predicts outcomes, allowing for informed improvements and adjustments. 
\end{frame}

\begin{frame}[fragile]
    \frametitle{Accuracy}
    \begin{itemize}
        \item \textbf{Definition}: Ratio of correctly predicted instances to total instances.
        \item \textbf{Formula}:
        \begin{equation}
            \text{Accuracy} = \frac{\text{TP} + \text{TN}}{\text{TP} + \text{TN} + \text{FP} + \text{FN}}
        \end{equation}
        \item Where:
        \begin{itemize}
            \item TP = True Positives
            \item TN = True Negatives
            \item FP = False Positives
            \item FN = False Negatives
        \end{itemize}
    \end{itemize}
\end{frame}

\begin{frame}[fragile]
    \frametitle{Accuracy Example}
    \begin{itemize}
        \item Example: Medical diagnosis dataset
        \begin{itemize}
            \item \textbf{Total Patients}: 100
            \item \textbf{Correctly Diagnosed Cases}: 90
            \item \textbf{Incorrect Cases}: 10
        \end{itemize}
        \item \textbf{Calculation}:
        \begin{equation}
            \text{Accuracy} = \frac{90}{100} = 0.90 = 90\%
        \end{equation}
    \end{itemize}
\end{frame}

\begin{frame}[fragile]
    \frametitle{Precision and Recall}
    \begin{itemize}
        \item \textbf{Precision}: Ratio of correctly predicted positive observations to total predicted positives.
        \begin{equation}
            \text{Precision} = \frac{\text{TP}}{\text{TP} + \text{FP}}
        \end{equation}
        \item \textbf{Recall}: Ratio of correctly predicted positive observations to all actual positives.
        \begin{equation}
            \text{Recall} = \frac{\text{TP}}{\text{TP} + \text{FN}}
        \end{equation}
    \end{itemize}
\end{frame}

\begin{frame}[fragile]
    \frametitle{Precision and Recall Example}
    \begin{itemize}
        \item Example: Spam email classification
        \begin{itemize}
            \item \textbf{TP}: 30 (correctly identified spam)
            \item \textbf{FP}: 10 (legitimate emails identified as spam)
            \item \textbf{FN}: 5 (spam emails not identified)
        \end{itemize}
        \item \textbf{Calculations}:
        \begin{align*}
            \text{Precision} & = \frac{30}{30 + 10} = 0.75 = 75\% \\
            \text{Recall} & = \frac{30}{30 + 5} = 0.86 = 86\%
        \end{align*}
    \end{itemize}
\end{frame}

\begin{frame}[fragile]
    \frametitle{F1 Score}
    \begin{itemize}
        \item \textbf{Definition}: Harmonic mean of precision and recall.
        \begin{equation}
            \text{F1 Score} = 2 \times \frac{\text{Precision} \times \text{Recall}}{\text{Precision} + \text{Recall}}
        \end{equation}
        \item \textbf{Example Calculation}:
        \begin{equation}
            \text{F1 Score} = 2 \times \frac{0.75 \times 0.86}{0.75 + 0.86} \approx 0.80
        \end{equation}
    \end{itemize}
\end{frame}

\begin{frame}[fragile]
    \frametitle{Area Under the ROC Curve (AUC-ROC)}
    \begin{itemize}
        \item \textbf{Definition}: Measures the model's ability to distinguish between classes.
        \item \textbf{Key Points}:
        \begin{itemize}
            \item AUC ranges from 0 to 1; 1 indicates perfect classification.
            \item Provides insight into the model's performance across all thresholds.
        \end{itemize}
        \item \textbf{Example}: A credit risk assessment model with AUC = 0.85 indicates strong predictive capability.
    \end{itemize}
\end{frame}

\begin{frame}[fragile]
    \frametitle{Key Takeaways}
    \begin{itemize}
        \item \textbf{Understanding the right metric}: Different situations necessitate different evaluation metrics.
        \item \textbf{Real-world applicability}: Using actual datasets solidifies concepts.
        \item \textbf{Balance precision and recall}: Depending on application, prioritization may differ.
    \end{itemize}
    \begin{block}{Next Steps}
        In the next section, we will explore a comparative analysis of these metrics and discuss their application based on specific model requirements and data characteristics.
    \end{block}
\end{frame}

\begin{frame}[fragile]
    \frametitle{Comparative Analysis - Introduction}
    \begin{block}{Introduction to Model Evaluation Metrics}
        Evaluating the performance of machine learning models is crucial for understanding their effectiveness and reliability. Different metrics provide insights based on the characteristics of the task and the data. 
        This slide presents a comparative analysis of various evaluation metrics commonly used in model assessment.
    \end{block}
\end{frame}

\begin{frame}[fragile]
    \frametitle{Comparative Analysis - Key Metrics Overview}
    \begin{enumerate}
        \item \textbf{Accuracy}
            \begin{itemize}
                \item \textbf{Definition}: The ratio of correctly predicted observations to the total observations.
                \item \textbf{Formula}:
                \[
                \text{Accuracy} = \frac{TP + TN}{TP + TN + FP + FN}
                \]
                \item \textbf{When to Use}: Best for balanced datasets where classes are equally represented.
            \end{itemize}
        
        \item \textbf{Precision}
            \begin{itemize}
                \item \textbf{Definition}: The ratio of true positive predictions to the total predicted positives.
                \item \textbf{Formula}:
                \[
                \text{Precision} = \frac{TP}{TP + FP}
                \]
                \item \textbf{When to Use}: Important when the cost of false positives is high (e.g., spam detection).
            \end{itemize}

        \item \textbf{Recall (Sensitivity)}
            \begin{itemize}
                \item \textbf{Definition}: The ratio of true positive predictions to the actual positives.
                \item \textbf{Formula}:
                \[
                \text{Recall} = \frac{TP}{TP + FN}
                \]
                \item \textbf{When to Use}: Crucial in scenarios where missing positive cases is costly (e.g., disease detection).
            \end{itemize}
    \end{enumerate}
\end{frame}

\begin{frame}[fragile]
    \frametitle{Comparative Analysis - Remaining Metrics}
    \begin{enumerate}[start=4]
        \item \textbf{F1-Score}
            \begin{itemize}
                \item \textbf{Definition}: The harmonic mean of precision and recall.
                \item \textbf{Formula}:
                \[
                F1 = 2 \times \frac{\text{Precision} \times \text{Recall}}{\text{Precision} + \text{Recall}}
                \]
                \item \textbf{When to Use}: Suitable for imbalanced datasets where both false positives and false negatives are of concern.
            \end{itemize}

        \item \textbf{ROC-AUC}
            \begin{itemize}
                \item \textbf{Definition}: Measures the model's ability to distinguish between classes, providing an aggregate performance metric across all classification thresholds.
                \item \textbf{When to Use}: Good for evaluating models at various thresholds, particularly when dealing with class imbalance.
            \end{itemize}
    \end{enumerate}
\end{frame}

\begin{frame}[fragile]
    \frametitle{Comparative Analysis - Insights and Conclusion}
    \begin{block}{Comparative Insights}
        \begin{tabular}{|c|c|c|}
            \hline
            Metric & Best Use Case & Limitation \\
            \hline
            Accuracy & Balanced classes & Misleading with imbalanced classes \\
            \hline
            Precision & High false positive cost & Ignores false negatives \\
            \hline
            Recall & High false negative cost & Ignores false positives \\
            \hline
            F1-Score & Imbalanced datasets & Complexity in interpreting the balance \\
            \hline
            ROC-AUC & Varying class thresholds & Does not provide specific class performance \\
            \hline
        \end{tabular}
    \end{block}

    \begin{block}{Key Points to Emphasize}
        \begin{itemize}
            \item Understanding context is crucial in choosing an evaluation metric.
            \item Handle data imbalance carefully; F1-Score and AUC are better metrics than accuracy.
            \item Consider trade-offs between metrics, especially false positives vs. false negatives.
        \end{itemize}
    \end{block}
    
    \begin{block}{Conclusion}
        Selecting the appropriate evaluation metric is vital for accurately assessing model performance. Understanding each metric's advantages and limitations guides better model selection and improvements.
    \end{block}
\end{frame}

\begin{frame}[fragile]
    \frametitle{Conclusion and Future Directions - Key Takeaways}
    \begin{enumerate}
        \item \textbf{Importance of Evaluation Metrics:}
        \begin{itemize}
            \item Essential for assessing machine learning model performance.
            \item Helps in understanding aspects like accuracy, precision, recall, and F1 score.
        \end{itemize}
        
        \item \textbf{Choosing the Right Metric:}
        \begin{itemize}
            \item Depends on the use case; examples include:
            \begin{itemize}
                \item Accuracy for balanced classes.
                \item Precision and Recall for imbalanced datasets (e.g., fraud detection).
                \item F1 Score for a balance between precision and recall.
            \end{itemize}
        \end{itemize}

        \item \textbf{Comparative Analysis of Metrics:}
        \begin{itemize}
            \item Analyzing multiple metrics unveils hidden insights.
            \item High accuracy does not always imply good performance, e.g., precision may suffer.
        \end{itemize}
    \end{enumerate}
\end{frame}

\begin{frame}[fragile]
    \frametitle{Conclusion and Future Directions - Emerging Trends}
    \begin{enumerate}
        \item \textbf{Fairness and Bias Detection:}
        \begin{itemize}
            \item Focus on ensuring models are unbiased.
            \item Fairness metrics evaluate performance across demographic groups.
        \end{itemize}

        \item \textbf{Automated and Continuous Evaluation:}
        \begin{itemize}
            \item Importance of continuous evaluation in dynamic environments.
            \item Adapts to real-time data to maintain model accuracy.
        \end{itemize}

        \item \textbf{Use of Ensemble Metrics:}
        \begin{itemize}
            \item Strategies like voting or stacking require evaluation of multiple models.
        \end{itemize}

        \item \textbf{Interpretable Metrics:}
        \begin{itemize}
            \item Demand for metrics that provide insights into model performance.
            \item Enhances trust and transparency in AI systems.
        \end{itemize}
    \end{enumerate}
\end{frame}

\begin{frame}[fragile]
    \frametitle{Key Formulas and Final Thoughts}
    \begin{block}{Key Formulas}
        \textbf{F1 Score:} Balances precision and recall
        \begin{equation}
            F1 = 2 \times \frac{(\text{Precision} \times \text{Recall})}{(\text{Precision} + \text{Recall})}
        \end{equation}

        \textbf{Precision and Recall Definitions:}
        \begin{itemize}
            \item Precision: \( \frac{TP}{TP + FP} \)
            \item Recall: \( \frac{TP}{TP + FN} \)
        \end{itemize}
        Where:
        \begin{itemize}
            \item \( TP \) = True Positives
            \item \( FP \) = False Positives
            \item \( FN \) = False Negatives
        \end{itemize}
    \end{block}

    \textbf{Final Thoughts:}
    - Evaluate models continuously in response to new data and ethical considerations.
    - Leverage emerging trends and metrics for effective model assessment.
\end{frame}


\end{document}