\documentclass[aspectratio=169]{beamer}

% Theme and Color Setup
\usetheme{Madrid}
\usecolortheme{whale}
\useinnertheme{rectangles}
\useoutertheme{miniframes}

% Additional Packages
\usepackage[utf8]{inputenc}
\usepackage[T1]{fontenc}
\usepackage{graphicx}
\usepackage{booktabs}
\usepackage{listings}
\usepackage{amsmath}
\usepackage{amssymb}
\usepackage{xcolor}
\usepackage{tikz}
\usepackage{pgfplots}
\pgfplotsset{compat=1.18}
\usetikzlibrary{positioning}
\usepackage{hyperref}

% Custom Colors
\definecolor{myblue}{RGB}{31, 73, 125}
\definecolor{mygray}{RGB}{100, 100, 100}
\definecolor{mygreen}{RGB}{0, 128, 0}
\definecolor{myorange}{RGB}{230, 126, 34}
\definecolor{mycodebackground}{RGB}{245, 245, 245}

% Set Theme Colors
\setbeamercolor{structure}{fg=myblue}
\setbeamercolor{frametitle}{fg=white, bg=myblue}
\setbeamercolor{title}{fg=myblue}
\setbeamercolor{section in toc}{fg=myblue}
\setbeamercolor{item projected}{fg=white, bg=myblue}
\setbeamercolor{block title}{bg=myblue!20, fg=myblue}
\setbeamercolor{block body}{bg=myblue!10}
\setbeamercolor{alerted text}{fg=myorange}

% Set Fonts
\setbeamerfont{title}{size=\Large, series=\bfseries}
\setbeamerfont{frametitle}{size=\large, series=\bfseries}
\setbeamerfont{caption}{size=\small}
\setbeamerfont{footnote}{size=\tiny}

% Code Listing Style
\lstdefinestyle{customcode}{
  backgroundcolor=\color{mycodebackground},
  basicstyle=\footnotesize\ttfamily,
  breakatwhitespace=false,
  breaklines=true,
  commentstyle=\color{mygreen}\itshape,
  keywordstyle=\color{blue}\bfseries,
  stringstyle=\color{myorange},
  numbers=left,
  numbersep=8pt,
  numberstyle=\tiny\color{mygray},
  frame=single,
  framesep=5pt,
  rulecolor=\color{mygray},
  showspaces=false,
  showstringspaces=false,
  showtabs=false,
  tabsize=2,
  captionpos=b
}
\lstset{style=customcode}

% Custom Commands
\newcommand{\hilight}[1]{\colorbox{myorange!30}{#1}}
\newcommand{\source}[1]{\vspace{0.2cm}\hfill{\tiny\textcolor{mygray}{Source: #1}}}
\newcommand{\concept}[1]{\textcolor{myblue}{\textbf{#1}}}
\newcommand{\separator}{\begin{center}\rule{0.5\linewidth}{0.5pt}\end{center}}

% Footer and Navigation Setup
\setbeamertemplate{footline}{
  \leavevmode%
  \hbox{%
  \begin{beamercolorbox}[wd=.3\paperwidth,ht=2.25ex,dp=1ex,center]{author in head/foot}%
    \usebeamerfont{author in head/foot}\insertshortauthor
  \end{beamercolorbox}%
  \begin{beamercolorbox}[wd=.5\paperwidth,ht=2.25ex,dp=1ex,center]{title in head/foot}%
    \usebeamerfont{title in head/foot}\insertshorttitle
  \end{beamercolorbox}%
  \begin{beamercolorbox}[wd=.2\paperwidth,ht=2.25ex,dp=1ex,center]{date in head/foot}%
    \usebeamerfont{date in head/foot}
    \insertframenumber{} / \inserttotalframenumber
  \end{beamercolorbox}}%
  \vskip0pt%
}

% Turn off navigation symbols
\setbeamertemplate{navigation symbols}{}

% Title Page Information
\title[Course Introduction]{Week 1: Course Introduction}
\author[J. Smith]{John Smith, Ph.D.}
\institute[University Name]{
  Department of Computer Science\\
  University Name\\
  \vspace{0.3cm}
  Email: email@university.edu\\
  Website: www.university.edu
}
\date{\today}

% Document Start
\begin{document}

\frame{\titlepage}

\begin{frame}[fragile]
    \frametitle{Course Introduction - Part 1}
    \begin{block}{Overview of the Course Structure and Objectives}
        \begin{itemize}
            \item \textbf{Welcome to the Course}
                \begin{itemize}
                    \item Objective: Provide foundational knowledge and skills in Data Mining.
                    \item Importance: Extracting meaningful patterns from large datasets has broad applications in business, healthcare, and social sciences.
                \end{itemize}
        \end{itemize}
    \end{block}
\end{frame}

\begin{frame}[fragile]
    \frametitle{Course Introduction - Part 2}
    \begin{block}{Course Structure}
        \begin{itemize}
            \item \textbf{Modules:}
                \begin{enumerate}
                    \item Week 1: Course Introduction - Overview and expectations
                    \item Week 2: What is Data Mining? - Significance of data mining
                    \item Subsequent Weeks: Advanced techniques such as Classification, Clustering, Regression Analysis, and Association Rule Learning.
                \end{enumerate}
            \item \textbf{Assignments:} Weekly tasks to reinforce concepts.
            \item \textbf{Capstone Project:} Apply techniques to a real-world dataset.
        \end{itemize}
    \end{block}
\end{frame}

\begin{frame}[fragile]
    \frametitle{Course Introduction - Part 3}
    \begin{block}{Learning Objectives and Motivation}
        \begin{itemize}
            \item Understand basic data mining concepts and their applications in real-world scenarios.
            \item Explore recent advancements in AI and how data mining enhances learning algorithms (e.g., ChatGPT).
            \item \textbf{Motivation for Data Mining:}
                \begin{itemize}
                    \item Real-world relevance in decision-making and data analysis.
                    \item Achievements through data mining in business and healthcare.
                \end{itemize}
        \end{itemize}
    \end{block}

    \begin{block}{Next Steps}
        \begin{itemize}
            \item Prepare for the upcoming module on "What is Data Mining?" and think about its applications in your field.
            \item Review assigned readings for Week 1.
        \end{itemize}
    \end{block}
\end{frame}

\begin{frame}[fragile]{What is Data Mining? - Introduction}
    \begin{block}{Introduction to Data Mining}
        Data mining is the process of discovering patterns, correlations, and trends by analyzing large sets of data. Through this process, significant insights are unveiled that can support decision-making and predictive analysis in various domains.
    \end{block}

    \begin{itemize}
        \item Understand the role of data mining in extracting value from large datasets.
        \item Recognize its significance in various industries and applications.
    \end{itemize}
\end{frame}

\begin{frame}[fragile]{What is Data Mining? - Importance}
    \begin{block}{Why is Data Mining Important?}
        \begin{enumerate}
            \item \textbf{Exponential Growth of Data}: Vast amounts of data are generated daily. Efficient methods are needed to extract valuable information from this immense data.
            
            \item \textbf{Competitive Advantage}: Organizations leverage data mining for insights into customer behavior, which helps in creating targeted marketing strategies and improving operational efficiencies.
            
            \item \textbf{Applications Across Industries}:
            \begin{itemize}
                \item \textbf{Healthcare}: Predict disease outbreaks and improve patient care.
                \item \textbf{Finance}: Detect fraud through pattern analysis.
                \item \textbf{Retail}: Optimize inventory based on customer purchase habits.
            \end{itemize}
        \end{enumerate}
    \end{block}
\end{frame}

\begin{frame}[fragile]{What is Data Mining? - Recent Developments}
    \begin{block}{Recent Developments}
        Data mining is closely linked to artificial intelligence (AI). For instance, AI applications like ChatGPT use data mining techniques to enhance language models by analyzing vast textual datasets. This enhances understanding of context, sentiment, and user intent.
    \end{block}

    \begin{block}{Key Takeaways}
        \begin{itemize}
            \item Data mining is crucial for deriving actionable insights from large datasets.
            \item It is essential for competitive business strategies and informed decisions.
            \item Techniques include classification, clustering, and regression.
            \item Impactful in modern applications, especially in AI, driving innovations.
        \end{itemize}
    \end{block}

    \begin{block}{Concluding Note}
        As we continue this course, we will explore techniques, tools, methodologies of data mining, and its ethical considerations.
    \end{block}
\end{frame}

\begin{frame}[fragile]
  \frametitle{Motivations for Data Mining - Introduction}
  Data mining is the process of discovering patterns and extracting valuable insights from large datasets. It plays a crucial role in transforming raw data into actionable knowledge, enabling organizations to make informed decisions.
\end{frame}

\begin{frame}[fragile]
  \frametitle{Motivations for Data Mining - Why Do We Need Data Mining?}
  \begin{enumerate}
    \item \textbf{Volume of Data}:
      \begin{itemize}
        \item The world generates approximately \textbf{2.5 quintillion bytes of data} every day.
        \item This massive volume necessitates powerful tools to analyze and extract meaningful information.
      \end{itemize}
      
    \item \textbf{Complexity of Data}:
      \begin{itemize}
        \item Data comes in various forms: structured, unstructured, and semi-structured.
        \item Data mining helps handle this complexity by utilizing algorithmic techniques to discern patterns and relationships.
      \end{itemize}
  \end{enumerate}
\end{frame}

\begin{frame}[fragile]
  \frametitle{Motivations for Data Mining - Real-World Examples}
  \begin{enumerate}[resume]
    \item \textbf{Trends in Big Data}:
      \begin{itemize}
        \item Companies like Amazon and Netflix utilize data mining to analyze user behavior.
        \item Amazon recommends products based on past purchases, while Netflix suggests shows based on viewing history.
      \end{itemize}
    
    \item \textbf{Competitive Advantage}:
      \begin{itemize}
        \item Businesses leverage data mining to enhance customer service and optimize operations.
        \item For instance, Walmart uses data mining techniques to predict inventory needs, reducing losses and improving sales.
      \end{itemize}
      
    \item \textbf{Applications in AI}:
      \begin{itemize}
        \item Modern AI applications, like \textbf{ChatGPT}, thrive on insights garnered from data mining, analyzing massive amounts of text data.
      \end{itemize}
  \end{enumerate}
\end{frame}

\begin{frame}[fragile]
  \frametitle{Motivations for Data Mining - Key Points and Conclusion}
  \begin{block}{Key Points}
    \begin{itemize}
      \item Data mining is essential for making sense of the vast amounts of data we produce.
      \item It allows businesses to uncover hidden patterns that can inform strategy.
      \item Organizations can enhance decision-making and operational efficiency through effective data mining.
    \end{itemize}
  \end{block}
  
  In essence, data mining is crucial for organizations to remain competitive in leveraging the vast datasets generated in our digital world.
\end{frame}

\begin{frame}[fragile]
    \frametitle{Overview of Data Mining Techniques}
    \begin{block}{Introduction to Data Mining}
        Data mining is a powerful technology that helps organizations extract valuable insights from vast amounts of data.
        Understanding the motivations behind data mining is key to appreciating its role in today’s data-driven world.
    \end{block}
\end{frame}

\begin{frame}[fragile]
    \frametitle{Key Data Mining Techniques - Part 1}
    \begin{enumerate}
        \item \textbf{Classification}
        \begin{itemize}
            \item \textbf{Definition}: Assigning items in a dataset to target categories or classes.
            \item \textbf{Purpose}: Predicting categorical labels based on input features.
            \item \textbf{Example}: Email filtering (spam vs. not spam).
            \item \textbf{Common Algorithms}: Logistic Regression, Decision Trees, Random Forest, Neural Networks.
        \end{itemize}
        
        \item \textbf{Clustering}
        \begin{itemize}
            \item \textbf{Definition}: Groups similar data points based on characteristics without prior labels.
            \item \textbf{Purpose}: Identifying inherent groupings within data.
            \item \textbf{Example}: Customer segmentation based on purchasing behavior.
            \item \textbf{Common Algorithms}: K-means, Hierarchical Clustering, DBSCAN.
        \end{itemize}
    \end{enumerate}
\end{frame}

\begin{frame}[fragile]
    \frametitle{Key Data Mining Techniques - Part 2}
    \begin{enumerate}
        \setcounter{enumi}{2}
        \item \textbf{Association Rule Learning}
        \begin{itemize}
            \item \textbf{Definition}: Identifies interesting relationships between variables in large datasets.
            \item \textbf{Purpose}: Discovering patterns and rules that describe large portions of the data.
            \item \textbf{Example}: Market Basket Analysis (items commonly bought together).
            \item \textbf{Common Methods}: Apriori algorithm, FP-Growth.
        \end{itemize}

        \item \textbf{Regression}
        \begin{itemize}
            \item \textbf{Definition}: Predicts a continuous outcome variable based on one or more predictors.
            \item \textbf{Purpose}: Quantifying the relationship between dependent and independent variables.
            \item \textbf{Example}: Predicting real estate prices.
            \item \textbf{Common Techniques}: Linear Regression, Polynomial Regression.
        \end{itemize}

        \item \textbf{Anomaly Detection}
        \begin{itemize}
            \item \textbf{Definition}: Identifies rare items or observations that differ significantly from the majority of the data.
            \item \textbf{Purpose}: Useful in fraud detection and monitoring system health.
            \item \textbf{Example}: Fraudulent transaction detection.
            \item \textbf{Common Methods}: Statistical tests, Isolation Forest, Autoencoders.
        \end{itemize}
    \end{enumerate}
\end{frame}

\begin{frame}[fragile]
    \frametitle{Conclusion and Key Points}
    \begin{block}{Why Do We Need Data Mining?}
        Data mining techniques support decision-making processes across various industries, including finance, healthcare, and retail.
        Applications of AI, such as ChatGPT, rely on data mining for analyzing interactions, learning user preferences, and improving response accuracy.
    \end{block}

    \begin{itemize}
        \item Data mining transforms raw data into actionable insights.
        \item Each technique serves unique purposes and specific use-cases.
        \item Understanding the right technique is critical for effective analysis and application.
    \end{itemize}

    \begin{block}{Summary}
        Familiarity with core data mining techniques equips you to leverage data effectively in future projects.
    \end{block}
\end{frame}

\begin{frame}[fragile]
    \frametitle{Classification Techniques - Introduction}
    \begin{block}{What is Classification?}
        Classification is a fundamental task in data mining and machine learning. 
        The goal is to predict the categorical label of new observations based on historical data.
        It is crucial for applications such as spam detection and disease diagnosis.
    \end{block}

    \begin{block}{Why Do We Need Data Mining?}
        Data mining uncovers actionable insights from large datasets, aiding informed decision-making. Examples include:
        \begin{itemize}
            \item \textbf{Healthcare}: Classifying patients based on risk to provide timely interventions.
            \item \textbf{Finance}: Detecting fraudulent transactions by analyzing patterns.
        \end{itemize}
    \end{block}
\end{frame}

\begin{frame}[fragile]
    \frametitle{Classification Techniques - Common Methods}
    \begin{enumerate}
        \item \textbf{Logistic Regression}
        \begin{itemize}
            \item Predicts probability for binary classification using the logistic function.
            \item Formula: 
            \begin{equation}
                P(Y=1 | X) = \frac{1}{1 + e^{-(\beta_0 + \beta_1X_1 + \beta_2X_2 + ... + \beta_nX_n)}}
            \end{equation}
            \item \textbf{Example}: Predicting spam emails based on features.
            \item \textbf{Key Points}:
                \begin{itemize}
                    \item Simple to implement.
                    \item Interpretable coefficients.
                \end{itemize}
        \end{itemize}

        \item \textbf{Decision Trees}
        \begin{itemize}
            \item A tree-like model that splits datasets into branches based on feature values.
            \item \textbf{Example}: Classifying a customer based on age, income, and credit score.
            \item \textbf{Key Points}:
                \begin{itemize}
                    \item Easy to visualize.
                    \item Can handle numerical and categorical data.
                \end{itemize}
        \end{itemize}
    \end{enumerate}
\end{frame}

\begin{frame}[fragile]
    \frametitle{Classification Techniques - Neural Networks}
    \begin{itemize}
        \item \textbf{Neural Networks}
        \begin{itemize}
            \item Algorithms modeled after the human brain with interconnected nodes.
            \item Captures complex patterns through multiple layers (deep learning).
            \item \textbf{Example}: Image classification, such as identifying objects in photos.
            \item \textbf{Key Points}:
                \begin{itemize}
                    \item Highly flexible and can approximate almost any function.
                    \item Requires large datasets and significant computational resources.
                    \item Recent applications include natural language processing, e.g., ChatGPT.
                \end{itemize}
        \end{itemize}
    \end{itemize}

    \begin{block}{Summary of Key Takeaways}
        \begin{itemize}
            \item Each technique has strengths and weaknesses; selection depends on the problem.
            \item Logistic regression for binary outcomes, decision trees for interpretability, and neural networks for complex data.
            \item Classification models are pivotal in modern AI applications, enhancing interactivity and accuracy.
        \end{itemize}
    \end{block}
\end{frame}

\begin{frame}[fragile]
    \frametitle{Clustering Techniques}
    \begin{block}{Introduction}
        Clustering is an unsupervised machine learning technique used to group a set of objects such that objects in the same group (or cluster) are more similar to one another than to those in other groups. This method uncovers inherent structures in data without pre-labeled outcomes.
    \end{block}
\end{frame}

\begin{frame}[fragile]
    \frametitle{Why Do We Need Clustering?}
    \begin{itemize}
        \item \textbf{Data Exploration}: Identifies patterns or group trends in exploratory data analysis.
        \item \textbf{Market Segmentation}: Groups customers based on purchasing behavior.
        \item \textbf{Anomaly Detection}: Identifies unusual data points that don't fit expected patterns.
        \item \textbf{Image Segmentation}: Divides an image into segments for easier analysis in computer vision tasks.
    \end{itemize}
\end{frame}

\begin{frame}[fragile]
    \frametitle{Common Clustering Techniques}
    \begin{enumerate}
        \item \textbf{K-Means Clustering}
        \begin{itemize}
            \item Partitions data into K clusters, assigning data points to the nearest mean.
            \item \textbf{Algorithm Steps:}
            \begin{itemize}
                \item Choose number of clusters (K).
                \item Initialize K centroids randomly.
                \item Assign each data point to the nearest centroid.
                \item Recalculate centroids and repeat until convergence.
            \end{itemize}
            \item \textbf{Example:} Customer segmentation based on purchasing habits.
            \begin{equation}
            J = \sum_{i=1}^{k} \sum_{j=1}^{n} ||x_j^{(i)} - \mu_i||^2
            \end{equation}
            where \( \mu_i \) is the centroid of cluster \( i \) and \( x_j^{(i)} \) are the points in cluster \( i \).
        \end{itemize}
        
        \item \textbf{Hierarchical Clustering}
        \begin{itemize}
            \item Builds a hierarchy of clusters either by merging or splitting clusters.
            \item \textbf{Algorithm Steps:}
            \begin{itemize}
                \item Calculate distances between data points.
                \item Combine the closest clusters.
                \item Repeat until all points are clustered.
            \end{itemize}
            \item \textbf{Example:} Taxonomy in species classification.
        \end{itemize}
        
        \item \textbf{DBSCAN}
        \begin{itemize}
            \item Groups points closely located, utilizing distance measurement and point density.
            \item \textbf{Key Parameters:}
            \begin{itemize}
                \item \( \epsilon \): Max distance to consider points in the same neighborhood.
                \item minPts: Minimum points in a dense region.
            \end{itemize}
            \item \textbf{Example:} Clustering geographic data points.
        \end{itemize}
    \end{enumerate}
\end{frame}

\begin{frame}[fragile]
    \frametitle{Applications of Clustering}
    \begin{itemize}
        \item \textbf{Customer Segmentation}: Tailoring marketing efforts based on different customer groups.
        \item \textbf{Social Network Analysis}: Identifying communities through user interactions.
        \item \textbf{Image Compression}: Reducing color usage by grouping similar colors.
    \end{itemize}

    \begin{block}{Key Points}
        \begin{itemize}
            \item Clustering is unsupervised, requiring no labeled data.
            \item The choice of technique depends on the use case and data characteristics.
            \item Preprocessing steps (like normalization) are crucial for accurate results.
        \end{itemize}
    \end{block}
\end{frame}

\begin{frame}[fragile]
    \frametitle{Conclusion}
    Clustering techniques are powerful tools for exploratory data analysis, enabling the discovery of patterns and relationships in datasets. As data continues to grow, especially in fields like AI and machine learning, effective clustering becomes essential for deriving actionable insights and making informed decisions.
\end{frame}

\begin{frame}
    \frametitle{Advanced Topics in Data Mining}
    \begin{block}{Introduction}
        Data Mining encompasses a wide array of techniques designed to discover patterns and insights from large datasets. 
        As we delve deeper into the field, we encounter advanced topics like \textbf{Generative Models} and \textbf{Natural Language Processing (NLP)}. 
        These concepts empower us to extract meaning and create new data based on learned patterns, essential in many real-world applications.
    \end{block}
\end{frame}

\begin{frame}
    \frametitle{Generative Models}
    \begin{block}{Definition}
        Generative models are a class of statistical models that can generate new data points from an underlying distribution by learning the joint probability distributions of inputs and outputs.
    \end{block}
    
    \begin{itemize}
        \item \textbf{Purpose:} Understand how data is generated, allowing creation of new instances similar to the training dataset.
        \item \textbf{Common Types:} 
            \begin{itemize}
                \item Variational Autoencoders (VAEs)
                \item Generative Adversarial Networks (GANs)
            \end{itemize}
    \end{itemize}
    
    \begin{block}{Example}
        \textbf{GANs in Art Generation:} GANs can generate realistic images of art that do not exist by training on a dataset of existing artworks.
    \end{block}
    
    \begin{block}{Diagram}
        \begin{verbatim}
                 +---------+
                 | Generator|
                 +---------+
                     |     
            Fake Image/ Data
                     |
      +-------------+--------------+
      |                            |
 +-------------+          +-------------+
 | Discriminator|          | Training D. |
 +-------------+          +-------------+
       |  Real Image             |
       +-------------------------+
        \end{verbatim}
    \end{block}
\end{frame}

\begin{frame}
    \frametitle{Natural Language Processing (NLP)}
    \begin{block}{Definition}
        NLP involves the use of algorithms to analyze and manipulate human language, enabling machines to understand, interpret, and respond to textual and spoken language.
    \end{block}
    
    \begin{itemize}
        \item \textbf{Applications:}
            \begin{itemize}
                \item Chatbots
                \item Translation services
                \item Sentiment analysis
                \item Text summarization
            \end{itemize}
        \item \textbf{Techniques:} 
            \begin{itemize}
                \item Tokenization
                \item Part-of-speech tagging
                \item Named entity recognition
            \end{itemize}
    \end{itemize}
    
    \begin{block}{Example}
        \textbf{ChatGPT:} Natural language models like ChatGPT utilize NLP for conversational purposes, generating contextually relevant responses based on user input.
    \end{block}
    
    \begin{block}{Code Snippet}
    \begin{lstlisting}[language=Python]
import nltk
nltk.download('punkt')
from nltk.tokenize import word_tokenize

text = "Hello, how are you?"
tokens = word_tokenize(text)
print(tokens)  # Output: ['Hello', ',', 'how', 'are', 'you', '?']
    \end{lstlisting}
    \end{block}
\end{frame}

\begin{frame}
    \frametitle{Conclusion and Summary Points}
    Understanding Generative Models and NLP opens the door to numerous innovative applications in AI, paving the way for smarter technologies that can learn from and interact with human behaviors. 
    These fields represent the cutting-edge of data mining and are critical for future advancements, including project tasks in this course.
    
    \begin{itemize}
        \item \textbf{Generative Models:} Allow for the creation of new data based on learned patterns.
        \item \textbf{NLP:} Enables interactions between computers and humans in natural language.
        \item \textbf{Application Impact:} Both are crucial for driving forward the capabilities of data-driven applications like ChatGPT.
    \end{itemize}
\end{frame}

\begin{frame}
    \frametitle{Hands-On Experience}
    Structure of the hands-on project to apply data mining techniques on real datasets.
\end{frame}

\begin{frame}
    \frametitle{Why Data Mining?}
    \begin{block}{Importance of Data Mining}
        Data mining is critical in various industries as it enables organizations to make data-driven decisions. 
        Modern technologies, such as ChatGPT, effectively utilize data mining techniques to gain insights from massive datasets.
    \end{block}
\end{frame}

\begin{frame}
    \frametitle{Project Components}
    \begin{enumerate}
        \item \textbf{Dataset Selection}
            \begin{itemize}
                \item Choose a dataset relevant to your interests.
                \item Examples include:
                    \begin{itemize}
                        \item Twitter sentiment analysis data
                        \item E-commerce sales data
                        \item Clinical patient records
                    \end{itemize}
            \end{itemize}
        
        \item \textbf{Data Preprocessing}
            \begin{itemize}
                \item Clean and prepare your data.
                \item Tasks include:
                    \begin{itemize}
                        \item Handle missing values
                        \item Normalize or standardize data
                        \item Convert categorical data into numerical format
                    \end{itemize}
            \end{itemize}
    \end{enumerate}
\end{frame}

\begin{frame}
    \frametitle{Project Components Continued}
    \begin{enumerate}
        \setcounter{enumi}{2} % Continue numbering
        \item \textbf{Exploratory Data Analysis (EDA)}
            \begin{itemize}
                \item Analyze the dataset for patterns and trends.
                \item Techniques include:
                    \begin{itemize}
                        \item Visualization (e.g., histograms, scatter plots)
                        \item Descriptive statistics (mean, median, mode)
                    \end{itemize}
                \item Example: Use a scatter plot to identify correlations.
            \end{itemize}
        
        \item \textbf{Model Selection}
            \begin{itemize}
                \item Choose techniques based on your project goal.
                \item Examples include:
                    \begin{itemize}
                        \item Decision Trees for classification
                        \item K-Means for clustering
                    \end{itemize}
            \end{itemize}
    \end{enumerate}
\end{frame}

\begin{frame}[fragile]
    \frametitle{Model Implementation}
    Implement the selected models using Python or R:
    
    \begin{lstlisting}[language=Python]
    from sklearn.datasets import load_iris
    from sklearn.tree import DecisionTreeClassifier
    from sklearn.model_selection import train_test_split
    
    # Load dataset
    data = load_iris()
    X = data.data
    y = data.target
    
    # Train-Test Split
    X_train, X_test, y_train, y_test = train_test_split(X, y, test_size=0.3, random_state=42)
    
    # Create Decision Tree model
    model = DecisionTreeClassifier()
    model.fit(X_train, y_train)
    
    # Predictions
    predictions = model.predict(X_test)
    \end{lstlisting}
\end{frame}

\begin{frame}
    \frametitle{Model Evaluation and Reporting Findings}
    \begin{itemize}
        \item \textbf{Model Evaluation}
            \begin{itemize}
                \item Assess effectiveness using metrics:
                    \begin{itemize}
                        \item Accuracy
                        \item Precision and Recall
                        \item F1-score
                    \end{itemize}
                \item Example: Use a confusion matrix to analyze predictions.
            \end{itemize}
        
        \item \textbf{Reporting Findings}
            \begin{itemize}
                \item Document process, findings, and insights.
                \item Summarize methodology and results in a report or presentation.
            \end{itemize}
    \end{itemize}
\end{frame}

\begin{frame}
    \frametitle{Key Takeaways}
    \begin{itemize}
        \item Importance of Data Mining and its role in decision-making.
        \item Real-World Application of techniques in modern technologies.
        \item Iterative Nature of Data Analysis—be open to revisiting earlier steps.
    \end{itemize}
\end{frame}

\begin{frame}[fragile]
    \frametitle{Performance Evaluation of Models - Introduction}
    \begin{block}{Introduction to Performance Evaluation}
        In data mining and machine learning, model effectiveness is gauged through specific metrics, crucial for understanding generalization to new data and problem-solving.
        Selecting the right metric enables fine-tuning and successful application of models.
    \end{block}
\end{frame}

\begin{frame}[fragile]
    \frametitle{Performance Evaluation of Models - Key Metrics}
    \begin{block}{Key Evaluation Metrics}
        \begin{enumerate}
            \item \textbf{Accuracy}
            \begin{itemize}
                \item \textbf{Definition}: Ratio of correctly predicted instances to total instances.
                \item \textbf{Formula}:
                \begin{equation}
                \text{Accuracy} = \frac{\text{True Positives} + \text{True Negatives}}{\text{Total Instances}}
                \end{equation}
                \item \textbf{Interpretation}: High accuracy indicates good performance, but it's not reliable for imbalanced datasets.
                \item \textbf{Example}: 80 correct out of 100 instances gives 80\% accuracy.
            \end{itemize}

            \item \textbf{F1-Score}
            \begin{itemize}
                \item \textbf{Definition}: Balances precision and recall.
                \item \textbf{Formula}:
                \begin{equation}
                \text{F1-Score} = 2 \times \frac{\text{Precision} \times \text{Recall}}{\text{Precision} + \text{Recall}}
                \end{equation}
                \item \textbf{Where}:
                \begin{equation}
                \text{Precision} = \frac{\text{True Positives}}{\text{True Positives} + \text{False Positives}}
                \end{equation}
                \begin{equation}
                \text{Recall} = \frac{\text{True Positives}}{\text{True Positives} + \text{False Negatives}}
                \end{equation}
                \item \textbf{Interpretation}: Useful for uneven class distributions, balancing positive finds vs false positives.
            \end{itemize}
        \end{enumerate}
    \end{block}
\end{frame}

\begin{frame}[fragile]
    \frametitle{Performance Evaluation of Models - Examples}
    \begin{block}{Example of F1-Score Calculation}
        In a dataset with 100 instances (30 positive, 70 negative):
        
        \begin{itemize}
            \item Model predicts:
                \begin{itemize}
                    \item 25 True Positives
                    \item 5 False Positives
                    \item 5 False Negatives
                \end{itemize}

            \item \textbf{Calculations}:
                \begin{equation}
                \text{Precision} = \frac{25}{25 + 5} = 0.833 \quad (83.3\%)
                \end{equation}
                \begin{equation}
                \text{Recall} = \frac{25}{25 + 5} = 0.833 \quad (83.3\%)
                \end{equation}
                \begin{equation}
                \text{F1-Score} = 2 \times \frac{0.833 \times 0.833}{0.833 + 0.833} \approx 0.833 \quad (83.3\%)
                \end{equation}
        \end{itemize}
    \end{block}
\end{frame}

\begin{frame}[fragile]
    \frametitle{Performance Evaluation of Models - Key Points and Conclusion}
    \begin{block}{Key Points}
        \begin{itemize}
            \item \textbf{Accuracy} is intuitive but can be misleading in imbalanced scenarios.
            \item \textbf{F1-Score} provides a more balanced view, suitable for rare positive instances.
            \item Evaluate multiple metrics for comprehensive performance analysis.
        \end{itemize}
    \end{block}
    
    \begin{block}{Conclusion}
        Performance evaluation is crucial for model development. Understanding metrics like accuracy and F1-score enables informed decisions and model enhancements, applicable to real-world data mining, such as improving systems like ChatGPT.
    \end{block}
\end{frame}

\begin{frame}[fragile]
    \frametitle{Collaborative Learning Approach - Overview}
    \begin{block}{Overview of Collaborative Learning}
        Collaborative learning is an educational approach where students work together to complete tasks or projects. This promotes teamwork and shared responsibility, enhancing learning outcomes and developing essential skills such as:
        \begin{itemize}
            \item Communication
            \item Critical Thinking
            \item Problem-solving
        \end{itemize}
    \end{block}
    
    \begin{block}{Why Collaborative Learning?}
        \begin{enumerate}
            \item \textbf{Diverse Perspectives:}  Leads to richer discussions and innovative solutions.
            \item \textbf{Peer Learning:} Reinforces understanding and builds confidence.
            \item \textbf{Preparation for Real-world Scenarios:} Simulates environments requiring collaboration.
        \end{enumerate}
    \end{block}
\end{frame}

\begin{frame}[fragile]
    \frametitle{Collaborative Learning Approach - Group Projects}
    \begin{block}{Group Projects in This Course}
        You will participate in various group projects that emphasize collaboration. Expect the following:
        \begin{itemize}
            \item \textbf{Team Formation:} Diverse groups of 4-6 members to leverage varied strengths.
            \item \textbf{Project Topics:} Real-world applications of data mining, such as analyzing datasets or creating predictive models.
            \item \textbf{Collaboration Tools:} Use of Google Workspace, Trello, Slack for seamless communication and project management.
        \end{itemize}
    \end{block}
\end{frame}

\begin{frame}[fragile]
    \frametitle{Collaborative Learning Approach - Key Points}
    \begin{block}{Key Points to Remember}
        \begin{itemize}
            \item \textbf{Roles and Responsibilities:} Each member will assume specific roles contributing to project success.
            \item \textbf{Assessment Criteria:} Projects evaluated based on group dynamics, quality, and individual contributions with clear rubrics.
            \item \textbf{Feedback Mechanism:} Continuous constructive feedback encouraged throughout the project timeline.
        \end{itemize}
    \end{block}

    \begin{block}{Conclusion}
        Group projects boost your knowledge of data mining concepts and strengthen your soft skills. Embrace collaboration to maximize learning potential!
    \end{block}
\end{frame}

\begin{frame}[fragile]
    \frametitle{Course Resources and Computing Requirements - Overview}
    To maximize your learning experience in this course, it is essential to familiarize yourself with the necessary resources, software applications, and computing requirements. 
    \begin{itemize}
        \item Required Resources
        \item Software Requirements
        \item Computing Requirements
    \end{itemize}
\end{frame}

\begin{frame}[fragile]
    \frametitle{Course Resources and Computing Requirements - Required Resources}
    \begin{enumerate}
        \item \textbf{Textbooks and Reading Materials}
        \begin{itemize}
            \item \textbf{Primary Textbook}: Ensure you have access to the designated textbook that covers key concepts thoroughly.
            \item \textbf{Supplementary Resources}: Additional readings will be provided throughout the course to deepen your understanding.
        \end{itemize}

        \item \textbf{Online Platforms}
        \begin{itemize}
            \item \textbf{Course LMS}: Regularly access the LMS for assignments, grades, and announcements.
            \item \textbf{Collaboration Tools}: Utilize platforms like Google Workspace or Microsoft Teams for group projects.
        \end{itemize}
    \end{enumerate}
\end{frame}

\begin{frame}[fragile]
    \frametitle{Course Resources and Computing Requirements - Software and Computing}
    \begin{block}{Software Requirements}
        \begin{enumerate}
            \item \textbf{Data Analysis Tools}
            \begin{itemize}
                \item \textbf{Python with Jupyter Notebooks}: Install Anaconda for interactive coding.
                \item \textbf{R and RStudio}: Optional software useful for statistical analysis.
            \end{itemize}

            \item \textbf{Statistical and Visualization Tools}
            \begin{itemize}
                \item \textbf{Excel or Google Sheets}: Essential for basic data manipulation and visualization.
                \item \textbf{Tableau or Power BI}: Advanced data visualization tools for group projects.
            \end{itemize}
        \end{enumerate}
    \end{block}

    \begin{block}{Computing Requirements}
        \begin{enumerate}
            \item \textbf{Hardware Specifications}
            \begin{itemize}
                \item \textbf{Minimum}: 8 GB RAM, 2.5 GHz processor, and at least 20 GB free disk space.
                \item \textbf{Recommended}: 16 GB RAM for better performance during data-heavy tasks.
            \end{itemize}

            \item \textbf{Internet Access}: A stable internet connection is vital to access course materials and participate in online classes.
        \end{enumerate}
    \end{block}
\end{frame}

\begin{frame}[fragile]
    \frametitle{Course Resources and Computing Requirements - Key Points and Summary}
    \begin{itemize}
        \item Ensure you set up your software before the first class to avoid delays.
        \item Familiarize yourself with collaborative tools early on, as they are crucial for group projects.
        \item Stay updated with any recommended readings or resources provided throughout the course.
    \end{itemize}

    \textbf{Summary:} Being equipped with the right resources, software, and computing specifications will facilitate your learning and enhance collaboration. Prepare accordingly for a smooth start.
\end{frame}

\begin{frame}[fragile]
    \frametitle{Course Policies}
    Important academic integrity and accessibility policies to follow.
\end{frame}

\begin{frame}[fragile]
    \frametitle{Academic Integrity Policy}
    \begin{block}{Definition}
        Academic integrity refers to the ethical guidelines and principles that govern how academic work is conducted. This means submitting your original work, accurately citing sources, and refraining from cheating or plagiarism.
    \end{block}
    \begin{itemize}
        \item \textbf{Plagiarism:} Using someone else's work, ideas, or data without proper attribution.
        \item \textbf{Collaboration vs. Cheating:} Collaboration is encouraged, but submitting work that is not your own or allowing others to copy your work is prohibited.
    \end{itemize}
\end{frame}

\begin{frame}[fragile]
    \frametitle{Academic Integrity - Examples}
    \begin{itemize}
        \item \textbf{Example of Plagiarism:} Copying text from a website without using quotation marks or a citation.
        \item \textbf{Example of Acceptable Collaboration:} Studying together and discussing concepts while writing your own assignments.
    \end{itemize}
\end{frame}

\begin{frame}[fragile]
    \frametitle{Accessibility Policy}
    \begin{block}{Definition}
        Accessibility ensures that all students, regardless of ability, have equal access to learning materials and opportunities in the course.
    \end{block}
    \begin{itemize}
        \item \textbf{Accommodations:} Students with disabilities may request accommodations (e.g., extended time for exams, accessible resources).
        \item \textbf{Inclusivity:} All course materials will be available in multiple formats (e.g., text, video, audio) to support diverse learning preferences.
    \end{itemize}
\end{frame}

\begin{frame}[fragile]
    \frametitle{Accessibility - Example}
    \begin{itemize}
        \item Providing written transcripts for video lectures to assist students who are deaf or hard of hearing.
    \end{itemize}
\end{frame}

\begin{frame}[fragile]
    \frametitle{Consequences of Policy Violations}
    \begin{itemize}
        \item \textbf{Academic Integrity Violations:} May result in penalties including failing grades, suspension, or expulsion.
        \item \textbf{Accessibility Violations:} May lead to corrective actions to ensure future courses meet required accessibility standards.
    \end{itemize}
\end{frame}

\begin{frame}[fragile]
    \frametitle{Resources for Support}
    \begin{itemize}
        \item \textbf{Academic Integrity Office:} Contact for support and clarification on integrity issues.
        \item \textbf{Accessibility Services:} Reach out for assistance in obtaining necessary accommodations.
    \end{itemize}
\end{frame}

\begin{frame}[fragile]
    \frametitle{Summary}
    \begin{enumerate}
        \item \textbf{Academic Integrity:} Adhere to ethical standards in your work.
        \item \textbf{Accessibility:} Ensure equal opportunities for all students.
        \item \textbf{Consequences:} Be aware of the penalties for violations.
        \item \textbf{Resources:} Utilize available support services.
    \end{enumerate}
    \begin{block}{Commitment}
        Let's commit to maintaining academic integrity and fostering an accessible learning space for everyone!
    \end{block}
\end{frame}

\begin{frame}[fragile]
    \frametitle{Syllabus Overview - Introduction}
    \begin{block}{Welcome to the Course}
       Welcome to the course! This syllabus serves as a roadmap for our journey together. It outlines essential components, including the weekly schedule, major topics, and expectations for the semester.
    \end{block}
\end{frame}

\begin{frame}[fragile]
    \frametitle{Syllabus Overview - Weekly Schedule}
    \begin{block}{Weekly Schedule - Breakdown}
        \begin{itemize}
            \item **Week 1:** Course Introduction
                \begin{itemize}
                    \item Overview of the syllabus and academic integrity.
                \end{itemize}
            \item **Week 2:** Understanding Core Concepts
                \begin{itemize}
                    \item Introduction to specific concepts based on course focus.
                \end{itemize}
            \item **Week 3:** Data Mining Techniques
                \begin{itemize}
                    \item Overview of data mining: definitions and techniques, e.g., classification, clustering, regression.
                \end{itemize}
            \item **Week 4:** Recent Advances in AI
                \begin{itemize}
                    \item Discussion on AI applications like ChatGPT and data mining relevance.
                \end{itemize}
            \item **Week 5:** Practical Applications
                \begin{itemize}
                    \item Case studies in healthcare, marketing, and finance.
                \end{itemize}
            \item **Week 6:** Ethical Considerations
                \begin{itemize}
                    \item Exploring privacy and data use in data mining.
                \end{itemize}
            \item **Weeks 7-8:** Group Project Preparation
            \item **Week 9:** Midterm Review
            \item **Weeks 10-11:** Advanced Topics
            \item **Week 12:** Presentations
            \item **Weeks 13-14:** Future Trends
            \item **Week 15:** Course Wrap-Up and Reflection
        \end{itemize}
    \end{block}
\end{frame}

\begin{frame}[fragile]
    \frametitle{Syllabus Overview - Key Points}
    \begin{block}{Key Points to Emphasize}
        \begin{enumerate}
            \item Importance of Understanding the Syllabus:
                \begin{itemize}
                    \item Ensures students are well-prepared for the course structure and expectations.
                \end{itemize}
            \item Structured Learning Path:
                \begin{itemize}
                    \item Weekly topics build upon each other, culminating in projects and presentations.
                \end{itemize}
            \item Real-World Relevance:
                \begin{itemize}
                    \item The course prepares students for practical scenarios in their future careers.
                \end{itemize}
        \end{enumerate}
    \end{block}
\end{frame}

\begin{frame}[fragile]
    \frametitle{Syllabus Overview - Engagement and Participation}
    \begin{block}{Engagement and Participation}
        \begin{itemize}
            \item **Class Participation:**
                \begin{itemize}
                    \item Encourage active participation through discussions and questions.
                \end{itemize}
            \item **Office Hours:**
                \begin{itemize}
                    \item Regular office hours will be available for additional support.
                \end{itemize}
        \end{itemize}
    \end{block}
\    
    By understanding the syllabus and the course schedule, students can ensure they remain aligned with both course outcomes and personal learning goals.
\end{frame}

\begin{frame}[fragile]
    \frametitle{Assessment Strategy - Overview}
    Assessment plays a crucial role in measuring students' understanding and engagement with course material. This slide highlights the types of assessments we will be using throughout the course, their respective weights, and how they will contribute to your final grade.
\end{frame}

\begin{frame}[fragile]
    \frametitle{Types of Assessments}
    \begin{enumerate}
        \item \textbf{Quizzes (20\%)}
            \begin{itemize}
                \item \textbf{Description:} Short, frequent quizzes following each module to check understanding.
                \item \textbf{Purpose:} Reinforces key concepts and ensures you are keeping up with the material.
                \item \textbf{Example:} Quiz on the course syllabus and expectations after the first class.
            \end{itemize}

        \item \textbf{Assignments (30\%)}
            \begin{itemize}
                \item \textbf{Description:} More in-depth written assignments reflecting on weekly topics.
                \item \textbf{Purpose:} Encourages deeper exploration of subjects.
                \item \textbf{Example:} A 2-page essay on “The Importance of Data Mining in Today's AI Applications.”
            \end{itemize}

        \item \textbf{Midterm Exam (25\%)}
            \begin{itemize}
                \item \textbf{Description:} A comprehensive exam covering all topics discussed in the first half of the course.
                \item \textbf{Purpose:} Tests overall understanding and retention of the course material.
                \item \textbf{Example:} Multiple-choice and short-answer questions regarding data mining techniques.
            \end{itemize}

        \item \textbf{Final Project (25\%)}
            \begin{itemize}
                \item \textbf{Description:} A practical project applying data mining techniques to a real-world problem.
                \item \textbf{Purpose:} Provides hands-on experience and demonstrates the application of course knowledge.
                \item \textbf{Example:} A detailed report and presentation on a data mining project analyzing AI algorithms.
            \end{itemize}
    \end{enumerate}
\end{frame}

\begin{frame}[fragile]
    \frametitle{Assessment Strategy - Key Points}
    \begin{itemize}
        \item \textbf{Consistency is Key:} Quizzes and assignments help you stay engaged and understand material continuously, rather than cramming before exams.
        
        \item \textbf{Application of Knowledge:} The final project allows you to apply what you've learned in a practical context, bridging theory with real-world applications.
        
        \item \textbf{Continuous Feedback:} Regular assessments will provide immediate feedback, allowing you to identify areas of strength and those needing improvement.
    \end{itemize}

    \textbf{Conclusion:} Understanding this assessment strategy helps you better navigate course expectations and utilize feedback to enhance your learning experience. Engaging actively with assignments and seeking help when needed is crucial for your understanding and growth in this subject.
\end{frame}

\begin{frame}[fragile]
    \frametitle{Introduction to Data Mining Applications in AI}
    \begin{block}{What is Data Mining?}
        \begin{itemize}
            \item \textbf{Definition:} The process of discovering patterns and extracting valuable information from large data sets.
            \item \textbf{Motivation:} The need for tools to handle exponential data growth in various fields, enabling decision-making and insights.
        \end{itemize}
    \end{block}
\end{frame}

\begin{frame}[fragile]
    \frametitle{Role of Data Mining in AI}
    Data mining enhances AI applications by:
    \begin{itemize}
        \item \textbf{Feeding AI Models with Quality Data:}
        High-quality datasets from data mining form the foundation for robust AI models like ChatGPT.
        
        \item \textbf{Improving Learning Algorithms:}
        Optimizes machine learning algorithms by identifying significant patterns for better predictions and responses.
        
        \item \textbf{Supporting Personalization:}
        Enables AI to understand user preferences through analysis of interactions, tailoring responses accordingly.
    \end{itemize}
\end{frame}

\begin{frame}[fragile]
    \frametitle{Examples of Data Mining Techniques in AI}
    Key techniques include:
    \begin{itemize}
        \item \textbf{Clustering:} Groups similar data points.
        \begin{itemize}
            \item \textit{Example:} Clustering user queries to identify common topics and enhance response accuracy in ChatGPT.
        \end{itemize}
        
        \item \textbf{Classification:} Assigns labels based on learned features.
        \begin{itemize}
            \item \textit{Example:} Classifying user intents in ChatGPT (e.g., questions, requests).
        \end{itemize}
        
        \item \textbf{Association Rule Learning:} Finds relationships in data.
        \begin{itemize}
            \item \textit{Example:} Utilizing associations to understand context in conversations.
        \end{itemize}
    \end{itemize}
\end{frame}

\begin{frame}[fragile]
    \frametitle{Key Points and Conclusion}
    \begin{block}{Key Points}
        \begin{itemize}
            \item Data mining is crucial for extracting actionable insights in AI.
            \item Techniques like clustering, classification, and association rules enhance AI performance, especially for conversational agents like ChatGPT.
            \item Continuous refinement of AI systems depends on effective data mining practices.
        \end{itemize}
    \end{block}
    
    \begin{block}{Concluding Thought}
        Data mining is a core component that drives modern AI capabilities, fostering innovation and improving user experience through intelligent data analysis.
    \end{block}
\end{frame}

\begin{frame}[fragile]
  \frametitle{Conclusion of Week 1 - Overview}
  \begin{block}{Summary}
    This week, we explored the foundations of data mining, its relevance in artificial intelligence (AI), and the expectations for next week.
  \end{block}
\end{frame}

\begin{frame}[fragile]
  \frametitle{Conclusion of Week 1 - Recap of Key Concepts}
  
  \begin{enumerate}
    \item \textbf{Understanding Data Mining:}
    \begin{itemize}
      \item Involves extracting insights from large datasets.
      \item Key techniques: clustering, classification, and association rule mining.
    \end{itemize}

    \item \textbf{Applications of Data Mining in AI:}
    \begin{itemize}
      \item Enhances AI algorithms by providing patterns and predictions.
      \item Applications like ChatGPT utilize data mining for user interactions.
    \end{itemize}

    \item \textbf{Importance of Data Mining in AI:}
    \begin{itemize}
      \item Motivation: Need for efficient data analysis with data growth.
      \item Example: Social media uses data mining to recommend content.
    \end{itemize}
  \end{enumerate}
\end{frame}

\begin{frame}[fragile]
  \frametitle{Conclusion of Week 1 - Expectations Moving Forward}

  \begin{enumerate}
    \item \textbf{In-Depth Exploration:}
    \begin{itemize}
      \item Next week: Detailed study of specific data mining techniques.
      \item Expect hands-on assignments with datasets.
    \end{itemize}

    \item \textbf{Greater Application to AI Developments:}
    \begin{itemize}
      \item Deeper insights into data mining in cutting-edge applications.
      \item Discussions on ethics and best practices.
    \end{itemize}

    \item \textbf{Collaborative Learning:}
    \begin{itemize}
      \item Engage in group discussions and peer reviews.
      \item Participation in forums encouraged.
    \end{itemize}
  \end{enumerate}
\end{frame}

\begin{frame}[fragile]
  \frametitle{Conclusion of Week 1 - Key Points and Next Steps}

  \begin{block}{Key Points to Remember}
    \begin{itemize}
      \item Data mining is critical for AI, converting raw data into information.
      \item Mastery of tools and techniques is essential for real-world applications.
      \item Stay proactive: Ask questions and engage in discussions.
    \end{itemize}
  \end{block}

  \begin{block}{Next Steps}
    \begin{itemize}
      \item Review notes and think about data mining applications in your field.
      \item Prepare for an interactive quiz on clustering and classification methods.
    \end{itemize}
  \end{block}
\end{frame}

\begin{frame}[fragile]
  \frametitle{Conclusion of Week 1 - Discussion Questions}
  
  \begin{itemize}
    \item How do you see data mining impacting industries you're interested in?
    \item Can you think of examples where AI could be enhanced through better data mining techniques?
  \end{itemize}

  \begin{block}{}
    Let’s embark on this exciting journey into the world of data and AI together!
  \end{block}
\end{frame}


\end{document}