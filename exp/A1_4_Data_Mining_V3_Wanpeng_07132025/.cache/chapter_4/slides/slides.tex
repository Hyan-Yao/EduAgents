\documentclass[aspectratio=169]{beamer}

% Theme and Color Setup
\usetheme{Madrid}
\usecolortheme{whale}
\useinnertheme{rectangles}
\useoutertheme{miniframes}

% Additional Packages
\usepackage[utf8]{inputenc}
\usepackage[T1]{fontenc}
\usepackage{graphicx}
\usepackage{booktabs}
\usepackage{listings}
\usepackage{amsmath}
\usepackage{amssymb}
\usepackage{xcolor}
\usepackage{tikz}
\usepackage{pgfplots}
\pgfplotsset{compat=1.18}
\usetikzlibrary{positioning}
\usepackage{hyperref}

% Custom Colors
\definecolor{myblue}{RGB}{31, 73, 125}
\definecolor{mygray}{RGB}{100, 100, 100}
\definecolor{mygreen}{RGB}{0, 128, 0}
\definecolor{myorange}{RGB}{230, 126, 34}
\definecolor{mycodebackground}{RGB}{245, 245, 245}

% Set Theme Colors
\setbeamercolor{structure}{fg=myblue}
\setbeamercolor{frametitle}{fg=white, bg=myblue}
\setbeamercolor{title}{fg=myblue}
\setbeamercolor{section in toc}{fg=myblue}
\setbeamercolor{item projected}{fg=white, bg=myblue}
\setbeamercolor{block title}{bg=myblue!20, fg=myblue}
\setbeamercolor{block body}{bg=myblue!10}
\setbeamercolor{alerted text}{fg=myorange}

% Set Fonts
\setbeamerfont{title}{size=\Large, series=\bfseries}
\setbeamerfont{frametitle}{size=\large, series=\bfseries}
\setbeamerfont{caption}{size=\small}
\setbeamerfont{footnote}{size=\tiny}

% Footer and Navigation Setup
\setbeamertemplate{footline}{
  \leavevmode%
  \hbox{%
  \begin{beamercolorbox}[wd=.3\paperwidth,ht=2.25ex,dp=1ex,center]{author in head/foot}%
    \usebeamerfont{author in head/foot}\insertshortauthor
  \end{beamercolorbox}%
  \begin{beamercolorbox}[wd=.5\paperwidth,ht=2.25ex,dp=1ex,center]{title in head/foot}%
    \usebeamerfont{title in head/foot}\insertshorttitle
  \end{beamercolorbox}%
  \begin{beamercolorbox}[wd=.2\paperwidth,ht=2.25ex,dp=1ex,center]{date in head/foot}%
    \usebeamerfont{date in head/foot}
    \insertframenumber{} / \inserttotalframenumber
  \end{beamercolorbox}}%
  \vskip0pt%
}

% Turn off navigation symbols
\setbeamertemplate{navigation symbols}{}

% Title Page Information
\title[Supervised Learning]{Weeks 4-5: Supervised Learning (Classification Techniques)}
\author[J. Smith]{John Smith, Ph.D.}
\institute[University Name]{
  Department of Computer Science\\
  University Name\\
  \vspace{0.3cm}
  Email: email@university.edu\\
  Website: www.university.edu
}
\date{\today}

% Document Start
\begin{document}

\frame{\titlepage}

\begin{frame}[fragile]
    \frametitle{Introduction to Supervised Learning}
    \begin{block}{Overview}
        Supervised learning is a branch of machine learning where algorithms learn from labeled training data. It is fundamental in data mining, enabling models to predict outcomes or classify data points based on observed patterns.
    \end{block}
\end{frame}

\begin{frame}[fragile]
    \frametitle{Why Do We Need Data Mining?}
    \begin{block}{Importance of Data Mining}
        Data mining is essential in extracting useful information from vast datasets, especially in today's data-rich world. Supervised learning plays a significant role in this process.
    \end{block}
    \begin{enumerate}
        \item \textbf{Decision Making:} Businesses use supervised learning to make data-driven choices, such as banks determining creditworthiness.
        \item \textbf{Improving Accuracy:} Historical data training enhances predictions, like forecasting customer behavior based on past habits.
        \item \textbf{Handling Complex Data:} Effective for managing complex relationships, e.g., healthcare data for disease diagnosis.
    \end{enumerate}
\end{frame}

\begin{frame}[fragile]
    \frametitle{Key Concepts in Supervised Learning}
    \begin{itemize}
        \item \textbf{Labeled Data:} Each input point has a corresponding output label (e.g., emails labeled as 'spam' or 'not spam').
        \item \textbf{Training and Testing Datasets:} Datasets are split into training to build models and testing to evaluate their performance.
        \item \textbf{Common Algorithms:}
            \begin{itemize}
                \item \textbf{Decision Trees:} Visualize decisions and consequences.
                \item \textbf{Support Vector Machines (SVM):} Find optimal hyperplanes to separate classes.
                \item \textbf{Neural Networks:} Mimic human brain functioning to recognize complex patterns.
            \end{itemize}
    \end{itemize}
\end{frame}

\begin{frame}[fragile]
    \frametitle{Recent Applications in AI}
    \begin{block}{Applications of Supervised Learning}
        Supervised learning is central to many AI applications:
    \end{block}
    \begin{itemize}
        \item \textbf{ChatGPT:} Utilizes extensive datasets to learn and predict conversational patterns, ensuring contextually relevant responses.
        \item \textbf{Healthcare:} Uses classification techniques for predicting and diagnosing diseases, thereby improving patient care.
    \end{itemize}
\end{frame}

\begin{frame}[fragile]
    \frametitle{Summary and Key Points}
    \begin{itemize}
        \item Supervised learning requires labeled data to train models.
        \item Crucial for informed decision-making, accuracy improvement, and handling complex datasets.
        \item Key algorithms encompass Decision Trees, SVMs, and Neural Networks.
        \item Significant real-world applications span across finance and healthcare, including advancements in technologies like ChatGPT.
    \end{itemize}
\end{frame}

\begin{frame}[fragile]
    \frametitle{Real-World Applications of Classification - Introduction}
    \begin{itemize}
        \item Classification is a vital technique in supervised learning.
        \item It categorizes data into predefined classes.
        \item Significantly impacts sectors like finance and healthcare.
        \item Understanding real-world applications highlights its importance and effectiveness.
    \end{itemize}
\end{frame}

\begin{frame}[fragile]
    \frametitle{Real-World Applications of Classification - Finance}
    \begin{enumerate}
        \item \textbf{Fraud Detection}
            \begin{itemize}
                \item Used to identify fraudulent transactions.
                \item Models trained on historical data distinguish between legitimate and suspicious activities.
                \item \textbf{Example:} A bank might use logistic regression to analyze features like amount, time, and location to predict fraud.
            \end{itemize}
            
        \item \textbf{Credit Scoring}
            \begin{itemize}
                \item Classifies applicants based on default likelihood for loans.
                \item Classification algorithms like decision trees and support vector machines are utilized.
                \item \textbf{Example:} A model evaluates if an applicant qualifies for a loan based on credit history and income.
            \end{itemize}
    \end{enumerate}
\end{frame}

\begin{frame}[fragile]
    \frametitle{Real-World Applications of Classification - Healthcare}
    \begin{enumerate}
        \item \textbf{Disease Diagnosis}
            \begin{itemize}
                \item Assists clinicians in diagnosing diseases from medical data.
                \item Analyzes symptoms, test results, and patient histories.
                \item \textbf{Example:} Random forests classify patients as diabetic or not based on blood sugar levels and BMI.
            \end{itemize}

        \item \textbf{Patient Risk Assessment}
            \begin{itemize}
                \item Predicts the risk of developing certain conditions.
                \item Helps in providing tailored interventions.
                \item \textbf{Example:} Logistic regression assesses heart disease risk, categorizing patients as low, medium, or high based on variables.
            \end{itemize}
    \end{enumerate}
\end{frame}

\begin{frame}[fragile]
    \frametitle{Key Points and Conclusion}
    \begin{block}{Key Points to Emphasize}
        \begin{itemize}
            \item Classification techniques enable actionable insights through data categorization.
            \item Significant enhancements in decision-making processes in finance and healthcare.
            \item Classification is a cornerstone of data mining, driving efficiency in applications.
        \end{itemize}
    \end{block}
    
    \begin{block}{Conclusion}
        Understanding and leveraging classification techniques is crucial across diverse fields. Recent innovations in AI illustrate the impact of classification on decision-making and daily life.
    \end{block}
\end{frame}

\begin{frame}[fragile]
    \frametitle{Next Steps}
    In the following slide, we will delve into one of the fundamental classification algorithms: \textbf{Logistic Regression}, exploring its mechanics and applications.
\end{frame}

\begin{frame}[fragile]
    \frametitle{What is Logistic Regression?}
    % Introduction to logistic regression as a classification technique
    \begin{block}{Introduction to Logistic Regression}
        Logistic regression is a statistical method used for binary classification problems, where the outcome variable is categorical (specifically, it has two possible outcomes).
    \end{block}
    
    \begin{block}{Why is it Needed?}
        Understanding the relationship between independent variables (features) and a dependent binary variable (outcome) is crucial in various fields. For instance, in healthcare, we might want to predict whether a patient has a disease (Yes/No) based on various risk factors.
    \end{block}
\end{frame}

\begin{frame}[fragile]
    \frametitle{How Logistic Regression Works}
    % Explanation of how logistic regression functions
    \begin{block}{Logit Function}
        Logistic regression uses a logistic function to model the probability that a given input point belongs to a certain class. The logit function can be expressed as:

        \begin{equation}
        P(Y=1|X) = \frac{1}{1 + e^{-(\beta_0 + \beta_1X_1 + \beta_2X_2 + ... + \beta_nX_n)}}
        \end{equation}
        
        Here:
        \begin{itemize}
            \item \( P(Y=1|X) \) is the probability of the outcome being 1 given input features \( X \).
            \item \( \beta_0 \) is the intercept, while \( \beta_1, \beta_2, ..., \beta_n \) are the coefficients representing the impact of each feature.
        \end{itemize}
    \end{block}

    \begin{block}{Interpreting Coefficients}
        Each coefficient represents the change in the log-odds of the outcome for a one-unit increase in the feature, holding other features constant.
    \end{block}
\end{frame}

\begin{frame}[fragile]
    \frametitle{Example: Predicting Customer Churn}
    % A case study example related to logistic regression
    \begin{block}{Case Study}
        Consider a telecom company wanting to predict if a customer will leave (churn) based on features such as:
        \begin{itemize}
            \item Monthly Bill Amount
            \item Number of Complaints
            \item Contract Type
        \end{itemize}
        
        If logistic regression finds significant relationships among these features, the company can target at-risk customers with tailored retention strategies.
    \end{block}

    \begin{block}{Key Points to Emphasize}
        \begin{enumerate}
            \item Logistic regression is ideal for binary outcomes.
            \item It provides probabilistic interpretations, crucial for understanding prediction certainties.
            \item This technique is widely used in fields like finance, healthcare, and marketing.
        \end{enumerate}
    \end{block}
\end{frame}

\begin{frame}[fragile]
    \frametitle{Conclusion}
    % Summary of logistic regression's importance
    Logistic regression is a powerful tool in supervised learning for classification tasks. It helps practitioners derive actionable insights from data and lays the groundwork for understanding more complex classification algorithms that may follow, such as support vector machines (SVM) or neural networks.

    \begin{block}{Next Steps}
        Transition to the next slide to delve into the key concepts behind logistic regression, including odds, log-odds, and the logistic function.
    \end{block}
\end{frame}

\begin{frame}[fragile]
    \frametitle{Logistic Regression: Key Concepts - Introduction}
    \begin{itemize}
        \item Logistic regression is a classification technique in supervised learning.
        \item It estimates the probability of a binary outcome based on predictor variables.
        \item Key concepts to explore: 
        \begin{itemize}
            \item Odds
            \item Log-Odds (Logit)
            \item Logistic Function
        \end{itemize}
    \end{itemize}
\end{frame}

\begin{frame}[fragile]
    \frametitle{Logistic Regression: Key Concepts - Odds}
    \begin{block}{1. Odds}
        \begin{itemize}
            \item \textbf{Definition}: Odds represent the ratio of the probability of an event occurring to it not occurring.
            \item \textbf{Formula}:
            \begin{equation}
                \text{Odds}(Y=1) = \frac{P(Y=1)}{P(Y=0)}
            \end{equation}
            \item \textbf{Example}: If $P(Y=1) = 0.8$, then:
            \begin{equation}
                \text{Odds} = \frac{0.8}{0.2} = 4
            \end{equation}
            This means passing is four times more likely than failing.
        \end{itemize}
    \end{block}
\end{frame}

\begin{frame}[fragile]
    \frametitle{Logistic Regression: Key Concepts - Log-Odds and Logistic Function}
    \begin{block}{2. Log-Odds (Logit)}
        \begin{itemize}
            \item \textbf{Definition}: The log-odds is the natural logarithm of the odds.
            \item \textbf{Formula}:
            \begin{equation}
                \text{Log-Odds} = \log\left(\frac{P(Y=1)}{P(Y=0)}\right) = \log\left(\frac{P(Y=1)}{1 - P(Y=1)}\right)
            \end{equation}
            \item \textbf{Example}: Given odds of 4:
            \begin{equation}
                \text{Log-Odds} = \log(4) \approx 1.386
            \end{equation}
        \end{itemize}
    \end{block}

    \begin{block}{3. Logistic Function}
        \begin{itemize}
            \item \textbf{Definition}: The logistic function maps any real-valued number into the (0, 1) interval.
            \item \textbf{Formula}:
            \begin{equation}
                P(Y=1) = \frac{1}{1 + e^{-z}}
            \end{equation}
            where $z = \beta_0 + \beta_1X_1 + \beta_2X_2 + ... + \beta_nX_n$.
            \item \textbf{Example}: If $z=1$:
            \begin{equation}
                P(Y=1) = \frac{1}{1 + e^{-1}} \approx 0.731
            \end{equation}
            This indicates approximately a 73.1\% chance of being classified as '1'.
        \end{itemize}
    \end{block}
\end{frame}

\begin{frame}[fragile]
    \frametitle{Understanding the Logistic Model - Introduction}
    % Introduction to logistic regression
    \begin{itemize}
        \item Logistic regression is a statistical method used for binary classification.
        \item It estimates the probability that a given input belongs to a particular category.
        \item Essential in fields requiring predictions of binary outcomes, like marketing and medical diagnosis.
    \end{itemize}
\end{frame}

\begin{frame}[fragile]
    \frametitle{Understanding the Logistic Model - The Model}
    % Explanation of the logistic function
    \begin{itemize}
        \item The core of logistic regression is the \textbf{logistic function}:
        \begin{equation}
            P(Y=1|X) = \frac{1}{1 + e^{-(\beta_0 + \beta_1X_1 + \beta_2X_2 + ... + \beta_nX_n)}}
        \end{equation}
        \item \textbf{Where}:
        \begin{itemize}
            \item \(P(Y=1|X)\): Probability of output \(Y=1\) given input \(X\).
            \item \(\beta_0\): Intercept of the model.
            \item \(\beta_1, \beta_2, ..., \beta_n\): Coefficients representing the weight of each feature \(X\).
            \item \(e\): Base of the natural logarithm.
        \end{itemize}
    \end{itemize}
\end{frame}

\begin{frame}[fragile]
    \frametitle{Understanding the Logistic Model - Understanding Odds and Log-Odds}
    % Explanation of odds and log-odds
    \begin{itemize}
        \item \textbf{Odds}: Defined as:
        \begin{equation}
            \text{Odds} = \frac{P(Y=1|X)}{1 - P(Y=1|X)}
        \end{equation}
        \item \textbf{Log-Odds}: The logarithm of the odds:
        \begin{equation}
            \text{Log-Odds} = \log\left(\frac{P(Y=1|X)}{1 - P(Y=1|X)}\right) = \beta_0 + \beta_1X_1 + \beta_2X_2 + ... + \beta_nX_n
        \end{equation}
        \item \textbf{Interpretation}: Coefficients (\(\beta\)) indicate changes in log-odds with one-unit changes in predictor variables.
    \end{itemize}
\end{frame}

\begin{frame}[fragile]
    \frametitle{Slide 1: Overview of Performance Metrics}
    In the realm of supervised learning, evaluating the performance of classification models, such as logistic regression, is crucial. This evaluation enables us to understand how well our model predicts outcomes. Here, we discuss four essential performance metrics:
    
    \begin{enumerate}
        \item \textbf{Accuracy}
        \item \textbf{Precision}
        \item \textbf{Recall}
        \item \textbf{F1-score}
    \end{enumerate}
\end{frame}

\begin{frame}[fragile]
    \frametitle{Slide 2: Key Performance Metrics Explained}
    \begin{block}{1. Accuracy}
        \textbf{Definition}: The proportion of correctly predicted instances (both positive and negative) out of the total instances.
        
        \textbf{Formula}:
        \begin{equation}
            \text{Accuracy} = \frac{TP + TN}{TP + TN + FP + FN}
        \end{equation}
        Where:
        \begin{itemize}
            \item $TP$ = True Positives
            \item $TN$ = True Negatives
            \item $FP$ = False Positives
            \item $FN$ = False Negatives
        \end{itemize}
        \textbf{Example}: In a test of 100 patients, if 80 are correctly diagnosed (60 true positives, 20 true negatives), the accuracy would be:
        \begin{equation}
            \text{Accuracy} = \frac{60 + 20}{100} = 0.80 \text{ or } 80\%
        \end{equation}
    \end{block}
\end{frame}

\begin{frame}[fragile]
    \frametitle{Slide 3: Key Performance Metrics Explained (cont.)}
    \begin{block}{2. Precision}
        \textbf{Definition}: The proportion of true positives to the total predicted positives, indicating the accuracy of positive class predictions.
        
        \textbf{Formula}:
        \begin{equation}
            \text{Precision} = \frac{TP}{TP + FP}
        \end{equation}
        \textbf{Example}: If out of 40 predicted positive cases, 30 are true positives and 10 are false positives, then:
        \begin{equation}
            \text{Precision} = \frac{30}{30 + 10} = 0.75 \text{ or } 75\%
        \end{equation}
    \end{block}

    \begin{block}{3. Recall (Sensitivity)}
        \textbf{Definition}: The proportion of true positives to the actual positives, measuring how well the model identifies positive instances.
        
        \textbf{Formula}:
        \begin{equation}
            \text{Recall} = \frac{TP}{TP + FN}
        \end{equation}
        \textbf{Example}: If there are 50 actual positive cases and 30 are correctly predicted (true positives), the recall is:
        \begin{equation}
            \text{Recall} = \frac{30}{30 + 20} = 0.60 \text{ or } 60\%
        \end{equation}
    \end{block}
\end{frame}

\begin{frame}[fragile]
    \frametitle{Slide 4: Key Performance Metrics Explained (cont.)}
    \begin{block}{4. F1-score}
        \textbf{Definition}: The harmonic mean of precision and recall, providing a balance between the two metrics. Useful when the class distribution is imbalanced.
        
        \textbf{Formula}:
        \begin{equation}
            \text{F1-score} = 2 \times \frac{\text{Precision} \times \text{Recall}}{\text{Precision} + \text{Recall}}
        \end{equation}
        \textbf{Example}: Given previous precision (75\%) and recall (60\%):
        \begin{equation}
            \text{F1-score} = 2 \times \frac{0.75 \times 0.60}{0.75 + 0.60} \approx 0.6667 \text{ or } 66.67\%
        \end{equation}
    \end{block}

    \begin{block}{Key Points to Emphasize}
        - Understanding these metrics helps gauge the effectiveness of logistic regression and informs improvements needed.
        - In applications like ChatGPT and AI systems, metrics guide model tuning and evaluation in critical scenarios.
    \end{block}
\end{frame}

\begin{frame}[fragile]
    \frametitle{Limitations of Logistic Regression - Introduction}
    \begin{block}{Introduction}
        Logistic regression is a widely used statistical method for binary classification problems. However, despite its popularity, it has several limitations that can affect its performance. 
    \end{block}
    \begin{itemize}
        \item Understanding these limitations is crucial for model selection and application.
    \end{itemize}
\end{frame}

\begin{frame}[fragile]
    \frametitle{Limitations of Logistic Regression - Key Points}
    \begin{enumerate}
        \item \textbf{Linearity Assumption}
        \begin{itemize}
            \item Assumes a linear relationship between predictors and log-odds of the outcome.
            \item \textit{Example:} Non-linear links, like age and heart disease, can lead to poor predictions.
            \item \textit{Consequence:} Underfitting when the true relationship is non-linear.
        \end{itemize}
        
        \item \textbf{Sensitivity to Outliers}
        \begin{itemize}
            \item Outliers can disproportionately affect model outputs and coefficients.
            \item \textit{Example:} Extreme values skewing results.
            \item \textit{Consequence:} Biased estimates if outliers are present.
        \end{itemize}
    \end{enumerate}
\end{frame}

\begin{frame}[fragile]
    \frametitle{Limitations of Logistic Regression - More Key Points}
    \begin{enumerate}[resume]
        \item \textbf{Multicollinearity}
        \begin{itemize}
            \item Assumes independent variables are not highly correlated.
            \item \textit{Example:} Height and weight correlation can complicate estimations.
            \item \textit{Consequence:} Confuses interpretation of coefficients.
        \end{itemize}
        
        \item \textbf{Binary Outcomes Only}
        \begin{itemize}
            \item Only suitable for binary outcomes without adaptations.
            \item \textit{Example:} Classifying multiple flower types isn’t feasible without modifications.
            \item \textit{Consequence:} Necessitates complex methods for multi-class scenarios.
        \end{itemize}
    \end{enumerate}
\end{frame}

\begin{frame}[fragile]
    \frametitle{Limitations of Logistic Regression - Final Key Points}
    \begin{enumerate}[resume]
        \item \textbf{Requires Large Sample Sizes}
        \begin{itemize}
            \item Better performance with larger datasets for accurate estimations.
            \item \textit{Example:} Small studies may yield unstable, not generalizable coefficients.
            \item \textit{Consequence:} May lead to overfitting.
        \end{itemize}
        
        \item \textbf{Imbalanced Datasets}
        \begin{itemize}
            \item Struggles with predicting minority classes in imbalanced datasets.
            \item \textit{Example:} Medical scenarios with 90% healthy individuals.
            \item \textit{Consequence:} High accuracy but poor recall and precision for rare classes.
        \end{itemize}
    \end{enumerate}
\end{frame}

\begin{frame}[fragile]
    \frametitle{Limitations of Logistic Regression - Conclusion}
    \begin{block}{Key Takeaway}
        While logistic regression is a powerful tool, it is essential to understand its limitations. Data scientists should consider these limitations when selecting classification techniques and explore alternative methods (e.g., decision trees, support vector machines) for improved performance.
    \end{block}
\end{frame}

\begin{frame}[fragile]
    \frametitle{Formula for Logistic Regression}
    \begin{equation}
        P(Y=1|X) = \frac{1}{1 + e^{-(\beta_0 + \beta_1X_1 + \beta_2X_2 + ... + \beta_nX_n)}}
    \end{equation}
    Where: 
    \begin{itemize}
        \item $P(Y=1|X)$ = Probability of the positive class
        \item $\beta_0$ = Intercept
        \item $\beta_n$ = Coefficients for each predictor variable $X_n$
    \end{itemize}
\end{frame}

\begin{frame}[fragile]
    \frametitle{Introduction to Decision Trees - Part 1}
    \begin{block}{What are Decision Trees?}
        \begin{itemize}
            \item \textbf{Definition}: A decision tree is a supervised learning model used for classification and regression tasks. It works by recursively splitting the dataset into subsets based on the value of input features, ultimately leading to a decision at the leaf nodes.
            \item \textbf{Structure}: Composed of nodes (decision points), branches (outcomes), and leaves (final decisions or classifications).
        \end{itemize}
    \end{block}
    
    \begin{block}{Why Use Decision Trees?}
        \begin{itemize}
            \item \textbf{Interpretability}: Easy to understand and interpret; visually represent decision paths.
            \item \textbf{No Need for Feature Scaling}: Unlike algorithms like SVM, decision trees do not require feature normalization.
            \item \textbf{Handles Non-Linearity}: Can capture complex relationships by creating multiple splitting points.
        \end{itemize}
    \end{block}
\end{frame}

\begin{frame}[fragile]
    \frametitle{Introduction to Decision Trees - Part 2}
    \begin{block}{How Decision Trees Work}
        Decision trees use criteria such as Gini Impurity, Information Gain, or Mean Squared Error to determine where to split the data.
        \begin{itemize}
            \item \textbf{Gini Impurity}: Measures the impurity of a node. Lower values indicate better splits.
            \item \textbf{Information Gain}: Based on the reduction in entropy after a dataset is split.
        \end{itemize}
    \end{block}

    \begin{block}{Example of a Decision Tree}
        \begin{lstlisting}[language=Python]
                     [Outlook]
                     /   |   \
                Sunny   Rain   Overcast
                 /       |       \
              [Humidity]   [Windy] 
               /   \         /     \
           High   Normal   True   False
           /         \      /        \
         No          Yes   Yes        No
        \end{lstlisting}
    \end{block}
\end{frame}

\begin{frame}[fragile]
    \frametitle{Introduction to Decision Trees - Part 3}
    \begin{block}{Key Advantages}
        \begin{itemize}
            \item \textbf{Versatile}: Can handle both categorical and continuous data.
            \item \textbf{No Assumptions}: Unlike linear regression, it does not assume a linear relationship between features and the outcome.
        \end{itemize}
    \end{block}

    \begin{block}{Key Limitations}
        \begin{itemize}
            \item \textbf{Overfitting}: Decision trees can become too complex (very deep) and may overfit the training data.
            \item \textbf{Instability}: Small changes in data can lead to different trees being generated.
        \end{itemize}
    \end{block}

    \begin{block}{Applications of Decision Trees}
        \begin{itemize}
            \item \textbf{Medical Diagnosis}: Classifying patients based on symptoms and diagnostic tests.
            \item \textbf{Credit Scoring}: Assessing the likelihood of default based on financial history.
            \item \textbf{Customer Segmentation}: Classifying customers into groups for targeted marketing.
        \end{itemize}
    \end{block}
\end{frame}

\begin{frame}[fragile]
    \frametitle{Summary and Next Steps}
    \begin{block}{Summary}
        \begin{itemize}
            \item \textbf{Decision Trees} are a powerful classification technique, easy to understand, versatile, but can be prone to overfitting.
            \item They are applicable in various domains, making them an essential tool in data mining and machine learning.
        \end{itemize}
    \end{block}

    \begin{block}{Next Steps}
        In the following slide, we will delve into \textbf{how to construct a decision tree} using specific algorithms like CART.
    \end{block}
\end{frame}

\begin{frame}
    \frametitle{Building a Decision Tree - Overview}
    \begin{block}{Overview of Decision Trees}
        A decision tree is a predictive model that maps observations about an item to conclusions about its target value. 
        It is widely used in machine learning for classification tasks. Understanding how to construct a decision tree is essential for leveraging its capacity to model decision-making processes.
    \end{block}
\end{frame}

\begin{frame}
    \frametitle{Building a Decision Tree - Key Concepts}
    \begin{itemize}
        \item \textbf{Data Preparation:}
        \begin{itemize}
            \item Ensure your dataset is clean and well-structured.
            \item Split the dataset into training and testing sets.
        \end{itemize}
        
        \item \textbf{Choosing a Splitting Algorithm:}
        \begin{itemize}
            \item \textbf{CART (Classification and Regression Trees)} is commonly used for its robustness and simplicity.
            \item Other algorithms: ID3, C4.5.
        \end{itemize}
    \end{itemize}
\end{frame}

\begin{frame}
    \frametitle{Building a Decision Tree - Steps Using CART}
    \begin{enumerate}
        \item \textbf{Select the Best Feature to Split On:}
        \begin{itemize}
            \item Compute impurity measure (e.g., Gini impurity).
            \item \textbf{Gini Impurity Formula:}
            \begin{equation}
                Gini = 1 - \sum (p_i)^2
            \end{equation}
        \end{itemize}
        
        \item \textbf{Split the Data:}
        \begin{itemize}
            \item Divide based on the selected feature, creating child nodes.
        \end{itemize}
        
        \item \textbf{Repeat the Process:}
        \begin{itemize}
            \item Continue splitting until stopping criteria are met.
        \end{itemize}
        
        \item \textbf{Prune the Tree (Optional):}
        \begin{itemize}
            \item To avoid overfitting, remove less important branches.
        \end{itemize}
    \end{enumerate}
\end{frame}

\begin{frame}
    \frametitle{Building a Decision Tree - Example}
    \begin{block}{Example Scenario}
        Given a dataset with features like Age, Income, and Marital Status to predict customer purchase behavior (Yes/No):
    \end{block}
    \begin{enumerate}
        \item Compute Gini impurity at root.
        \item Identify 'Income' as the best feature and split into 'High Income' and 'Low Income' branches.
        \item Repeat for each branch until a condition is met.
    \end{enumerate}
\end{frame}

\begin{frame}[fragile]
    \frametitle{Building a Decision Tree - Code Snippet}
    \begin{block}{Python using Scikit-learn}
    \begin{lstlisting}[language=Python]
from sklearn.tree import DecisionTreeClassifier

# Initialize the classifier
clf = DecisionTreeClassifier(criterion='gini', max_depth=3)

# Fit the model
clf.fit(X_train, y_train)

# Make predictions
predictions = clf.predict(X_test)
    \end{lstlisting}
    \end{block}
\end{frame}

\begin{frame}
    \frametitle{Building a Decision Tree - Key Points}
    \begin{itemize}
        \item Decision Trees provide a clear and interpretable model.
        \item The choice of splitting algorithms significantly affects performance.
        \item Pruning is essential to prevent overfitting and enhance generalization.
    \end{itemize}
\end{frame}

\begin{frame}[fragile]
    \frametitle{Evaluating Decision Trees - Introduction}
    \begin{itemize}
        \item Decision trees: popular machine learning technique for classification.
        \item Intuitive and easy to visualize.
        \item Importance of evaluation: understand effectiveness and identify overfitting.
    \end{itemize}
\end{frame}

\begin{frame}[fragile]
    \frametitle{Evaluating Decision Trees - Understanding Overfitting}
    \begin{itemize}
        \item \textbf{Definition}: Overfitting happens when a model captures noise instead of the actual signal, leading to:
        \begin{itemize}
            \item High accuracy on training data.
            \item Low accuracy on validation/test datasets.
        \end{itemize}
        \item \textbf{Example}: A decision tree trained on noisy data captures fluctuations, resulting in a complex model.
    \end{itemize}
\end{frame}

\begin{frame}[fragile]
    \frametitle{Evaluating Decision Trees - Evaluation Metrics}
    \begin{enumerate}
        \item \textbf{Accuracy}  
        \begin{itemize}
            \item Definition: Ratio of correct predictions to total instances.
            \item Formula:  
                \[
                \text{Accuracy} = \frac{\text{Number of Correct Predictions}}{\text{Total Predictions}}
                \]
        \end{itemize}

        \item \textbf{Precision}  
        \begin{itemize}
            \item Definition: Ratio of true positives to sum of true positives and false positives.
            \item Formula:  
                \[
                \text{Precision} = \frac{TP}{TP + FP}
                \]
        \end{itemize}

        \item \textbf{Recall (Sensitivity)}  
        \begin{itemize}
            \item Definition: Ratio of true positives to sum of true positives and false negatives.
            \item Formula:  
                \[
                \text{Recall} = \frac{TP}{TP + FN}
                \]
        \end{itemize}

        \item \textbf{F1 Score}  
        \begin{itemize}
            \item Definition: Harmonic mean of precision and recall.
            \item Formula:  
                \[
                F1 = 2 \times \frac{\text{Precision} \times \text{Recall}}{\text{Precision} + \text{Recall}}
                \]
        \end{itemize}

        \item \textbf{Confusion Matrix}
        \begin{itemize}
            \item Structure used to evaluate performance:
            \[
            \begin{array}{|c|c|c|}
            \hline
            & \text{Predicted Positive} & \text{Predicted Negative} \\
            \hline
            \text{Actual Positive} & TP & FN \\
            \hline
            \text{Actual Negative} & FP & TN \\
            \hline
            \end{array}
            \]
        \end{itemize}
    \end{enumerate}
\end{frame}

\begin{frame}[fragile]
    \frametitle{Evaluating Decision Trees - Strategies to Mitigate Overfitting}
    \begin{itemize}
        \item \textbf{Pruning}: Simplifying the tree by removing insignificant splits.
        \item \textbf{Cross-Validation}: Implement k-fold cross-validation to validate generalization.
        \item \textbf{Setting Depth Limit}: Control the tree depth to avoid overfitting.
        \item \textbf{Minimum Samples Split}: Require a minimum number of samples in a node before splitting.
    \end{itemize}
\end{frame}

\begin{frame}[fragile]
    \frametitle{Evaluating Decision Trees - Key Points & Conclusion}
    \begin{itemize}
        \item Evaluation metrics quantify decision tree performance and detect overfitting.
        \item Balancing bias and variance is essential for effective models.
        \item Always validate with a separate test set to ensure robustness.
    \end{itemize}
    \begin{block}{Conclusion}
        Evaluating decision trees with appropriate metrics and strategies is crucial for reliable machine learning applications.
    \end{block}
\end{frame}

\begin{frame}[fragile]
    \frametitle{Random Forests Explained - Introduction}
    Random Forests is an advanced ensemble learning technique primarily used for classification and regression tasks. 
    \begin{itemize}
        \item Constructs multiple decision trees during training.
        \item Outputs mode of predictions for classification, mean for regression.
        \item Improves accuracy and controls overfitting.
    \end{itemize}
    \begin{block}{Why Do We Need Random Forests?}
        Decision trees can be simple and interpretable but are prone to overfitting and high variance due to sensitivity to data fluctuations.
    \end{block}
\end{frame}

\begin{frame}[fragile]
    \frametitle{Random Forests Explained - Advantages}
    \begin{enumerate}
        \item \textbf{Improved Accuracy:} Reduces overfitting, leads to greater accuracy.
        \item \textbf{Robust to Noise:} Less impacted by noise through training on random subsets.
        \item \textbf{Feature Importance:} Helps understand variable contributions.
        \item \textbf{Handles High Dimensionality:} Effective with many features.
        \item \textbf{Less Hyperparameter Tuning:} Simpler parameter configuration.
    \end{enumerate}
\end{frame}

\begin{frame}[fragile]
    \frametitle{Random Forests Explained - Example Scenario}
    Imagine using a healthcare dataset to predict a patient's disease based on features such as age and cholesterol levels.
    \begin{itemize}
        \item A single decision tree may overfit the training data and perform poorly on new data.
        \item A Random Forest creates multiple trees from different data subsets, leading to greater predictive accuracy and generalization.
    \end{itemize}
    \begin{block}{Key Points to Emphasize}
        \begin{itemize}
            \item Builds multiple decision trees to enhance performance.
            \item Reduces overfitting compared to single decision trees.
            \item Provides insights into impactful features for predictions.
        \end{itemize}
    \end{block}
\end{frame}

\begin{frame}[fragile]
    \frametitle{Random Forests Explained - Summary and Conclusion}
    \begin{itemize}
        \item Random Forests effectively aggregate multiple decision trees.
        \item Address the limitations of single decision trees while generating higher predictive power.
        \item Set the stage for exploring ensemble methods and bagging principles in future slides.
    \end{itemize}
\end{frame}

\begin{frame}[fragile]
    \frametitle{How Random Forests Work - Introduction}
    \begin{block}{Ensemble Methods}
        Ensemble methods combine predictions from multiple models to improve accuracy and robustness.
    \end{block}
    \begin{itemize}
        \item **Single Decision Trees**: Prone to overfitting - perform well on training data but poorly on unseen data.
        \item **Random Forests**: Reduce overfitting by leveraging multiple decision trees, enhancing performance.
    \end{itemize}
\end{frame}

\begin{frame}[fragile]
    \frametitle{How Random Forests Work - Bagging Principle}
    \begin{block}{Bagging (Bootstrap Aggregating)}
        An ensemble technique that creates multiple subsets of the training data by sampling with replacement.
    \end{block}
    \begin{enumerate}
        \item \textbf{Data Sampling}: Create subsets from original dataset through random sampling with replacement.
        \item \textbf{Model Training}: Train a decision tree on each subset.
        \item \textbf{Prediction Aggregation}: 
        \begin{itemize}
            \item Classification: Majority voting.
            \item Regression: Average of predictions.
        \end{itemize}
    \end{enumerate}
\end{frame}

\begin{frame}[fragile]
    \frametitle{How Random Forests Work - Functionality and Example}
    \begin{block}{Randomness in Features}
        A random subset of features is chosen for splitting at each node during tree construction.
    \end{block}
    \begin{itemize}
        \item **Steps of Random Forests**:
        \begin{enumerate}
            \item Sampling: Create numerous bootstrapped datasets.
            \item Tree Construction: Build decision trees using random features.
            \item Aggregation: Combine predictions for final output.
        \end{enumerate}
        \item **Example**: Classifying Iris Species
        \begin{itemize}
            \item Each tree votes for the species, and the species with the most votes is chosen.
        \end{itemize}
    \end{itemize}
\end{frame}

\begin{frame}[fragile]
    \frametitle{How Random Forests Work - Advantages and Summary}
    \begin{block}{Advantages of Random Forests}
        \begin{itemize}
            \item Improved accuracy through averaging predictions.
            \item Robustness to overfitting by aggregating models.
            \item Insight into feature importance for predictions.
        \end{itemize}
    \end{block}
    \begin{block}{Summary of Key Points}
        \begin{itemize}
            \item Ensemble methods enhance model performance and reduce overfitting.
            \item Bagging utilizes random sampling to create diverse models.
            \item Random Forests aggregate predictions from multiple decision trees for robust classification.
        \end{itemize}
    \end{block}
    \begin{block}{Closing Thought}
        Random Forests are powerful in data mining, applicable in fields like medicine, finance, and AI systems.
    \end{block}
\end{frame}

\begin{frame}[fragile]
    \frametitle{Evaluating Random Forests - Introduction}
    \begin{block}{Introduction to Performance Evaluation}
        Evaluating the performance of a Random Forest model is critical to understanding its effectiveness in classification tasks.
        \begin{itemize}
            \item Ensemble learning technique
            \item Performance metrics capture robustness and generalization
            \item Tailored metrics for unique strengths
        \end{itemize}
    \end{block}
\end{frame}

\begin{frame}[fragile]
    \frametitle{Evaluating Random Forests - Key Performance Metrics}
    \begin{enumerate}
        \item \textbf{Accuracy}
            \begin{itemize}
                \item Definition: Ratio of correctly predicted instances to total instances
                \item Formula: 
                \begin{equation}
                \text{Accuracy} = \frac{\text{TP} + \text{TN}}{\text{TP} + \text{TN} + \text{FP} + \text{FN}}
                \end{equation}
                \item Example: 85 correct predictions out of 100 yields 85\%.
            \end{itemize}

        \item \textbf{Precision}
            \begin{itemize}
                \item Definition: Ratio of true positives to predicted positives
                \item Formula: 
                \begin{equation}
                \text{Precision} = \frac{\text{TP}}{\text{TP} + \text{FP}}
                \end{equation}
                \item Example: If 30 predicted positives but only 20 are actual, precision = 67\%.
            \end{itemize}

        \item \textbf{Recall (Sensitivity)}
            \begin{itemize}
                \item Definition: Ratio of true positives to actual positives
                \item Formula: 
                \begin{equation}
                \text{Recall} = \frac{\text{TP}}{\text{TP} + \text{FN}}
                \end{equation}
                \item Example: 40 correct out of 50 actual gives recall = 80\%.
            \end{itemize}
    \end{enumerate}
\end{frame}

\begin{frame}[fragile]
    \frametitle{Evaluating Random Forests - More Metrics}
    \begin{enumerate}
        \setcounter{enumi}{3} % Continue numbering from the previous frame
        \item \textbf{F1 Score}
            \begin{itemize}
                \item Definition: Harmonic mean of precision and recall
                \item Formula: 
                \begin{equation}
                F1 = 2 \times \frac{\text{Precision} \times \text{Recall}}{\text{Precision} + \text{Recall}}
                \end{equation}
                \item Example: For precision = 0.67 and recall = 0.80, F1 = 0.73.
            \end{itemize}

        \item \textbf{ROC AUC}
            \begin{itemize}
                \item Definition: Graphical representation of classifier’s performance
                \item Interpretation: AUC = 0.5 (no discrimination) to AUC = 1.0 (perfect discrimination)
            \end{itemize}

        \item \textbf{Cross-Validation}
            \begin{itemize}
                \item Definition: Technique to assess the generalization of the model
                \item Benefits: Mitigates overfitting and improves performance estimation
            \end{itemize}
    \end{enumerate}
\end{frame}

\begin{frame}[fragile]
    \frametitle{Evaluating Random Forests - Key Takeaways}
    \begin{block}{Summary}
        \begin{itemize}
            \item Random Forests improve accuracy and robustness through multiple trees
            \item Evaluate models using multiple metrics for comprehensive insights
            \item Understanding these metrics ensures reliability in classification tasks
        \end{itemize}
    \end{block}
\end{frame}

\begin{frame}
    \titlepage
\end{frame}

\begin{frame}
    \frametitle{Introduction to Classification Techniques}
    \begin{block}{Overview}
        In supervised learning, classification techniques help us predict categorical outcomes based on input features. We will cover practical implementations of three popular classification techniques:
    \end{block}
    \begin{itemize}
        \item Logistic Regression
        \item Decision Trees
        \item Random Forests
    \end{itemize}
\end{frame}

\begin{frame}[fragile]
    \frametitle{1. Logistic Regression}
    \begin{block}{Explanation}
        Logistic Regression is a statistical method for predicting binary classes. The outcome is modeled using a logistic function to predict the probability of a certain class.
    \end{block}
    \begin{block}{Key Formula}
        \begin{equation}
            P(Y=1|X) = \frac{1}{1 + e^{-(\beta_0 + \beta_1X_1 + \beta_2X_2 + ... + \beta_nX_n)}}
        \end{equation}
        where \( P \) is the probability of the event occurring, and \( \beta \) represents model parameters.
    \end{block}

    \begin{block}{Python Code Example}
        \begin{lstlisting}[language=Python]
from sklearn.model_selection import train_test_split
from sklearn.linear_model import LogisticRegression
from sklearn.datasets import load_iris

# Load dataset
data = load_iris()
X, y = data.data, data.target

# Splitting the dataset
X_train, X_test, y_train, y_test = train_test_split(X, y, test_size=0.2, random_state=42)

# Create logistic regression model
model = LogisticRegression(max_iter=200)
model.fit(X_train, y_train)

# Making predictions
predictions = model.predict(X_test)
        \end{lstlisting}
    \end{block}
    \begin{block}{Key Points}
        \begin{itemize}
            \item Best for binary classification problems.
            \item Assumes a linear relationship between features and class probability.
        \end{itemize}
    \end{block}
\end{frame}

\begin{frame}[fragile]
    \frametitle{2. Decision Trees}
    \begin{block}{Explanation}
        Decision Trees partition the data into subsets based on feature values, with nodes testing features and leaves representing the predicted class.
    \end{block}

    \begin{block}{Python Code Example}
        \begin{lstlisting}[language=Python]
from sklearn.tree import DecisionTreeClassifier

# Create a decision tree model
dtree = DecisionTreeClassifier()
dtree.fit(X_train, y_train)

# Making predictions
dtree_predictions = dtree.predict(X_test)
        \end{lstlisting}
    \end{block}
    \begin{block}{Key Points}
        \begin{itemize}
            \item Intuitive and easy to interpret.
            \item Prone to overfitting; can be mitigated by pruning or setting a maximum depth.
        \end{itemize}
    \end{block}
\end{frame}

\begin{frame}[fragile]
    \frametitle{3. Random Forests}
    \begin{block}{Explanation}
        Random Forests are an ensemble method that builds multiple decision trees for improved accuracy and robustness by aggregating predictions from various trees.
    \end{block}

    \begin{block}{Python Code Example}
        \begin{lstlisting}[language=Python]
from sklearn.ensemble import RandomForestClassifier

# Create random forest model
rf = RandomForestClassifier(n_estimators=100)
rf.fit(X_train, y_train)

# Making predictions
rf_predictions = rf.predict(X_test)
        \end{lstlisting}
    \end{block}
    \begin{block}{Key Points}
        \begin{itemize}
            \item Reduces the risk of overfitting compared to a single decision tree.
            \item Handles both numerical and categorical data well.
        \end{itemize}
    \end{block}
\end{frame}

\begin{frame}
    \frametitle{Conclusion and Takeaways}
    \begin{block}{Conclusion}
        Understanding and implementing these classification techniques form the backbone of many predictive models in machine learning. Through Python, we can apply these techniques, enabling data-driven decision-making across various applications, including AI advancements like ChatGPT.
    \end{block}
    \begin{block}{Takeaway Outline}
        \begin{itemize}
            \item \textbf{Logistic Regression:} Binary classification technique; provides probabilities of outcomes.
            \item \textbf{Decision Trees:} Simple model structure; prone to overfitting.
            \item \textbf{Random Forests:} Ensemble method; improves accuracy and reduces overfitting.
        \end{itemize}
    \end{block}
\end{frame}

\begin{frame}[fragile]
    \frametitle{Slide 1: Introduction to Classification Techniques}
    \begin{block}{Overview}
        Classification is a fundamental task in machine learning aimed at predicting categorical labels based on input features. This presentation compares three popular techniques: 
        \begin{itemize}
            \item Logistic Regression
            \item Decision Trees
            \item Random Forests
        \end{itemize}
        Understanding their applications, strengths, and weaknesses is crucial for selecting the appropriate algorithm for a given problem.
    \end{block}
\end{frame}

\begin{frame}[fragile]
    \frametitle{Slide 2: Logistic Regression}
    \begin{block}{Concept}
        A statistical model predicting the probability of a binary outcome (0 or 1) based on one or more predictor variables.
    \end{block}
    \begin{block}{Applications}
        \begin{itemize}
            \item Credit scoring
            \item Healthcare (diagnosing diseases)
            \item Marketing (predicting customer churn)
        \end{itemize}
    \end{block}
    \begin{block}{Strengths}
        \begin{itemize}
            \item Simple and interpretable
            \item Produces useful probabilities for decision-making
            \item Handles both binary and multinomial outcomes
        \end{itemize}
    \end{block}
    \begin{block}{Weaknesses}
        \begin{itemize}
            \item Assumes a linear relationship between features and log odds
            \item Poor performance on non-linear data
        \end{itemize}
    \end{block}
\end{frame}

\begin{frame}[fragile]
    \frametitle{Slide 3: Decision Trees}
    \begin{block}{Concept}
        A non-linear model that splits data into branches based on feature values, leading to a decision (label).
    \end{block}
    \begin{block}{Applications}
        \begin{itemize}
            \item Finance (risk assessment)
            \item Healthcare (treatment decisions)
            \item Customer behavior analysis
        \end{itemize}
    \end{block}
    \begin{block}{Strengths}
        \begin{itemize}
            \item Intuitive and easy to visualize
            \item Can handle both numerical and categorical data
            \item No need for scaling or normalization
        \end{itemize}
    \end{block}
    \begin{block}{Weaknesses}
        \begin{itemize}
            \item Prone to overfitting with too many branches
            \item Sensitive to small changes in data
        \end{itemize}
    \end{block}
\end{frame}

\begin{frame}[fragile]
    \frametitle{Slide 4: Random Forests}
    \begin{block}{Concept}
        An ensemble method that builds multiple decision trees and merges their outputs to improve accuracy and control overfitting.
    \end{block}
    \begin{block}{Applications}
        \begin{itemize}
            \item Ecology (species classification)
            \item Finance (fraud detection)
            \item Image classification
        \end{itemize}
    \end{block}
    \begin{block}{Strengths}
        \begin{itemize}
            \item High accuracy by averaging multiple trees
            \item Robust to overfitting
            \item Handles missing values well
        \end{itemize}
    \end{block}
    \begin{block}{Weaknesses}
        \begin{itemize}
            \item Less interpretable than a single decision tree
            \item Requires more computational resources
        \end{itemize}
    \end{block}
\end{frame}

\begin{frame}[fragile]
    \frametitle{Slide 5: Key Comparisons}
    \begin{table}[ht]
        \centering
        \begin{tabular}{|l|l|l|l|}
            \hline
            \textbf{Technique} & \textbf{Strengths} & \textbf{Weaknesses} & \textbf{Typical Applications} \\ \hline
            Logistic Regression & Simple, interpretable, outputs probabilities & Assumes linearity & Credit scoring, healthcare, marketing \\ \hline
            Decision Trees & Intuitive, handles various data types & Prone to overfitting & Finance, healthcare, behavior analysis \\ \hline
            Random Forests & High accuracy, robust to overfitting & Less interpretable, resource-intensive & Ecology, finance, image classification \\ \hline
        \end{tabular}
        \caption{Comparison of Classification Techniques}
    \end{table}
\end{frame}

\begin{frame}[fragile]
    \frametitle{Slide 6: Conclusion and Key Takeaways}
    \begin{block}{Conclusion}
        Choosing the appropriate classification technique is vital for effective data analysis. Understanding the strengths and weaknesses helps in informed decision-making.
    \end{block}
    \begin{block}{Key Takeaways}
        \begin{itemize}
            \item Each technique has distinct characteristics suited for specific scenarios.
            \item Consider data characteristics (linearity, size, types) when selecting.
            \item Ensemble methods generally provide better performance but risk interpretability.
        \end{itemize}
    \end{block}
\end{frame}

\begin{frame}[fragile]
    \frametitle{Slide 7: Illustrative Example}
    \begin{block}{Example Use Case}
        For a dataset predicting customer churn:
        \begin{itemize}
            \item Logistic Regression quantifies churn probability.
            \item Decision Trees outline the decision path.
            \item Random Forests enhance predictive power by combining multiple decisions.
        \end{itemize}
    \end{block}
\end{frame}

\begin{frame}[fragile]
    \frametitle{Slide 8: References}
    \begin{itemize}
        \item [1] "Introduction to Statistical Learning" by Gareth James et al.
        \item [2] "Hands-On Machine Learning with Scikit-Learn, Keras, and TensorFlow" by Aurélien Géron
    \end{itemize}
\end{frame}

\begin{frame}[fragile]
    \frametitle{Slide 9: Additional Notes}
    This section serves to help students grasp the essentials of classification techniques and their comparative strengths and weaknesses in supervised learning contexts.
\end{frame}

\begin{frame}[fragile]
    \frametitle{Conclusion and Future Directions - Overview}
    \begin{itemize}
        \item Wrap-up of supervised learning, especially classification techniques.
        \item Significant impact across various domains.
        \item Summary of key themes and future trends.
    \end{itemize}
\end{frame}

\begin{frame}[fragile]
    \frametitle{Key Concepts Recap}
    \begin{itemize}
        \item \textbf{Supervised Learning}: Algorithms that learn from labeled data to make predictions on unseen data.
        \item \textbf{Classification Techniques}:
            \begin{itemize}
                \item \textbf{Logistic Regression}: Foundation for binary classification; useful for probabilistic outputs.
                \item \textbf{Decision Trees}: Visualize decisions and outcomes; interpretable but can overfit.
                \item \textbf{Random Forests}: Ensemble method for robustness and improved accuracy through averaging.
            \end{itemize}
    \end{itemize}
\end{frame}

\begin{frame}[fragile]
    \frametitle{Future Directions in Classification}
    \begin{enumerate}
        \item \textbf{Integration of Advanced Techniques}:
            \begin{itemize}
                \item \textbf{Deep Learning}: Neural networks for complex datasets (e.g., images, text).
                \item \textbf{Transfer Learning}: Models that leverage knowledge from previous tasks for better efficiency.
            \end{itemize}
        \item \textbf{Explainable AI (XAI)}:
            \begin{itemize}
                \item Importance of transparency in machine learning models.
                \item Techniques like SHAP and LIME for model interpretability.
            \end{itemize}
        \item \textbf{Ethics and Fairness}:
            \begin{itemize}
                \item Addressing bias in sensitive areas (e.g., hiring, lending).
                \item Frameworks for ensuring fairness and accountability.
            \end{itemize}
        \item \textbf{Real-time Data Processing}:
            \begin{itemize}
                \item Need for immediate insights; real-time classification for dynamic adaptation.
            \end{itemize}
    \end{enumerate}
\end{frame}

\begin{frame}[fragile]
    \frametitle{Applications and Key Takeaways}
    \begin{itemize}
        \item Example: \textbf{ChatGPT} benefits from advanced classification techniques.
        \item Continuous evolution in the field of classification techniques.
        \item Key takeaways:
            \begin{itemize}
                \item Critical role of classification across industries.
                \item Future advancements: integrating deep learning, enhancing interpretability, and ensuring fairness.
                \item Importance of ongoing research for ethical and effective models.
            \end{itemize}
    \end{itemize}
\end{frame}


\end{document}