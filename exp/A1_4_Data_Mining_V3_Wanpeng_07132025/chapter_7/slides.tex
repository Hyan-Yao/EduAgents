\documentclass[aspectratio=169]{beamer}

% Theme and Color Setup
\usetheme{Madrid}
\usecolortheme{whale}
\useinnertheme{rectangles}
\useoutertheme{miniframes}

% Additional Packages
\usepackage[utf8]{inputenc}
\usepackage[T1]{fontenc}
\usepackage{graphicx}
\usepackage{booktabs}
\usepackage{listings}
\usepackage{amsmath}
\usepackage{amssymb}
\usepackage{xcolor}
\usepackage{tikz}
\usepackage{pgfplots}
\pgfplotsset{compat=1.18}
\usetikzlibrary{positioning}
\usepackage{hyperref}

% Custom Colors
\definecolor{myblue}{RGB}{31, 73, 125}
\definecolor{mygray}{RGB}{100, 100, 100}
\definecolor{mygreen}{RGB}{0, 128, 0}
\definecolor{myorange}{RGB}{230, 126, 34}
\definecolor{mycodebackground}{RGB}{245, 245, 245}

% Set Theme Colors
\setbeamercolor{structure}{fg=myblue}
\setbeamercolor{frametitle}{fg=white, bg=myblue}
\setbeamercolor{title}{fg=myblue}
\setbeamercolor{section in toc}{fg=myblue}
\setbeamercolor{item projected}{fg=white, bg=myblue}
\setbeamercolor{block title}{bg=myblue!20, fg=myblue}
\setbeamercolor{block body}{bg=myblue!10}
\setbeamercolor{alerted text}{fg=myorange}

% Set Fonts
\setbeamerfont{title}{size=\Large, series=\bfseries}
\setbeamerfont{frametitle}{size=\large, series=\bfseries}
\setbeamerfont{caption}{size=\small}
\setbeamerfont{footnote}{size=\tiny}

% Document Start
\begin{document}

\frame{\titlepage}

\begin{frame}[fragile]
  \titlepage
\end{frame}

\begin{frame}[fragile]
  \frametitle{Motivation for Data Mining}
  
  \begin{itemize}
    \item In today's data-driven world, vast amounts of data are generated every second.
    \item Data mining enables the extraction of valuable insights from this data.
    \item Many real-world applications, such as customer recommendation systems and fraud detection, rely on effective data mining.
  \end{itemize}
  
\end{frame}

\begin{frame}[fragile]
  \frametitle{Overview of Unsupervised Learning}
  
  Unsupervised learning is a type of machine learning where the model is trained on unlabeled data. Unlike supervised learning, which relies on input-output pairs to learn a mapping, unsupervised learning seeks to identify patterns or structures within data without explicit instructions.

  \begin{itemize}
    \item **Unlabeled Data**: Data that has no predefined categories or labels.
    \item **Pattern Recognition**: Detecting underlying structures or clusters based on similarities and differences.
    \item **Dimensionality Reduction**: Reducing the number of variables while retaining essential information.
  \end{itemize}
  
\end{frame}

\begin{frame}[fragile]
  \frametitle{Significance and Applications}
  
  \begin{block}{Significance in Data Mining}
    Unsupervised learning plays a crucial role in data mining by enabling the discovery of insights that would otherwise remain hidden.
  \end{block}
  
  \begin{itemize}
    \item **Market Basket Analysis**: Identifying products frequently purchased together.
    \item **Customer Segmentation**: Grouping customers by purchasing behavior.
    \item **Anomaly Detection**: Detecting unusual patterns that may indicate fraud or failures.
  \end{itemize}

\end{frame}

\begin{frame}[fragile]
  \frametitle{Examples of Unsupervised Learning}
  
  \begin{enumerate}
    \item **Clustering**
      \begin{itemize}
        \item Algorithm: K-Means Clustering
        \item Use Case: Grouping customers into segments for targeted advertising based on behavior.
      \end{itemize}
      \begin{itemize}
        \item Steps:
          \begin{itemize}
            \item Choose the number of clusters, $k$.
            \item Randomly select $k$ centroids.
            \item Assign each data point to the nearest centroid.
            \item Update centroid positions based on group means.
            \item Iterate until centroids stabilize.
          \end{itemize}
      \end{itemize}

    \item **Association Rule Learning**
      \begin{itemize}
        \item Algorithm: Apriori Algorithm
        \item Use Case: Discovering items frequently bought together.
      \end{itemize}
  \end{enumerate}
  
\end{frame}

\begin{frame}[fragile]
  \frametitle{Key Benefits and Concluding Thoughts}
  
  \begin{itemize}
    \item **Error Reduction**: Unsupervised models can enhance subsequent supervised learning tasks by reducing noise.
    \item **Versatility**: Applicable across various fields like finance and healthcare.
  \end{itemize}
  
  Understanding unsupervised learning is paramount for leveraging vast datasets to drive innovation and efficiency.

  \begin{block}{Outline for Discussion}
    \begin{itemize}
      \item Definition and basics of unsupervised learning
      \item Importance in data mining and real-world significance
      \item Examples of algorithms and their applications
      \item Key benefits and concluding remarks
    \end{itemize}
  \end{block}

\end{frame}

\begin{frame}[fragile]
    \frametitle{Motivation Behind Unsupervised Learning}
    \begin{block}{Introduction}
        Unsupervised learning is a critical component of data mining and machine learning. This approach helps to identify patterns within data without prior labeling, uncovering hidden structures in complex datasets.
    \end{block}
    \begin{block}{Key Points}
        \begin{itemize}
            \item Leverages abundant unlabeled data to extract meaningful patterns.
            \item Applications span various industries, including finance, healthcare, and marketing.
            \item Complements supervised learning, providing a holistic approach to data analysis.
        \end{itemize}
    \end{block}
\end{frame}

\begin{frame}[fragile]
    \frametitle{Why Do We Need Unsupervised Learning?}

    \begin{enumerate}
        \item \textbf{Exploration of Unlabeled Data}
            \begin{itemize}
                \item Insight derivation from unlabeled data (e.g., customer behaviors).
                \item \textit{Example:} Retailers analyze purchase patterns to identify segments without predefined categories.
            \end{itemize}
        
        \item \textbf{Pattern Recognition}
            \begin{itemize}
                \item Detects natural groupings in datasets.
                \item \textit{Example:} Google Photos categorizes images into clusters for easy navigation.
            \end{itemize}
    \end{enumerate}
\end{frame}

\begin{frame}[fragile]
    \frametitle{Continuing Need for Unsupervised Learning}

    \begin{enumerate}[resume]
        \item \textbf{Dimensionality Reduction}
            \begin{itemize}
                \item Simplifies high-dimensional data while preserving essential information.
                \item \textit{Example:} PCA in genomics reduces complexity, aiding in visualization of gene relationships.
            \end{itemize}

        \item \textbf{Anomaly Detection}
            \begin{itemize}
                \item Identifies outliers or unusual data points that deviate from the expected model.
                \item \textit{Example:} Financial institutions use unsupervised learning to detect fraudulent transactions by analyzing spending patterns.
            \end{itemize}
    \end{enumerate}
\end{frame}

\begin{frame}[fragile]
    \frametitle{Conclusion and Further Learning}
    
    \begin{block}{Conclusion}
        Unsupervised learning is vital for understanding complex datasets that traditional methods may overlook. Its evolving capabilities are increasingly important for maximizing insights from big data.
    \end{block}

    \begin{block}{Outline for Further Learning}
        \begin{itemize}
            \item Defining Unsupervised Learning
            \item Understanding Clustering Techniques
            \item Applications in AI (e.g., ChatGPT and data mining)
        \end{itemize}
    \end{block}

    \begin{block}{References}
        \begin{itemize}
            \item Data Mining: Concepts and Techniques by Jiawei Han \& Micheline Kamber
            \item Principal Component Analysis (PCA) Review and Applications
        \end{itemize}
    \end{block}
\end{frame}

\begin{frame}[fragile]
    \frametitle{Clustering Techniques Overview}
    % Overview of clustering techniques and their significance
    Clustering is an unsupervised learning technique used to group objects based on similarity.
    \begin{itemize}
        \item **Pattern Discovery**: Identify distinct groups within data to reveal hidden structures.
        \item **Data Summarization**: Simplify data analysis by aggregating large datasets.
        \item **Dimensionality Reduction**: Replace data points with cluster centroids.
        \item **Preprocessing**: Serve as a preliminary step for further analyses.
    \end{itemize}
\end{frame}

\begin{frame}[fragile]
    \frametitle{Clustering in Unsupervised Learning}
    % Explaining the context of clustering in unsupervised learning
    Clustering is a fundamental method in:
    \begin{itemize}
        \item **Unsupervised Learning**: Analyzing unlabeled datasets to find groupings.
        \item Applications include:
            \begin{itemize}
                \item Market segmentation
                \item Social network analysis
                \item Image processing
                \item Genomics
            \end{itemize}
    \end{itemize}
\end{frame}

\begin{frame}[fragile]
    \frametitle{Examples of Clustering Applications}
    % Discussing practical applications and examples
    \begin{enumerate}
        \item **Market Segmentation**:
            \begin{itemize}
                \item Segment customers based on purchasing behavior.
                \item Example: Frequent buyers, occasional buyers, one-time shoppers.
            \end{itemize}
        \item **Image Analysis**:
            \begin{itemize}
                \item Grouping pixels in images by color/texture.
                \item Illustration: Landscape image segmented into sky, land, and water.
            \end{itemize}
        \item **Genomics**:
            \begin{itemize}
                \item Identify gene expression patterns for disease classification.
            \end{itemize}
    \end{enumerate}
\end{frame}

\begin{frame}[fragile]
    \frametitle{Key Points and Conclusion}
    % Summary of key points and closing remarks
    \begin{block}{Key Points}
        \begin{itemize}
            \item Clustering reveals structure in unlabeled data.
            \item Applicable to both numerical and categorical data.
            \item Enhances performance in AI applications (e.g., recommendation systems).
        \end{itemize}
    \end{block}
    \vspace{0.5cm}
    Clustering is crucial in unsupervised learning for understanding complex datasets and providing insights. 
\end{frame}

\begin{frame}[fragile]
    \frametitle{Types of Clustering Methods}
    \begin{block}{Introduction to Clustering Methods}
        Clustering is a crucial technique in unsupervised learning, allowing us to group similar data points based on feature similarity. Different clustering methods have varying strengths and weaknesses, catering to different types of datasets and analytical needs.
    \end{block}
\end{frame}

\begin{frame}[fragile]
    \frametitle{1. K-means Clustering}
    \begin{itemize}
        \item \textbf{Description}: K-means is a partitioning method that divides a dataset into K distinct clusters by minimizing the variance within each cluster.
        \item \textbf{How it Works}:
        \begin{enumerate}
            \item Choose the number of clusters (K).
            \item Initialize K centroids randomly.
            \item Assign each data point to the nearest centroid.
            \item Recalculate the centroids based on the assigned data points.
            \item Repeat steps 3-4 until convergence.
        \end{enumerate}
        \item \textbf{Use Cases}:
        \begin{itemize}
            \item Customer segmentation in marketing.
            \item Image compression.
            \item Document clustering.
        \end{itemize}
        \item \textbf{Key Point}: Sensitive to the initial placement of centroids and the value of K. Requires scaling of data for optimal performance.
    \end{itemize}
\end{frame}

\begin{frame}[fragile]
    \frametitle{2. Hierarchical Clustering}
    \begin{itemize}
        \item \textbf{Description}: Builds a tree of clusters. It can be either agglomerative (bottom-up) or divisive (top-down).
        \item \textbf{How it Works} (Agglomerative Method):
        \begin{enumerate}
            \item Treat each data point as a single cluster.
            \item Merge the closest pair of clusters.
            \item Repeat until one cluster is formed or a desired number of clusters is reached.
        \end{enumerate}
        \item \textbf{Use Cases}:
        \begin{itemize}
            \item Taxonomy development in biology.
            \item Social network analysis.
            \item Gene expression data analysis.
        \end{itemize}
        \item \textbf{Key Point}: Produces a dendrogram. Good for small datasets but can be computationally expensive for larger sets.
    \end{itemize}
\end{frame}

\begin{frame}[fragile]
    \frametitle{3. DBSCAN}
    \begin{itemize}
        \item \textbf{Description}: Identifies clusters based on the density of data points in a region, effectively handling noise and outliers.
        \item \textbf{How it Works}:
        \begin{enumerate}
            \item Specify parameters: Epsilon ($\epsilon$) as the radius and MinPts as the minimum number of points required to form a dense region.
            \item For each point, check the density of its neighborhood.
            \item Form clusters from high-density regions, marking points outside as noise.
        \end{enumerate}
        \item \textbf{Use Cases}:
        \begin{itemize}
            \item Geospatial clustering (e.g., location data).
            \item Anomaly detection.
        \end{itemize}
        \item \textbf{Key Point}: Does not require specification of the number of clusters beforehand, can discover clusters of arbitrary shapes. Efficient for large datasets.
    \end{itemize}
\end{frame}

\begin{frame}[fragile]
    \frametitle{Conclusion and Summary Points}
    \begin{block}{Conclusion}
        Understanding these clustering methods provides fundamental tools for analyzing and interpreting complex datasets. Each method serves different purposes and is applicable to various fields, from marketing analytics to biological research.
    \end{block}
    \begin{itemize}
        \item Clustering helps in grouping data similar by features.
        \item K-means is easy to understand but sensitive to initial conditions.
        \item Hierarchical clustering provides a visual structure but may struggle with larger datasets.
        \item DBSCAN excels in finding arbitrary shaped clusters and handling noise.
    \end{itemize}
\end{frame}

\begin{frame}[fragile]
    \frametitle{Formula for K-means Objective Function}
    \begin{equation}
        J = \sum_{i=1}^K \sum_{x_j \in C_i} \lVert x_j - \mu_i \rVert^2
    \end{equation}
    Where $J$ is the total variance, $x_j$ are data points, $C_i$ is the i-th cluster, and $\mu_i$ is the centroid of the i-th cluster.
\end{frame}

\begin{frame}[fragile]
    \frametitle{K-means Clustering - Introduction}
    \begin{block}{What is K-means Clustering?}
        K-means clustering is a popular unsupervised learning algorithm used for partitioning a dataset into distinct groups called clusters.
    \end{block}
    
    \begin{block}{Purpose of K-means}
        The main aim is to group a set of data points into K clusters, where each data point belongs to the cluster with the nearest mean value.
    \end{block}
    
    \begin{block}{Why Do We Need K-means Clustering?}
        \begin{itemize}
            \item Uncover hidden patterns in big data.
            \item Segment customers for targeted marketing.
            \item Identify gene clusters in biology.
        \end{itemize}
    \end{block}
\end{frame}

\begin{frame}[fragile]
    \frametitle{K-means Clustering - Working Principle}
    \begin{enumerate}
        \item \textbf{Initialization}:
            \begin{itemize}
                \item Choose the number of clusters \(K\).
                \item Randomly select \(K\) data points as initial centroids.
            \end{itemize}
        \item \textbf{Assignment Step}:
            \begin{itemize}
                \item Assign each data point to the nearest centroid using:
                \begin{equation}
                    d(x, c) = \sqrt{\sum_{i=1}^{n} (x_i - c_i)^2}
                \end{equation}
            \end{itemize}
        \item \textbf{Update Step}:
            \begin{itemize}
                \item Calculate new centroids:
                \begin{equation}
                    c_k = \frac{1}{|S_k|} \sum_{x_i \in S_k} x_i
                \end{equation}
            \end{itemize}
        \item \textbf{Convergence Check}:
            \begin{itemize}
                \item Repeat steps until centroids change minimally.
            \end{itemize}
    \end{enumerate}
\end{frame}

\begin{frame}[fragile]
    \frametitle{K-means Clustering - Use Cases}
    \begin{block}{Key Features of K-means}
        \begin{itemize}
            \item Scalability: Efficient for large datasets.
            \item Simplicity: Easy to implement and interpret.
            \item Sensitivity: Affected by initial centroid placement or outliers.
        \end{itemize}
    \end{block}

    \begin{block}{Real-world Applications}
        \begin{itemize}
            \item Customer segmentation for targeted marketing.
            \item Anomaly detection in security or fraud.
            \item Image compression reducing color counts.
        \end{itemize}
    \end{block}
    
    \begin{block}{Conclusion}
        K-means clustering helps unlock insights from data, showcasing its versatility across different sectors.
    \end{block}
\end{frame}

\begin{frame}[fragile]
    \titlepage
\end{frame}

\begin{frame}[fragile]
    \frametitle{Introduction to Hierarchical Clustering}
    \begin{itemize}
        \item Hierarchical clustering is a powerful unsupervised learning technique.
        \item Groups data points into clusters based on similarities.
        \item Provides visual representation of data relationships (dendrogram).
    \end{itemize}
    \begin{block}{Why use Hierarchical Clustering?}
        \begin{itemize}
            \item Identifies natural groupings in data.
            \item Suitable for exploratory data analysis with unknown number of clusters.
            \item Dendrogram helps illustrate the arrangement of clusters.
        \end{itemize}
    \end{block}
\end{frame}

\begin{frame}[fragile]
    \frametitle{Types of Hierarchical Clustering}
    Hierarchical clustering can be classified into two main types:
    \begin{enumerate}
        \item \textbf{Agglomerative Clustering} (Bottom-Up Approach)
            \begin{itemize}
                \item Starts with each data point as an individual cluster.
                \item Merges clusters based on a distance metric until one cluster remains.
                \item Common distance metrics:
                    \begin{itemize}
                        \item Single Linkage: minimum distance between points
                        \item Complete Linkage: maximum distance between points
                        \item Average Linkage: mean distance between points
                    \end{itemize}
                \item \textbf{Example}: Points A, B, and C group first, then merge with D if closer than E or F.
            \end{itemize}
        
        \item \textbf{Divisive Clustering} (Top-Down Approach)
            \begin{itemize}
                \item Begins with a single cluster containing all data points.
                \item Sequentially splits the most dissimilar clusters.
                \item \textbf{Example}: Separate A, B, and C from D and E if they are much closer to each other initially.
            \end{itemize}
    \end{enumerate}
\end{frame}

\begin{frame}[fragile]
    \frametitle{Dendrogram Representation}
    \begin{itemize}
        \item A dendrogram visually represents clustering results:
        \begin{itemize}
            \item \textbf{X-Axis}: Data points or clusters.
            \item \textbf{Y-Axis}: Distance or dissimilarity between clusters.
        \end{itemize}
        \item The height at which two clusters are merged indicates their distance.
    \end{itemize}
    \begin{block}{Key Points to Emphasize}
        \begin{itemize}
            \item Choice of linkage method affects dendrogram structure.
            \item Cuts at different heights reveal different levels of clustering granularity.
        \end{itemize}
    \end{block}
\end{frame}

\begin{frame}[fragile]
    \frametitle{Applications of Hierarchical Clustering}
    \begin{itemize}
        \item \textbf{Gene Expression Analysis}: Grouping similar genes based on expression profiles.
        \item \textbf{Document Clustering}: Organizing documents into categories based on content.
        \item \textbf{Market Segmentation}: Identifying distinct customer segments for targeted marketing strategies.
    \end{itemize}

    \begin{block}{Key Takeaways}
        \begin{itemize}
            \item Essential for understanding complex data patterns.
            \item Flexible approach that does not require pre-specified clusters.
            \item Dendrograms provide insightful visual analysis.
        \end{itemize}
    \end{block}
\end{frame}

\begin{frame}[fragile]
    \frametitle{Conclusion and Questions}
    \begin{itemize}
        \item Understanding hierarchical clustering, particularly through agglomerative and divisive methods, is vital for effective data analysis.
        \item Enhances data mining techniques widely used in modern AI applications.
    \end{itemize}
    \begin{block}{Question to Ponder}
        How can the choice of distance metric in hierarchical clustering influence the results, and how might it differ from other clustering techniques like K-means or DBSCAN?
    \end{block}
\end{frame}

\begin{frame}
  \frametitle{DBSCAN: Density-Based Clustering}
  \begin{block}{Overview}
    DBSCAN (Density-Based Spatial Clustering of Applications with Noise) is a density-based clustering algorithm that groups closely packed points and identifies outliers in low-density areas.
  \end{block}
  \begin{itemize}
    \item No need to specify the number of clusters beforehand.
    \item Handles noise effectively.
  \end{itemize}
\end{frame}

\begin{frame}
  \frametitle{DBSCAN: Core Concepts}
  \begin{itemize}
    \item \textbf{Density}: Defined by the number of points (minPts) within a specified radius ($\epsilon$).
    \item \textbf{Core Points}: Have at least minPts neighboring points within $\epsilon$.
    \item \textbf{Border Points}: Not core, but within the neighborhood of a core point.
    \item \textbf{Noise Points}: Neither core nor border points.
  \end{itemize}
\end{frame}

\begin{frame}
  \frametitle{DBSCAN: How It Works}
  \begin{enumerate}
    \item Select an arbitrary point: If it’s a core point, a cluster is formed.
    \item Join the neighborhood: Add all neighboring points to the cluster.
    \item Expand the cluster: Repeat for every new core neighbor, checking neighbors of neighbors.
    \item Mark as noise: Remaining points not in a cluster are classified as noise.
  \end{enumerate}
\end{frame}

\begin{frame}
  \frametitle{Advantages of DBSCAN}
  \begin{itemize}
    \item No need to specify number of clusters.
    \item Can discover clusters of arbitrary shapes.
    \item Robust to noise and outliers, enhancing clustering quality.
  \end{itemize}
\end{frame}

\begin{frame}
  \frametitle{DBSCAN: Suitable Use Cases}
  \begin{itemize}
    \item Spatial data analysis: Geographical clustering patterns (e.g., earthquake epicenters).
    \item Image processing: Grouping similar colors or textures in pixel data.
    \item Market segmentation: Discovering segmentations in consumer behavior data.
  \end{itemize}
\end{frame}

\begin{frame}[fragile]
  \frametitle{DBSCAN: Example}
  Consider a dataset of geographic coordinates of stores. Using DBSCAN, we can group closely located stores into clusters, identifying standalone stores as noise or outliers.
\end{frame}

\begin{frame}[fragile]
  \frametitle{DBSCAN: Key Points}
  \begin{itemize}
    \item \textbf{Parameters}: 
      \begin{itemize}
        \item $\epsilon$: Maximum distance for points to be neighbors.
        \item minPts: Minimum points to form a dense region (typically set to 4 for 2D data).
      \end{itemize}
    \item \textbf{Performance}: Efficient with large datasets and performs well on data with varying density.
  \end{itemize}
\end{frame}

\begin{frame}[fragile]
  \frametitle{DBSCAN: Pseudocode and Implementation}
  \begin{lstlisting}[language=Python]
from sklearn.cluster import DBSCAN
import numpy as np

# Example data: 2D points
X = np.array([[1, 2], [2, 2], [2, 3], [8, 7], [8, 8], [25, 80]])

# Initialize DBSCAN
db = DBSCAN(eps=3, min_samples=2).fit(X)

# Get the cluster labels
labels = db.labels_
print(labels)  # Output will indicate clustering results
  \end{lstlisting}
\end{frame}

\begin{frame}
  \frametitle{Conclusion}
  DBSCAN is a powerful clustering technique suitable for datasets that contain noise and require identifying arbitrary-shaped clusters. It effectively distinguishes noise from significant data points, providing robust insights across various applications.
\end{frame}

\begin{frame}[fragile]
    \frametitle{Dimensionality Reduction Overview}
    \begin{block}{What is Dimensionality Reduction?}
        Dimensionality Reduction is the process of reducing the number of random variables under consideration, obtaining a set of principal variables. It is crucial in various data analysis techniques, particularly unsupervised learning and clustering.
    \end{block}
\end{frame}

\begin{frame}[fragile]
    \frametitle{Why Do We Need Dimensionality Reduction?}
    \begin{itemize}
        \item \textbf{Curse of Dimensionality:} High dimensionality can lead to several issues:
        \begin{enumerate}
            \item Increased Computational Cost: Higher dimensions require more computational power and time for analysis.
            \item Overfitting: Models trained on high-dimensional data are more likely to fit noise rather than real data patterns.
            \item Sparsity: Data points become sparse, making it difficult to identify clusters accurately.
        \end{enumerate}
        \item \textbf{Example:} A dataset with 10,000 features but only 100 samples tends to be sparse, complicating clustering identification.
    \end{itemize}
\end{frame}

\begin{frame}[fragile]
    \frametitle{Importance of Dimensionality Reduction in Clustering}
    \begin{itemize}
        \item \textbf{Improves Visualization:} Reducing dimensions allows for better interpretation of complex datasets.
        \item \textbf{Enhances Clustering Results:} Removing noise and redundant features allows algorithms to focus on relevant patterns.
        \item \textbf{Reduces Overfitting:} Fewer dimensions lead to simpler models that generalize better on unseen data.
    \end{itemize}
    \begin{block}{Example Illustration}
        Data originally in a 10-dimensional space can be transformed into 2D or 3D while retaining critical information.
    \end{block}
\end{frame}

\begin{frame}[fragile]
    \frametitle{Key Techniques of Dimensionality Reduction}
    \begin{itemize}
        \item \textbf{Principal Component Analysis (PCA):} Projects data onto directions of maximum variance.
        \item \textbf{t-Distributed Stochastic Neighbor Embedding (t-SNE):} Maintains the proximity of similar instances while separating dissimilar instances in lower dimensions.
    \end{itemize}
\end{frame}

\begin{frame}[fragile]
    \frametitle{Conclusion}
    Dimensionality Reduction is a foundational concept in unsupervised learning that enhances clustering performance, aids visualization, and improves model interpretability. 
    \begin{itemize}
        \item Mastering this concept helps in understanding techniques and their applications, such as in AI tools like ChatGPT.
    \end{itemize}
\end{frame}

\begin{frame}[fragile]
    \frametitle{Principal Component Analysis (PCA)}
    \begin{block}{Overview}
        PCA is a technique for reducing the dimensionality of data while preserving variance and structure. 
        It is particularly useful in data science for visualizing and interpreting high-dimensional datasets.
    \end{block}
\end{frame}

\begin{frame}[fragile]
    \frametitle{1. Introduction to PCA}
    \begin{itemize}
        \item \textbf{Motivation}: Handling high-dimensional datasets in data science can complicate visualization and pattern interpretation. PCA addresses this issue.
        \item \textbf{Use Case}: In image processing, large datasets can be reduced to a few components that capture essential features, facilitating analysis.
    \end{itemize}
\end{frame}

\begin{frame}[fragile]
    \frametitle{2. Concept and Definition}
    \begin{itemize}
        \item \textbf{Definition}: PCA transforms data into a new coordinate system where the greatest variance lies along the first principal component, followed by the second, and so on.
        \item \textbf{Working Principle}: PCA identifies principal components that maximize data variance in specified directions.
    \end{itemize}
\end{frame}

\begin{frame}[fragile]
    \frametitle{3. Mathematical Foundation}
    \begin{enumerate}
        \item \textbf{Data Standardization}:
        \begin{equation}
            Z = \frac{X - \mu}{\sigma}
        \end{equation}
        
        \item \textbf{Covariance Matrix}:
        \begin{equation}
            C = \frac{1}{n-1} Z^T Z
        \end{equation}
        
        \item \textbf{Eigenvalues and Eigenvectors}:
        \begin{itemize}
            \item Select top \( k \) eigenvectors associated with the largest \( k \) eigenvalues to form the new feature space.
        \end{itemize}
        
        \item \textbf{Projection}:
        \begin{equation}
            Y = Z V_k
        \end{equation}
    \end{enumerate}
\end{frame}

\begin{frame}[fragile]
    \frametitle{4. Practical Application}
    \begin{itemize}
        \item \textbf{Example}: PCA can reduce a dataset with hundreds of dimensions to two or three, making visualization and clustering easier.
        \item \textbf{AI Applications}: PCA aids in exploratory data analysis, speeds up training times, and improves performance in algorithms like ChatGPT by simplifying high-dimensional data processing.
    \end{itemize}
\end{frame}

\begin{frame}[fragile]
    \frametitle{5. Key Points to Emphasize}
    \begin{itemize}
        \item PCA is fundamental for data preprocessing and visualization.
        \item It tackles the curse of dimensionality by retaining significant features.
        \item PCA preserves maximum variance in reduced datasets, enhancing their usability for analysis.
    \end{itemize}
\end{frame}

\begin{frame}[fragile]
    \frametitle{Summary}
    \begin{block}{Conclusion}
        Principal Component Analysis (PCA) effectively reduces dimensionality while preserving essential variance in high-dimensional datasets, making it a crucial technique for data analysis and machine learning enhancements.
    \end{block}
\end{frame}

\begin{frame}[fragile]
    \frametitle{t-distributed Stochastic Neighbor Embedding (t-SNE) - Overview}
    % Overview of t-SNE as a dimensionality reduction technique
    t-SNE is a powerful technique for dimensionality reduction well-suited for visualizing high-dimensional data by converting similarities between data points into probabilities. It effectively maintains the local structure of data, which is critical for analyzing complex datasets.
    
    \begin{itemize}
        \item Converts high-dimensional data into lower dimensions (2D/3D).
        \item Preserves local data structures, enhancing interpretability.
        \item Addresses limitations of conventional methods like PCA.
    \end{itemize}
\end{frame}

\begin{frame}[fragile]
    \frametitle{Why Use t-SNE?}
    % Reasons for using t-SNE for dimensionality reduction
    High-dimensional datasets can be challenging to visualize and interpret. Conventional methods may not capture intrinsic data structures effectively. t-SNE addresses these challenges by:

    \begin{itemize}
        \item \textbf{Preserving Local Clusters:} Emphasizes similarity probabilities, ensuring close points in high-dimensional space remain close in lower-dimensional representation.
        \item \textbf{Non-linear Processes:} Captures complex patterns and non-linear relationships, revealing hidden structures in the data.
    \end{itemize}
\end{frame}

\begin{frame}[fragile]
    \frametitle{How t-SNE Works}
    % Detailed explanation of the t-SNE methodology
    t-SNE consists of several key steps:

    \begin{enumerate}
        \item \textbf{Probability Distribution:} Construct a probability distribution for each data point measuring its similarity to other points.
        \begin{equation}
            p_{j|i} = \frac{exp(-\|x_i - x_j\|^2 / 2\sigma_i^2)}{\sum_{k \neq i} exp(-\|x_i - x_k\|^2 / 2\sigma_i^2)}
        \end{equation}
        
        \item \textbf{Symmetrization:} Ensure symmetry in the probability distribution.
        \begin{equation}
            p_{ij} = \frac{p_{j|i} + p_{i|j}}{2N}
        \end{equation}

        \item \textbf{Low-Dimensional Representation:} Minimize divergence between high and low-dimensional distributions using Kullback-Leibler divergence:
        \begin{equation}
            C = \sum_{i \neq j} p_{ij} \log \left( \frac{p_{ij}}{q_{ij}} \right)
        \end{equation}
        where \( q_{ij} \) is modeled using the Student's t-distribution.
    \end{enumerate}
\end{frame}

\begin{frame}[fragile]
    \frametitle{Use Cases of t-SNE}
    % Examples of t-SNE applications in various domains
    t-SNE is used in various fields for effective visualization, including:

    \begin{itemize}
        \item \textbf{Image Visualization:} Visualizing clusters in image datasets (e.g., grouping similar handwritten digits).
        \item \textbf{Gene Expression Analysis:} Understanding relationships between different gene expressions in biological data.
        \item \textbf{Natural Language Processing:} Clustering word embeddings to reveal similar words or topics within documents.
    \end{itemize}
\end{frame}

\begin{frame}[fragile]
    \frametitle{Key Points and Conclusion}
    % Summary and conclusion about the effectiveness of t-SNE
    Key points to emphasize:
    \begin{itemize}
        \item \textbf{Local Structure Preservation:} Retains local configurations while downplaying global structures.
        \item \textbf{Scalability:} Computationally intensive; performance may decrease with larger datasets.
        \item \textbf{Parameter Sensitivity:} The choice of perplexity significantly impacts embedding quality.
    \end{itemize}

    t-SNE serves as a valuable tool for revealing hidden structures in high-dimensional data, facilitating interpretable visualizations across diverse domains.
\end{frame}

\begin{frame}[fragile]
    \frametitle{Introduction to Generative Models}
    \begin{block}{Overview of Generative Models}
        Generative models are a class of machine learning models designed to generate new data instances that resemble a given training dataset. They capture the underlying distribution of data to create new outputs.
    \end{block}
\end{frame}

\begin{frame}[fragile]
    \frametitle{1. What are Generative Models?}
    \begin{itemize}
        \item \textbf{Definition}: Learn the joint probability distribution of the input data \( P(X) \).
        \item \textbf{Goal}: Understand the structure of data and generate new, similar examples.
    \end{itemize}
    \begin{block}{Outline}
        \begin{itemize}
            \item Definition and Purpose
            \item Importance in Various Domains
        \end{itemize}
    \end{block}
\end{frame}

\begin{frame}[fragile]
    \frametitle{2. Why do we need Generative Models?}
    \begin{itemize}
        \item \textbf{Data Augmentation}: Synthesize additional training data, crucial in scenarios where data is scarce.
        \item \textbf{Creativity and Art}: Used to create art, music, and design.
        \item \textbf{Understanding Data}: Help in better understanding data distribution, improving the performance of discriminative models.
    \end{itemize}
    \begin{block}{Outline}
        \begin{itemize}
            \item Key Characteristics and Applications
        \end{itemize}
    \end{block}
\end{frame}

\begin{frame}[fragile]
    \frametitle{3. Applications of Generative Models}
    \begin{itemize}
        \item \textbf{Natural Language Processing}: Models like ChatGPT generate human-like text.
        \item \textbf{Image Generation}: Programs such as DALL-E and StyleGAN create realistic images.
        \item \textbf{Drug Discovery}: Design novel molecules for pharmaceutical development.
    \end{itemize}
    \begin{block}{Outline}
        \begin{itemize}
            \item Summary
            \item Transition to types of generative models
        \end{itemize}
    \end{block}
\end{frame}

\begin{frame}[fragile]
    \frametitle{4. Key Characteristics}
    \begin{itemize}
        \item \textbf{Unsupervised Learning}: Mostly learn without labeled data, suitable for real-world situations.
        \item \textbf{Complex Data Distribution Modeling}: Excel at modeling multi-modal distributions, generating diverse outputs.
    \end{itemize}
\end{frame}

\begin{frame}[fragile]
    \frametitle{Summary}
    Generative models play a vital role in modern AI applications from creative industries to scientific advancements. They enable systems to understand and replicate the complexities of real-world data, leading to countless opportunities in various domains.
    
    \begin{block}{Transition}
        Next, we will explore different types of generative models and their specific applications in depth.
    \end{block}
\end{frame}

\begin{frame}[fragile]
    \frametitle{Types of Generative Models}
    
    \begin{block}{Overview}
        Generative models are revolutionary tools in machine learning that allow systems to create new, synthetic data points resembling real-world examples. In this slide, we will discuss two prominent types of generative models: \textbf{Generative Adversarial Networks (GANs)} and \textbf{Variational Autoencoders (VAEs)}.
    \end{block}
\end{frame}

\begin{frame}[fragile]
    \frametitle{Generative Adversarial Networks (GANs)}
    
    \begin{block}{Concept}
        GANs are composed of two neural networks: a \textit{generator} and a \textit{discriminator}. The generator creates fake data, while the discriminator evaluates data authenticity (real vs. fake).
    \end{block}

    \begin{itemize}
        \item \textbf{Motivation:} GANs learn the underlying distribution of the data, enabling the generation of new samples.
        \item \textbf{Example:} A GAN trained on artwork data can produce novel pieces resembling those of renowned artists.
    \end{itemize}

    \begin{block}{Key Points}
        \begin{itemize}
            \item Produce high-quality outputs (e.g., images, audio).
            \item Applications include:
                \begin{itemize}
                    \item Image Super-resolution
                    \item Video Generation
                    \item Image-to-image translation (e.g., sketches to photographs)
                \end{itemize}
        \end{itemize}
    \end{block}
\end{frame}

\begin{frame}[fragile]
    \frametitle{Variational Autoencoders (VAEs)}
    
    \begin{block}{Concept}
        VAEs encode data into a lower-dimensional latent space while maintaining the capability to sample from this space to generate new data points.
    \end{block}

    \begin{itemize}
        \item \textbf{Motivation:} To model internal variations of the data and generate new samples from a learned distribution.
        \item \textbf{Example:} A VAE trained on facial images learns to represent facial features in a compressed form to generate new faces.
    \end{itemize}

    \begin{block}{Key Points}
        \begin{itemize}
            \item Achieve a balance between reconstruction quality and generating diverse samples.
            \item Applications include:
                \begin{itemize}
                    \item Text generation
                    \item Drug discovery
                    \item Anomaly detection
                \end{itemize}
        \end{itemize}
    \end{block}
    
    \begin{equation}
        \text{Loss} = \text{Reconstruction Loss} + \beta \times \text{KL Divergence}
    \end{equation}
    Where:
    \begin{itemize}
        \item \textbf{Reconstruction Loss:} Measures output's match to input.
        \item \textbf{KL Divergence:} Measures difference between learned and prior distributions.
    \end{itemize}
\end{frame}

\begin{frame}[fragile]
    \frametitle{Summary and Next Steps}
    
    \begin{block}{Summary}
        Both GANs and VAEs play critical roles in generative modeling, each with unique approaches and applications. Understanding these models can empower exploration in AI fields such as content generation and simulation.
    \end{block}
    
    \textbf{Next Steps:} Dive deeper into GANs in the following slide to explore their mechanisms and real-world applications.
\end{frame}

\begin{frame}[fragile]
    \frametitle{Introduction to GANs}
    \begin{itemize}
        \item \textbf{Motivation}: Generative models create new data instances that resemble existing data.
        \item Introduced by Ian Goodfellow in 2014, GANs have revolutionized fields such as:
        \begin{itemize}
            \item Image processing
            \item Art creation
            \item Realistic simulation
        \end{itemize}
    \end{itemize}
\end{frame}

\begin{frame}[fragile]
    \frametitle{How GANs Work}
    \begin{itemize}
        \item \textbf{Architecture}:
        \begin{itemize}
            \item \textbf{Generator (G)}: Creates fake data.
            \item \textbf{Discriminator (D)}: Evaluates and distinguishes between real and fake data.
        \end{itemize}
        
        \item \textbf{Training Process}:
        \begin{enumerate}
            \item G generates a batch of synthetic data from random noise.
            \item D receives real data (from training dataset) and fake data (from G).
            \item D outputs a probability score indicating real or fake.
            \item G aims to maximize D's mistakes, while D aims to minimize its errors.
        \end{enumerate}
    \end{itemize}
\end{frame}

\begin{frame}[fragile]
    \frametitle{Loss Functions and Key Points}
    \begin{itemize}
        \item \textbf{Loss Functions}:
        \begin{equation}
            L_D = - (E[\log D(x)] + E[\log(1 - D(G(z)))] )
        \end{equation}
        \begin{equation}
            L_G = - E[\log D(G(z))]
        \end{equation}
        
        \item \textbf{Key Points to Emphasize}:
        \begin{itemize}
            \item \textbf{Adversarial Training}: Continuous improvements in data generation quality.
            \item \textbf{Convergence}: Occurs when D can no longer distinguish between real and fake data.
        \end{itemize}
    \end{itemize}
\end{frame}

\begin{frame}[fragile]
    \frametitle{Applications of GANs}
    \begin{enumerate}
        \item \textbf{Image Generation}: High-quality images that can trick human observers (e.g. StyleGAN).
        \item \textbf{Data Augmentation}: Generating synthetic data to enhance datasets, especially useful with limited data.
        \item \textbf{Super Resolution Imaging}: Enhancing images by creating higher-resolution details.
        \item \textbf{Video Generation \& Prediction}: Generating video frames based on previous sequences.
        \item \textbf{Text-to-Image Generation}: Examples like DALL-E generating images from textual descriptions.
    \end{enumerate}
\end{frame}

\begin{frame}[fragile]
    \frametitle{Recent Impact and Further Learning}
    \begin{itemize}
        \item \textbf{Recent Impact}: Advancements in AI applications, tools for design, entertainment, and virtual reality.
    \end{itemize}

    \begin{block}{Outline for Further Learning}
        \begin{enumerate}
            \item Introduction and Motivation of GANs
            \item Architectural Overview: Generator and Discriminator
            \item Detailed Training Process
            \item Mathematical Formulation: Loss Functions
            \item Key Applications of GANs in Various Fields
            \item Summary of Recent Developments and Trends
        \end{enumerate}
    \end{block}
\end{frame}

\begin{frame}[fragile]
    \frametitle{Variational Autoencoders (VAEs) - Introduction}
    \begin{block}{Overview}
        Variational Autoencoders (VAEs) are generative models that learn compact representations of data distributions inspired by Bayesian inference.
        They combine autoencoders and probabilistic modeling to generate new data resembling the training set.
    \end{block}
    
    \begin{itemize}
        \item Complex Data Modeling
        \item Utilization of Latent Variables
        \item Integration with Deep Learning
    \end{itemize}
\end{frame}

\begin{frame}[fragile]
    \frametitle{VAE Architecture}
    \begin{enumerate}
        \item \textbf{Encoder (Inference Network)}:
            \begin{itemize}
                \item Maps input data \( x \) to a latent space \( z \).
                \item Outputs mean \( \mu \) and variance \( \sigma^2 \):
                \begin{equation}
                    q(z|x) \sim \mathcal{N}(\mu, \sigma^2 I)
                \end{equation}
            \end{itemize}
        
        \item \textbf{Latent Space}:
            \begin{itemize}
                \item Compressed representation of input data.
                \item Uses the reparameterization trick for sampling.
            \end{itemize}
        
        \item \textbf{Decoder (Generative Network)}:
            \begin{itemize}
                \item Converts latent variables \( z \) to reconstructed data \( x' \).
                \begin{equation}
                    p(x|z) = \text{data distribution}(\text{Decoder}(z))
                \end{equation}
            \end{itemize}

        \item \textbf{Loss Function}:
            \begin{equation}
                \mathcal{L} = \mathbb{E}_{q(z|x)}[\log p(x|z)] - D_{KL}(q(z|x) || p(z))
            \end{equation}
        \end{enumerate}
\end{frame}

\begin{frame}[fragile]
    \frametitle{Comparison with GANs}
    \begin{itemize}
        \item \textbf{Training Objective}:
            \begin{itemize}
                \item VAEs minimize reconstruction error and KL divergence.
                \item GANs focus on adversarial loss between generator and discriminator.
            \end{itemize}
        
        \item \textbf{Outcome of Model}:
            \begin{itemize}
                \item VAEs yield meaningful latent space distributions.
                \item GANs often produce sharper images but may suffer from mode collapse.
            \end{itemize}

        \item \textbf{Applications}:
            \begin{itemize}
                \item VAEs: Feature extraction, diverse outputs, anomaly detection.
                \item GANs: High-fidelity image generation, artistic applications.
            \end{itemize}
    \end{itemize}
    
    \begin{block}{Key Points}
        \begin{itemize}
            \item Blend generative modeling with neural networks.
            \item Regularization via KL divergence ensures smooth latent spaces.
            \item Facilitate unsupervised learning tasks and novel sample generation.
        \end{itemize}
    \end{block}
\end{frame}

\begin{frame}[fragile]
    \frametitle{Example Code Snippet (PyTorch)}
    \begin{lstlisting}[language=Python]
import torch
from torch import nn

class VAE(nn.Module):
    def __init__(self, input_dim, latent_dim):
        super(VAE, self).__init__()
        self.encoder = nn.Sequential(
            nn.Linear(input_dim, 128),
            nn.ReLU(),
            nn.Linear(128, latent_dim * 2)  # Output both mu and logvar
        )
        self.decoder = nn.Sequential(
            nn.Linear(latent_dim, 128),
            nn.ReLU(),
            nn.Linear(128, input_dim),
            nn.Sigmoid()
        )

    def encode(self, x):
        encoded = self.encoder(x)
        mu, logvar = encoded.chunk(2, dim=-1)
        return mu, logvar

    def reparameterize(self, mu, logvar):
        std = torch.exp(0.5 * logvar)
        eps = torch.randn_like(std)
        return mu + eps * std

    def decode(self, z):
        return self.decoder(z)

    def forward(self, x):
        mu, logvar = self.encode(x)
        z = self.reparameterize(mu, logvar)
        return self.decode(z), mu, logvar
    \end{lstlisting}
\end{frame}

\begin{frame}[fragile]
    \frametitle{Conclusion}
    Variational Autoencoders are pivotal in unsupervised learning due to their probabilistic framework. Their architecture enables effective modeling of complex data distributions, making them an essential tool in machine learning applications.
\end{frame}

\begin{frame}[fragile]
  \frametitle{Real-world Applications of Unsupervised Learning - Introduction}
  \begin{block}{Introduction to Unsupervised Learning}
    Unsupervised learning is a type of machine learning where algorithms learn patterns from unlabelled data without explicit guidance. It is crucial for discovering hidden structures in data, making it invaluable across various sectors.
  \end{block}
  \begin{block}{Motivations for Using Unsupervised Learning}
    \begin{itemize}
        \item \textbf{Data Exploration}: Uncover insights that are not easily observable.
        \item \textbf{Pattern Recognition}: Identify natural groupings in data.
        \item \textbf{Feature Engineering}: Discover relevant features for supervised tasks.
    \end{itemize}
  \end{block}
\end{frame}

\begin{frame}[fragile]
  \frametitle{Real-world Applications of Unsupervised Learning - Marketing}
  \begin{block}{Applications in Marketing}
    \begin{itemize}
        \item \textbf{Customer Segmentation}
        \begin{itemize}
            \item \textbf{Technique}: Clustering algorithms (e.g., K-Means, DBSCAN).
            \item \textbf{Example}: Group customers based on purchasing behavior.
            \item \textbf{Outcome}: Enhanced targeting and improved conversion rates.
        \end{itemize}
    \end{itemize}
  \end{block}
\end{frame}

\begin{frame}[fragile]
  \frametitle{Real-world Applications of Unsupervised Learning - Finance & Healthcare}
  \begin{block}{Applications in Finance}
    \begin{itemize}
        \item \textbf{Anomaly Detection}
        \begin{itemize}
            \item \textbf{Technique}: Isolation Forests, One-Class SVM.
            \item \textbf{Example}: Identifying fraudulent transactions by detecting outliers.
            \item \textbf{Outcome}: Early detection of fraud and increased security.
        \end{itemize}
        \item \textbf{Portfolio Management}
        \begin{itemize}
            \item \textbf{Technique}: Dimensionality Reduction (e.g., PCA).
            \item \textbf{Example}: Optimizing asset allocation by reducing data complexity.
            \item \textbf{Outcome}: Better decision-making based on emerging trends.
        \end{itemize}
    \end{itemize}
  \end{block}
  
  \begin{block}{Applications in Healthcare}
    \begin{itemize}
        \item \textbf{Patient Clustering}
        \begin{itemize}
            \item \textbf{Technique}: Hierarchical Clustering.
            \item \textbf{Example}: Grouping patients to personalize healthcare plans.
            \item \textbf{Outcome}: Improved patient care and treatment strategies.
        \end{itemize}
    \end{itemize}
  \end{block}
\end{frame}

\begin{frame}[fragile]
  \frametitle{Key Points and Conclusion}
  \begin{block}{Key Points to Emphasize}
    \begin{itemize}
        \item \textbf{Versatility}: Applicable across diverse domains.
        \item \textbf{Data-Driven Insights}: Enables leveraging large volumes of unstructured data.
        \item \textbf{Foundation for Supervised Learning}: Serves as a preliminary step for supervised analyses.
    \end{itemize}
  \end{block}
  
  \begin{block}{Conclusion}
    Unsupervised learning provides tools for data exploration, uncovering hidden patterns. As the complexity and volume of data continue to grow, these techniques will play a critical role in innovation and informed decision-making across industries.
  \end{block}
\end{frame}

\begin{frame}[fragile]
    \frametitle{Conclusion and Future Directions - Conclusion}
    \begin{block}{Key Points}
        \begin{itemize}
            \item Unsupervised learning identifies patterns without labeled outcomes.
            \item Techniques explored:
                \begin{itemize}
                    \item Clustering (e.g., K-Means, Hierarchical Clustering)
                    \item Dimensionality Reduction (e.g., PCA)
                    \item Anomaly Detection (e.g., fraud detection)
                \end{itemize}
            \item **Purpose**: To uncover hidden structures in large datasets and drive decision-making.
        \end{itemize}
    \end{block}
\end{frame}

\begin{frame}[fragile]
    \frametitle{Conclusion and Future Directions - Future Directions}
    \begin{block}{Emerging Trends and Challenges}
        \begin{enumerate}
            \item **Integration with Supervised Learning**
                \begin{itemize}
                    \item Enhances performance using both labeled and unlabeled data.
                    \item Example: Semi-supervised learning.
                \end{itemize}
            \item **Scalability and Efficiency**
                \begin{itemize}
                    \item Addressing performance in real-time massive datasets.
                    \item Developing algorithms with mini-batch processing and distributed computing.
                \end{itemize}
            \item **Deep Learning and Neural Networks**
                \begin{itemize}
                    \item Innovations like GANs and Autoencoders blend neural networks with unsupervised learning.
                \end{itemize}
        \end{enumerate}
    \end{block}
\end{frame}

\begin{frame}[fragile]
    \frametitle{Conclusion and Future Directions - Future Directions (cont.)}
    \begin{block}{Additional Trends}
        \begin{enumerate}
            \setcounter{enumi}{3}
            \item **Explainability and Interpretability**
                \begin{itemize}
                    \item Critical for trust in complex models, especially in healthcare.
                \end{itemize}
            \item **Real-world Applications**
                \begin{itemize}
                    \item Expanding into fields like cybersecurity anomaly detection and e-commerce customer segmentation.
                    \item Vital for AI applications (e.g., ChatGPT) in understanding user behavior.
                \end{itemize}
        \end{enumerate}
        \vspace{0.5cm}
        \textbf{Final Thoughts:} Unsupervised learning can transform data utilization by addressing upcoming challenges with advanced models.
    \end{block}
\end{frame}


\end{document}