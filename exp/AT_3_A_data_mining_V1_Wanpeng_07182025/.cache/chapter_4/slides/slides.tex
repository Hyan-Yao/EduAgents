\documentclass[aspectratio=169]{beamer}

% Theme and Color Setup
\usetheme{Madrid}
\usecolortheme{whale}
\useinnertheme{rectangles}
\useoutertheme{miniframes}

% Additional Packages
\usepackage[utf8]{inputenc}
\usepackage[T1]{fontenc}
\usepackage{graphicx}
\usepackage{booktabs}
\usepackage{listings}
\usepackage{amsmath}
\usepackage{amssymb}
\usepackage{xcolor}
\usepackage{tikz}
\usepackage{pgfplots}
\pgfplotsset{compat=1.18}
\usetikzlibrary{positioning}
\usepackage{hyperref}

% Custom Colors
\definecolor{myblue}{RGB}{31, 73, 125}
\definecolor{mygray}{RGB}{100, 100, 100}
\definecolor{mygreen}{RGB}{0, 128, 0}
\definecolor{myorange}{RGB}{230, 126, 34}
\definecolor{mycodebackground}{RGB}{245, 245, 245}

% Set Theme Colors
\setbeamercolor{structure}{fg=myblue}
\setbeamercolor{frametitle}{fg=white, bg=myblue}
\setbeamercolor{title}{fg=myblue}
\setbeamercolor{section in toc}{fg=myblue}
\setbeamercolor{item projected}{fg=white, bg=myblue}
\setbeamercolor{block title}{bg=myblue!20, fg=myblue}
\setbeamercolor{block body}{bg=myblue!10}
\setbeamercolor{alerted text}{fg=myorange}

% Set Fonts
\setbeamerfont{title}{size=\Large, series=\bfseries}
\setbeamerfont{frametitle}{size=\large, series=\bfseries}
\setbeamerfont{caption}{size=\small}
\setbeamerfont{footnote}{size=\tiny}

% Document Start
\begin{document}

\frame{\titlepage}

\begin{frame}[fragile]
    \frametitle{Introduction to Classification Techniques}
    \begin{block}{Overview of Classification in Data Mining}
        Classification is a fundamental technique in data mining used to categorize data into predefined classes or labels. It is a predictive modeling technique where the goal is to predict the target class for each record in the dataset based on input features.
    \end{block}
\end{frame}

\begin{frame}[fragile]
    \frametitle{Importance of Classification in Predictive Modeling}
    \begin{itemize}
        \item \textbf{Decision Making:} Enhances decision-making processes in fields such as finance, healthcare, and marketing through automation based on historical data.
        
        \item \textbf{Performance Metrics:} Evaluating classification algorithms improves model accuracy, precision, recall, and F1-score, which are crucial for many applications.
        
        \item \textbf{Real-world Applications:}
        \begin{itemize}
            \item \textit{Finance:} Credit scoring (classifying loan applicants as low, medium, or high risk).
            \item \textit{Healthcare:} Diagnosing diseases based on symptoms and patient history.
            \item \textit{Marketing:} Predicting customer churn (classifying customers likely to leave).
        \end{itemize}
    \end{itemize}
\end{frame}

\begin{frame}[fragile]
    \frametitle{Key Points and Conclusion}
    \begin{itemize}
        \item \textbf{Classification vs. Regression:} Classification predicts categorical outcomes, while regression predicts continuous outcomes (e.g., spam detection vs. house price prediction).
        
        \item \textbf{Popular Classification Algorithms:} 
        \begin{itemize}
            \item Decision Trees
            \item Support Vector Machines (SVM)
            \item k-Nearest Neighbors (k-NN)
            \item Neural Networks
        \end{itemize}
        
        \item \textbf{Data Preprocessing:} Data must be prepared through techniques like data cleaning, normalization, and feature selection to improve model accuracy.
    \end{itemize}
    
    \begin{block}{Example}
        For instance, a dataset might include features like age, income, and credit score. A classification algorithm could determine if a client should be given a loan (Class: "Approved" or "Denied").
    \end{block}
    
    \begin{block}{Conclusion}
    Classification is vital for extracting insights from data, allowing businesses and researchers to predict outcomes and make informed decisions. We'll explore specific classification algorithms, starting with Decision Trees.
    \end{block}
\end{frame}

\begin{frame}[fragile]{Decision Trees - Introduction}
    \begin{block}{Introduction to Decision Trees}
        Decision Trees are a popular and powerful machine learning method used for classification and regression tasks. 
        They represent decisions and their possible consequences, visually resembling a tree structure whereby each internal node denotes a feature (attribute), each branch represents a decision rule, and each leaf node signifies an outcome (class label).
    \end{block}
\end{frame}

\begin{frame}[fragile]{Decision Trees - Structure}
    \begin{block}{Structure of Decision Trees}
        \begin{itemize}
            \item \textbf{Root Node}: Represents the entire dataset and the first split based on the most significant attribute.
            \item \textbf{Internal Nodes}: Decision nodes where the dataset is split based on certain conditions.
            \item \textbf{Leaf Nodes}: Terminal nodes that deliver a class label or a numerical value in regression.
        \end{itemize}
    \end{block}
\end{frame}

\begin{frame}[fragile]{Decision Trees - Key Algorithms}
    \begin{block}{Key Algorithms}
        \begin{enumerate}
            \item \textbf{C4.5 Algorithm}:
            \begin{itemize}
                \item Extension of the ID3 algorithm.
                \item Utilizes information gain ratio to determine splits.
                \item Handles categorical and continuous attributes.
                \item Accommodates missing values and generates pruned trees to avoid overfitting.
            \end{itemize}
            
            \item \textbf{CART (Classification and Regression Trees)}:
            \begin{itemize}
                \item Uses Gini impurity or mean squared error (for regression) to split nodes.
                \item Produces binary trees with each internal node branching into only two children.
                \item Provides a straightforward methodology for pruning the tree post-creation for enhanced generalization.
            \end{itemize}
        \end{enumerate}
    \end{block}
\end{frame}

\begin{frame}[fragile]{Decision Trees - Practical Applications}
    \begin{block}{Practical Applications}
        \begin{itemize}
            \item \textbf{Healthcare}: Diagnosing diseases based on patient data.
            \item \textbf{Finance}: Credit scoring and risk assessment.
            \item \textbf{Marketing}: Customer segmentation and churn prediction.
            \item \textbf{Manufacturing}: Decision-making processes in quality control.
        \end{itemize}
    \end{block}
\end{frame}

\begin{frame}[fragile]{Decision Trees - Key Points}
    \begin{block}{Key Points to Emphasize}
        \begin{itemize}
            \item Highly interpretable, allowing users to understand decision-making processes easily.
            \item Versatile, as they handle both numerical and categorical data efficiently.
            \item Prone to overfitting, especially with complex trees that model noise in the training data.
        \end{itemize}
    \end{block}
\end{frame}

\begin{frame}[fragile]{Decision Trees - Illustration Example}
    \begin{block}{Illustration Example}
        Consider a dataset with three attributes: Weather (Sunny, Rainy), Temperature (Hot, Mild, Cool), and Humidity (High, Normal). The decision tree might look like:
    
        \begin{verbatim}
                  Weather
                  /     \
               Sunny    Rainy
               /          \
         Humidity        Temperature
         /      \         /      \
      High    Normal   Hot      Cool
       |           |      |         |
     No         Yes    Yes       No
        \end{verbatim}
    \end{block}
\end{frame}

\begin{frame}[fragile]{Decision Trees - Illustrative Code Snippet}
    \begin{block}{Code Snippet}
        Here’s a simple code snippet to build a decision tree using the \texttt{scikit-learn} library in Python:
        \begin{lstlisting}[language=Python]
from sklearn.tree import DecisionTreeClassifier
from sklearn.model_selection import train_test_split

# Sample Dataset
X = [[0, 0], [1, 1], [1, 0], [0, 1]]
y = [0, 1, 1, 0]   # Class labels

# Splitting Dataset
X_train, X_test, y_train, y_test = train_test_split(X, y, test_size=0.3)

# Decision Tree Classifier
clf = DecisionTreeClassifier()
clf.fit(X_train, y_train)

# Making Predictions
predictions = clf.predict(X_test)
        \end{lstlisting}
    \end{block}
\end{frame}

\begin{frame}[fragile]{Decision Trees - Conclusion}
    \begin{block}{Conclusion}
        Decision Trees are an essential part of classification techniques, offering a simple yet effective way to build predictive models. 
        Understanding their structure, algorithms, and potential applications is crucial for utilizing these models effectively in various domains. 
        Next, we will explore the advantages and disadvantages of Decision Trees in depth.
    \end{block}
\end{frame}

\begin{frame}[fragile]
    \frametitle{Advantages and Disadvantages of Decision Trees}
    Decision Trees are a popular machine learning algorithm used for both classification and regression tasks. They create a model in a tree-like structure representing decisions and their consequences.
\end{frame}

\begin{frame}[fragile]
    \frametitle{Advantages of Decision Trees}
    \begin{itemize}
        \item \textbf{Interpretability:} 
        Easy to understand and follow through the tree structure.
        \item \textbf{No Need for Feature Scaling:} 
        Unlike SVMs and K-means, scaling is unnecessary for Decision Trees.
        \item \textbf{Non-Linear Relationships:} 
        They can effectively capture non-linear relationships without linearity assumptions.
        \item \textbf{Automatic Feature Selection:} 
        Implicit feature selection ignores irrelevant features during the splitting process.
    \end{itemize}
\end{frame}

\begin{frame}[fragile]
    \frametitle{Disadvantages of Decision Trees}
    \begin{itemize}
        \item \textbf{Overfitting:} 
        Deep trees can learn noise in training data, failing on unseen data.
        \item \textbf{Instability:} 
        Small data changes can lead to significantly different trees.
        \item \textbf{Bias Towards Splits with More Levels:} 
        The model may favor complex features, leading to potential bias.
        \item \textbf{Limited Predictive Power:} 
        Struggles with datasets containing interdependent features.
    \end{itemize}
\end{frame}

\begin{frame}[fragile]
    \frametitle{Key Points and Conclusion}
    \begin{block}{Key Points}
        \begin{itemize}
            \item \textbf{Interpretability} helps stakeholders understand decisions.
            \item \textbf{Overfitting} is a major challenge; techniques like pruning can help.
            \item Trees need tuning to avoid bias and instability.
        \end{itemize}
    \end{block}
    \begin{block}{Conclusion}
        Decision Trees provide a powerful method for classification tasks. Understanding their advantages and disadvantages supports effective utilization.
    \end{block}
\end{frame}

\begin{frame}[fragile]
    \frametitle{Example Code Snippet}
    \begin{lstlisting}[language=Python]
from sklearn.tree import DecisionTreeClassifier
# Sample Data
X = [[0, 0], [1, 1], [0, 1], [1, 0]]
y = [0, 1, 1, 0]

# Create Decision Tree Classifier
clf = DecisionTreeClassifier()
clf.fit(X, y)

# Predict
predictions = clf.predict([[0, 0], [1, 1]])
print(predictions)
    \end{lstlisting}
\end{frame}

\begin{frame}[fragile]
  \frametitle{Naive Bayes Classifier - Overview}
  Naive Bayes is a family of probabilistic algorithms based on Bayes' theorem, primarily used for classification tasks. 
  \begin{itemize}
      \item Operates under the assumption of feature independence given the class label.
      \item Effective in various real-world applications despite the "naive" independence assumption.
  \end{itemize}
\end{frame}

\begin{frame}[fragile]
  \frametitle{Naive Bayes Classifier - Key Concepts}
  \begin{enumerate}
      \item \textbf{Bayes' Theorem}:
        \begin{equation}
        P(A|B) = \frac{P(B|A) \cdot P(A)}{P(B)}
        \end{equation}
        where \( P(A|B) \) is the posterior probability of class \( A \) given feature \( B \).
      
      \item \textbf{Independence Assumption}:
        \begin{itemize}
            \item Assumes all features contribute independently to the likelihood of the outcome.
            \item Simplification leads to efficient computations, particularly in high-dimensional data.
        \end{itemize}
  \end{enumerate}
\end{frame}

\begin{frame}[fragile]
  \frametitle{Naive Bayes Classifier - Types & Use Cases}
  \textbf{Types of Naive Bayes Classifiers:}
  \begin{itemize}
      \item \textbf{Gaussian Naive Bayes}: Assumes features follow a normal distribution; useful for continuous data.
      \item \textbf{Multinomial Naive Bayes}: Designed for categorical data; effective for text classification.
      \item \textbf{Bernoulli Naive Bayes}: Suitable for binary features; indicates presence/absence of features.
  \end{itemize}
  
  \textbf{Use Cases in Text Classification:}
  \begin{itemize}
      \item Spam Detection: Classifying emails as "spam" or "not spam".
      \item Sentiment Analysis: Evaluating movie reviews as positive or negative.
      \item Topic Classification: Categorizing news articles into various topics.
  \end{itemize}
\end{frame}

\begin{frame}[fragile]
    \frametitle{Mathematics Behind Naive Bayes - Introduction}
    \begin{block}{Bayes' Theorem}
        Bayes' Theorem is a foundational concept in probability theory that relates the conditional and marginal probabilities of random events. It allows us to update our beliefs based on new evidence.
    \end{block}
    
    \begin{equation}
    P(A|B) = \frac{P(B|A) \cdot P(A)}{P(B)}
    \end{equation}
    
    \begin{itemize}
        \item \textbf{P(A|B)}: Probability of event A occurring given that B is true (posterior).
        \item \textbf{P(B|A)}: Probability of event B occurring given that A is true (likelihood).
        \item \textbf{P(A)}: Probability of event A (prior).
        \item \textbf{P(B)}: Probability of event B (evidence).
    \end{itemize}
\end{frame}

\begin{frame}[fragile]
    \frametitle{Mathematics Behind Naive Bayes - Conditional Probability}
    \begin{block}{Understanding Conditional Probability}
        Conditional probability is the probability of one event occurring given the occurrence of another event, and it is foundational for understanding Bayes' Theorem.
    \end{block}
    
    \begin{itemize}
        \item For example, consider a class of students and their study habits (Event A) relative to passing an exam (Event B).
        \item Let \( P(B|A) = 0.7 \) (70% of studious students pass) and \( P(A) = 0.8 \) (80% of students study).
        \item The overall probability of passing is given by:
        \begin{equation}
        P(B) = P(B|A) \cdot P(A) + P(B|A') \cdot P(A')
        \end{equation}
    \end{itemize}
\end{frame}

\begin{frame}[fragile]
    \frametitle{Mathematics Behind Naive Bayes - Naive Bayes Assumptions}
    \begin{block}{Naive Bayes Assumptions}
        Naive Bayes applies Bayes' Theorem to classification problems based on the assumption that all features are independent given the class label.
    \end{block}
    
    \begin{equation}
    P(C|F_1, F_2, \ldots, F_n) \propto P(C) \cdot P(F_1|C) \cdots P(F_n|C)
    \end{equation}

    \begin{itemize}
        \item \textbf{Goal}: Find the class \( C \) that maximizes \( P(C|F_1, F_2, \ldots, F_n) \).
        \item \textbf{Independence Assumption}: Features are conditionally independent given the class label, simplifying computation.
        \item Despite its simplicity, Naive Bayes performs well in practice, especially in text classification.
    \end{itemize}
\end{frame}

\begin{frame}[fragile]
    \frametitle{Mathematics Behind Naive Bayes - Example}
    \begin{block}{Illustrative Example}
        Classifying an email as "spam" or "not spam" using Naive Bayes principles.
    \end{block}
    
    \begin{itemize}
        \item Given:
        \begin{itemize}
            \item \( P(\text{Spam}) = 0.4 \)
            \item \( P(\text{Not Spam}) = 0.6 \)
            \item \( P(\text{"win"|Spam}) = 0.9 \)
            \item \( P(\text{"win"|Not Spam}) = 0.1 \)
        \end{itemize}
        \item To evaluate:
        \begin{equation}
        P(\text{Spam|“win”}) \propto P(\text{"win"|Spam}) \cdot P(\text{Spam}) 
        \end{equation}
    \end{itemize}
\end{frame}

\begin{frame}[fragile]
    \frametitle{Pros and Cons of Naive Bayes - Overview}
    \begin{block}{Overview}
        Naive Bayes is a family of probabilistic algorithms based on Bayes' theorem, primarily used for classification. 
        It is particularly famous for its application in natural language processing, spam detection, and document classification. 
        We will discuss both the strengths and limitations of the Naive Bayes algorithm.
    \end{block}
\end{frame}

\begin{frame}[fragile]
    \frametitle{Pros and Cons of Naive Bayes - Strengths}
    \begin{block}{Strengths of Naive Bayes}
        \begin{enumerate}
            \item \textbf{Speed and Efficiency}
            \begin{itemize}
                \item Computationally efficient with small training data for parameter estimation.
                \item Executes quickly even with large datasets.
            \end{itemize}
            
            \item \textbf{Simplicity and Interpretability}
            \begin{itemize}
                \item Easy to implement with straightforward statistical insights.
                \item Outputs probabilities that clarify confidence in predictions.
            \end{itemize}
            
            \item \textbf{Good Performance with Imbalanced Data}
            \begin{itemize}
                \item Effective in situations where one class significantly outnumbers another.
                \item Surprising performance even with limited training data for some classes.
            \end{itemize}
            
            \item \textbf{Handles High Dimensionality Well}
            \begin{itemize}
                \item Useful in text classification with many features and limited data points.
            \end{itemize}
        \end{enumerate}
    \end{block}
\end{frame}

\begin{frame}[fragile]
    \frametitle{Pros and Cons of Naive Bayes - Limitations}
    \begin{block}{Limitations of Naive Bayes}
        \begin{enumerate}
            \item \textbf{Independence Assumption}
            \begin{itemize}
                \item Assumes features are independent given the class label, which may not be true.
            \end{itemize}
            
            \item \textbf{Limited Expressiveness}
            \begin{itemize}
                \item May perform poorly on complex datasets with significant feature interactions.
            \end{itemize}
            
            \item \textbf{Sensitive to Irrelevant Features}
            \begin{itemize}
                \item Equal weighting of all features can lead to classification errors in noisy data.
            \end{itemize}
            
            \item \textbf{Zero Probability Problem}
            \begin{itemize}
                \item Assigning a zero probability to unobserved features; can use Laplace smoothing to mitigate this.
            \end{itemize}
        \end{enumerate}
    \end{block}
\end{frame}

\begin{frame}[fragile]
    \frametitle{Key Points and Conclusion}
    \begin{block}{Key Points to Remember}
        \begin{itemize}
            \item Naive Bayes is fast and simple, ideal for high-dimensional data but has limitations.
            \item Consider data context; it excels in some scenarios (e.g., text classification) but may struggle with feature interactions.
            \item Balancing pros and cons aids in determining its suitability for specific tasks.
        \end{itemize}
    \end{block}

    \begin{block}{Conclusion}
        Understanding Naive Bayes's strengths and weaknesses is essential for effective application in machine learning, helping to leverage its speed while being aware of its limitations.
    \end{block}
\end{frame}

\begin{frame}
  \frametitle{k-Nearest Neighbors (k-NN)}
  \begin{block}{Definition}
    The k-Nearest Neighbors (k-NN) algorithm is a simple, yet powerful supervised learning technique used for classification and regression tasks. The essence of k-NN is to predict the label of a data point based on the labels of its 'k' nearest neighbors in the feature space.
  \end{block}
\end{frame}

\begin{frame}
  \frametitle{k-Nearest Neighbors (k-NN) - How it Works}
  \begin{enumerate}
    \item \textbf{Data Preparation}:
      \begin{itemize}
        \item Gather a labeled dataset containing features and corresponding labels.
      \end{itemize}
    \item \textbf{Distance Calculation}:
      \begin{itemize}
        \item Compute the distance from the input data point to all other data points.
        \item Common distance metrics include:
          \begin{itemize}
            \item \textbf{Euclidean Distance}: 
            \begin{equation}
            d(p, q) = \sqrt{\sum_{i=1}^{n}(p_i - q_i)^2}
            \end{equation}
            \item \textbf{Manhattan Distance}: 
            \begin{equation}
            d(p, q) = \sum_{i=1}^{n}|p_i - q_i|
            \end{equation}
          \end{itemize}
      \end{itemize}
  \end{enumerate}
\end{frame}

\begin{frame}
  \frametitle{k-Nearest Neighbors (k-NN) - Voting Mechanism}
  \begin{enumerate}[resume]
    \item \textbf{Identifying Neighbors}:
      \begin{itemize}
        \item Identify the 'k' closest data points based on computed distances.
      \end{itemize}
    \item \textbf{Voting Mechanism}:
      \begin{itemize}
        \item For classification, the predicted label is typically the mode of the 'k' neighbors.
        \item For regression, the prediction may be the average of the values of the 'k' nearest neighbors.
      \end{itemize}
    \item \textbf{Choosing 'k'}:
      \begin{itemize}
        \item A smaller 'k' can be noisy while a larger 'k' may dilute the result by including points from other classes.
      \end{itemize}
  \end{enumerate}
\end{frame}

\begin{frame}
  \frametitle{k-Nearest Neighbors (k-NN) - Applications}
  \begin{itemize}
    \item \textbf{Image Recognition}:
      \begin{itemize}
        \item Classifying images based on pixel data by comparing similarities in feature space.
      \end{itemize}
    \item \textbf{Recommendation Systems}:
      \begin{itemize}
        \item Suggesting products based on user preferences similar to others.
      \end{itemize}
    \item \textbf{Medical Diagnosis}:
      \begin{itemize}
        \item Classifying types of diseases based on patient symptoms and medical records.
      \end{itemize}
  \end{itemize}
\end{frame}

\begin{frame}[fragile]
  \frametitle{k-Nearest Neighbors (k-NN) - Example Code}
  \begin{block}{Python Code Snippet}
    \begin{lstlisting}[language=Python]
from sklearn.neighbors import KNeighborsClassifier

# Sample Data
X_train = [[0, 0], [1, 1], [1, 0], [0, 1]]  # Features
y_train = [0, 1, 1, 0]                       # Labels

# Create k-NN Classifier
knn = KNeighborsClassifier(n_neighbors=3)

# Fit Model
knn.fit(X_train, y_train)

# Predict
prediction = knn.predict([[0.5, 0.5]])
print(f"Predicted Class: {prediction}")
    \end{lstlisting}
  \end{block}
\end{frame}

\begin{frame}[fragile]
  \frametitle{Distance Metrics in k-NN}
  
  \begin{block}{Overview of Distance Metrics}
    k-Nearest Neighbors (k-NN) is a classification algorithm that relies on distance metrics to identify nearest neighbors. The choice of distance metric can notably impact performance. Two commonly used metrics are:
    \begin{itemize}
      \item Euclidean Distance
      \item Manhattan Distance
    \end{itemize}
  \end{block}
\end{frame}

\begin{frame}[fragile]
  \frametitle{Euclidean Distance}
  
  \begin{block}{Definition}
    The Euclidean distance measures the straight-line distance between two points in Euclidean space, using the Pythagorean theorem.
  \end{block}

  \begin{equation}
    d_{\text{Euclidean}}(P, Q) = \sqrt{\sum_{i=1}^{n} (p_i - q_i)^2}
  \end{equation}
  
  where \( P = (p_1, p_2, \ldots, p_n) \) and \( Q = (q_1, q_2, \ldots, q_n) \) are two n-dimensional points.

  \begin{itemize}
    \item \textbf{Key Points:}
    \begin{itemize}
      \item Sensitive to scale: Large differences in one dimension can disproportionately impact the distance.
      \item Best for continuous data with a 'natural geometric' relationship.
    \end{itemize}
    
    \item \textbf{Example:}
    \[
    d_{\text{Euclidean}}(P(1, 2), Q(4, 6)) = \sqrt{(1-4)^2 + (2-6)^2} = 5
    \]
  \end{itemize}
\end{frame}

\begin{frame}[fragile]
  \frametitle{Manhattan Distance}
  
  \begin{block}{Definition}
    Manhattan distance measures distance in a grid-based system, summing absolute differences along each dimension.
  \end{block}

  \begin{equation}
    d_{\text{Manhattan}}(P, Q) = \sum_{i=1}^{n} |p_i - q_i|
  \end{equation}

  \begin{itemize}
    \item \textbf{Key Points:}
    \begin{itemize}
      \item Robust against outliers by treating all dimensions equally.
      \item Useful for categorical data or data along axis-aligned paths.
    \end{itemize}
    
    \item \textbf{Example:}
    \[
    d_{\text{Manhattan}}(P(1, 2), Q(4, 6)) = |1-4| + |2-6| = 7
    \]
  \end{itemize}
\end{frame}

\begin{frame}[fragile]
  \frametitle{Impact on Performance}
  
  \begin{itemize}
    \item \textbf{Choice of Metric:} The impact of the chosen metric can vary based on data characteristics.
    \begin{itemize}
      \item \textbf{High-dimensional data:} Euclidean distance may become less effective.
      \item \textbf{Heterogeneous features:} Manhattan distance may perform better with mixed types of data.
    \end{itemize}
    
    \item \textbf{Conclusion:} The choice of distance metric significantly affects classification results. Understanding the characteristics of each metric aids in better model selection, improving accuracy and effectiveness.
  \end{itemize}
\end{frame}

\begin{frame}[fragile]
  \frametitle{Benefits and Challenges of k-NN - Overview}
  
  \begin{itemize}
      \item Popular technique for classification tasks
      \item Intuitive approach
      \item Comes with both benefits and challenges
  \end{itemize}
\end{frame}

\begin{frame}[fragile]
  \frametitle{Benefits of k-NN}
  
  \begin{enumerate}
      \item \textbf{Simplicity}:
      \begin{itemize}
          \item Straightforward and easy to understand
          \item Key steps: distance calculation and majority class classification
      \end{itemize}
      
      \item \textbf{No Training Phase}:
      \begin{itemize}
          \item Works directly with the training dataset
          \item Ready to classify immediately
      \end{itemize}
      
      \item \textbf{Versatile}:
      \begin{itemize}
          \item Applicable for both classification and regression tasks
      \end{itemize}
      
      \item \textbf{Effectiveness with Large Datasets}:
      \begin{itemize}
          \item Performs well with sufficient labeled data
      \end{itemize}
  \end{enumerate}
\end{frame}

\begin{frame}[fragile]
  \frametitle{Challenges of k-NN}
  
  \begin{enumerate}
      \item \textbf{Computationally Expensive}:
      \begin{itemize}
          \item Requires distance calculations for all training points
          \item Slower predictions with large datasets
          \item \textit{Example:} For 10,000 points and $k=5$, compute distance to all 10,000 for each new point.
      \end{itemize}
      
      \item \textbf{Memory Intensive}:
      \begin{itemize}
          \item Stores the entire training dataset, requiring significant memory
      \end{itemize}
      
      \item \textbf{Sensitivity to Irrelevant Features}:
      \begin{itemize}
          \item Performance can degrade with irrelevant features
          \item \textit{Example:} Adding "favorite color" in a dataset of personal attributes can confuse the algorithm.
      \end{itemize}
      
      \item \textbf{Curse of Dimensionality}:
      \begin{itemize}
          \item As features increase, distance becomes less meaningful
          \item Classification becomes harder in high-dimensional spaces
          \item \textit{Illustration:} Points closer in 2D may become indistinguishable in higher dimensions.
      \end{itemize}
  \end{enumerate}
\end{frame}

\begin{frame}[fragile]
    \frametitle{Comparison of Classification Techniques - Overview}
    In this slide, we compare three popular classification techniques: 
    \textbf{Decision Trees}, \textbf{Naive Bayes}, and \textbf{k-Nearest Neighbors (k-NN)}. 
    The comparison focuses on key criteria including:
    \begin{itemize}
        \item \textbf{Accuracy}
        \item \textbf{Speed}
        \item \textbf{Interpretability}
    \end{itemize}
    This provides a clearer understanding of when to use each method.
\end{frame}

\begin{frame}[fragile]
    \frametitle{Comparison of Classification Techniques - Decision Trees}
    \begin{itemize}
        \item \textbf{Definition}: A tree-like model used to make decisions based on a series of questions.
        \item \textbf{Accuracy}: Performs well in capturing complex patterns but can overfit to noise if too deep.
        \item \textbf{Speed}: Moderate; training time varies by tree depth, fast for predictions.
        \item \textbf{Interpretability}: Highly interpretable; the tree structure is easily visualized.
    \end{itemize}
    \textbf{Example}: A Decision Tree classifying a product purchase might ask:
    \begin{itemize}
        \item Is age > 30?
        \item Yes $\rightarrow$ Next question: Is income > \$50,000?
    \end{itemize}
\end{frame}

\begin{frame}[fragile]
    \frametitle{Comparison of Classification Techniques - Naive Bayes and k-NN}
    \textbf{Naive Bayes}
    \begin{itemize}
        \item \textbf{Definition}: A probabilistic classifier based on Bayes’ Theorem, assuming feature independence.
        \item \textbf{Accuracy}: Performs well for large datasets, especially effective in text classification.
        \item \textbf{Speed}: Very fast in training and prediction.
        \item \textbf{Interpretability}: Somewhat interpretable; probabilities are understandable, independence may not always hold.
    \end{itemize}
    
    \textbf{Example}: In spam detection, it calculates likelihood based on word frequency (e.g., "free" and "win").
    
    \vspace{1em}
    \textbf{k-Nearest Neighbors (k-NN)}
    \begin{itemize}
        \item \textbf{Definition}: An instance-based learning algorithm that classifies based on the majority class of k-nearest samples.
        \item \textbf{Accuracy}: Can yield high accuracy but performance deteriorates with irrelevant features.
        \item \textbf{Speed}: Slow during prediction, negligible training time.
        \item \textbf{Interpretability}: Less interpretable; complex in high-dimensional spaces.
    \end{itemize}
    
    \textbf{Example}: To classify a fruit, if nearest neighbors are mainly apples, it classifies as an apple.
\end{frame}

\begin{frame}[fragile]
    \frametitle{Comparison of Classification Techniques - Key Points & Summary}
    \begin{block}{Key Points to Emphasize}
        \begin{itemize}
            \item Use \textbf{Decision Trees} for easy interpretation and visual relationships.
            \item Choose \textbf{Naive Bayes} for speed in text classification tasks.
            \item Opt for \textbf{k-NN} for accurate predictions when the dataset is well-conditioned.
        \end{itemize}
    \end{block}
    
    \vspace{1em}
    \begin{tabular}{|l|l|l|l|}
        \hline
        \textbf{Technique} & \textbf{Accuracy} & \textbf{Speed} & \textbf{Interpretability} \\
        \hline
        Decision Trees & High (risk of overfitting) & Moderate (depends on depth) & High (visual structure) \\
        \hline
        Naive Bayes & Good (especially for text) & Very Fast & Moderate (probabilistic) \\
        \hline
        k-NN & Variable (depends on k) & Slow (during prediction) & Low (complexity in high dimensions) \\
        \hline
    \end{tabular}
    
    \vspace{1em}
    By understanding these distinctions, we can select the most appropriate classification technique for various applications.
\end{frame}

\begin{frame}[fragile]
    \frametitle{Applications of Classification Techniques - Introduction}
    Classification techniques are vital in a range of industries, offering practical solutions to complex problems. This slide explores how classification methods—like Decision Trees, Naive Bayes, and k-Nearest Neighbors (k-NN)—are used in finance, healthcare, and marketing.
\end{frame}

\begin{frame}[fragile]
    \frametitle{Applications - Finance}
    \begin{itemize}
        \item \textbf{Credit Scoring}: Financial institutions use classification algorithms to predict loan applicants' likelihood of default.
        \begin{itemize}
            \item Models such as logistic regression or decision trees analyze data (e.g., income, credit history, employment status).
            \item Classifies applicants into "low-risk" or "high-risk" categories.
        \end{itemize}
        \item \textbf{Example}: A model predicts the likelihood of default based on past behavior; applicants with similar profiles can be grouped for risk assessments.
    \end{itemize}
\end{frame}

\begin{frame}[fragile]
    \frametitle{Applications - Healthcare}
    \begin{itemize}
        \item \textbf{Disease Diagnosis}: Classification techniques help diagnose diseases using patient data.
        \begin{itemize}
            \item Models analyze symptoms, medical history, and lab results to classify patients into categories such as "healthy," "needs monitoring," or "requires immediate treatment."
        \end{itemize}
        \item \textbf{Example}: A Naive Bayes classifier identifies whether a patient has diabetes based on features like age, weight, and blood sugar level.
    \end{itemize}
\end{frame}

\begin{frame}[fragile]
    \frametitle{Applications - Marketing}
    \begin{itemize}
        \item \textbf{Customer Segmentation}: Companies use classification to categorize customers based on behavior, preferences, and demographics.
        \begin{itemize}
            \item This helps in targeting marketing campaigns effectively.
        \end{itemize}
        \item \textbf{Example}: Using k-NN, an e-commerce platform classifies customers into "frequent buyers," "occasional buyers," or "non-buyers."
    \end{itemize}
\end{frame}

\begin{frame}[fragile]
    \frametitle{Key Points to Emphasize}
    \begin{itemize}
        \item \textbf{Importance of Accuracy}: In finance, small errors can lead to significant implications; rigorous testing and validation are essential.
        \item \textbf{Interpretability}: Decision Trees are favored in healthcare due to their easy-to-understand visualizations.
        \item \textbf{Adaptive Models}: Classification techniques improve as new data becomes available, enabling responsive decision-making.
    \end{itemize}
\end{frame}

\begin{frame}[fragile]
    \frametitle{Formula Recap - Confusion Matrix}
    A tool to evaluate the performance of a classification model, helping visualize True Positives (TP), True Negatives (TN), False Positives (FP), and False Negatives (FN).

    \begin{center}
        \textbf{Confusion Matrix Layout}:
        \begin{equation}
        \begin{array}{|c|c|c|}
        \hline
        & \text{Actual Positive} & \text{Actual Negative} \\
        \hline
        \text{Predicted Positive} & TP & FP \\
        \hline
        \text{Predicted Negative} & FN & TN \\
        \hline
        \end{array}
        \end{equation}
    \end{center}
\end{frame}

\begin{frame}[fragile]
    \frametitle{Conclusion and Next Steps}
    Classification techniques are powerful tools applied across various fields to tackle real-world challenges. Understanding these applications paves the way for practical implementations and informed decision-making.

    \textbf{Prepare for Next Slide}: 
    In the upcoming lab session, you will gain hands-on experience implementing classification algorithms using Python, reinforcing these concepts through practical applications.
\end{frame}

\begin{frame}[fragile]
    \frametitle{Hands-on Lab Session}
    \begin{block}{Overview}
        In this hands-on lab session, we will implement the classification algorithms discussed in previous lectures using Python. 
        The goal is to apply theoretical knowledge in a practical context and understand the workings of these algorithms through coding.
    \end{block}
\end{frame}

\begin{frame}
    \frametitle{Learning Objectives}
    \begin{itemize}
        \item Acquire hands-on experience with classification techniques.
        \item Understand data preprocessing and model training in Python.
        \item Evaluate the performance of classification models.
        \item Gain familiarity with popular libraries: Scikit-learn, Pandas, and Matplotlib.
    \end{itemize}
\end{frame}

\begin{frame}[fragile]
    \frametitle{Concepts to Cover}
    \begin{enumerate}
        \item \textbf{Data Preparation}
        \begin{itemize}
            \item \textbf{Loading Data}: Use datasets from Scikit-learn or import with Pandas.
            \item \textbf{Exploratory Data Analysis (EDA)}: Visualize data, identify missing values, generate summary statistics.
        \end{itemize}
        
        \item \textbf{Implementing Classification Algorithms}
        \begin{itemize}
            \item \textbf{Choose an Algorithm}: Examples include:
            \begin{itemize}
                \item Logistic Regression
                \item Decision Trees
                \item Support Vector Machines (SVM)
                \item K-Nearest Neighbors (KNN)
            \end{itemize}
            \item \textbf{Code Snippet}:
            \begin{lstlisting}[language=Python]
from sklearn.datasets import load_iris
from sklearn.model_selection import train_test_split
from sklearn.ensemble import RandomForestClassifier
from sklearn import metrics

# Load dataset
data = load_iris()
X = data.data
y = data.target

# Split the dataset into training and testing sets
X_train, X_test, y_train, y_test = train_test_split(X, y, test_size=0.3, random_state=42)

# Create a Random Forest Classifier
model = RandomForestClassifier()

# Train the model
model.fit(X_train, y_train)

# Predictions
y_pred = model.predict(X_test)

# Evaluate the model
accuracy = metrics.accuracy_score(y_test, y_pred)
print(f'Accuracy: {accuracy}')
            \end{lstlisting}
        \end{itemize}
        
        \item \textbf{Model Evaluation}
        \begin{itemize}
            \item Measure accuracy, precision, recall, and F1-score.
            \item Use confusion matrix for detailed performance insights.
            \item \textbf{Code to Evaluate Model}:
            \begin{lstlisting}[language=Python]
from sklearn.metrics import confusion_matrix, classification_report

# Confusion Matrix
cm = confusion_matrix(y_test, y_pred)
print("Confusion Matrix:\n", cm)

# Classification Report
report = classification_report(y_test, y_pred)
print("Classification Report:\n", report)
            \end{lstlisting}
        \end{itemize}
    \end{enumerate}
\end{frame}

\begin{frame}
    \frametitle{Key Points to Emphasize}
    \begin{itemize}
        \item \textbf{Data Preprocessing is Critical}: Ensure the data is clean before model training.
        \item \textbf{Model Selection}: The choice of algorithm significantly affects performance; understand the principles behind each.
        \item \textbf{Evaluation Metrics}: Utilize various metrics to assess model quality; accuracy alone may not suffice.
    \end{itemize}
\end{frame}

\begin{frame}
    \frametitle{Conclusion and Next Steps}
    By the end of this lab session, students will be equipped with practical skills to implement and evaluate classification algorithms in Python. 
    The hands-on experience reinforces theoretical concepts discussed earlier.

    \textbf{Next Steps}:
    After completing this lab, we will have a recap followed by a Q\&A session to clarify concepts and address any uncertainties you may have.
    
    \begin{block}{Reminder}
        Feel free to ask questions during the lab, and remember, the best way to solidify your understanding is through practice!
    \end{block}
\end{frame}

\begin{frame}[fragile]
    \frametitle{Conclusion and Q\&A}
    Recap of key points covered on classification techniques and an open floor for questions.
\end{frame}

\begin{frame}[fragile]
    \frametitle{Conclusion of Classification Techniques}
    
    \begin{block}{Key Points Recap}
        \begin{enumerate}
            \item \textbf{Definition of Classification}:
                \begin{itemize}
                    \item Classification is a supervised learning technique to categorize data into predefined classes or labels.
                    \item The key goal is to identify the class of new observations based on past data.
                \end{itemize}
        \end{enumerate}
    \end{block}
\end{frame}

\begin{frame}[fragile]
    \frametitle{Common Classification Algorithms}
    
    \begin{block}{Key Algorithms}
        \begin{itemize}
            \item \textbf{Logistic Regression}: Estimates probabilities using a logistic function; best for binary outcomes.
            \item \textbf{Decision Trees}: Splits data into branches to make decisions; intuitive and easy to visualize.
            \item \textbf{Support Vector Machines (SVM)}: Finds the optimal hyperplane to separate classes in high-dimensional space.
            \item \textbf{K-Nearest Neighbors (KNN)}: Classifies based on the closest training examples; non-parametric method.
            \item \textbf{Random Forests}: Ensemble method combining multiple decision trees to improve accuracy and prevent overfitting.
            \item \textbf{Neural Networks}: Simulates human brain function; powerful for complex patterns and large datasets.
        \end{itemize}
    \end{block}
\end{frame}

\begin{frame}[fragile]
    \frametitle{Evaluation Metrics and Best Practices}

    \begin{block}{Evaluation Metrics}
        \begin{itemize}
            \item \textbf{Accuracy}: \( \frac{TP + TN}{TP + TN + FP + FN} \)
            \item \textbf{Precision}: \( \frac{TP}{TP + FP} \) – Measures correctness of positive predictions.
            \item \textbf{Recall (Sensitivity)}: \( \frac{TP}{TP + FN} \) – Measures the ability to find all relevant instances.
            \item \textbf{F1 Score}: \( 2 \times \frac{Precision \times Recall}{Precision + Recall} \) – Harmonic mean of precision and recall.
        \end{itemize}
    \end{block}

    \begin{block}{Best Practices}
        \begin{itemize}
            \item Pre-process data (handle missing values, scaling).
            \item Split your dataset into training, validation, and test sets to ensure model generalization.
            \item Utilize cross-validation to assess model performance on unseen data.
            \item Tune hyperparameters to enhance model efficacy (e.g., using Grid Search).
        \end{itemize}
    \end{block}
\end{frame}

\begin{frame}[fragile]
    \frametitle{Practical Application and Open Questions}

    \begin{block}{Practical Application}
        During our \textbf{Hands-on Lab Session}, we applied these classification techniques using Python libraries like scikit-learn to implement algorithms and evaluate their performance.
    \end{block}

    \begin{block}{Open Floor for Questions}
        Please feel free to ask clarifying questions or share your experiences with classification. Consider discussing challenges from the lab session or specific scenarios where different algorithms may apply.
    \end{block}
\end{frame}


\end{document}