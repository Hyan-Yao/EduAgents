\documentclass{beamer}

% Theme choice
\usetheme{Madrid} % You can change to e.g., Warsaw, Berlin, CambridgeUS, etc.

% Encoding and font
\usepackage[utf8]{inputenc}
\usepackage[T1]{fontenc}

% Graphics and tables
\usepackage{graphicx}
\usepackage{booktabs}

% Code listings
\usepackage{listings}
\lstset{
basicstyle=\ttfamily\small,
keywordstyle=\color{blue},
commentstyle=\color{gray},
stringstyle=\color{red},
breaklines=true,
frame=single
}

% Math packages
\usepackage{amsmath}
\usepackage{amssymb}

% Colors
\usepackage{xcolor}

% TikZ and PGFPlots
\usepackage{tikz}
\usepackage{pgfplots}
\pgfplotsset{compat=1.18}
\usetikzlibrary{positioning}

% Hyperlinks
\usepackage{hyperref}

% Title information
\title{Chapter 6: Anomaly Detection}
\author{Your Name}
\institute{Your Institution}
\date{\today}

\begin{document}

\frame{\titlepage}

\begin{frame}[fragile]
    \frametitle{Introduction to Anomaly Detection}
    \begin{block}{Overview}
        Anomaly detection, also known as outlier detection, identifies unexpected items or events in a dataset. 
        It is critical for detecting:
        \begin{itemize}
            \item Fraud
            \item Security breaches
            \item Equipment failures
        \end{itemize}
    \end{block}
\end{frame}

\begin{frame}[fragile]
    \frametitle{Key Concepts of Anomaly Detection}
    \begin{enumerate}
        \item \textbf{Definition}: Identifying data points that significantly deviate from the majority. 
        \item \textbf{Relevance}:
        \begin{itemize}
            \item \textbf{Data Integrity}: Enhances data quality by identifying errors or fraud.
            \item \textbf{Predictive Maintenance}: Forecasts equipment malfunctions.
            \item \textbf{Fraud Detection}: Spots unusual patterns in transactions.
        \end{itemize}
    \end{enumerate}
\end{frame}

\begin{frame}[fragile]
    \frametitle{Real-World Applications}
    \begin{enumerate}
        \item \textbf{Financial Sector}:
        \begin{itemize}
            \item Example: Sudden high-value transactions from unusual locations trigger alerts.
        \end{itemize}
        \item \textbf{Healthcare}:
        \begin{itemize}
            \item Example: A spike in a patient's heart rate may indicate an emergency.
        \end{itemize}
        \item \textbf{Manufacturing}:
        \begin{itemize}
            \item Example: Unusual temperature readings may signal machinery malfunction.
        \end{itemize}
        \item \textbf{Network Security}:
        \begin{itemize}
            \item Example: Sudden surges in traffic could indicate a DDoS attack.
        \end{itemize}
    \end{enumerate}
\end{frame}

\begin{frame}[fragile]
    \frametitle{Key Points to Emphasize}
    \begin{itemize}
        \item Anomaly detection is essential for proactive decision-making.
        \item Various algorithms like statistical tests, clustering, and machine learning can be employed.
        \item Understanding the normal behavior of systems is crucial for effective anomaly detection.
    \end{itemize}
\end{frame}

\begin{frame}[fragile]
    \frametitle{Conclusion and Key Formula}
    Conclusion: Anomaly detection is vital for maintaining integrity and security across industries.

    \begin{block}{Z-Score Method}
        An anomaly is identified if:
        \begin{equation}
        Z = \frac{(X - \mu)}{\sigma}
        \end{equation}
        Where:
        \begin{itemize}
            \item \(X\) = a single data point
            \item \(\mu\) = mean of the dataset
            \item \(\sigma\) = standard deviation of the dataset
        \end{itemize}
    \end{block}
\end{frame}

\begin{frame}[fragile]
    \frametitle{What is Anomaly Detection? - Definition}
    Anomaly detection is a data mining technique that identifies patterns in data that do not conform to expected behavior. 
    \begin{itemize}
        \item Anomalies are unusual data points that can provide critical insights.
        \item They often indicate errors, fraud, or significant changes in underlying processes.
    \end{itemize}
\end{frame}

\begin{frame}[fragile]
    \frametitle{What is Anomaly Detection? - Importance}
    \begin{enumerate}
        \item \textbf{Insight Extraction:} 
            \begin{itemize}
                \item Reveals valuable insights into system behavior and unrecognized patterns.
            \end{itemize}
        \item \textbf{Fraud Detection:} 
            \begin{itemize}
                \item Critical in financial sectors to identify fraudulent transactions.
            \end{itemize}
        \item \textbf{Quality Control:} 
            \begin{itemize}
                \item Ensures product quality by identifying faults in manufacturing processes.
            \end{itemize}
        \item \textbf{Network Security:}
            \begin{itemize}
                \item Helps to identify potential security breaches or cyber-attacks through traffic monitoring.
            \end{itemize}
        \item \textbf{Healthcare Monitoring:}
            \begin{itemize}
                \item Identifies sudden changes in patient conditions for timely interventions.
            \end{itemize}
    \end{enumerate}
\end{frame}

\begin{frame}[fragile]
    \frametitle{What is Anomaly Detection? - Example and Summary}
    \textbf{Illustrative Example:} 
    Imagine a bank monitoring credit card transactions. 
    \begin{itemize}
        \item Normal monthly purchases: \$500. 
        \item Anomaly: Multiple transactions totaling \$20,000 in a single day.
        \item Flagged for investigation to prevent fraud.
    \end{itemize}
    
    \textbf{Summary:}
    Anomaly detection is a powerful tool within data mining that enhances understanding of data behavior and informs proactive decision-making across industries.
\end{frame}

\begin{frame}[fragile]
    \frametitle{Types of Anomalies - Overview}
    \begin{block}{Anomaly Definition in Context}
        Anomalies, or outliers, are observations that deviate significantly from the expected pattern in data. Understanding the types of anomalies is crucial in selecting appropriate detection techniques. We will explore three primary types:
        \begin{itemize}
            \item Point Anomalies
            \item Contextual Anomalies
            \item Collective Anomalies
        \end{itemize}
    \end{block}
\end{frame}

\begin{frame}[fragile]
    \frametitle{Types of Anomalies - Point Anomalies}
    \begin{block}{1. Point Anomalies}
        \begin{itemize}
            \item \textbf{Definition}: A point anomaly refers to a single data instance that is radically different from the rest of the data.
            \item \textbf{Example}: A user logging in from an unusual geographic location (e.g., a country they have never accessed before).
            \item \textbf{Key Points}:
            \begin{itemize}
                \item Simple to detect using statistical thresholds (e.g., Z-score, IQR).
                \item Commonly used in fraud detection and network security.
            \end{itemize}
        \end{itemize}
    \end{block}
\end{frame}

\begin{frame}[fragile]
    \frametitle{Types of Anomalies - Contextual and Collective Anomalies}
    \begin{block}{2. Contextual Anomalies}
        \begin{itemize}
            \item \textbf{Definition}: Occur when a data point is considered anomalous in a specific context but may be normal in another context.
            \item \textbf{Example}: A temperature reading of 30°C in winter vs. summer; acceptable in summer but anomalous in winter.
            \item \textbf{Key Points}:
            \begin{itemize}
                \item Requires contextual information to ascertain normalcy (e.g., time, location).
                \item Arises in time-series analysis and environmental monitoring.
            \end{itemize}
        \end{itemize}
    \end{block}
    
    \begin{block}{3. Collective Anomalies}
        \begin{itemize}
            \item \textbf{Definition}: Appear as a group of data points that together exhibit an abnormal pattern.
            \item \textbf{Example}: Sudden spikes in web traffic suggesting a DDoS attack, where individual spikes are less significant.
            \item \textbf{Key Points}:
            \begin{itemize}
                \item Useful in detecting patterns of behavior (e.g., network traffic analysis).
                \item Often analyzed using methods like clustering or sequence analysis.
            \end{itemize}
        \end{itemize}
    \end{block}
\end{frame}

\begin{frame}[fragile]
    \frametitle{Summary of Anomalies}
    \begin{block}{Summary}
        Understanding the types of anomalies is essential for effective anomaly detection:
        \begin{itemize}
            \item \textbf{Point Anomalies}: Individual deviations.
            \item \textbf{Contextual Anomalies}: Depend on the surrounding context.
            \item \textbf{Collective Anomalies}: Arise from a group behavior pattern.
        \end{itemize}
        By identifying and categorizing anomalies, we can apply suitable detection techniques.
    \end{block}
\end{frame}

\begin{frame}[fragile]
    \frametitle{Further Exploration}
    \begin{block}{Next Steps}
        Familiarize yourself with techniques used to detect these anomalies, as we will discuss in the next slide, \textit{Techniques for Anomaly Detection}.
    \end{block}
\end{frame}

\begin{frame}[fragile]
    \frametitle{Chapter 6: Anomaly Detection}
    \frametitle{Techniques for Anomaly Detection}
    Anomaly detection identifies unusual patterns that do not conform to expected behavior within a dataset. 
    It is essential in various fields, such as fraud detection, network security, and industrial monitoring. 
    We will explore three primary techniques:
    \begin{itemize}
        \item Statistical Methods
        \item Machine Learning Approaches
        \item Hybrid Methods
    \end{itemize}
\end{frame}

\begin{frame}[fragile]
    \frametitle{1. Statistical Methods}
    Statistical methods leverage mathematical concepts to identify anomalies based on statistical characteristics.

    \begin{block}{Key Concepts}
        \begin{itemize}
            \item \textbf{Assumption of Normal Distribution:} Many methods assume data follows a normal distribution.
            \item \textbf{Outlier Detection:} Points significantly outside the mean are deemed anomalies.
        \end{itemize}
    \end{block}

    \begin{block}{Examples}
        \begin{itemize}
            \item \textbf{Z-Score:} Indicates how many standard deviations a data point is from the mean.
            \begin{equation}
                Z = \frac{(X - \mu)}{\sigma}
            \end{equation}
            Where:
            \begin{itemize}
                \item \(X\) = data point
                \item \(\mu\) = mean of dataset
                \item \(\sigma\) = standard deviation of dataset
            \end{itemize}
            
            \item \textbf{Interquartile Range (IQR):} Measures the spread of the middle 50\% of data.
            \begin{equation}
                IQR = Q3 - Q1 
            \end{equation}
            Points below \(Q1 - 1.5 \times IQR\) or above \(Q3 + 1.5 \times IQR\) are considered outliers.
        \end{itemize}
    \end{block}
\end{frame}

\begin{frame}[fragile]
    \frametitle{2. Machine Learning Approaches}
    Machine learning techniques learn from data and adapt to new patterns over time.

    \begin{block}{Key Concepts}
        \begin{itemize}
            \item \textbf{Supervised Learning:} Uses labeled data with known anomalies.
            \item \textbf{Unsupervised Learning:} Detects anomalies without labeled data based on patterns.
        \end{itemize}
    \end{block}

    \begin{block}{Examples}
        \begin{itemize}
            \item \textbf{Isolation Forest:} Constructs trees that isolate observations; anomalies are isolated quicker.
            \item \textbf{Support Vector Machine (SVM):} Finds a hyperplane separating normal data from anomalies.
        \end{itemize}
    \end{block}
\end{frame}

\begin{frame}[fragile]
    \frametitle{3. Hybrid Methods}
    Hybrid methods combine statistical and machine learning techniques for improved anomaly detection.

    \begin{block}{Key Concepts}
        \begin{itemize}
            \item \textbf{Flexibility in Detection:} Combines strengths of both approaches to adapt to different datasets.
            \item \textbf{Integration of Domain Knowledge:} Incorporates specific heuristics for enhanced accuracy.
        \end{itemize}
    \end{block}

    \begin{block}{Examples}
        \begin{itemize}
            \item \textbf{Statistical Features with Machine Learning:} Using statistical measures as features enhances detection.
            \item \textbf{Ensemble Methods:} Combining predictions from multiple models improves detection rates.
        \end{itemize}
    \end{block}
\end{frame}

\begin{frame}[fragile]
    \frametitle{Key Points to Remember}
    \begin{itemize}
        \item Different techniques serve various needs based on the type of anomalies and dataset nature.
        \item Statistical methods are simpler and more interpretable, while machine learning approaches handle complex datasets.
        \item Hybrid methods leverage both approaches for more robust systems.
    \end{itemize}
    By understanding these techniques, you can select the most appropriate one for your anomaly detection needs. More detailed discussions will follow in subsequent slides.
\end{frame}

\begin{frame}[fragile]
    \frametitle{Statistical Methods - Overview}
    \begin{block}{Overview of Statistical Anomaly Detection Techniques}
        In anomaly detection, statistical methods are fundamental techniques that help identify outliers or anomalies in data. The two commonly used statistical methods include \textbf{Z-Score} and \textbf{Interquartile Range (IQR)}.
    \end{block}
\end{frame}

\begin{frame}[fragile]
    \frametitle{Statistical Methods - Z-Score}
    \begin{block}{1. Z-Score Method}
        \begin{itemize}
            \item \textbf{Concept}: Measures how many standard deviations a data point is from the mean.
            \item \textbf{Formula}:
            \begin{equation}
            Z = \frac{(X - \mu)}{\sigma}
            \end{equation}
            Where:
            \begin{itemize}
                \item $X$ = data point
                \item $\mu$ = mean of the dataset
                \item $\sigma$ = standard deviation of the dataset
            \end{itemize}
            \item \textbf{Interpretation}: 
            \begin{itemize}
                \item A Z-Score greater than 3 or less than -3 indicates an outlier.
            \end{itemize}
            \item \textbf{Example}:
            \begin{itemize}
                \item Dataset: [70, 75, 80, 85, 90, 100]
                \item Mean ($\mu$): 83.33, Standard Deviation ($\sigma$): 10.41
                \item Calculate Z-Score for 100:
                \begin{equation}
                Z = \frac{(100 - 83.33)}{10.41} \approx 1.60
                \end{equation}
                \item Since 1.60 is not greater than 3, it may not be an anomaly.
            \end{itemize}
        \end{itemize}
    \end{block}
\end{frame}

\begin{frame}[fragile]
    \frametitle{Statistical Methods - IQR}
    \begin{block}{2. Interquartile Range (IQR) Method}
        \begin{itemize}
            \item \textbf{Concept}: Focuses on the spread of the middle 50\% of the data and identifies anomalies based on distance from the interquartile range.
            \item \textbf{Calculation}:
            \begin{enumerate}
                \item Compute Q1 (25th percentile) and Q3 (75th percentile).
                \item Calculate IQR:
                \begin{equation}
                \text{IQR} = Q3 - Q1
                \end{equation}
                \item Determine the lower and upper boundaries:
                \begin{equation}
                \text{Lower Bound} = Q1 - 1.5 \times \text{IQR}
                \end{equation}
                \begin{equation}
                \text{Upper Bound} = Q3 + 1.5 \times \text{IQR}
                \end{equation}
            \end{enumerate}
            \item \textbf{Example}:
            \begin{itemize}
                \item Dataset: [2, 6, 7, 10, 12, 15, 19]
                \item Q1: 6, Q3: 12, IQR: 6
                \item Lower Bound: $6 - (1.5 \times 6) = -3$
                \item Upper Bound: $12 + (1.5 \times 6) = 21$
                \item Data points outside this range are considered anomalies.
            \end{itemize}
        \end{itemize}
    \end{block}
\end{frame}

\begin{frame}[fragile]
    \frametitle{Machine Learning Approaches}
    \begin{block}{Overview of Anomaly Detection}
        Anomaly detection identifies rare items or events in data that differ significantly from the majority. This presentation covers three prominent algorithms:
        \begin{itemize}
            \item Isolation Forest
            \item Support Vector Machine (SVM)
            \item Neural Networks
        \end{itemize}
    \end{block}
\end{frame}

\begin{frame}[fragile]
    \frametitle{Isolation Forest}
    \begin{block}{Concept}
        The Isolation Forest isolates observations in a dataset, assuming anomalies are easier to isolate than normal instances.
    \end{block}
    \begin{block}{How it Works}
        \begin{itemize}
            \item Builds an ensemble of isolation trees by selecting a feature and a split value.
            \item Anomalies have shorter average path lengths since they are rare.
        \end{itemize}
    \end{block}
    \begin{block}{Key Points}
        \begin{itemize}
            \item Efficient for large datasets.
            \item Non-parametric; no assumption of underlying distribution.
        \end{itemize}
    \end{block}
\end{frame}

\begin{frame}[fragile]
    \frametitle{Isolation Forest - Example}
    \begin{block}{Example}
        In a fraud detection system:
        \begin{itemize}
            \item Normal transactions cluster densely.
            \item Fraudulent ones are isolated, resulting in short path lengths.
        \end{itemize}
    \end{block}
\end{frame}

\begin{frame}[fragile]
    \frametitle{Support Vector Machine (SVM)}
    \begin{block}{Concept}
        SVM is a supervised learning algorithm adapted for anomaly detection using One-class SVM to identify normal observation boundaries.
    \end{block}
    \begin{block}{How it Works}
        \begin{itemize}
            \item Locates a hyperplane that separates normal data from potential outliers.
            \item Outliers are any points outside the defined boundary.
        \end{itemize}
    \end{block}
    \begin{block}{Key Points}
        \begin{itemize}
            \item Sensitive to the choice of kernel function (linear, polynomial, RBF).
            \item Effective in high-dimensional spaces.
        \end{itemize}
    \end{block}
\end{frame}

\begin{frame}[fragile]
    \frametitle{SVM - Example}
    \begin{block}{Example}
        In network intrusion detection:
        \begin{itemize}
            \item SVM models classify normal traffic patterns.
            \item Deviations are marked as potential intrusions.
        \end{itemize}
    \end{block}
\end{frame}

\begin{frame}[fragile]
    \frametitle{Neural Networks}
    \begin{block}{Concept}
        Deep learning models, such as autoencoders and recurrent neural networks, learn complex data patterns for anomaly detection.
    \end{block}
    \begin{block}{How it Works}
        \begin{itemize}
            \item \textbf{Autoencoders:} Compress input into lower-dimensional space to detect anomalies based on high reconstruction error.
            \item \textbf{Recurrent Neural Networks (RNNs):} Capture temporal patterns for detecting anomalies in sequential data.
        \end{itemize}
    \end{block}
    \begin{block}{Key Points}
        \begin{itemize}
            \item Capable of capturing intricate nonlinear relationships.
            \item Requires substantial amounts of labeled training data.
        \end{itemize}
    \end{block}
\end{frame}

\begin{frame}[fragile]
    \frametitle{Neural Networks - Example}
    \begin{block}{Example}
        In monitoring industrial equipment:
        \begin{itemize}
            \item Autoencoders learn normal operating conditions.
            \item Significant deviations are flagged as anomalies.
        \end{itemize}
    \end{block}
\end{frame}

\begin{frame}[fragile]
    \frametitle{Conclusion}
    \begin{block}{Key Takeaways}
        \begin{itemize}
            \item \textbf{Isolation Forest}: Quick and effective for large datasets.
            \item \textbf{SVM}: Powerful boundary-based method for clearly defined classes.
            \item \textbf{Neural Networks}: Highly flexible and capable of learning complex representations, but requires more data.
        \end{itemize}
    \end{block}
    By understanding these methodologies, we can effectively employ anomaly detection techniques tailored to our data characteristics.
\end{frame}

\begin{frame}[fragile]
    \frametitle{Additional Resources}
    \begin{block}{Formulas}
        \begin{equation}
            \text{Error} = ||X - \hat{X}||^2
        \end{equation}
        Reconstruction error for autoencoders.
    \end{block}
    \begin{block}{Code Snippet}
        \begin{lstlisting}[language=Python]
        from sklearn.ensemble import IsolationForest
        model = IsolationForest()
        model.fit(data)
        predictions = model.predict(data)
        \end{lstlisting}
    \end{block}
\end{frame}

\begin{frame}[fragile]
    \frametitle{Evaluation Metrics - Introduction}
    \begin{block}{Understanding Evaluation Metrics in Anomaly Detection}
        To effectively evaluate the performance of anomaly detection algorithms, we use several key metrics: Precision, Recall, F1-Score, and ROC-AUC. Each of these metrics serves to indicate how well the model identifies anomalies compared to non-anomalies.
    \end{block}
\end{frame}

\begin{frame}[fragile]
    \frametitle{Evaluation Metrics - Precision and Recall}
    \begin{block}{1. Precision}
        \begin{itemize}
            \item \textbf{Definition:} Measures the accuracy of positive predictions. It calculates the proportion of correctly identified anomalies out of all instances flagged as anomalies.
            \item \textbf{Formula:}  
            \[
            \text{Precision} = \frac{\text{True Positives (TP)}}{\text{True Positives (TP)} + \text{False Positives (FP)}}
            \]
            \item \textbf{Example:} If a model predicts 100 anomalies, but only 80 are actual anomalies, then:
            \[
            \text{Precision} = \frac{80}{80 + 20} = 0.80 \text{ or } 80\%
            \]
        \end{itemize}
    \end{block}

    \begin{block}{2. Recall (Sensitivity)}
        \begin{itemize}
            \item \textbf{Definition:} Indicates the model's ability to identify all relevant instances. It measures the proportion of actual anomalies correctly identified.
            \item \textbf{Formula:}  
            \[
            \text{Recall} = \frac{\text{True Positives (TP)}}{\text{True Positives (TP)} + \text{False Negatives (FN)}}
            \]
            \item \textbf{Example:} If there are 100 actual anomalies and the model identifies 80, then:
            \[
            \text{Recall} = \frac{80}{80 + 20} = 0.80 \text{ or } 80\%
            \]
        \end{itemize}
    \end{block}
\end{frame}

\begin{frame}[fragile]
    \frametitle{Evaluation Metrics - F1-Score and ROC-AUC}
    \begin{block}{3. F1-Score}
        \begin{itemize}
            \item \textbf{Definition:} The harmonic mean of Precision and Recall, useful for imbalanced classes.
            \item \textbf{Formula:}  
            \[
            \text{F1-Score} = 2 \times \frac{\text{Precision} \times \text{Recall}}{\text{Precision} + \text{Recall}}
            \]
            \item \textbf{Example:} If Precision = 0.80 and Recall = 0.80, then:
            \[
            \text{F1-Score} = 2 \times \frac{0.80 \times 0.80}{0.80 + 0.80} = 0.80
            \]
        \end{itemize}
    \end{block}

    \begin{block}{4. ROC-AUC}
        \begin{itemize}
            \item \textbf{Definition:} The ROC curve plots the true positive rate against the false positive rate at various thresholds. AUC evaluates the model's discrimination ability.
            \item \textbf{Interpretation:}
            \begin{itemize}
                \item AUC = 1: Perfect model
                \item AUC = 0.5: No discriminative power
                \item AUC < 0.5: Worse than random chance
            \end{itemize}
            \item \textbf{Example:} A model with an AUC of 0.85 indicates strong ability to distinguish anomalies from normal data.
        \end{itemize}
    \end{block}
\end{frame}

\begin{frame}[fragile]
    \frametitle{Use Cases of Anomaly Detection}
    
    \begin{block}{Understanding Anomaly Detection}
        Anomaly detection refers to identifying patterns in data that do not conform to expected behavior. This is crucial across various domains where it is essential to flag unusual data points, which may indicate significant events or errors.
    \end{block}
\end{frame}

\begin{frame}[fragile]
    \frametitle{Applications in Various Domains}

    \begin{enumerate}
        \item \textbf{Finance}
        \begin{itemize}
            \item Credit Card Fraud Detection: Financial institutions utilize anomaly detection to monitor transactions in real-time.
            \begin{itemize}
                \item Example: Transactions deviating from typical geographic areas can trigger fraud alerts.
            \end{itemize}
        \end{itemize}
        
        \item \textbf{Healthcare}
        \begin{itemize}
            \item Patient Monitoring: Wearable devices collect data, alerting when vital signs deviate from normal ranges.
            \begin{itemize}
                \item Example: A sudden spike in heart rate may prompt immediate medical evaluation.
            \end{itemize}
        \end{itemize}
    \end{enumerate}
\end{frame}

\begin{frame}[fragile]
    \frametitle{Further Applications}

    \begin{enumerate}    
        \setcounter{enumi}{2}
        \item \textbf{Cybersecurity}
        \begin{itemize}
            \item Intrusion Detection Systems (IDS): Identifying unusual patterns of network traffic to flag potential breaches.
            \begin{itemize}
                \item Example: Sudden large downloads by users with infrequent access to sensitive data can indicate insider threats.
            \end{itemize}
        \end{itemize}

        \item \textbf{Fraud Detection in Retail}
        \begin{itemize}
            \item Retail Transaction Monitoring: Anomaly detection helps identify unusual purchasing patterns.
            \begin{itemize}
                \item Example: Frequent returns of high-value items without valid reasons may warrant investigation.
            \end{itemize}
        \end{itemize}
    \end{enumerate}
\end{frame}

\begin{frame}[fragile]
    \frametitle{Key Points and Conclusion}

    \begin{block}{Key Points to Emphasize}
        \begin{itemize}
            \item \textbf{Importance:} Timely identification of anomalies can prevent financial loss, enhance patient outcomes, and secure data integrity.
            \item \textbf{Complexity:} Real-world applications often involve large datasets and complex environments necessitating sophisticated algorithms.
            \item \textbf{Integration:} Systems must be seamlessly integrated into existing workflows for effective real-time monitoring.
        \end{itemize}
    \end{block}

    \begin{block}{Conclusion}
        Anomaly detection plays a vital role across diverse fields by enabling proactive responses to potential issues. The discussed implementations show how critical this technology is for safeguarding finances, health, data, and overall systems integrity.
    \end{block}
\end{frame}

\begin{frame}[fragile]
    \frametitle{Challenges in Anomaly Detection - Overview}
    Anomaly detection identifies rare items or events significantly differing from the majority of data. 
    This slide discusses three significant challenges:
    \begin{itemize}
        \item \textbf{Class Imbalance}
        \item \textbf{Real-time Processing}
        \item \textbf{High-dimensional Data}
    \end{itemize}
\end{frame}

\begin{frame}[fragile]
    \frametitle{Challenges in Anomaly Detection - Class Imbalance}
    \begin{block}{Definition}
        Class imbalance occurs when the number of normal instances significantly outweighs anomalies.
        \textit{Example:} In fraud detection, legitimate transactions outnumber fraudulent ones drastically.
    \end{block}

    \begin{block}{Challenges}
        \begin{itemize}
            \item \textbf{Impact on Learning Algorithms}: Algorithms may be biased towards majority class (normal instances).
            \item \textbf{Examples}: High accuracy can mask poor performance in detecting critical anomalies.
        \end{itemize}
    \end{block}

    \begin{block}{Possible Solutions}
        \begin{itemize}
            \item \textbf{Resampling Techniques}: Oversampling/minority class or undersampling/majority class.
            \item \textbf{Cost-sensitive Learning}: Assign higher costs to misclassification of minority class instances.
        \end{itemize}
    \end{block}
\end{frame}

\begin{frame}[fragile]
    \frametitle{Challenges in Anomaly Detection - Real-time Processing and High-dimensional Data}
    \begin{block}{Real-time Processing}
        \begin{itemize}
            \item \textbf{Definition}: Analyzing and detecting anomalies as data is generated in real time (milliseconds).
            \item \textbf{Challenges}: 
            \begin{itemize}
                \item \textbf{Volume of Data}: High-speed data generation complicates process.
                \item \textbf{Latency Requirements}: Immediate detection is essential in applications like cybersecurity.
            \end{itemize}
            \item \textbf{Possible Solutions}:
            \begin{itemize}
                \item \textbf{Stream Processing Frameworks}: Utilize frameworks like Apache Kafka or Flink.
                \item \textbf{Lightweight Models}: Develop models that predict faster without major accuracy loss.
            \end{itemize}
        \end{itemize}
    \end{block}

    \begin{block}{High-dimensional Data}
        \begin{itemize}
            \item \textbf{Definition}: Data with a high number of features, leading to complexities.
            \item \textbf{Challenges}:
            \begin{itemize}
                \item \textbf{Curse of Dimensionality}: Sparsely populated space complicates anomaly detection.
                \item \textbf{Overfitting}: Models may capture noise rather than significant patterns.
            \end{itemize}
            \item \textbf{Possible Solutions}:
            \begin{itemize}
                \item \textbf{Dimensionality Reduction}: Techniques like PCA or t-SNE.
                \item \textbf{Feature Selection}: Retain only pertinent features for effective detection.
            \end{itemize}
        \end{itemize}
    \end{block}
\end{frame}

\begin{frame}[fragile]
    \frametitle{Conclusion - Summary of Key Concepts}
    Anomaly detection refers to the process of identifying patterns in data that do not conform to expected behavior. It plays a crucial role in various domains such as:
    \begin{itemize}
        \item \textbf{Finance:} Fraud detection
        \item \textbf{Healthcare:} Monitoring patient vitals
        \item \textbf{Cybersecurity:} Identifying intrusions
    \end{itemize}
\end{frame}

\begin{frame}[fragile]
    \frametitle{Conclusion - Importance of Optimizing Methods}
    Optimizing anomaly detection methods ensures that systems can effectively:
    \begin{enumerate}
        \item \textbf{Minimize False Positives:}
            \begin{itemize}
                \item Encourages trust in detection systems.
                \item Example: High false positive rates in credit card fraud detection can lead to customer dissatisfaction.
            \end{itemize}
        \item \textbf{Maximize True Positives:}
            \begin{itemize}
                \item Critical in applications like healthcare where missing an anomaly can lead to severe risks.
            \end{itemize}
        \item \textbf{Enhance Speed and Efficiency:}
            \begin{itemize}
                \item Real-time detection is essential (e.g., cybersecurity must analyze network traffic instantly).
            \end{itemize}
        \item \textbf{Handle High-Dimensional Data:}
            \begin{itemize}
                \item Essential for efficiently processing large datasets, such as social media analysis.
            \end{itemize}
    \end{enumerate}
\end{frame}

\begin{frame}[fragile]
    \frametitle{Conclusion - Key Takeaways}
    \begin{itemize}
        \item \textbf{Challenges Recap:} Issues like class imbalance and high-dimensionality complicate detection.
        \item \textbf{Multi-domain Application:} Essential across sectors for asset protection and enhanced decision-making.
        \item \textbf{Future Directions:} Ongoing research needed for adaptive algorithms tackling evolving data complexities.
    \end{itemize}
    
    \textbf{Concluding Thoughts:} The landscape of anomaly detection is evolving. Emphasizing optimization is key to mitigate risks and ensure operational stability across industries.
\end{frame}


\end{document}