\documentclass[aspectratio=169]{beamer}

% Theme and Color Setup
\usetheme{Madrid}
\usecolortheme{whale}
\useinnertheme{rectangles}
\useoutertheme{miniframes}

% Additional Packages
\usepackage[utf8]{inputenc}
\usepackage[T1]{fontenc}
\usepackage{graphicx}
\usepackage{booktabs}
\usepackage{listings}
\usepackage{amsmath}
\usepackage{amssymb}
\usepackage{xcolor}
\usepackage{tikz}
\usepackage{pgfplots}
\pgfplotsset{compat=1.18}
\usetikzlibrary{positioning}
\usepackage{hyperref}

% Custom Colors
\definecolor{myblue}{RGB}{31, 73, 125}
\definecolor{mygray}{RGB}{100, 100, 100}
\definecolor{mygreen}{RGB}{0, 128, 0}
\definecolor{myorange}{RGB}{230, 126, 34}
\definecolor{mycodebackground}{RGB}{245, 245, 245}

% Set Theme Colors
\setbeamercolor{structure}{fg=myblue}
\setbeamercolor{frametitle}{fg=white, bg=myblue}
\setbeamercolor{title}{fg=myblue}
\setbeamercolor{section in toc}{fg=myblue}
\setbeamercolor{item projected}{fg=white, bg=myblue}
\setbeamercolor{block title}{bg=myblue!20, fg=myblue}
\setbeamercolor{block body}{bg=myblue!10}
\setbeamercolor{alerted text}{fg=myorange}

% Set Fonts
\setbeamerfont{title}{size=\Large, series=\bfseries}
\setbeamerfont{frametitle}{size=\large, series=\bfseries}
\setbeamerfont{caption}{size=\small}
\setbeamerfont{footnote}{size=\tiny}

% Document Start
\begin{document}

\frame{\titlepage}

\begin{frame}[fragile]
    \frametitle{Capstone Project Work Overview}
    \begin{block}{Introduction to Capstone Project Work}
        The capstone project is a pivotal component of our course, designed to integrate the knowledge and skills accumulated throughout the term. This week marks a unique opportunity for you to collaborate with your peers, harnessing collective learning to solve complex problems in real-world scenarios.
    \end{block}
\end{frame}

\begin{frame}[fragile]
    \frametitle{Key Concepts Explained}
    \begin{enumerate}
        \item \textbf{Integration of Course Learnings:}
            \begin{itemize}
                \item You have acquired diverse knowledge—from theoretical foundations to practical skills in AI and application development.
                \item \textit{Example:} Use machine learning algorithms from your coursework to analyze real data sets and derive insights.
            \end{itemize}
        
        \item \textbf{Collaboration Among Teams:}
            \begin{itemize}
                \item Teamwork fosters creativity, promotes diverse perspectives, and enhances problem-solving capabilities.
                \item \textit{Example:} Engage in brainstorming sessions to pool knowledge and craft project directions.
            \end{itemize}
    \end{enumerate}
\end{frame}

\begin{frame}[fragile]
    \frametitle{Key Points to Emphasize}
    \begin{itemize}
        \item \textbf{Expectations:} Approach the capstone project as a real-world scenario, applying concepts learned in class.
        \item \textbf{Collaboration Tools:} Utilize platforms (e.g., GitHub, Trello, Slack) for efficient communication and transparency.
        \item \textbf{Feedback Mechanism:} Implement a system for constructive criticism and feedback to promote continuous improvement.
    \end{itemize}

    \begin{block}{Conclusion}
        Your capstone project is not just an assessment; it’s an opportunity to synthesize your learning, work collaboratively, and contribute to innovative solutions. Embrace this experience as a fundamental step in your proficiency journey.
    \end{block}
\end{frame}

\begin{frame}[fragile]
    \frametitle{Objectives of Capstone Project - Overview}
    \begin{itemize}
        \item Applying AI Concepts
        \item Developing Teamwork
        \item Enhancing Ethical Considerations in Application Development
    \end{itemize}
\end{frame}

\begin{frame}[fragile]
    \frametitle{Objectives of Capstone Project - Applying AI Concepts}
    \begin{block}{Explanation}
        The capstone project provides a practical platform for students to utilize and integrate the AI concepts learned throughout the course, including:
        \begin{itemize}
            \item Machine learning algorithms
            \item Natural language processing
            \item Data analysis techniques
        \end{itemize}
    \end{block}
    \begin{block}{Example}
        For instance, a team may choose to develop an AI-powered chatbot using natural language processing to enhance user interaction on a website. This allows students to implement theoretical knowledge in a real-world scenario, deepening their understanding of AI technologies.
    \end{block}
\end{frame}

\begin{frame}[fragile]
    \frametitle{Objectives of Capstone Project - Developing Teamwork}
    \begin{block}{Explanation}
        Collaboration is a crucial skill in both academic and professional settings. During the capstone project, students will:
        \begin{itemize}
            \item Work in diverse teams
            \item Foster cooperation, communication, and collective problem-solving
        \end{itemize}
    \end{block}
    \begin{block}{Example}
        A team might assign roles based on members’ strengths—such as project manager, lead developer, or UX designer—encouraging effective task distribution and promoting teamwork. This experience mimics real-world software development environments.
    \end{block}
\end{frame}

\begin{frame}[fragile]
    \frametitle{Objectives of Capstone Project - Enhancing Ethical Considerations}
    \begin{block}{Explanation}
        As technology evolves, so do the ethical considerations surrounding its use. Students are encouraged to:
        \begin{itemize}
            \item Assess the societal impacts of AI technology
            \item Ensure responsible implementation
        \end{itemize}
    \end{block}
    \begin{block}{Example}
        When creating an AI model for predictive analytics, students should consider issues like bias in training data and transparency in decision-making, promoting applications that align with ethical standards and respect user privacy.
    \end{block}
\end{frame}

\begin{frame}[fragile]
    \frametitle{Key Points to Emphasize}
    \begin{itemize}
        \item Integration of Skills: Synthesize concepts from the entire curriculum, demonstrating comprehensive understanding of AI.
        \item Collaboration Skills: Strong teamwork improves project outcomes and prepares students for the collaborative nature of the tech industry.
        \item Ethical Responsibility: Students should consider the ‘how’, ‘what’, and ‘why’ of their project decisions, ensuring mindful development of AI applications.
    \end{itemize}
\end{frame}

\begin{frame}[fragile]
    \frametitle{Team Formation and Roles}
    \begin{block}{Importance of Team Roles}
        \begin{itemize}
            \item Team roles refer to specific responsibilities and functions assigned to each team member.
            \item Each role is crucial for the holistic success of a project.
            \item Diverse roles enhance collaboration, ensuring effective problem-solving and innovation.
            \item Clear roles reduce confusion and ensure accountability.
        \end{itemize}
    \end{block}
\end{frame}

\begin{frame}[fragile]
    \frametitle{Key Team Roles}
    \begin{enumerate}
        \item \textbf{Project Manager}
            \begin{itemize}
                \item Oversees project progress, timelines, and resources.
                \item Example: Creating a Gantt chart to visualize timelines.
            \end{itemize}
        \item \textbf{Developer}
            \begin{itemize}
                \item Responsible for coding and application development tasks.
                \item Example: Using Agile methodology for project increments.
            \end{itemize}
        \item \textbf{Designer}
            \begin{itemize}
                \item Focuses on user interface and user experience design.
                \item Example: Creating wireframes for main features.
            \end{itemize}
        \item \textbf{Quality Assurance (QA) Specialist}
            \begin{itemize}
                \item Tests the application for functionality and performance.
                \item Example: Implementing automated testing scripts.
            \end{itemize}
        \item \textbf{Content/Research Lead}
            \begin{itemize}
                \item Conducts research and contributes relevant content.
                \item Example: Gathering data for AI algorithms.
            \end{itemize}
        \item \textbf{Data Analyst} (specific to AI projects)
            \begin{itemize}
                \item Analyzes data sets to inform project decisions.
                \item Example: Interpreting results from machine learning models.
            \end{itemize}
    \end{enumerate}
\end{frame}

\begin{frame}[fragile]
    \frametitle{Effective Collaboration Strategies}
    \begin{itemize}
        \item \textbf{Regular Communication}
            \begin{itemize}
                \item Establish daily or weekly check-ins.
                \item Use collaborative tools (e.g., Slack, Microsoft Teams).
            \end{itemize}
        \item \textbf{Defined Goals and Objectives}
            \begin{itemize}
                \item Clear understanding of project goals for every member.
            \end{itemize}
        \item \textbf{Conflict Resolution}
            \begin{itemize}
                \item Implement a structured approach to address conflicts.
                \item Example: Establish a "cooling-off" period followed by discussion.
            \end{itemize}
        \item \textbf{Feedback Mechanism}
            \begin{itemize}
                \item Foster an environment for constructive feedback.
                \item Example: Holding retrospective meetings after milestones.
            \end{itemize}
    \end{itemize}
\end{frame}

\begin{frame}[fragile]
  \frametitle{Project Planning and Management}
  \begin{block}{Overview}
    Effective project planning and management is essential for ensuring that a project meets its objectives and is completed on time and within budget. 
    This slide covers the frameworks and methodologies for organizing tasks, setting milestones, and optimizing team collaboration.
  \end{block}
\end{frame}

\begin{frame}[fragile]
  \frametitle{Key Concepts}
  \begin{enumerate}
    \item \textbf{Project Milestones}
      \begin{itemize}
        \item Significant checkpoints or deliverables in a project timeline.
        \item Example: Completing a project proposal by Week 2 or delivering a prototype by Week 6.
      \end{itemize}
      
    \item \textbf{Task Breakdown}
      \begin{itemize}
        \item Decomposing larger tasks into smaller, actionable items aids in resource allocation and progress tracking.
        \item Example: Instead of "Developing a Marketing Strategy," use tasks like "Market Research," "Competitor Analysis," and "Plan Outreach Channels."
      \end{itemize}
      
    \item \textbf{Timelines}
      \begin{itemize}
        \item A visual representation of the project schedule indicating when tasks need to be completed.
        \item Example: A Gantt Chart displays tasks on a horizontal timeline, showcasing dependencies and overlapping work.
      \end{itemize}
  \end{enumerate}
\end{frame}

\begin{frame}[fragile]
  \frametitle{Framework for Project Planning}
  \begin{enumerate}
    \item \textbf{Define Goals \& Objectives}
      \begin{itemize}
        \item Clearly outline what the project aims to achieve by utilizing SMART criteria.
      \end{itemize}

    \item \textbf{Identify Stakeholders}
      \begin{itemize}
        \item Determine who has a vested interest in the project and involve them in the planning process.
      \end{itemize}

    \item \textbf{Create a Work Breakdown Structure (WBS)}
      \begin{itemize}
        \item A hierarchical decomposition of all the work needed to complete the project.
      \end{itemize}

    \item \textbf{Develop a Schedule}
      \begin{itemize}
        \item Use software tools (e.g., Microsoft Project, Asana) for task scheduling and prioritization.
        \item Example Gantt Chart Structure:
          \begin{itemize}
            \item Task 1: Initial Research | Duration: Week 1–2
            \item Task 2: Development Phase | Duration: Week 3–5
            \item Task 3: Testing \& Feedback | Duration: Week 6
          \end{itemize}
      \end{itemize}
  \end{enumerate}
\end{frame}

\begin{frame}[fragile]
    \frametitle{Research and Ideation}
    \begin{block}{Methods for Brainstorming and Selecting Viable AI Projects}
        Research and Ideation refers to the process of generating ideas and selecting the most feasible projects to implement. It is a critical first step in project development, 
        particularly for Artificial Intelligence (AI), where the possibilities are vast but require careful consideration of practicality, ethics, and societal impact.
    \end{block}
\end{frame}

\begin{frame}[fragile]
    \frametitle{Brainstorming Techniques}
    \begin{enumerate}
        \item \textbf{Mind Mapping}: Visual representation of ideas connecting to a central theme. Start with "AI Solutions" at the center, branching into subcategories like healthcare, finance, education, etc.
        \item \textbf{Brainwriting}: A group activity where participants write down ideas independently before sharing, encouraging contributions from all team members.
        \item \textbf{SCAMPER Method}:
        \begin{itemize}
            \item \textbf{Substitute}: What can you replace in a current process?
            \item \textbf{Combine}: Can you merge two ideas for a better outcome?
            \item \textbf{Adapt}: How can you tweak an existing idea?
            \item \textbf{Modify}: How can you enhance or change the form?
            \item \textbf{Put to another use}: Can the product serve another purpose?
            \item \textbf{Eliminate}: What can you remove to simplify the idea?
            \item \textbf{Reverse}: What can you rearrange for better results?
        \end{itemize}
    \end{enumerate}
\end{frame}

\begin{frame}[fragile]
    \frametitle{Evaluating Ideas and Ethical Implications}
    \begin{block}{Criteria for Selection}
        \begin{itemize}
            \item \textbf{Feasibility}: Can the project be realistically executed given the current technology and resources?
            \item \textbf{Market Need}: Assess the demand for the proposed AI solution.
            \item \textbf{Innovation}: Does the idea bring something new or improve existing solutions?
            \item \textbf{Scalability}: Can the project grow to serve a larger audience?
        \end{itemize}
    \end{block}
    
    \begin{block}{Ethical Implications}
        \begin{itemize}
            \item \textbf{Responsible AI}: Evaluate how AI projects affect privacy, fairness, and security. Important questions to consider include:
            \begin{itemize}
                \item Does the project respect user privacy?
                \item Are there potential biases in the AI model?
                \item How will the data be used and shared?
            \end{itemize}
            \item \textbf{Compliance}: Adhere to legal regulations such as GDPR.
        \end{itemize}
    \end{block}
\end{frame}

\begin{frame}[fragile]
    \frametitle{Example AI Project Ideas}
    \begin{itemize}
        \item \textbf{Healthcare Prediction Models}: 
        Use AI to predict patient outcomes based on historical data. 
        Evaluate its impact on privacy and biases in medical datasets.
        
        \item \textbf{AI for Environmental Monitoring}: 
        Monitor air quality using AI-driven data analysis. 
        Consider ethical implications like data collection methods and impact on local communities.
    \end{itemize}
\end{frame}

\begin{frame}[fragile]
    \frametitle{Technical Implementation - Overview}
    This slide provides guidelines for implementing your chosen AI solutions, integrating necessary tools and frameworks to facilitate the development process. By following these guidelines, you will efficiently translate your ideas into functional AI systems.
\end{frame}

\begin{frame}[fragile]
    \frametitle{Technical Implementation - Project Scope and Languages}
    \begin{enumerate}
        \item \textbf{Define the Project Scope}
        \begin{itemize}
            \item \textbf{Clarify Objectives:} Set clear, measurable goals for what your AI solution should accomplish.
            \item \textbf{Use Cases:} Identify the specific use cases that your project will address. This will guide your technical implementation.
        \end{itemize}
        
        \item \textbf{Select Programming Languages}
        \begin{itemize}
            \item \textbf{Python:} Highly recommended for AI due to its simplicity and wide range of libraries (e.g., TensorFlow, PyTorch).
            \begin{block}{Example: Building a Machine Learning Model in Python}
            \begin{lstlisting}[language=Python]
import pandas as pd
from sklearn.model_selection import train_test_split
from sklearn.linear_model import LogisticRegression

# Load dataset
data = pd.read_csv("data.csv")
X = data.drop("target", axis=1)
y = data["target"]

# Split data into training and testing sets
X_train, X_test, y_train, y_test = train_test_split(X, y, test_size=0.2)

# Instantiate and train model
model = LogisticRegression()
model.fit(X_train, y_train)

# Predictions
predictions = model.predict(X_test)
            \end{lstlisting}
            \end{block}
        \end{itemize}
    \end{enumerate}
\end{frame}

\begin{frame}[fragile]
    \frametitle{Technical Implementation - Frameworks and Tools}
    \begin{enumerate}
        \setcounter{enumi}{2}
        \item \textbf{Choose Frameworks and Libraries}
        \begin{itemize}
            \item \textbf{TensorFlow:} Excellent for deep learning applications.
            \item \textbf{PyTorch:} Preferred for projects requiring flexibility in model design.
            \item \textbf{Scikit-Learn:} Ideal for traditional machine learning tasks.
        \end{itemize}

        \item \textbf{Data Management Tools}
        \begin{itemize}
            \item \textbf{Pandas:} Essential for data manipulation and analysis.
            \item \textbf{SQL Databases:} (like MySQL or PostgreSQL) for storing large datasets.
        \end{itemize}

        \item \textbf{Model Training and Evaluation}
        \begin{itemize}
            \item \textbf{Training:} Use libraries like Keras or built-in functionalities of TensorFlow/PyTorch to train your models.
            \item \textbf{Evaluation metrics:} Include accuracy, precision, recall, and F1-score to assess your model’s performance.
            \begin{block}{Formula for F1-Score}
            \begin{equation}
                F1 = 2 \cdot \frac{Precision \cdot Recall}{Precision + Recall}
            \end{equation}
            \end{block}
        \end{itemize}
    \end{enumerate}
\end{frame}

\begin{frame}[fragile]
    \frametitle{Technical Implementation - Deployment and Conclusion}
    \begin{enumerate}
        \setcounter{enumi}{5}
        \item \textbf{Deployment}
        \begin{itemize}
            \item \textbf{Flask/Django:} For creating web applications to deploy your model as a service.
            \item \textbf{Docker:} To containerize your application for easier deployment and scaling.
        \end{itemize}
    \end{enumerate}

    \textbf{Key Points to Emphasize:}
    \begin{itemize}
        \item Align your technical implementation with the objectives defined in the project scope.
        \item Document your code and implementation process to facilitate future iterations and maintenance.
        \item Ensure that ethical considerations are integrated throughout your implementation process.
    \end{itemize}

    \textbf{Conclusion:} 
    Implementing your AI solution involves careful selection of programming languages, frameworks, and tools tailored to your project's needs. Remember, iterative testing and evaluation are vital for refining your AI model's performance.
\end{frame}

\begin{frame}[fragile]
    \frametitle{Ethical Considerations in AI Projects - Overview}
    \begin{block}{Understanding Ethical Dilemmas in AI}
        Ethical considerations in AI refer to the moral implications and responsibilities associated with the development and deployment of artificial intelligence technologies.
    \end{block}

    \begin{block}{Importance of Addressing Ethical Dilemmas}
        \begin{itemize}
            \item \textbf{Trust and Adoption:} Users are more likely to trust transparent and ethical AI systems.
            \item \textbf{Social Impact:} Ensures benefits of AI extend to all, avoiding harm or bias.
            \item \textbf{Regulatory Compliance:} Compliance with laws like GDPR is crucial to avoid legal repercussions.
        \end{itemize}
    \end{block}
\end{frame}

\begin{frame}[fragile]
    \frametitle{Ethical Considerations in AI Projects - Common Issues}
    \begin{block}{Common Ethical Issues in AI}
        \begin{enumerate}
            \item \textbf{Bias and Fairness:} AI can perpetuate existing biases. \\
            \textit{Example:} A hiring AI may favor specific demographics.
            \item \textbf{Privacy:} Must respect user consent and privacy. \\
            \textit{Example:} Facial recognition raises surveillance concerns.
            \item \textbf{Accountability:} Complexity in determining responsibility for AI decisions. \\
            \textit{Example:} Liability in autonomous vehicle accidents.
            \item \textbf{Transparency:} AI algorithms should be explainable. \\
            \textit{Example:} Providing insights on AI decisions enhances trust.
        \end{enumerate}
    \end{block}
\end{frame}

\begin{frame}[fragile]
    \frametitle{Ethical Considerations in AI Projects - Actionable Solutions}
    \begin{block}{Actionable Solutions to Address Ethics in AI}
        \begin{enumerate}
            \item \textbf{Diverse Data Collection:} Ensure data is representative to minimize bias.
            \item \textbf{Implementing AI Ethics Guidelines:} Adopt ethical frameworks for guidance.
            \item \textbf{User Involvement:} Engage end-users in AI design for diverse perspectives.
            \item \textbf{Regular Audits:} Conduct periodic reviews for social impact and compliance.
            \item \textbf{Transparency Measures:} Develop user-friendly documentation for AI functionalities.
        \end{enumerate}
    \end{block}

    \begin{block}{Key Points to Emphasize}
        Ethics in AI is crucial for acceptance and societal benefit; addressing these issues fosters innovation and builds a responsible AI ecosystem.
    \end{block}
\end{frame}

\begin{frame}[fragile]
    \frametitle{Data Management Practices}
    \begin{block}{Overview of Data Management in AI}
        The integrity and quality of data are paramount in AI applications. Effective data management practices encompass three key phases: collection, cleaning, and preprocessing. These practices ensure that our AI models are built on reliable, ethical data.
    \end{block}
\end{frame}

\begin{frame}[fragile]
    \frametitle{Data Collection: Best Practices}
    \begin{itemize}
        \item \textbf{Define Objectives:} Clearly outline what you aim to achieve with the data. \\
        *Example:* For a housing price prediction model, gather relevant property features, neighborhood characteristics, and historical prices.
        
        \item \textbf{Diversify Data Sources:} Utilize multiple sources to create a comprehensive dataset, capturing various perspectives and minimizing bias. \\
        *Example:* Combine public datasets (e.g., government stats) with private datasets (e.g., real estate listings).
        
        \item \textbf{Informed Consent:} Ensure ethical data collection by obtaining consent from participants and informing them of data usage.
    \end{itemize}
\end{frame}

\begin{frame}[fragile]
    \frametitle{Data Cleaning: Essential Steps}
    \begin{itemize}
        \item \textbf{Identify and Handle Missing Values:} Analyze for missing data and:
        \begin{itemize}
            \item Fill gaps with average values (mean/mode)
            \item Remove records with excessive missing data
        \end{itemize}
        \begin{lstlisting}[language=Python]
df.fillna(df.mean(), inplace=True)  # Filling missing values with the mean
        \end{lstlisting}

        \item \textbf{Remove Duplicates:} Eliminate duplicate records to ensure uniqueness.
        \begin{lstlisting}[language=Python]
df.drop_duplicates(inplace=True)  # Removes duplicate rows in the DataFrame
        \end{lstlisting}

        \item \textbf{Correct Inconsistent Data:} Standardize formats and fix mislabeled records for uniformity.
    \end{itemize}
\end{frame}

\begin{frame}[fragile]
    \frametitle{Data Preprocessing: Preparing for Analysis}
    \begin{itemize}
        \item \textbf{Normalization/Standardization:} Scale numerical data to reduce computational discrepancies.
        \begin{lstlisting}[language=Python]
from sklearn.preprocessing import StandardScaler
scaler = StandardScaler()
df[['col1', 'col2']] = scaler.fit_transform(df[['col1', 'col2']])
        \end{lstlisting}

        \item \textbf{Encoding Categorical Data:} Convert categorical variables into numerical format for compatibility using techniques like one-hot encoding.
        \begin{lstlisting}[language=Python]
df = pd.get_dummies(df, columns=['category_column'])  # One-hot encoding
        \end{lstlisting}

        \item \textbf{Data Splitting:} Divide the dataset into training and testing sets.
        \begin{lstlisting}[language=Python]
from sklearn.model_selection import train_test_split
train, test = train_test_split(df, test_size=0.2, random_state=42)
        \end{lstlisting}
    \end{itemize}
\end{frame}

\begin{frame}[fragile]
    \frametitle{Key Points to Emphasize}
    \begin{itemize}
        \item \textbf{Quality Over Quantity:} High-quality data is more valuable than vast amounts of poor-quality data.
        \item \textbf{Ethics in Data Usage:} Prioritize ethical considerations, especially regarding privacy and consent.
        \item \textbf{Iterate and Improve:} Data management is iterative; continuously monitor and refine practices based on new challenges.
    \end{itemize}
\end{frame}

\begin{frame}[fragile]
    \frametitle{Collaboration Tools and Techniques - Overview}
    \begin{block}{Overview}
        Effective collaboration is crucial for the success of any capstone project, particularly in team environments. 
        This slide presents various tools and techniques to enhance communication and project management among team members, ensuring that everyone stays aligned and informed throughout the project lifecycle.
    \end{block}
\end{frame}

\begin{frame}[fragile]
    \frametitle{Collaboration Tools and Techniques - Key Collaboration Tools}
    \begin{block}{Key Collaboration Tools}
        \begin{enumerate}
            \item \textbf{Communication Platforms}
                \begin{itemize}
                    \item \textbf{Slack}: Real-time messaging with direct messages and group channels.
                          \begin{itemize}
                              \item \textit{Example}: Team channels for different project aspects (e.g., design, research).
                          \end{itemize}
                    \item \textbf{Microsoft Teams}: Integrates chat, video meetings, and file collaboration.
                          \begin{itemize}
                              \item \textit{Illustration}: Scheduling team meetings for project updates.
                          \end{itemize}
                \end{itemize}
            \item \textbf{Project Management Tools}
                \begin{itemize}
                    \item \textbf{Trello}: Visual tool using boards and cards to organize tasks.
                          \begin{itemize}
                              \item \textit{Key Feature}: Drag-and-drop functionality for task management.
                          \end{itemize}
                    \item \textbf{Asana}: Organizes, tracks, and manages tasks with due dates.
                          \begin{itemize}
                              \item \textit{Example}: Assigning tasks for milestones.
                          \end{itemize}
                \end{itemize}
            \item \textbf{Document Collaboration}
                \begin{itemize}
                    \item \textbf{Google Workspace}: Real-time collaboration on documents.
                          \begin{itemize}
                              \item \textit{Feature}: Comments and suggestions for peer input.
                          \end{itemize}
                    \item \textbf{Dropbox Paper}: Collaborative document-editing tool.
                \end{itemize}
        \end{enumerate}
    \end{block}
\end{frame}

\begin{frame}[fragile]
    \frametitle{Collaboration Tools and Techniques - Techniques for Effective Collaboration}
    \begin{block}{Techniques for Effective Collaboration}
        \begin{enumerate}
            \item \textbf{Regular Stand-up Meetings}: Short meetings to sync on progress.
                  \begin{itemize}
                      \item \textit{Frequency}: Daily or weekly check-ins.
                  \end{itemize}
            \item \textbf{Defined Roles and Responsibilities}: Outlining duties to ensure accountability.
                  \begin{itemize}
                      \item \textit{Example}: A RACI matrix (Responsible, Accountable, Consulted, Informed).
                  \end{itemize}
            \item \textbf{Version Control Systems (e.g., Git)}: For managing code changes collaboratively.
                  \begin{itemize}
                      \item \textit{Diagram}: Visualizing branches and merging processes.
                  \end{itemize}
        \end{enumerate}
    \end{block}
\end{frame}

\begin{frame}[fragile]
    \frametitle{Collaboration Tools and Techniques - Key Points to Emphasize}
    \begin{block}{Key Points to Emphasize}
        \begin{itemize}
            \item \textbf{Customization}: Select tools that fit team needs and workflows.
            \item \textbf{Feedback Loops}: Encourage constructive feedback throughout the project.
            \item \textbf{Documentation}: Maintain clear records of meetings, decisions, and changes.
        \end{itemize}
    \end{block}
    Using these tools and techniques will significantly enhance team collaboration, ensuring smoother project execution and a successful capstone experience.
\end{frame}

\begin{frame}[fragile]
    \frametitle{Final Deliverables - Overview}
    \begin{block}{Summary of Expected Outputs}
        In the final phase of your capstone project, you are expected to produce key deliverables that showcase your work and collaborative efforts. These deliverables encapsulate your learning journey and the impact of your project.
    \end{block}
\end{frame}

\begin{frame}[fragile]
    \frametitle{Final Deliverables - 1. Project Report}
    \begin{enumerate}
        \item \textbf{Definition:} A comprehensive document detailing the process, methods, findings, and conclusions of your project.
        \item \textbf{Key Components:}
        \begin{itemize}
            \item \textbf{Introduction:} Project background, objectives, significance.
            \item \textbf{Literature Review:} Summary of existing research relevant to your project.
            \item \textbf{Methodology:} Techniques and tools used during the project.
            \item \textbf{Results:} Findings presented with charts, graphs, or tables.
            \item \textbf{Discussion:} Interpretation of results and implications.
            \item \textbf{Conclusion and Recommendations:} Key takeaways and future directions.
        \end{itemize}
        \item \textbf{Example:} For a software development project, include features, development process, user testing results, and feedback.
    \end{enumerate}
\end{frame}

\begin{frame}[fragile]
    \frametitle{Final Deliverables - 2. Presentations}
    \begin{enumerate}
        \item \textbf{Definition:} Deliver a presentation summarizing your project for classmates, faculty, or stakeholders.
        \item \textbf{Components of an Effective Presentation:}
        \begin{itemize}
            \item \textbf{Engaging Introduction:} Use an interesting fact or question.
            \item \textbf{Clear Structure:} Organized flow (Introduction → Body → Conclusion).
            \item \textbf{Visual Aids:} Use slides to support your narrative.
        \end{itemize}
        \item \textbf{Example:} A 10-minute presentation with each team member discussing contributions, followed by Q\&A.
    \end{enumerate}
\end{frame}

\begin{frame}[fragile]
    \frametitle{Final Deliverables - 3. Peer Feedback Process}
    \begin{enumerate}
        \item \textbf{Purpose:} Engage with peers to critically assess each other's work, improving quality of final outputs.
        \item \textbf{Process:}
        \begin{itemize}
            \item \textbf{Draft Submissions:} Submit drafts for peer review.
            \item \textbf{Feedback Sessions:} Participate in structured sessions for constructive comments.
            \item \textbf{Incorporation of Feedback:} Revise documents based on peer insights.
        \end{itemize}
        \item \textbf{Example:} Adjust your report based on feedback to clarify findings or reshape sections.
    \end{enumerate}
\end{frame}

\begin{frame}[fragile]
    \frametitle{Final Deliverables - Key Points}
    \begin{itemize}
        \item Focus on \textbf{quality} and \textbf{clarity} in final deliverables.
        \item Ensure reports are \textbf{well-researched} and documented.
        \item Utilize \textbf{peer feedback} to enhance your work—collaboration is essential.
        \item Prepare thoroughly for your presentation to convey your project's significance.
    \end{itemize}
\end{frame}

\begin{frame}[fragile]
    \frametitle{Final Deliverables - Final Note}
    Remember, your final deliverables reflect your learning, teamwork, and ability to communicate complex ideas. Effort is key to presenting your work in the best light possible!
\end{frame}

\begin{frame}[fragile]
    \frametitle{Peer Feedback and Reflection - Importance of Peer Reviews}
    \begin{block}{Enhancing Quality}
        \begin{itemize}
            \item Peer feedback provides diverse perspectives.
            \item It highlights strengths and identifies areas for improvement.
            \item Example: Peers may suggest clearer explanations or additional data.
        \end{itemize}
    \end{block}

    \begin{block}{Encouraging Collaboration}
        \begin{itemize}
            \item Fosters a collaborative environment.
            \item Students learn from each other and develop communication skills.
            \item Example: Critiquing each other's work builds a sense of community.
        \end{itemize}
    \end{block}

    \begin{block}{Building Critical Thinking Skills}
        \begin{itemize}
            \item Evaluating others' work enhances critical thinking.
            \item Writing constructive feedback helps in analyzing data and methodologies.
        \end{itemize}
    \end{block}
\end{frame}

\begin{frame}[fragile]
    \frametitle{Peer Feedback and Reflection - Reflecting on Learnings}
    \begin{block}{Self-Assessment \& Growth}
        \begin{itemize}
            \item Reflecting on your experience aids in self-assessment.
            \item Key Reflection Questions:
            \begin{itemize}
                \item What aspects of my project management were effective?
                \item In what areas did I struggle?
            \end{itemize}
        \end{itemize}
    \end{block}

    \begin{block}{Integration of Feedback}
        \begin{itemize}
            \item Consider peer feedback and its influence on final submissions.
            \item Example: Critiques on methodology can lead to understanding best practices.
        \end{itemize}
    \end{block}
\end{frame}

\begin{frame}[fragile]
    \frametitle{Peer Feedback and Reflection - Documenting Insights}
    \begin{block}{Documenting Insights}
        \begin{itemize}
            \item Keep a learning journal summarizing insights.
            \item Key Points to Document:
            \begin{itemize}
                \item New strategies or ideas from peers.
                \item Enhanced skills through project collaboration.
                \item Challenges faced and strategies to address them.
            \end{itemize}
        \end{itemize}
    \end{block}

    \begin{block}{Conclusion}
        Engaging in peer feedback is essential for learning, enhancing collaboration, and facilitating self-reflection. It improves the quality of work and prepares students for future endeavors.
    \end{block}
\end{frame}

\begin{frame}[fragile]
  \frametitle{Next Steps after Capstone Project - Overview}
  \begin{block}{Overview}
    After completing your capstone project, it's essential to reflect on the skills and experiences gained. This slide outlines critical opportunities for applying your capstone learnings in professional settings within the AI field.
  \end{block}
\end{frame}

\begin{frame}[fragile]
  \frametitle{Key Skills Developed During the Capstone Project}
  \begin{enumerate}
    \item \textbf{Project Management and Team Collaboration}
      \begin{itemize}
        \item Coordinated tasks among team members and set deadlines.
        \item Example: Utilizing Agile methodologies for iterative development.
      \end{itemize}
    \item \textbf{Technical Skills}
      \begin{itemize}
        \item Proficiency in programming languages (e.g., Python, R).
        \item Example: Implementing logistic regression for classification tasks.
      \end{itemize}
    \item \textbf{Problem-Solving and Analytical Thinking}
      \begin{itemize}
        \item Analyzed data patterns for decision-making processes.
        \item Example: Identifying customer behaviors for targeted marketing strategies.
      \end{itemize}
    \item \textbf{Communication \& Presentation}
      \begin{itemize}
        \item Created impactful visualizations and shared project outcomes.
        \item Example: Presenting results using tools like Tableau or Power BI.
      \end{itemize}
  \end{enumerate}
\end{frame}

\begin{frame}[fragile]
  \frametitle{Applications in Future Career Pathways}
  \begin{enumerate}
    \item \textbf{Career Advancement in AI}
      \begin{itemize}
        \item Enhance your resume by emphasizing project management and technical expertise.
      \end{itemize}
    \item \textbf{Continued Learning}
      \begin{itemize}
        \item Engage in online courses or certifications to build upon skills.
      \end{itemize}
    \item \textbf{Networking Opportunities}
      \begin{itemize}
        \item Attend industry meetups and share your capstone project on platforms like LinkedIn.
      \end{itemize}
    \item \textbf{Portfolio Development}
      \begin{itemize}
        \item Include your capstone project as a case study to highlight challenges and solutions.
      \end{itemize}
    \item \textbf{Contributing to Open Source Projects}
      \begin{itemize}
        \item Gain experience by contributing to open-source AI projects.
      \end{itemize}
  \end{enumerate}
\end{frame}


\end{document}