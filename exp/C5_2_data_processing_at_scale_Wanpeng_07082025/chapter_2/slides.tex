\documentclass[aspectratio=169]{beamer}

% Theme and Color Setup
\usetheme{Madrid}
\usecolortheme{whale}
\useinnertheme{rectangles}
\useoutertheme{miniframes}

% Additional Packages
\usepackage[utf8]{inputenc}
\usepackage[T1]{fontenc}
\usepackage{graphicx}
\usepackage{booktabs}
\usepackage{listings}
\usepackage{amsmath}
\usepackage{amssymb}
\usepackage{xcolor}
\usepackage{tikz}
\usepackage{pgfplots}
\pgfplotsset{compat=1.18}
\usetikzlibrary{positioning}
\usepackage{hyperref}

% Custom Colors
\definecolor{myblue}{RGB}{31, 73, 125}
\definecolor{mygray}{RGB}{100, 100, 100}
\definecolor{mygreen}{RGB}{0, 128, 0}
\definecolor{myorange}{RGB}{230, 126, 34}
\definecolor{mycodebackground}{RGB}{245, 245, 245}

% Set Theme Colors
\setbeamercolor{structure}{fg=myblue}
\setbeamercolor{frametitle}{fg=white, bg=myblue}
\setbeamercolor{title}{fg=myblue}
\setbeamercolor{section in toc}{fg=myblue}
\setbeamercolor{item projected}{fg=white, bg=myblue}
\setbeamercolor{block title}{bg=myblue!20, fg=myblue}
\setbeamercolor{block body}{bg=myblue!10}
\setbeamercolor{alerted text}{fg=myorange}

% Set Fonts
\setbeamerfont{title}{size=\Large, series=\bfseries}
\setbeamerfont{frametitle}{size=\large, series=\bfseries}
\setbeamerfont{caption}{size=\small}
\setbeamerfont{footnote}{size=\tiny}

% Footer and Navigation Setup
\setbeamertemplate{footline}{
  \leavevmode%
  \hbox{%
  \begin{beamercolorbox}[wd=.3\paperwidth,ht=2.25ex,dp=1ex,center]{author in head/foot}%
    \usebeamerfont{author in head/foot}\insertshortauthor
  \end{beamercolorbox}%
  \begin{beamercolorbox}[wd=.5\paperwidth,ht=2.25ex,dp=1ex,center]{title in head/foot}%
    \usebeamerfont{title in head/foot}\insertshorttitle
  \end{beamercolorbox}%
  \begin{beamercolorbox}[wd=.2\paperwidth,ht=2.25ex,dp=1ex,center]{date in head/foot}%
    \usebeamerfont{date in head/foot}
    \insertframenumber{} / \inserttotalframenumber
  \end{beamercolorbox}}%
  \vskip0pt%
}

% Turn off navigation symbols
\setbeamertemplate{navigation symbols}{}

% Title Page Information
\title[Distributed Systems and Data Management]{Weeks 5-8: Distributed Systems and Data Management}
\author[J. Smith]{John Smith, Ph.D.}
\institute[University Name]{
  Department of Computer Science\\
  University Name\\
  \vspace{0.3cm}
  Email: email@university.edu\\
  Website: www.university.edu
}
\date{\today}

% Document Start
\begin{document}

\frame{\titlepage}

\begin{frame}[fragile]
    \frametitle{Introduction to Distributed Systems}
    \begin{block}{What are Distributed Systems?}
        A distributed system is a network of independent computers that work together to achieve a common goal. 
        Each computer (often referred to as a node) has its own local memory and processing power, but they communicate 
        and coordinate tasks over a network. This architecture allows distributed systems to function as a single 
        coherent system, despite the geographical dispersion of their components.
    \end{block}
\end{frame}

\begin{frame}[fragile]
    \frametitle{Importance in Data Management}
    \begin{enumerate}
        \item \textbf{Scalability}: Allows handling increased loads by adding more nodes (e.g., AWS).
        \item \textbf{Fault Tolerance}: Continues operation despite node failures through redundancy (e.g., Google’s BigTable).
        \item \textbf{Performance}: Executes tasks concurrently for improved processing speed (e.g., Hadoop).
        \item \textbf{Data Locality}: Minimizes data transfer times by storing data close to processing locations. 
    \end{enumerate}
\end{frame}

\begin{frame}[fragile]
    \frametitle{Example: Social Media Platforms}
    Consider how social media platforms like Facebook manage user data:
    \begin{itemize}
        \item \textbf{Data Distribution}: User-generated content is stored across global data centers.
        \item \textbf{Load Balancing}: Incoming requests are distributed evenly across servers.
        \item \textbf{Redundancy}: Copies of user data are maintained in various locations to prevent loss.
    \end{itemize}
\end{frame}

\begin{frame}[fragile]
    \frametitle{Key Points to Emphasize}
    \begin{itemize}
        \item \textbf{Decentralization}: Resources and computations are spread out, reducing bottlenecks.
        \item \textbf{Inter-node Communication}: Effective communication is essential (using protocols like HTTP, gRPC).
        \item \textbf{Complexity}: Challenges include network latency, data consistency, and failure management.
    \end{itemize}
\end{frame}

\begin{frame}[fragile]
    \frametitle{Conclusion: The Future of Data Management}
    Distributed systems drive innovations in cloud computing, big data analytics, and decentralized applications. 
    Understanding these concepts is crucial for harnessing technology to solve complex problems in today’s 
    data-driven world.
\end{frame}

\begin{frame}[fragile]
    \frametitle{Next Topic Preview}
    In the following slide, we will explore 
    \textbf{"Understanding Data Processing at Scale,"} diving deeper into how distributed systems enable 
    large-scale data processing. 
\end{frame}

\begin{frame}[fragile]{Understanding Data Processing at Scale - Part 1}
    \frametitle{Introduction to Data Processing at Scale}
    \begin{block}{Overview}
        Data processing at scale refers to efficiently handling and analyzing large volumes of data in distributed systems. Traditional techniques fall short due to modern applications generating massive datasets.
    \end{block}
\end{frame}

\begin{frame}[fragile]{Understanding Data Processing at Scale - Part 2}
    \frametitle{Key Concepts}
    \begin{enumerate}
        \item \textbf{Distributed Systems}
            \begin{itemize}
                \item A network of independent computers that appears as a single system.
                \item Enhances reliability, availability, and performance.
            \end{itemize}
        \item \textbf{Scalability}
            \begin{itemize}
                \item Capability of a system to handle growth.
                \item \textit{Horizontal scalability} (adding machines) vs. \textit{Vertical scalability} (adding power).
            \end{itemize}
        \item \textbf{Data Partitioning}
            \begin{itemize}
                \item Divides datasets into manageable chunks for parallel processing.
                \item Example: Splitting user activity logs by time or user ID.
            \end{itemize}
    \end{enumerate}
\end{frame}

\begin{frame}[fragile]{Understanding Data Processing at Scale - Part 3}
    \frametitle{Key Concepts Continued}
    \begin{enumerate}
        \setcounter{enumi}{3} % To continue numbering from previous frame
        \item \textbf{Data Replication}
            \begin{itemize}
                \item Duplicating data across nodes for durability and availability.
                \item Improves read performance by serving queries from various copies.
            \end{itemize}
        \item \textbf{Data Consistency}
            \begin{itemize}
                \item Ensuring all nodes see the same data at the same time.
                \item Uses models to balance trade-offs (e.g., CAP Theorem).
            \end{itemize}
    \end{enumerate}
\end{frame}

\begin{frame}[fragile]{Understanding Data Processing at Scale - Part 4}
    \frametitle{Processing Frameworks}
    \begin{block}{MapReduce Overview}
        A programming model for processing large datasets:
        \begin{itemize}
            \item \textbf{Map}: Processes input data and produces key-value pairs.
            \item \textbf{Reduce}: Aggregates results of the map function.
        \end{itemize}
    \end{block}
    
    \begin{lstlisting}[language=Python]
def map(key, value):
    # Process input data and emit intermediate key-value pairs
    emit(key, value)

def reduce(key, values):
    # Aggregate values and emit the final result
    result = aggregate(values)
    emit(key, result)
    \end{lstlisting}
\end{frame}

\begin{frame}[fragile]{Understanding Data Processing at Scale - Part 5}
    \frametitle{Example Case Study: Analytics in E-Commerce}
    An e-commerce company processing millions of transactions:
    \begin{itemize}
        \item Identify trending products
        \item Improve customer recommendations
        \item Optimize inventory management
    \end{itemize}
    Using a distributed framework, they can:
    \begin{enumerate}
        \item Partition transaction data by product categories.
        \item Process transactions in parallel across nodes.
        \item Replicate data for backup and quicker access.
    \end{enumerate}
\end{frame}

\begin{frame}[fragile]{Understanding Data Processing at Scale - Part 6}
    \frametitle{Key Takeaways}
    \begin{itemize}
        \item Understanding data processing at scale is crucial in today's data-driven world.
        \item Distributed systems offer the necessary infrastructure for handling vast amounts of data efficiently.
        \item Techniques like partitioning, replication, and consistency models facilitate scalable data processing.
        \item Familiarity with frameworks like MapReduce is essential for working with big data.
    \end{itemize}
    By grasping these concepts, we set the stage for exploring specific frameworks and tools, such as Hadoop, in the next section.
\end{frame}

\begin{frame}[fragile]
    \frametitle{Hadoop Overview}
    \begin{block}{What is Hadoop?}
        Hadoop is an open-source framework designed for distributed storage and processing of large datasets across clusters of computers. It scales up from a single server to thousands of machines, each providing local computation and storage.
    \end{block}
    
    \begin{itemize}
        \item \textbf{Scalability:} Handles petabytes of data by adding nodes.
        \item \textbf{Fault Tolerance:} Automatically replicates data for high availability.
        \item \textbf{Cost-Effective:} Runs on commodity hardware for vast data storage.
    \end{itemize}
\end{frame}

\begin{frame}[fragile]
    \frametitle{Core Components of Hadoop}
    \begin{enumerate}
        \item \textbf{Hadoop Distributed File System (HDFS)}
            \begin{itemize}
                \item \textbf{Function:} Provides reliable, distributed storage.
                \item \textbf{Key Characteristics:}
                    \begin{itemize}
                        \item Breaks large files into smaller blocks (default 128 MB) across the cluster.
                        \item Replicates blocks (default 3 copies) for fault tolerance.
                    \end{itemize}
            \end{itemize}
        
        \item \textbf{YARN (Yet Another Resource Negotiator)}
            \begin{itemize}
                \item \textbf{Function:} Manages and schedules resources in a Hadoop cluster.
                \item \textbf{Key Components:}
                    \begin{itemize}
                        \item ResourceManager: Oversees cluster resources.
                        \item NodeManager: Manages resource usage on individual nodes.
                    \end{itemize}
            \end{itemize}
        
        \item \textbf{MapReduce}
            \begin{itemize}
                \item \textbf{Function:} Programming model for processing large datasets.
                \item \textbf{Process:}
                    \begin{itemize}
                        \item \textbf{Map Phase:} Processes input data into key-value pairs.
                        \item \textbf{Reduce Phase:} Aggregates the key-value pairs into final output.
                    \end{itemize}
            \end{itemize}
    \end{enumerate}
\end{frame}

\begin{frame}[fragile]
    \frametitle{Role in Distributed Data Processing}
    Hadoop plays a crucial role in processing large datasets by dividing tasks and running them in parallel across nodes, enabling:
    \begin{itemize}
        \item Efficient data analysis on massive scales.
        \item Quick retrieval and processing of unstructured data from various sources.
        \item Support for diverse data types, such as text, images, and videos.
    \end{itemize}

    \begin{block}{Example: Analyzing Log Files}
    \begin{itemize}
        \item \textbf{Input Data:} Millions of log entries, each entry a line in a text file.
        \item \textbf{HDFS:} Stores log files distributed across the cluster.
        \item \textbf{MapReduce:}
            \begin{itemize}
                \item \textbf{Map:} Each mapper reads a log file and emits a key-value pair like (IP address, 1).
                \item \textbf{Reduce:} The reducer sums the values for each IP to count total hits.
            \end{itemize}
    \end{itemize}
    \end{block}
\end{frame}

\begin{frame}[fragile]
  \frametitle{MapReduce Fundamentals}
  \begin{block}{What is MapReduce?}
    MapReduce is a programming model used for processing and generating large data sets with a distributed algorithm on a cluster. It was introduced by Google and efficiently handles data processing by dividing tasks into manageable parts that can be executed in parallel.
  \end{block}
\end{frame}

\begin{frame}[fragile]
  \frametitle{Key Concepts}
  \begin{enumerate}
    \item \textbf{Map Function:}
    \begin{itemize}
      \item \textbf{Purpose:} Transform input data into a set of key-value pairs.
      \item \textbf{Process:} Each input record is processed by the Map function, outputting intermediate key-value pairs.
    \end{itemize}
    
    \pause
    
    \item \textbf{Reduce Function:}
    \begin{itemize}
      \item \textbf{Purpose:} Takes intermediate key-value pairs and reduces them to a smaller set of values.
      \item \textbf{Process:} Groups all intermediate values by key and combines them to produce the final output.
    \end{itemize}
  \end{enumerate}
\end{frame}

\begin{frame}[fragile]
  \frametitle{Operational Steps}
  \begin{enumerate}
    \item \textbf{Input Data Splitting:}
    \begin{itemize}
      \item Large datasets are divided into smaller, manageable chunks (input splits).
    \end{itemize}

    \item \textbf{Mapping:}
    \begin{itemize}
      \item Each piece of data is processed by the Map function running in parallel across different nodes.
    \end{itemize}

    \item \textbf{Shuffling and Sorting:}
    \begin{itemize}
      \item The intermediate key-value pairs from Map tasks are shuffled and sorted, with pairs having the same key grouped together.
    \end{itemize}

    \item \textbf{Reducing:}
    \begin{itemize}
      \item The Reduce function processes sorted key-value pairs and produces the final output.
    \end{itemize}

    \item \textbf{Output:}
    \begin{itemize}
      \item The results from Reduce tasks are written to the distributed file system.
    \end{itemize}
  \end{enumerate}
\end{frame}

\begin{frame}[fragile]
  \frametitle{MapReduce Example Flow}
  \begin{block}{Example Flow}
    Given a dataset of sentences, here’s a simplified flow:
    \begin{enumerate}
      \item \textbf{Input:} Text: "Hello World Hello"
      \item \textbf{Mapper Output:} \{("Hello", 1), ("World", 1), ("Hello", 1)\}
      \item \textbf{Shuffle and Sort:} Grouping by key: \{("Hello", [1, 1]), ("World", [1])\}
      \item \textbf{Reducer Output:} \{("Hello", 2), ("World", 1)\}
    \end{enumerate}
  \end{block}
\end{frame}

\begin{frame}[fragile]
  \frametitle{Code Snippet Example}
  Here is a simple pseudo-code for a MapReduce word count:
  \begin{lstlisting}[language=Python]
def map_function(document):
    for word in document.split():
        emit(word, 1)

def reduce_function(word, counts):
    total_count = sum(counts)
    emit(word, total_count)
  \end{lstlisting}
\end{frame}

\begin{frame}[fragile]
  \frametitle{Conclusion}
  \begin{block}{Key Points to Emphasize}
    \begin{itemize}
      \item \textbf{Scalability:} MapReduce can efficiently handle massive datasets across many machines.
      \item \textbf{Fault Tolerance:} If a node fails, tasks can be redistributed to other nodes without losing progress.
      \item \textbf{Simplicity:} Users can focus on defining Map and Reduce functions without needing to manage hardware.
    \end{itemize}
  \end{block}

  \begin{block}{Conclusion}
    MapReduce is a powerful model for processing large datasets efficiently in a distributed manner. Understanding its fundamentals is crucial for leveraging tools like Hadoop for big data applications.
  \end{block}
\end{frame}

\begin{frame}[fragile]
    \frametitle{Spark Overview - Introduction}
    \begin{block}{Introduction to Apache Spark}
        Apache Spark is an open-source distributed computing system designed for large-scale data processing. It simplifies big data analytics through its in-memory data processing capabilities, allowing multiple tasks and operations to be executed simultaneously on a dataset.
    \end{block}
\end{frame}

\begin{frame}[fragile]
    \frametitle{Spark Overview - Architecture}
    \begin{block}{Key Components of Spark Architecture}
        \begin{itemize}
            \item \textbf{Driver Program:} The main function that runs the user's code, creating SparkContext to coordinate tasks.
            \item \textbf{Cluster Manager:} Manages resources within the cluster (options: Apache Mesos, Hadoop YARN, Spark's standalone manager).
            \item \textbf{Worker Nodes:} Machines executing tasks; each runs executors that perform data processing and storage.
            \item \textbf{Tasks:} Smallest unit of work in Spark, executed by worker nodes in parallel.
        \end{itemize}
    \end{block}
    
    \begin{block}{Spark Architecture Diagram}
        \centering
        \includegraphics[width=0.8\linewidth]{spark_architecture_diagram} % Use the proper command to include the diagram
    \end{block}
\end{frame}

\begin{frame}[fragile]
    \frametitle{Spark Overview - Advantages}
    \begin{block}{Advantages of Apache Spark over Hadoop}
        \begin{enumerate}
            \item \textbf{Speed:} 
                Spark processes data in-memory, performing analytics up to 100 times faster than Hadoop MapReduce for certain applications.
            
            \item \textbf{Ease of Use:}
                High-level APIs in Java, Scala, Python, and R, with built-in modules for SQL, streaming, machine learning, and graph processing.

            \item \textbf{Unified Platform:}
                Provides a single framework for batch processing, real-time analytics, and machine learning, simplifying the development process.

            \item \textbf{Fault Tolerance:}
                Maintains data lineage for resilient distributed datasets (RDDs), allowing recomputation of lost data.
        \end{enumerate}
    \end{block}
\end{frame}

\begin{frame}[fragile]
    \frametitle{Spark Overview - Example and Conclusion}
    \begin{block}{Code Snippet Example (Spark with Python)}
        \begin{lstlisting}[language=Python]
from pyspark import SparkContext

# Initialize Spark Context
sc = SparkContext("local", "Simple App")

# Create an RDD from a list
data = [1, 2, 3, 4, 5]
rdd = sc.parallelize(data)

# Perform a transformation and an action
squared_rdd = rdd.map(lambda x: x ** 2)
print(squared_rdd.collect())  # Output: [1, 4, 9, 16, 25]

# Stop the Spark Context
sc.stop()
        \end{lstlisting}
    \end{block}

    \begin{block}{Conclusion}
        Apache Spark is a powerful framework for processing big data efficiently. Understanding its architecture and advantages can help leverage its capabilities in distributed data management and analytics.
    \end{block}
\end{frame}

\begin{frame}[fragile]
  \frametitle{Overview of Data Ingestion}
  Data ingestion refers to the process of collecting data from various sources and moving it into a storage system (like a data lake or data warehouse) where it can be processed and analyzed. 
  In distributed systems, efficient data ingestion is crucial as it must handle large volumes of data from multiple sources, ensuring low latency and high throughput.
\end{frame}

\begin{frame}[fragile]
  \frametitle{Key Ingestion Techniques}
  \begin{enumerate}
    \item \textbf{Batch Ingestion}
      \begin{itemize}
        \item \textbf{Description:} Data is collected and processed in large blocks at scheduled intervals.
        \item \textbf{Use Cases:} Financial reports, end-of-day processing.
        \item \textbf{Example:} A retail company consolidating sales data from all stores every night.
      \end{itemize}
      \begin{block}{Illustration}
      \begin{verbatim}
   +-----------------------------------+
   |           Batch Ingestion         |
   |                                   |
   |   1. Collect Data (e.g., hourly)  |
   |   2. Store in Batches             |
   |   3. Process Periodically          |
   +-----------------------------------+
      \end{verbatim}
      \end{block}
      
    \item \textbf{Streaming Ingestion}
      \begin{itemize}
        \item \textbf{Description:} Continuous real-time ingestion of data as it arrives.
        \item \textbf{Use Cases:} Social media feeds, sensor data, real-time analytics.
        \item \textbf{Example:} Monitoring IoT device data as it comes in.
      \end{itemize}
      \textbf{Key Point:} This technique requires a robust infrastructure to handle the continuous flow and ensure data integrity.
\end{enumerate}
\end{frame}

\begin{frame}[fragile]
  \frametitle{Streaming Ingestion - Code Snippet}
  \begin{block}{Code Snippet}
  \begin{lstlisting}[language=Python]
from pyspark import SparkContext
from pyspark.streaming import StreamingContext

sc = SparkContext("local[2]", "NetworkWordCount")
ssc = StreamingContext(sc, 1)  # 1 second batch interval
lines = ssc.socketTextStream("localhost", 9999)
words = lines.flatMap(lambda line: line.split(" "))
wordCounts = words.map(lambda word: (word, 1)).reduceByKey(lambda a, b: a + b)
wordCounts.pprint()
  \end{lstlisting}
  \end{block}

  \begin{enumerate}
    \setcounter{enumi}{2}
    \item \textbf{Micro-Batch Ingestion}
      \begin{itemize}
        \item \textbf{Description:} A hybrid approach that processes incoming data in small chunks more frequently than in traditional batch processing.
        \item \textbf{Use Cases:} Near real-time analytics with less latency than traditional batch.
        \item \textbf{Example:} Apache Spark Streaming, where data is ingested and processed every few seconds.
      \end{itemize}
  \end{enumerate}
\end{frame}

\begin{frame}[fragile]
  \frametitle{Choosing the Right Technique}
  \begin{itemize}
    \item \textbf{Data Volume:} High volumes favor batch or micro-batch; lower volumes can use streaming.
    \item \textbf{Latency Requirements:} Real-time needs necessitate streaming ingestion.
    \item \textbf{Complexity and Cost:} Evaluate infrastructure costs for the desired ingestion speed and volume.
  \end{itemize}
\end{frame}

\begin{frame}[fragile]
  \frametitle{Conclusion & Key Takeaways}
  Understanding data ingestion techniques is crucial for leveraging distributed systems effectively. 
  By selecting the appropriate method for your data sources and processing needs, organizations can enhance data availability and accelerate insights.
  
  \textbf{Key Takeaways:}
  \begin{itemize}
    \item Batch, Streaming, and Micro-Batch are the primary ingestion techniques.
    \item Each method has unique use cases based on requirements for volume, speed, and complexity.
    \item Efficient data ingestion supports powerful analytics capabilities in distributed systems.
  \end{itemize}
\end{frame}

\begin{frame}[fragile]
  \frametitle{Next Steps}
  Transition into Data Processing Strategies, examining how to handle ingested data effectively in a distributed environment.
\end{frame}

\begin{frame}[fragile]
  \frametitle{Data Processing Strategies - Introduction}
  \begin{block}{Introduction to Data Processing Strategies}
    Data processing in distributed systems is crucial for handling vast amounts of data. 
    The two primary strategies are \textbf{Batch Processing} and \textbf{Stream Processing}. 
    Understanding their differences and applications is essential for optimizing data workflows.
  \end{block}
\end{frame}

\begin{frame}[fragile]
  \frametitle{Data Processing Strategies - Batch Processing}
  \begin{block}{1. Batch Processing}
    Batch processing involves accumulating a large volume of data and processing it all at once.
  \end{block}
  \begin{itemize}
    \item \textbf{Characteristics:}
    \begin{itemize}
      \item Suitable for large data sets.
      \item Delayed results after data collection.
      \item Efficient for intensive tasks without real-time analysis.
    \end{itemize}
    
    \item \textbf{Use Cases:}
    \begin{itemize}
      \item Monthly sales reporting.
      \item Data warehousing.
      \item ETL (Extract, Transform, Load) processes.
    \end{itemize}
    
    \item \textbf{Example:} 
    An e-commerce company processes daily transaction logs at midnight to generate insights on sales and inventory.
  \end{itemize}
\end{frame}

\begin{frame}[fragile]
  \frametitle{Data Processing Strategies - Stream Processing}
  \begin{block}{2. Stream Processing}
    Stream processing handles data in real-time as it arrives.
  \end{block}
  \begin{itemize}
    \item \textbf{Characteristics:}
    \begin{itemize}
      \item Continuous processing on-the-fly.
      \item Immediate results for real-time analytics.
      \item Handles smaller, discrete data points.
    \end{itemize}
    
    \item \textbf{Use Cases:}
    \begin{itemize}
      \item Real-time fraud detection.
      \item Social media sentiment analysis.
      \item Monitoring IoT devices.
    \end{itemize}
    
    \item \textbf{Example:} 
    A financial institution uses stream processing for real-time transaction monitoring to quickly identify fraud.
  \end{itemize}
\end{frame}

\begin{frame}[fragile]
  \frametitle{Key Differences Between Batch and Stream Processing}
  \begin{table}[ht]
    \centering
    \begin{tabular}{|l|l|l|}
      \hline
      \textbf{Feature} & \textbf{Batch Processing} & \textbf{Stream Processing} \\
      \hline
      Data Timing & Scheduled intervals & Continuous flow \\
      Latency & Higher (minutes to hours) & Low (milliseconds to seconds) \\
      Storage & Writes to disk after batch complete & Processes in memory, yielding data immediately \\
      Complexity & Typically simpler & More complex; requires real-time handling \\
      \hline
    \end{tabular}
  \end{table}
\end{frame}

\begin{frame}[fragile]
  \frametitle{When to Use Which Strategy}
  \begin{itemize}
    \item \textbf{Batch Processing} is ideal when:
    \begin{itemize}
      \item Real-time insights are not critical.
      \item Data volume is large and does not require continuous updates.
    \end{itemize}

    \item \textbf{Stream Processing} is best for:
    \begin{itemize}
      \item Real-time decision-making.
      \item Immediate reaction to data events.
    \end{itemize}
  \end{itemize}
\end{frame}

\begin{frame}[fragile]
  \frametitle{Conclusion}
  \begin{block}{Conclusion}
    Choosing the right data processing strategy is fundamental to meeting business needs and leveraging data effectively. 
    Understanding the characteristics and differences between batch and stream processing helps organizations optimize their data architectures.
  \end{block}
\end{frame}

\begin{frame}[fragile]
  \frametitle{Code Snippets}
  \begin{block}{Example: Batch Processing}
    \begin{lstlisting}[language=Python]
import pandas as pd

# Batch Process Example - Monthly Sales Reporting
data = pd.read_csv('sales_data.csv')
monthly_report = data.groupby(['month']).agg({'sales': 'sum'})
monthly_report.to_csv('monthly_report.csv')
    \end{lstlisting}
  \end{block}
  
  \begin{block}{Example: Stream Processing}
    \begin{lstlisting}[language=Python]
from kafka import KafkaConsumer

# Stream Process Example - Real-Time Transaction Monitoring
consumer = KafkaConsumer('transactions', bootstrap_servers='localhost:9092')

for message in consumer:
    process_transaction(message.value)
    \end{lstlisting}
  \end{block}
\end{frame}

\begin{frame}[fragile]
    \frametitle{Distributed Database Architectures - Introduction}
    \begin{itemize}
        \item Distributed databases are not confined to a single location.
        \item They are spread across multiple physical locations.
        \item The slide explores design and implementation, focusing on scalability.
    \end{itemize}
\end{frame}

\begin{frame}[fragile]
    \frametitle{Distributed Database Architectures - Key Concepts}
    \begin{enumerate}
        \item \textbf{Definition of Distributed Databases}
            \begin{itemize}
                \item Collection of interconnected databases across various locations.
                \item Provides a unified interface for data access.
            \end{itemize}
        
        \item \textbf{Types of Distributed Database Architectures}
            \begin{itemize}
                \item \textbf{Homogeneous}
                    \begin{itemize}
                        \item All nodes use the same DBMS and data structure. 
                        \item \textit{Example}: A network of servers using MySQL.
                    \end{itemize}
                \item \textbf{Heterogeneous}
                    \begin{itemize}
                        \item Nodes may use different DBMS and structures.
                        \item \textit{Example}: PostgreSQL, MongoDB, and Oracle DB.
                    \end{itemize}
            \end{itemize}
    \end{enumerate}
\end{frame}

\begin{frame}[fragile]
    \frametitle{Distributed Database Architectures - Design Considerations}
    \begin{itemize}
        \item \textbf{Data Distribution}
            \begin{itemize}
                \item Affects performance; strategies include:
                    \begin{itemize}
                        \item \textbf{Horizontal Partitioning (Sharding)}: Distributes rows.
                        \item \textbf{Vertical Partitioning}: Splits columns.
                    \end{itemize}
            \end{itemize}

        \item \textbf{Replication}
            \begin{itemize}
                \item Ensures high availability and fault tolerance.
                \item \textit{Example}: Two nodes replicating essential data.
            \end{itemize}
    \end{itemize}
\end{frame}

\begin{frame}[fragile]
    \frametitle{Scalability Aspects of Distributed Databases}
    \begin{itemize}
        \item Scalability is crucial for managing increased workloads.
        \item \textbf{Types of Scalability}
            \begin{itemize}
                \item \textbf{Vertical Scalability (Scaling Up)}: Adding resources to existing machines. 
                    \begin{itemize}
                        \item Limited by hardware constraints.
                    \end{itemize}
                \item \textbf{Horizontal Scalability (Scaling Out)}: Adding machines to the system.
                    \begin{itemize}
                        \item More flexible and preferred in distributed systems.
                    \end{itemize}
            \end{itemize}
    \end{itemize}
\end{frame}

\begin{frame}[fragile]
    \frametitle{Challenges and Examples in Distributed Databases}
    \begin{itemize}
        \item \textbf{Challenges in Scaling}
            \begin{itemize}
                \item \textbf{Consistency vs. Availability}:
                    \begin{itemize}
                        \item CAP theorem limits guarantees to two of three aspects.
                        \item \textit{Example}: In the event of a network partition, inconsistency may be allowed for availability.
                    \end{itemize}
                \item \textbf{Network Latency}: 
                    \begin{itemize}
                        \item More nodes can increase latency in data retrieval.
                    \end{itemize}
            \end{itemize}
        
        \item \textbf{Example: Google Spanner}
            \begin{itemize}
                \item A globally distributed database utilizing synchronous replication and sharding.
                \item Achieves consistency and availability; scales out seamlessly.
            \end{itemize}
    \end{itemize}
\end{frame}

\begin{frame}[fragile]
    \frametitle{Conclusion and Key Points}
    \begin{itemize}
        \item Understanding distributed database architectures is essential for distributed systems.
        \item \textbf{Key Points to Remember}
            \begin{itemize}
                \item Types: Homogeneous vs. Heterogeneous.
                \item Importance of data distribution and replication.
                \item Scalability achieved through horizontal scaling, while facing CAP theorem challenges.
            \end{itemize}
    \end{itemize}
\end{frame}

\begin{frame}[fragile]
    \frametitle{Query Processing in Distributed Systems - Overview}
    \begin{block}{Optimization Techniques}
        Optimization techniques for executing queries on distributed databases are critical for minimizing latency and maximizing throughput.
    \end{block}
    \begin{itemize}
        \item Efficient query processing across multiple nodes is essential.
        \item Key strategies include:
        \begin{itemize}
            \item Query Decomposition
            \item Data Locality Optimization
            \item Predicate Pushdown
            \item Cost-Based Optimization
            \item Parallel Query Processing
        \end{itemize}
    \end{itemize}
\end{frame}

\begin{frame}[fragile]
    \frametitle{Query Processing in Distributed Systems - Techniques}
    \begin{enumerate}
        \item \textbf{Query Decomposition}
            \begin{itemize}
                \item Definition: Break complex queries into smaller sub-queries.
                \item Example: Retrieve orders and then join with customer data.
                \item Benefits: Reduces complexity and allows parallel execution.
            \end{itemize}
        
        \item \textbf{Data Locality Optimization}
            \begin{itemize}
                \item Definition: Place data closer to the computation.
                \item Techniques: 
                  \begin{itemize}
                      \item Replication
                      \item Partitioning
                  \end{itemize}
                \item Example: Placing region-specific data closer to users to reduce response time.
            \end{itemize}
    \end{enumerate}
\end{frame}

\begin{frame}[fragile]
    \frametitle{Query Processing in Distributed Systems - Techniques Continued}
    \begin{enumerate}
        \setcounter{enumi}{2}  % Continue numbering
        \item \textbf{Predicate Pushdown}
            \begin{itemize}
                \item Definition: Apply filter conditions early in execution.
                \item Example: Push `WHERE quantity > 0` down to the data retrieval step.
            \end{itemize}
        
        \item \textbf{Cost-Based Optimization}
            \begin{itemize}
                \item Definition: Use statistics to choose the efficient execution plan.
                \item Key Elements: Query execution time, resource utilization, network traffic.
                \item Illustration:
                \begin{lstlisting}
                Execution Plan:
                1. Direct access to the index (low cost)
                2. Full table scan (high cost)
                Preferred Plan: 1
                \end{lstlisting}
            \end{itemize}
        
        \item \textbf{Parallel Query Processing}
            \begin{itemize}
                \item Definition: Distribute execution across multiple units.
                \item Example:
                \begin{lstlisting}
                Query: SELECT SUM(sales) 
                FROM sales_table 
                WHERE year = 2023 
                Groups:
                - Node A processes data for Q1
                - Node B processes data for Q2
                \end{lstlisting}
            \end{itemize}
    \end{enumerate}
\end{frame}

\begin{frame}[fragile]
    \frametitle{Key Points to Emphasize}
    \begin{itemize}
        \item \textbf{Scalability}: Optimized query processing enhances scalability.
        \item \textbf{Performance}: Each technique improves response times and system performance.
        \item \textbf{Data Distribution}: Understanding data distribution impacts query performance.
    \end{itemize}
    \begin{block}{Conclusion}
        Mastering optimization techniques ensures effective query handling, leading to improved user experiences and resource utilization.
    \end{block}
\end{frame}

\begin{frame}
    \frametitle{Proficiency in Industry Tools - Introduction}
    \begin{block}{Overview}
        Proficiency in key industry-standard tools is crucial for effective data management and application deployment. This slide introduces four essential tools:
    \end{block}
    \begin{itemize}
        \item Amazon Web Services (AWS)
        \item Kubernetes
        \item PostgreSQL
        \item NoSQL Databases
    \end{itemize}
\end{frame}

\begin{frame}[fragile]
    \frametitle{Proficiency in Industry Tools - AWS}
    \begin{block}{Amazon Web Services (AWS)}
        \begin{itemize}
            \item \textbf{Overview}: Comprehensive cloud services platform with over 200 services.
            \item \textbf{Key Features}:
            \begin{itemize}
                \item Scalability: Adjust resources based on demand.
                \item Cost-Effectiveness: Pay-as-you-go pricing.
                \item Security: Robust protection for data.
            \end{itemize}
            \item \textbf{Example Use Case}: Deploy a web application using AWS EC2 and store data in AWS S3.
        \end{itemize}
    \end{block}
\end{frame}

\begin{frame}[fragile]
    \frametitle{Proficiency in Industry Tools - Kubernetes}
    \begin{block}{Kubernetes}
        \begin{itemize}
            \item \textbf{Overview}: An open-source platform for automating deployment and management of application containers.
            \item \textbf{Key Features}:
            \begin{itemize}
                \item Container Orchestration: Manages the lifecycle of containers.
                \item Load Balancing: Distributes traffic for optimal performance.
                \item Self-Healing: Automatically restarts or replaces failed containers.
            \end{itemize}
            \item \textbf{Example Use Case}: Manage microservices for an e-commerce application.
        \end{itemize}
    \end{block}
    \begin{lstlisting}[language=yaml, caption=Sample Kubernetes Deployment YAML]
apiVersion: apps/v1
kind: Deployment
metadata:
  name: web-app
spec:
  replicas: 3
  selector:
    matchLabels:
      app: web-app
  template:
    metadata:
      labels:
        app: web-app
    spec:
      containers:
      - name: web-app
        image: myorg/web-app:latest
        ports:
        - containerPort: 80
    \end{lstlisting}
\end{frame}

\begin{frame}[fragile]
    \frametitle{Proficiency in Industry Tools - PostgreSQL}
    \begin{block}{PostgreSQL}
        \begin{itemize}
            \item \textbf{Overview}: Advanced, open-source relational database management system (RDBMS).
            \item \textbf{Key Features}:
            \begin{itemize}
                \item ACID Compliance: Ensures transaction reliability.
                \item Rich Data Types: Supports JSON, arrays, and custom types.
                \item Extensibility: Enables custom functions and data types.
            \end{itemize}
            \item \textbf{Example Use Case}: Store relational data for analytics in a data warehouse using PostgreSQL.
        \end{itemize}
    \end{block}
    \begin{lstlisting}[language=sql, caption=Example of creating a table in PostgreSQL]
CREATE TABLE users (
  id SERIAL PRIMARY KEY,
  username VARCHAR(50) UNIQUE NOT NULL,
  created_at TIMESTAMP DEFAULT CURRENT_TIMESTAMP
);
    \end{lstlisting}
\end{frame}

\begin{frame}
    \frametitle{Proficiency in Industry Tools - NoSQL Databases}
    \begin{block}{NoSQL Databases}
        \begin{itemize}
            \item \textbf{Overview}: Designed for unstructured or semi-structured data, allowing flexible data modeling.
            \item \textbf{Key Features}:
            \begin{itemize}
                \item Schema Flexibility: Adaptable to various data formats.
                \item Horizontal Scalability: Scale out by adding more servers.
            \end{itemize}
            \item \textbf{Example Types}:
            \begin{itemize}
                \item Document Stores (e.g., MongoDB).
                \item Key-Value Stores (e.g., Redis).
            \end{itemize}
            \item \textbf{Example Use Case}: Use MongoDB for a content management system with varied article structures.
        \end{itemize}
    \end{block}
\end{frame}

\begin{frame}
    \frametitle{Proficiency in Industry Tools - Key Points}
    \begin{itemize}
        \item Familiarity with these tools enhances employability in data engineering and DevOps roles.
        \item Understanding each tool's context aids in selecting the right solution for specific issues.
        \item Real-world implementations often combine multiple tools for optimal performance and scalability.
    \end{itemize}
\end{frame}

\begin{frame}
    \frametitle{Proficiency in Industry Tools - Conclusion}
    \begin{block}{Conclusion}
        Mastering AWS, Kubernetes, PostgreSQL, and NoSQL databases equips you to navigate the complexities of distributed systems and build resilient and scalable applications.
    \end{block}
    \begin{block}{Next Steps}
        We will explore the fundamentals of developing data pipelines in cloud environments.
    \end{block}
\end{frame}

\begin{frame}[fragile]
    \frametitle{Developing Data Pipelines}
    % Overview
    \begin{block}{Overview}
        Data pipelines are essential components of data management in cloud environments. 
        They facilitate the movement, processing, and storage of data, enabling organizations to derive insights and make informed decisions. 
        This presentation introduces the fundamentals of creating and managing data pipelines, focusing on key concepts, technologies, and best practices.
    \end{block}
\end{frame}

\begin{frame}[fragile]
    \frametitle{What is a Data Pipeline?}
    % Definition of Data Pipeline
    A data pipeline consists of several processing steps:
    \begin{enumerate}
        \item \textbf{Data Ingestion}: Importing data from various sources (e.g., databases, APIs).
        \item \textbf{Data Processing}: Transforming data to fit operational needs (e.g., filtering, aggregating).
        \item \textbf{Data Storage}: Storing processed data in databases or data warehouses for querying and analysis.
        \item \textbf{Data Analysis}: Extracting insights through analytics and visualization tools.
    \end{enumerate}
\end{frame}

\begin{frame}[fragile]
    \frametitle{Key Concepts}
    % ETL vs ELT
    \begin{block}{ETL vs. ELT}
        \begin{itemize}
            \item \textbf{ETL (Extract, Transform, Load)}: Data is extracted from the source, transformed, and then loaded into storage.
            \item \textbf{ELT (Extract, Load, Transform)}: Data is extracted and loaded into storage before transformation, used in scalable cloud environments.
        \end{itemize}
    \end{block}

    % Batch vs Real-time Processing
    \begin{block}{Batch vs. Real-time Processing}
        \begin{itemize}
            \item \textbf{Batch Processing}: Data collected and processed at specified intervals (e.g., nightly reports).
            \item \textbf{Real-time Processing}: Data processed continuously as it arrives (e.g., live analytics).
        \end{itemize}
    \end{block}
\end{frame}

\begin{frame}[fragile]
    \frametitle{Cloud Tools for Data Pipelines}
    % Cloud Tools
    \begin{itemize}
        \item \textbf{AWS Lambda}: Enables serverless data processing without server management.
        \item \textbf{Apache Airflow}: An open-source tool for scheduling and orchestrating complex workflows.
        \item \textbf{Apache Kafka}: A distributed streaming platform for handling real-time data feeds.
    \end{itemize}
\end{frame}

\begin{frame}[fragile]
    \frametitle{Example of a Data Pipeline}
    % Example Scenario
    \textbf{Scenario: A retail company wants to analyze customer purchases.}
    \begin{enumerate}
        \item \textbf{Ingestion}: Data from POS systems and online stores is collected.
        \item \textbf{Processing}: The pipeline filters fraudulent transactions and aggregates data by product category.
        \item \textbf{Storage}: The cleaned data is stored in Amazon Redshift for analysis.
        \item \textbf{Analysis}: BI tools generate reports and dashboards for marketing insights.
    \end{enumerate}
\end{frame}

\begin{frame}[fragile]
    \frametitle{Best Practices}
    % Best Practices
    \begin{enumerate}
        \item \textbf{Modularity}: Design pipelines as a series of modular steps for easy updates and maintenance.
        \item \textbf{Monitoring \& Alerting}: Implement monitoring to track performance and set up alerts for failures.
        \item \textbf{Data Quality Checks}: Regularly validate data at various stages to ensure accuracy and completeness.
    \end{enumerate}
\end{frame}

\begin{frame}[fragile]
    \frametitle{Code Snippet (Python Example)}
    % Code Example
    \begin{lstlisting}[language=Python]
import pandas as pd

# Simple ETL example
# Extract
data = pd.read_csv('sales_data.csv')

# Transform
data_cleaned = data[data['amount'] > 0]  # Filter out invalid entries

# Load
data_cleaned.to_sql('cleaned_sales', con=database_connection)
    \end{lstlisting}
\end{frame}

\begin{frame}[fragile]
    \frametitle{Key Points to Remember}
    \begin{itemize}
        \item Data pipelines automate data flow, enhancing efficiency and decision-making.
        \item Choosing the right processing approach (ETL vs. ELT) depends on the use case.
        \item Leveraging cloud tools simplifies the deployment and management of data pipelines.
    \end{itemize}
\end{frame}

\begin{frame}[fragile]
    \frametitle{Conclusion}
    % Conclusion
    Understanding how to develop and manage data pipelines is critical for efficient data management and analytics in cloud environments. 
    As students progress, they will explore further concepts related to collaboration in data projects, building upon the foundations laid here.
\end{frame}

\begin{frame}[fragile]
    \frametitle{Teamwork in Data Projects}
    \begin{block}{Importance of Collaboration}
        Collaboration is a cornerstone of successful data projects. Here’s why teamwork is essential:
    \end{block}
    \begin{enumerate}
        \item \textbf{Diverse Skill Sets}
        \begin{itemize}
            \item Data projects require expertise in various domains.
            \item A collaborative team ensures a well-rounded approach.
        \end{itemize}

        \item \textbf{Innovative Problem-Solving}
        \begin{itemize}
            \item Collaborative teams generate innovative ideas.
            \item Fosters creative solutions to challenges.
        \end{itemize}
    \end{enumerate}
\end{frame}

\begin{frame}[fragile]
    \frametitle{Project Management for Scalability}
    Effective project management is crucial for maintaining scalability:
    \begin{enumerate}
        \item \textbf{Agile Methodology}
        \begin{itemize}
            \item Agile approaches like Scrum allow adaptation to changes.
            \item Short sprints promote accountability.
        \end{itemize}

        \item \textbf{Version Control}
        \begin{itemize}
            \item Tools like Git manage changes in codebases.
            \item Ensures all members work on the latest version.
        \end{itemize}
    \end{enumerate}
    \begin{block}{Code Snippet Example}
        \begin{lstlisting}[language=bash]
git clone https://github.com/username/project.git
git checkout -b feature-branch
git commit -m "Add new data processing feature"
git push origin feature-branch
        \end{lstlisting}
    \end{block}
\end{frame}

\begin{frame}[fragile]
    \frametitle{Communication Tools and Conclusion}
    Utilizing communication tools enhances team efficiency:
    \begin{itemize}
        \item \textbf{Slack or Microsoft Teams}: For real-time communication.
        \item \textbf{Trello or JIRA}: To manage tasks and track progress.
    \end{itemize}

    \begin{block}{Key Points to Emphasize}
        \begin{itemize}
            \item \textbf{Synergy}: Collective output is greater than individual contributions.
            \item \textbf{Clear Roles}: Defines responsibilities to reduce confusion.
            \item \textbf{Regular Check-ins}: Promotes transparency and quick pivots.
        \end{itemize}
    \end{block}

    \begin{block}{Conclusion}
        Teamwork is essential for scalable, efficient, and resilient data processing solutions.
    \end{block}
\end{frame}

\begin{frame}[fragile]
  \frametitle{Critical Thinking and Troubleshooting}
  \begin{block}{Overview}
    Distributed systems often bring complexities that can lead to various issues in data management. Developing critical thinking and troubleshooting skills is essential for effectively resolving these challenges.
  \end{block}
\end{frame}

\begin{frame}[fragile]
  \frametitle{Key Concepts}
  \begin{enumerate}
    \item \textbf{Critical Thinking in Data Management}
      \begin{itemize}
        \item \textbf{Definition}: Ability to think clearly and rationally; essential for evaluating information and making evidence-based decisions.
        \item \textbf{Application}: Guides identification of problems, hypothesis formation, and evaluation of solutions during troubleshooting.
      \end{itemize}
      
    \item \textbf{Troubleshooting Steps}
      \begin{itemize}
        \item \textbf{Identification}: Recognize issues (e.g., slow response times, data inconsistencies).
        \item \textbf{Analysis}: Investigate the root cause.
          \begin{itemize}
            \item Logs Analysis: Check system logs for warnings or errors.
            \item Performance Monitoring: Use tools to check resource utilization (CPU, memory, etc.).
          \end{itemize}
        \item \textbf{Solution Generation}: Brainstorm potential fixes considering feasibility and impact.
        \item \textbf{Implementation and Testing}: Apply the solution, monitor results, and document the process for future reference.
      \end{itemize}
  \end{enumerate}
\end{frame}

\begin{frame}[fragile]
  \frametitle{Example Scenario}
  \begin{block}{Problem Statement}
    A distributed data storage system is experiencing high latency.
  \end{block}
  \begin{enumerate}
    \item \textbf{Identification}: Alerts indicate increased response times beyond acceptable thresholds.
    \item \textbf{Analysis}: Examine logs and find a spike in resource usage by a specific service.
    \item \textbf{Solution Generation}: Consider load balancing or query optimization.
    \item \textbf{Implementation}: Adjust server configurations and test performance.
  \end{enumerate}
\end{frame}

\begin{frame}[fragile]
  \frametitle{Key Points to Emphasize}
  \begin{itemize}
    \item \textbf{Proactive Monitoring}: Regular monitoring can preempt many issues.
    \item \textbf{Document Everything}: Record every step for future reference.
    \item \textbf{Collaborate}: Work with team members for diverse perspectives.
  \end{itemize}
\end{frame}

\begin{frame}[fragile]
  \frametitle{Quick Tips for Effective Troubleshooting}
  \begin{itemize}
    \item \textbf{Maintain a Check-list}: Create a checklist of common issues and their resolutions.
    \item \textbf{Follow a Systematic Approach}: Methodically go through identification, analysis, solution generation, and testing.
    \item \textbf{Stay Curious}: Cultivate a mindset of inquiry; asking questions leads to better solutions.
  \end{itemize}
\end{frame}

\begin{frame}[fragile]
    \frametitle{Ethical Considerations in Data Management - Overview}
    \begin{block}{Overview}
        Ethical considerations in data management are crucial for ensuring data privacy, integrity, and trustworthiness. Understanding the ethical implications becomes essential for professionals as we increasingly depend on distributed systems for storing and processing data.
    \end{block}
\end{frame}

\begin{frame}[fragile]
    \frametitle{Ethical Considerations in Data Management - Key Concepts}
    \begin{enumerate}
        \item \textbf{Data Privacy}
            \begin{itemize}
                \item \textbf{Definition}: The right of individuals to control how their personal information is collected and used.
                \item \textbf{Example}: Companies collecting user data without consent, such as social media platforms using personal data for targeted advertising, contravene ethical practices.
            \end{itemize}
    
        \item \textbf{Data Integrity}
            \begin{itemize}
                \item \textbf{Definition}: Maintaining and assuring the accuracy and consistency of data over its entire lifecycle.
                \item \textbf{Importance}: Ethical data management practices require that data is reliable and secure, e.g., financial transactions.
            \end{itemize}

        \item \textbf{Informed Consent}
            \begin{itemize}
                \item \textbf{Definition}: The process of obtaining permission from individuals before collecting or utilizing their data.
                \item \textbf{Best Practice}: Clear communication about data collection, usage, and duration is vital.
            \end{itemize}
    \end{enumerate}
\end{frame}

\begin{frame}[fragile]
    \frametitle{Ethical Considerations in Data Management - Best Practices and Case Study}
    \begin{block}{Best Practices}
        \begin{itemize}
            \item \textbf{Data Minimization}: Collect only necessary data for defined purposes.
            \item \textbf{Transparency}: Organizations should be open about data management practices and policies.
            \item \textbf{Security Measures}: Implement robust security protocols to protect data from unauthorized access.
        \end{itemize}
    \end{block}

    \begin{block}{Example Case Study}
        \textbf{Cambridge Analytica Scandal}: Highlighted the misuse of personal data without consent, emphasizing the importance of ethical practices in data management.
    \end{block}
\end{frame}

\begin{frame}[fragile]
    \frametitle{Ethical Considerations in Data Management - Guidelines and Conclusion}
    \begin{block}{Ethical Guidelines}
        \begin{enumerate}
            \item \textbf{Adhere to Legislation}: Follow laws such as GDPR and CCPA for data protection.
            \item \textbf{Conduct Ethical Audits}: Assess data practices regularly for compliance with ethical standards.
            \item \textbf{Engage Stakeholders}: Involve stakeholders in discussions about data management policies.
        \end{enumerate}
    \end{block}

    \begin{block}{Conclusion}
        By focusing on ethical considerations in data management, organizations enhance credibility and trust, improving relationships with clients and stakeholders.
    \end{block}
\end{frame}

\begin{frame}[fragile]
  \frametitle{Course Wrap-Up: Summary of Key Points in Distributed Systems and Data Management}
  As we conclude our exploration of distributed systems and data management, let’s summarize the key concepts and principles we have covered over the past weeks.
\end{frame}

\begin{frame}[fragile]
  \frametitle{1. Distributed Systems}
  \begin{itemize}
    \item \textbf{Definition}: A distributed system is a network of independent computers that appears to its users as a single coherent system.
    \item \textbf{Characteristics}:
      \begin{itemize}
        \item \textbf{Scalability}: Systems can grow by adding nodes without affecting performance.
        \item \textbf{Fault Tolerance}: The system continues functioning in the event of a node failure.
        \item \textbf{Concurrency}: Multiple processes can operate simultaneously.
      \end{itemize}
    \item \textbf{Example}: Web services like Google Search utilize distributed systems.
  \end{itemize}
\end{frame}

\begin{frame}[fragile]
  \frametitle{2. Key Architectures}
  \begin{itemize}
    \item \textbf{Client-Server Architecture}: Clients request resources, and servers provide them.
    \item \textbf{Peer-to-Peer (P2P) Architecture}: All nodes have equal responsibilities, sharing resources directly.
    \item \textbf{Microservices}: Structures an application as a collection of loosely coupled services.
  \end{itemize}
\end{frame}

\begin{frame}[fragile]
  \frametitle{3. Data Management Approaches}
  \begin{itemize}
    \item \textbf{Data Distribution}:
      \begin{itemize}
        \item \textbf{Sharding}: Dividing data into smaller parts across various nodes.
        \item \textbf{Replication}: Maintaining copies of data on multiple nodes for availability.
      \end{itemize}
    \item \textbf{Consistency Models}:
      \begin{itemize}
        \item \textbf{Strong Consistency}: Immediate updates visible to all nodes.
        \item \textbf{Eventual Consistency}: Guarantees updates will propagate to all nodes eventually.
      \end{itemize}
  \end{itemize}
\end{frame}

\begin{frame}[fragile]
  \frametitle{4. Ethical Considerations and Real-world Applications}
  \begin{itemize}
    \item \textbf{Ethical Considerations}:
      \begin{itemize}
        \item Importance of ethics in data management focusing on privacy, transparency, and integrity.
      \end{itemize}
    \item \textbf{Real-world Applications}:
      \begin{itemize}
        \item \textbf{Social Media Platforms}: Manage large user-generated content.
        \item \textbf{Financial Services}: Ensure secure transaction processing across locations.
      \end{itemize}
  \end{itemize}
\end{frame}

\begin{frame}[fragile]
  \frametitle{Key Takeaways and Next Steps}
  \begin{itemize}
    \item Understanding the architecture and principles of distributed systems is essential for modern software development.
    \item Effective data management is critical for reliability, scalability, and user trust.
    \item Ethical considerations should guide data management practices.
  \end{itemize}
  \textbf{Next Steps}: Prepare for our next session on future trends in distributed data processing.
\end{frame}

\begin{frame}[fragile]
  \frametitle{Future Trends in Distributed Data Processing - Overview}
  As data continues to grow exponentially, distributed data processing is evolving rapidly. This slide discusses the key emerging trends and how they may reshape the industry landscape.
\end{frame}

\begin{frame}[fragile]
  \frametitle{Trends in Distributed Data Processing - Part 1}
  \begin{enumerate}
    \item \textbf{Serverless Computing}
    \begin{itemize}
      \item \textbf{Description}: Serverless architectures allow developers to focus on writing code without managing server infrastructure.
      \item \textbf{Impact}: Facilitates automatic scaling, faster deployments, and reduced operational costs.
      \item \textbf{Example}: AWS Lambda allows you to run code in response to events, such as uploads to Amazon S3, eliminating the need to provision servers upfront.
    \end{itemize}

    \item \textbf{Edge Computing}
    \begin{itemize}
      \item \textbf{Description}: Processing data closer to where it is generated (e.g., IoT devices) rather than relying solely on centralized cloud servers.
      \item \textbf{Impact}: Reduces latency, optimizes bandwidth, and enhances reliability in real-time applications.
      \item \textbf{Example}: Autonomous vehicles analyze data from on-board sensors in real-time to make split-second decisions without waiting for cloud processing.
    \end{itemize}
  \end{enumerate}
\end{frame}

\begin{frame}[fragile]
  \frametitle{Trends in Distributed Data Processing - Part 2}
  \begin{enumerate}
    \setcounter{enumi}{2}
    \item \textbf{Multi-Cloud Strategies}
    \begin{itemize}
      \item \textbf{Description}: Organizations are leveraging multiple cloud services to avoid vendor lock-in and improve resilience.
      \item \textbf{Impact}: Increases flexibility in resource allocation and strengthens disaster recovery strategies.
      \item \textbf{Example}: A company might use Amazon Web Services (AWS) for machine learning, Google Cloud for data analytics, and Microsoft Azure for application hosting.
    \end{itemize}

    \item \textbf{AI and Machine Learning Integration}
    \begin{itemize}
      \item \textbf{Description}: Incorporating AI and ML into data processing pipelines to automate data analysis and enhance decision-making.
      \item \textbf{Impact}: Increases efficiency, improves accuracy in predictions, and provides deeper insights.
      \item \textbf{Example}: Predictive analytics platforms utilize machine learning algorithms to analyze trends and forecast future data patterns.
    \end{itemize}

    \item \textbf{Enhanced Data Security Mechanisms}
    \begin{itemize}
      \item \textbf{Description}: As distributed systems evolve, so do the methods of securing data, including encryption and decentralized identity management.
      \item \textbf{Impact}: Protects sensitive information from breaches and builds trust with users.
      \item \textbf{Example}: Blockchain technology is increasingly being used to secure transactions by creating a decentralized ledger that is tamper-proof.
    \end{itemize}
  \end{enumerate}
\end{frame}

\begin{frame}[fragile]
  \frametitle{Conclusion & Key Points}
  \begin{block}{Conclusion}
    The future of distributed data processing is characterized by innovation and adaptability. Keeping abreast of these trends is essential for professionals to leverage the full potential of distributed systems and to drive competitive advantage in their industries.
  \end{block}

  \begin{block}{Key Points to Emphasize}
    \begin{itemize}
      \item The shift towards serverless and edge computing facilitates efficiency and speed.
      \item Multi-cloud approaches bolster resilience and flexibility.
      \item AI/ML adoption revolutionizes data processing capabilities.
      \item Security advancements are crucial in protecting data integrity in distributed environments.
    \end{itemize}
  \end{block}
\end{frame}


\end{document}