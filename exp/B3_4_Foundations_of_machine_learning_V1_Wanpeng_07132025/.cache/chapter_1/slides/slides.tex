\documentclass[aspectratio=169]{beamer}

% Theme and Color Setup
\usetheme{Madrid}
\usecolortheme{whale}
\useinnertheme{rectangles}
\useoutertheme{miniframes}

% Additional Packages
\usepackage[utf8]{inputenc}
\usepackage[T1]{fontenc}
\usepackage{graphicx}
\usepackage{booktabs}
\usepackage{listings}
\usepackage{amsmath}
\usepackage{amssymb}
\usepackage{xcolor}
\usepackage{tikz}
\usepackage{pgfplots}
\pgfplotsset{compat=1.18}
\usetikzlibrary{positioning}
\usepackage{hyperref}

% Custom Colors
\definecolor{myblue}{RGB}{31, 73, 125}
\definecolor{mygray}{RGB}{100, 100, 100}
\definecolor{mygreen}{RGB}{0, 128, 0}
\definecolor{myorange}{RGB}{230, 126, 34}
\definecolor{mycodebackground}{RGB}{245, 245, 245}

% Set Theme Colors
\setbeamercolor{structure}{fg=myblue}
\setbeamercolor{frametitle}{fg=white, bg=myblue}
\setbeamercolor{title}{fg=myblue}
\setbeamercolor{section in toc}{fg=myblue}
\setbeamercolor{item projected}{fg=white, bg=myblue}
\setbeamercolor{block title}{bg=myblue!20, fg=myblue}
\setbeamercolor{block body}{bg=myblue!10}
\setbeamercolor{alerted text}{fg=myorange}

% Set Fonts
\setbeamerfont{title}{size=\Large, series=\bfseries}
\setbeamerfont{frametitle}{size=\large, series=\bfseries}
\setbeamerfont{caption}{size=\small}
\setbeamerfont{footnote}{size=\tiny}

% Document Start
\begin{document}

\frame{\titlepage}

\begin{frame}[fragile]
    \titlepage
\end{frame}

\begin{frame}[fragile]
    \frametitle{What is Machine Learning?}

    \begin{block}{Definition}
        Machine Learning (ML) is a subset of artificial intelligence that enables software applications to become more accurate in predicting outcomes without being explicitly programmed for each task.
    \end{block}
    
    \begin{block}{How it Works}
        \begin{enumerate}
            \item \textbf{Data Input:} Collection of various forms of data.
            \item \textbf{Learning Process:} Algorithms analyze data, identify patterns, and create models.
            \item \textbf{Predictions/Decisions:} Models predict future outcomes or classify new data.
        \end{enumerate}
    \end{block}
\end{frame}

\begin{frame}[fragile]
    \frametitle{Significance of Machine Learning}

    \begin{itemize}
        \item \textbf{Automation of Tasks:} Frees human capacity for higher-level decision-making.
        \item \textbf{Data Insights:} Extracts insights from vast amounts of data.
        \item \textbf{Personalization:} Enhances user experience through tailored content recommendations.
    \end{itemize}
\end{frame}

\begin{frame}[fragile]
    \frametitle{Real-World Examples of Machine Learning}

    \begin{enumerate}
        \item \textbf{Healthcare:} Assists in diagnosing diseases earlier.
        \item \textbf{Finance:} Utilized for credit scoring and fraud detection.
        \item \textbf{Transportation:} Enables self-driving cars to navigate safely.
        \item \textbf{Social Media:} Analyzes user behavior for better feature enhancements.
    \end{enumerate}
\end{frame}

\begin{frame}[fragile]
    \frametitle{Key Points to Emphasize}

    \begin{itemize}
        \item \textbf{Data-Driven Decision Making:} Transforms business operations through reliance on data.
        \item \textbf{Interdisciplinary Approach:} Combines statistics, computer science, and domain knowledge.
        \item \textbf{Continuous Learning:} Models improve as they encounter more data.
    \end{itemize}
\end{frame}

\begin{frame}[fragile]
    \frametitle{Simplified Illustration}

    \begin{block}{Learning Process Flow}
        \begin{center}
            \textbf{Data Collection} $\rightarrow$ \textbf{Model Training} $\rightarrow$ \textbf{Predictions/Decisions}
        \end{center}
        % Here you could include a diagram if desired.
    \end{block}
\end{frame}

\begin{frame}[fragile]
    \frametitle{Course Objectives - Overview}
    In this course, we will embark on a journey to understand the fascinating field of Machine Learning (ML). 
    By the end of this course, you will acquire the knowledge and skills necessary to apply machine learning techniques to solve real-world problems.
\end{frame}

\begin{frame}[fragile]
    \frametitle{Course Objectives - Key Learning Objectives}
    \begin{enumerate}
        \item \textbf{Understanding Basic Concepts}
        \begin{itemize}
            \item Gain a foundational understanding of ML and its importance.
            \item \textit{Example:} Netflix movie recommendations based on viewing history.
        \end{itemize}

        \item \textbf{Types of Machine Learning}
        \begin{itemize}
            \item Learn the difference between supervised and unsupervised learning.
            \item \textit{Example:} Predicting house prices (supervised) vs. customer segmentation (unsupervised).
        \end{itemize}

        \item \textbf{Key Algorithms and Techniques}
        \begin{itemize}
            \item Explore fundamental algorithms like linear regression and decision trees.
            \item \textit{Illustration:} Decision tree splitting data for classification tasks.
        \end{itemize}
    \end{enumerate}
\end{frame}

\begin{frame}[fragile]
    \frametitle{Course Objectives - Practical Applications and Skills}
    \begin{enumerate}[resume]
        \item \textbf{Practical Applications}
        \begin{itemize}
            \item Examine ML applications in healthcare, finance, and entertainment.
            \item \textit{Example:} Image recognition for disease detection in healthcare.
        \end{itemize}

        \item \textbf{Data Preparation and Feature Engineering}
        \begin{itemize}
            \item Understand the impact of data quality on model performance.
            \item \textit{Key Point:} "Garbage in, garbage out" — quality input is crucial.
        \end{itemize}

        \item \textbf{Model Evaluation and Validation}
        \begin{itemize}
            \item Learn about methods to evaluate model performance (accuracy, precision, recall).
            \item \textit{Example:} How evaluation metrics can affect model perception.
        \end{itemize}

        \item \textbf{Hands-On Programming Skills}
        \begin{itemize}
            \item Introduction to coding in Python and essential libraries (e.g., scikit-learn, TensorFlow).
            \item \begin{lstlisting}[language=Python]
from sklearn.model_selection import train_test_split
from sklearn.linear_model import LinearRegression

# Sample data
X = [[1], [2], [3], [4]]
y = [1, 2, 3, 4]

# Train-test split
X_train, X_test, y_train, y_test = train_test_split(X, y, test_size=0.2)

# Creating and training the model
model = LinearRegression()
model.fit(X_train, y_train)
            \end{lstlisting}
        \end{itemize}
    \end{enumerate}
\end{frame}

\begin{frame}[fragile]
    \frametitle{Course Objectives - Conclusion}
    This course will empower you with both theoretical knowledge and practical skills to apply machine learning in innovative ways. 
    We will encourage you to think critically and creatively about the possibilities and challenges of this transformative technology. 
    Get ready to explore a world full of questions, solutions, and the art of machine learning!
\end{frame}

\begin{frame}[fragile]
    \frametitle{Introduction}
    \begin{block}{Overview}
        Machine Learning (ML) is transforming various industries by enabling systems to learn from data, identify patterns, and make decisions with minimal human intervention. Let’s explore some of the exciting real-world applications of ML and how they are making an impact in our daily lives.
    \end{block}
\end{frame}

\begin{frame}[fragile]
    \frametitle{Applications in Healthcare}
    \begin{itemize}
        \item \textbf{Predictive Analytics}: 
            ML algorithms analyze patient data to predict disease outbreaks and patient outcomes.
            \begin{itemize}
                \item Example: Assessing disease likelihood based on genetics and lifestyle.
            \end{itemize}
        \item \textbf{Medical Imaging}: 
            Techniques such as convolutional neural networks (CNNs) enhance image recognition in diagnostic imaging.
            \begin{itemize}
                \item Example: Early detection of cancer through improved X-ray and MRI analysis.
            \end{itemize}
        \item \textbf{Example}:
            IBM Watson Health utilizes ML to recommend customized cancer treatment plans.
    \end{itemize}
\end{frame}

\begin{frame}[fragile]
    \frametitle{Applications in Finance and Autonomous Vehicles}
    \begin{block}{Finance}
        \begin{itemize}
            \item \textbf{Fraud Detection}: 
                Identifies unusual patterns indicating potential fraud in transactions.
            \item \textbf{Algorithmic Trading}: 
                Analyzes market data to make trade decisions faster than humans.
            \item \textbf{Example}:
                Citibank uses ML to detect fraudulent credit card transactions.
        \end{itemize}
    \end{block}
    
    \begin{block}{Autonomous Vehicles}
        \begin{itemize}
            \item \textbf{Self-Driving Technology}: 
                Processes sensor and GPS data for navigation and decision-making.
            \item \textbf{Enhanced Safety Features}: 
                ML powers features like lane detection and automatic braking.
            \item \textbf{Example}: 
                Tesla’s Autopilot continually improves through data collection.
        \end{itemize}
    \end{block}
\end{frame}

\begin{frame}[fragile]
    \frametitle{Key Points and Conclusion}
    \begin{itemize}
        \item \textbf{Data-Driven Decision Making}:
            ML enables informed decisions based on data, reducing reliance on assumptions.
        \item \textbf{Continuously Evolving}:
            As systems learn from more data, their accuracy and efficiency improve.
        \item \textbf{Interdisciplinary Nature}:
            ML impacts various fields such as healthcare, finance, and engineering.
    \end{itemize}
    \begin{block}{Conclusion}
        ML is set to profoundly change our world. Understanding its applications is vital as it continues to evolve across multiple sectors.
    \end{block}
\end{frame}

\begin{frame}[fragile]
    \frametitle{Discussion Questions}
    \begin{itemize}
        \item Which areas do you think will be most transformed by machine learning in the next 5 years?
        \item How can we ensure ethical applications of machine learning in critical areas such as healthcare and finance?
    \end{itemize}
\end{frame}

\begin{frame}[fragile]
    \frametitle{Supervised Learning - Definition}
    \begin{block}{What is Supervised Learning?}
        Supervised Learning is a type of machine learning where an algorithm learns from a labeled dataset. This means that the input data is paired with the correct output, allowing the model to learn the relationships between them.
    \end{block}
    \begin{itemize}
        \item \textbf{Labeled Data:} The training dataset includes both input features and corresponding target outputs.
        \item \textbf{Training and Testing Phases:} The model is trained on the labeled data and then tested on unseen data.
        \item \textbf{Goal:} To create a model that can accurately predict outcomes for new, unseen data instances.
    \end{itemize}
\end{frame}

\begin{frame}[fragile]
    \frametitle{Supervised Learning - Examples}
    \begin{block}{Examples of Supervised Learning}
        \begin{enumerate}
            \item \textbf{Classification:}
                \begin{itemize}
                    \item \textbf{Definition:} A task where the output variable is a category (discrete value).
                    \item \textbf{Example:} Email spam detection (classifying emails as 'spam' or 'not spam').
                \end{itemize}
            
            \item \textbf{Regression:}
                \begin{itemize}
                    \item \textbf{Definition:} A task where the output variable is a continuous value.
                    \item \textbf{Example:} Predicting house prices based on features like size, location, and number of bedrooms.
                \end{itemize}
        \end{enumerate}
    \end{block}
\end{frame}

\begin{frame}[fragile]
    \frametitle{Supervised Learning - Key Points}
    \begin{block}{Key Points to Emphasize}
        \begin{itemize}
            \item \textbf{Applications:}
                \begin{itemize}
                    \item Healthcare: Predicting patient outcomes.
                    \item Finance: Credit scoring or fraud detection.
                    \item Speech Recognition: Translating spoken words into text.
                \end{itemize}
            \item \textbf{Common Algorithms:}
                \begin{itemize}
                    \item Linear Regression (for regression tasks).
                    \item Decision Trees (for both classification and regression tasks).
                    \item Support Vector Machines (SVM) (primarily for classification).
                \end{itemize}
            \item \textbf{Performance Metrics:}
                \begin{itemize}
                    \item Classification Accuracy: Percentage of correctly predicted labels.
                    \item Mean Absolute Error (MAE): Average error between predicted and actual values for regression.
                \end{itemize}
        \end{itemize}
    \end{block}
\end{frame}

\begin{frame}[fragile]
    \frametitle{Supervised Learning - Visual Representation}
    \begin{block}{Diagram}
        \begin{center}
        \large
        Input Features (X) --------------> Model (Learning) ------------------> Predicted Output (Y) \\ 
               |                                                  | \\
               V                                                  V \\
          Labeled Data (X, Y)                              New Unseen Data (X') 
        \end{center}
    \end{block}
\end{frame}

\begin{frame}[fragile]
    \frametitle{Unsupervised Learning - Definition and Key Characteristics}
    \begin{block}{Definition}
        Unsupervised Learning is a type of machine learning that deals with data that has no labeled responses. The model learns patterns and structures from the input data without guidance on the expected outputs.
    \end{block}

    \begin{block}{Key Characteristics}
        \begin{itemize}
            \item \textbf{No Labels:} The algorithm uncovers the underlying structure of the data without guidance.
            \item \textbf{Exploration:} It aids in understanding data distributions and identifying patterns.
            \item \textbf{Dimensionality Reduction:} Simplifies data while retaining important information.
        \end{itemize}
    \end{block}
\end{frame}

\begin{frame}[fragile]
    \frametitle{Unsupervised Learning - Techniques}
    \begin{enumerate}
        \item \textbf{Clustering}
            \begin{itemize}
                \item \textbf{Definition:} Grouping objects so that those in the same group are more similar to each other.
                \item \textbf{Example:} Identifying customer groups in retail based on purchasing habits.
                \item \textbf{Common Algorithms:} K-Means, Hierarchical Clustering, DBSCAN.
            \end{itemize}
        
        \item \textbf{Association}
            \begin{itemize}
                \item \textbf{Definition:} Finding interesting relationships between variables in large databases.
                \item \textbf{Example:} Customers who buy bread are likely to buy butter.
                \item \textbf{Common Algorithms:} Apriori, FP-Growth.
            \end{itemize}
    \end{enumerate}
\end{frame}

\begin{frame}[fragile]
    \frametitle{Unsupervised Learning - Key Points and Conclusion}
    \begin{block}{Key Points to Emphasize}
        \begin{itemize}
            \item \textbf{Applications:} Market segmentation, anomaly detection, social network analysis.
            \item \textbf{Exploratory Data Analysis (EDA):} Crucial for visualizing data and discovering patterns before predictive modeling.
            \item \textbf{Caveats:} May be less interpretable than supervised learning; results need further analysis for actionable insights.
        \end{itemize}
    \end{block}

    \begin{block}{Conclusion}
        Unsupervised learning helps uncover complex datasets without labeled data limitations. Techniques like clustering and association provide valuable insights across various sectors.
    \end{block}
\end{frame}

\begin{frame}[fragile]
    \frametitle{Comparison between Supervised and Unsupervised Learning - Part 1}
    \begin{block}{Overview}
        This slide summarizes the key differences and use cases of Supervised and Unsupervised Learning.
    \end{block}
    
    \begin{itemize}
        \item \textbf{Supervised Learning:} Trained on labeled data.
        \item \textbf{Unsupervised Learning:} Trained on unlabeled data.
    \end{itemize}
\end{frame}

\begin{frame}[fragile]
    \frametitle{Comparison between Supervised and Unsupervised Learning - Part 2}
    
    \begin{block}{Supervised Learning}
        \begin{itemize}
            \item \textbf{Definition:} Model trained on labeled data, mapping inputs to outputs.
            \item \textbf{Goal:} Apply learned mapping to new, unseen data.
        \end{itemize}
    \end{block}

    \begin{block}{Unsupervised Learning}
        \begin{itemize}
            \item \textbf{Definition:} Model trained on unlabeled data to discover patterns.
            \item \textbf{Goal:} Identify hidden structures within the data.
        \end{itemize}
    \end{block}
\end{frame}

\begin{frame}[fragile]
    \frametitle{Examples of Learning Approaches}
    \begin{block}{Supervised Learning Example}
        \textbf{Use Case:} Email Spam Detection
        \begin{itemize}
            \item \textbf{Input Data:} Email features (e.g., keywords, sender).
            \item \textbf{Output:} "Spam" or "Not Spam."
        \end{itemize}
    \end{block}

    \begin{block}{Unsupervised Learning Example}
        \textbf{Use Case:} Customer Segmentation
        \begin{itemize}
            \item \textbf{Input Data:} Customer purchasing behavior.
            \item \textbf{Output:} Groups of similar customers for targeted marketing.
        \end{itemize}
    \end{block}
\end{frame}

\begin{frame}[fragile]
    \frametitle{Key Points and Summary Table}
    \begin{block}{Key Points}
        \begin{itemize}
            \item \textbf{Data Requirement:}
                \begin{itemize}
                    \item Supervised Learning requires labeled data.
                    \item Unsupervised Learning works with raw data.
                \end{itemize}
            \item \textbf{Applications:}
                \begin{itemize}
                    \item Supervised Learning for classification and regression.
                    \item Unsupervised Learning for clustering and association.
                \end{itemize}
        \end{itemize}
    \end{block}

    \begin{table}
        \centering
        \begin{tabular}{|l|l|l|}
            \hline
            \textbf{Aspect} & \textbf{Supervised Learning} & \textbf{Unsupervised Learning} \\
            \hline
            Input Data & Labeled data & Unlabeled data \\
            Goal & Map inputs to outputs & Discover patterns \\
            Common Techniques & Classification, Regression & Clustering, Association \\
            Example Use Case & Predicting house prices & Customer segmentation \\
            \hline
        \end{tabular}
    \end{table}
\end{frame}

\begin{frame}[fragile]
    \frametitle{Conclusion}
    Understanding the differences between Supervised and Unsupervised Learning is essential for selecting the appropriate approach based on your data and problem context. Consider how these methods can be applied in real-world scenarios as you continue your study in machine learning.
\end{frame}

\begin{frame}[fragile]
    \frametitle{Importance of Data Quality}
    \begin{block}{Overview}
        Data is the "new oil" in AI and ML. Like oil, data needs refining for quality before training models. Quality affects model performance and decision-making.
    \end{block}
\end{frame}

\begin{frame}[fragile]
    \frametitle{Importance of Data Quality - Definition}
    \begin{block}{What is Data Quality?}
        Data quality is the condition of a dataset, which includes:
        \begin{itemize}
            \item Accuracy
            \item Completeness
            \item Consistency
            \item Reliability
            \item Timeliness
        \end{itemize}
        High-quality data enhances predictions; poor quality leads to misleading outcomes.
    \end{block}
\end{frame}

\begin{frame}[fragile]
    \frametitle{Key Aspects of Data Quality}
    \begin{enumerate}
        \item \textbf{Accuracy}: Correct data entries (e.g., age of individuals).
        \item \textbf{Completeness}: All necessary information is included.
        \item \textbf{Consistency}: Uniform entries across all datasets.
        \item \textbf{Reliability}: Trustworthiness of data sources.
        \item \textbf{Timeliness}: Use of up-to-date data.
    \end{enumerate}
\end{frame}

\begin{frame}[fragile]
    \frametitle{Examples of Impact on Model Performance}
    \begin{itemize}
        \item \textbf{Predictive Maintenance}: Faulty sensor data can lead to inaccurate predictions and increased downtime.
        \item \textbf{Credit Scoring Models}: Incomplete or outdated data may result in lost opportunities for applicants or risky loans.
    \end{itemize}
\end{frame}

\begin{frame}[fragile]
    \frametitle{Questions to Ponder}
    \begin{itemize}
        \item How can we ensure high-quality data collection?
        \item What processes can validate data before model training?
        \item How may bias in datasets affect AI decision-making?
    \end{itemize}
\end{frame}

\begin{frame}[fragile]
    \frametitle{Conclusion and Takeaway}
    \begin{block}{Conclusion}
        Emphasizing data quality is critical in machine learning. High-quality data leads to better model performance and decision-making.
    \end{block}
    \begin{block}{Key Takeaway}
        "Garbage in, garbage out" highlights the necessity of high-quality data in AI applications.
    \end{block}
\end{frame}

\begin{frame}[fragile]
    \frametitle{Key Techniques in Machine Learning - Introduction}
    \begin{block}{Overview}
        Machine Learning (ML) techniques can be broadly categorized into three fundamental approaches: 
        \begin{itemize}
            \item Classification
            \item Regression
            \item Clustering
        \end{itemize}
        Each method serves distinct types of problems and utilizes different algorithms and models.
    \end{block}
\end{frame}

\begin{frame}[fragile]
    \frametitle{Key Techniques in Machine Learning - Classification}
    \begin{block}{A. Classification}
        \begin{itemize}
            \item \textbf{Definition}: A supervised learning technique that assigns labels to instances based on input features, predicting discrete outcomes (categories).
            \item \textbf{Example}: A spam detection system classifies emails as either "Spam" or "Not Spam" based on features like content and sender.
            \item \textbf{Common Algorithms}: 
                \begin{itemize}
                    \item Decision Trees
                    \item Random Forests
                    \item Support Vector Machines (SVM)
                    \item Neural Networks
                \end{itemize}
        \end{itemize}
    \end{block}
\end{frame}

\begin{frame}[fragile]
    \frametitle{Key Techniques in Machine Learning - Regression and Clustering}
    \begin{block}{B. Regression}
        \begin{itemize}
            \item \textbf{Definition}: Predicting a continuous outcome variable based on one or more input features. 
            \item \textbf{Example}: A real estate price prediction model utilizes features like square footage and location to estimate house prices.
            \item \textbf{Common Algorithms}: 
                \begin{itemize}
                    \item Linear Regression
                    \item Polynomial Regression
                    \item Ridge Regression
                \end{itemize}
        \end{itemize}
    \end{block}

    \begin{block}{C. Clustering}
        \begin{itemize}
            \item \textbf{Definition}: An unsupervised learning technique that groups similar data points based on features.
            \item \textbf{Example}: Customer segmentation in marketing based on purchasing behavior enables targeted marketing strategies.
            \item \textbf{Common Algorithms}: 
                \begin{itemize}
                    \item K-Means
                    \item Hierarchical Clustering
                    \item DBSCAN (Density-Based Spatial Clustering of Applications with Noise)
                \end{itemize}
        \end{itemize}
    \end{block}
\end{frame}

\begin{frame}[fragile]
    \frametitle{Key Techniques in Machine Learning - Key Points}
    \begin{block}{Key Points to Emphasize}
        \begin{itemize}
            \item \textbf{Relevance of Techniques}: Understanding the problem type is crucial for selecting the appropriate ML technique.
            \item \textbf{Supervised vs Unsupervised}:
            \begin{itemize}
                \item Classification and Regression are supervised methods.
                \item Clustering is an unsupervised method.
            \end{itemize}
        \end{itemize}
        \textbf{Visual Aid:} Consider a flowchart that starts with "What type of output do we need?" branching into "Discrete", "Continuous", and "Groups".
    \end{block}
\end{frame}

\begin{frame}[fragile]
    \frametitle{Introduction to Data Preprocessing}
    \begin{block}{Overview}
        Data preprocessing is crucial in the machine learning pipeline. It involves cleaning and transforming raw data to improve model performance.
    \end{block}
\end{frame}

\begin{frame}[fragile]
    \frametitle{Key Concepts in Data Preprocessing}
    \begin{enumerate}
        \item \textbf{Data Cleaning}
        \begin{itemize}
            \item Handling Missing Values
            \item Removing Duplicates
            \item Correcting Data Types
        \end{itemize}
        \item \textbf{Data Transformation}
        \begin{itemize}
            \item Normalization / Standardization
            \item Encoding Categorical Variables
        \end{itemize}
        \item \textbf{Feature Engineering}
        \begin{itemize}
            \item Importance of creating new features
        \end{itemize}
    \end{enumerate}
\end{frame}

\begin{frame}[fragile]
    \frametitle{Detailed Discussion on Data Cleaning}
    \begin{block}{Data Cleaning Techniques}
        \begin{itemize}
            \item \textbf{Handling Missing Values:}
            Techniques include removing records or imputing missing values (mean, median, mode).\\
            \textbf{Example:} Replace missing scores with average.
            \item \textbf{Removing Duplicates:}
            Identifying and eliminating repeated entries.
            \item \textbf{Correcting Data Types:}
            Ensuring correct formats for data representation.
        \end{itemize}
    \end{block}
\end{frame}

\begin{frame}[fragile]{Example Code Snippet for Data Cleaning}
    \begin{lstlisting}[language=Python]
import pandas as pd

# Load dataset
data = pd.read_csv("students_grades.csv")

# Handling missing values
data.fillna(data.mean(), inplace=True)

# Removing duplicates
data.drop_duplicates(inplace=True)

# Normalization
from sklearn.preprocessing import MinMaxScaler
scaler = MinMaxScaler()
data[['height']] = scaler.fit_transform(data[['height']])

print(data.head())
    \end{lstlisting}
\end{frame}

\begin{frame}[fragile]
    \frametitle{Evaluation Metrics - Introduction}
    \begin{block}{Overview}
        In machine learning, evaluating the performance of a model is crucial. 
        This helps us understand how well our model performs and guides improvements.
        This series of slides introduces common evaluation metrics used across different types of machine learning tasks.
    \end{block}
\end{frame}

\begin{frame}[fragile]
    \frametitle{Evaluation Metrics - Key Metrics}
    \begin{enumerate}
        \item \textbf{Accuracy}
            \begin{itemize}
                \item Measures the proportion of correctly predicted instances out of the total instances.
                \item \text{Formula:} 
                \[
                \text{Accuracy} = \frac{\text{Number of Correct Predictions}}{\text{Total Predictions}}
                \]
                \item \textbf{Example:} If a model makes 80 correct predictions out of 100, accuracy = 80\%.
            \end{itemize}
        
        \item \textbf{Precision}
            \begin{itemize}
                \item Assesses the quality of positive predictions.
                \item \text{Formula:}
                \[
                \text{Precision} = \frac{\text{True Positives}}{\text{True Positives} + \text{False Positives}}
                \]
                \item \textbf{Example:} If a model predicts 30 positives, 20 of which are true, precision = \( \frac{20}{30} = 0.67 \) or 67\%.
            \end{itemize}
        
        \item \textbf{Recall (Sensitivity)}
            \begin{itemize}
                \item Measures how many actual positive instances are correctly predicted.
                \item \text{Formula:}
                \[
                \text{Recall} = \frac{\text{True Positives}}{\text{True Positives} + \text{False Negatives}}
                \]
                \item \textbf{Example:} If 25 actual positives exist and the model corrects 20, recall = \( \frac{20}{25} = 0.8 \) or 80\%.
            \end{itemize}
    \end{enumerate}
\end{frame}

\begin{frame}[fragile]
    \frametitle{Evaluation Metrics - F1 Score and Confusion Matrix}
    \begin{enumerate}
        \setcounter{enumi}{3}
        \item \textbf{F1 Score}
            \begin{itemize}
                \item The harmonic mean of precision and recall, balancing both.
                \item \text{Formula:}
                \[
                \text{F1 Score} = 2 \times \frac{\text{Precision} \times \text{Recall}}{\text{Precision} + \text{Recall}}
                \]
                \item \textbf{Example:} If precision = 0.67 and recall = 0.8, then F1 Score = \( 2 \times \frac{0.67 \times 0.8}{0.67 + 0.8} \approx 0.73 \).
            \end{itemize}

        \item \textbf{Confusion Matrix}
            \begin{itemize}
                \item A table that visualizes the performance of a classification model.
                \item Components:
                \begin{itemize}
                    \item True Positive (TP): Correctly predicted positive cases.
                    \item True Negative (TN): Correctly predicted negative cases.
                    \item False Positive (FP): Incorrectly predicted positive cases.
                    \item False Negative (FN): Incorrectly predicted negative cases.
                \end{itemize}
                \item \textbf{Example:}
                \begin{center}
                    \begin{tabular}{|c|c|c|}
                        \hline
                        & \textbf{Predicted Positives} & \textbf{Predicted Negatives} \\
                        \hline
                        \textbf{Actual Positives} & TP & FN \\
                        \hline
                        \textbf{Actual Negatives} & FP & TN \\
                        \hline
                    \end{tabular}
                \end{center}
            \end{itemize}
    \end{enumerate}
\end{frame}

\begin{frame}[fragile]
    \frametitle{Engaging with Real-World Data}
    \begin{block}{Importance of Real-World Data}
        Engaging with real-world data enhances practical skills in machine learning by bridging theoretical concepts and their applications.
    \end{block}
\end{frame}

\begin{frame}[fragile]
    \frametitle{Understanding the Importance of Real-World Data Experiences}
    \begin{itemize}
        \item Practical application of theoretical knowledge
        \item Development of critical thinking skills
        \item Exposure to diverse data types
        \item Collaborative learning through case studies
        \item Hands-on experience with tools and technologies
    \end{itemize}
\end{frame}

\begin{frame}[fragile]
    \frametitle{Key Concepts}
    \begin{enumerate}
        \item \textbf{Practical Application:} 
        Real-world data challenges often require students to analyze datasets.
        \item \textbf{Critical Thinking:} 
        Students encounter issues like missing values and must decide on appropriate solutions.
        \item \textbf{Diverse Data:} 
        Working with structured, unstructured, and time-series data enhances learning.
        \item \textbf{Collaboration:} 
        Team-based case studies simulate real-world working environments.
        \item \textbf{Hands-On Tools:} 
        Familiarity with frameworks like TensorFlow and Scikit-learn is essential.
    \end{enumerate}
\end{frame}

\begin{frame}[fragile]
    \frametitle{Practical Examples}
    \begin{block}{Kaggle Competitions}
        Participation exposes students to large datasets and real-time feedback.
    \end{block}
    \begin{block}{Predicting Diabetes}
        Using the PIMA Indians Diabetes Database to predict diabetes likelihood.
    \end{block}
\end{frame}

\begin{frame}[fragile]
    \frametitle{Key Takeaways}
    \begin{itemize}
        \item Real-world datasets lead to deeper understanding of machine learning.
        \item Practical experiences build confidence and problem-solving skills.
        \item Case studies prepare students for real-world data challenges.
    \end{itemize}
\end{frame}

\begin{frame}[fragile]
    \frametitle{Final Project Overview - Introduction}
    \begin{block}{Introduction to Your Final Project}
        As we wrap up this chapter on Machine Learning, it's time to put the concepts we've learned into practice. The final project serves as an exciting opportunity for you to apply machine learning techniques to real-world problems, fostering an understanding that goes beyond theory.
    \end{block}
\end{frame}

\begin{frame}[fragile]
    \frametitle{Final Project Overview - What to Expect}
    \begin{itemize}
        \item \textbf{Real-World Applications:} Choose a real-world dataset or problem for your project, such as:
        \begin{itemize}
            \item Predicting stock prices
            \item Classifying flowers based on petal size
        \end{itemize}

        \item \textbf{Learning Objectives:} The project will help you:
        \begin{itemize}
            \item \textbf{Integrate Knowledge:} Combine various machine learning techniques learned during the course.
            \item \textbf{Develop Practical Skills:} Gain hands-on experience in data preprocessing, model selection, evaluation metrics, and hyperparameter tuning.
        \end{itemize}

        \item \textbf{Problem-Solving:} Choose a problem that interests you; this can be driven by personal passion, societal issues, or academic curiosity.
    \end{itemize}
\end{frame}

\begin{frame}[fragile]
    \frametitle{Final Project Overview - Example Scenarios}
    \begin{itemize}
        \item \textbf{Project Idea 1: Health Prediction Model}
        \begin{itemize}
            \item Use a health dataset to predict disease likelihood.
            \item Tools: Python (libraries like Scikit-learn or TensorFlow).
        \end{itemize}

        \item \textbf{Project Idea 2: Image Classification}
        \begin{itemize}
            \item Train a model to classify images using datasets like CIFAR-10.
            \item Tools: TensorFlow or PyTorch.
        \end{itemize}

        \item \textbf{Project Idea 3: Sentiment Analysis}
        \begin{itemize}
            \item Analyze customer reviews to determine sentiment.
            \item Tools: Python with NLTK or SpaCy.
        \end{itemize}
    \end{itemize}
\end{frame}

\begin{frame}[fragile]
    \frametitle{Importance of Collaboration - Overview}
    \begin{block}{Description}
        Emphasizing the role of teamwork in machine learning projects and learning outcomes.
    \end{block}
\end{frame}

\begin{frame}[fragile]
    \frametitle{Importance of Collaboration - Concepts}
    \begin{itemize}
        \item Collaboration combines diverse skills and perspectives.
        \item Essential for tackling complex machine learning projects.
        \item Encompasses multiple stages: data collection, cleaning, modeling, testing, and deployment.
    \end{itemize}
\end{frame}

\begin{frame}[fragile]
    \frametitle{Examples of Collaboration in Machine Learning}
    \begin{enumerate}
        \item \textbf{Data Scientists and Domain Experts:} Healthcare professionals offer insights for developing predictive models.
        \item \textbf{Software Engineers and Data Engineers:} Collaborative efforts ensure the efficient use of data infrastructure and application integration.
        \item \textbf{Cross-functional Teams:} Integration of machine learning engineers, product managers, and UX designers to balance technical feasibility with user experience.
    \end{enumerate}
\end{frame}

\begin{frame}[fragile]
    \frametitle{Key Points to Emphasize}
    \begin{itemize}
        \item \textbf{Diverse Skill Sets:} Innovation stems from varied expertise.
        \item \textbf{Problem Solving:} Joint brainstorming leads to better solutions.
        \item \textbf{Learning Opportunities:} Collaboration enhances knowledge sharing and learning outcomes.
    \end{itemize}
\end{frame}

\begin{frame}[fragile]
    \frametitle{Engaging Questions}
    \begin{itemize}
        \item How could combining expertise from different fields enhance your machine learning project?
        \item Can you think of a situation where teamwork led to a breakthrough solution in your experiences?
    \end{itemize}
\end{frame}

\begin{frame}[fragile]
    \frametitle{Summary of Collaboration Importance}
    In summary, effective collaboration in machine learning is crucial for success. It fosters a continuous learning environment, leading to improved project outcomes and user engagement. As you embark on your final project, consider how teamwork can enhance your learning experience.
\end{frame}

\begin{frame}[fragile]
    \frametitle{Conclusion of Chapter 1}
    \begin{block}{Introduction to Machine Learning}
        Machine Learning (ML) is a subset of artificial intelligence (AI) that enables systems to learn from data, improve performance, and make decisions without explicit programming for each task.
    \end{block}
\end{frame}

\begin{frame}[fragile]
    \frametitle{Key Concepts in Machine Learning}
    \begin{enumerate}
        \item \textbf{Types of Machine Learning:}
        \begin{itemize}
            \item \textbf{Supervised Learning}: Learning from labeled data.
            \item \textbf{Unsupervised Learning}: Finding patterns in unlabeled data.
            \item \textbf{Reinforcement Learning}: Learning through trial and error with feedback.
        \end{itemize}
        \item \textbf{Common Applications:}
        \begin{itemize}
            \item Image Recognition
            \item Natural Language Processing
            \item Predictive Analysis
        \end{itemize}
    \end{enumerate}
\end{frame}

\begin{frame}[fragile]
    \frametitle{Takeaway Points and Reflection}
    \begin{block}{Takeaway Points}
        \begin{itemize}
            \item Data is Crucial: Quality and quantity directly impact ML effectiveness.
            \item Iterative Process: Ongoing experimentation and evaluation are key.
            \item Ethics in ML: Considerations around data usage and decision-making.
        \end{itemize}
    \end{block}
    \begin{block}{Reflection Questions}
        \begin{itemize}
            \item How do you envision the role of machine learning in your field?
            \item What ethical considerations should we keep in mind with ML?
        \end{itemize}
    \end{block}
\end{frame}

\begin{frame}[fragile]
    \frametitle{Questions \& Discussion}
    \begin{block}{Engaging with Machine Learning Concepts}
        In this session, we'll delve into your questions and insights on the foundational concepts of Machine Learning (ML). 
        This discussion aims to deepen our understanding and encourage creative thinking about ML's applications.
    \end{block}
\end{frame}

\begin{frame}[fragile]
    \frametitle{Key Discussion Points}

    \begin{enumerate}
        \item \textbf{Understanding Machine Learning:}
        \begin{itemize}
            \item \textbf{Definition:} A subset of artificial intelligence enabling systems to learn from data and make decisions autonomously.
            \item \textbf{Types of Learning:}
            \begin{itemize}
                \item \textbf{Supervised Learning:} Learns from labeled datasets.
                \item \textbf{Unsupervised Learning:} Finds patterns in unlabeled data.
                \item \textbf{Reinforcement Learning:} Learn through trial and error to achieve a goal.
            \end{itemize}
        \end{itemize}
    \end{enumerate}
\end{frame}

\begin{frame}[fragile]
    \frametitle{Real-World Applications and Encouragement for Participation}

    \begin{itemize}
        \item \textbf{Real-World Applications:}
        \begin{itemize}
            \item \textbf{Healthcare:} Predicting disease outbreaks.
            \item \textbf{Finance:} Detecting fraud in banking.
            \item \textbf{Entertainment:} Recommender systems in platforms like Netflix and Spotify.
            \item \textbf{Autonomous Vehicles:} Navigation within self-driving cars.
        \end{itemize}
        \item \textbf{Inspiring Questions:}
        \begin{itemize}
            \item How do we ensure fairness in ML models?
            \item How might ML change unexplored industries?
            \item What are the ethical implications of ML technologies in daily life?
        \end{itemize}
        \item \textbf{Encouragement for Participation:}
        \begin{itemize}
            \item Share your perspectives on ML applications.
            \item Ask questions about concepts we’ve discussed.
            \item Share any fascinating ML applications from your experience.
        \end{itemize}
    \end{itemize}
\end{frame}

\begin{frame}[fragile]
    \frametitle{Next Steps - Overview}
    \begin{block}{Introduction to Machine Learning - What's Coming Up?}
        As we delve deeper into the world of Machine Learning (ML), we will explore how foundational concepts lead us to advanced topics. Here’s a roadmap of what lies ahead and why it matters:
    \end{block}
\end{frame}

\begin{frame}[fragile]
    \frametitle{Key Points \& Upcoming Concepts}
    \begin{enumerate}
        \item \textbf{Types of Learning}:
        \begin{itemize}
            \item \textbf{Supervised Learning}:
            \begin{itemize}
                \item Learning from labeled data (e.g., spam email detection).
            \end{itemize}
            \item \textbf{Unsupervised Learning}:
            \begin{itemize}
                \item Identifying patterns without labels (e.g., customer clustering).
            \end{itemize}
        \end{itemize}

        \item \textbf{Algorithms \& Models}:
        \begin{itemize}
            \item Introduction to algorithms like Linear Regression and Neural Networks.
        \end{itemize}

        \item \textbf{Evaluation Metrics}:
        \begin{itemize}
            \item Measuring model success: accuracy, precision, recall, F1 score.
        \end{itemize}
    \end{enumerate}
\end{frame}

\begin{frame}[fragile]
    \frametitle{Real-World Applications \& Ethics}
    \begin{enumerate}
        \setcounter{enumi}{3}
        \item \textbf{Real-World Applications}:
        \begin{itemize}
            \item Case studies in healthcare (diagnosing diseases) and finance (fraud detection).
            \item Example: Predicting stock prices based on historical data.
        \end{itemize}

        \item \textbf{Ethical Considerations}:
        \begin{itemize}
            \item Discussing bias in algorithms and data privacy issues.
            \item Discussion Prompt: How can we ensure our models treat all individuals equitably?
        \end{itemize}
    \end{enumerate}
\end{frame}

\begin{frame}[fragile]
    \frametitle{Building the Bridge to Advanced Topics}
    \begin{block}{Conclusion}
        Each upcoming session builds on foundational knowledge, preparing you to:
        \begin{itemize}
            \item Engage with complex models like Neural Networks.
            \item Explore fields like Natural Language Processing, Computer Vision, and Reinforcement Learning.
        \end{itemize}
        Remember, as we proceed, keeping an open mind and reflecting on real-world problems is key to our exploration!
    \end{block}
\end{frame}


\end{document}