\documentclass[aspectratio=169]{beamer}

% Theme and Color Setup
\usetheme{Madrid}
\usecolortheme{whale}
\useinnertheme{rectangles}
\useoutertheme{miniframes}

% Additional Packages
\usepackage[utf8]{inputenc}
\usepackage[T1]{fontenc}
\usepackage{graphicx}
\usepackage{booktabs}
\usepackage{listings}
\usepackage{amsmath}
\usepackage{amssymb}
\usepackage{xcolor}
\usepackage{tikz}
\usepackage{pgfplots}
\pgfplotsset{compat=1.18}
\usetikzlibrary{positioning}
\usepackage{hyperref}

% Custom Colors
\definecolor{myblue}{RGB}{31, 73, 125}
\definecolor{mygray}{RGB}{100, 100, 100}
\definecolor{mygreen}{RGB}{0, 128, 0}
\definecolor{myorange}{RGB}{230, 126, 34}
\definecolor{mycodebackground}{RGB}{245, 245, 245}

% Set Theme Colors
\setbeamercolor{structure}{fg=myblue}
\setbeamercolor{frametitle}{fg=white, bg=myblue}
\setbeamercolor{title}{fg=myblue}
\setbeamercolor{section in toc}{fg=myblue}
\setbeamercolor{item projected}{fg=white, bg=myblue}
\setbeamercolor{block title}{bg=myblue!20, fg=myblue}
\setbeamercolor{block body}{bg=myblue!10}
\setbeamercolor{alerted text}{fg=myorange}

% Set Fonts
\setbeamerfont{title}{size=\Large, series=\bfseries}
\setbeamerfont{frametitle}{size=\large, series=\bfseries}
\setbeamerfont{caption}{size=\small}
\setbeamerfont{footnote}{size=\tiny}

% Code Listing Style
\lstdefinestyle{customcode}{
  backgroundcolor=\color{mycodebackground},
  basicstyle=\footnotesize\ttfamily,
  breakatwhitespace=false,
  breaklines=true,
  commentstyle=\color{mygreen}\itshape,
  keywordstyle=\color{blue}\bfseries,
  stringstyle=\color{myorange},
  numbers=left,
  numbersep=8pt,
  numberstyle=\tiny\color{mygray},
  frame=single,
  framesep=5pt,
  rulecolor=\color{mygray},
  showspaces=false,
  showstringspaces=false,
  showtabs=false,
  tabsize=2,
  captionpos=b
}
\lstset{style=customcode}

% Custom Commands
\newcommand{\hilight}[1]{\colorbox{myorange!30}{#1}}
\newcommand{\source}[1]{\vspace{0.2cm}\hfill{\tiny\textcolor{mygray}{Source: #1}}}
\newcommand{\concept}[1]{\textcolor{myblue}{\textbf{#1}}}
\newcommand{\separator}{\begin{center}\rule{0.5\linewidth}{0.5pt}\end{center}}

% Footer and Navigation Setup
\setbeamertemplate{footline}{
  \leavevmode%
  \hbox{%
  \begin{beamercolorbox}[wd=.3\paperwidth,ht=2.25ex,dp=1ex,center]{author in head/foot}%
    \usebeamerfont{author in head/foot}\insertshortauthor
  \end{beamercolorbox}%
  \begin{beamercolorbox}[wd=.5\paperwidth,ht=2.25ex,dp=1ex,center]{title in head/foot}%
    \usebeamerfont{title in head/foot}\insertshorttitle
  \end{beamercolorbox}%
  \begin{beamercolorbox}[wd=.2\paperwidth,ht=2.25ex,dp=1ex,center]{date in head/foot}%
    \usebeamerfont{date in head/foot}
    \insertframenumber{} / \inserttotalframenumber
  \end{beamercolorbox}}%
  \vskip0pt%
}

% Turn off navigation symbols
\setbeamertemplate{navigation symbols}{}

% Title Page Information
\title[Chapter 12: Introduction to Advanced Topics]{Chapter 12: Introduction to Advanced Topics}
\author[J. Smith]{John Smith, Ph.D.}
\institute[University Name]{
  Department of Computer Science\\
  University Name\\
  \vspace{0.3cm}
  Email: email@university.edu\\
  Website: www.university.edu
}
\date{\today}

% Document Start
\begin{document}

\frame{\titlepage}

\begin{frame}[fragile]
    \frametitle{Introduction to Chapter 12 - Overview of Advanced Topics}
    As we venture into Chapter 12, our focus shifts towards some of the most exciting and advanced topics in machine learning, particularly emphasizing \textbf{Reinforcement Learning (RL)}. 
    This chapter aims to provide a scientific understanding as well as context and relevance through inspiring questions and relatable examples.
\end{frame}

\begin{frame}[fragile]
    \frametitle{Introduction to Chapter 12 - What is Reinforcement Learning?}
    Reinforcement Learning is a type of machine learning inspired by behavioral psychology. It involves training an agent to make decisions based on feedback from its environment. Here are key concepts:
    \begin{itemize}
        \item \textbf{Agent}: The learner or decision-maker (e.g., a robot, a software program).
        \item \textbf{Environment}: Everything the agent interacts with (e.g., a game, a real-world scenario).
        \item \textbf{Actions}: Choices made by the agent that affect the environment (e.g., moving left, jumping, picking an item).
        \item \textbf{Rewards}: Signals indicating the success of an action (e.g., points earned, reaching a goal).
        \item \textbf{State}: The current situation of the agent in the environment (e.g., agent’s position in a game).
    \end{itemize}
\end{frame}

\begin{frame}[fragile]
    \frametitle{Introduction to Chapter 12 - Importance of Reinforcement Learning}
    \begin{enumerate}
        \item \textbf{Real-World Applications:}
        \begin{itemize}
            \item \textbf{Game Playing:} RL has led to breakthroughs in game AI, such as \textbf{AlphaGo}, which defeated world champions in the game of Go.
            \item \textbf{Robotics:} Training robots to navigate and interact within unpredictable environments.
            \item \textbf{Recommendation Systems:} Personalizing content (like videos or ads) based on users' actions.
        \end{itemize}
        \item \textbf{Learning Paradigm:} Reinforcement Learning differs from supervised learning, focusing on learning from exploration and trial-and-error, making it powerful in dynamic environments where rules may not be explicitly defined.
    \end{enumerate}
\end{frame}

\begin{frame}[fragile]
    \frametitle{Introduction to Chapter 12 - Engaging Example}
    Imagine teaching a dog to fetch a ball. Every time it successfully retrieves the ball, you give it a treat (reward). If it doesn’t fetch (wrong action), it gets no treat. 
    Over time, the dog learns to associate fetching the ball with getting a reward. 
    This is the essence of RL—learning through interaction and feedback.
\end{frame}

\begin{frame}[fragile]
    \frametitle{Introduction to Chapter 12 - Key Points to Emphasize}
    \begin{itemize}
        \item \textbf{Exploration vs. Exploitation:} The agent must balance trying new actions (exploration) and leveraging known actions that yield high rewards (exploitation).
        \item \textbf{Long vs. Short-Term Reward:} Agents learn to maximize the cumulative reward over time, often requiring strategies that involve considering future impacts of their actions.
    \end{itemize}
\end{frame}

\begin{frame}[fragile]
    \frametitle{Introduction to Chapter 12 - Conclusion}
    As we dive deeper into reinforcement learning in the next slides, think about the implications of RL across various industries and its potential to solve complex problems. 
    This chapter will inspire you to ponder questions like: 
    \begin{itemize}
        \item How can RL transform everyday tasks?
        \item What ethical considerations arise when deploying RL in critical systems?
    \end{itemize}
    Together, we will explore these intriguing facets of reinforcement learning and its role in shaping the future of artificial intelligence!
\end{frame}

\begin{frame}[fragile]{What is Reinforcement Learning? - Part 1}
    \begin{block}{Definition}
        Reinforcement Learning (RL) is a subset of Machine Learning (ML) where an agent learns to make decisions by interacting with its environment. 
    \end{block}
    
    \begin{block}{Key Differences}
        Unlike supervised learning, where the model learns from labeled data, in RL, the agent learns from the consequences of its actions, similar to how humans learn through trial and error.
    \end{block}
\end{frame}

\begin{frame}[fragile]{What is Reinforcement Learning? - Part 2}
    \frametitle{Key Principles of Reinforcement Learning}
  
    \begin{itemize}
        \item \textbf{Agent}: The decision-maker (e.g., a robot or game player).
        \item \textbf{Environment}: Everything the agent interacts with (e.g., a chess board).
        \item \textbf{Actions}: Moves available to the agent (e.g., move left or right).
        \item \textbf{Rewards}: Feedback received from environment actions (positive or negative).
        \item \textbf{Policy}: Strategy used by the agent to decide on actions.
    \end{itemize}
\end{frame}

\begin{frame}[fragile]{What is Reinforcement Learning? - Part 3}
    \frametitle{Example and Key Points}
    
    \begin{block}{Example}
        Teaching a pet to fetch:
        \begin{itemize}
            \item The pet (agent) retrieves the ball (action).
            \item Receives praise or treats (reward).
            \item Develops a fetching strategy (policy).
        \end{itemize}
    \end{block}

    \begin{itemize}
        \item \textbf{Trial and Error Learning}: The agent learns by experimenting and improving.
        \item \textbf{Reward Structure}: Critical for guiding learning; immediate vs delayed rewards influence strategies.
        \item \textbf{Exploration vs. Exploitation}: Balancing new actions with known successful actions is key.
    \end{itemize}
\end{frame}

\begin{frame}[fragile]{What is Reinforcement Learning? - Part 4}
    \begin{block}{Summary}
        Reinforcement Learning offers a powerful approach to decision-making, enabling agents to learn optimal behaviors through environmental interactions. 
        Its principles underpin many AI innovations, from game-playing bots to robotics and autonomous vehicles, all leveraging rewards and adaptive strategies.
    \end{block}
\end{frame}

\begin{frame}[fragile]
    \frametitle{Key Terminology - Introduction to Key Concepts}
    In the realm of Reinforcement Learning (RL), understanding key terminology is crucial to grasp how agents interact with their environments.
    This foundational knowledge enables effective engagement with advanced topics later on.
\end{frame}

\begin{frame}[fragile]
    \frametitle{Key Terminology - Agents and Environments}
    \begin{enumerate}
        \item \textbf{Agents}
        \begin{itemize}
            \item \textbf{Definition}: An agent is an entity that makes decisions by interacting with an environment.
            \item \textbf{Example}: A robotic vacuum cleaner navigates around a room (environment) to clean effectively.
        \end{itemize}
        
        \item \textbf{Environments}
        \begin{itemize}
            \item \textbf{Definition}: The environment encompasses everything that the agent interacts with while performing tasks.
            \item \textbf{Example}: In a video game, the digital world including obstacles, rewards, and characters.
        \end{itemize}
    \end{enumerate}
\end{frame}

\begin{frame}[fragile]
    \frametitle{Key Terminology - Actions, Rewards, and Policies}
    \begin{enumerate}
        \setcounter{enumi}{2} % To continue numbering from previous frame
        \item \textbf{Actions}
        \begin{itemize}
            \item \textbf{Definition}: Actions are choices made by an agent to perform tasks within the environment.
            \item \textbf{Example}: For a chess-playing AI, possible actions include moving a pawn or capturing an opponent's piece.
        \end{itemize}
        
        \item \textbf{Rewards}
        \begin{itemize}
            \item \textbf{Definition}: Rewards are signals received from the environment that inform the agent about the success or failure of its actions.
            \item \textbf{Example}: Collecting coins in a video game as a reward for successfully navigating to a certain location.
        \end{itemize}
        
        \item \textbf{Policies}
        \begin{itemize}
            \item \textbf{Definition}: A policy is a strategy that dictates what actions an agent should take in different states.
            \item \textbf{Example}: A navigation app may have a deterministic policy to always choose the shortest route or a stochastic one based on traffic data.
        \end{itemize}
    \end{enumerate}
\end{frame}

\begin{frame}[fragile]
    \frametitle{The Learning Process - Overview}
    The learning process for agents in Reinforcement Learning (RL) involves:
    \begin{itemize}
        \item **Trial and Error**: A fundamental method of learning by interacting with the environment.
        \item Key components include:
        \begin{itemize}
            \item Agents
            \item Environment
            \item Actions
            \item Rewards
        \end{itemize}
    \end{itemize}
\end{frame}

\begin{frame}[fragile]
    \frametitle{Key Concepts of Learning}
    \begin{enumerate}
        \item **Agents**: Entities that perceive their environment and act to maximize performance (e.g., a robot in a maze).
        \item **Environment**: Everything an agent interacts with, providing feedback (e.g., maze walls and exit points).
        \item **Actions**: Choices impacting the agent's state (e.g., moving, turning).
        \item **Rewards**: Signals from the environment based on actions, guiding behavior (e.g., positive reward for reaching the exit).
    \end{enumerate}
\end{frame}

\begin{frame}[fragile]
    \frametitle{Example Scenario and Summary}
    \textbf{Example Scenario:}
    Imagine a dog learning to fetch a ball:
    \begin{itemize}
        \item The dog (agent) receives the ball (action) and navigates the yard (environment).
        \item Successful retrieval brings praise (positive reward); incorrect direction leads to no reward (negative outcome).
    \end{itemize}

    \textbf{Key Points to Emphasize:}
    \begin{itemize}
        \item **Exploration vs. Exploitation**: Balance between trying new actions and using known rewarding actions.
        \item **Feedback Loop**: Continuous improvement based on environmental feedback.
        \item **Self-Improvement**: Learning without pre-labeled data in an uncertain world.
    \end{itemize}
\end{frame}

\begin{frame}[fragile]
    \frametitle{Difference from Supervised and Unsupervised Learning - Overview}
    \begin{block}{Clear Explanations of Concepts}
        \begin{enumerate}
            \item \textbf{Supervised Learning:}
            \begin{itemize}
                \item \textbf{Definition:} Trained using labeled data where input and correct output are paired.
                \item \textbf{Example:} Teaching a child to recognize fruits through labeled pictures.
            \end{itemize}
            
            \item \textbf{Unsupervised Learning:}
            \begin{itemize}
                \item \textbf{Definition:} Involves data without explicit labels, focusing on finding patterns.
                \item \textbf{Example:} A child grouping fruits by similarity without knowing their names.
            \end{itemize}
            
            \item \textbf{Reinforcement Learning:}
            \begin{itemize}
                \item \textbf{Definition:} Learning through interactions with an environment, receiving rewards or penalties.
                \item \textbf{Example:} Training a pet with rewards and penalties based on behavior.
            \end{itemize}
        \end{enumerate}
    \end{block}
\end{frame}

\begin{frame}[fragile]
    \frametitle{Difference from Supervised and Unsupervised Learning - Key Points}
    \begin{block}{Key Points to Emphasize}
        \begin{itemize}
            \item \textbf{Data Requirement:}
            \begin{itemize}
                \item Supervised Learning: Needs labeled data.
                \item Unsupervised Learning: Works with unlabeled data.
                \item Reinforcement Learning: Operates through trial and error.
            \end{itemize}

            \item \textbf{Feedback Mechanism:}
            \begin{itemize}
                \item Supervised Learning: Direct feedback on predictions.
                \item Unsupervised Learning: No feedback; discovers patterns independently.
                \item Reinforcement Learning: Feedback via rewards and penalties.
            \end{itemize}

            \item \textbf{Applications:}
            \begin{itemize}
                \item Supervised Learning: Classification tasks such as spam detection.
                \item Unsupervised Learning: Clustering tasks like market segmentation.
                \item Reinforcement Learning: Applications in gaming and robotics.
            \end{itemize}
        \end{itemize}
    \end{block}
\end{frame}

\begin{frame}[fragile]
    \frametitle{Summary Comparison Table}
    \begin{table}[ht]
        \centering
        \begin{tabular}{|l|l|l|l|l|}
            \hline
            \textbf{Learning Type} & \textbf{Feedback} & \textbf{Data Type} & \textbf{Goal} & \textbf{Example} \\
            \hline
            Supervised Learning & Direct feedback & Labeled data & Predict outcomes & Classifying emails as spam \\
            \hline
            Unsupervised Learning & No feedback & Unlabeled data & Discover patterns & Grouping customers by behavior \\
            \hline
            Reinforcement Learning & Reward/Punishment & Interaction-based & Maximize reward over time & Training a robot to navigate \\
            \hline
        \end{tabular}
    \end{table}
\end{frame}

\begin{frame}[fragile]
    \frametitle{Engagement Questions}
    \begin{block}{Discussion Points}
        \begin{itemize}
            \item Can you think of a real-world situation where reinforcement learning might be more beneficial than supervised or unsupervised learning?
            \item How might a child learn differently from a supervised versus an unsupervised setup? What are the implications for designing learning systems in technology?
        \end{itemize}
    \end{block}
\end{frame}

\begin{frame}[fragile]
  \frametitle{Types of Reinforcement Learning}
  \begin{block}{Overview}
    Reinforcement Learning (RL) allows an agent to learn decision-making through interaction with an environment, aiming to maximize cumulative rewards.
  \end{block}
\end{frame}

\begin{frame}[fragile]
  \frametitle{Types of Reinforcement Learning - Model-Free vs Model-Based}
  
  \begin{enumerate}
    \item \textbf{Model-Free RL:}
    \begin{itemize}
      \item Learns directly from experiences, no explicit model of the environment.
      \item Key Characteristics:
      \begin{itemize}
        \item Trial and Error Learning.
        \item No Environment Prediction.
      \end{itemize}
      \item Common Algorithms:
      \begin{itemize}
        \item Q-Learning
        \item Deep Q-Networks (DQN)
      \end{itemize}
    \end{itemize}
    
    \item \textbf{Model-Based RL:}
    \begin{itemize}
      \item Constructs a model of the environment to predict outcomes and refine policies.
      \item Key Characteristics:
      \begin{itemize}
        \item Environment Simulation.
        \item Planning Capability.
      \end{itemize}
      \item Common Algorithms:
      \begin{itemize}
        \item Dyna-Q
        \item Monte Carlo Tree Search (MCTS)
      \end{itemize}
    \end{itemize}
  \end{enumerate}
  
\end{frame}

\begin{frame}[fragile]
  \frametitle{Examples and Key Points}
  
  \begin{block}{Examples}
    \textbf{Model-Free Example:} A robot learns to navigate a maze through trial and error.
    
    \textbf{Model-Based Example:} An AI agent playing chess simulating future moves before deciding.
  \end{block}

  \begin{block}{Key Points}
    \begin{itemize}
      \item Exploration vs Exploitation in model-free approaches.
      \item Efficiency of model-based methods in learning environments.
      \item Applications in fields such as robotics and game AI.
    \end{itemize}
  \end{block}
  
  \textbf{Engaging Question:} What challenges do you think an agent faces in unpredictable environments using these RL approaches?
  
\end{frame}

\begin{frame}[fragile]
    \frametitle{Applications of Reinforcement Learning - Introduction}
    \begin{block}{What is Reinforcement Learning?}
        Reinforcement Learning (RL) is a powerful framework for solving complex problems by leveraging the interaction of agents with environments. 
        It allows agents to learn by taking actions, receiving feedback in the form of rewards or penalties, and adjusting strategies to maximize cumulative rewards.
    \end{block}
    
    \begin{block}{Key Takeaway}
        RL has made significant strides and found versatile applications across various fields, including robotics, gaming, and finance.
    \end{block}
\end{frame}

\begin{frame}[fragile]
    \frametitle{Applications of Reinforcement Learning - Real-World Examples}
    \begin{enumerate}
        \item \textbf{Robotics} 
        \begin{itemize}
            \item \textit{Example: Robot Navigation}
            \begin{itemize}
                \item RL enables robots to autonomously navigate complex environments.
                \item A robotic vacuum learns efficient cleaning paths through exploration and rewards associated with cleaning time.
            \end{itemize}
            \item \textit{Key Points}
            \begin{itemize}
                \item Enhances adaptability and learning from experience.
            \end{itemize}
        \end{itemize}

        \item \textbf{Gaming}
        \begin{itemize}
            \item \textit{Example: Atari Games}
            \begin{itemize}
                \item Algorithms like Deep Q-Networks (DQN) achieve success in classic Atari games by maximizing scores from raw pixel inputs.
                \item In Breakout, the RL agent learns to control the paddle through past experiences.
            \end{itemize}
            \item \textit{Key Points}
            \begin{itemize}
                \item Achieves superhuman performance without prior knowledge.
            \end{itemize}
        \end{itemize}
        
        \item \textbf{Finance}
        \begin{itemize}
            \item \textit{Example: Algorithmic Trading}
            \begin{itemize}
                \item RL is applied in trading strategies, allowing agents to maximize returns while minimizing losses.
            \end{itemize}
            \item \textit{Key Points}
            \begin{itemize}
                \item Adapts to rapidly changing market conditions.
            \end{itemize}
        \end{itemize}
    \end{enumerate}
\end{frame}

\begin{frame}[fragile]
    \frametitle{Benefits and Summary of Reinforcement Learning}
    \begin{block}{Potential Benefits of Reinforcement Learning}
        \begin{itemize}
            \item \textbf{Autonomous Learning:} Reduces need for human guidance.
            \item \textbf{Flexibility:} Effective in unpredictable environments with undefined rules.
            \item \textbf{Performance Optimization:} Capable of discovering superior strategies compared to traditional methods.
        \end{itemize}
    \end{block}
    
    \begin{block}{Conclusion}
        Reinforcement Learning is transforming industries through smarter systems, with applications in robotics, gaming, and finance that illustrate its potential for innovation and addressing complex challenges.
    \end{block}
    
    \begin{block}{Questions for Discussion}
        - What other domains could benefit from RL applications?
        - How can we ensure the ethical use of RL systems, especially in finance?
    \end{block}
\end{frame}

\begin{frame}[fragile]
    \frametitle{Challenges in Reinforcement Learning}
    This slide discusses key challenges such as exploration vs. exploitation, reward shaping, and scalability.
\end{frame}

\begin{frame}[fragile]
    \frametitle{1. Exploration vs. Exploitation}
    
    \begin{block}{Concept Explanation}
        In Reinforcement Learning (RL), an agent must balance between:
        \begin{itemize}
            \item \textbf{Exploration:} Trying new actions to discover their effects.
            \item \textbf{Exploitation:} Leveraging known actions that yield the highest rewards.
        \end{itemize}
    \end{block}

    \begin{block}{Example}
        - \textbf{Exploration:} An agent in a game might try unexpected strategies.
        - \textbf{Exploitation:} Repeating a known effective shortcut once discovered.
    \end{block}

    \begin{block}{Key Points}
        \begin{itemize}
            \item Balancing exploration and exploitation is crucial for optimizing learning.
            \item Techniques such as $\epsilon$-greedy are widely used to manage this balance.
        \end{itemize}
    \end{block}
\end{frame}

\begin{frame}[fragile]
    \frametitle{2. Reward Shaping}
    
    \begin{block}{Concept Explanation}
        Reward shaping modifies the reward signal to make it more informative for the agent, facilitating faster learning.
    \end{block}

    \begin{block}{Example}
        - Incremental rewards can be given for each step taken towards a goal, rather than just at completion.
    \end{block}

    \begin{block}{Key Points}
        \begin{itemize}
            \item Properly designed reward functions can enhance learning efficiency.
            \item Care must be taken to avoid unintended behaviors stemming from poorly considered rewards.
        \end{itemize}
    \end{block}
\end{frame}

\begin{frame}[fragile]
    \frametitle{3. Scalability}
    
    \begin{block}{Concept Explanation}
        As environmental complexity increases, challenges in RL can scale dramatically.
    \end{block}

    \begin{block}{Example}
        - An agent trained in a simple grid-world might struggle in a complex urban environment.
    \end{block}

    \begin{block}{Key Points}
        \begin{itemize}
            \item Techniques like function approximation and hierarchical reinforcement learning can help.
            \item Algorithms must be designed to manage complexity without losing learning benefits.
        \end{itemize}
    \end{block}
\end{frame}

\begin{frame}[fragile]
    \frametitle{Summary and Discussion}
    
    \begin{block}{Summary}
        Reinforcement Learning faces challenges such as:
        \begin{itemize}
            \item Exploration vs. Exploitation
            \item Reward Shaping
            \item Scalability
        \end{itemize}
        Navigating these challenges requires innovative strategies and a deep understanding.
    \end{block}

    \begin{block}{Discussion Questions}
        \begin{enumerate}
            \item How might changing the reward structure impact the learning process in real-world applications?
            \item What strategies could you propose to handle the exploration-exploitation trade-off in a dynamic environment?
        \end{enumerate}
    \end{block}
\end{frame}

\begin{frame}[fragile]
    \frametitle{Deep Reinforcement Learning}
    \begin{block}{Definition}
        Deep Reinforcement Learning (DRL) integrates deep learning techniques with reinforcement learning (RL) to manage high-dimensional state and action spaces effectively.
    \end{block}
\end{frame}

\begin{frame}[fragile]
    \frametitle{Understanding Deep Reinforcement Learning}
    \begin{itemize}
        \item \textbf{Reinforcement Learning (RL):} A paradigm where an agent learns decisions through interactions with the environment, receiving feedback as rewards or penalties.
        \item \textbf{Deep Learning:} A subset of machine learning that utilizes neural networks with multiple layers to learn complex data representations.
    \end{itemize}
\end{frame}

\begin{frame}[fragile]
    \frametitle{Why Combine Deep Learning with RL?}
    \begin{itemize}
        \item \textbf{Complex Environments:} DRL effectively tackles high-dimensional state/action spaces seen in scenarios like video games and robotics.
        \item \textbf{Feature Extraction:} Deep learning automates the feature extraction process, aiding RL agents in understanding their environments better.
    \end{itemize}
\end{frame}

\begin{frame}[fragile]
    \frametitle{Examples of DRL in Action}
    \begin{itemize}
        \item \textbf{Atari Games:} Algorithms such as DQN train agents using pixel data to maximize game scores via CNNs.
        \item \textbf{Robotics:} Used in tasks like navigation and manipulation, where robots learn by trial and error to complete complex actions.
    \end{itemize}
\end{frame}

\begin{frame}[fragile]
    \frametitle{Key Components of DRL}
    \begin{itemize}
        \item \textbf{Agent:} Learner or decision-maker.
        \item \textbf{Environment:} The context in which the agent operates.
        \item \textbf{Reward Signal:} Feedback for actions taken, aiming to maximize cumulative reward.
        \item \textbf{Policy:} Strategy to determine actions based on the current state.
        \item \textbf{Value Function:} Estimates expected return from a state or state-action pair.
    \end{itemize}
\end{frame}

\begin{frame}[fragile]
    \frametitle{Example Architecture: DQN}
    \begin{block}{Deep Q-Network (DQN)}
        A DRL architecture combining Q-learning with deep neural networks.
    \end{block}

    \begin{itemize}
        \item \textbf{Q-Learning Simplified:} 
        The agent estimates action values at different states.
        \item \textbf{Key Update Equation:}
            \begin{equation}
                Q(s, a) \leftarrow Q(s, a) + \alpha \left[ r + \gamma \max Q(s', a') - Q(s, a) \right]
            \end{equation}
        % Where:
        % \begin{itemize}
        %     \item \(s\) = current state
        %     \item \(a\) = action taken
        %     \item \(r\) = reward received
        %     \item \(s'\) = next state
        %     \item $\alpha$ = learning rate
        %     \item $\gamma$ = discount factor
        % \end{itemize}
        
        \item \textbf{Neural Network Role:} 
        Estimates \(Q(s, a)\) from the visual input of the state, approximating action quality without complete state-action space knowledge.
    \end{itemize}
\end{frame}

\begin{frame}[fragile]
    \frametitle{Engaging Questions for Exploration}
    \begin{itemize}
        \item How might DRL impact industries beyond gaming, such as healthcare or finance?
        \item What ethical considerations emerge with the application of DRL to real-world problems?
        \item How can we enhance DRL algorithms to be more efficient and less data-dependent?
    \end{itemize}
\end{frame}

\begin{frame}[fragile]
    \frametitle{Key Takeaways}
    \begin{itemize}
        \item DRL combines deep learning's feature extraction abilities with the goal of maximizing rewards in RL.
        \item It is effective in complex decision-making tasks where traditional methods may fail.
        \item Exploring DRL can lead to advancements in fields such as robotics, gaming, and autonomous systems.
    \end{itemize}
\end{frame}

\begin{frame}[fragile]
    \frametitle{Case Study: AlphaGo}
    \begin{block}{Introduction to AlphaGo}
        AlphaGo is an AI program developed by DeepMind. It gained fame in 2016 after defeating top Go player Lee Sedol. 
        The complexity of Go, with about $10^{170}$ possible moves, makes it a significant challenge for AI.
    \end{block}
\end{frame}

\begin{frame}[fragile]
    \frametitle{Key Concepts in AlphaGo}
    \begin{enumerate}
        \item \textbf{Reinforcement Learning}:
            \begin{itemize}
                \item Definition: AI learns to achieve goals by taking actions and receiving rewards.
                \item AlphaGo's usage: It optimizes decision-making through trial and error to learn winning moves.
            \end{itemize}
        \item \textbf{Deep Learning}:
            \begin{itemize}
                \item Definition: Utilizes deep neural networks to interpret data.
                \item Integration: Combines with reinforcement learning for move evaluation and prediction.
            \end{itemize}
    \end{enumerate}
\end{frame}

\begin{frame}[fragile]
    \frametitle{How AlphaGo Works}
    \begin{enumerate}
        \item \textbf{Training Process}:
            \begin{itemize}
                \item Self-Play: Played thousands of games against itself, enhancing strategies over time.
                \item Supervised Learning: Initially trained on historical games from skilled players.
            \end{itemize}
        \item \textbf{Move Selection}:
            \begin{itemize}
                \item Policy Network: Predicts winning probabilities for each possible move.
                \item Value Network: Assesses winning likelihood from different board positions.
            \end{itemize}
    \end{enumerate}
\end{frame}

\begin{frame}[fragile]
    \frametitle{Key Points and Insights}
    \begin{block}{Key Points}
        \begin{itemize}
            \item The Game of Go demands strategic foresight and tactical thinking.
            \item AlphaGo symbolizes a groundbreaking achievement in AI, showcasing the synergy of learning techniques.
            \item Its methods offer potential applications in fields like healthcare, finance, and robotics.
        \end{itemize}
    \end{block}
    \begin{block}{Additional Insights}
        \begin{itemize}
            \item Ethical Considerations: The rise of AI raises discussions on its ethical usage.
            \item Future Applications: AlphaGo's architecture can tackle complex real-world issues beyond gaming.
        \end{itemize}
    \end{block}
\end{frame}

\begin{frame}[fragile]
    \frametitle{Evaluating Reinforcement Learning Models - Introduction}
    \begin{itemize}
        \item Evaluating RL model performance is crucial for real-world applications.
        \item Differences from traditional supervised learning:
        \begin{itemize}
            \item RL requires a nuanced approach.
            \item Performance assessment involves sequential decision-making and delayed rewards.
        \end{itemize}
    \end{itemize}
\end{frame}

\begin{frame}[fragile]
    \frametitle{Evaluating Reinforcement Learning Models - Key Evaluation Methods}
    \begin{enumerate}
        \item \textbf{Reward Metrics}
            \begin{itemize}
                \item \textbf{Total Reward}: Sum of all rewards in an episode.
                \item \textbf{Average Reward}: Total reward divided by the number of episodes.
            \end{itemize}
        \item \textbf{Discounted Reward}
            \begin{itemize}
                \item Future rewards are discounted using a factor $ \gamma $.
                \item \begin{equation}
                    R_t = r_t + \gamma r_{t+1} + \gamma^2 r_{t+2} + \ldots
                \end{equation}
            \end{itemize}
        \item \textbf{Success Rate}
            \begin{itemize}
                \item Percentage of episodes achieving a predefined goal.
            \end{itemize}
        \item \textbf{Learning Efficiency}
            \begin{itemize}
                \item Measured by cumulative reward and improvement over time.
            \end{itemize}
    \end{enumerate}
\end{frame}

\begin{frame}[fragile]
    \frametitle{Evaluating Reinforcement Learning Models - Example Application}
    \begin{itemize}
        \item \textbf{AlphaGo Example}
            \begin{itemize}
                \item Evaluation focused on win rates against human champions.
                \item Key metrics used:
                    \begin{itemize}
                        \item Win rate in matches.
                        \item Average score per game.
                        \item Learning curve visualization.
                    \end{itemize}
            \end{itemize}
    \end{itemize}
\end{frame}

\begin{frame}[fragile]
    \frametitle{Evaluating Reinforcement Learning Models - Conclusion}
    \begin{itemize}
        \item Importance of diverse metrics for capturing effectiveness.
        \item Emphasis on dynamic evaluation over time.
        \item Consideration of contextual relevance and benchmarking against baselines.
    \end{itemize}
\end{frame}

\begin{frame}[fragile]
    \frametitle{Future Directions in Reinforcement Learning}
    
    \begin{block}{Overview}
        Reinforcement learning (RL) is evolving rapidly, presenting exciting avenues for research and innovation.
    \end{block}
\end{frame}

\begin{frame}[fragile]
    \frametitle{Key Trends and Areas for Future Research - Part 1}
    
    \begin{enumerate}
        \item \textbf{Integration with Neural Architectures}
        \begin{itemize}
            \item New architectures like Transformers and U-Nets adapted for RL.
            \item Example: Enhancing sequence prediction in robotics and finance.
        \end{itemize}

        \item \textbf{Multi-Agent Reinforcement Learning (MARL)}
        \begin{itemize}
            \item Focus on environments with multiple interacting agents.
            \item Example: Autonomous vehicles coordinating for traffic efficiency.
        \end{itemize}

        \item \textbf{Imitation Learning and Transfer Learning}
        \begin{itemize}
            \item Leveraging human or agent knowledge for efficiency.
            \item Example: Robots learning navigation from human demonstrations.
        \end{itemize}
    \end{enumerate}
\end{frame}

\begin{frame}[fragile]
    \frametitle{Key Trends and Areas for Future Research - Part 2}
    
    \begin{enumerate}
        \setcounter{enumi}{3} % Continue numbering from the previous frame
        \item \textbf{Safe and Ethical AI}
        \begin{itemize}
            \item Developing RL systems with safety and ethics in decision-making.
            \item Example: Adhering to ethical guidelines in medical decisions.
        \end{itemize}

        \item \textbf{Explainable Reinforcement Learning (XRL)}
        \begin{itemize}
            \item Understanding decision-making in complex RL systems.
            \item Example: Mechanisms to interpret RL agent's actions.
        \end{itemize}

        \item \textbf{Plug-and-Play RL}
        \begin{itemize}
            \item Modular components for easy integration in diverse industries.
            \item Example: Adaptable RL algorithms for energy management.
        \end{itemize}
    \end{enumerate}
\end{frame}

\begin{frame}[fragile]
    \frametitle{Future Directions: Engaging Thoughts}
    
    \begin{block}{Key Points to Emphasize}
        \begin{itemize}
            \item Interdisciplinary Collaboration: Between fields like neuroscience, economics, and robotics.
            \item Customization and Personalization: Tailored RL applications for user needs.
            \item Sustainability Focus: Addressing global challenges with RL strategies.
        \end{itemize}
    \end{block}

    \begin{block}{Thought Questions}
        \begin{itemize}
            \item How can RL transform industries outside of traditional tech sectors?
            \item In what ways can we ensure the future of RL is both beneficial and ethical?
        \end{itemize}
    \end{block}
\end{frame}

\begin{frame}[fragile]
    \frametitle{Summary of Key Points - Part 1}
    \begin{block}{1. What is Reinforcement Learning?}
        \begin{itemize}
            \item \textbf{Definition}: A type of machine learning where an agent learns to make decisions by taking actions in an environment to maximize cumulative rewards.
            \item \textbf{Key Concepts}:
            \begin{itemize}
                \item \textbf{Agent}: The learner or decision-maker.
                \item \textbf{Environment}: The context or system within which the agent operates.
                \item \textbf{Actions}: Choices made by the agent that affect the state of the environment.
                \item \textbf{Rewards}: Feedback from the environment representing the success of actions taken.
            \end{itemize}
        \end{itemize}
    \end{block}
    
    \begin{block}{2. Core Components of RL}
        \begin{itemize}
            \item \textbf{State}: The current situation of the agent in the environment.
            \item \textbf{Policy}: A strategy for determining the action to take in a given state.
            \item \textbf{Value Function}: A prediction of future rewards to evaluate favorable states.
            \item \textbf{Q-Value (Action-Value) Function}: Evaluates expected utility of taking a specific action in a state.
        \end{itemize}
    \end{block}
\end{frame}

\begin{frame}[fragile]
    \frametitle{Summary of Key Points - Part 2}
    \begin{block}{3. Learning Paradigms}
        \begin{itemize}
            \item \textbf{Model-Free vs. Model-Based}:
            \begin{itemize}
                \item \textbf{Model-Free Learning}: Learns from rewards without a model. E.g., Q-Learning, SARSA.
                \item \textbf{Model-Based Learning}: Builds a model of the environment for decision-making. E.g., dynamic programming.
            \end{itemize}
        \end{itemize}
    \end{block}

    \begin{block}{4. Exploration vs. Exploitation}
        \begin{itemize}
            \item \textbf{Trade-Off}: Balancing exploration of new actions and exploiting known rewarding actions.
            \item \textbf{Example}: Choosing between trying a new strategy or repeating a successful one.
        \end{itemize}
    \end{block}
\end{frame}

\begin{frame}[fragile]
    \frametitle{Summary of Key Points - Part 3}
    \begin{block}{5. Applications of Reinforcement Learning}
        \begin{itemize}
            \item \textbf{Gaming}: AI models like AlphaGo.
            \item \textbf{Robotics}: Teaching robots through trial and error.
            \item \textbf{Healthcare}: Personalizing treatment plans based on patient responses.
        \end{itemize}
    \end{block}

    \begin{block}{6. Future Directions}
        \begin{itemize}
            \item Integrating with Neural Networks: Using deep learning to handle complex data.
            \item Transfer Learning: Adapting knowledge across domains.
            \item Multi-Agent Systems: Cooperation among multiple agents for complex tasks.
        \end{itemize}
    \end{block}

    \begin{block}{Key Takeaways}
        \begin{itemize}
            \item RL is about learning from interaction to maximize rewards.
            \item Balancing exploration and exploitation is critical.
            \item Diverse applications are leading to advancements in AI.
        \end{itemize}
    \end{block}
\end{frame}

\begin{frame}[fragile]
  \frametitle{Discussion Questions - Overview}
  \begin{block}{Exploring the Potential and Ethical Considerations of Reinforcement Learning (RL)}
    Reinforcement Learning (RL) is a powerful tool within artificial intelligence, but its applications and implications raise important discussions. Here are engaging questions designed to prompt thoughtful conversations:
  \end{block}
\end{frame}

\begin{frame}[fragile]
  \frametitle{Potential Applications of RL}
  \begin{enumerate}
    \item \textbf{Impactful Industries:}
      \begin{itemize}
        \item In what industries do you think reinforcement learning could have the most significant impact, and why?
        \item \textbf{Example:} Think about healthcare: RL can optimize treatment plans for patients by recommending personalized medication schedules based on trial and error outcomes.
      \end{itemize}
    
    \item \textbf{Enhancing User Experiences:}
      \begin{itemize}
        \item How can reinforcement learning enhance user experiences in everyday technologies, such as mobile applications or smart home devices?
        \item \textbf{Example:} Consider how an RL-powered virtual assistant could learn a user's preferences over time, providing increasingly relevant suggestions.
      \end{itemize}
  \end{enumerate}
\end{frame}

\begin{frame}[fragile]
  \frametitle{Ethical Considerations and Societal Implications}
  \begin{enumerate}
    \item \textbf{Ethical Dilemmas:}
      \begin{itemize}
        \item What ethical dilemmas might arise from using reinforcement learning in decision-making processes, such as autonomous vehicles?
        \item \textbf{Example:} Discuss scenarios where an autonomous vehicle must make choices that could harm individuals, raising questions about how to program moral values into machines.
      \end{itemize}
    
    \item \textbf{Bias in Training Data:}
      \begin{itemize}
        \item How could biases in training data affect the outcomes produced by RL systems?
        \item \textbf{Example:} If an RL algorithm learns from data that reflects societal biases (like hiring data favoring a particular demographic), it might perpetuate discrimination.
      \end{itemize}

    \item \textbf{Long-term Societal Impacts:}
      \begin{itemize}
        \item What are the potential long-term societal impacts if RL becomes widely implemented in fields like surveillance or law enforcement?
        \item \textbf{Example:} Discuss the balance between security and privacy and consider how RL algorithms could be used to monitor citizens.
      \end{itemize}
  
    \item \textbf{Ensuring Accountability:}
      \begin{itemize}
        \item How do we ensure accountability for decisions made by RL systems, especially in critical sectors like finance or healthcare?
      \end{itemize}
  \end{enumerate}
\end{frame}

\begin{frame}[fragile]
    \frametitle{Additional Resources - Introduction}
    \begin{block}{Overview}
        Reinforcement Learning (RL) is a dynamic area of machine learning focused on how agents ought to take actions in an environment to maximize cumulative reward. To deepen your understanding of RL, we recommend various resources, including:
    \end{block}
    \begin{itemize}
        \item Books
        \item Online Courses
        \item Research Papers
        \item Online Communities
    \end{itemize}
\end{frame}

\begin{frame}[fragile]
    \frametitle{Additional Resources - Recommended Books}
    \begin{enumerate}
        \item \textbf{"Reinforcement Learning: An Introduction"} by Richard S. Sutton and Andrew G. Barto
        \begin{itemize}
            \item \textbf{Overview}: Foundational text explaining RL principles and methods.
            \item \textbf{Key Points}: Markov Decision Processes (MDPs), policy gradients, and value functions.
        \end{itemize}

        \item \textbf{"Deep Reinforcement Learning Hands-On"} by Maxim Lapan
        \begin{itemize}
            \item \textbf{Overview}: Practical guide incorporating deep learning into RL with hands-on projects.
            \item \textbf{Key Points}: Implementing deep Q-networks and policy-based methods.
        \end{itemize}
    \end{enumerate}
\end{frame}

\begin{frame}[fragile]
    \frametitle{Additional Resources - Online Courses and Research Papers}
    \begin{block}{Online Courses}
        \begin{enumerate}
            \item \textbf{Coursera: "Reinforcement Learning Specialization" by the University of Alberta}
            \item \textbf{edX: "Practical Deep Learning for Coders" by Fast.ai}
        \end{enumerate}
    \end{block}
    
    \begin{block}{Research Papers}
        \begin{enumerate}
            \item \textbf{"Playing Atari with Deep Reinforcement Learning"} by Mnih et al. (2013)
            \item \textbf{"Mastering Chess and Shogi by Self-Play with a General Reinforcement Learning Algorithm"} by Silver et al. (2018)
        \end{enumerate}
    \end{block}
\end{frame}

\begin{frame}[fragile]
    \frametitle{Additional Resources - Online Communities and Key Points}
    \begin{block}{Online Communities and Platforms}
        \begin{itemize}
            \item \textbf{OpenAI Gym}: Toolkit for developing and comparing RL algorithms. \textit{Website}: \url{https://gym.openai.com/}
            \item \textbf{Kaggle}: Platform for data science competitions, providing datasets and collaborative tools. \textit{Website}: \url{https://www.kaggle.com/}
        \end{itemize}
    \end{block}
    
    \begin{block}{Key Points to Emphasize}
        \begin{itemize}
            \item Diverse Learning Pathways
            \item Practical Application
            \item Community Engagement
        \end{itemize}
    \end{block}
\end{frame}

\begin{frame}[fragile]
    \frametitle{Q\&A Session - Introduction}
    \begin{block}{Description}
    Open floor for any questions about the concepts discussed in this chapter. 
    This session is designed to clarify any uncertainties and deepen understanding of the advanced topics discussed.
    \end{block}
\end{frame}

\begin{frame}[fragile]
    \frametitle{Q\&A Session - Key Topics}
    \begin{enumerate}
        \item \textbf{Advanced Reinforcement Learning Strategies}
        \begin{itemize}
            \item Techniques to improve learning efficiency and performance
            \item Examples: transfer learning, meta-learning, multi-agent systems
        \end{itemize}
        
        \item \textbf{Neural Network Architectures}
        \begin{itemize}
            \item Recent developments: Transformers, U-Nets, Diffusion Models
            \item Differences from traditional neural networks
        \end{itemize}
        
        \item \textbf{Applications of Advanced Topics}
        \begin{itemize}
            \item \textbf{Healthcare:} Predictive analytics in patient diagnosis
            \item \textbf{Autonomous Vehicles:} RL for navigation and decision-making
            \item \textbf{Natural Language Processing:} Improved language translation and understanding with Transformers
        \end{itemize}
    \end{enumerate}
\end{frame}

\begin{frame}[fragile]
    \frametitle{Q\&A Session - Engagement Questions}
    \begin{itemize}
        \item What aspects of reinforcement learning do you find most intriguing, and why?
        \item Can anyone share examples of where you’ve seen advanced neural network models in action?
        \item What challenges do you anticipate in applying these advanced techniques in real-world scenarios?
    \end{itemize}
\end{frame}

\begin{frame}[fragile]
    \frametitle{Q\&A Session - Encouraging Reflection}
    \begin{block}{Considerations for Questions}
    - How can we apply the theoretical aspects learned in this chapter to practical problems?
    - In what ways do you think advancements in neural network architectures can impact future technological development?
    \end{block}
    
    \begin{block}{Final Thoughts}
    Don't hesitate to raise any topic for discussion! This session will help build a clear understanding of advanced topics.
    \end{block}
\end{frame}


\end{document}